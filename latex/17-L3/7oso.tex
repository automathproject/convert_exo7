\uuid{7oso}
\exo7id{2219}
\auteur{matos}
\organisation{exo7}
\datecreate{2008-04-23}
\isIndication{false}
\isCorrection{true}
\chapitre{Autre}
\sousChapitre{Autre}

\contenu{
\texte{
Montrer que
}
\begin{enumerate}
    \item \question{cond$_2 (A) = \mu_n(A)/\mu_1(A)$ avec $\mu_n(A)$ et $\mu_1(A)$ respectivement la plus grande et la plus petite valeur singuli\`ere de $A$;}
\reponse{$\|A\|_2^2 =\rho (A^*A)=\max_i \lambda_i(A^*A) =\mu_1^2(A)$ la plus grande valeur singuli\`ere de $A$

$\|A^{-1}\|_2^2=\rho (A^{-1}(A^{-1})^*)=\max_i \lambda_i((A^*A)^{-1}=\frac{1}{\mu_n(A)^2}$ avec $\mu_n(A)$ la plus petite valeur singuli\`ere de $A$. Donc
$$\mbox{cond}_2(A)=\|A\|_2 \|A^{-1}\|_2=\mu_n(A)/\mu_1(A)$$}
    \item \question{si $A$ est normale alors
$$\mbox{cond}_2(A) =\frac{\max_i |\lambda_i(A)|}{\min_i |\lambda_i(A)|} ;$$}
\reponse{Si $A$ est normale alors $\|A\|_2=\rho (A) $ rayon spectral. Donc
$$A^{-1}=UD^{-1}U^* \Rightarrow (A^{-1})^* A^{-1} =U(D^{-1})^* D^{-1}U^*\Rightarrow \rho ((A^{-1})^*A^{-1})=1/\min_i |\lambda_i(A)|^2$$
$$\mbox{cond}_2(A)=\max |\lambda_i(A)|/\min |\lambda_i(A)|$$}
    \item \question{Si $A \in\Rr^{n\times n}$ est inversible, $Q\in\Rr^{n\times n}$ orthogonale alors
$$\mbox{cond}_2(A)=\mbox{cond}_2(AQ)=\mbox{cond}_2(QA)$$}
\reponse{cond$_2(QA)=\|QA\|_2 \|A^{-1}Q^*\|_2=\|A\|_2\|A^{-1}\|_2=$cond$_2(A)$.}
\end{enumerate}
}
