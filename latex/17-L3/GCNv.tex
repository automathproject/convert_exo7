\uuid{GCNv}
\exo7id{2232}
\auteur{matos}
\organisation{exo7}
\datecreate{2008-04-23}
\isIndication{false}
\isCorrection{false}
\chapitre{Autre}
\sousChapitre{Autre}

\contenu{
\texte{
Soit $Z=\left(\begin{array}{cc} c&s\\ -s&c\end{array}\right)$ avec $c^2 + s^2=1$.
On d\'efinit $\rho$ par
$$\rho = \left\{\begin{array}{lrr}
1 &\mbox{si} & c=0\\
1/2\mbox{sign}(c) s  &\mbox{si} & |s|< |c|\\
2\mbox{sign}(s)/c  &\mbox{si} & |c| \leq |s|
\end{array}\right.$$
}
\begin{enumerate}
    \item \question{Comment reconstruire $\pm Z$ \`a partir de $\rho$?}
    \item \question{Soit $Q$ une matrice orthogonale produit de $n$ rotations de Givens: $Q=J_1 \cdots J_n$. Comment peut--on stocker de la fa\c con la plus \'economique $Q$ sous forme factoris\'ee?}
    \item \question{Modifier l'algorithme de Givens pour r\'eduire $A$ \`a la forme triangulaire sup\'erieure ($QA=R$, $Q$ matrice produit de rotations de Givens) en stockant sur place ( donc dans $A$) toute l'information n\'ecessaire \`a reconstruire $Q$.}
    \item \question{Ecrire l'algorithme qui, \`a partir des r\'esultats de l'algorithme pr\'ec\'edent permet de reconstruire $Q$.}
\end{enumerate}
}
