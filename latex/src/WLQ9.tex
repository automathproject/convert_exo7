\uuid{WLQ9}
\exo7id{5777}
\auteur{rouget}
\organisation{exo7}
\datecreate{2010-10-16}
\isIndication{false}
\isCorrection{true}
\chapitre{Espace euclidien, espace normé}
\sousChapitre{Produit scalaire, norme}

\contenu{
\texte{
Soit $E$ un $\Rr$-espace vectoriel muni d'une norme, notée $\|\;\|$, vérifiant l'identité du parallélogramme. Montrer que cette norme est hilbertienne.
}
\reponse{
Soit $N$ une norme sur $E$ vérifiant $\forall(x,y)\in E^2$ $(N(x+y))^2+(N(x-y))^2 = 2((N(x))^2+(N(y))^2)$.

Il faut montrer que la norme $N$ est associée à un produit scalaire $B$. Si $B$ existe, $B$ est nécessairement défini par

\begin{center}
$\forall(x,y)\in E^2$, $B(x,y) =\frac{1}{4}((N(x+y))^2-(N(x-y))^2)$.
\end{center}

Réciproquement, 

\textbullet~Pour tout $x\in E$, $B(x,x) =\frac{1}{4}((N(2x))^2-(N(0))^2)=\frac{1}{4}(4(N(x))^2-0) =(N(x))2$ et donc $\forall x\in E$, $B(x,x)\geqslant0$ puis $B(x,x)=0\Leftrightarrow x=0$. De plus, $\forall x\in E$, $N(x)=\sqrt{B(x,x)}$.

\textbullet~$\forall (x,y)\in E^2$, $B(y,x)=\frac{1}{4}((N(y+x))^2-(N(y-x))^2)=\frac{1}{4}((N(x+y))^2-(N(x-y))^2)=B(x,y)$.

\textbullet~Vérifions alors que l'application $B$ est bilinéaire.

1) Montrons que $\forall(x,y,z)\in E^3$, $B(x+y,z)+B(x-y,z)=2B(x,z)$.

\begin{align*}\ensuremath
B(x+y,z) + B(x-y,z)&=\frac{1}{4}((N(x+y+z))^2-(N(x+y-z))^2 +(N(x-y+z))^2-(N(x-y-z))^2)\\
 &=\frac{1}{4}((N(x+y+z))^2+(N(x-y+z))^2)-((N(x+y-z))^2+(N(x-y-z))^2)\\
 &=\frac{1}{4}(2(N(x+z))^2 +(N(y))^2)-2((N(x-z))^2 +(N(y))^2)\;(\text{par hypothèse sur}\;N)\\
 &=\frac{2}{4}((N(x+z))^2-(N(x-z))^2) =2B(x,z).
\end{align*}

2) Montrons que $\forall(x,z)\in E^2$, $B(2x,z) = 2B(x,z)$. Tout d'abord, $B(0,z)=\frac{1}{4}((N(z))^2-(N(-z))^2)=0$ puis d'après 1)

\begin{align*}\ensuremath
B(2x,z)&= B(x+x,z)+B(x-x,z) = 2B(x,z).
\end{align*}

3) Montrons que $\forall(x,y,z)\in E^3$, $B(x,z) + B(y,z) = B(x+y,z)$.

\begin{align*}\ensuremath
B(x,z) + B(y,z)&= B\left(\frac{x+y}{2}+\frac{x-y}{2},z\right)+ B\left(\frac{x+y}{2}-\frac{x-y}{2},z\right)\\
 &=2B\left(\frac{x+y}{2},z\right)\;(\text{d'après 1)})\\
 &= B(x+y,z)\;(\text{d'après 2)}).
\end{align*}

4) Montrons que $\forall n\in\Nn$, $\forall(x,y)\in E^2$, $B(nx,y) = nB(x,y)$.

\textbullet~~C'est vrai pour $n = 0$ et $n=1$.

\textbullet~Soit $n\geqslant0$. Supposons que $\forall(x,y)\in E^2$, $B(nx,y) = nB(x,y)$ et $B((n+1)x,y) = (n+1)B(x,y)$. Alors

\begin{center}
$B((n+2)x,y) + B(nx,y) = B((n+2)x+nx,y) = B(2(n+1)x,y) = 2B((n+1)x,y)$,
\end{center}

et donc, par hypothèse de récurrence, $B((n+2)x,y)=2(n+1)B(x,y)-nB(x,y) =(n+2)B(x,y)$.

5) Montrons que $\forall n\in\Zz$, $\forall(x,y)\in E^2$, $B(nx,y) = nB(x,y)$. Le résultat est acquis pour $n\geqslant 0$. Pour $n\in\Nn$,

\begin{center}
$B(nx,y) + B(-nx,y) = B(0,y) = 0$ et donc $B(-nx,y) =- B(nx,y)=-nB(x,y)$,
\end{center}

6) Montrons que $\forall n\in\Nn^*$, $\forall(x,y)\in E^2$, $B\left(\frac{1}{n}x,y\right) =\frac{1}{n}B(x,y)$.

\begin{center}
$B(x,y) =B\left(\frac{1}{n}nx,y\right) =nB\left(\frac{1}{n}x,y\right)$ et donc $B\left(\frac{1}{n}x,y\right) =\frac{1}{n}B(x,y)$.
\end{center}

7) Montrons que $\forall r\in\Qq$, $\forall(x,y)\in E^2$, $B(rx,y) = rB(x,y)$. Soient $(p,q)\in\Zz\times\Nn^*$ puis $r=\frac{p}{q}$.

\begin{center}
$B(rx,y) = B\left(\frac{p}{q}x,y\right) =pB\left(\frac{1}{q}x,y\right)=\frac{p}{q}B(x,y) = rB(x,y)$.
\end{center}

8) Montrons que $\forall\lambda\in\Rr$, $\forall(x,y)\in E^2$, $B(\lambda x,y) =\lambda B(x,y)$. Soit $\lambda$ un réel. Puisque $\Qq$ est dense dans $\Rr$, il existe une

suite de rationnels $(r_n)_{n\in\Nn}$ convergente de limite $\lambda$.

Maintenant, l'application $\begin{array}[t]{cccc}N~:&(E,N)&\rightarrow&(\Rr,|\;|)\\
 &x&\mapsto&N(x)
 \end{array}$ est continue sur $E$ car $1$-Lipschitzienne sur $E$. Donc
 
\begin{center}
$B(\lambda x,y) =B(\lim_{n \rightarrow +\infty}r_nx,y)=\lim_{n \rightarrow +\infty}B(r_nx,y)=\lim_{n \rightarrow +\infty}r_nB(x,y)=\lambda B(x,y)$.
\end{center}

Finalement, l'application $B$ est une forme bilinéaire symétrique définie positive et donc un produit scalaire. Puisque $\forall x\in E$, $N(x)=\sqrt{B(x,x)}$, $N$ est la norme associée à ce produit scalaire. On a montré que

\begin{center}
\shadowbox{
toute norme vérifiant l'identité du parallélogramme est une norme hilbertienne.
}
\end{center}
}
}
