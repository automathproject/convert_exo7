\uuid{gjOq}
\exo7id{5128}
\auteur{rouget}
\organisation{exo7}
\datecreate{2010-06-30}
\isIndication{false}
\isCorrection{true}
\chapitre{Nombres complexes}
\sousChapitre{Forme cartésienne, forme polaire}

\contenu{
\texte{
On note $U$ l'ensemble des nombres complexes de module $1$. Montrer que~:

$$\forall z\in\Cc,\;(z\in U\setminus\{-1\}\Leftrightarrow\exists x\in\Rr/\;z=\frac{1+ix}{1-ix}).$$
}
\reponse{
Soient $x\in\Rr$ et $z=\frac{1+ix}{1-ix}$. Puisque $1-ix\neq0$, $z$ est bien défini et
$|z|=\frac{|1+ix|}{|1-ix|}=\frac{|1+ix|}{|\overline{1+ix}|}=1$. Enfin,
$z=\frac{-1+ix+2}{1-ix}=-1+\frac{2}{1-ix}\neq-1$. On a montré que~:
$$\forall x\in\Rr,\;\frac{1+ix}{1-ix}\in U\setminus\{-1\}.$$
Réciproquement, soit $z\in U\setminus\{-1\}$. Il existe un réel $\theta\notin\pi+2\pi\Zz$ tel que $z=e^{i\theta}$. Mais
alors,

$$z=e^{i\theta}=\frac{e^{i\theta/2}}{e^{-i\theta/2}}=\frac{\cos\frac{\theta}{2}+i\sin\frac{\theta}{2}}
{\cos\frac{\theta}{2}-i\sin\frac{\theta}{2}}=\frac{\cos\frac{\theta}{2}(1+i\tan\frac{\theta}{2})}
{\cos\frac{\theta}{2}(1-i\tan\frac{\theta}{2})}=\frac{1+i\tan\frac{\theta}{2}}
{1-i\tan\frac{\theta}{2}}\;(\cos\frac{\theta}{2}\neq0\;\mbox{car}\;\frac{\theta}{2}\notin\frac{\pi}{2}+\pi\Zz),
$$
et $z$ est bien sous la forme voulue avec $x=\tan\frac{\theta}{2}$.
}
}
