\uuid{C5m6}
\exo7id{6882}
\auteur{gammella}
\organisation{exo7}
\datecreate{2012-05-29}
\isIndication{false}
\isCorrection{true}
\chapitre{Analyse vectorielle}
\sousChapitre{Forme différentielle, champ de vecteurs, circulation}

\contenu{
\texte{
On considère la forme différentielle $$\omega= \frac{-y}{x^2+y^2}dx+\frac{x}{x^2+y^2} dy.$$
}
\begin{enumerate}
    \item \question{Dans quel domaine cette forme différentielle est-elle définie ?}
\reponse{La forme $\omega= \frac{-y}{x^2+y^2}dx+\frac{x}{x^2+y^2} dy$ est définie
sur $\Rr^2 \setminus \{(0,0)\}$.}
    \item \question{Calculer l'intégrale curviligne $ \int_{C} \omega$ o\`u $C$ est le cercle de centre $O$ et
 de rayon $1$, parcouru dans le sens direct.}
\reponse{Paramétrons le cercle $C$ par $x=\cos t$, $y=\sin t$ avec $t\in [0;2\pi]$.
On obtient :
\begin{align*}
  \int_{C} \omega&=& \int_{0}^{2\pi}& (-\sin t(-\sin t)+\cos t(\cos t)) dt &\\
 &=&\int_0^{2\pi} &\sin^2 t+\cos^2 t dt&\\
 &=& \int_0^{2\pi}& 1 dt& \\
 &=& 2\pi.&&
 \end{align*}}
    \item \question{La forme $\omega$ est-elle exacte ?}
\reponse{La forme $\omega$ n'est pas exacte, sinon son intégrale curviligne sur la courbe
fermée $C$ serait nulle et cela contredirait notre résultat de la
question précédente. Remarquons cependant que
$$
 \frac{\partial}{\partial y}( \frac{-y}{x^2+y^2})=
 \frac{\partial}{\partial x} ( \frac{x}{x^2+y^2})=  \frac{y^2-x^2}{(x^2+y^2)^2}.$$
En fait, avec cet exemple, on voit que dans le théorème de Poincaré, l'hypothèse que l'ouvert doit \^etre étoilé,
est indispensable. Ici $\Rr^2 \setminus \{(0,0)\}$ n'est pas étoilé, c'est un domaine "troué". De plus,
$\int_{C} \omega$ n'est pas nulle car le cercle entoure le "trou".}
\end{enumerate}
}
