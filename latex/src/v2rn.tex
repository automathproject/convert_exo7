\uuid{v2rn}
\exo7id{4448}
\auteur{quercia}
\organisation{exo7}
\datecreate{2010-03-14}
\isIndication{false}
\isCorrection{true}
\chapitre{Série numérique}
\sousChapitre{Autre}

\contenu{
\texte{

}
\begin{enumerate}
    \item \question{Prouver la convergence de la série de terme général
    $u_n = \frac1{n\ln^2 n}$.}
    \item \question{On note $S_n = \sum_{k=2}^n u_k$ et $S = \sum_{k=2}^\infty u_k$.
    Montrer que $\frac1{\ln(n+1)} \le S-S_n \le \frac1{\ln n}$ pour $n \ge 2$.}
    \item \question{Montrer que si $S_n$ est une valeur approchée de $S$ à $10^{-3}$ près
    alors $n > 10^{434}$.}
    \item \question{On suppose disposer d'une machine calculant un million de termes de la
    série par seconde avec 12 chiffres significatifs. Peut-on obtenir une valeur
    approchée de $S$ à $10^{-3}$ près ?
    (Remarque : 1 an $\approx$ 32 millions de secondes)}
    \item \question{Donner une valeur approchée de $S$ à $10^{-3}$ près.}
\reponse{
$S_n + \frac1{\ln(n+1)} \le S \le S_n + \frac1{\ln n}$.
             Pour $n = 60$ : $2.06857 < S < 2.06956$.
}
\end{enumerate}
}
