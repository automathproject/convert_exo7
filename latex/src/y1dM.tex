\uuid{y1dM}
\exo7id{6658}
\auteur{queffelec}
\organisation{exo7}
\datecreate{2011-10-16}
\isIndication{false}
\isCorrection{false}
\chapitre{Fonction logarithme et fonction puissance}
\sousChapitre{Fonction logarithme et fonction puissance}

\contenu{
\texte{

}
\begin{enumerate}
    \item \question{Montrer que $\Re(\cos z)>0$ si $|\Re z|<{\pi\over2}$. En déduire une
détermination holo\-morphe du logarithme de  $\cos z$ dans
$\{|\Re z|<{\pi\over2}\}$.}
    \item \question{Montrer que l'on peut définir une fonction holomorphe $f(z)=\hbox{Log}
{1+iz\over 1-iz}$ sur l'ouvert $U={\Cc}\backslash S$ où $S=\{ix\ ;\
|x|\geq1\}$.}
    \item \question{Montrer que l'on peut définir une fonction holomorphe $f(z)=\hbox{Log}
\sqrt{z^3-1}$ sur un ouvert $U$ à déterminer (où $\sqrt{}$ désigne la
détermination principale de la racine).}
\end{enumerate}
}
