\uuid{OAcf}
\exo7id{5954}
\auteur{tumpach}
\organisation{exo7}
\datecreate{2010-11-11}
\isIndication{false}
\isCorrection{true}
\chapitre{Espace L^p}
\sousChapitre{Espace Lp}

\contenu{
\texte{

}
\begin{enumerate}
    \item \question{Soit $a, b \geq 0$ et soit $p,q \in (1, +\infty)$ tel que
$\frac{1}{p} + \frac{1}{q} = 1$ \emph{(on dit que $p$ et $q$ sont
conjugu\'es au sens de Young)}.  Montrer l'in\'egalit\'e de
Young~:
$$
ab \leq \frac{1}{p} a^p + \frac{1}{q} b^q.
$$
On pourra consid\'erer la fonction
$\theta~:\mathbb{R}^{+}\rightarrow \mathbb{R}$ d\'efinie par
$\theta(a) = \frac{1}{p} a^{p} + \frac{1}{q} b^{q} - ab$.}
\reponse{Soit $a, b \geq 0$ et soit $p, q \in(1, +\infty)$ tel que
$\frac{1}{p} + \frac{1}{q} = 1$. La fonction
$\theta~:\mathbb{R}^{+}\rightarrow \mathbb{R}$ d\'efinit par
$\theta(a) = \frac{1}{p} a^{p} + \frac{1}{q} b^{q} - ab$ est
d\'erivable et~:
$$
\theta'(a) =  a^{p-1} - b.
$$
Cette d\'eriv\'ee s'annule lorsque $a = b^{\frac{1}{p-1}}$, est
n\'egative pour  $a < b^{\frac{1}{p-1}}$ et positive pour $a >
b^{\frac{1}{p-1}}$. On a
$$
\theta(b^{\frac{1}{p-1}}) = \frac{1}{p} b^{\frac{p}{p-1}} +
\frac{1}{q}b^q- b^{1+\frac{1}{p-1}} = 0.
$$
Ainsi $\theta(a)\geq 0$, i.e. $$ ab \leq \frac{1}{p} a^p +
\frac{1}{q} b^q.
$$
%L'\'egalit\'e correspond \`a $\theta(a) = 0$, c'est-\`a-dire \`a
%$a = b^{\frac{1}{p-1}}$.}
    \item \question{Soit de nouveau $p,q \in (1, +\infty)$ tel que $\frac{1}{p}
+ \frac{1}{q} = 1$ et $f\in L^{p}(\mu)$, $g\in L^{q}(\mu)$. En
utilisant la question pr\'ec\'edente, montrer que pour tout
$\lambda>0$
$$
\int_{\Omega} |fg|\,d\mu \leq \frac{\lambda^{p}}{p}\int_{\Omega}
|f|^{p}\,d\mu + \frac{\lambda^{-q}}{q} \int_{\Omega}
|g|^{q}\,d\mu.
$$
Optimiser cette in\'egalit\'e par rapport \`a $\lambda$ et montrer
l'in\'egalit\'e de H\"older~:
$$
\| fg\|_{1} \leq \|f\|_{p} \,\|g\|_{q}.
$$
 Cette
in\'egalit\'e est-elle vraie pour $p=1$ et $q=+\infty$ ?}
\reponse{Soit $f\in L^{p}(\mu)$ et $g\in L^{q}(\mu)$. D'apr\`es la
question pr\'ec\'edente,  pour tout $\lambda>0$ et pour
$\mu$-presque tout $x$~:
$$
 |f\,g|(x) = |\lambda f(x) \cdot \frac{g(x)}{\lambda}| \leq \frac{\lambda^{p}}{p}
|f(x)|^{p} + \frac{\lambda^{-q}}{q}  |g(x)|^{q}.
$$
Ainsi
$$
\int_{\Omega} |f\,g|\,d\mu \leq \frac{\lambda^{p}}{p}\int_{\Omega}
|f|^{p}\,d\mu + \frac{\lambda^{-q}}{q} \int_{\Omega}
|g|^{q}\,d\mu.
$$
Posons $$\Phi(\lambda) = \frac{\lambda^{p}}{p}\int_{\Omega}
|f|^{p}\,d\mu + \frac{\lambda^{-q}}{q} \int_{\Omega}
|g|^{q}\,d\mu.$$ La fonction $\Phi$ est d\'erivable et~:
$$
\Phi'(\lambda) = \lambda^{p-1} \|f\|_{p}^{p} -
\lambda^{-q-1}\|g\|_{q}^{q}.
$$
Cette d\'eriv\'ee s'annule pour $\lambda_{1} :=
\left(\frac{\|g\|_{q}^{q}}{\|f\|_{p}^{p}}
\right)^{\frac{1}{p+q}}$, est n\'egative pour $\lambda\leq
\lambda_{1} $ et positive pour $\lambda \geq \lambda_{1}$. Ainsi
le minimum de $\Phi$ vaut~:
\begin{equation*}
\begin{array}{ll}
\Phi(\lambda_{1}) &=
\frac{1}{p}\left(\frac{\|g\|_{q}^{q}}{\|f\|_{p}^{p}}
\right)^{\frac{p}{p+q}} \|f\|_{p}^{p} + \frac{1}{q}
\left(\frac{\|g\|_{q}^{q}}{\|f\|_{p}^{p}}
\right)^{-\frac{q}{p+q}} \|g\|_{q}^{q}\\
& =
\frac{1}{p}\|g\|_{q}^{\frac{qp}{p+q}}\,\|f\|_{p}^{\frac{qp}{p+q}}
+
\frac{1}{q}\|g\|_{q}^{\frac{qp}{p+q}}\,\|f\|_{p}^{\frac{qp}{p+q}}
= \| f\|_{p}\,\|g\|_{q}.
\end{array}
\end{equation*}
On en d\'eduit l'in\'egalit\'e de H\"older~:
$$
\| f\,g\|_{1} \leq \|f\|_{p} \,\|g\|_{q}.
$$
Si $f\in L^{1}(\mu)$ et $g\in L^{\infty}(\mu)$, alors  $|g(x)|
\leq \|g\|_{\infty}$ pour presque tout $x\in\Omega$ et
$$
\int_{\Omega} |f g|\,d\mu \leq \|g\|_{\infty}\int_{\Omega}
|f|\,d\mu,
$$
i.e. $\|f g\|_{1}\leq \|g\|_{\infty} \|f\|_{1}$.}
    \item \question{Soient $p$ et  $p'$ dans $[1, +\infty[$ (pas
n\'ecessairement conjugu\'es). Montrer que si $f$ appartient \`a $
L^{p}(\mu) \cap L^{p'}(\mu)$, alors $f$ appartient \`a $
L^{r}(\mu)$ pour tout $r$ compris entre $p$ et $p'$.}
\reponse{Soient $p, p' \in [1, +\infty)$. On suppose $p< p'$. Soit
$p<r<p'$. On a
$$
|f|^{r} = |f|^r \mathbf{1}_{|f|>1} + |f|^r \mathbf{1}_{|f|<1} ~\leq~ |f|^{p'}
\mathbf{1}_{|f|>1} + |f|^{p}\mathbf{1}_{|f|<1}.
$$
On en d\'eduit que
$$
\int_{\Omega} |f|^{r}\,d\mu ~\leq~ \int_{\Omega}|f|^{p'}\,d\mu +
\int_{\Omega} |f|^{p}\,d\mu ~<+\infty,
$$
donc $f$ appartient \`a $L^{r}(\mu)$.}
    \item \question{Montrer que si $\mu$ est une mesure finie alors
$$L^{\infty}(\mu) \subset \bigcap_{p\geq 1} L^{p}(\mu),
$$
et, pour tout $f$,
$$
\lim_{p\rightarrow+\infty}\|f\|_{p} = \|f\|_{\infty}.
$$}
\reponse{Supposons que  $\mu$ soit une mesure finie et soit $f\in
L^{\infty}(\mu)$. Alors
$$
|f(x)| \leq \|f\|_{\infty}
$$
pour presque tout $x\in\Omega$. Ainsi pour tout $p$
$$
\int_{\Omega} |f|^{p}\,d\mu ~\leq~\|f\|_{\infty}^{p}\int_{\Omega}
1\,d\mu ~=~ \|f\|_{\infty}^{p} \mu(\Omega)~< ~+\infty,
$$
ce qui implique que $f\in L^{p}(\mu)$. En particulier, $f$
appartient \`a l'intersection $\bigcap_{p\geq 1} L^{p}(\mu)$. De
plus, pour tout $p$, on a~:
$$
\|f\|_{p} ~\leq ~\|f\|_{\infty} ~\mu(\Omega)^{\frac{1}{p}},
$$
ce qui implique que
$$
\lim_{p\rightarrow+\infty}\|f\|_{p}~ \leq ~\|f\|_{\infty}.
$$
D'autre part, pour tout $0<\varepsilon< \| f \|_\infty$ , on a
$$
\int_{\Omega} |f|^{p}\,d\mu ~\geq~
\int_{|f|>\left(\|f\|_{\infty}-\varepsilon\right)}|f|^{p}\,d\mu
~\geq ~\left(\|f\|_{\infty}-\varepsilon\right)^{p} \mu\bigg(
|f|>\left(\|f\|_{\infty}-\varepsilon\right)\bigg).
$$
Ainsi pour tout $p$, il vient
$$
\|f\|_{p} ~\geq~ \left(\|f\|_{\infty}-\varepsilon\right) \mu\bigg(
|f|>\left(\|f\|_{\infty}-\varepsilon\right)\bigg)^{\frac{1}{p}}.
$$
Puisque $\lim_{p\rightarrow +\infty}\mu\left(
|f|>\left(\|f\|_{\infty}-\varepsilon\right)\right)^{\frac{1}{p}} =
1$, il en d\'ecoule que
$$
\lim_{p\rightarrow +\infty}\|f\|_{p} ~\geq~
\|f\|_{\infty}-\varepsilon.
$$
Comme $\varepsilon$ peut \^etre choisi arbitrairement petit, on a
$$
\lim_{p\rightarrow +\infty}\|f\|_{p} ~\geq~ \|f\|_{\infty},
$$
donc finalement $\lim_{p\rightarrow +\infty}\|f\|_{p} =
\|f\|_{\infty}$.}
    \item \question{Montrer que si  $f \in L^p(\mu)$ et
$g\in L^q(\mu)$ avec $\frac 1p + \frac 1q = \frac 1r$, alors
$f\cdot g\in L^r(\mu)$ et
$$
 \| fg\|_{r} \leq \|f\|_{p}\|g\|_{q}.
$$}
\reponse{Posons $f_1:=f^r$ et $g_1:=g^r$. On a $f_1\in
L^{\frac{p}{r}}(\mu)$ et $g_1\in L^{\frac{q}{r}}(\mu)$. Notons que
l'identit\'{e} $\frac{1}{p}+\frac{1}{q}=\frac{1}{r}$ entra\^{i}ne
que $\frac{p}{r}, \frac{q}{r}>1$ et que les nombres $\frac{p}{r}$
et $ \frac{q}{r}$ sont conjugu\'{e}s au sens de Young. Par
l'in\'{e}galit\'{e} de H\"{o}lder on a donc
$$\int_{\Omega}(fg)^r d\mu = \int_{\Omega} f_1 g_1 \,d\mu \leq
\left(\int_{\Omega} f_1^{\frac{p}{r}}d\mu\right)^{\frac{r}{p}}
\left(\int_{\Omega} g_1^{\frac{q}{r}}d\mu\right)^{\frac{r}{q}}
=\left(\int_{\Omega} f^p
d\mu\right)^{\frac{r}{p}}\left(\int_{\Omega} g^q
d\mu\right)^{\frac{r}{q}}.$$ D'o\`{u}, finalement,
 $$\|fg\|_r\leq \|f\|_p\|g\|_q.$$}
\end{enumerate}
}
