\uuid{a1aJ}
\exo7id{3324}
\auteur{quercia}
\organisation{exo7}
\datecreate{2010-03-09}
\isIndication{false}
\isCorrection{false}
\chapitre{Groupe, anneau, corps}
\sousChapitre{Algèbre, corps}

\contenu{
\texte{
Soient $ K, \mathbb{L}$ deux corps avec $ K \subset \mathbb{L}$.

Un élément $\alpha \in \mathbb{L}$ est dit algébrique sur $ K$ s'il existe
un polynôme non nul $P \in  K[X]$ tel que $P(\alpha) = 0$.
}
\begin{enumerate}
    \item \question{Montrer que $\alpha$ est algébrique sur $ K$ si et seulement si $ K[\alpha]$ est
      un $ K$-ev de dimension finie.}
    \item \question{On suppose que $\alpha$ et $\beta$ sont algébriques sur $ K$.
      Montrer que $\alpha + \beta$ et $\alpha\beta$ sont algébriques sur $ K$
      (étudier $ K[\alpha,\beta]$).}
\end{enumerate}
}
