\uuid{0IB7}
\exo7id{7555}
\auteur{mourougane}
\organisation{exo7}
\datecreate{2021-08-10}
\isIndication{false}
\isCorrection{false}
\chapitre{Théorème des résidus}
\sousChapitre{Théorème des résidus}

\contenu{
\texte{
On paramètre le cercle $C_r$ de centre $0$ et de rayon $r>0$ orienté dans le sens trigonométrique 
du plan complexe en définissant pour $t\in\Rr$, $\xi(t)$ 
comme le point d'intersection de la droite d'équation $y=t(x+r)$ avec le cercle $C_r$ différent de $-r$.
}
\begin{enumerate}
    \item \question{Montrer que $\xi(t)=r\frac{1+it}{1-it}$.}
    \item \question{Vérifier que $\xi$ est dérivable sur $\Rr$ et calculer $\frac{\xi'(t)}{\xi(t)}$.}
    \item \question{En déduire que $$\int_{C_r} \frac{dz}{z}=2i\int_{-\infty}^{+\infty}\frac{dt}{1+t^2}.$$}
    \item \question{En déduire que $$\int_{-\infty}^{+\infty}\frac{dt}{1+t^2}=\pi.$$}
\end{enumerate}
}
