\uuid{USTX}
\exo7id{6864}
\auteur{bodin}
\organisation{exo7}
\datecreate{2012-04-13}
\isIndication{true}
\isCorrection{true}
\chapitre{Calcul d'intégrales}
\sousChapitre{Intégration par parties}

\contenu{
\texte{
Calculer les primitives suivantes par intégration par parties.
}
\begin{enumerate}
    \item \question{$\int x^2 \ln x \, dx$}
\reponse{$\int x^2 \ln x \, dx$

Considérons l'intégration par parties avec $u=\ln x$ et $v'=x^2$.
On a donc $u'=\frac 1x$ et $v = \frac{x^3}3$.
Donc
\begin{align*}
\int  \ln x \times x^2\, dx 
  &= \int uv' = \big[ uv \big] - \int u'v \\
  &= \left[ \ln x \times \frac{x^3}3 \right] - \int  \frac 1x\times\frac{x^3}3  \, dx \\
  &= \left[ \ln x  \times \frac{x^3}3\right] - \int \frac{x^2}3 \, dx \\
  &= \frac{x^3}3 \ln x - \frac{x^3}9 + c \\
\end{align*}}
    \item \question{$\int x \arctan x \, dx$}
\reponse{$\int x \arctan x \, dx$

Considérons l'intégration par parties avec $u=\arctan x$ et $v'=x$.
On a donc $u'=\frac 1{1+x^2}$ et $v = \frac{x^2}2$.
Donc
\begin{align*}
\int  \arctan x \times x \, dx 
&= \int uv' = \big[ uv \big] - \int u'v \\
&= \left[ \arctan x \times \frac{x^2}2  \right] - \int \frac 1{1+x^2} \times\frac{x^2}2  \, dx  \\
&= \left[  \arctan x \times \frac{x^2}2\right] - \frac12 \int \left( 1 -  \frac 1{1+x^2} \right)\, dx  \\
&= \frac{x^2}2  \arctan x -\frac 12 x + \frac 12 \arctan x + c \\
&= \frac 12 (1+x^2) \arctan x -\frac 12 x+ c \\
\end{align*}}
    \item \question{$\int \ln x \, dx$ \quad  puis \quad  $\int (\ln x)^2 \, dx$}
\reponse{$\int \ln x \, dx$ puis $\int (\ln x)^2 \, dx$

Pour la primitive $\int \ln x \, dx$, regardons l'intégration par parties avec $u=\ln x$ et $v'=1$.
Donc $u' = \frac 1x$ et $v=x$.
\begin{align*}
\int \ln x \, dx
&= \int uv' = \big[ uv \big] - \int u'v \\
&= \left[ \ln x \times x \right] - \int \frac 1x \times x \, dx \\
&= \left[  \ln x \times x \right] - \int 1 \, dx \\
&= x\ln x - x + c \\
\end{align*}

\bigskip

Par la primitive $\int (\ln x)^2 \, dx$ soit l'intégration par parties définie par $u=(\ln x)^2$ et  $v'=1$.
Donc $u' = 2 \frac 1x \ln x$ et $v=x$.
\begin{align*}
\int (\ln x)^2 \, dx
&= \int uv' = \big[ uv \big] - \int u'v \\
&= \left[ x (\ln x)^2 \right] - 2 \int \ln x \, dx \\
&= x(\ln x)^2 -2 (x\ln x - x) + c \\
\end{align*}
Pour obtenir la dernière ligne on a utilisé la primitive calculée précédemment.}
    \item \question{$\int \cos x\exp x \, dx$}
\reponse{Notons $I=\int \cos x\exp x \, dx$.

Regardons l'intégration par parties avec $u=\exp x$ et $v'=\cos x$.
Alors  $u' = \exp x$ et $v=\sin x$.

Donc

$$I = \int \cos x\exp x \,dx= \big[ \sin x \exp x \big] - \int \sin x \exp x\,dx$$

Si l'on note $J = \int \sin x \exp x\,dx$, alors on a obtenu
\begin{equation}
\label{eq:intI}
I = \big[ \sin x \exp x \big] - J  
\end{equation}
 
Pour calculer $J$ on refait une deuxième intégration par parties
avec  $u=\exp x$ et $v'=\sin x$.
Ce qui donne
$$J = \int \sin x \exp x\,dx = \big[ -\cos x \exp x \big] - \int -\cos x \exp x\,dx
= \big[ -\cos x \exp x \big] + I$$
Nous avons ainsi une deuxième équation :
\begin{equation}
\label{eq:intJ}
J = \big[ -\cos x \exp x \big] + I
\end{equation}

Repartons de l'équation (\ref{eq:intI}) dans laquelle on remplace $J$ par la formule obtenue dans l'équation 
(\ref{eq:intJ}).


$$I=\big[ \sin x \exp x \big] - J  = \big[ \sin x \exp x \big] - \big[ -\cos x \exp x \big] - I$$
D'où
$$2I = \big[ \sin x \exp x \big] + \big[ \cos x \exp x \big]$$

Ce qui nous permet de calculer notre intégrale :
$$I= \frac12 (\sin x + \cos x) \exp x + c.$$}
\indication{\begin{enumerate}
  \item Pour $\int x^2 \ln x \, dx$ poser $v'=x^2$, $u=\ln x$.

  \item Pour $\int x \arctan x \, dx$ poser $v'=x$ et $u= \arctan x$.

  \item Pour les deux il faut faire une intégration par parties avec $v'=1$.

  \item Pour $\int \cos x\exp x \,dx$ il faut faire deux intégrations par parties.
\end{enumerate}}
\end{enumerate}
}
