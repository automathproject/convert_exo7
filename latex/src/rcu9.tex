\uuid{rcu9}
\exo7id{6688}
\auteur{queffelec}
\organisation{exo7}
\datecreate{2011-10-16}
\isIndication{false}
\isCorrection{false}
\chapitre{Formule de Cauchy}
\sousChapitre{Formule de Cauchy}

\contenu{
\texte{

}
\begin{enumerate}
    \item \question{Montrer qu'il existe sur $\C\setminus [-1,1]$ une détermination
holomorphe de $(z^2-1)^{-1/2}$ qui prend la valeur 1 pour $z=\sqrt 2$.
Unicité ? On désignera par $f$ cette détermination.}
    \item \question{Pour $z\in \C\setminus{\Rr}$, on désigne par $\gamma _z$ le
segment $\mathopen] 0,z]$ orienté de 0 vers $z$ et on pose
$$F(z)=\int_{\gamma _z}f(\zeta )\ d\zeta .$$

\begin{enumerate}}
    \item \question{Montrer que $F$ est holomorphe sur $\C\setminus {\Rr}$ (quelle en
est la dérivée $F'$ ?).}
    \item \question{Etudier $\lim_{z'\to z}F(z')$ lorsque $z\in {\Rr}\setminus [-1,1]$.}
    \item \question{En déduire que $f$ n'a pas de primitive sur $\C\setminus [-1,1]$.}
\end{enumerate}
}
