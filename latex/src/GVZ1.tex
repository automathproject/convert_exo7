\uuid{GVZ1}
\exo7id{5136}
\auteur{rouget}
\organisation{exo7}
\datecreate{2010-06-30}
\isIndication{false}
\isCorrection{true}
\chapitre{Nombres complexes}
\sousChapitre{Autre}

\contenu{
\texte{
Pour $z\in\Cc$, on pose $\ch z=\frac{1}{2}(e^z+e^{-z})$, $\sh z=\frac{1}{2}(e^z-e^{-z})$ et $\tanh z=\frac{\sh z}{\ch
z}$.
}
\begin{enumerate}
    \item \question{Quels sont les nombres complexes $z$ pour lesquels $\tanh z$ existe~?}
\reponse{Soit $z\in\Cc$. $\sh z$ et $\ch z$ sont définis et donc, $\tanh z$ existe si et seulement si $\ch z\neq0$. Or,

$$\ch z=0\Leftrightarrow e^z+e^{-z}=0\Leftrightarrow e^{2z}=-1\Leftrightarrow e^{2z}=e^{i\pi}\Leftrightarrow2z\in i\pi+2i\pi\Zz\Leftrightarrow z\in i\left(\frac{\pi}{2}+\pi\Zz\right).$$
$\tanh z$ existe si et seulement si $z\notin i\left(\frac{\pi}{2}+\pi\Zz\right)$.}
    \item \question{Résoudre dans $\Cc$ l'équation $\tanh z=0$.}
\reponse{Soit $z\notin i\left(\frac{\pi}{2}+\pi\Zz\right)$.

$$\tanh z=0\Leftrightarrow\sh z=0\Leftrightarrow e^z=e^{-z}\Leftrightarrow e^{2z}=1\Leftrightarrow 2z\in2i\pi\Zz\Leftrightarrow z\in i\pi\Zz.$$
Comme $i\left(\frac{\pi}{2}+\pi\Zz\right)\cap i\pi\Zz=\varnothing$, $\tanh z=0$ si et seulement si $z\in i\pi\Zz$.}
    \item \question{Résoudre dans $\Cc$ le système $\left\{
\begin{array}{l}
|\Im z|<\frac{\pi}{2}\\
|\tanh z|<1
\end{array}\right.$.}
\reponse{Soit $z\notin i\left(\frac{\pi}{2}+\pi\Zz\right)$. Posons $z=x+iy$ où $(x,y)\in\Rr^2$.

\begin{align*}
|\tanh z|<1&\Leftrightarrow|e^z-e^{-z}|^2<|e^z+e^{-z}|^2\Leftrightarrow(e^z-e^{-z})(e^{{\bar z}}-e^{-{\bar z}})<(e^z+e^{-z})(e^{{\bar z}}+e^{-{\bar z}})\\
 &\Leftrightarrow-e^{z-{\bar z}}-e^{-(z-{\bar z})}<e^{z-{\bar z}}+e^{-(z-{\bar z})}\Leftrightarrow2(e^{2iy}+e^{-2iy})>0\\
 &\Leftrightarrow\cos(2y)>0
\end{align*}
Par suite,

$$\left\{
\begin{array}{l}
|\Im z|<\frac{\pi}{2}\\
|\tanh z|<1
\end{array}
\right.
\Leftrightarrow
\left\{
\begin{array}{l}
|y|<\frac{\pi}{2}\\
\cos(2y)>0
\end{array}
\right.
\Leftrightarrow
|y|<\frac{\pi}{4}\Leftrightarrow z\in\Delta.$$}
    \item \question{Montrer que la fonction $\tanh$ réalise une bijection de $\Delta=\{z\in\Cc/\;|\Im z|<\frac{\pi}{4}\}$ sur
$U=\{z\in\Cc/\;|z|<1\}$.}
\reponse{Soit $z\in\Delta$.
D'après 1), $\tanh z$ existe et d'après 3), $|\tanh z|<1$. Donc $z\in\Delta\Rightarrow\tanh z\in U$. Ainsi, $\tanh$ est une
application de $\Delta$ dans $U$.
Soit alors $Z\in U$ et $z\in\Delta$.

$$\tanh z=Z\Leftrightarrow\frac{e^{2z}-1}{e^{2z}+1}=Z\Leftrightarrow e^{2z}=\frac{1+Z}{1-Z}.$$
Puisque $Z\neq-1$, $\frac{1+Z}{1-Z}\neq0$ et on peut poser $\frac{1+Z}{1-Z}=re^{i\theta}$ où $r\in\Rr_+^*$ et
$\theta\in]-\pi,\pi]$.

Par suite,

\begin{align*}
e^{2z}=\frac{1+Z}{1-Z}&\Leftrightarrow e^{2z}=re^{i\theta}\Leftrightarrow e^{2x}=r\;\mbox{et}\;2y\in\theta+2\pi\Zz\\
 &\Leftrightarrow x=\frac{1}{2}\ln r\;\mbox{et}\;y\in\frac{\theta}{2}+\pi\Zz
\end{align*}
Maintenant, on ne peut avoir $\theta=\pi$. Dans le cas contraire, on aurait $\frac{1+Z}{1-Z}=-r\in\Rr_-^*$ puis
$Z=\frac{r+1}{r-1}\in\Rr$. Par suite, puisque $|Z|<1$, on aurait $Z\in]-1,1[$ et donc $\frac{1+Z}{1-Z}\in\Rr_+^*$ ce
qui est une contradiction. Donc, $\theta\in]-\pi,\pi[$ puis $\frac{\theta}{2}\in]-\frac{\pi}{2},\frac{\pi}{2}[$.
Mais alors,

$$\left\{
\begin{array}{l}
\tanh z=Z\\
z\in\Delta
\end{array}
\right.
\Leftrightarrow\left\{
\begin{array}{l}
x=\frac{1}{2}\ln r\\
y=\frac{\theta}{2}\rule{0mm}{6mm}
\end{array}
\right..$$
Ainsi, tout élément $Z$ de $U$ a un et un seul antécédent $z$ dans $\Delta$ (à savoir
$z=\frac{1}{2}\ln\left|\frac{1+Z}{1-Z}\right|+\frac{i}{2}\mbox{Arg}\left(\frac{1+Z}{1-Z}\right)$ où 
$\mbox{Arg}\left(\frac{1+Z}{1-Z}\right)$ désigne l'argument de $\frac{1+Z}{1-Z}$ qui est dans $]-\pi,\pi[$).
Finalement, $\tanh$ réalise donc une bijection de $\Delta$ sur $U$.}
\end{enumerate}
}
