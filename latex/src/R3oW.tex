\uuid{R3oW}
\exo7id{5067}
\auteur{rouget}
\organisation{exo7}
\datecreate{2010-06-30}
\isIndication{false}
\isCorrection{true}
\chapitre{Nombres complexes}
\sousChapitre{Trigonométrie}

\contenu{
\texte{
Calculer $\cos\frac{\pi}{8}$ et $\sin\frac{\pi}{8}$.
}
\reponse{
$\cos^2\frac{\pi}{8}=\frac{1}{2}\left(1+\cos(2\times\frac{\pi}{8})\right)=\frac{1}{2}\left(1+\frac{\sqrt{2}}{2}\right)=\frac{2+\sqrt{2}}{4}$,
et puisque $\cos\frac{\pi}{8}>0$,

\begin{center}
\shadowbox{$\cos\frac{\pi}{8}=\frac{1}{2}\sqrt{2+\sqrt{2}}$.}
\end{center}
De même, 
puisque $\sin\frac{\pi}{8}>0$, $\sin\frac{\pi}{8}=\sqrt{\frac{1}{2}\left(1-\cos(2\times\frac{\pi}{8})\right)}$ 
et
\begin{center}
\shadowbox{
$\sin\frac{\pi}{8}=\frac{1}{2}\sqrt{2-\sqrt{2}}$.
}
\end{center}
}
}
