\uuid{r33A}
\exo7id{476}
\auteur{bodin}
\organisation{exo7}
\datecreate{1998-09-01}
\isIndication{true}
\isCorrection{true}
\chapitre{Propriétés de R}
\sousChapitre{Maximum, minimum, borne supérieure}

\contenu{
\texte{
Soient $A$ et $B$ deux parties born\'ees de $\Rr$.
On note $A+B = \{ a+b \mid (a,b)\in A\times B \}$.
}
\begin{enumerate}
    \item \question{Montrer que $\sup A + \sup B$ est un majorant de $A+B$.}
    \item \question{Montrer que $\sup(A+B)=\sup A + \sup B$.}
\reponse{
Soient $A$ et $B$ deux parties born\'ees de $\R$.
On sait que $\sup A$ est un majorant de $A$, c'est-\`a-dire,
pour tout $a\in A$, $a\leqslant \sup A$. De m\^eme, pour tout $b\in B$, $b\le
\sup B$. On veut montrer que $\sup A+\sup B$ est un majorant de
$A+B$. Soit donc $x\in A+B$. Cela signifie que $x$ est de la forme
$a+b$ pour un $a\in A$ et un $b\in B$. Or $a\leqslant \sup A$, et $b \le
\sup B$, donc $x=a+b\leqslant \sup A+\sup B$. Comme ce raisonnement est
valide pour tout $x\in A+B$ cela signifie que  $\sup A+\sup B$ est
un majorant de $A+B$.
On veut montrer que, quel que soit
$\epsilon>0$, $\sup A +\sup B-\epsilon$ n'est pas un majorant de $A+B$. On
prend donc un $\epsilon >0$ quelconque, et on veut montrer que $\sup A
+\sup B-\epsilon$ ne majore pas $A+B$. On s'interdit donc dans la
suite de modifier $\epsilon$. Comme $\sup A$ est le plus petit des
majorants de $A$, $\sup A-\epsilon/2$ n'est pas un majorant de $A$.
Cela signifie qu'il existe un \'el\'ement $a$ de $A$ tel que
$a>\sup A-\epsilon/2$. {\em Attention: $\sup A-\epsilon/2$ n'est pas
forc\'ement dans $A$ ; $\sup A$ non plus.} De la m\^eme mani\`ere, il existe $b\in B$ tel que
$b>\sup B-\epsilon/2$. Or l'\'el\'ement $x$ d\'efini par $x=a+b$ est
un \'el\'ement de $A+B$, et il v\'erifie $x>(\sup A-\epsilon/2)+(\sup
B-\epsilon/2)=\sup A +\sup B-\epsilon.$ Ceci implique que $\sup A +\sup
B-\epsilon$ n'est pas un majorant de $A+B$.
$\sup A+\sup B$
est un majorant de $A+B$ d'apr\`es la partie 1. Mais, d'apr\`es la
partie 2., d\`es qu'on prend un $\epsilon>0$, $\sup A+\sup B -\epsilon$
n'est pas un majorant de $A+B$. Donc $\sup A+\sup B$ est bien le
plus petit des majorants de $A+B$, c'est donc la borne supérieure de $A+B$. Autrement dit
 $\sup (A+B)= \sup A +\sup B$.
}
\indication{Il faut revenir à la définition de la borne supérieure d'un ensemble borné :
c'est le plus petit des majorants. En particulier la borne supérieure est un majorant.}
\end{enumerate}
}
