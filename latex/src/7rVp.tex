\uuid{7rVp}
\exo7id{7387}
\auteur{mourougane}
\organisation{exo7}
\datecreate{2021-08-10}
\isIndication{false}
\isCorrection{false}
\chapitre{Groupe, anneau, corps}
\sousChapitre{Autre}

\contenu{
\texte{
On admet que $271$ et $281$ sont deux nombres premiers.
}
\begin{enumerate}
    \item \question{Peut-on écrire $271$ comme somme de deux carrés ? Si oui, faîtes le. On pourra trouver une racine $c$ de $-1$ modulo $271$, puis calculer le $\pgcd(271, c+i)$ dans l'anneau $\Z[i]$ des entiers de Gauss.}
    \item \question{Peut-on écrire $281$ comme somme de deux carrés ? Si oui, faîtes le. On pourra trouver une racine $c$ de $-1$ modulo $281$, puis calculer le $\pgcd(281, c+i)$ dans l'anneau $\Z[i]$ des entiers de Gauss.}
\end{enumerate}
}
