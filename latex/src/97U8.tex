\uuid{97U8}
\exo7id{2091}
\auteur{bodin}
\organisation{exo7}
\datecreate{2008-02-04}
\isIndication{false}
\isCorrection{true}
\chapitre{Calcul d'intégrales}
\sousChapitre{Théorie}

\contenu{
\texte{
Soit $f\,:\;\R\to\R$ une fonction continue sur $\R$ et $F(x)=\int_0^x
f(t)d t$. Répondre par vrai ou faux aux affirmations suivantes:
}
\begin{enumerate}
    \item \question{$F$ est continue sur $\R$.}
\reponse{Vrai.}
    \item \question{$F$ est dérivable sur $\R$ de dérivée $f$.}
\reponse{Vrai.}
    \item \question{Si $f$ est croissante sur $\R$ alors $F$ est croissante sur $\R$.}
\reponse{Faux ! Attention aux valeurs négatives par exemple pour $f(x)=x$ alors $F$ est décroissante sur $]-\infty,0]$ et croissante sur $[0,+\infty[$.}
    \item \question{Si $f$ est positive sur $\R$ alors $F$ est positive sur $\R$.}
\reponse{Faux. Attention aux valeurs négatives par exemple pour $f(x)=x^2$ alors $F$ est négative sur $]-\infty,0]$ et positive sur $[0,+\infty[$.}
    \item \question{Si $f$ est positive sur $\R$ alors $F$ est croissante sur $\R$.}
\reponse{Vrai.}
    \item \question{Si $f$ est $T$-périodique sur $\R$ alors $F$ est $T$-périodique sur $\R$.}
\reponse{Faux. Faire le calcul avec la fonction $f(x) = 1+\sin(x)$ par exemple.}
    \item \question{Si $f$ est paire alors $F$ est impaire.}
\reponse{Vrai.}
\end{enumerate}
}
