\uuid{JkAf}
\exo7id{2419}
\auteur{drutu}
\organisation{exo7}
\datecreate{2007-10-01}
\isIndication{false}
\isCorrection{true}
\chapitre{Espace topologique, espace métrique}
\sousChapitre{Espace topologique, espace métrique}

\contenu{
\texte{
Soit $\R$ et soit $\mathcal{T}$ une collection de sous-ensembles de $\R$
contenant $\emptyset$, $\R$ et tous les complementaires
d'ensembles finis. Est-ce une topologie sur $\R$ ? Est-ce une
topologie s\'epar\'ee ?
}
\reponse{
Il faut donc d\'emontrer que la collection de sous-ensembles de
$\R$ contenant $\emptyset$, $\R$ et tous les ensembles finis
v\'erifie les propri\'et\'es d'une collection d'ensembles
ferm\'es~:
\begin{itemize}
    \item toute int\'ersection d'ensembles ferm\'es est ferm\'e;
    \item toute r\'eunion finie d'ensembles ferm\'es est ferm\'e;
    \item $\emptyset$ et tout l'espace sont des ferm\'es.
\end{itemize}

Les trois propri\'et\'es sont \'evidemment v\'erifi\'ees dans ce
cas.


La topologie ainsi d\'efinie sur $\R$ n'est pas s\'epar\'ee. En
effet deux ouverts non-vides $\Omega$ et $\Omega'$ sont sous la
forme $\Omega=\R \setminus F$ et $\Omega'=\R \setminus F'$, o\`u
$F,F'$ sont ou bien finis ou bien vides. Alors $\Omega \cap
\Omega'=\R\setminus \left( F\cup F'\right)$ n'est pas vide, car
sinon ceci impliquerait que $\R = F\cup F'$ est finie ou vide, ce
qui est faux.
}
}
