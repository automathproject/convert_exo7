\uuid{Uo7u}
\exo7id{119}
\auteur{ridde}
\organisation{exo7}
\datecreate{1999-11-01}
\isIndication{true}
\isCorrection{true}
\chapitre{Logique, ensemble, raisonnement}
\sousChapitre{Logique}

\contenu{
\texte{
Montrer que 
$$\forall \epsilon >0 \quad \exists N \in \Nn \text{ tel que }
 (n \geq N \Rightarrow 2-\epsilon < \frac{2n + 1}{n + 2} < 2 + \epsilon).$$
}
\indication{En fait, on a toujours : $\frac{2n+1}{n+2} \leq 2$.
Puis chercher une condition sur $n$ pour que
l'in\'egalit\'e 
$$2-\epsilon < \frac{2n + 1}{n + 2}$$
soit vraie.}
\reponse{
Remarquons d'abord que pour $n \in \Nn$, $\frac{2n+1}{n+2} \leq 2$
car $2n+1 \leq 2(n+2)$.
\'Etant donn\'e $\epsilon > 0$, nous avons donc 
$$\forall n \in \Nn \quad \frac{2n+1}{n+2} < 2 + \epsilon$$
Maintenant nous cherchons une condition sur $n$ pour que
l'in\'egalit\'e 
$$2-\epsilon < \frac{2n + 1}{n + 2}$$
soit vraie.
\begin{align*}
2-\epsilon < \frac{2n + 1}{n + 2} 
    &\Leftrightarrow (2-\epsilon)(n+2) < 2n+1 \\
    &\Leftrightarrow 3  < \epsilon (n+2) \\
   &\Leftrightarrow n >  \frac{3}{\epsilon}-2 \\
\end{align*}

Ici $\epsilon$ nous est donn\'e, nous prenons un $N\in \Nn$ tel
que $N > \frac{3}{\epsilon}-2$, alors pour tout $n \geq N$ nous avons
$n \geq N >  \frac{3}{\epsilon}-2$ et par
cons\'equent: $2-\epsilon < \frac{2n + 1}{n + 2}$.
Conclusion: \'etant donn\'e $\epsilon > 0$, nous avons trouv\'e un 
$N\in \Nn$ tel que pour tout $n \geq N$ on ait
$2-\epsilon < \frac{2n + 1}{n + 2}$ et $\frac{2n+1}{n+2} < 2 + \epsilon$.

En fait nous venons de prouver que la suite 
de terme $(2n+1)/(n+2)$ tend vers $2$ quand $n$ tend vers $+\infty$.
}
}
