\uuid{iZ4V}
\exo7id{5951}
\auteur{tumpach}
\organisation{exo7}
\datecreate{2010-11-11}
\isIndication{false}
\isCorrection{true}
\chapitre{Théorème de convergence dominée}
\sousChapitre{Théorème de convergence dominée}

\contenu{
\texte{
Montrer que
}
\begin{enumerate}
    \item \question{$\displaystyle \lim_{n \to \infty} \int_0^n \left(1-\frac{x}{n}\right)^n
x^{m}dx = m !$ \ (pour tout $m\in \mathbf{N}$).}
\reponse{Pour tout $x\in \mathbb{R}_+$
et $n\in \mathbb{N}$ on a
$$\left(1-\frac{x}{n}\right)^n \leq e^{-x}.$$ En effet, comme \, $\ln
y\leq y-1$ \, pour $y>0$, on a \, $\ln y^{-\frac{1}{n}}\leq
y^{-\frac{1}{n}}-1$,\, c'est-\`a-dire $\left(1 - \frac{\ln y}{n}
\right)^{n} \leq y^{-1}$. Ainsi, en posant $x = \ln y$, il vient
$\left(1-\frac{x}{n}\right)^n \leq e^{-x}$. De plus,
$$
\lim_{n\rightarrow+\infty} \left(1-\frac{x}{n} \right)^{n} =
\lim_{n\rightarrow+\infty}e^{n\ln\left(1-\frac{x}{n} \right)} =
\lim_{n\rightarrow+\infty}e^{n\left(-\frac{x}{n} +
\frac{x}{n}\varepsilon\left(\frac{x}{n} \right) \right)},
$$
o\`u $\lim_{u\rightarrow 0}\varepsilon(u) = 0$. Ainsi
$\lim_{n\rightarrow+\infty} \left(1-\frac{x}{n} \right)^{n} =
e^{-x}$.


Posons $f_n(x)=\left(1-\frac{x}{n}\right)^n x^m\mathbf{1}_{[0,n]}$. Alors
en utilisant le th\'{e}or\`{e}me de convergence domin\'{e}e et
sachant que \, $ \Gamma(m+1)=\int\limits_0^\infty e^{-x}x^m dx
=m!$, on obtient le r\'{e}sultat.}
    \item \question{$\displaystyle \lim_{n \to \infty} \int_0^n \left(1+\frac{x}{n}\right)^n
e^{-2x}dx = 1$.}
\reponse{Soit $f_n(x)=\left(1+\frac{x}{n}\right)^n e^{-2x}\mathbf{1}_{[0,n]}$.
Comme la suite $\{f_n(x)\}$ est croissante et \,
$\lim\limits_{n\rightarrow \infty}f_n(x)=e^{-x}$, on obtient le
r\'{e}sultat en appliquant le th\'{e}or\`{e}me de convergence
monotone.}
\end{enumerate}
}
