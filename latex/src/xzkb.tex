\uuid{xzkb}
\exo7id{2793}
\auteur{burnol}
\organisation{exo7}
\datecreate{2009-12-15}
\isIndication{false}
\isCorrection{true}
\chapitre{Fonction holomorphe}
\sousChapitre{Fonction holomorphe}

\contenu{
\texte{
Cet exercice propose une variante pour développer la
théorie de la fonction exponentielle.\medskip
}
\begin{enumerate}
    \item \question{On se donne une fonction $f$ qui est $n+1$-fois dérivable au sens
  complexe  sur le disque ouvert $D(0,R)$ (on
  sait qu'une fois suffit mais on ne va pas utiliser ce
  théorème difficile ici). Soit
  $z\in D(0,R)$. En appliquant la formule de Taylor avec
  reste intégral de Lagrange à la fonction de la variable
  réelle $t\mapsto g(t) = f(tz)$ pour $0\leq t\leq 1$,
  prouver:
\[ f(z) = f(0) + f'(0) z + \frac{f^{(2)}(0)}2 z^2 + \dots +
  \frac{f^{(n)}(0)}{n!} z^n + z^{n+1}\int_0^1
  \frac{(1-t)^{n}}{n!} f^{(n+1)}(tz)\,dt\]}
\reponse{La formule de Taylor avec reste int\'egral est
$$g(b)= g(a) +g'(a)(b-a) +...+ \frac{g^{(n)} (a)}{n!} (b-a)^n+ \int_a^b \frac{(b-t)^n}{n!} g^{(n+1)}(t)\, dt$$
puis on remplace avec $a=0$ et $b=1$.}
    \item \question{On suppose que $f$ est dérivable au sens complexe une fois sur
  $D(0,R)$ et vérifie $f'=f$ et $f(0) = 1$. Montrer que $f$
  est infiniment dérivable au sens complexe. En utilisant
  la question précédente montrer :
\[ \left| f(z) - \sum_{k=0}^n \frac{z^k}{k!}\right| \leq
   (\sup_{|w|\leq |z|} |f(w)|)
   \frac{|z|^{n+1}}{(n+1)!}\]
et en déduire que, pour tout $z\in \Cc$ on a : $f(z) = \sum_{k=0}^\infty \frac{z^k}{k!}$.}
\reponse{Si $f'=f$ et si $f$ est $n$--fois d\'erivable au sens complexe, alors
$\lim_{h \to 0}(f^{(n)} (z+h) -f^{(n)} (z))/h
=\lim_{h \to 0}(f^{(n-1)} (z+h) -f^{(n-1)} (z))/h=f^{(n)}(z)$.
Par r\'ecurrence on en d\'eduit, d'une part, que $f$ est infiniment d\'erivable et, d'autre part, que $f^{(n)}(z)=f(z)$
pour tout $n\geq 0$. En particulier, $f^{(n)} (0)=1$ pour tout $n\geq 0$. En utilisant la formule de Taylor de la question précédente
on a donc
$$\begin{aligned}
\left| f(z) - \sum_{k=0}^n \frac{z^k}{k!} \right| &\leq |z|^{n+1} \int_0^1 \frac{(1-u)^n}{n!} |f^{(n+1)} (uz)| \, du\\
&\leq |z|^{n+1} \sup_{|w|\leq |z|} |f^{(n+1)} (w)|\int_0^1 \frac{(1-u)^n}{n!}\, du
\leq \sup_{|w|\leq |z|} |f (w)|\frac{|z|^{n+1} }{(n+1)!}.
\end{aligned}$$
Cette derni\`ere expression tend vers $0$ lorsque $n\to \infty$. D'o\`u $f(z)=\sum_{k=0}^\infty \frac{z^k}{k!}$.}
    \item \question{Réciproquement on considère la fonction $F(z) =
  \sum_{k=0}^\infty \frac{z^k}{k!}$. Vérifier que le rayon
  de convergence est infini. Établir par un calcul direct
  que $F'(0)$ existe et vaut $1$. En utilisant le théorème sur
  les séries doubles, montrer $F(z+w) = F(z)F(w)$. En
  déduire ensuite que $F$ est holomorphe sur $\Cc$ et vérifie
  $F' = F$.}
\reponse{Fixons $z\in \Cc$ et notons $a_k=\frac{z^k}{k!}$. Alors :
$$ \left| \frac{a_{k+1}}{a_k}\right| = |z| \frac{1}{k+1} \to 0 $$
lorsque $k\to \infty$. On en d\'eduit que le rayon de convergence de cette s\'erie est $\infty$ (d'Alembert) et que
$F$ est holomorphe sur $\C$. De plus :
$$F'(z) = \sum_{k\geq 1} \frac{kz^{k-1}}{k!}=F(z)$$
pour tout $z\in \Cc$. Par le th\'eor\`eme sur les s\'eries doubles (en fait l'exercice \ref{ex:burnol1.1.6})
$$\begin{aligned}
F(z)F(w) &= \sum_{k=0}^\infty \sum_{j=0}^k \frac{z^j}{j!}\frac{w^{k-j}}{(k-j)!}
=\sum_{k=0}^\infty \frac{1}{k!} \sum_{j=0}^k \frac{k!}{j! (k-j)!} z^j w^{k-j}\\
&= \sum_{k=0}^\infty \frac{1}{k!} (z+w)^k =F(z+w).
\end{aligned}$$}
\end{enumerate}
}
