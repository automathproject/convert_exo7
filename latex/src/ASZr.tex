\uuid{ASZr}
\exo7id{2509}
\auteur{queffelec}
\organisation{exo7}
\datecreate{2009-04-01}
\isIndication{false}
\isCorrection{false}
\chapitre{Différentiabilité, calcul de différentielles}
\sousChapitre{Différentiabilité, calcul de différentielles}

\contenu{
\texte{
Soit $E$ un espace de
Banach et ${\cal L}(E)$ l'espace des endomorphismes lin\'eaires
continus de $E$.
}
\begin{enumerate}
    \item \question{Soit $A\in{\cal L}(E)$; montrer que l'application
$\varphi:t\in{\Rr}\to e^{tA}$ est d\'erivable et calculer sa
d\'eriv\'ee.}
    \item \question{On suppose que la norme de $E$ est associ\'ee au produit
scalaire $\langle \cdot,\cdot\rangle$. Soit $x\in E$. Montrer que
l'application $\Phi:t\to \langle e^{tA}x,e^{tA}x\rangle$ est
d\'erivable et calculer sa d\'eriv\'ee.}
    \item \question{On suppose que $A$ est antisym\'etrique. Montrer que pour
tout $t$, $e^{tA}$ est unitaire.}
\end{enumerate}
}
