\uuid{oN7A}
\exo7id{5767}
\auteur{rouget}
\organisation{exo7}
\datecreate{2010-10-16}
\isIndication{false}
\isCorrection{true}
\chapitre{Intégration}
\sousChapitre{Intégrale de Riemann dépendant d'un paramètre}

\contenu{
\texte{
Pour $x\in\Rr$, on pose $F(x)=\int_{0}^{1}\frac{e^{-x^2(1+t^2)}}{1+t^2}\;dt$ et $G(x)=\left(\int_{0}^{x}e^{-t^2}\;dt\right)^2$.
}
\begin{enumerate}
    \item \question{Montrer que $F$ est de classe $C^1$ sur $\Rr$ et préciser $F'$.}
\reponse{Soit $A>0$. Soit $\begin{array}[t]{cccc}
\Phi~:&[-A,A]\times[0,1]&\rightarrow&\Rr\\
 &(x,t)&\mapsto&\frac{e^{-x^2(1+t^2)}}{1+t^2}
\end{array}$.

\textbullet~Pour chaque $x$ de $[-A,A]$, la fonction $t\mapsto F(x,t)$ est continue sur le segment $[0,1]$ et donc intégrable sur ce segment.

\textbullet~La fonction $\Phi$ admet sur $[-A,A]\times[0,1]$ une dérivée partielle par rapport à sa première variable $x$ définie par :

\begin{center}
$\forall(x,t)\in[-A,A]\times[0,1]$, $\frac{\partial \Phi}{\partial x}(x,t)=-2xe^{-x^2(1+t^2)}$.
\end{center}

De plus,

- pour chaque $x\in[-A,A]$, la fonction $t\mapsto \frac{\partial \Phi}{\partial x}(x,t)$ est continue par morceaux sur le segment $[0,1]$,

- pour chaque $t\in[0,1]$, la fonction $x\mapsto \frac{\partial \Phi}{\partial x}(x,t)$ est continue par morceaux sur $\Rr$,

- pour chaque $(x,t)\in[-A,A]\times[0,1]$, $\left|\frac{\partial \Phi}{\partial x}(x,t)\right|\leqslant 2A=\varphi(t)$, la fonction $\varphi$ étant continue et donc intégrable sur le

segment $[0,1]$.

D'après le théorème de dérivation des intégrales à paramètres (théorème de \textsc{Leibniz}), la fonction $F$ est de classe $C^1$ sur $[-A,A]$ et sa dérivée s'obtient en dérivant sous le signe somme. Ceci étant vrai pour tout $A>0$, $F$ est de classe $C^1$ sur $\Rr$ et

\begin{center}
\shadowbox{
$\forall x\in\Rr$, $F'(x)=-2x\int_{0}^{1}e^{-x^2(1+t^2)}\;dt$.
}
\end{center}}
    \item \question{Montrer que $G$ est de classe $C^1$ sur $\Rr$ et préciser $G'$.}
\reponse{La fonction $x\mapsto e^{-x^2}$ est continue sur $\Rr$. On en déduit que la fonction $x\mapsto\int_{0}^{x}e^{-t^2}\;dt$ est de classe $C^1$ sur $\Rr$. Il en est de même de la fonction $G$ et pour tout réel $x$,

\begin{center}
\shadowbox{
$G'(x)=2e^{-x^2}\int_{0}^{x}e^{-t^2}\;dt$.
}
\end{center}}
    \item \question{Montrer que la fonction $F+G$  est constante sur $\Rr$.}
\reponse{Soit $x\in\Rr^*$. En posant $u=xt$, on obtient

\begin{center}
$F'(x)=-2x\int_{0}^{1}e^{-x^2(1+t^2)}\;dt=-2e^{-x^2}\int_{0}^{1}e^{-(xt)^2}\;xdt=-e^{-x^2}\int_{0}^{x}e^{-u^2}\;du=-G'(x)$,
\end{center}

cette égalité restant vraie quand $x=0$ par continuité des fonctions $F'$ et $G'$ sur $\Rr$.

Ainsi, $F'+G'=0$ et donc $\forall x\in\Rr$, $F(x)+G(x)=F(0)+G(0)=\int_{0}^{1}\frac{1}{1+t^2}\;dt=\frac{\pi}{4}$.

\begin{center}
\shadowbox{
$\forall x\in\Rr$, $F(x)+G(x)=\frac{\pi}{4}$.
}
\end{center}}
    \item \question{Déterminer $\lim_{x \rightarrow +\infty}F(x)$.}
\reponse{Pour $x\in\Rr$,

\begin{center}
$|F(x)|=\int_{0}^{1}\frac{e^{-x^2(1+t^2)}}{1+t^2}\;dt\leqslant e^{-x^2}\int_{0}^{1}\frac{1}{1+t^2}\;dt=\frac{\pi}{4e^{x^2}}$,
\end{center}

et puisque $\lim_{x \rightarrow +\infty}\frac{\pi}{4e^{x^2}}=0$, on a montré que

\begin{center}
\shadowbox{
$\lim_{x \rightarrow +\infty}F(x)=0$.
}
\end{center}}
    \item \question{En déduire $I$.}
\reponse{Pour $x>0$, on a $\int_{0}^{x}e^{-t^2}\;dt\geqslant0$ et donc d'après la question 3),

\begin{center}
$\int_{0}^{x}e^{-t^2}\;dt=\sqrt{G(x)}=\sqrt{\frac{\pi}{2}-F(x)}$.
\end{center}

La question 4) permet alors d'affirmer que $\lim_{x \rightarrow +\infty}G(x)=\frac{\sqrt{\pi}}{2}$ et donc que

\begin{center}
\shadowbox{
$\int_{0}^{+\infty}e^{-t^2}\;dt=\frac{\sqrt{\pi}}{2}$.
}
\end{center}}
\end{enumerate}
}
