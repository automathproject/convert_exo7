\uuid{ji2V}
\exo7id{3828}
\auteur{quercia}
\organisation{exo7}
\datecreate{2010-03-11}
\isIndication{false}
\isCorrection{true}
\chapitre{Espace euclidien, espace normé}
\sousChapitre{Espaces vectoriels hermitiens}

\contenu{
\texte{

}
\begin{enumerate}
    \item \question{Soit $\varphi : {\R^n} \to \R$ définie par
    $\varphi(x_1,\dots,x_n) =  \int_{t=0}^1 (1+tx_1+\dots+t^nx_n)^2\,d t$.
    Montrer que $\varphi$ admet un minimum absolu et le calculer lorsque $n=3$.}
\reponse{$\frac1{16}$.}
    \item \question{Même question avec
    $\psi(x_1,\dots,x_n) =  \int_{t=0}^{+\infty} e^{-t}(1+tx_1+\dots+t^nx_n)^2\,d t$.}
\reponse{$\frac14$.}
\end{enumerate}
}
