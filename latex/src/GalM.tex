\uuid{GalM}
\exo7id{3167}
\auteur{quercia}
\organisation{exo7}
\datecreate{2010-03-08}
\isIndication{false}
\isCorrection{true}
\chapitre{Polynôme, fraction rationnelle}
\sousChapitre{Autre}

\contenu{
\texte{
D{\'e}terminer les poly{\^o}mes $P \in \R_{2n-1}(X)$ tels que
$P(X)+1$ est multiple de $(X-1)^n$ et
$P(X)-1$ est multiple de $(X+1)^n$.
}
\reponse{
$P(X) = -1 + Q(X)\times(X-1)^n \Leftrightarrow (X+1)^n \mid Q(X)(X-1)^n-2
\Leftrightarrow X^n \mid Q(X-1)(X-2)^n-2$.
Soit $2 = A(X)(X-2)^n + X^nB(X)$ la division suivant les puissances croissantes
de $2$ par $(X-2)^n$ {\`a} l'ordre~$n$.
On obtient $X^n\mid Q(X-1)-A(X)$ soit $Q(X) = A(X+1) + X^nR(X)$ et
$\deg(P)< 2n \Leftrightarrow R=0$.
Calcul de $A(X)$ par d{\'e}veloppement limit{\'e}~:  
$\frac{1}{(1+x)^n} = \sum_{k=0}^{n-1}\binom{-n}{k}x^k + O(x^n)$ donc~:
$$
A(X) = \frac{(-1)^n}{2^{n-1}}\sum_{k=0}^{n-1}\binom{-n}{k}\frac{(-1)^kX^k}{2^k}
     = \sum_{k=0}^{n-1}C_{n+k-1}^k(-1)^n\frac{X^k}{2^{n+k-1}}
$$
}
}
