\uuid{SdmZ}
\exo7id{3725}
\auteur{quercia}
\organisation{exo7}
\datecreate{2010-03-11}
\isIndication{false}
\isCorrection{true}
\chapitre{Espace euclidien, espace normé}
\sousChapitre{Forme quadratique}

\contenu{
\texte{
Soit $A = (a_{ij})\in\mathcal{M}_n(\R)$ symétrique. On dit que $A$ est à diagonale
faiblement dominante si pour tout $i$ on a $a_{ii}\ge \sum_{j\ne i}|a_{ij}|$
et que $A$ est à diagonale fortement dominante si pour tout $i$
on a $a_{ii}> \sum_{j\ne i}|a_{ij}|$.

Montrer que si $A$ est à diagonale fortement dominante alors $A$ est définie positive
et si $A$ est à diagonale faiblement dominante alors $A$ est positive.
}
\reponse{
Pour $A$ à diagonale fortement dominante, récurrence sur $n$.

Soit $(e_1,\dots,e_n)$ la base dans laquelle $A$ est la matrice de $q$.
$\Delta_{n-1}(A) \ne 0$ donc il existe des coefficients
$\alpha_1,\dots,\alpha_{n-1}$ tels que 
$u_n = e_n-\sum_{i<n}\alpha_i e_i$ soit $q$-orthogonal à $e_1,\dots,e_{n-1}$ et il faut montrer que
$q(u_n) > 0$ ce qui résulte de $|\alpha_i| \le 1$ en considérant la $i$-ème ligne
de $A$.
}
}
