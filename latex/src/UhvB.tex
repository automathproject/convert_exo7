\uuid{UhvB}
\exo7id{7603}
\auteur{mourougane}
\organisation{exo7}
\datecreate{2021-08-10}
\isIndication{false}
\isCorrection{true}
\chapitre{Autre}
\sousChapitre{Autre}

\contenu{
\texte{

}
\begin{enumerate}
    \item \question{Soit $D$ un ouvert de $\Cc$, $c$ un point de $D$ et $f : D-\{c\}\to \Cc$ une application holomorphe.
Rappeler la définition du résidu de $f$ en $c$.}
\reponse{Soit $r>0$ tel que $\overline{\Delta_r(c)}$ est un disque fermé inclus dans $D$.
Le résidu de $f$ en $c$ est 
$$Res_c(f):=\frac{1}{2i\pi}\int_{\partial\Delta_r(c)}f(z)dz.$$
Cette intégrale ne dépend pas de $r$ tel que $\overline{\Delta_r(c)}$ est inclus dans $D$.}
    \item \question{Soit $D$ un ouvert de $\Cc$ et $f : D-\{c\}\to \Cc$ une application holomorphe.
 On suppose que $f$ admet une primitive sur $D-\{c\}$.
 Que peut-on dire du résidu de $f$ en $c$ ?}
\reponse{Si $f$ admet une primitive sur $D-\{c\}$, l'intégrale de $f$ sur tout chemin fermé de $D-\{c\}$ est nulle.
 En particulier, le résidu de $f$ en $c$ est nul.}
    \item \question{Soit $c$ un point de $\Delta$ et $f :\Delta-\{c\}\to\Cc$ 
  une application holomorphe qui admet un pôle d'ordre $3$ en $c$.
  Sous quelle condition $f$ admet-elle une primitive sur $\Delta-\{c\}$ ?}
\reponse{D'après la question précédente, il est nécessaire que le résidu de $f$ en $c$ soit nul.
  Réciproquement, si le résidu de $f$ en $c$ est nul, montrons que $f$ admet une primitive sur $\Delta$.
  Par le développement au voisinage de $c$, on sait qu'il existe $a_{-3}, a_{-2}, a_{-1}\in\Cc$ et une fonction $\tilde f$
  holomorphe sur $\Delta$ tels que
  $$\forall z\in\Delta,\ \ f(z)=\frac{a_{-3}}{(z-c)^3}+ \frac{a_{-2}}{(z-c)^2}+ \frac{a_{-1}}{z-c}+ \tilde{f}(z).$$
  Comme le résidu de $f$ en $c$ est nul, $a_{-1}=0$.
  L'application $z\mapsto \frac{a_{-3}}{(z-c)^3}+ \frac{a_{-2}}{(z-c)^2}$ admet 
  $z\mapsto -\frac{a_{-3}}{2(z-c)^2}- \frac{a_{-2}}{z-c}$ comme primitive sur $\Delta-\{c\}$.
  Puisque $\Delta$ est étoilé et que $\tilde{f}$ est holomorphe sur $\Delta$, elle admet une primitive sur $\Delta$
  donc sur $\Delta-\{c\}$.
  En conclusion, si le résidu de $f$ en $c$ est nul, $f$ admet-elle une primitive sur $\Delta-\{c\}$.}
\end{enumerate}
}
