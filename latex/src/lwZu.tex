\uuid{lwZu}
\exo7id{7691}
\auteur{mourougane}
\organisation{exo7}
\datecreate{2021-08-11}
\isIndication{false}
\isCorrection{true}
\chapitre{Sous-variété}
\sousChapitre{Sous-variété}

\contenu{
\texte{
Calculer l'aire de l'ellipsoïde $\mathcal{E}$ d'équation $x^2+y^2+5z^2=1$.
}
\reponse{
On paramètre $\mathcal{E}$ en coordonnées sphériques par $F(\theta,\varphi)=
(\cos\theta\sin\varphi,\sin\theta\sin\varphi,1/\sqrt{5}\cos\varphi)$ avec $\theta\in[0,2\pi[$ et 
$\varphi\in [0,\pi]$.
Le plan tangent est engendré par les vecteurs
$\frac{\partial F}{\partial \theta}=(-\sin\theta\sin\varphi,\cos\theta\sin\varphi,0)$ 
et $\frac{\partial F}{\partial \varphi}=(\cos\theta\cos\varphi,\sin\theta\cos\varphi,-1/\sqrt{5}\sin\varphi)$.
Dans cette base, la matrice de la première forme fondamentale est 
$\begin{pmatrix}\sin^2\varphi&0\\0&\cos^2\varphi+1/5\sin^2\varphi\end{pmatrix}$.
L'élément de volume est $1/\sqrt{5}\sin\varphi\sqrt{1+4\cos^2\varphi}d\theta d\varphi$.
L'aire de l'ellipsoïde est donc
\begin{eqnarray*}
 A[\mathcal{E}]&=&\int_{\varphi=0}^{\pi}\int_{\theta=0}^{2\pi}\frac{1}{\sqrt{5}}
\sin\varphi\sqrt{1+4\cos^2\varphi} d\varphi d\theta\\
&=&2\pi\int_{\varphi=0}^{\pi}\frac{1}{\sqrt{5}}
\sin\varphi\sqrt{1+4\cos^2\varphi}d\varphi\\
&=&\frac{2\pi}{\sqrt{5}}\int_{t=-1}^1\sqrt{1+4t^2}dt=\frac{\pi}{\sqrt{5}}(2\sqrt{5}+\ln(2+\sqrt{5}))\\
&=&\pi(2+\frac{\ln(2+\sqrt{5})}{\sqrt{5}}).
\end{eqnarray*}
}
}
