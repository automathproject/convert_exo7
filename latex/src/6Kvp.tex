\uuid{6Kvp}
\exo7id{3766}
\auteur{quercia}
\organisation{exo7}
\datecreate{2010-03-11}
\isIndication{false}
\isCorrection{true}
\chapitre{Espace euclidien, espace normé}
\sousChapitre{Endomorphismes auto-adjoints}

\contenu{
\texte{
Diagonaliser dans une base orthonormée :
}
\begin{enumerate}
    \item \question{$A = \begin{pmatrix} 6 &-2 &2 \cr -2 &5 &0 \cr 2 &0 &7 \cr \end{pmatrix}$.}
\reponse{$P = \frac13\begin{pmatrix}2 &-1 &2 \cr 2 &2 &-1 \cr -1 &2 &2 \cr\end{pmatrix}$,
             $D = \text{Diag}(3\ 6\ 9)$.}
    \item \question{$A = \frac19\begin{pmatrix} 23 &2 &-4 \cr 2 &26 &2 \cr -4 &2 &23 \cr \end{pmatrix}$.}
\reponse{$P = \frac13\begin{pmatrix}2 &-1 &2 \cr 2 &2 &-1 \cr -1 &2 &2 \cr\end{pmatrix}$,
             $D = \text{Diag}(3\ 3\ 2)$.}
\end{enumerate}
}
