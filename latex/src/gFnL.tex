\uuid{gFnL}
\exo7id{1242}
\auteur{roussel}
\organisation{exo7}
\datecreate{2001-09-01}
\isIndication{false}
\isCorrection{false}
\chapitre{Développement limité}
\sousChapitre{Calculs}

\contenu{
\texte{
Pour chacune des fonctions suivantes, donner les conditions sur $\epsilon (x)$
pour que ces fonctions soient des d\'eveloppements limit\'es ~au voisinage d'un point et \`a un ordre
que vous pr\'eciserez.
}
\begin{enumerate}
    \item \question{$f_1(x)=x-\displaystyle{\frac{x^3}{3}}+x^2\epsilon (x)$}
    \item \question{$f_2(x)=1-\displaystyle{\frac{2}{x^2}}+\displaystyle{\frac{1}{x^3}}+\displaystyle{\frac{1}{x^3}}\epsilon
(x)$}
    \item \question{$f_3(x)=(x-2)+\displaystyle{\frac{(x-2)^2}{5}}+(x-2)^3\epsilon (x)$}
    \item \question{$f_4(x)=x^2-x+1+\displaystyle{\frac{1}{x}}+\displaystyle{\frac{1}{x}}\epsilon (x)$}
    \item \question{$f_5(x)=x^3+3x^2-x+1+(x-1)^2\epsilon (x)$}
    \item \question{$f_6(x)=(x-2)^2+(x-2)-2+(x-2)\epsilon (x)$}
    \item \question{$f_7(x)=\{ 2x+x^2+1+x^2\epsilon (x)\}\{-x+3+x^2-x^3\epsilon (x)\}$}
\end{enumerate}
}
