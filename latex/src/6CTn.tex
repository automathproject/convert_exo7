\uuid{6CTn}
\exo7id{4802}
\auteur{quercia}
\organisation{exo7}
\datecreate{2010-03-16}
\isIndication{false}
\isCorrection{true}
\chapitre{Topologie}
\sousChapitre{Topologie des espaces vectoriels normés}

\contenu{
\texte{
Soient $E,F$ deux $\R$-espaces vectoriels norm{\'e}s, $F$ {\'e}tant complet.
Soit $f$ une application continue de $E$ dans $F$ telle qu'il existe $M\in\R^+$
v{\'e}rifiant~:
$$\forall\ x,y\in E,\ \|f(x+y)-f(x)-f(y)\| \le M.$$
}
\begin{enumerate}
    \item \question{Dans le cas $M=0$ montrer que $f$ est lin{\'e}aire. Ce r{\'e}sultat subsiste-t-il
    si $E$ et $F$ sont des $\C$-ev~?}
\reponse{$f(rx) = rf(x)$ pour tout $r\in\N$ par r{\'e}currence, puis pour
    tout $r\in\Z$ par diff{\'e}rence, pour tout $r\in\Q$ par quotient et enfin
    pour tout $r\in\R$ par densit{\'e}. Dans le cas de $\C$-ev $f$ est $\R$-lin{\'e}aire
    mais pas forc{\'e}ment $\C$-lin{\'e}aire, ctrex~: $z \mapsto\overline z$ de $\C$
    dans $\C$.}
    \item \question{On suppose $M>0$. Soit pour $x\in E$ et $n\in\N$~: $f_n(x) = 2^{-n}f(2^nx)$.
    Montrer que la suite $(f_n)$ converge simplement sur~$E$.}
\reponse{$\|f_{n+1}(x) - f_n(x)\|\le M2^{-n-1}$ donc la s{\'e}rie t{\'e}lescopique
    $\sum(f_{n+1}(x) - f_n(x))$ est uniform{\'e}ment convergente.}
    \item \question{On note $g = \lim_{n\to\infty}f_n$. Montrer que $g$ est une application
    lin{\'e}aire continue et que c'est l'unique application lin{\'e}aire telle que
    $f-g$ soit born{\'e}e.}
\reponse{$\|f_n(x+y) - f_n(x) - f_n(y)\|\le M2^{-n}$ donc
     $\|g(x+y) - g(x) - g(y)\|\le 0$ et $g$ est continue (limite uniforme des $f_n$)
     d'o{\`u} $g$ est lin{\'e}aire continue.
     $\|f(x) - g(x)\| = \|\sum_{k=0}^\infty(f_{k}(x) - f_{k+1}(x))\|
                        \le 2M$
     donc $f-g$ est born{\'e}e.
     Si $h$ est une application lin{\'e}aire telle que $f-h$ est born{\'e}e alors $g-h$ est aussi
     born{\'e}e ce qui entra{\^\i}ne $g=h$ par lin{\'e}arit{\'e}.}
\end{enumerate}
}
