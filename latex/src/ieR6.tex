\uuid{ieR6}
\exo7id{5031}
\auteur{quercia}
\organisation{exo7}
\datecreate{2010-03-17}
\isIndication{false}
\isCorrection{false}
\chapitre{Courbes planes}
\sousChapitre{Propriétés métriques : longueur, courbure,...}

\contenu{
\texte{
Soit $\mathcal{C}$ une courbe plane paramétrée par une abscisse curviligne $s$.
Soit $s_0$ fixé.
}
\begin{enumerate}
    \item \question{Donner le DL à l'ordre 2 de $M_s$ pour $s \to s_0$ dans le repère
      de Frenet en $M_{s_0}$.}
    \item \question{On suppose $c(s_0) \ne 0$. Montrer que pour $h$ assez petit, les points
      $M_{s_0-h}$, $M_{s_0}$, $M_{s_0+h}$ ne sont pas alignés.}
    \item \question{Soit $\Gamma_h$ le cercle circonscrit à ces trois points, et $R_h$
      son rayon. Chercher $\lim_{h\to 0} R_h$.}
\end{enumerate}
}
