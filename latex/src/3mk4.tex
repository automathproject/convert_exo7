\uuid{3mk4}
\exo7id{5810}
\auteur{rouget}
\organisation{exo7}
\datecreate{2010-10-16}
\isIndication{false}
\isCorrection{true}
\chapitre{Espace euclidien, espace normé}
\sousChapitre{Forme quadratique}

\contenu{
\texte{
Soit $S$ une matrice symétrique réelle, définie positive. Pour $(x_1,...,x_n)\in\Rr^n$, on pose 

\begin{center}
$Q((x_1,...,x_n)) = -\text{det}\left(\begin{array}{cccc}
0&x_1&\ldots&x_n\\
x_1& & & \\
\vdots& &S& \\
x_n& & & 
\end{array}
\right)$.
\end{center}

Montrer que $Q$ est une forme quadratique définie positive.
}
\reponse{
Posons $X=\left(
\begin{array}{c}
x_1\\
\vdots\\
x_n
\end{array}
\right)$ et $A =\left(\begin{array}{cccc}
0&x_1&\ldots&x_n\\
x_1& & & \\
\vdots& &S& \\
x_n& & & 
\end{array}
\right)=\left(
\begin{array}{cc}
0&{^t}X\\
X&S
\end{array}\right)$.

Un calcul par blocs fournit $\left(
\begin{array}{cc}
0&{^t}X\\
X&S
\end{array}\right)\left(
\begin{array}{cc}
1&0\\
0&S^{-1}
\end{array}\right)=\left(
\begin{array}{cc}
0&{^t}XS^{-1}\\
X&I_n
\end{array}\right)$  puis

\begin{center}
$\left(
\begin{array}{cc}
0&{^t}X\\
X&S
\end{array}\right)\left(
\begin{array}{cc}
1&0\\
0&S^{-1}
\end{array}\right)\left(
\begin{array}{cc}
-1&0\\
X&I_n
\end{array}\right)=\left(
\begin{array}{cc}
0&{^t}XS^{-1}\\
X&I_n
\end{array}\right)\left(
\begin{array}{cc}
-1&0\\
X&I_n
\end{array}\right)=\left(
\begin{array}{cc}
{^t}XS^{-1}X&{^t}XS^{-1}\\
0&I_n
\end{array}\right)$.
\end{center}

On en déduit que $\text{det}(A)\times\text{det}(S^{-1})\times(-1)={^t}XS^{-1}X$ puis que $Q(X)=-\text{det}(A)={^t}X((\text{det}(S))S^{-1})X ={^t}XS'X$ où $S'=(\text{det}(S))S^{-1}$.

Maintenant, la matrice $S$ est définie positive et donc ses valeurs propres sont des réels strictement positifs. Les valeurs propres de la matrice $S'$ sont les $\frac{\text{det}(S)}{\lambda}$ où $\lambda$ décrit le spectre de $S$ et donc la matrice $S'$ est aussi une matrice symétrique définie positive. $Q$ est donc une forme quadratique définie positive.
}
}
