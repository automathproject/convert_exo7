\uuid{s8KU}
\exo7id{2100}
\auteur{bodin}
\organisation{exo7}
\datecreate{2008-02-04}
\isIndication{true}
\isCorrection{true}
\chapitre{Calcul d'intégrales}
\sousChapitre{Somme de Riemann}

\contenu{
\texte{
Calculer la limite des suites suivantes :
}
\begin{enumerate}
    \item \question{$\displaystyle u_n=n\sum_{k=0}^{n-1}\frac 1{k^2+n^2}$}
\reponse{Soit 
$$u_n =n \sum_{k=0}^{n-1}\frac 1{k^2+n^2} = \frac 1n  \sum_{k=0}^{n-1}\frac 1{1+\big(\frac k n \big)^2}.$$
En posant $f(x) = \frac 1 {1+x^2}$ nous venons d'écrire la somme de Riemann correspondant à 
$\int_0^1 f(x) dx$. Cette intégrale se calcule facilement : 
$$\int_0^1 f(t) dt = \int_0^1 \frac {dx} {1+x^2} = \big[\arctan x\big]_0^1 = \frac \pi 4.$$
La somme de Riemann $u_n$ convergeant vers $\int_0^1 f(x) dx$ nous venons de montrer que
$(u_n)$ converge vers $\frac \pi 4$.}
    \item \question{$\displaystyle v_n=\prod\limits_{k=1}^n\left(1+\frac{k^2}{n^2}\right) ^{\frac 1n}$}
\reponse{Soit $v_n=\prod\limits_{k=1}^n\left(1+\frac{k^2}{n^2}\right) ^{\frac 1n}$, notons 
$$w_n = \ln v_n = \sum_{k=1}^n \ln\left( \left(1+\frac{k^2}{n^2}\right)^{\frac 1n} \right) 
= \frac 1 n \sum_{k=1}^n \ln \left(1+\frac{k^2}{n^2}\right).$$
En posant $g(x) = \ln (1+x^2)$ nous reconnaissons la somme de Riemann correspondant à
$I = \int_0^1 g(x)dx$.

Calculons cette intégrale : 
\begin{align*}
 I &= \int_0^1 g(x)dx = \int_0^1 \ln(1+x^2) dx \\
   &= \big[x\ln(1+x^2)\big]_0^1 - \int_0^1 x \frac{2x}{1+x^2}dx \quad \text{par intégration par parties} \\
   &= \ln 2 -2 \int_0^1 1-\frac 1{1+x^2} dx \\
   &= \ln 2  - 2\big[x-\arctan x\big]_0^1 \\
   &= \ln 2 - 2 + \frac \pi 2. \\
\end{align*}


Nous venons de prouver que $w_n=\ln v_n$ converge vers $I=\ln 2 - 2 + \frac \pi 2$,
donc $v_n = \exp w_n$ converge vers $\exp(\ln 2 - 2 + \frac \pi 2) = 2e^{\frac \pi 2 -2}$.
Bilan $(v_n)$ a pour limite $2e^{\frac \pi 2 -2}$.}
\indication{On pourra essayer de reconnaître des sommes de Riemann, puis calculer des intégrales. 
Pour le produit composer par la fonction $\ln$, afin de transformer le produit en une somme.}
\end{enumerate}
}
