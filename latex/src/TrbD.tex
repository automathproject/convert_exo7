\uuid{TrbD}
\exo7id{5944}
\auteur{tumpach}
\organisation{exo7}
\datecreate{2010-11-11}
\isIndication{false}
\isCorrection{true}
\chapitre{Théorème de convergence dominée}
\sousChapitre{Théorème de convergence dominée}

\contenu{
\texte{
Soit $f\in\mathcal{M}^{+}(\Omega, \Sigma)$ tel que $\int_{\Omega}
f\,d\mu ~<~+\infty$. Montrer que $$\mu\{ x\in\Omega, ~f(x) =
+\infty\} = 0. $$ 
On pourra consid\'erer les fonctions $f_{n} = n\mathbf{1}_{\{f\geq n\}}$.
}
\reponse{
On a
$$\mu\left(f =
+\infty\right) = \mu\left(\cap_{n\in\mathbb{N}}\{f \geq
n\}\right).
$$
Puisque les ensembles $A_{n} := \{f \geq n\}$ v\'erifient $A_1
\supset A_{2} \supset A_{3}\dots$ et $\mu(A_i) < +\infty$ ($i=1,2,\ldots$), par continuit\'e de la mesure,
on a~:
$$\mu\left(f =
+\infty\right) = \lim_{n\rightarrow+\infty} \mu\left(f\geq n
\right).
$$
Or, comme $f$ est \`a valeurs positives, les fonctions $f_{n}$
d\'efinies par $f_{n} = n\mathbf{1}_{\{f\geq n\}}$ v\'erifient $f_{n}\leq
f$. Ainsi
$$
\int_{\Omega} f_{n}\,d\mu  = \int_{\Omega} n \mathbf{1}_{\{f\geq
n\}}\,d\mu = n \mu\left(f\geq n \right) ~\leq ~\int_{\Omega}
f\,d\mu <+\infty.
$$
On en d\'eduit que $$\mu\left(f\geq n \right) ~\leq~
\frac{1}{n}\int_{\Omega} f\,d\mu \rightarrow 0,$$ donc
$$\mu\left(f = +\infty\right) = 0.$$
}
}
