\uuid{Ocf5}
\exo7id{7143}
\auteur{megy}
\organisation{exo7}
\datecreate{2017-04-05}
\isIndication{true}
\isCorrection{true}
\chapitre{Géométrie affine euclidienne}
\sousChapitre{Géométrie affine euclidienne du plan}

\contenu{
\texte{
% puissance, triangles rectangles, semblables
Soit $ABC$ un triangle rectangle en $A$, et $H$ le pied de la hauteur issue de $A$. Montrer $AH^2 = HB\cdot HC$.
}
\indication{Utiliser des triangles semblables, ou bien utiliser la puissance d'un point par rapport à un cercle.}
\reponse{
Soit $A'$ le symétrique de $A$ par rapport à $[BC]$. Alors, $BACA'$ est inscriptible, et la puissance de $H$ par rapport à son cercle circonscrit est 
\[ p_{\mathcal C}(H) = HB\cdot HC=  HA\cdot HA' = HA^2.\]

\textbf{Deuxième solution, n'utilisant pas la puissance d'un point par rapport à un cercle} : 

Les triangles $ABC$, $ABH$ et $ACH$ sont semblables car ils ont à chaque fois deux (donc trois) angles identiques. Les rapports de longueurs de côtés homologues sont donc égaux, ce qui donne
\[ \frac{AB}{AC} = \frac{HB}{HA} = \frac{HA}{HC}\]
d'où on tire $HA^2 = HB\cdot HC$, ce qu'il fallait démontrer.
}
}
