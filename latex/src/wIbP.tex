\uuid{wIbP}
\exo7id{3528}
\auteur{quercia}
\organisation{exo7}
\datecreate{2010-03-10}
\isIndication{false}
\isCorrection{true}
\chapitre{Réduction d'endomorphisme, polynôme annulateur}
\sousChapitre{Polynôme caractéristique, théorème de Cayley-Hamilton}

\contenu{
\texte{
On considère le polynôme défini par :
$\forall\ n \in \N,\ P_n(x) = x^n - \sum_{i=0}^{n-1} \alpha_ix^i$ avec
$\alpha_0 > 0$ et $\alpha_i \ge 0$ pour $1 \le i \le n-1$.
}
\begin{enumerate}
    \item \question{Montrer qu'il existe une unique racine dans $\R^{+*}$ pour $P_n$.}
\reponse{La fonction $f_n : x  \mapsto \frac{P_n(x)}{x^n}$ croît strictement
             de $-\infty$ à 1 quand $x$ varie de $0$ à $+\infty$.}
    \item \question{Soit $A = \begin{pmatrix} 1       &1      &0      &\dots  &0      \cr
                        2       &0      &1      &\ddots &\vdots \cr
                        \vdots  &\vdots &\ddots &\ddots &0      \cr
                        \vdots  &\vdots &       &\ddots &1      \cr
                        n       &0      &\dots  &\dots  &0      \cr\end{pmatrix}$.
    Montrer que $A$ admet une unique valeur propre réelle strictement positive.}
\reponse{$\chi_A(x) = (-1)^n\Bigl(x^n - \sum_{k=1}^n kx^{n-k}\Bigr)$.}
\end{enumerate}
}
