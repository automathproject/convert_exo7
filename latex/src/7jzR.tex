\uuid{7jzR}
\exo7id{6011}
\auteur{quinio}
\organisation{exo7}
\datecreate{2011-05-20}
\isIndication{false}
\isCorrection{true}
\chapitre{Probabilité discrète}
\sousChapitre{Variable aléatoire discrète}

\contenu{
\texte{
Si dans une population une personne sur cent est un centenaire,
quelle est la probabilité de trouver au moins un centenaire parmi $100$
personnes choisies au hasard ? Et parmi $200$ personnes ?
}
\reponse{
La probabilité $p=\frac{1}{100}$ étant faible, on peut appliquer la
loi de Poisson d'espérance $100p=1$ au nombre $X$ de centenaires pris
parmi cent personnes. On cherche donc: $P[X\geq 1]=1-P[X=0]=1-e^{-1}\simeq
63\%$.

Sur un groupe de $200$ personnes: l'espérance est 2 donc: $P[X'\geq 1]=1-e^{-2}\simeq 86\%.$
La probabilité des événements : $[X' =1]$ et $[X' =2]$
sont les mêmes et valent: $0.14$.
Ainsi, sur 200 personnes, la probabilité de trouver exactement un
centenaire vaut $0.14$, égale à la probabilité de trouver
exactement deux centenaires. Cette valeur correspond au maximum de probabilité 
pour une loi de Poisson d'espérance $2$ et se généralise.
Si $X$ obeit à une loi de Poisson d'espérance $K$, alors le maximum de
probabilité est obtenu pour les événements $[X=K-1]$ et $[X=K].$
}
}
