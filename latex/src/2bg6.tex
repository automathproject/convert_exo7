\uuid{2bg6}
\exo7id{5993}
\auteur{quinio}
\organisation{exo7}
\datecreate{2011-05-20}
\isIndication{false}
\isCorrection{true}
\chapitre{Probabilité discrète}
\sousChapitre{Probabilité conditionnelle}

\contenu{
\texte{
Une fête réunit $35$ hommes, $40$ femmes, $25$ enfants ; sur
une table, il y a $3$ urnes $H$, $F$, $E$ contenant des boules de couleurs dont
respectivement $10$\%, $40$\%, $80$\% de boules noires. Un présentateur
aux yeux bandés désigne une personne au hasard et lui demande de
tirer une boule dans l'urne $H$ si cette personne est un homme, dans l'urne $F$
si cette personne est une femme, dans l'urne $E$ si cette personne est un
enfant. La boule tirée est noire : quelle est la probabilité pour que
la boule ait été tirée par un homme? une femme? un enfant? Le présentateur n'est pas plus magicien que vous et moi et pronostique le
genre de la personne au hasard : que doit-il dire pour avoir le moins de
risque d'erreur?
}
\reponse{
C'est évidemment le même que le précédent (exercice \ref{exo:quinio11}), seul le contexte
est différent : il suffit d'adapter les calculs faits.
En pronostiquant un enfant, le présentateur a une chance sur deux
environ de ne pas se tromper.
}
}
