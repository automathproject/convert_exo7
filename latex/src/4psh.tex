\uuid{4psh}
\exo7id{2680}
\auteur{matexo1}
\organisation{exo7}
\datecreate{2002-02-01}
\isIndication{false}
\isCorrection{true}
\chapitre{Singularité}
\sousChapitre{Singularité}

\contenu{
\texte{
{\`A} l'aide de la formule
$$ f^{(n)}(a) = {n!\over 2i\pi } \int_\Gamma {f(z)\over (z-a)^{n+1}} \,dz$$
o{\`u} $f$ est m{\'e}romorphe dans un domaine contenant le contour simple 
$\Gamma$, et $a$ est
un point int{\'e}rieur {\`a} $\Gamma$, montrer que l'on a
$$ \left(x^n\over n!\right)^2 = {1\over 2i\pi } 
\int_C {x^n e^{xz}\over n! z^{n+1}} \,dz$$
o{\`u} $C$ est le cercle unit{\'e} de $\C$. En d{\'e}duire que l'on a
$$ \sum_{n=0}^{+\infty } {x^{2n}\over \left(n!\right)^2} = {1\over 2\pi }
 \int_0^{2\pi } e^{2x\cos \theta}\,d\theta. $$
}
\reponse{
On pose $f(z) = e^{xz}$. Alors $f^{(n)} (0)= x^n$, donc
$$ x^n = {n!\over 2i\pi } \int_C {e^{xz}\over z^{n+1}}\,dz $$
d'o{\`u} la formule demand{\'e}e par multiplication par $x^n/(n!)^2$.

Sur $C$, on a $|z|=1$, et donc la s{\'e}rie $\sum x^n /(n! z^{n})$ est
uniform{\'e}ment convergente par rapport {\`a} $z$ (et sa limite est bien s{\^u}r {\'e}gale {\`a}
$e^{x/z}$). Donc on peut inverser les signes $\sum$ et $\int$:
$$\begin{array}{cccccc}\sum_{n=0}^{+\infty } \left(x^n\over n!\right)^2
&= {1\over 2i\pi }\sum_{n=0}^{+\infty } \int_C {e^{xz}\over z}\,\left({x^n\over n! z^n}\right)\,dz \cr
&= {1\over 2i\pi } \int_C {1\over z}\,e^{xz}\,e^{x\over z}\ dz \cr
&= {1\over 2i\pi } \int_0^{2\pi } e^{-i\theta} e^{x(e^{i\theta}+e^{-i\theta})} \,{i e^{i\theta}\,d\theta}
 &(z = e^{i\theta})\cr
&= {1\over 2\pi } \int_0^{2\pi } e^{2x\cos\theta}\,d\theta.
\end{array}$$
}
}
