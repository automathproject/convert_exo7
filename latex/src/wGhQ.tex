\uuid{wGhQ}
\exo7id{3349}
\auteur{quercia}
\organisation{exo7}
\datecreate{2010-03-09}
\isIndication{false}
\isCorrection{false}
\chapitre{Application linéaire}
\sousChapitre{Image et noyau, théorème du rang}

\contenu{
\texte{
Soit $E$ un ev de dimension finie et $f \in \mathcal{L}(E)$.
On pose $N_k = \mathrm{Ker}(f^k)$ et $I_k = \Im(f^k)$.
}
\begin{enumerate}
    \item \question{Montrer que la suite $(N_k)$ est croissante (pour l'inclusion)
    et que la suite $(I_k)$ est décroissante.}
    \item \question{Soit $p$ tel que $N_p = N_{p+1}$. Justifier l'existence de $p$ et
    montrer que $N_{p+1} = N_{p+2} = \dots = N_{p+k} = \dots$}
    \item \question{Montrer que les suites $(N_k)$ et $(I_k)$ sont stationnaires à partir du
    même rang $p$.}
    \item \question{Montrer que $N_p \oplus I_p = E$.}
    \item \question{Montrer que la suite $(\dim(N_{k+1})-\dim(N_k))$ est décroissante.}
\end{enumerate}
}
