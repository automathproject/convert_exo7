\uuid{IMAp}
\exo7id{2909}
\auteur{quercia}
\organisation{exo7}
\datecreate{2010-03-08}
\isIndication{false}
\isCorrection{false}
\chapitre{Dénombrement}
\sousChapitre{Binôme de Newton et combinaison}

\contenu{
\texte{
Soient $a,b,c \in \N$.
D{\'e}montrer que $\sum_{k=0}^c\,C_a^kC_b^{c-k} = C_{a+b}^c$ $\ldots$
}
\begin{enumerate}
    \item \question{En calculant de deux mani{\`e}res $(1+x)^a(1+x)^b$.}
    \item \question{En cherchant le nombre de parties de cardinal $c$ dans $E\cup F$, o{\`u}
    $E$ et $F$ sont des ensembles disjoints de cardinaux $a$ et $b$.}
    \item \question{Application : Soient $n,p,q \in \N$.
    Montrer que $\sum_{k=0}^q\,C_q^kC_n^{p+k} = C_{n+q}^{p+q}$.}
\end{enumerate}
}
