\uuid{K23W}
\exo7id{5434}
\auteur{exo7}
\organisation{exo7}
\datecreate{2010-07-06}
\isIndication{false}
\isCorrection{true}
\chapitre{Développement limité}
\sousChapitre{Calculs}

\contenu{
\texte{
Soient $a>0$ et $b>0$. Pour $n\in\Nn^*$ et $x\in\Rr$, on pose $f_n(x)=\left(1+\frac{x}{n}\right)^n$.
}
\begin{enumerate}
    \item \question{Equivalent simple quand $n$ tend vers $+\infty$ de $f_n(a+b)-f_n(a)f_n(b)$.}
\reponse{$$f_n(a)=\text{exp}\left(n\ln\left(1+\frac{a}{n}\right)\right)\underset{n\rightarrow+\infty}{=}\text{exp}\left(a-\frac{a^2}{2n}+o\left(\frac{1}{n}\right)\right)= e^a\left(1-\frac{a^2}{2n}+o\left(\frac{1}{n}\right)\right).$$
En remplaçant $a$ par $b$ ou $a+b$, on obtient

\begin{align*}\ensuremath
f_n(a+b)-f_n(a)f_n(b)&\underset{n\rightarrow+\infty}{=}e^{a+b}\left(1-\frac{(a+b)^2}{2n}\right)-e^a\left(1-\frac{a^2}{2n}\right)e^b\left(1-\frac{b^2}{2n}\right)+o\left(\frac{1}{n}\right)\\
 &=e^{a+b}\frac{-(a+b)^2+a^2+b^2}{2n}+o\left(\frac{1}{n}\right)=-\frac{ab\;e^{a+b}}{n}+o\left(\frac{1}{n}\right).
\end{align*}
Donc, si $ab\neq0$, $f_n(a+b)-f_n(a)f_n(b)\underset{n\rightarrow+\infty}{\sim}-\frac{ab\;e^{a+b}}{n}$. Si $ab=0$, il est clair que $f_n(a+b)-f_n(a)f_n(b)=0$.}
    \item \question{Même question pour $e^{-a}f_n(a)-1+\frac{a^2}{2n}$.}
\reponse{$e^{-a}f_n(a)\underset{n\rightarrow+\infty}{=}\text{exp}\left(-a+\left(a-\frac{a^2}{2n}+\frac{a^3}{3n^2}\right)+o\left(\frac{1}{n^2}\right)\right)
=1+\left(-\frac{a^2}{2n}+\frac{a^3}{3n^2}\right)+\frac{1}{2}\left(-\frac{a^2}{2n}\right)^2+o\left(\frac{1}{n^2}\right)$, et donc

$$e^{-a}f_n(a)-1+\frac{a^2}{2n}\underset{n\rightarrow+\infty}{\sim}\left(\frac{a^3}{3}+\frac{a^4}{8}\right)\frac{1}{n^2}.$$}
\end{enumerate}
}
