\uuid{UPMY}
\exo7id{4551}
\auteur{quercia}
\organisation{exo7}
\datecreate{2010-03-14}
\isIndication{false}
\isCorrection{true}
\chapitre{Suite et série de fonctions}
\sousChapitre{Autre}

\contenu{
\texte{
Pour $x\in\R^+$ et $n\in\N$, $n\ge 2$ on pose $f_n(x) = \frac{xe^{-nx}}{\ln n}$
et $S(x) = \sum_{n=2}^\infty f_n(x)$ sous réserve de convergence.
}
\begin{enumerate}
    \item \question{\'Etudier la convergence simple, normale, uniforme de la série $\sum f_n$
    sur $\R^+$.}
    \item \question{Montrer que $S$ est de classe $\mathcal{C}^1$ sur $\R^{+*}$.}
    \item \question{Montrer que $S$ n'est pas dérivable à droite en $0$.}
    \item \question{Montrer que $x^kS(x)$ tend vers $0$ en $+\infty$ pour tout $k\in\N$.}
\reponse{
Il y a convergence normale sur tout intervalle $[a,+\infty[$ avec $a>0$.
    Il n'y a pas convergence normale au voisinage de $0$ car
    $\sup\Bigl\{\frac{xe^{-nx}}{\ln n},\ x\ge 0\Bigr\} = \frac1{en\ln n}$ atteint pour $x=\frac1n$ et
    $\sum\frac1{n\ln n}$ diverge (série de Bertrand).
    Par contre il y a convergence uniforme sur $[0,+\infty[$ car
    $$0\le \sum_{k=n}^\infty f_k(x) \le \frac1{\ln n}\sum_{k=n}^\infty xe^{-kx}
       = \frac{xe^{-nx}}{\ln n(1-e^{-x})} \le \frac{\sup\{t/(1-e^{-t}),\ t\ge 0\}}{\ln n}.$$
Lorsque $x\to0^+$, $\frac{S(x)-S(0)}x = \sum_{n=2}^\infty\frac{e^{-nx}}{\ln n}
    \to \sum_{n=2}^\infty\frac1{\ln n} = +\infty$ par convergence monotone.
}
\end{enumerate}
}
