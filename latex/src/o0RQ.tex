\uuid{o0RQ}
\exo7id{7103}
\auteur{megy}
\organisation{exo7}
\datecreate{2017-01-21}
\isIndication{true}
\isCorrection{true}
\chapitre{Géométrie affine euclidienne}
\sousChapitre{Géométrie affine euclidienne du plan}

\contenu{
\texte{
%rotation, octogone, analyse-synthèse
Construire  un octogone régulier inscrit dans un carré donné (c'est-à-dire un octogone dont quatre des huit cotés s'appuient sur les cotés du carré).
}
\indication{Procéder par analyse-synthèse et considérer des rotations.}
\reponse{
On construit l'image du carré par une rotation d'angle $\pi/4$. Les points d'intersection des deux carrés forment un octogone régulier qui répond à la question.
}
}
