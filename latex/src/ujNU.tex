\uuid{ujNU}
\exo7id{2789}
\auteur{burnol}
\organisation{exo7}
\datecreate{2009-12-15}
\isIndication{false}
\isCorrection{true}
\chapitre{Fonction holomorphe}
\sousChapitre{Fonction holomorphe}

\contenu{
\texte{
Retrouver le résultat de l'exercice précédent (l'exercice \ref{ex:burnol1.1.6}) de manière
  plus indirecte en montrant que les coefficients $c_n =
     \sum_{j=0}^n a_j b_{n-j}$ sont 
  ceux de la série de Taylor à l'origine de la fonction
  holomorphe $k(z)=f(z)g(z)$.
}
\reponse{
Il suffit d'utiliser la formule de Leibniz de l'exercice \ref{ex:burnol1.1.5} et le fait que le coefficient $a_n$
du d\'eveloppement de $f$ \`a l'origine est $a_n=f^{(n)}(0)/n!$.
}
}
