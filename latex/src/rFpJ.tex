\uuid{rFpJ}
\exo7id{4599}
\auteur{quercia}
\organisation{exo7}
\datecreate{2010-03-14}
\isIndication{false}
\isCorrection{true}
\chapitre{Série entière}
\sousChapitre{Equations différentielles}

\contenu{
\texte{
On pose $f(x) = \frac{\Arcsin x}{\sqrt{1-x^2}}$.
}
\begin{enumerate}
    \item \question{Montrer que $f$ admet un développement en série entière au voisinage de $0$
    et préciser le rayon de convergence.}
\reponse{Produit de deux séries $ \Rightarrow  R \ge 1$.
          Lorsque $x\to1^-$   $f(x) \to +\infty  \Rightarrow  R = 1$.}
    \item \question{Chercher une équation différentielle d'ordre 1 vérifiée par $f$.
    En déduire les coefficients du développement en série entière de $f$.}
\reponse{$(1-x^2)y' = xy+1  \Rightarrow  (n+2)a_{n+2} = (n+1)a_n  \Rightarrow 
              a_{2k} = a_0\frac{C_{2k}^k}{4^k}$,
             $a_{2k+1} = a_1\frac{4^k}{(2k+1)C_{2k}^k}$.\par
             $a_0 = 0$, $a_1 = 1  \Rightarrow 
             y = \sum_{k=0}^\infty \frac{4^kx^{2k+1}}{(2k+1)C_{2k}^k}$.}
    \item \question{Donner le développement en série entière de $\Arcsin^2 x$.}
\reponse{$\Arcsin^2x = 2 \int_{t=0}^x f(t)\,d t
              = \sum_{k=1}^\infty \frac{2^{2k-1}x^{2k}}{k^2C_{2k}^k}$.}
\end{enumerate}
}
