\uuid{vcqM}
\exo7id{3274}
\auteur{quercia}
\organisation{exo7}
\datecreate{2010-03-08}
\isIndication{false}
\isCorrection{true}
\chapitre{Polynôme, fraction rationnelle}
\sousChapitre{Fraction rationnelle}

\contenu{
\texte{

}
\begin{enumerate}
    \item \question{Soit $F \in {\C(X)}$ telle que $F(e^{2i\pi/n}X) = F(X)$.
    Montrer qu'il existe une unique fraction $G \in {\C(X)}$ telle que $F(X) = G(X^n)$.}
    \item \question{Application : Simplifier
    $\sum_{k=0}^{n-1} \frac {X+e^{2ik\pi/n}}{X-e^{2ik\pi/n}}$.}
\reponse{
$n\frac {X^n+1}{X^n-1}$.
}
\end{enumerate}
}
