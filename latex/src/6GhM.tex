\uuid{6GhM}
\exo7id{4965}
\auteur{quercia}
\organisation{exo7}
\datecreate{2010-03-17}
\isIndication{false}
\isCorrection{true}
\chapitre{Géométrie affine euclidienne}
\sousChapitre{Géométrie affine euclidienne de l'espace}

\contenu{
\texte{
Soit $ABCD$ un tétraèdre dont les quatre faces ont même aire.
Montrer que les côtés non coplanaires ont deux à deux mêmes longueurs.
}
\reponse{
\begin{itemize}
  \item Soient $B'$, $C'$ les projetés orthogonaux de $C$, $D$ sur $(AB)$. 
Le triangle $ABC$ a pour aire $\frac12 AB \cdot CC'$.
Le triangle $ABD$ a pour aire $\frac12 AB \cdot DD'$.
Comme ces triangles ont la même aire alors $CC'=DD'$.

  \item Notons $I$ le milieu de $[C,D]$ et $I'$ son projeté orthogonal sur $(AB)$.
Par définition la droite $(II')$ est perpendiculaire à $(AB)$.
Montrons qu'elle est aussi perpendiculaire à $(CD)$.
Par Pythagore dans $I'CC'$ : $I'C^2 = I'C'^2 + CC'^2$.
Par Pythagore dans $IDD'$ : $I'D^2 = I'D'^2 + DD'^2$.
Comme $I$ est le milieu de $[C,D]$, alors par projection orthogonale, $I'$ est le milieu de $[C',D']$, donc $I'C'=I'D'$. D'autre part $CC'=DD'$ et ainsi $I'C=I'D$. Le triangle $I'CD$ étant isocèle en $I'$, alors $(II')$ est perpendiculaire à $(CD)$.

Bilan : on a montré que la perpendiculaire commune à $(AB)$ et $(CD)$ est $(II')$ et passe donc par le milieu de $[C,D]$.

\item On refait le même raisonnement en projetant cette fois sur $(CD)$, pour trouver que la perpendiculaire commune à $(AB)$ et $(CD)$ passe par le milieu $J$ de $[A,B]$. Comme la perpendiculaire commune est unique et coupe $(AB)$ et $I'$ et en $J$ alors $I'=J$. 
Bilan : la perpendiculaire commune à $(AB)$ et $(CD)$ est la droite $(IJ)$ où
$I$ est le milieu de $[C,D]$ et $J$ le milieu de $[A,B]$.

\item On considère le retournement d'axe $(IJ)$ (rotation d'angle $\pi$). Alors $A$ s'envoie sur $B$, $C$ s'envoie sur $D$. Le segment $[A,C]$ s'envoie sur $[B,D]$. Un retournement préserve les longueurs, donc $AC = BD$.

Conclusion : les deux côtés opposés $AC$ et $BD$ ont même longueur.
\end{itemize}
}
}
