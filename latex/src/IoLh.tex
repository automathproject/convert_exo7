\uuid{IoLh}
\exo7id{5613}
\auteur{rouget}
\organisation{exo7}
\datecreate{2010-10-16}
\isIndication{false}
\isCorrection{true}
\chapitre{Matrice}
\sousChapitre{Inverse, méthode de Gauss}

\contenu{
\texte{
Soit $n$ un entier naturel supérieur ou égal à $2$ et $\omega=e^{2i\pi/n}$.

Soit $A =(\omega^{(j-1)(k-1)})_{1\leqslant j,k\leqslant n}$. Montrer que $A$ est inversible et calculer $A^{-1}$.
}
\reponse{
Calculons $A\overline{A}$. Soit $(j,k)\in\llbracket1,n\rrbracket^2$. Le coefficient ligne $j$, colonne $k$ de $A\overline{A}$ vaut

\begin{center}
$\sum_{u=1}^{n}\omega^{(j-1)(u-1)}\omega^{-(u-1)(k-1)}=\sum_{u=1}^{n}\left(\omega^{j-k}\right)^{u-1}$.
\end{center}

\textbullet~Si $j=k$, ce coefficient vaut $n$.

\textbullet~Si $j\neq k$, puisque $j-k$ est strictement compris entre $-n$ et $n$ et que $j-k$ n'est pas nul, $\omega^{j-k}$ est différent de $1$. Le coefficient ligne $j$, colonne $k$, de $A\overline{A}$ est donc égal à $\frac{1-\left(\omega^{j-k}\right)^{n}}{1-\omega^{j-k}}=\frac{1-1}{1-\omega^{j-k}}= 0$.

Finalement, $A\overline{A}=nI_n$. Ainsi, $A$ est inversible à gauche et donc inversible, d'inverse $A^{-1}=\frac{1}{n}\overline{A}$.
}
}
