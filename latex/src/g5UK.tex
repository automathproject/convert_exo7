\uuid{g5UK}
\exo7id{3619}
\auteur{quercia}
\organisation{exo7}
\datecreate{2010-03-10}
\isIndication{false}
\isCorrection{true}
\chapitre{Réduction d'endomorphisme, polynôme annulateur}
\sousChapitre{Sous-espace stable}

\contenu{
\texte{
Soit $f\in\mathcal{L}(\R^3)$ ayant pour matrice $M=\left(\begin{smallmatrix}1&1&1\cr1&1&1\cr-1&1&1\cr\end{smallmatrix}\right)$
dans la base canonique de~$\R^3$. Déterminer les sous-espaces de~$\R^3$ stables par~$f$.
}
\reponse{
$\mathrm{Spec}(f) = \{0,1,2\}$ donc $f$ est diagonalisable et chaque sous-espace propre est de dimension~$1$.
Comme la restriction d'un diagonalisable à un sous-espace stable est encore diagonalisable,
les sous-espaces stables par~$f$ sont les huit sous-sommes de~$E_0\oplus E_1\oplus E_2$.
}
}
