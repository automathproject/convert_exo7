\uuid{Kta3}
\exo7id{4105}
\auteur{quercia}
\organisation{exo7}
\datecreate{2010-03-11}
\isIndication{false}
\isCorrection{true}
\chapitre{Equation différentielle}
\sousChapitre{Equations différentielles linéaires}

\contenu{
\texte{

}
\begin{enumerate}
    \item \question{Soit $f$ de classe $\mathcal{C}^1$ de $[0,\pi]$ dans $\R$ telle que $f(0) = f(\pi) = 0$.
Montrer que $ \int_0^\pi f^2 \le  \int_0^\pi f'^2$.

{\it Indication~: prolonger $f$ en une fonction impaire $2\pi$-périodique.}}
\reponse{Après le prolongement indiqué on peut appliquer le relation
de Parseval à $f$ et $f'$ sachant que $c_0(f) = 0$ par imparité et
$|c_n(f)| = |c_n(f')|/n\le |c_n(f')|$ pour $n\ne0$.}
    \item \question{Soit une fonction $q$ de classe $\mathcal{C}^1$ sur $[0,\pi]$, à valeurs
dans $]-\infty,1[$. Montrer que l'unique fonction $x$ de classe $\mathcal{C}^2$
s'annulant en $0$ et en $\pi$ et vérifiant l'équation différentielle
$x''(t) + q(t)x(t) = 0$ est la fonction nulle.}
\reponse{$x''(t) + q(t)x(t) = 0  \Rightarrow   \int_0^\pi x'^2 = \Bigl[xx'\Bigr]_0^\pi -  \int_0^\pi xx''
=  \int_0^\pi qx^2  \Rightarrow  x'=0  \Rightarrow  x=0$.

Rmq: il n'est pas nécessaire d'avoir $q$ de classe $\mathcal{C}^1$.}
    \item \question{Soit $f$ une fonction de classe $\mathcal{C}^1$ sur $[0,\pi]$ et deux réels $a,b$ fixés.
Montrer qu'il existe une unique solution $x$ de classe $\mathcal{C}^2$ vérifiant $x(0) = a$,
$x(\pi) = b$ et $x''(t) + q(t)x(t) = f(t)$.}
\reponse{Il existe $x_0$ de classe $\mathcal{C}^2$ vérifiant l'équation différentielle.
Par différence avec $x_0$ on se ramène au cas $f = 0$ et il faut montrer
que l'application $\varphi : {\cal S} \to {\R^2}, x \mapsto {(x(0),x(\pi))}$ est bijective,
en notant $\cal S$ l'espace des solutions de l'équation homogène $x'' + qx = 0$.
Or $\varphi$ est linéaire et est injective d'après la question précédente, c'est
donc une bijection car $\dim{\cal S} = 2$.

Remarque~: l'hypothèse $f$ de classe $\mathcal{C}^1$ est inutile, continue suffit.}
\end{enumerate}
}
