\uuid{tcm3}
\exo7id{1426}
\auteur{ortiz}
\organisation{exo7}
\datecreate{1999-04-01}
\isIndication{false}
\isCorrection{true}
\chapitre{Groupe, anneau, corps}
\sousChapitre{Groupe de permutation}

\contenu{
\texte{

}
\begin{enumerate}
    \item \question{Montrer que $\mathcal{S}_n$ est isomorphe \`a un sous-groupe de $\mathcal{A}_{n+2}.$}
    \item \question{Montrer que $\mathcal{S}_4$ n'est pas isomorphe \`a un sous-groupe de $\mathcal{A}_5.$}
    \item \question{Montrer que $\mathcal{S}_5$ n'est pas isomorphe \`a un sous-groupe de $\mathcal{A}_6.$}
\reponse{
$\mathcal{S}_{N}$ est l'ensemble des permutations de l'ensemble
$\{ 1,2,\ldots,N \}$. Dans $\mathcal{S}_{n+2}$ notons $\tau$ la
permutation $(n+1,n+2)$. Nous définissons une application $\phi :
\mathcal{S}_{n} \longrightarrow \mathcal{S}_{n+2}$ par les
relations
$$ \phi(\sigma) = \sigma \text{ si } \epsilon(\sigma) = +1 \quad ;
\quad \phi(\sigma) =  \sigma \circ \tau \text{ sinon ;}
$$
o\`u $\epsilon$ désigne la signature. Alors $\phi$ est un
morphisme de groupe, de plus quelque soit $\sigma \in
\mathcal{S}_{n}$ alors $\epsilon (\phi(\sigma))=+1$ (si
$\epsilon(\sigma)= +1$ c'est clair, sinon $\epsilon
(\phi(\sigma))=\epsilon(\sigma)\times \epsilon(\tau) = (-1)\times
(-1) = +1$). Donc $\phi (\mathcal{S}_{n})$ est un sous-groupe de
$\mathcal{A}_{n+2}$.

Enfin $\phi$ est injective : en effet soit $\sigma$ tel que
$\phi(\sigma)=\mathrm{id}$. Soit $\epsilon(\sigma) = +1 $ et alors
$\phi(\sigma)= \sigma = \mathrm{id}$ ; soit $\epsilon(\sigma) = -1$ et
alors $\phi(\sigma)= \sigma \circ \tau$, pour $j \in
\{1,2,\ldots,n\}$
 $j = \phi(\sigma)(j)= \sigma \circ \tau(j) = \sigma(j)$, et donc
quelque soit $j\in \{1,2,\ldots,n\}$ $\sigma(j)=j$ et donc $\sigma
= \mathrm{id}$. On vient de démontrer que la composée de deux permutations
à supports disjoint est l'identité si et seulement si les
permutations sont déjà l'identité !


Notons encore $\phi : \mathcal{S}_{n} \longrightarrow
\phi(\mathcal{S}_{n} )$ le morphisme induit par $\phi$. Il est
injectif et surjectif, donc $\mathcal{S}_{n}$ est isomorphe
$\phi(\mathcal{S}_{n})$ qui est un sous-groupe de
$\mathcal{A}_{n+2}$.
$\mathcal{A}_5$ est  de cardinal $5 !/2=60$, et comme $24 = \mathrm{Card} \mathcal{S}_4$ ne divise
pas $60$    alors $\mathcal{A}_5$  n'a pas de sous-groupe d'ordre
$24$.
C'est un peu plus délicat car $\mathrm{Card} \mathcal{S}_5 = 5 ! = 120$ divise
$\mathrm{Card} \mathcal{A}_6 = 6 !/2 = 360$ donc l'argument ci-dessus
n'est pas valide. Cependant s'il existe un isomorphisme entre
$\mathcal{S}_5$ et un sous-groupe de $\mathcal{A}_6$ alors un
cycle d'ordre $5$ de $\mathcal{S}_5$ est envoyé sur une
permutation $\sigma \in \mathcal{A}_6$ d'ordre $5$.

Décomposons $\sigma$ en produit de cycles à supports disjoints,
$\sigma = \sigma_1 \circ \sigma_2 \circ \cdots$. Comme les cycles
$\sigma_i$ sont à supports disjoints, il y a au plus trois cycles
(de longueur $\geq 2$) dans la décomposition (car dans
$\mathcal{A}_6$ on peut permuter au plus $6$ éléments).
\begin{itemize}
Le cas $\sigma = \sigma_1$ n'est pas possible car alors $\sigma_1$ serait un cycle d'ordre $5$ et donc
de signature $-1$ dans $\mathcal{A}_6$.
Si $\sigma = \sigma_1 \circ \sigma_2$ alors les longueurs de $\sigma_1$
  et $\sigma_2$ sont $(4,2)$ ou $(2,2)$, et l'ordre de leur composée
  $\sigma_1 \circ \sigma_2$ est donc $4$ ou $2$ mais pas $5$.
Si $\sigma = \sigma_1 \circ \sigma_2 \circ \sigma_3$ alors
les $\sigma_i$ sont des transpositions, et la signature de
$\sigma$ est alors $-1$ ce qui contredit $\sigma \in
\mathcal{A}_6$.
\end{itemize}
}
\end{enumerate}
}
