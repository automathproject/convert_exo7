\uuid{0kAM}
\exo7id{7831}
\auteur{mourougane}
\organisation{exo7}
\datecreate{2021-08-11}
\isIndication{false}
\isCorrection{false}
\chapitre{Forme bilinéaire}
\sousChapitre{Forme bilinéaire}

\contenu{
\texte{
Soit $\mathcal{Q}$  une quadrique de $P(E)$  (muni d'un repère
projectif) d'équation $q(x)=0$ o{ù} $q$ est la forme quadratique d'une
forme bilinéaire symétrique non dégénérée $f$ sur un espace vectoriel
$E$. Si $\vec{A}$ et $\vec{B}$ sont deux sous-espaces orthogonaux dans
$E$, on note $P(\vec{A})\perp P(\vec{B})$. On appelle hyperplan
polaire d'un point $A=P(\vec{A})$ de $P(E)$ l'hyperplan projectif
$A^\perp:=P(\vec{A}^\perp)$.
}
\begin{enumerate}
    \item \question{On munit de plan projectif d'un repère. Déterminer une équation
de la droite polaire du point $M(x_0,y_0,1)$ par rapport à la
quadrique d'équation $x^2+y^2-z^2=0$.  La représenter dans l'espace
affine d'équation $z\not=0$.}
    \item \question{Soit $F$ un sous-espace non-isotrope de $E$. Soit $A$ et $B$
 deux points de $P(F)$. Montrer si $A\perp B$ pour $f$ alors $A\perp
 B$ pour $f_{|F}$.}
    \item \question{Soit $\mathcal{Q}$ une quadrique de $P^1(K)$ dont l'image est
 composée des deux points $A$ et $B$. Montrer en utilisant un bon
 repère que pour tout $M$ in $P^1(K)$,
$$M\perp N\iff M \textrm{ et } N \textrm{ sont conjugués harmoniques
 par rapport à } M \textrm{ et } N.$$}
    \item \question{En déduire une construction géométrique de la polaire d'un point
 par rapport à une conique.}
\end{enumerate}
}
