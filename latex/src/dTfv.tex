\uuid{dTfv}
\exo7id{1428}
\auteur{legall}
\organisation{exo7}
\datecreate{1998-09-01}
\isIndication{false}
\isCorrection{true}
\chapitre{Sous-groupe distingué}
\sousChapitre{Sous-groupe distingué}

\contenu{
\texte{
Soit $  G  $ un groupe, $  H  $ et $  K  $ deux
sous-groupes d'ordre fini de $  G  $ tels que $  H\cap K= \{ e_G\}
.$
}
\begin{enumerate}
    \item \question{Montrer que le cardinal de $  HK  $ est \'egal  $  \vert H\vert \vert K\vert   .$}
\reponse{Soit $\phi : H\times K \rightarrow HK$ d\'efinie par
$\phi(h,k) = hk$. Montrons que $\phi$ est bijective : $\phi$ est
surjective par d\'efinition de $HK$ et si $\phi(h,k)=\phi(h',k')$
alors $hk=h'k'$ et donc ${h'}^{-1}h = k'k^{-1}$ or $H\cap K =\{
e_G \}$ et donc ${h'}^{-1}h = e_G$ et donc $h=h'$, de m\^eme
$k=k'$ et donc $\phi$ est injective.

Comme $\phi$ est bijective $\mathrm{Card} H\times K = \mathrm{Card} HK$ et donc
$\mathrm{Card} HK = \mathrm{Card} H . \mathrm{Card} K$.}
    \item \question{En d\'eduire que si $  \vert G\vert =pq  $ o\`u $  p   $ est premier
et $  p>q  $ alors $  G  $ a au plus un sous-groupe d'ordre $  p  .$ Montrer que si ce
sous-groupe existe il est distingu\'e dans $  G  .$}
\reponse{Supposons qu'il existe deux sous-groupes $H$ et $K$ distincts et d'ordre
$p$. Montrons d'abord que $H\cap K =\{ e_G \}$. En effet $H\cap K$
est un sous-groupe de $H$ et donc le cardinal de $H\cap K$ divise
$\mathrm{Card} H = p$ avec $p$ premier. Or comme $H\not= K$ alors $H\cap K
\not= H$ et donc $\mathrm{Card} H\cap K =1$, c'est ce que nous voulions
d\'emontrer.

Maintenant d'apr\`es la premi\`ere question $HK$ est un
ensemble de cardinal $p^2$ dans le groupe  $G$ de cardinal $pq
< p^2$. Donc il ne peut exister deux sous-groupe d'ordre $p$.

Supposons maintenant que $H$ soit un sous-groupe d'ordre $p$,
c'est donc l'unique sous-groupe d'ordre $p$ d'apr\`es ce que nous
venons de d\'emontrer. Pour $g\in G$ le sous-groupe $gHg^{-1}$ est
du m\^eme ordre que $H$ (car pour $g$ fix\'e le morphisme
$\theta_g$ de $G$ dans $G$, $\theta_g(h) =ghg^{-1}$ est un
automorphisme et en particulier un biction donc $\mathrm{Card}
\theta_g(H)=\mathrm{Card} H$ ). Par cons\'equent $gHg^{-1} = H$ et donc
$H$ est un sous-groupe distingu\'e.}
\end{enumerate}
}
