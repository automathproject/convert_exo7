\uuid{Pc4x}
\exo7id{4502}
\auteur{quercia}
\organisation{exo7}
\datecreate{2010-03-14}
\isIndication{false}
\isCorrection{true}
\chapitre{Série numérique}
\sousChapitre{Familles sommables}

\contenu{
\texte{
Déterminer l'ensemble de définition de $f(x) = \sum_{k=2}^\infty\frac{(-1)^k}{x+k}$.
Montrer que $f$ est de classe $\mathcal{C}^\infty$ sur son domaine et la développer
en série entière.
}
\reponse{
La série converge pour tout $x\notin\{-2,-3,\dots\}$ car le critère
des séries alternées s'applique à partir d'un certain rang (fonction de~$x$).
Il en va de même pour toutes les séries obtenues par dérivations successives
terme à terme, et ces séries convergent localement uniformément (le reste
d'une série vérifant le CSA est majoré en valeur absolue par la valeur
absolue du premier terme figurant dans le reste) donc $f$ est $\mathcal{C}^\infty$.

Pour $|x|<2$ on a
$$f(x) = \sum_{k=2}^\infty\sum_{n=0}^\infty\frac{(-1)^k}k\Bigl(\frac{-x}k\Bigr)^n
      = \sum_{k=2}^\infty\sum_{n=0}^\infty\frac{(-1)^{k+n}}{k^{n+1}}x^n
      = (1-\ln2) + \sum_{n=1}^\infty(-1)^n(1-(1-2^{-n})\zeta(n+1))x^n.$$
}
}
