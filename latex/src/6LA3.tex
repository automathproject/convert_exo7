\uuid{6LA3}
\exo7id{2675}
\auteur{matexo1}
\organisation{exo7}
\datecreate{2002-02-01}
\isIndication{false}
\isCorrection{true}
\chapitre{Théorème des résidus}
\sousChapitre{Théorème des résidus}

\contenu{
\texte{
D{\'e}velopper la fonction
$$f(x) = {1\over 2-\cos x} $$ 
en s{\'e}rie de Fourier, en calculant les coefficients par
la m{\'e}thode des r{\'e}sidus.
}
\reponse{
% correction ajoutée le 18/11/2018
La fonction est paire, donc ses coefficients $b_n$ sont nuls et on calcule ici ses coefficients $a_n$.

Pour $n\geq 1$, on a : 
\[ a_n 
= \frac{1}{\pi} \int_0^{2\pi}f(t)\cos(nt)dt
= \frac{1}{\pi} \int_0^{2\pi}\frac{\cos(nt)}{2-\cos(t)}dt
\]

On peut écrire 
\begin{align*}
a_n &= \Re \frac{1}{\pi} \int_0^{2\pi}f(t)e^{int}dt\\
&= \Re \frac{1}{\pi} \int_{\mathcal C}\frac{z^n}{2-\frac12 \left(z+z^{-1}\right)}\frac{dz}{iz}\\
&= \Re \frac{1}{i\pi} \int_{\mathcal C}\frac{2z^n}{4z-z^2-1}dz\\
&= 4\Re \sum_\alpha Res\left(\frac{z^n}{4z-z^2-1},\alpha\right)
\end{align*}
Comme $n\geq 1$, il n'y a qu'un seul pôle dans le disque : le pôle simple $2-\sqrt 3$. Le résidu en ce pôle vaut $\frac{(2-\sqrt 3)^n}{2\sqrt 3}$ qui est  déjà un nombre réel.\\

On en déduit que 
\[ a_n = \frac{2}{\sqrt 3}(2-\sqrt 3)^n.\]

Le coefficient $a_0$ est obtenu de la même façon, mais il faut diviser par deux, vu la définition particulière lorsque $a=0$. On obtient donc $a_0 = \frac{1}{\sqrt 3}$.

Finalement, on a donc :
$$ f(x) = {1\over\sqrt 3}\left[1+ 2\sum_{n=0}^{+\infty } 
(2-\sqrt 3)^n \cos nx\right].$$
}
}
