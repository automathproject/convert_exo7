\uuid{4c5g}
\exo7id{5948}
\auteur{tumpach}
\organisation{exo7}
\datecreate{2010-11-11}
\isIndication{false}
\isCorrection{true}
\chapitre{Théorème de convergence dominée}
\sousChapitre{Théorème de convergence dominée}

\contenu{
\texte{
Soit $(\Omega, \Sigma, \mu)$ un espace mesur\'e.  On dit que
$f_{n}$ converge vers $f$ \emph{en mesure} si pour tout
$\varepsilon$,
$$\lim_{n\rightarrow+\infty} \mu\{x\in\Omega, ~|f_{n}(x) - f(x)| ~> \varepsilon\} ~ =
0.
$$
Montrer que si $f_{n}\rightarrow f$ en mesure, alors il existe une
sous-suite $\{f_{n_{k}}\}_{k\in\mathbb{N}}$ de
$\{f_{n}\}_{n\in\mathbb{N}}$ qui converge vers $f$ $\mu$-presque
partout.
}
\reponse{
On cherche une sous-suite $\{ f_{n_{k}}\}_{n\in\mathbb{N}}$ de $\{
f_{n}\}_{n\in\mathbb{N}}$ telle que pour $\mu$-presque tout
$x\in\Omega$, \'etant donn\'e un $\varepsilon>0$, il existe un
$k\in\mathbb{N}$ (d\'ependant \`a priori de $x$) v\'erifiant
$j\geq k \Rightarrow |f_{n_{j}}(x) - f(x)| < \varepsilon$. Il
suffit de montrer que pour $\mu$-presque tout $x$, il existe un
$k\in\mathbb{N}$ tel que $j\geq k \Rightarrow |f_{n_{j}}(x) -
f(x)| < \frac{1}{2^{j}} \leq \frac{1}{2^k}.$ Cela revient \`a
montrer que le compl\'ementaire de l'ensemble
$$
A := \bigcup_{k=1}^{\infty} \bigcap_{j\geq k} \left\{|f_{n_{j}} -
f|< \frac{1}{2^{j}}\right\}
$$
est de mesure nulle. Or
$$
A^{c} = \bigcap_{k=1}^{\infty} \bigcup_{j\geq k}\left\{|f_{n_{j}}
- f|\geq \frac{1}{2^{j}}\right\}.
$$
Posons $B_{k} := \bigcup_{j\geq k}\left\{|f_{n_{j}} - f|\geq
\frac{1}{2^{j}}\right\}$. On a $B_{1} \supset B_{2} \supset B_3
\dots$ avec $B_1$ de mesure fini ; donc par continuit\'e de la mesure, il vient~:
$$
\mu(A^{c}) = \lim_{k\rightarrow +\infty} \mu(B_{k}).
$$
Par $\sigma$-additivit\'e, on a~:
$$
\mu(B_{k}) ~\leq~\sum_{j\geq k} \mu\left( \left\{|f_{n_{j}} -
f|\geq \frac{1}{2^{j}}\right\}\right).
$$
On d\'efinit alors la sous-suite $\{ f_{n_{k}}\}_{n\in\mathbb{N}}$
de la mani\`ere suivante. Puisque $f_{n}$ converge vers $f$ en
mesure, il existe un indice $n_1$ tel que pour $n\geq n_1$,
$$
\mu\left( \left\{|f_{n} - f|\geq \frac{1}{2}\right\}\right) \leq
\frac{1}{2}.
$$
Il existe un indice $n_{2}> n_1$ tel que pour $n\geq n_2$,
$$
\mu\left( \left\{|f_{n} - f|\geq \frac{1}{2^2}\right\}\right) \leq
\frac{1}{2^2},
$$
et ainsi de suite: pour tout $k\in\mathbb{N}$, il existe un
$n_{k}> n_{k-1}$, tel que pour $n\geq n_{k}$
$$
\mu\left( \left\{|f_{n} - f|\geq \frac{1}{2^k}\right\}\right) \leq
\frac{1}{2^k}.
$$
Pour cette sous-suite on a alors~:
$$
\mu(B_{k}) ~\leq~\sum_{j\geq k} \mu\left( \left\{|f_{n_{j}} -
f|\geq \frac{1}{2^{j}}\right\}\right) ~\leq ~\sum_{j\geq k}
\frac{1}{2^j} = \frac{1}{2^{k-1}}.
$$
On a bien
$$
\mu(A^{c}) = \lim_{k\rightarrow +\infty} \mu(B_{k}) = 0.
$$
}
}
