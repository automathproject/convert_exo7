\uuid{2FON}
\exo7id{1625}
\auteur{barraud}
\organisation{exo7}
\datecreate{2003-09-01}
\isIndication{false}
\isCorrection{false}
\chapitre{Réduction d'endomorphisme, polynôme annulateur}
\sousChapitre{Diagonalisation}

\contenu{
\texte{
Soit $A\in\mathcal{M}_{n}(\R)$. Montrer que si $\lambda$ est une valeur propre
complexe de $A$, alors $\bar{\lambda}$ est aussi une valeur propre de
$A$. De m\^{e}me, montrer que si $x$ est un vecteur propre complexe de
$A$, alors $\bar{x}$ (o\`{u} $\bar{x}$ d\'{e}signe le vecteur dont les
composantes sont les conjugu\'{e}es des composantes de $x$) est aussi un
vecteur propre complexe de $A$.

Diagonaliser
 $
  A=
  \begin{pmatrix}
    -1 &  1 &  0 \\
     0 & -1 &  1 \\
     1 &  0 & -1
  \end{pmatrix}
 $.
}
}
