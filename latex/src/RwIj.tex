\uuid{RwIj}
\exo7id{3}
\auteur{cousquer}
\organisation{exo7}
\datecreate{2003-10-01}
\isIndication{true}
\isCorrection{true}
\chapitre{Nombres complexes}
\sousChapitre{Forme cartésienne, forme polaire}

\contenu{
\texte{
Écrire sous la forme $a+ib$ les nombres complexes suivants :
}
\begin{enumerate}
    \item \question{Nombre de module $2$ et d'argument $\pi /3$.}
\reponse{$z_1 = 2 e^{i\frac \pi 3} = 2(\cos \frac \pi 3 + i \sin \frac \pi 3) = 2 (\frac 12+ i\frac{\sqrt3}{2}) = 1+i\sqrt 3$.}
    \item \question{Nombre de module $3$ et d'argument $-\pi /8$.}
\reponse{$z_2 = 3e^{-i\frac \pi 8} = 3\cos {\pi \over 8}-3i\sin{\pi \over8}={3\sqrt{2+\sqrt2}\over 2}-{3i\sqrt{2-\sqrt2}\over 2}$.


Il nous reste à expliquer comment nous avons calculé $\cos \frac \pi 8$ et $\sin \frac\pi 8$: 
posons $\theta=\frac{\pi}{8}$,
alors $2\theta = \frac \pi 4$ et donc $\cos(2\theta)= \frac {\sqrt 2}{2} = \sin(2\theta)$.
Mais $\cos(2\theta) = 2\cos^2 \theta - 1$. Donc $\cos^2\theta = \frac{\cos(2\theta) + 1}{2} = \frac 14 (2 + \sqrt 2)$.
Et ensuite  $\sin^2 \theta = 1- \cos^2 \theta = \frac 14 (2 - \sqrt 2)$.
Comme $0 \le \theta = \frac \pi 8 \le \frac \pi 2$, $\cos \theta$ et $\sin \theta$ sont des nombres positifs. Donc
$$\cos \frac \pi 8 = \frac 12 \sqrt{2 + \sqrt 2} \quad, \quad \sin \frac \pi 8 = \frac 12 \sqrt{2 - \sqrt 2}.$$}
\indication{Il faut bien connaître ses formules trigonométriques.
En particulier si l'on connait $\cos(2\theta)$ ou $\sin(2\theta)$ on sait
calculer $\cos \theta$ et $\sin \theta$.}
\end{enumerate}
}
