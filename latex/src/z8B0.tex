\uuid{z8B0}
\exo7id{5263}
\auteur{rouget}
\organisation{exo7}
\datecreate{2010-07-04}
\isIndication{false}
\isCorrection{true}
\chapitre{Matrice}
\sousChapitre{Propriétés élémentaires, généralités}

\contenu{
\texte{
Montrer que $\{\frac{1}{\sqrt{1-x^2}}\left(
\begin{array}{cc}
1&x\\
x&1
\end{array}
\right),\;x\in]-1,1[\}$ est un groupe pour la multiplication des matrices.
}
\reponse{
Pour $x\in]-1,1[$, posons $M(x)=\frac{1}{\sqrt{1-x^2}}\left(
\begin{array}{cc}
1&x\\
x&1
\end{array}
\right)$. Posons ensuite $G=\{M(x),\;x\in]-1,1[\}$.

Soit alors $x\in]-1,1[$. Posons $a=\Argth x$ de sorte que $x=\mbox{th}a$. On a 

$$M(x)=\frac{1}{\sqrt{1-x^2}}\left(
\begin{array}{cc}
1&x\\
x&1
\end{array}
\right)=\ch a\left(
\begin{array}{cc}
1&\tanh a\\
\tanh a&1
\end{array}
\right)=\left(
\begin{array}{cc}
\ch a&\sh a\\
\sh a&\ch a
\end{array}
\right).$$

Posons, pour $a\in\Rr$, $N(a)=\left(
\begin{array}{cc}
\ch a&\sh a\\
\sh a&\ch a
\end{array}
\right)$. On a ainsi $\forall x\in]-1,1[,\;M(x)=N(\Argth x)$ ou aussi, $\forall a\in\Rr,\;N(a)=M(\mbox{th}a)$. Par suite, $G=\{N(a),\;a\in\Rr\}$.

Soit alors $(a,b)\in\Rr^2$.

\begin{align*}\ensuremath
N(a)N(b)&=\left(
\begin{array}{cc}
\ch a&\sh a\\
\sh a&\ch a
\end{array}
\right)\left(
\begin{array}{cc}
\ch b&\sh b\\
\sh b&\ch b
\end{array}
\right)=\left(
\begin{array}{cc}
\ch a\ch b+\sh a\sh b&\sh a\ch b+\sh b\ch a\\
\sh a\ch b+\sh b\ch a&\ch a\ch b+\sh a\sh b
\end{array}
\right)
\\
 &=\left(
\begin{array}{cc}
\ch(a+b)&\sh(a+b)\\
\sh(a+b)&\ch(a+b)
\end{array}
\right)
=N(a+b).
\end{align*}

Montrons alors que $G$ est un sous-groupe de $(\mathcal{GL}_2(\Rr),\times)$.

$N(0)=I_2\in G$ et donc $G$ est non vide.

$\forall a\in\Rr,\;\mbox{det}(N(a))=\ch^2a-\sh^2a=1\neq0$ et donc $G\subset\mathcal{GL}_2(\Rr)$.

$\forall(a,b)\in\Rr^2,\;N(a)N(b)=N(a+b)\in G$.

$\forall a\in\Rr,\;(N(a))^{-1}=\left(
\begin{array}{cc}
\ch a&-\sh a\\
-\sh a&\ch a
\end{array}
\right)=N(-a)\in G$.

On a montré que $G$ est un sous-groupe de $(\mathcal{GL}_2(\Rr),\times)$.
}
}
