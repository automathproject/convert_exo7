\uuid{8zOW}
\exo7id{5609}
\auteur{rouget}
\organisation{exo7}
\datecreate{2010-10-16}
\isIndication{false}
\isCorrection{true}
\chapitre{Matrice}
\sousChapitre{Autre}

\contenu{
\texte{
On appelle idéal bilatère de l'anneau $(\mathcal{M}_n(\Kk),+,\times)$ tout sous-ensemble I de $\mathcal{M}_n(\Kk)$ tel que

\begin{center}
a) $(I,+)$ est un groupe et b) $\forall A\in I$,  $\forall M\in\mathcal{M}_n(\Kk)$, $AM\in I$ et $MA\in I$.
\end{center}

Déterminer tous les idéaux bilatères de l'anneau $(\mathcal{M}_n(\Kk),+,\times)$.
}
\reponse{
$\{0\}$ est un idéal bilatère de l'anneau $\mathcal{M}_n(\Kk),+,\times)$.

Soit $I$ un idéal non nul de de l'anneau $\mathcal{M}_n(\Kk),+,\times)$. Montrons que $I=\mathcal{M}_n(\Kk)$.

Il existe une matrice $A$ non nulle dans $I$. Pour tout quadruplet d'indices $(i,j,k,l)$, $I$ contient le produit

\begin{center}
$E_{i,j}AE_{k,l}= \sum_{1\leqslant u,v\leqslant n}^{}a_{u,v}E_{i,j}E_{u,v}E_{k,l}=a_{j,k}E_{i,l}$.
\end{center}

$A$ est non nulle et on peut choisir $j$ et $k$ tels que $a_{j,k}$ soit non nul. $I$ contient alors $a_{j,k}E_{i,l}\frac{1}{a_{j,k}}I_n= E_{i,l}$. Finalement $I$ contient toutes les matrices élémentaires et donc encore toutes les sommes du type $\sum_{1\leqslant i,j\leqslant n}^{}m_{i,j}I_nE_{i,j}=(m_{i,j})_{1\leqslant i,j\leqslant n}$, c'est-à-dire $\mathcal{M}_n(\Kk)$ tout entier.

\begin{center}
\shadowbox{
Les idéaux bilatères de l'anneau $\mathcal{M}_n(\Kk),+,\times)$ sont $\{0\}$ et $\mathcal{M}_n(\Kk)$.
}
\end{center}
}
}
