\uuid{jN4l}
\exo7id{4377}
\auteur{quercia}
\organisation{exo7}
\datecreate{2010-03-12}
\isIndication{false}
\isCorrection{true}
\chapitre{Intégration}
\sousChapitre{Intégrale de Riemann dépendant d'un paramètre}

\contenu{
\texte{
\smallskip
Soit $f(x) =  \int_{t=0}^1\frac{1-t}{\ln t}t^x\,d t$.
Étudier le domaine de définition de~$f$, sa dérivabilité, puis calculer $f(x)$.
}
\reponse{
Il y a convergence si et seulement si $x>-1$.
$f'(x) =  \int_{t=0}^1 (1-t)t^x\,d t = \frac1{x+1} - \frac1{x+2}$,
donc $f(x) = \ln\Bigl(\frac{x+1}{x+2}\Bigr) + C$ et $f(x)\to 0$ (lorsque $x\to+\infty$) d'où $C=0$.
}
}
