\uuid{DE0m}
\exo7id{6926}
\auteur{ruette}
\organisation{exo7}
\datecreate{2013-01-24}
\isIndication{false}
\isCorrection{true}
\chapitre{Probabilité continue}
\sousChapitre{Densité de probabilité}

\contenu{
\texte{
Soit $X_1,\ldots, X_n$ des variables aléatoires indépendantes de même loi 
exponentielle
$\mathcal{E}(1)$ et $Z=\min(X_1,\ldots,X_n)$. Déterminer la loi de $Z$.
}
\reponse{
\def\I1{{ \rm 1\:\!\!\! l}}
$P(Z>t)=P(X_1>t, X_2>T,\ldots, X_n>t)=P(X_1>t)P(X_2>t)\cdots P(X_n>t)$
par indépendance. Comme $P(X>t)=1-P(X\leq t)$, on a
$\displaystyle
1-F_Z(t)=\prod_{1\leq i\leq n} (1-F_{X_i}(t)).$\\
Comme $F_{X_i}(t)=1-e^{-t}$ si $t\geq 0$ et $F_{X_i}(t)=0$ sinon, on trouve
que $F_Z(t)=1-e^{-nt}$ si $t\geq 0$ et $F_Z(t)=0$ sinon. 
La densité de $Z$ est donc $\I1_{\Rr_+}ne^{-nt}$. Par conséquent, 
$Z$ est de loi exponentielle de paramètre $n$.
}
}
