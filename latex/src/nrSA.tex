\uuid{nrSA}
\exo7id{5503}
\auteur{rouget}
\organisation{exo7}
\datecreate{2010-07-10}
\isIndication{false}
\isCorrection{true}
\chapitre{Géométrie affine euclidienne}
\sousChapitre{Géométrie affine euclidienne de l'espace}

\contenu{
\texte{
Matrice dans la base canonique orthonormée directe de la rotation vectorielle de $\Rr^3$ autour de $(1,2,2)$ qui transforme $j$ en $k$.
}
\reponse{
Soit $r$ la rotation cherchée. Notons $u$ le vecteur $\frac{1}{3}(1,2,2)$ ($u$ est unitaire) et $\theta$ l'angle de $r$. $r$ est la rotation d'angle $\theta$ autour du vecteur unitaire $u$.  On sait que pour tout vecteur $v$ de $\Rr^3$

\begin{center}
$r(v)=(\cos\theta)v+(1-\cos\theta)(v.u)u+(\sin\theta) u\wedge v\quad(*)$
\end{center}
et en particulier que $[v,r(v),u]=\sin\theta\|v\wedge u\|^2$. L'égalité $r(j)=k$ fournit

\begin{center}
$\sin\theta\|j\wedge u\|^2=\left[j,r(j),u\right]=\left[u,j,k\right]=\frac{1}{3}\left|\begin{array}{ccc}
1&0&0\\
2&1&0\\
3&0&1
\end{array}
\right|=\frac{1}{3}$.
\end{center}

Comme $u\wedge j=\frac{1}{3}(i+2j+2k)\wedge j=-\frac{2}{3}j+\frac{1}{3}k$, on a $\|j\wedge u\|^2=\frac{5}{9}$ et donc $\sin\theta=\frac{3}{5}$. L'égalité $r(j)=k$ fournit ensuite

\begin{center}
$k=(\cos\theta)j+(1-\cos\theta)\times\frac{2}{3}\times\frac{1}{3}(i+2j+2k)+\frac{3}{5}\times\frac{1}{3}(i+2j+2k)\wedge j$
\end{center}
En analysant la composante en $i$, on en déduit que $\frac{2}{9}(1-\cos\theta)-\frac{2}{5}=0$ et donc $\cos\theta=-\frac{4}{5}$. Ainsi, pour tout vecteur $v=(x,y,z)$ de $\Rr^3$, l'égalité $(*)$ s'écrit

\begin{align*}\ensuremath
r(v)&=-\frac{4}{5}(x,y,z)+\frac{9}{5}\times\frac{1}{3}\times\frac{1}{3}(x+2y+2z)(1,2,2)+\frac{3}{5}\times\frac{1}{3}
(2z-2y,2x-z,-2x+y)\\
 &=\frac{1}{5}(-4x+(x+2y+2z)+(2z-2y),-4y+2(x+2y+2z)+(2x-z),-4z+2(x+2y+2z)+(-2x+y))\\
 &=\frac{1}{5}(-3x+4z,4x+3z,5y)=\frac{1}{5}\left(
 \begin{array}{ccc}
 -3&0&4\\
 4&0&3\\
 0&5&0
 \end{array}
 \right)\left(
 \begin{array}{c}
 x\\
 y\\
 z
 \end{array}
 \right)
\end{align*}
La matrice cherchée est

\begin{center}
\shadowbox{
$\left(
 \begin{array}{ccc}
 -\frac{3}{5}&0&\frac{4}{5}\\
\rule[-4mm]{0mm}{10mm} \frac{4}{5}&0&\frac{3}{5}\\
 0&1&0
 \end{array}
 \right)$.
 }
 \end{center}
}
}
