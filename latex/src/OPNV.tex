\uuid{OPNV}
\exo7id{5968}
\auteur{tumpach}
\organisation{exo7}
\datecreate{2010-11-11}
\isIndication{false}
\isCorrection{true}
\chapitre{Espace L^p}
\sousChapitre{Espace Lp}

\contenu{
\texte{
Soit $\{f_{n}\}_{n\in\mathbb{N}}$ la suite de fonctions d\'efinies
par~:
$$
f_{n}(x) = \sqrt{n} \mathbf{1}_{[n, n+\frac{1}{n}]}(x).
$$
}
\begin{enumerate}
    \item \question{Montrer que $f_{n}$ converge faiblement vers $0$ dans
$L^{2}([0, +\infty[)$ mais ne converge pas fortement dans $L^2([0,
+\infty[)$.}
\reponse{Quelque soit $g$ continue \`a support compact,
$$
\int_{[0, +\infty[} f_{n}(x) g(x)\,dx = \sqrt{n}
\int_{n}^{n+\frac{1}{n}} g(x)\,dx \rightarrow 0
$$
quand $n\rightarrow +\infty$. Par densit\'e des fonctions
continues \`a support compact, $f_{n}$ converge faiblement vers
$0$. Comme $f_n$ converge presque partout vers $0$ on conclut
comme pr\'ec\'edemment que $f_n$ ne converge pas fortement vers
$0$ dans $L^2([0, +\infty[)$ car
$$
\|f_{n}\|_{2} = 1.
$$}
    \item \question{Montrer que $f_{n}$ converge fortement vers $0$ dans
$L^{p}([0, +\infty[)$ pour $p<2$.}
\reponse{Pour $p<2$, on a~:
$$
\int_{[0, +\infty[} |f_{n}(x)|\,dx = \int_{n}^{n+\frac{1}{n}}
n^{\frac{p}{2}}\,dx = n^{ \frac{p}{2}-1} \rightarrow 0,
$$
donc $f_{n}$ converge fortement vers $0$ dans $L^{p}([0,
+\infty[)$.}
\end{enumerate}
}
