\uuid{RP8Y}
\exo7id{185}
\auteur{bodin}
\organisation{exo7}
\datecreate{1998-09-01}
\isIndication{true}
\isCorrection{true}
\chapitre{Injection, surjection, bijection}
\sousChapitre{Application}

\contenu{
\texte{
Soient $f : \Rr \rightarrow \Rr$ et $g : \Rr \rightarrow \Rr$ telles que $f(x) = 3x+1$ et $g(x)=x^2-1$.
A-t-on $f\circ g=g\circ f$ ?
}
\indication{Prouver que l'\'egalit\'e est fausse.}
\reponse{
Si $f\circ g=g\circ f$ alors 
$$\forall x \in \Rr \ \ f\circ g (x) = g\circ f(x).$$
Nous allons montrer que c'est faux, en exhibant un contre-exemple.
Prenons $x=0$. Alors $f\circ g (0) = f(-1) = -2$, et
$g\circ f(0) = g(1) = 0$ donc $f\circ g (0) \not= g\circ f(0)$.
Ainsi $f\circ g \not= g\circ f$.
}
}
