\uuid{E1Mw}
\exo7id{2545}
\auteur{tahani}
\organisation{exo7}
\datecreate{2009-04-01}
\isIndication{false}
\isCorrection{true}
\chapitre{Difféomorphisme, théorème d'inversion locale et des fonctions implicites}
\sousChapitre{Difféomorphisme, théorème d'inversion locale et des fonctions implicites}

\contenu{
\texte{
Donner l'allure de $C=\{(x,y)
\in \mathbb{R}^2; x^4+y^3-y^2+x-y=0\}$ au voisinage des points
$(0,0)$ et $(1,1)$.
}
\reponse{
Posons $f(x,y)=x^4+y^3-x^2-y^2+x-y$, $f(0,0)=0$ et $f(1,1)=0$.
$\mathbb{R}$ est un espace de Banach et $f$ est de classe $C^1$
car polynomiale.
$$\frac{\partial f}{\partial y}=3y^2-2y-1$$
\'Etude au point $(0,0)$, $\frac{\partial f}{\partial y}(0,0)=-1$,
c'est un isomorphisme de $\mathbb{R}$. Nous sommes dans les
conditions d'application du th\'eor\`eme des fonctions implicites.
Il existe $I$ contenant $0$, $J$ contenant $0$ et $g: I
\rightarrow J$, $C^1$ tel que $g(0)=0$ et $f(x,g(x))=0, \forall x
\in I$. On a
$$x^4+(g(x))^3-x^2-(g(x))^2+x-g(x)=0$$
En d\'erivant on obtient:
$$4x^3+3g^2(x)g'(x)-2x-2g(x)g'(x)+1-g'(x)=0$$
d'o\`u $g'(0)=1$. On d\'erive encore:
$$12x^2+6g(x)g'(x)^2+3g^2(x)g''(x)-2-2g'(x)^2-2g(x)g''(x)-g''(x)=0$$
d'o\`u $$g''(0)=-4.$$ \'Etude au point $(1,1)$, $\frac{\partial
f}{\partial y}(1,1)=0$. Ce n'est plus un diff\'eo, on ne peut pas
appliquer le th\'eor\`eme des fonctions implicites. Dans ce cas,
on prend la d\'eriv\'ee par rapport \`a la premì\`ere variable.
$$\frac{\partial f}{\partial x}=4x^3-2x+1$$
et donc $\frac{\partial f}{\partial x}(1,1)=3$. Donc, d'apr\`es le
th\'eor\`eme des fonctions implicites, il existe $I$ contenant
$1$, $J$ contenant $1$ et $g:I \rightarrow J$ de classe $C^1$ tels
que $g(1)=1$ et $f(g(x),x)=0, \forall y \in I$. On a
$$g(y)^4-g^2(y)+g(y)+y^3-y^2-y=0$$
En d\'erivant
$$4g^3g'-2gg'+g'+3y^2-2y-1=0$$
d'o\`u $4g'(1)-g'(1)=0$ et donc $g'(1)=0$.
$$12g^2(g')^2+4g^3g''-2gg''-2(g')^2+g''+6y-2=0$$
d'o\`u $g''(1)=-4/3$.
}
}
