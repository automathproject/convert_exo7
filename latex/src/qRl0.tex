\uuid{qRl0}
\exo7id{6861}
\auteur{gijs}
\organisation{exo7}
\datecreate{2011-10-16}
\isIndication{false}
\isCorrection{false}
\chapitre{Autre}
\sousChapitre{Autre}

\contenu{
\texte{
Soit $c$ un point singulier essentiel d'une fonction
$f$ holomorphe dans un disque pointé $U =\{ z\in {\Cc}\vert\ 0<\vert
z-c\vert <\rho \}$. Le but de l'exercice est de démontrer que $f$ n'est
injective dans aucun voisinage pointé de $c$.
}
\begin{enumerate}
    \item \question{Montrer que pour tout $\gamma \in{\Cc}$ et tout $\varepsilon >0$, il
existe $z'\in U$ et $\varepsilon '>0$ tels que
$$\overline{D (f(z'),\varepsilon ')}\subset f(U) \cap 
D(\gamma ,\varepsilon ),$$
où $D(a,r)$ désigne le disque ouvert de centre $a$ et de rayon $r$.
On pourra utiliser le théorème de Casorati-Weierstrass, puis remarquer que
$f(U)$ est ouvert (la fonction $f$ est holomorphe donc ouverte).}
    \item \question{Pour $n\ge 1$, soit $U_n $ le disque pointé $\{ z\in {\Cc}\vert\ 0<\vert
z-c\vert <\rho /n \}$. Soient $\gamma _0\in{\Cc}$ et $\varepsilon
_0>0$. Construire par récurrence une suite strictement décroissante
$(\varepsilon _n)_{n\ge 1}$ de réels strictement positifs et une suite
$(z_n)_{n\ge 1}$, $z_n\in U_{n-1}$, vérifiant
\begin{eqnarray*}
\overline{D (f(z_1),\varepsilon_1 )}&\subset& f(U) \cap 
D(\gamma_0 ,\varepsilon _0),\\
\overline{D (f(z_{n+1}),\varepsilon_{n+1} )}&\subset& f(U_{n+1}) \cap 
D(f(z_n) ,\varepsilon _n)\ \ {\rm pour\ }n\ge 1.\\
\end{eqnarray*}
En déduire qu'il existe  $a\in D(\gamma _0,\varepsilon _0)$  et une suite
$(c_n)_{n\ge 0}$ de points de $U$ distincts deux à deux tels que
$$\lim_{n\to +\infty}c_n= c{\rm\ \ et\ \ }\forall n,\ \ f(c_n)=a.$$
Conclure.}
\end{enumerate}
}
