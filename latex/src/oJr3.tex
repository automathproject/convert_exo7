\uuid{oJr3}
\exo7id{4391}
\auteur{quercia}
\organisation{exo7}
\datecreate{2010-03-12}
\isIndication{false}
\isCorrection{true}
\chapitre{Intégration}
\sousChapitre{Intégrale multiple}

\contenu{
\texte{

}
\begin{enumerate}
    \item \question{Calculer $A = \iint_{0\le y\le x\le 1}^{} \frac{d xd y}{(1+x^2)(1+y^2)}$.}
    \item \question{Démontrer la convergence des intégrales :\par
    $B =  \int_{\theta=0}^{\pi/4} \frac{\ln(2\cos^2\theta)}{2\cos2\theta}  d\theta$,
    $C =  \int_{\theta=0}^{\pi/4} \frac{\ln(2\sin^2\theta)}{2\cos2\theta}  d\theta$,
    et $D =  \int_{t=0}^1 \frac{\ln t}{1-t^2}\,d t$.}
    \item \question{Démontrer que $A = B$ (passer en coordonnées polaires dans $A$).}
    \item \question{Calculer $B+C$ et $B-C$ en fonction de $D$.}
    \item \question{En déduire les valeurs de $C$ et $D$.}
\reponse{
$2A = \left( \int_{t=0}^1 \frac{d t}{1+t^2}\right)^2  \Rightarrow  A = \frac{\pi^2}{32}$.
$B+C = \frac D2 $, $B-C = -D$.
$C = -\frac{3\pi^2}{32}$, $D = -\frac{\pi^2}8$.
}
\end{enumerate}
}
