\uuid{S214}
\exo7id{1800}
\auteur{ridde}
\organisation{exo7}
\datecreate{1999-11-01}
\isIndication{true}
\isCorrection{true}
\chapitre{Fonction de plusieurs variables}
\sousChapitre{Différentiabilité}

\contenu{
\texte{
Soit la fonction $f \colon \R^{2} \longrightarrow \R$ d\'efinie par
\[
\begin{array}{lll}
f(x, y)= xy\frac{x^{2}-y^{2}}
{x^{2} + y^{2}}& \mathrm{ si } &(x, y) \neq (0,0),\\
 f(0, 0)= 0
\end{array}
\]
\'Etudier la continuit\'e de $f$. Montrer que $f$ est de classe $C^{1}$.
}
\indication{Il est \'evident que, en tout point $(x,y)$ distinct de l'origine,
la fonction $f$ est continue et que les d\'eriv\'ees partielles y
existent et sont continues. Il suffit de montrer que
$f$ est continue en $(0,0)$ et que les d\'eriv\'ees partielles y
existent et y sont continues.}
\reponse{
Puisque $\left|\frac{x^{2}-y^{2}}{x^{2} + y^{2}}\right|$ reste born\'e,
\[
\mathrm{lim}_{(x,y)\to (0,0)}f(x,y)= \mathrm{lim}_{(x,y)\to (0,0)}xy\frac{x^{2}-y^{2}}{x^{2} + y^{2}}
= \mathrm{lim}_{(x,y)\to (0,0)}xy =0
\]
d'o\`u $f$ est continue en $(0,0)$.
De m\^eme,
\begin{align*}
\mathrm{lim}_{(x,y)\to (0,0)}\frac{f(x,y)}x&= \mathrm{lim}_{(x,y)\to (0,0)}y\frac{x^{2}-y^{2}}{x^{2} + y^{2}}
= \mathrm{lim}_{(x,y)\to (0,0)}y =0
\\
\mathrm{lim}_{(x,y)\to (0,0)}\frac{f(x,y)}y&= \mathrm{lim}_{(x,y)\to (0,0)}x\frac{x^{2}-y^{2}}{x^{2} + y^{2}}
= \mathrm{lim}_{(x,y)\to (0,0)}x =0
\end{align*}
d'o\`u les d\'eriv\'ees partielles
$\frac{\partial f}{\partial x}(0,0)$ et $\frac{\partial f}{\partial y}(0,0)$
existent, et
$\frac{\partial f}{\partial x}(0,0)=0$ et $\frac{\partial f}{\partial y}(0,0)=0$.
En plus, en dehors de l'origine,
\begin{align*}
\frac{\partial f}{\partial x}&= \frac{f(x,y)}x + xy \frac{\partial}{\partial x}
\frac{x^{2}-y^{2}}{x^{2} + y^{2}} = \frac{f(x,y)}x +  4\frac{x^{2}y^{3}}{(x^{2} + y^{2})^2}
\\
\frac{\partial f}{\partial y}&= \frac{f(x,y)}y +  xy \frac{\partial}{\partial y}
\frac{x^{2}-y^{2}}{x^{2} + y^{2}}  = \frac{f(x,y)}x +  4\frac{x^{3}y^{2}}{(x^{2} + y^{2})^2} .
\end{align*}
Puisque
\[
\mathrm{lim}_{(x,y) \to (0,0)}\frac{x^{2}y^{3}}{(x^{2} + y^{2})^2}=0,\ 
\mathrm{lim}_{(x,y) \to (0,0)}\frac{x^{3}y^{2}}{(x^{2} + y^{2})^2}=0,
\]
il s'ensuit que
\[
\mathrm{lim}_{(u,v) \to (0,0)}\frac{\partial f}{\partial x}(u,v) =0,\ 
\mathrm{lim}_{(u,v) \to (0,0)}\frac{\partial f}{\partial y}(u,v) =0,
\]
d'o\`u les d\'eriv\'ees partielles $\frac{\partial f}{\partial x}$ et
$\frac{\partial f}{\partial y}$ sont continues en $(0,0)$.
}
}
