\uuid{jBiA}
\exo7id{6697}
\auteur{queffelec}
\organisation{exo7}
\datecreate{2011-10-16}
\isIndication{false}
\isCorrection{false}
\chapitre{Formule de Cauchy}
\sousChapitre{Formule de Cauchy}

\contenu{
\texte{
Soit $f$ une fonction entière ; on pose, pour $r\in {\Rr}_+^*$,
$$M(r)=\sup_{\vert z\vert =r}\vert f(z)\vert$$
}
\begin{enumerate}
    \item \question{On suppose qu'il existe $p\in {\Nn}$ tel que 
$$\lim_{r\to +\infty}{M(r)\over r^{p+1}}=0$$
Montrer qu'alors $f$ est un polynôme de degré au plus $p$.}
    \item \question{On suppose qu'il existe $R\ge 0$, $K>0$ et $p\in {\Nn}$ tels que 
$$\vert z\vert  >R\Longrightarrow \vert f(z)\vert \le K\vert z\vert ^p$$
Montrer qu'alors $f$ est un polynôme de degré au plus $p$. Montrer que si
de plus $R=0$ alors $f$ est un monôme de degré $p$.}
    \item \question{En déduire que si $f$ vérifie 
$$\forall z\in \C,\ \vert f'(z)\vert \le \vert z\vert$$
alors $f$ est de la forme $f(z)=a+bz^2$ avec $\vert b\vert \le 1/2$.}
\end{enumerate}
}
