\uuid{9HFP}
\exo7id{5378}
\auteur{rouget}
\organisation{exo7}
\datecreate{2010-07-06}
\isIndication{false}
\isCorrection{true}
\chapitre{Déterminant, système linéaire}
\sousChapitre{Système linéaire, rang}

\contenu{
\texte{
Résoudre le système~:~$x_1+x_2=0$, $x_{k-1}+x_k+x_{k+1}=0$ pour $k=2,...,n-1$, $x_{n-1}+x_n=0$.
}
\reponse{
Soit $D_n$ le déterminant du système pour $n\geq3$.

En développant ce déterminant suivant sa première colonne, on obtient la relation de récurrence~:
 
$$\forall n\geq5,\;D_n=D_{n-1}-D_{n-2},$$

ce qui fournit aisément par récurrence, en tenant compte de $D_3=D_4=-1$~:

$$\forall k\geq1,\;D_{3k}=D_{3k+1}=(-1)^k\;\mbox{et}\;D_{3k+2}=0.$$

Pour $n$ élément de $3\Nn^*\cup(1+3\Nn^*)$, le système est de \textsc{Cramer} et homogène et admet donc une et une seule solution à savoir la solution nulle.

Pour $n=3k+2$, puisque $D_n=0$ mais que le mineur de format $n-1$ constitué des $n-1$ premières lignes et colonnes est $D_{n-1}$ et est donc non nul, le système est homogène de rang $n-1$ et l'ensemble des solutions est un sous-espace vectoriel de $\Rr^n$ de dimension $1$. On trouve aisément $\mathcal{S}=\{\lambda(1,-1,0,1,-1,0...,1,-1),\;;\lambda\in\Rr\}$.
}
}
