\uuid{tuNk}
\exo7id{7006}
\auteur{megy}
\organisation{exo7}
\datecreate{2016-04-26}
\isIndication{false}
\isCorrection{true}
\chapitre{Nombres complexes}
\sousChapitre{Géométrie}

\contenu{
\texte{
Soit $ABC$ un triangle direct. Soient $P, Q, R$ tels que $CBP$, $ACQ$ et $BAR$ soient des triangles équilatéraux directs. On note $U, V, W$ les centres de gravité respectifs de ces trois triangles équilatéraux. Montrer que $UVW$ est équilatéral, de même centre de gravité que $ABC$, en utilisant la caractérisation des triangles équilatéraux.
}
\reponse{
%Cet exercice est semblable à celui sur le théorème de Van Aubel. 

Par construction, $A$ est l'image de $B$ par la rotation de centre $W$ et d'angle $2\pi/3$. En posant  $j=e^{2i\pi/3}$, on a donc 
\[ a-w = j(b-w), \]
autrement dit 

\[ w = \frac{a-jb}{1-j}.\]
On obtient de même $u=\frac{b-jc}{1-j}$ et $v=\frac{c-ja}{1-j}$.





On en déduit tout d'abord que 

\[ u+v+w = \frac{1}{1-j}\left(a-jb+b-jc+c-ja\right) = a+b+c. \]

Donc $\frac{1}{3} \left(u+v+w\right) = \frac{1}{3} \left(a+b+c\right)$, c'est-à-dire que les triangles $ABC$ et $UVW$ ont même centre de gravité. Ensuite, par la caractérisation des triangles équilatéraux, $UVW$ est équilatéral direct ssi $u+jv+j^2w=0$. Or:

\[ u+jv+j^2w = \frac{1}{1-j}\left(b-jc+j(c-ja)+j^2(c-ja)\right)=0.\]

Donc $UVW$ est bien équilatéral direct.
}
}
