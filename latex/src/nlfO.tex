\uuid{nlfO}
\exo7id{7692}
\auteur{mourougane}
\organisation{exo7}
\datecreate{2021-08-11}
\isIndication{false}
\isCorrection{true}
\chapitre{Sous-variété}
\sousChapitre{Sous-variété}

\contenu{
\texte{
Calculer une application de Weingarten du paraboloïde hyperbolique $S$ d'équation $z=x^2-y^2$
au point $p(0,0,0)$. En déduire sa courbure de Gauss et ses directions principales.
}
\reponse{
On paramètre la surface $S$ par
$F(u,v)=(u,v,u^2-v^2)$.
L'espace tangent en $p$ est engendré par $(\frac{\partial F}{\partial u})_p=(1,0,2u)_p=(1,0,0)$ et
$(\frac{\partial F}{\partial v})_p=(0,1,-2v)_p=(0,1,0)$.
La matrice de la première forme fondamentale dans cette base est en $p$ de paramètre $(0,0)$
$\begin{pmatrix}
 1+4u^2&-4uv\\-4uv&1+4v^2
\end{pmatrix}=Id$.
 La matrice de la seconde forme fondamentale se calcule à l'aide du vecteur normal 
$n=(\frac{\partial F}{\partial u}\times\frac{\partial F}{\partial v})_p=(0,0,1)$
par
$\begin{pmatrix}
 <\frac{\partial^2 F}{\partial u^2},n>&<\frac{\partial^2 F}{\partial u\partial v},n>\\
<\frac{\partial^2 F}{\partial u\partial v},n>&<\frac{\partial^2 F}{\partial v^2},n>
 \end{pmatrix}=
\begin{pmatrix}
 2&0\\0&-2
\end{pmatrix}$.
Comme la matrice de la première forme fondamentale est l'identité,
cette dernière matrice est aussi celle de l'endomorphisme de Weingarten.
La courbure de Gauss est donc le déterminant $-4$.
Les deux directions principales sont dirigées par les vecteurs
 $\frac{\partial F}{\partial u}_p=(1,0,0)$ et
$\frac{\partial F}{\partial v}_p=(0,1,0)$.
}
}
