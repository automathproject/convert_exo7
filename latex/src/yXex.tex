\uuid{yXex}
\exo7id{7054}
\auteur{megy}
\organisation{exo7}
\datecreate{2017-01-08}
\isIndication{true}
\isCorrection{true}
\chapitre{Géométrie affine euclidienne}
\sousChapitre{Géométrie affine euclidienne du plan}

\contenu{
\texte{
% tags triangle rectangle, cercle circonscrit
On donne deux cercles $\mathcal C$ et $\mathcal C'$ de rayons $r < r'$, de centres $O$ et $O'$, disjoints et extérieurs l'un à l'autre. On admet qu'il existe quatre tangentes communes  à $\mathcal C$ et $\mathcal C'$. L'objectif est de les construire.
}
\begin{enumerate}
    \item \question{(Analyse) Soit $\mathcal D$ une tangente commune. On note $A$ et $A'$ les points de contact de $\mathcal D$ avec les deux cercles. Que peut-on dire de la parallèle à $(AA')$ passant par $O$ et de son intersection avec $(O'A')$ ?}
\reponse{Les droites sont perpendiculaires.}
    \item \question{(Synthèse) En déduire une construction du point d'intersection de ces deux droites, puis ces deux droites et enfin de $\mathcal D$. Tracer les quatre tangentes communes de cette façon.}
\reponse{Pour les tangentes communes extérieures, le cercle de centre $O'$ et de rayon $r'-r$ intersecte le cercle de diamètre $[OO']$ en deux points $C$ et $D$. Les droites $(O'C)$ et $(O'D)$ coupent $\mathcal C'$ en deux points $A'$ et $B'$. Les tangentes extérieures sont les parallèles à $(OC)$ et $(OD)$ passant par $A'$ et $B'$. Pour les tangentes intérieures, utiliser le cercle de centre $O'$ et de rayon $r'+r$.}
\indication{\begin{enumerate}
\item À quelle distance de $O'$ se situe ce point d'intersection ?
\item Utiliser  le cercle de centre $O'$ et de rayon $r'-r$ pour une tangente \og extérieure\fg{} ou bien $r'+r$ pour une  tangente \og intérieure\fg, et un autre cercle.
\end{enumerate}}
\end{enumerate}
}
