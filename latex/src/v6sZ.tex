\uuid{v6sZ}
\exo7id{4256}
\auteur{quercia}
\organisation{exo7}
\datecreate{2010-03-12}
\isIndication{false}
\isCorrection{false}
\chapitre{Calcul d'intégrales}
\sousChapitre{Autre}

\contenu{
\texte{
Soit $f : {[a,b]} \to {\R^+}$ continue non identiquement nulle.
On pose $I_n =  \int_{t=a}^b f^n(t)\,d t$ et $u_n = \sqrt[n]{I_n}$.

Soit $M = \max\{ f(x)\text{ tel que } a \le x \le b \}$ et $c \in {[a,b]}$ tel que $f(c) = M$.
}
\begin{enumerate}
    \item \question{Comparer $M$ et $u_n$.}
    \item \question{En utilisant la continuité de $f$ en $c$, démontrer que :
      $\forall\ \varepsilon \in{]0,M[}$ il existe $\delta > 0$ tel que
      $I_n \ge \delta(M-\varepsilon)^n$.}
    \item \question{En déduire $\lim_{n\to\infty} u_n$.}
\end{enumerate}
}
