\uuid{G6dg}
\exo7id{5517}
\auteur{rouget}
\organisation{exo7}
\datecreate{2010-07-15}
\isIndication{false}
\isCorrection{true}
\chapitre{Géométrie affine dans le plan et dans l'espace}
\sousChapitre{Géométrie affine dans le plan et dans l'espace}

\contenu{
\texte{
Angle des plans $x+2y+2z=3$ et $x+y=0$.
}
\reponse{
Soient $(P)$ le plan d'équation $x+2y+2z=3$ et $(P')$ le plan d'équation $x+y=0$. L'angle entre $(P)$ et $(P')$ est l'angle entre les vecteurs normaux $\overrightarrow{n}(1,2,2)$ et $\overrightarrow{n'}(1,1,0)$ :

\begin{center}
$\left(\widehat{\overrightarrow{n},\overrightarrow{n'}}\right)=\Arccos\left(\frac{\overrightarrow{n}.\overrightarrow{n'}}{\|\overrightarrow{n}\|\|\overrightarrow{n'}\|}\right)=\Arccos\left(\frac{3}{3\sqrt{2}}\right)=\Arccos\left(\frac{1}{\sqrt{2}}\right)=\frac{\pi}{4}$.
\end{center}
}
}
