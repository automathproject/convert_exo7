\uuid{qkyu}
\exo7id{2691}
\auteur{matexo1}
\organisation{exo7}
\datecreate{2002-02-01}
\isIndication{false}
\isCorrection{false}
\chapitre{Courbes planes}
\sousChapitre{Autre}

\contenu{
\texte{
Un cercle de rayon $R$ roule sans glisser sur l'axe ${\rm O}x$ dans le sens des
$x$ croissants. Soit {\cal C} la courbe d{\'e}crite par le point M li{\'e} {\`a} la
circonf{\'e}rence qui, dans la position initiale, co\"{\i}ncide avec l'origine
O\,(cyclo\"{\i}de). Soit ${\rm M}(\theta)$ la position du point M quand le cercle
a tourn{\'e} d'un angle $\theta$ {\`a} partir de la position initiale
et $\Omega(\theta)$ le point de contact correspondant entre la circonf{\'e}rence et
l'axe ${\rm O}x$.
\begin{itemize}
\item D{\'e}terminer l'abscisse de $\Omega(\theta)$ et les coordonn{\'e}es
$x(\theta)$ et $y(\theta)$ du point ${\rm M}(\theta)$. Montrer que la courbe
{\cal C} est p{\'e}riodique et repr{\'e}senter graphiquement la premi{\`e}re
p{\'e}riode. 
\item D{\'e}terminer, en fonction de $\theta$, le vecteur tangent
$d\overrightarrow{\rm OM}/d\theta$, le vecteur tangent unitaire
$\overrightarrow{\rm T}$, et l'{\'e}l{\'e}ment de longueur $ds$. 
\item D{\'e}terminer
le vecteur normal unitaire $\overrightarrow{\rm N}$ et le rayon de courbure
$\rho$ au point param{\'e}tr{\'e} par $\theta$. Montrer que le centre de courbure est
situ{\'e} sur la droite d{\'e}finie par ${\rm M}(\theta)$ et $\Omega(\theta)$, et
pr{\'e}ciser sa position sur cette droite. 

\item(facultatif) L'angle de rotation
est d{\'e}fini en fonction du temps par la fonction $\theta(t)$. Calculer le
vecteur vitesse $\overrightarrow{v}$ du point M {\`a} l'instant $t$. Montrer que
$\overrightarrow{v}$ s'exprime en fonction de $\Omega{\rm M}$, $d\theta/dt$ et
$\overrightarrow{\rm T}$, et donner un vecteur $\overrightarrow{\omega}$
orthogonal au plan du mouvement tel que
$\overrightarrow{v}=\overrightarrow{\omega}\land \overrightarrow{\Omega {\rm
M}}$. Obtenir g{\'e}om{\'e}triquement {\`a} un instant donn{\'e} quelconque le vecteur
vitesse d'un point P quelconque de la circonf{\'e}rence.
\end{itemize}
}
}
