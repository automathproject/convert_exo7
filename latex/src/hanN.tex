\uuid{hanN}
\exo7id{807}
\auteur{cousquer}
\organisation{exo7}
\datecreate{2003-10-01}
\isIndication{false}
\isCorrection{true}
\chapitre{Calcul d'intégrales}
\sousChapitre{Longueur, aire, volume}

\contenu{
\texte{
On appelle \emph{tore} la figure obtenue par révolution d'un cercle
de rayon $r$ autour d'une droite de son plan passant à distance $R$ de son
centre (on suppose $r<R$). Calculer l'aire $A$ du tore, et son volume~$V$.
}
\reponse{
$A=4\pi^2Rr,\quad V=2\pi^2Rr^2$.
}
}
