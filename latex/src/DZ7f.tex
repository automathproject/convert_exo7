\uuid{DZ7f}
\exo7id{3476}
\auteur{quercia}
\organisation{exo7}
\datecreate{2010-03-10}
\isIndication{false}
\isCorrection{true}
\chapitre{Déterminant, système linéaire}
\sousChapitre{Système linéaire, rang}

\contenu{
\texte{
Soit $A \in \mathcal{M}_n(\R)$ et $E = \{ M \in \mathcal{M}_n(\R)$ tq $MA = 0 \}$.
    Quelle est la structure de $E$, sa dimension~?
}
\reponse{
$E$ est un sev et un idéal à gauche de $\mathcal{M}_n(\R)$.
	     Il est isomorphe à $\mathcal{L}({H,\R^n})$ où $H$ est un supplémentaire de
	     $\Im A$ dans $\R^n$. $\dim E = n(n-\mathrm{rg}(A))$.
}
}
