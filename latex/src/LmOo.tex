\uuid{LmOo}
\exo7id{6977}
\auteur{blanc-centi}
\organisation{exo7}
\datecreate{2014-05-06}
\isIndication{true}
\isCorrection{true}
\chapitre{Fonctions circulaires et hyperboliques inverses}
\sousChapitre{Fonctions hyperboliques et hyperboliques inverses}

\contenu{
\texte{
Soit $a$ et $b$ deux réels positifs tels que $a^2-b^2=1$. Résoudre le système
$$\left\{\begin{array}{l}
\ch(x)+\ch(y)=2a\\
\sh(x)+\sh(y)=2b
\end{array}\right.$$
}
\indication{Poser $X=e^x$ et $Y=e^y$ et se ramener à un système d'équations du type somme-produit.}
\reponse{
\begin{eqnarray*}
(S)\ \left\{\begin{array}{l}
\ch(x)+\ch(y)=2a\\
\sh(x)+\sh(y)=2b
\end{array}\right.
&\Longleftrightarrow&
\left\{\begin{array}{l}
e^x+e^{-x}+e^y+e^{-y}=4a\\
e^x-e^{-x}+e^y-e^{-y}=4b
\end{array}\right.\\
 &\Longleftrightarrow&
\left\{\begin{array}{l}
e^x+e^y=2a+2b\\
e^x-e^{-x}+e^y-e^{-y}=4b
\end{array}\right.\\
 &\Longleftrightarrow&
\left\{\begin{array}{l}
e^x+e^y=2a+2b\\
-e^{-x}-e^{-y}=2b-2a
\end{array}\right.\\
 &\Longleftrightarrow&
\left\{\begin{array}{l}
e^x+e^y=2(a+b)\\
\frac{1}{e^{x}}+\frac{1}{e^{y}}=2(a-b)
\end{array}\right.
\end{eqnarray*}
ce qui donne, en posant  $X=e^x$ et $Y=e^y$:
\begin{eqnarray*}
(S)&\Longleftrightarrow&
\left\{\begin{array}{l}
X+Y=2(a+b)\\
\frac{1}{X}+\frac{1}{Y}=2(a-b)
\end{array}\right.\\
&\Longleftrightarrow&
\left\{\begin{array}{l}
X+Y=2(a+b)\\
\frac{X+Y}{XY}=2(a-b)
\end{array}\right.\\
&\Longleftrightarrow&
\left\{\begin{array}{l}
X+Y=2(a+b)\\
\frac{2(a+b)}{XY}=2(a-b)
\end{array}\right.
\end{eqnarray*}
Or $a\not=b$ puisque par hypothèse, $a^2-b^2=1$. Ainsi,
\begin{eqnarray*}
(S)&\Longleftrightarrow&
\left\{\begin{array}{l}
X+Y=2(a+b)\\
XY=\frac{a+b}{a-b}
\end{array}\right.\\
&\Longleftrightarrow& X\ \text{et}\ Y\ \text{sont les solutions de}\ z^2-2(a+b)z+\frac{a+b}{a-b}=0
\end{eqnarray*}

\medskip

\emph{Remarque :} On rappelle
que si $X, Y$ vérifient le système 
$\left\{\begin{array}{l}
X+Y=S\\
XY=P
\end{array}\right.$, alors $X$ et $Y$ sont les solutions de l'équation $z^2-Sz+P=0$.
\medskip

Or le discriminant du trinôme $z^2-2(a+b)z+\frac{a+b}{a-b}=0$ vaut 
$$\Delta=4(a+b)^2-4\frac{a+b}{a-b}=4(a+b)\left(a+b-\frac{1}{a-b}\right)=\frac{4(a+b)(a^2-b^2-1)}{a-b}=0$$ 
Il y a donc une racine double qui vaut $\frac{2(a+b)}{2}$, ainsi $X=Y=a+b$ et donc :
$$(S)\Longleftrightarrow e^x=e^y=a+b$$
On vérifie que $a+b\ge 0$ (car $a\ge 0$ et $b\ge 0$) et $a+b\not=0$ (car $a^2-b^2=1$).
Conclusion : le système $(S)$ admet une unique solution, donnée par $\big(x=\ln(a+b),y=\ln(a+b)\big)$.
}
}
