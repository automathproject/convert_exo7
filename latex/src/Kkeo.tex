\uuid{Kkeo}
\exo7id{2175}
\auteur{debes}
\organisation{exo7}
\datecreate{2008-02-12}
\isIndication{false}
\isCorrection{true}
\chapitre{Action de groupe}
\sousChapitre{Action de groupe}

\contenu{
\texte{
\label{ex:deb75}
(a) Montrer que tout $3$-cycle est un carr\'e. En d\'eduire
que le groupe altern\'e $A_n$ est engendr\'e par les carr\'es de
permutations.
\smallskip

(b) Montrer que $A_n$ est le seul sous-groupe de $S_n$ d'indice $2$.
}
\reponse{
(a) Un $3$-cycle $\omega$ est d'ordre $3$ et v\'erifie donc $\omega^3=1$ soit encore $\omega=
(\omega^2)^2$. Le groupe engendr\'e par tous les carr\'es de permutations dans $S_n$ contient
donc tous les $3$-cycles, et donc aussi le groupe qu'ils engendrent, c'est-\`a-dire $A_n$.
L'autre inclusion est facile puisque le carr\'e d'une permutation est toujours une 
permutation paire. 
\smallskip

(b) Si $H$ est un sous-groupe d'indice $2$ de $S_n$, il est distingu\'e. On a alors
$\sigma^2 \in H$ pour tout $\sigma \in S_n$ (cf exercice \ref{ex:le18}). D'apr\`es la question
(a), $H=A_n$.
}
}
