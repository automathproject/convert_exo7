\uuid{YWDL}
\exo7id{5932}
\auteur{tumpach}
\organisation{exo7}
\datecreate{2010-11-11}
\isIndication{false}
\isCorrection{true}
\chapitre{Autre}
\sousChapitre{Autre}

\contenu{
\texte{
\emph{Cet exercice fournit une autre m\'ethode de calcul du volume
de la boule unit\'e $\mathcal{B}_{n}$ de $\mathbb{R}^{n}$ et de
l'aire de la sph\`ere $\mathcal{S}_{n-1} \subset \mathbb{R}^{n}$.}
On conserve les notations de l'exercice pr\'ec\'edent.
}
\begin{enumerate}
    \item \question{Montrer que $\mathcal{V}_{n} = I_{n} \cdot
\mathcal{V}_{n-1}$, o\`u $I_{n} = \int_{0}^{\pi} \left(\sin \theta
\right)^{n} \,d\theta.$}
\reponse{On a
\begin{eqnarray*}
\mathcal{V}_n &= &\int_{\mathcal{B}_n} dx_1\dots dx_{n}=
\int_{-1}^{1} dx_1 \int_{\sum_{i=2}^n x_i^2 \leq 1 - x_1^2} dx_2
\dots dx_n\\
& = & \mathcal{V}_{n-1}\int_{-1}^{1}  \left(\sqrt{1-
x_1^2}\right)^{n-1} \,dx_1
\end{eqnarray*}
Posons $x_1=\cos \theta$, pour $\theta \in [0, \pi]$. Alors
$\sqrt{1-x_1^2}=|\sin \theta| = \sin \theta$ et $dx_1 =-\sin\theta
\;d\theta.$ On a donc
$$ \mathcal{V}_n = -\mathcal{V}_{n-1}\int\limits_\pi^0 (\sin \theta)^n d\theta
=\mathcal{V}_{n-1}\int\limits_0^\pi (\sin \theta)^n d\theta = I_n
\cdot \mathcal{V}_{n-1}.$$}
    \item \question{V\'erifier que $I_{n} =
\frac{n-1}{n} I_{n-2}.$}
\reponse{On a
\begin{eqnarray*}
I_n & = &\int\limits_0^\pi (\sin \theta)^n d\theta=\int\limits_0^\pi (\sin \theta)^{n-1}\; \sin \theta\;d\theta= \\
    & = &\left[-\cos\theta (\sin \theta)^{n-1} \right]_0^\pi + (n-1)\int\limits_0^\pi (\sin \theta)^{n-2}\; (\cos \theta)^2\;d\theta= \\
    & = &(n-1) \int\limits_0^\pi (\sin \theta)^{n-2} (1-(\sin \theta)^2) \;d\theta =(n-1)(I_{n-2}-I_n).\\
\end{eqnarray*}
Donc $I_n=\frac{n-1}{n}\cdot I_{n-2}\;\;$.}
    \item \question{Calculer $\mathcal{V}_{n}$ pour $n =
 1, 2, \dots, 7$.}
\reponse{On a $I_0=\pi,\;I_1 = 2.$ Donc
$I_2=\frac{\pi}{2}, \;
I_3=\frac{4}{3},\;I_4=\frac{3\pi}{8},\;I_5=\frac{16}{15},\;I_6=
\frac{15\pi}{48},\;I_7=\frac{32}{35}.$ 

Comme $\mathcal{V}_1=2$ on trouve:
$$\mathcal{V}_2=\pi,\;\mathcal{V}_3=\frac{4\pi}{3},\;\mathcal{V}_4=\frac{\pi^2}{2},\;
\mathcal{V}_5=\frac{8\pi^2}{15},\;\mathcal{V}_6=\frac{\pi^3}{6},\;\mathcal{V}_7=\frac{16}{105}\pi^3
.$$}
    \item \question{Calculer $\mathcal{A}_{n-1}$ pour $n = 
1, 2, \dots, 6$.}
\reponse{On a $\displaystyle\mathcal{V}_n =\int\limits_0^1
\int\limits_{\mathcal{S}_{n-1}}
  r^{n-1}dr d\sigma=\frac{1}{n}\mathcal{A}_{n-1},$ \,
d'o\`{u} \, $\mathcal{A}_{n-1}=n\,\mathcal{V}_n$. Donc on a
  $$\mathcal{A}_1=2\pi,\;\mathcal{A}_2=4\pi,\;\mathcal{A}_3=2\pi^2,\;
  \mathcal{A}_4=\frac{8}{3}\pi^2,\;\mathcal{A}_5=\pi^3,\;\mathcal{A}_6=\frac{16}{15}\pi^3
  .$$}
\end{enumerate}
}
