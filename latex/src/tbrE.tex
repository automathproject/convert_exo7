\uuid{tbrE}
\exo7id{5160}
\auteur{rouget}
\organisation{exo7}
\datecreate{2010-06-30}
\isIndication{false}
\isCorrection{true}
\chapitre{Propriétés de R}
\sousChapitre{Autre}

\contenu{
\texte{
Soit $n\in\Nn^*$ et $(x_1,x_2,...,x_n )\in[-1,1]^n$ tels que $x_1+x_2+...+x_n=0$.

Montrer que $|x_1+2x_2+...+nx_n|\leq E(\frac{n^2}{4})$.
}
\reponse{
Soient $n\in\Nn^*$ et $(x_1,x_2,...,x_n )\in[-1,1]^n$ tels que $x_1+x_2+...+x_n=0$.

On écrit

$$(x_1+2x_2+...+nx_n)=(x_1+x_2+...+x_n)+(x_2+x_3+....+x_n)+(x_3+...+x_n)+...+(x_{n-1}+x_n)+x_n,$$

avec $x_1+...+x_n=0$ et donc $x_2+...+x_n=-x_1$ ...

\begin{itemize}
\item[\textbf{1er cas.}] Si $n=2p$ est pair, alors $\frac{n^2}{4}=p^2$ et
donc, $E(\frac{n^2}{4})=p^2=\frac{n^2}{4}$. Dans ce cas, on peut écrire

\begin{align*}
|x_1+2x_2+...+nx_n|&\leq|x_1+x_2+...+x_{2p}|+|x_2+....+x_{2p}|+...+|x_p+...+x_{2p}|\\
 &\;+|x_{p+1}+...+x_{2p}|...+|x_{2p-1}+
x_{2p}|+|x_{2p}|\\
 &=0+|-x_1|+|-x_1-x_2|+...+|-x_{1}+...-x_{p-1}|\\
 &+|x_{p+1}+...+x_{2p}|...+|x_{2p-1}+x_{2p}|+|x_{2p}|\\
 &\leq0+1+2+...+(p-1)+p+(p-1)+...+1=2\frac{p(p-1)}{2}+p=p^2=E(\frac{n^2}{4})
\end{align*}

\item[\textbf{2ème cas.}] Si $n=2p+1$ est impair, alors $\frac{n^2}{4}=p^2+p+\frac{1}{4}$ et
donc, $E(\frac{n^2}{4})=p^2+p=\frac{n^2-1}{4}$. Dans ce cas, on peut écrire

\begin{align*}
|x_1+2x_2+...+nx_n|&\leq|x_1+x_2+...+x_{2p+1}|+...+|x_{p+1}+...+x_{2p+1}|\\
 &+|x_{p+2}+...+x_{2p+1}|...+|x_{2p+1}|\\
 &=0+|-x_1|+|-x_1-x_2|+...+|-x_{1}+...-x_{p}|\\
 &+|x_{p+2}+...+x_{2p+1}|...+|x_{2p+1}|\\
 &\leq0+1+2+...+(p-1)+p+p+(p-1)+...+1=2\frac{p(p+1)}{2}=p^2+p=E(\frac{n^2}{4})
\end{align*}

\end{itemize}

Dans tous les cas, on a montré que
\begin{center}
\shadowbox{
$\forall n\in\Nn^*,\;|x_1+2x_2+...+nx_n|\leq E(\frac{n^2}{4}).$
}
\end{center}
}
}
