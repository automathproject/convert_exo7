\uuid{jxrV}
\exo7id{7001}
\auteur{blanc-centi}
\organisation{exo7}
\datecreate{2015-07-04}
\isIndication{true}
\isCorrection{true}
\chapitre{Equation différentielle}
\sousChapitre{Résolution d'équation différentielle du premier ordre}

\contenu{
\texte{
\
}
\begin{enumerate}
    \item \question{\textbf{\'Equation de Bernoulli}
  \begin{enumerate}}
\reponse{\textbf{\'Equation de Bernoulli}
  \begin{enumerate}}
    \item \question{Montrer que l'équation de Bernoulli
    $$y'+a(x)y+b(x)y^n = 0 \qquad n \in \Zz \quad n \neq 0, n\neq 1$$
    se ramène à une équation linéaire par le changement de fonction 
    $z(x) = 1/ y(x)^{n-1}$.}
\reponse{On suppose qu'une solution $y$ ne s'annule pas.
    On divise l'équation $y'+a(x)y+b(x)y^n = 0$ par $y^n$, ce qui donne
    $$\frac{y'}{y^n} +a(x) \frac{1}{y^{n-1}} + b(x) = 0.$$
    On pose $z(x) = \frac{1}{y^{n-1}}$ et donc $z'(x) = (1-n)\frac{y'}{y^{n}}$.
    L'équation de Bernoulli devient une équation différentielle linéaire :
    $$\tfrac{1}{1-n} z' + a(x) z +b(x) = 0$$}
    \item \question{Trouver les solutions de l'équation $xy'+y-xy^3 = 0$.}
\reponse{\'Equation $xy'+y-xy^3 = 0$.
    
    Cherchons les solutions $y$ qui ne s'annulent pas. On peut alors diviser par $y^3$ pour obtenir :
    $$x \frac{y'}{y^3} + \frac{1}{y^2} - x=0$$
    On pose $z(x) = \frac{1}{y^2(x)}$, et donc $z'(x) = -2\frac{y'(x)}{y(x)^3}$.
    L'équation différentielle s'exprime alors $\frac{-1}{2} xz'+z -x = 0$, c'est-à-dire :
    $$xz'-2z = -2x.$$
    Les solutions  sur $\Rr$ de cette dernière équation sont les 
    $$z(x) =\begin{cases} \lambda_+ x^2+2x\ \text{si}\ x\ge 0 \cr\lambda_- x^2+2x\ 
    \text{si}\ x< 0 \end{cases} ,\quad\lambda_+,\lambda_- \in \Rr$$
    
    Comme on a posé $z(x) = \frac{1}{y^2(x)}$, on se retreint à un intervalle $I$ sur lequel $z(x)>0$: nécessairement $0\notin I$, donc on considère $z(x)=\lambda x^2+2x$, qui est strictement positif sur $I_\lambda$ où
$$I_\lambda=\begin{cases}
]0;+\infty[\quad\text{si}\ \lambda=0\\
\left]0;-\frac{2}{\lambda}\right[\quad\text{si}\ \lambda<0\\
\left]-\infty;-\frac{2}{\lambda}\right[\ \mathrm{ou}\ \left]0;+\infty\right[\quad\text{si}\ \lambda>0              
            \end{cases}
$$
On a $(y(x))^2 = \frac{1}{z(x)}$ pour tout $x\in I_\lambda$ et donc  $y(x) = \epsilon(x)\frac{1}{\sqrt{z(x)}}$, 
où $\epsilon(x)=\pm 1$. Or $y$ est continue sur l'intervalle $I_\lambda$, et 
ne s'annule pas par hypothèse: d'après le théorème des valeurs intermédiaires, $y$ 
ne peut pas prendre à la fois des valeurs strictement positives et des valeurs 
strictement négatives, donc $\epsilon(x)$ est soit constant égal à $1$, soit constant égal à $-1$.
    Ainsi les solutions cherchées sont les :
    $$y(x) = \frac{1}{\sqrt{\lambda x^2 + 2x}}\ \mathrm{ou}\ y(x) = \frac{-1}{\sqrt{\lambda x^2 + 2x}}\quad\mathrm{sur}\ I_\lambda\qquad (\lambda\in \Rr)$$
    
    Noter que la solution nulle est aussi solution.}
\indication{\begin{enumerate}
  \item
  \begin{enumerate}
    \item Se ramener à $\frac{1}{1-n}z'+a(x)z+b(x)=0$.
    \item $y = \pm\frac{1}{\sqrt{\lambda x^2 + 2x}}$ ou $y=0$.
  \end{enumerate}
    \item 
  \begin{enumerate}
    \item Remplacer $y$ par $u+y_0$.
    \item $y=\frac1x+\frac1{x\ln|x|+\lambda x}$ ou $y = \frac1x$.
  \end{enumerate}
\end{enumerate}}
\end{enumerate}
}
