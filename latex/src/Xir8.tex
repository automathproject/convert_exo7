\uuid{Xir8}
\exo7id{4952}
\auteur{quercia}
\organisation{exo7}
\datecreate{2010-03-17}
\isIndication{true}
\isCorrection{true}
\chapitre{Géométrie affine euclidienne}
\sousChapitre{Géométrie affine euclidienne du plan}

\contenu{
\texte{
Pour $\lambda \in \R$ on considère la droite $D_\lambda$ d'équation
cartésienne : $(1-\lambda^2)x + 2\lambda y = 4\lambda + 2$.

Montrer qu'il existe un point $M_0$ équidistant de toutes les droites
$D_\lambda$.
}
\indication{La distance d'un point $M_0(x_0,y_0)$ à une droite $D$ d'équation $ax+by+c=0$ est donnée par la formule
$d(M_0,D) = \frac{|ax_0+by_0+c_0|}{\sqrt{a^2+b^2}}$.}
\reponse{
Nous savons que la distance d'un point $M_0(x_0,y_0)$ à une droite $D$ d'équation $ax+by+c=0$ est donnée par la formule
$d(M_0,D) = \frac{|ax_0+by_0+c_0|}{\sqrt{a^2+b^2}}$.

Pour une droite $D_\lambda$ la formule donne :
$d(M_0,D_\lambda)  = \frac{|(1-\lambda^2)x_0 + 2\lambda y_0 - (4\lambda + 2)|}{\sqrt{(1-\lambda^2)^2+4\lambda^2}}$.

\bigskip
\textbf{Analyse.}

On cherche un point $M_0=(x_0,y_0)$ tel que pour tout $\lambda$, $d(M_0,D_\lambda)=k$ où $k\in \Rr$ est une constante.

L'égalité $d(M_0,D_\lambda)^2=k^2$ conduit à 
$$\Big((1-\lambda^2)x_0 + 2\lambda y_0 - (4\lambda + 2)\Big)^2 
= k^2 \Big( (1-\lambda^2)^2+4\lambda^2 \Big)$$
pour tout $\lambda\in \Rr$.
Nos inconnues sont $x_0,y_0,k$.
On regarde l'égalité comme une égalité de deux polynômes en la variable $\lambda$.

Pour ne pas avoir à tout développer on raffine un peu :
on identifie les termes de plus haut degré en $\lambda^4$ :
$x_0^2 \lambda^4=k^2\lambda^4$ donc $x_0^2=k^2$.

En évaluant l'égalité pour $\lambda=0$ cela donne $(x_0-2)^2=k^2$.
On en déduit $(x_0-2)^2=x_0^2$ dont la seule solution est $x_0=1$.
Ainsi $k=1$ (car $k>0$).

L'égalité pour $\lambda=+1$ donne $(2y_0-6)^2=4k^2$
et pour $\lambda=-1$ donne $(-2y_0+2)^2=4k^2$.
La seule solution est  $y_0=2$. 

\bigskip
\textbf{Synthèse.}
Vérifions que le point de coordonnées $M_0=(1,2)$ est situé à une distance $k=1$ de
toutes les droites $D_\lambda$.

Pour $(x_0,y_0)=(1,2)$, on trouve :
$d(M_0,D_\lambda)  = \frac{|(1-\lambda^2) + 4\lambda - (4\lambda + 2)|}{\sqrt{(1-\lambda^2)^2+4\lambda^2}}
= \frac{|\lambda^2+1|}{\sqrt{(\lambda^2+1)^2}} =  \frac{|\lambda^2+1|}{|\lambda^2+1|}=1$.
Donc $M_0=(1,2)$ est bien équidistant de toutes les droites $D_\lambda$.
}
}
