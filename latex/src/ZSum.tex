\uuid{ZSum}
\exo7id{2639}
\auteur{debievre}
\organisation{exo7}
\datecreate{2009-05-19}
\isIndication{true}
\isCorrection{true}
\chapitre{Fonction de plusieurs variables}
\sousChapitre{Différentielle seconde}

\contenu{
\texte{
Soit $f\colon \R^2\setminus\{(0,0)\}\longrightarrow \R$ 
une fonction
de classe $C^2$ et soient $r$ et $\theta$ les coordonn\'ees
polaires standard dans le plan de telle sorte que
l'association
\[
]0,+\infty[\times  ]0,2\pi[\longrightarrow  \R^2\setminus\{(0,0)\},
\quad
(r,\theta)\longmapsto (x,y)=(r\cos\theta, r\sin\theta),
\]
soit un changement de variables.
Soit $F$ la fonction d\'efinie par 
\[
F(r,\theta)=f(r\cos\theta, r\sin\theta).
\]
C'est \lq\lq l'expression de $f$ 
en coordonn\'ees polaires\rq\rq. 
Montrer que
\begin{equation}
\frac{\partial^2 f}{\partial x^2}(x,y)+\frac{\partial^2 f}{\partial y^2}(x,y)=
\frac{\partial^2F}{\partial r^2}(r,\theta)+\frac{1}{r}\frac{\partial F}{\partial r}(r,\theta)+\frac{1}{r^2}\frac{\partial^2 F}{\partial \theta^2}(r,\theta).
\label{laplace}
\end{equation}
Cette formule
calcule ``le Laplacien en coordonn\'ees polaires.'' 
L'exercice ne d\'epend pas de la connaissance
du Laplacien cependant.
}
\indication{\begin{enumerate}
 \item  Montrer que
$$\frac{\partial^2F}{\partial r^2}+\frac{1}{r}\frac{\partial F}{\partial r}=
\frac1r\frac{\partial}{\partial r} r\frac{\partial}{\partial r}F.$$
 \item Montrer que
$$r\frac{\partial F}{\partial r}=x\frac{\partial f}{\partial x}+y\frac{\partial f}{\partial y}.$$
 \item Montrer que
$$\frac{\partial F}{\partial \theta}=x\frac{\partial f}{\partial y}-y\frac{\partial f}{\partial x}$$
 \item Utiliser ces r\'esultats, puis calculer encore un peu pour obtenir le r\'esultat souhait\'e. 
\end{enumerate}}
\reponse{
$\frac{\partial }{\partial r}\left (r \frac{\partial F}{\partial r}\right)=
\frac{\partial F}{\partial r} +r \frac{\partial^2 F}{\partial r^2}$
$\frac{\partial F}{\partial r}=
\frac{\partial f}{\partial x}\frac xr+ \frac{\partial f}{\partial y}\frac yr$
$\frac{\partial F}{\partial \theta}=
\frac{\partial f}{\partial x}\frac{\partial x}{\partial \theta}
+ \frac{\partial f}{\partial y}\frac{\partial y}{\partial \theta}
= -y \frac{\partial f}{\partial x}+x\frac{\partial f}{\partial y}
$
En prenant la somme des trois \'equations suivantes
\begin{align*}
r^2 \frac{\partial^2 F}{\partial r^2}&=
x^2 \frac{\partial^2 f}{\partial x^2}
+2xy \frac{\partial^2 f}{\partial x \partial y}
+
y^2 \frac{\partial^2 f}{\partial y^2}
\\
r\frac{\partial F}{\partial r}&=
x\frac{\partial f}{\partial x}+ y\frac{\partial f}{\partial y}
\\
\frac{\partial^2 F}{\partial \theta^2}&=
x^2 \frac{\partial^2 f}{\partial y^2}
-2xy \frac{\partial^2 f}{\partial x \partial y}
+
y^2 \frac{\partial^2 f}{\partial x^2}
-
x \frac{\partial f}{\partial x}
-
y\frac{\partial f}{\partial y}
\end{align*}
on trouve le r\'esultat cherch\'e.
}
}
