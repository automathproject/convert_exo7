\uuid{2KV6}
\exo7id{6207}
\auteur{queffelec}
\organisation{exo7}
\datecreate{2011-10-16}
\isIndication{false}
\isCorrection{true}
\chapitre{Application linéaire bornée}
\sousChapitre{Application linéaire bornée}

\contenu{
\texte{
Soit $K$ un compact convexe d'un evn $E$. Soit $u$ une application linéaire
continue de $E$ dans $E$ telle que $u(K)\subset K$. On va montrer que $u$ a un
point fixe dans $K$.
}
\begin{enumerate}
    \item \question{On peut supposer que $0\notin K$. Pour chaque $n\geq1$, on désigne par $S_n$
l'application définie sur $E$ par 

$S_n(x)={1\over n}(x+u(x)+\cdots+u^{n-1}(x))$.
Montrer que $S_n(K)\subset K$.}
\reponse{Remarquons tout d'abord que si $0\in K$, l'application $u$ (linéaire) admet
$0$ comme point fixe dans $K$.  

On suppose donc $0\notin K$.
Si $x\in K$, $u^j(x)\in K$ pout tout $0\leq j\leq n-1$ et $S_n(x)\in K$ comme
combinaison convexe de ces $n$ points de $K$.}
    \item \question{Montrer que pour tous entiers $n_1,n_2,\ldots,n_k$ en nombre fini,

$S_{n_1}\circ\cdots\circ S_{n_k}(K)\subset S_{n_1}(K)\cap
S_{n_2}(K)\cap\cdots\cap S_{n_k}(K)$.

En déduire que $A=\cap_{n\geq1} S_{n}(K)$ est non vide.}
\reponse{Soit $x\in K$. Par 1., $S_{n_1}\circ\cdots\circ S_{n_k}(x)\in K$ et s'écrit
$S_{n_1}(S_{n_2}\circ\cdots\circ S_{n_k})(x)$ : il appartient donc à
$S_{n_1}(K)$. Par ailleurs, comme les applications linéaires $S_{n_j}$ commutent
entre elles (ce sont des polyn\^omes en $u$), il en va de même pour
$S_{n_2},\cdots, S_{n_k}$ d'où l'inclusion. 

Chaque $S_n$ étant continue et $K$ compact, les ensembles $S_{n}(K)$ sont eux
aussi compacts et inclus dans $K$; si $A=\cap_{n\geq1} S_{n}(K)$ était
vide, par la propriété de l'intersection finie, on pourrait trouver des
entiers $n_1,n_2,\ldots,n_k$ en nombre fini tels que $S_{n_1}(K)\cap
S_{n_2}(K)\cap\cdots\cap S_{n_k}(K)=\emptyset$. Or cette intersection contient
l'image de $K$ par $S_{n_1}\circ\cdots\circ S_{n_k}$ et ne peut être vide.}
    \item \question{Montrer que tout $x\in A$ est  point fixe de $u$.}
\reponse{Soit $a\in A$; pour tout $n$ il existe $x_n\in K$ tel que $a=S_n(x_n)$.
On va montrer que  $u(a)=a$ :

$$\begin{array}{ccc}
         u(a)-a&=& u(S_n(x_n))-S_n(x_n)\\
&&\\

        &=&\displaystyle{(n+1)S_{n+1}(x_n)\over n}-{x_n\over n}-S_n(x_n)
\\
&&\\

&=&\displaystyle{(n+1)S_{n+1}(x_n)-nS_n(x_n)-x_n\over n}\\
&&\\
&=&\displaystyle{u^n(x_n)-x_n\over n}
\end{array}$$

Mais $||u^n(x_n)-x_n||\leq \hbox{diam} K=d<+\infty$ puisque $K$ compact est
borné, et 
$||u(a)-a||\leq {d\over n}$ pour tout $n$ est donc nul.}
\end{enumerate}
}
