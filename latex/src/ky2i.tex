\uuid{ky2i}
\exo7id{6263}
\auteur{queffelec}
\organisation{exo7}
\datecreate{2011-10-16}
\isIndication{false}
\isCorrection{false}
\chapitre{Différentiabilité, calcul de différentielles}
\sousChapitre{Différentiabilité, calcul de différentielles}

\contenu{
\texte{
Soit $\Omega$ un ouvert convexe de $\Rr^n$ et $f:\Omega \to \Rr^n$ une
application de classe $C^1$ qui est injective sur $\Omega$ et
telle que $Df(x)$ soit injective pour tout $x\in \Omega$.
}
\begin{enumerate}
    \item \question{Montrer que, pour tous $a,b \in \Omega$,
$$ \| f(b) -f(a)- Df(a) (b-a)\| \leq \| b -a \| \sup _{c\in [a,b]} \| Df(c) -Df(a)\| \; .$$}
    \item \question{Soit $(f_n)$ une suite de fonctions de classe $C^1$ telle que $f_n \to f$ et $Df_n \to Df$
uniformément sur tout compact de $\Omega$. On va montrer:
{\it pour tout compact $K$ de $\Omega$ il existe $n_0$ tel que $f_n$ soit injective
sur $K$ pour $n\geq n_0$.} 
\begin{itemize}}
    \item \question{En raisonnant par l'absurde, montrer qu'il existerait $K$ compact et, pour une infinité d'entiers
$n$, des points $a_n, b_n \in K$ tels que $f_n(a_n) = f_n (b_n)$.}
    \item \question{Quitte à extraire, montrer qu'alors $b_n - a_n \to 0$.}
    \item \question{Utiliser (1.) pour en déduire une contradiction.
\end{itemize}}
\end{enumerate}
}
