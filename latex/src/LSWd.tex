\uuid{LSWd}
\exo7id{2683}
\auteur{matexo1}
\organisation{exo7}
\datecreate{2002-02-01}
\isIndication{false}
\isCorrection{true}
\chapitre{Suite et série de fonctions}
\sousChapitre{Convergence simple, uniforme, normale}

\contenu{
\texte{
A
}
\begin{enumerate}
    \item \question{Soient $a$ et $z$ deux réels. Soit $f$ une fonction de
classe $C^{n+1}$ sur le segment d'extrémités
$a$ et $z$ et $\phi$ un polynôme de degré $n$. Prouver que pour tout $t$ compris dans
l'intervalle $[0,1]$,
\[
{d\over dt}\sum_{m=1}^n (-1)^m (z-a)^m
                \phi^{(n-m)}(t) f^{(m)}\left(a+t(z-a)\right)
\]\[
  = -(z-a) \phi^{(n)}(t) f'\left(a+t(z-a)\right)
    + (-1)^n (z-a)^{n+1} \phi(t) f^{(n+1)}\left(a+t(z-a)\right) 
\]}
    \item \question{\begin{enumerate}}
    \item \question{Montrer que la fonction $t\mapsto {t \over e^t-1}$ est prolongeable par continuité
en zéro, que son prolongement est indéfiniment dérivable et admet des développements
limités en zéro de la forme:
$$ 1 - {t/2} + {b_1 t^2 \over 2!}
                                + {b_2 t^4 \over 4!} + \dots + {b_n t^{2n} \over (2n)!}
+o(t^{2n+1}),$$
où les $b_i$ sont des réels qu'on ne cherchera pas à déterminer.

Montrer que la dérivée $n^{i\grave eme}$ en zéro, notée $\phi_n(z)$, de la fonction
$t\mapsto t {e^{zt}-1 \over e^t-1}$
est un polynôme en $z$ de degré $n$ et que
$$\phi_n(z)= z^n - {1\over 2}nz^{n-1} + C_n^2 b_1 z^{n-2}
                + C_n^4 b_2 z^{n-4} + \dots
                + C_n^{2N} b_N z^{n-2N}$$
où $N=E({n-1\over2})$, $E$ désignant la fonction partie entière.}
    \item \question{Prouver que $nz^{n-1}=\phi_n(z+1) - \phi_n(z)$}
\reponse{
\vfill
{\leftskip=-1cm A\qquad  1.\par} 
$$\begin{aligned}
\qquad &{d\over dt}\sum_{m=1}^n (-1)^m (z-a)^m
                \phi^{(n-m)}(t) f^{(m)}\left(a+t(z-a)\right) \cr
        &=\sum_{m=1}^n (-1)^m (z-a)^m \Big[
                \phi^{(n-m+1)}(t)f^{(m)}\left(a+t(z-a)\right)  \cr
        &  \mkern260mu   +\phi^{(n-m)}(t) (z-a) f^{(m+1)}\left(a+t(z-a)\right)  \Big]  \cr
        &=\sum_{m=1}^n (-1)^m 
                (z-a)^m \phi^{(n-m+1)}(t)f^{(m)}\left(a+t(z-a)\right)   \cr
        &  \mkern220mu  + \sum_{m=1}^n (-1)^m(z-a)^{m+1} \phi^{(n-m)}(t)f^{(m+1)}\left(a+t(z-a)\right)   \cr
\end{aligned}
$$
Effectuons le changement d'indice de sommation $m=m'+1$ dans la deuxième somme; tous les
termes s'éliminent deux à deux, à l'exception du premier terme de la première somme
et du dernier terme de la deuxième somme, d'où le résultat demandé. 

\medskip
2.a. Plus généralement, on a le résultat suivant: si une fonction $f$ est nulle
en zéro et de classe $C^{n+1}$ sur un intervalle $I$ contenant zéro, $g(t)=f(t)/t$ est
prolongeable par continuité en zéro et son prolongement est de classe $C^n$.
Gardons nous de déduire fallacieusement ce résultat de l'existence d'un développement
limité d'ordre $n$ de $g$.
On montre à l'aide d'un développement limité d'ordre 1 de $f$ que $g$ est
prolongeable par continuité en $0$ en posant $\tilde g(0)=f'(0)$. Par ailleurs, $g$ est de
classe $C^{n+1}$ sur $I\backslash\{0\}$. 
Faisons l'hypothèse de
récurrence $g^{(k-1)}(t)={f^{(k)}(0) \over k} + {f^{(k+1)}(0) \over k+1} {t \over 1!}
+ \dots + {f^{(n+1)}(0) \over n+1} {t^{n-k+1} \over (n-k+1)!}+o(t^{n-k+1})$
$(1\leq k \leq n,\ t\neq0)$.
En dérivant $k$ fois l'identité $f(t)=tg(t)$, on obtient que
$\forall t\neq0 \ \ g^{(k)}(t)={ f^{(k)}(t) - k g^{(k-1)}(t) \over t }$.
Or $f^{(k)}(t)=f^{(k)}(0)+f^{(k+1)}(0){t\over1!}+\dots
+f^{(n+1)}(0){t^{n+1-k}\over(n-k)!}+o(t^{n+1-k})$.
En substituant ces développements limités dans l'identité précédente, on montre que
l'hypothèse de récurrence est vérifiée au rang suivant, donc pour tout entier
$k$ de 1 à $n+1$.
 Faisons l'hypothèse de récurrence que $g^{(k)}(0)$ existe
et est égale à $f^{(k+1)}(0)/(k+1)$ ($0\leq k\leq n$). Du développement limité
de $g^{(k)}$ (tronqué à l'ordre 1) et de l'hypothèse de
récurrence, il résulte que $g^{(k)}$ est continue et dérivable en zéro et que, si
$k<n$, $g^{(k+1)}(0)=f^{(k+2)}(0)/(k+2)$, ce qui prouve par récurrence que $g$ est
$n$ fois continument dérivable sur $I$.

Par conséquent, $t\mapsto{e^t-1 \over t}$ est prolongeable par continuité en une
fonction indéfiniment dérivable sur $\Rr$. Cette fonction ne s'annulant jamais,
son inverse est également définie et indéfiniment dérivable sur $\Rr$.



La fonction $t \mapsto {t \over e^t-1} + {t\over2}$ est paire car
$$\begin{aligned}
{-t \over e^{-t}-1} + {-t\over2}
&= {-t e^t \over 1-e^t} - {t\over2}  \cr
&= {t e^t - t + t \over e^t-1} - {t\over2}  \cr
&= {t \over e^t-1} + {t\over2}  \cr
\end{aligned}
$$
Donc  le développement limité de ${t \over e^t-1}$, dont l'existence est garantie par
le fait que la fonction est indéfiniment dérivable, est de la forme demandée par
l'énoncé.

Par conséquent
\begin{eqnarray*}
{t \over e^t-1} (e^{zt}-1) &=&
\Big( 1-{t\over2}+ {b_1 t^2 \over 2!} + {b_2 t^4 \over 4!} + \dots 
+  {b_N t^{2N} \over (2N)!} + o(t^n) \Big) 
\\
\times \Big( zt + {z^2 t^2 \over 2!} + \dots + {z^n t^n \over n!} + o(t^n)   \Big)   
 \end{eqnarray*}
et $\phi_n(z)/n!$ est le coefficient de $t^n$ dans ce développement:
$$\phi_n(z)/n!= {z^n\over n!} - {1\over2} {z^{n-1}\over(n-1)!}
+ {b_1\over2!}{z^{n-2}\over(n-2)!} + {b_2\over4!}{z^{n-4}\over(n-4)!}
+ \dots + {b_N\over(2N)!}{z^{n-2N}\over(n-2N)!},$$
d'où l'expression de $\phi_n$ demandée.

\medskip
2.b. 
$$\begin{aligned}
\phi_n(z+1)-\phi_n(z)&={d^n \over dt^n}_{| t=0}
                      \left(t{e^{zt}-1 \over e^t-1} - t{e^{(z+1)t}-1 \over e^t-1} \right) \cr
                     &={d^n \over dt^n}_{| t=0} \left(t e^{zt} \right) \cr
\end{aligned}
$$
Comme $t \mapsto t e^{zt}$ est de classe $C^\infty$ et que son développement limité
d'ordre $n$ en $t=0$ est $\sum_{k=0}^{n-1} {z^k\over k!} t^{k+1} + o(t^n)$, il vient
$\phi_n(z+1)-\phi_n(z)=nz^{n-1}$.


\medskip
3. $(i)$ est obtenue en dérivant autant de fois que nécessaire l'identité précédente
et en donnant à $z$ la valeur zéro.
   $(ii)$, $(iii)$, $(iv)$, $(v)$ et $(vi)$ sont des conséquences immédiates de 2.a.

4.a. On applique la question 1 au polynôme $\phi_{2n}$ de degré 2n et on
intègre entre $0$ et $1$. Il vient
\[
\sum_{m=1}^{2n} (-1)^m (z-a)^m \Big[ \phi_{2n}^{(2n-m)}(1) f^{(m)}(z)
- \phi_{2n}^{(2n-m)}(0) f^{(m)}(a)  \Big]
\]\[
 = -(2n)! \bigl( f(z)-f(a) \bigr) + (z-a)^{2n+1} \int_0^1 \phi_{2n}(t)f\Big( a+(z-a)t \Big)dt
\]
en tenant compte du fait que $\phi_{2n}^{(2n)}=(2n)!$

On obtient l'égalité demandée en substituant aux  dérivées itérées
de $\phi_{2n}$ les expressions déterminées dans la question 3.      

\medskip
4.b. Appliquons la question précédente en remplaçant $f$ par une
primitive de $F$ et $z$ par $\omega$. Il vient
\begin{eqnarray*}
0 &=& \int_a^{a+\omega} F(t)dt - {\omega\over2} (F(a+\omega)+F(a)) 
+ \sum_{m=1}^{n-1} b_m {(z-a)^{2m} \over (2m)!} \left[ F^{(2m-1)}(a+\omega) - F^{(2m-1)}(a) \right] 
\\
&& - {\omega^{2n+1}\over (2n)!} \int_0^1 \phi_{2n}(t)F^{(2n)}\Big( a+(z-a)t \Big)dt  
 \end{eqnarray*}
Lorsqu'on somme les égalités obtenues en remplaçant $a$ successivement
par lui-m{\^e}me, $a+\omega$, \dots, $a+(r-1)\omega$, on obtient le résultat demandé,
certains termes se simplifiant deux à deux.

\bigskip
{\leftskip=-1cm B\qquad 1. On a pour tout $x>0$ fixé \par}
$$u_k(x)=\ln \left( 1+{x\over k} \right)
                        + x \ln \left( 1-{k\over k+1} \right)
                        ={x\over k}-{x\over k+1}+O({1\over k^2})
                        =O({1\over k^2})$$
Donc la série $\sum_{k\geq1}u_k(x)$ converge.

\medskip
2. 
$$\begin{aligned}
\ln(x+1)+\sum_{k=1}^n &u_k(x+1)  \cr
        &=\ln(x+1) + \sum_{k=2}^{n+1} \ln(x+k)
                        + \sum_{k=1}^n \Big[ -\ln(k)
                                + (x+1) \big(\ln(k)-\ln(k+1)\big) \Big] \cr
        &=\sum_{k=1}^{n+1} \ln(x+k)                                
                        + \sum_{k=1}^n  x (\ln(k)-\ln(k+1)) 
                        - \sum_{k=1}^n \ln(k+1)        \cr
        &=\sum_{k=1}^n u_k(x) + \ln(x+n+1) - \ln(n+1)  \cr
\end{aligned}
$$
Lorsque $n$ tend vers $+\infty$,
$\ln(x+n+1) - \ln(n+1)=\ln \left( {x+n+1 \over n+1} \right)$ tend vers zéro;
on obtient donc par passage à la limite l'égalité souhaitée.

\medskip
3. Pour tout $k\geq1$ entier, $u_k(1)=0$, donc $G(1)=0$ et
on prouve aisément par récurrence à l'aide de la question précédente
que pour tout entier strictement positif $n$, $G(n+1)=\sum_{k=1}^n \ln(k)$,
égalité de laquelle on déduit immédiatement le résultat demandé.

\medskip
4. immédiat

\medskip
5. C'est une application directe de la question A.4.b.\par
$T_{p,n}(x,y)=-{1\over(2p)!}\int_0^1 \phi_{2p}(t)
                        \sum_{m=0}^{n-1} f^{(2p)}(m+t) dt
=-{1\over(2p)!} \int_0^n \phi_{(2p)} \big( t-E(t) \big) f^{(2p)}(t)dt$


\medskip
6. L'intégrande dans l'expression de $T_{p,n}(x,y)$ est majorée en valeur
absolue par le produit de la borne supérieure de la fonction continue
$\phi_{2p}$ sur le segment $[0,1]$ et de la valeur absolue de $f^{(2p)}$.

On prouve aisément par récurrence que
$f^{(m)}(t)=(-1)^{m-1}\left({1\over(y+t)^m}-{1\over(x+t)^m}\right)
= O \left( {1\over t^{m+1}} \right)$ (quand $t \to +\infty$).
Donc l'intégrale $\int_0^{+\infty} \phi_{(2p)} \big( t-E(t) \big) f^{(2p)}(t)dt$
est absolument convergente, ce qui prouve que $T_{p,n}(x,y)$ admet une limite
finie lorsque $n$ tend vers $+\infty$.



\medskip
7. D'après les questions 4 et 5,
$$\begin{aligned}
G(y)-G(x)&=\sum_{k=1}^{+\infty}\bigl[ \ln(y+k) - \ln(x+k)
                        + (y-x)\ln\left({k\over k+1}\right) \bigr] + \ln y - \ln x \cr
        &=\lim_{n\to +\infty}\Bigl\{\sum_{k=0}^n [\ln(y+k) - \ln(x+k)]
                 + (y-x)\ln{1\over n+1} \Bigr\}                          \cr 
        &\mkern-30mu =\lim_{n\to +\infty} \Biggl\{
                 (y+n)\ln(y+n) - (y+n) - y\ln y + y - (x+n)\ln(x+n)             \cr
        &\mkern40mu  + (x+n) + x\ln x - x
         + {1\over2} \bigl( \ln y - \ln x + \ln(y+n) - \ln(x+n) \bigr)          \cr
        &\mkern10mu +\sum_{h=1}^{p-1} {b_h\over(2h)!}
                \big( f^{(2h-1)}(n) - {1\over y^{2h-1}} + {1\over x^{2h-1}}\big) 
         + T_{p,n}(x,y) + (y-x)\ln{1\over n+1} \Biggr\}                  \cr
\end{aligned}
$$

Or $(y+n)\ln(y+n) - (x+n)\ln(x+n) = (y-x)\ln n + y - x + o(1)$
quand $n\to +\infty$. On obtient donc après simplifications
$G(y)-G(x)=g(x)-g(y)+R_n(x,y)$, ce qu'il fallait démontrer.

\medskip
8. D'après la question 5 et l'expression des dérivées successives de $f$ donnée dans
la question 6, $T_{p,n}(x,y)$ est majoré en valeur absolue par le produit d'une constante
et de l'intégrale
$\int_0^n \left| {1\over (y+t)^{2p}} - {1\over (x+t)^{2p}} \right| dt$.
$|R_p(x,y)|$ est majoré de la m{\^e}me façon en remplaçant la borne finale d'intégration
$n$ par $+\infty$. L'argument de la valeur absolue gardant un signe constant, l'intégrale
majorante est égale à
${1\over 2p-1} \left[ \ \left| {1\over (y+t)^{2p-1}}
                                - {1\over (x+t)^{2p-1}} \right| \ \right]_0^{+\infty}$
et on obtient ainsi l'estimée souhaitée.


\medskip
9. On a $g(m)=m\ln m - m - {1\over2}\ln m + o(1)$ et $G(m)=-\ln(m-1)!=-\ln m! + \ln m$ pour
$m$ entier, donc
le résultat demandé découle immédiatement de la formule de Stirling
$m! \sim \sqrt{2\pi} m^{m+{1\over2}} e^{-m}$.

\medskip
10. Le résultat demandé est obtenu à partir de l'égalité de la question 7 par passage à la
limite. On fait tendre $x$ vers $+\infty$ par valeurs entières et on tient compte de l'estimée
obtenue dans la question 8.

\medskip
11. En calculant les premiers termes du développement limité de la question A.2.a,
on trouve $b_1=1/6,\ b_2=-1/30,\ b_3=1/42$.
Des questions 3 et 10, il résulte que
$$\begin{aligned}
\ln(m!) &= -G(m) + \ln m        \cr
        &= m\ln m - m + {1\over2}\ln m + {1\over2}\ln2\pi
        + {1\over12} {1\over m} - {1\over360} {1\over m^3} + {1\over1260} {1\over m^5}
        + O({1\over m^7})  \cr
\end{aligned}
$$
}
\end{enumerate}
}
