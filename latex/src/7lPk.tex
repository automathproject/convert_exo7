\uuid{7lPk}
\exo7id{2479}
\auteur{matexo1}
\organisation{exo7}
\datecreate{2002-02-01}
\isIndication{false}
\isCorrection{true}
\chapitre{Réduction d'endomorphisme, polynôme annulateur}
\sousChapitre{Valeur propre, vecteur propre}

\contenu{
\texte{
On considère la matrice $N\times N$
\[                          
 M=\begin{pmatrix}
b & c & 0   &.. & 0 & 0\cr 
a & b & c & ..& 0 & 0\cr
0   & a & b & ..& 0 & 0\cr
..  &..   &..   &.. &.. & ..\cr
0   & 0   &0    &.. &b&c  \cr
0   & 0   &0    &.. &a    &b  \cr
\end{pmatrix}                       
\]     
où $a,b$ et $c$ sont trois nombres complexes, avec $c\neq 0$.            
On note $V$ un vecteur propre associé à la valeur propre $\lambda$ de $M$.\\
Ecrire les relations reliant les composantes de $V$.\\ 
Déterminer toutes les valeurs propres de $M$.
}
\reponse{
On a une suite récurrente à trois termes reliant les composantes  $v_i$ du vecteur
propre. On calcule le terme général de la suite en résolvant le 
polynôme caractéristique. Les deux constantes sont identifiées en 
écrivant que $v_0=v_{n+1}=0$. On trouve $n+1$ valeurs propres distinctes:
\[
\lambda_k=b+2c\left(\frac{a}{c}\right)^{1/2}cos\left(\frac{2 k \pi}{n+1}\right)
\mbox{~~~pour~~~}k=1,...n
\]
avec le vecteur propre $v^k$ associé, de composantes
\[
v_j^k=\left(\frac{a}{c}\right)^{j/2} sin\left(\frac{2 k j \pi}{n+1}\right)
\mbox{~~~pour~~~}j=1,...n
\]
}
}
