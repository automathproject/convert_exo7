\uuid{aJER}
\exo7id{1202}
\auteur{ridde}
\organisation{exo7}
\datecreate{1999-11-01}
\isIndication{false}
\isCorrection{true}
\chapitre{Suite}
\sousChapitre{Suite récurrente linéaire}

\contenu{
\texte{
Soit $ (u_n)$ d\'efinie par $u_0$ et $u_1$
strictement positifs et $u_{n + 1}
 = u_n + u_{n-1}$ pour $n \geq 1$.
}
\begin{enumerate}
    \item \question{Montrer que $\lim (\dfrac{u_{n + 1}}{u_n})$ existe et la d\'eterminer. Que
remarquez-vous ?}
    \item \question{Soit $a_n = \dfrac{u_{n + 1}}{u_n}$. Exprimer $a_{n + 1}$ en fonction de $a_n$.}
    \item \question{Montrer que $a_{2n}$ et $a_{2n + 1}$ sont adjacentes.}
    \item \question{D\'eterminer un rationnel $r$ tel que $\left|r-\frac{1 + \sqrt 5}2\right|< 10^{-3}$.}
\reponse{
L'équation caractéristique est :
$$r^2-r-1=0$$
dont les solution sont $\lambda = \frac{1-\sqrt5}{2}$ et $\mu =
\frac{1+\sqrt5}{2}$. Donc $u_n$ est de la forme
$$u_n = \alpha \lambda^n + \beta\mu^n$$
pour $\alpha, \beta$ des réels que nous allons calculer gr\^ace à
$u_0$ et $u_1$. En effet $u_0 = 1 = \alpha \lambda^0 + \beta\mu^0$
donc $\alpha+\beta = 1$. Et comme $u_1 =1= \alpha \lambda^1 +
\beta\mu^1$ nous obtenons $\alpha \frac{1-\sqrt5}{2} + \beta
\frac{1+\sqrt5}{2}=1$. En résolvant ces deux équations nous
obtenons $\alpha = \frac{1}{\sqrt5}\frac{\sqrt5-1}{2}=
\frac{1}{\sqrt5}(-\lambda)$ et $\beta =
\frac{1}{\sqrt5}\frac{1+\sqrt5}{2}= \frac{1}{\sqrt5}(\mu)$. Nous
écrivons donc pour finir :
$$u_n = \frac{1}{\sqrt5}\big( \mu^{n+1}-\lambda^{n+1}\big).$$
}
\end{enumerate}
}
