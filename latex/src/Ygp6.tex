\uuid{Ygp6}
\exo7id{5164}
\auteur{rouget}
\organisation{exo7}
\datecreate{2010-06-30}
\isIndication{false}
\isCorrection{true}
\chapitre{Espace vectoriel}
\sousChapitre{Définition, sous-espace}

\contenu{
\texte{
Soit $E$ le $\Rr$-espace vectoriel des applications de $[0,1]$ dans $\Rr$ (muni de $f+g$ et $\lambda.f$ usuels)
(ne pas hésiter à redémontrer que $E$ est un $\Rr$ espace vectoriel). Soit $F$ l'ensemble des applications de $[0,1]$
dans $\Rr$ vérifiant l'une des conditions suivantes~:

$$\begin{array}{llll}
1)\;f(0)+f(1)=0&2)\;f(0)=0&3)\;f(\frac{1}{2})=\frac{1}{4}&4)\;\forall x\in[0,1],\;f(x)+f(1-x)=0\\
5)\;\forall x\in[0,1],\;f(x)\geq0&6)\;2f(0)=f(1)+3
\end{array}$$

Dans quel cas $F$ est-il un sous-espace vectoriel de $E$~?
}
\reponse{
La fonction nulle est dans $F$ et en particulier, $F\neq\varnothing$.
Soient alors $(f,g)\in F^2$ et $(\lambda,\mu)\in\Rr^2$.

$$(\lambda f+\mu g)(0)+(\lambda f+\mu g)(1)=\lambda(f(0)+f(1))+\mu(g(0)+g(1))=0.$$

Par suite, $\lambda f+\mu g$ est dans $F$. On a montré que~:

$$F\neq\varnothing\;\mbox{et}\;\forall(f,g)\in F^2,\;\forall(\lambda,\mu)\in\Rr^2,\;\lambda f+\mu g\in F.$$

$F$ est donc un sous-espace vectoriel de $E$.
Même démarche et même conclusion .
$F$ ne contient pas la fonction nulle et n'est donc pas un sous-espace vectoriel de $E$.
La fonction nulle est dans $F$ et en particulier, $F\neq\varnothing$.
Soient alors $(f,g)\in F^2$ et $(\lambda,\mu)\in\Rr^2$.

Pour $x$ élément de $[0,1]$,
$$(\lambda f+\mu g)(x)+(\lambda f+\mu g)(1-x)= \lambda(f(x)+f(1-x))+\mu(g(x)+g(1-x))=0$$ et $\lambda f+\mu g$ est dans
$F$. $F$ est un sous-espace vectoriel de $E$.

\textbf{Remarque.} Les graphes des fonctions considérés sont symétriques par rapport au point $(\frac{1}{2},0)$.
$F$ contient la fonction constante $1$ mais pas son opposé la fonction constante $-1$ et n'est donc pas un
sous-espace vectoriel de $E$.
$F$ ne contient pas la fonction nulle et n'est donc pas un sous-espace vectoriel de $E$.
}
}
