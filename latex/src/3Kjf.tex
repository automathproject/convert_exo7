\uuid{3Kjf}
\exo7id{3905}
\auteur{quercia}
\organisation{exo7}
\datecreate{2010-03-11}
\isIndication{false}
\isCorrection{true}
\chapitre{Continuité, limite et étude de fonctions réelles}
\sousChapitre{Etude de fonctions}

\contenu{
\texte{
Résoudre les équations suivantes : 

% ---------------------------------------------- (–6+–2)cos‚ + (–6-–2)sin‚ = 2
}
\begin{enumerate}
    \item \question{$(\sqrt6 + \sqrt2)\cos\theta + (\sqrt6 - \sqrt2)\sin\theta = 2$.

% ------------------------------------------- sin‚ + sin2‚ + sin3‚ + sin4‚ = 0}
\reponse{$4\cos\left(\theta-\frac\pi{12}\right) = 2 \iff
               \theta \equiv \pm\frac\pi3 + \frac\pi{12} (\mathrm{mod}\, {2\pi})$.}
    \item \question{$\sin\theta + \sin2\theta + \sin3\theta + \sin4\theta = 0$.

% ---------------------------------------------------- cos‚ + cos2‚ + cos3‚ = 0}
\reponse{$\sin\theta + \dots + \sin4\theta
                  = 2\sin\theta\cos\theta(4\cos^2\theta + 2\cos\theta - 1)
                  = 4\sin( 5\theta/2 )\cos\theta\cos( \theta/2 )$

               $4\cos^2\theta + 2\cos\theta - 1 = 0 \iff
                  \cos\theta = \frac {\sqrt5 -1}4 = \cos( 2\pi/5 )
                  \text{ ou }
                  \cos\theta = -\frac {\sqrt5 +1}4 = \sin( 2\pi/5 )$

               $ \Rightarrow $ modulo $2\pi$,
               $\theta \in \{ 0, \pi, \pi/2, 3\pi/2, 2\pi/5, 4\pi/5,
                                   6\pi/5, 8\pi/5 \}$.}
    \item \question{$\cos\theta + \cos2\theta + \cos3\theta = 0$.

% -------------------------------------------------------- cos‚ - cos2‚ = sin3‚}
\reponse{$\cos\theta \in \left\{ -\frac 12, \pm\frac 1{\sqrt2} \right\} \iff
               \theta \equiv \pm\frac {2\pi}3 (\mathrm{mod}\, {2\pi})
               \text{ ou } \theta \equiv \frac \pi4 (\mathrm{mod}\, {\frac\pi2})$.}
    \item \question{$\cos\theta - \cos2\theta = \sin3\theta$.

% -------------------------------------------------------- cos‚ + cos7‚ = cos4‚}
\reponse{$2\sin(3\theta/2)\sin(\theta/2) = 2\sin(3\theta/2)\cos(3\theta/2)
             \iff
             \theta \equiv 0 (\mathrm{mod}\, {\frac {2\pi}3}) \text{ ou }
             \theta \equiv \frac \pi4 (\mathrm{mod}\, \pi) \text{ ou }
             \theta \equiv \frac {3\pi}2 (\mathrm{mod}\, {2\pi})$.}
    \item \question{$\cos\theta + \cos7\theta = \cos4\theta$.

% ----------------------------------------------------- cos2‚ + cos12‚ = –3cos5‚}
\reponse{$2\cos4\theta\cos3\theta = \cos4\theta \iff
             \theta \equiv \frac \pi8 (\mathrm{mod}\, {\frac \pi4}) \text{ ou }
             \theta \equiv \pm\frac \pi9 (\mathrm{mod}\, {\frac{2\pi}3})$.}
    \item \question{$\cos2\theta + \cos12\theta = \sqrt3\cos5\theta$.

% -------------------------------------------------------- sin7‚ - sin‚ = sin3‚}
\reponse{$2\cos7\theta\cos5\theta = \sqrt3\cos5\theta \iff
             \theta \equiv \frac \pi{10} (\mathrm{mod}\, {\frac \pi5}) \text{ ou }
             \theta \equiv \pm\frac \pi{42} (\mathrm{mod}\, {\frac{2\pi}7})$.}
    \item \question{$\sin7\theta - \sin\theta = \sin3\theta$.

% ----------------------------------------------- cos‚ sin3‚ + cos3‚sin‚  = 3/4}
\reponse{$\theta\equiv 0\mathrm{mod}\, {\frac\pi3} \text{ ou } \theta\equiv\pm\frac \pi{12} (\mathrm{mod}\, {\frac \pi2})$.}
    \item \question{$\cos^3\theta\sin3\theta + \cos3\theta\sin^3\theta = \frac 34$.

% ------------------------------------------------------ sin‚sin3‚ = sin5‚sin7‚}
\reponse{$\cos^3\theta\sin3\theta + \cos3\theta\sin^3\theta
                = \frac 34\sin4\theta
                 \Rightarrow  \theta \equiv \frac \pi8 (\mathrm{mod}\, {\frac \pi2})$.}
    \item \question{$\sin\theta\sin3\theta = \sin5\theta\sin7\theta$.

% --------------------------------------------------------------- 3tan‚ = 2cos‚}
\reponse{$\theta\equiv 0 (\mathrm{mod}\, {\frac \pi8})$.}
    \item \question{$3\tan\theta = 2\cos\theta$.

% --------------------------------------------------------------- tan4‚ = 4tan‚}
\reponse{$\sin\theta = \frac 12 \Leftrightarrow
           \theta\equiv \frac\pi6 (\mathrm{mod}\, {2\pi})$ ou
          $\theta\equiv \frac{5\pi}6 (\mathrm{mod}\, {2\pi})$.}
    \item \question{$\tan4\theta = 4\tan\theta$.

% ------------------------------------------------- cotan‚ - tan‚ = cos‚ + sin‚}
\reponse{$\theta \equiv 0,\pm\arctan\sqrt5 (\mathrm{mod}\,\pi)$.}
    \item \question{$\mathrm{cotan}\theta - \tan\theta = \cos\theta + \sin\theta$.

% ----------------------------------------- tan(x) + tan(y) = 1, tan(x+y) = 4/3}
\reponse{$\cos^2\theta - \sin^2\theta
           = \cos\theta\sin\theta(\cos\theta+\sin\theta)$.

          $\cos\theta + \sin\theta = 0 \iff
           \theta\equiv -\frac\pi4 (\mathrm{mod}\, \pi)$.

          $\cos\theta-\sin\theta = \cos\theta\sin\theta  \Rightarrow 
           (\cos\theta\sin\theta)^2 + 2\cos\theta\sin\theta = 1  \Rightarrow 
           \begin{cases}\cos\theta = \frac {\sqrt{2\sqrt2-1} + \sqrt2 - 1}2 \cr
        
                  \sin\theta = \frac {\sqrt{2\sqrt2-1} - \sqrt2 + 1}2.\end{cases}$

          Les valeurs trouvées conviennent.}
    \item \question{$\begin{cases}\tan x + \tan y = 1\cr \tan(x+y) = 4/3.\end{cases}$}
\reponse{$\tan x = \tan y = \frac 12$.}
\end{enumerate}
}
