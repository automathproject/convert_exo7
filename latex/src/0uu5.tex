\uuid{0uu5}
\exo7id{6013}
\auteur{quinio}
\organisation{exo7}
\datecreate{2011-05-20}
\isIndication{false}
\isCorrection{true}
\chapitre{Probabilité discrète}
\sousChapitre{Variable aléatoire discrète}

\contenu{
\texte{
Une population comporte en moyenne une personne mesurant plus de 1m90 sur
80 personnes.
Sur 100 personnes, calculer la probabilité qu'il y ait au moins une
personne mesurant plus de 1.90m (utiliser une loi de Poisson).
Sur 300 personnes, calculer la probabilité qu'il y ait au moins une
personne mesurant plus de 1.90m.
}
\reponse{
Le nombre $X$ de personnes mesurant plus de 1.90m parmi 100 obéit à
une loi de Poisson de paramètre $\frac{100}{80}$.

La probabilité qu'il y ait au moins une personne mesurant plus de 1.90m
est donc $1-P[X=0]=1-e^{-\frac{100}{80}}=1-e^{-\frac{5}{4}}=0.713\,50$.

Sur 300 personnes:
la probabilité qu'il y ait au moins une personne mesurant plus de 1.90m
est donc $1-P[Y=0]=1-e^{-\frac{300}{80}}=0.976\,48$.
}
}
