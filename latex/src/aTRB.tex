\uuid{aTRB}
\exo7id{5325}
\auteur{rouget}
\organisation{exo7}
\datecreate{2010-07-04}
\isIndication{false}
\isCorrection{true}
\chapitre{Polynôme, fraction rationnelle}
\sousChapitre{Racine, décomposition en facteurs irréductibles}

\contenu{
\texte{
Trouver tous les polynômes divisibles par leur dérivée.
}
\reponse{
On suppose que $n=\mbox{deg}P\geq1$.

On pose $P=\lambda(X-z_1)(X-z_2)...(X-z_n)$ où $\lambda$ est un complexe non nul et les $z_k$ sont des complexes pas nécessairement deux à deux distincts.

D'après l'exercice précédent, $\frac{P'}{P}=\sum_{k=1}^{n}\frac{1}{X-z_k}$.

Si $P$ est divisible par $P'$, $\exists(a,b)\in\Cc^2\setminus\{(0,0)\}/\;P=(aX+b)P'$ et donc $\exists(a,b)\in\Cc^2\setminus\{(0,0)\}/\;\frac{P'}{P}=\frac{1}{aX+b}$ ce qui montre que la fraction rationelle $\frac{P'}{P}$ a exactement un et un seul pôle complexe et donc que les $z_k$ sont confondus.

En résumé, si $P'$ divise $P$, $\exists(a,\lambda)\in\Cc^2/\;P=\lambda(X-a)^n$ et $\lambda\neq0$.

Réciproquement, si $P=\lambda(X-a)^n$ avec $\lambda\neq0$, alors $P'=n\lambda(X-a)^{n-1}$ divise $P$.

Les polynômes divisibles par leur dérivée sont les polynômes de la forme $\lambda(X-a)^n$, $\lambda\in\Cc\setminus\{0\}$, $n\in\Nn^*$, $a\in\Cc$.
}
}
