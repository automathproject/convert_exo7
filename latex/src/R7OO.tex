\uuid{R7OO}
\exo7id{5988}
\auteur{quinio}
\organisation{exo7}
\datecreate{2011-05-18}
\isIndication{false}
\isCorrection{true}
\chapitre{Probabilité discrète}
\sousChapitre{Probabilité et dénombrement}

\contenu{
\texte{
Lors d'une loterie de Noël, $300$ billets sont
vendus aux enfants de l'école ; $4$ billets sont gagnants.
J'achète $10$ billets, quelle est la probabilité pour que je gagne au moins un lot?
}
\reponse{
L'univers des possibles est ici l'ensemble des combinaisons de $10$ billets
parmi les $300$ ; il y en a $\binom{300}{10}$.
Je ne gagne rien si les $10$ billets achetés se trouvent parmi les $296$
billets perdants, ceci avec la probabilité : 
\begin{equation*}
\frac{\binom{296}{10}}{\binom{300}{10}}.
\end{equation*}

La probabilité cherchée est celle de l'événement complémentaire : 
\begin{equation*}
1-\frac{\binom{296}{10}}{\binom{300}{10}}\simeq 0.127.
\end{equation*}

La probabilité est environ $12.7$\% de gagner au moins un lot.
}
}
