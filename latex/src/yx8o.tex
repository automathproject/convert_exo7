\uuid{yx8o}
\exo7id{5504}
\auteur{rouget}
\organisation{exo7}
\datecreate{2010-07-15}
\isIndication{false}
\isCorrection{true}
\chapitre{Espace euclidien, espace normé}
\sousChapitre{Projection, symétrie}

\contenu{
\texte{
Dans $\Rr^3$, espace vectoriel euclidien orienté rapporté à la base orthonormée directe $(i,j,k)$, déterminer l'image du plan d'équation $x+y=0$ par
}
\begin{enumerate}
    \item \question{la symétrie orthogonale par rapport au plan d'équation $x-y+z=0$,}
\reponse{Soit $s$ la symétrie orthogonale par rapport au plan $P'$ d'équation $x-y+z=0$. $s(P)$ est le plan de vecteur normal $s(n)$. 
Or, le vecteur $n$ est dans $P'$  et donc $s(n)=n$ puis $s(P)=P$.

\begin{center}
\shadowbox{
$s(P)$ est le plan $P$.
}
\end{center}}
    \item \question{la symétrie orthogonale par rapport au vecteur $(1,1,1)$,}
\reponse{Notons $\sigma$ la symétrie orthogonale par rapport au vecteur $u=(1,1,1)$. $\sigma(P)$ est le plan de vecteur normal 

\begin{center}
$\sigma(n)=2\frac{n.u}{\|u\|^2}u-n=2\frac{2}{3}(1,1,1)-(1,1,0)=\frac{1}{3}(1,1,4)$.
\end{center}

\begin{center}
\shadowbox{
$\sigma(P)$ est le plan d'équation $x+y+4z=0$.
}
\end{center}}
    \item \question{par la rotation d'angle $\frac{\pi}{4}$ autour du vecteur $(1,1,1)$.}
\reponse{Notons $r$ la rotation d'angle $\frac{\pi}{4}$ autour du vecteur unitaire $u=\frac{1}{\sqrt{3}}(1,1,1)$. $r(P)$ est le plan de vecteur normal 
\begin{align*}\ensuremath
r(n)&=\left(\cos\frac{\pi}{4}\right)n+\left(1-\cos\frac{\pi}{4}\right)(n.u)u+\left(\sin\frac{\pi}{4}\right)u\wedge n\\
 &=\frac{1}{\sqrt{2}}(1,1,0)+\left(1-\frac{1}{\sqrt{2}}\right)\times\frac{2}{3}(1,1,1)+\frac{1}{\sqrt{2}}\times\frac{1}{\sqrt{3}}(-1,1,0)\\
 &=\frac{1}{3\sqrt{2}}(3+2(\sqrt{2}-1)-\sqrt{3},3+2(\sqrt{2}-1)+\sqrt{3},2(\sqrt{2}-1))=\frac{1}{3\sqrt{2}}(1+2\sqrt{2}-\sqrt{3},1+2\sqrt{2}+\sqrt{3},2(\sqrt{2}-1)).
\end{align*}

\begin{center}
\shadowbox{
$r(P)$ est le plan d'équation $(1+2\sqrt{2}-\sqrt{3})x+(1+2\sqrt{2}+\sqrt{3})y+2(\sqrt{2}-1)z=0$.
}
\end{center}}
\end{enumerate}
}
