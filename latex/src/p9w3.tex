\uuid{p9w3}
\exo7id{645}
\auteur{ridde}
\organisation{exo7}
\datecreate{1999-11-01}
\isIndication{true}
\isCorrection{true}
\chapitre{Continuité, limite et étude de fonctions réelles}
\sousChapitre{Continuité : théorie}

\contenu{
\texte{
Soient $I$ un intervalle de $\Rr$ et $f : I \rightarrow \Rr$ continue, telle que pour chaque $x \in I$, $f (x)^2 = 1$.
Montrer que $f = 1$  ou $f = -1$.
}
\indication{Ce n'est pas tr\`es dur mais il y a quand m\^eme quelque chose \`a d\'emontrer :
ce n'est pas parce que $f(x)$ vaut $+1$ ou $-1$ que la fonction est constante.
Raisonner par l'absurde et utiliser le th\'eor\`eme des valeurs interm\'ediaires.}
\reponse{
Comme $f(x)^2 = 1$ alors $f(x) = \pm 1$. 
Attention ! Cela ne veut pas dire que 
la fonction est constante \'egale \`a $1$ ou $-1$.
Supposons, par exemple, qu'il existe $x$ tel que $f(x)=+1$.
Montrons que $f$ est constante \'egale \`a $+1$.
S'il existe $y \not= x$ tel que $f(y) = -1$
alors $f$ est positive en $x$, n\'egative en $y$
et continue sur $I$. Donc, par le th\'eor\`eme des valeurs interm\'ediaires,
il existe $z$ entre $x$ et $y$ tel que $f(z) = 0$, ce qui contredit
$f(z)^2 = 1$. Donc $f$ est constante \'egale \`a $+1$.
}
}
