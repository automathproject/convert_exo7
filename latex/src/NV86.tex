\uuid{NV86}
\exo7id{7112}
\auteur{megy}
\organisation{exo7}
\datecreate{2017-01-21}
\isIndication{true}
\isCorrection{true}
\chapitre{Géométrie affine euclidienne}
\sousChapitre{Géométrie affine euclidienne du plan}

\contenu{
\texte{
% tags : similitude
Soit $ABCD$ et $A'B'C'D'$ deux carrés du plan inclus l'un dans l'autre. On suppose que ce sont des cartes routières de la même région, tracées à différentes échelles, posées l'une sur l'autre. Montrer qu'il existe un unique point dont les représentations sur les deux cartes coïncident. (Bonus : construire ce point.)
}
\indication{Similitude.}
\reponse{
Deux carrés sont toujours semblables, et une similitude qui n'est pas une translation admet toujours un point fixe, son centre.
}
}
