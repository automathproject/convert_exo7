\uuid{Ebdm}
\exo7id{5821}
\auteur{rouget}
\organisation{exo7}
\datecreate{2010-10-16}
\isIndication{false}
\isCorrection{true}
\chapitre{Conique}
\sousChapitre{Parabole}

\contenu{
\texte{
L'espace de dimension $3$ est rapporté à un repère orthonormé $(O,i,j,k)$. On note $(\Gamma)$ la courbe d'équations

\begin{center}
$\left\{
\begin{array}{l}
y=x^2+x+1\\
x+y+z-1=0
\end{array}
\right.$.
\end{center}

Montrer que $(\Gamma)$ est une parabole dont on déterminera le sommet, l'axe, le foyer et la directrice.
}
\reponse{
$(\Gamma)$ est l'intersection d'un cylindre parabolique de direction $(Oz)$ et d'un plan non perpendiculaire à la direction de ce cylindre. On choisit un repère orthonormé $\mathcal{R}'=(\Omega,X,Y,Z)$ dans lequel le plan d'équation $x+y+z-1= 0$ dans le repère $\mathcal{R}=(O,x,y,z)$ soit le plan $(\Omega,X,Y)$  ou encore le plan d'équation $Z=0$ dans le repère $\mathcal{R}'$.

On pose donc $Z =\frac{1}{\sqrt{3}}(x+y+z-1)$ puis par exemple $X =\frac{1}{\sqrt{2}}(x-y)$ et $Y=\frac{1}{\sqrt{6}}(x+y-2z)$ ce qui s?écrit

\begin{center}
$\left(
\begin{array}{c}
X\\
Y\\
Z
\end{array}
\right)=\left(
\begin{array}{ccc}
\frac{1}{\sqrt{2}}&-\frac{1}{\sqrt{2}}&0\\
\rule[-4mm]{0mm}{10mm}\frac{1}{\sqrt{6}}&\frac{1}{\sqrt{6}}&-\frac{2}{\sqrt{6}}\\
\frac{1}{\sqrt{3}}&\frac{1}{\sqrt{3}}&\frac{1}{\sqrt{3}}
\end{array}
\right)\left(
\begin{array}{c}
x\\
y\\
z
\end{array}
\right)+\left(
\begin{array}{c}
0\\
0\\
-\frac{1}{\sqrt{3}}
\end{array}
\right)$
\end{center}

ou encore

\begin{center}
$\left(
\begin{array}{c}
x\\
y\\
z
\end{array}
\right)=\left(
\begin{array}{ccc}
\frac{1}{\sqrt{2}}&\frac{1}{\sqrt{6}}&\frac{1}{\sqrt{3}}\\
\rule[-4mm]{0mm}{10mm}-\frac{1}{\sqrt{2}}&\frac{1}{\sqrt{6}}&\frac{1}{\sqrt{3}}\\
0&-\frac{2}{\sqrt{6}}&\frac{1}{\sqrt{3}}
\end{array}
\right)\left(
\begin{array}{c}
X\\
Y\\
Z
\end{array}
\right)+\left(
\begin{array}{c}
\frac{1}{3}\\
\rule[-4mm]{0mm}{10mm}\frac{1}{3}\\
\frac{1}{3}
\end{array}
\right)$
\end{center}

Dans le repère $\mathcal{R}'$, la courbe $(\Gamma)$ admet pour système d'équations 

\begin{center}
$\left\{
\begin{array}{l}
-\frac{1}{\sqrt{2}}X+\frac{1}{\sqrt{6}}Y+\frac{1}{\sqrt{3}}Z+\frac{1}{3}=\left(\frac{1}{\sqrt{2}}X+\frac{1}{\sqrt{6}}Y+\frac{1}{\sqrt{3}}Z+\frac{1}{3}\right)^2 +\left(\frac{1}{\sqrt{2}}X+\frac{1}{\sqrt{6}}Y+\frac{1}{\sqrt{3}}Z+\frac{1}{3}\right)+ 1\\
Z=0
\end{array}
\right.$
\end{center}

ou encore

\begin{center}
$\left\{
\begin{array}{l}
-\frac{1}{\sqrt{2}}X+\frac{1}{\sqrt{6}}Y+\frac{1}{3}=\left(\frac{1}{\sqrt{2}}X+\frac{1}{\sqrt{6}}Y+\frac{1}{3}\right)^2 +\left(\frac{1}{\sqrt{2}}X+\frac{1}{\sqrt{6}}Y+\frac{1}{3}\right)+ 1\\
Z=0
\end{array}
\right.$.
\end{center}

Continuons à deux coordonnées $X$ et $Y$ dans le plan $(\Omega,X,Y)$.

\begin{align*}\ensuremath
M(X,Y)\in(\Gamma)&\Leftrightarrow-\frac{1}{\sqrt{2}}X+\frac{1}{\sqrt{6}}Y+\frac{1}{3}=\left(\frac{1}{\sqrt{2}}X+\frac{1}{\sqrt{6}}Y+\frac{1}{3}\right)^2 +\left(\frac{1}{\sqrt{2}}X+\frac{1}{\sqrt{6}}Y+\frac{1}{3}\right)+ 1\\
 &\Leftrightarrow\frac{1}{6}(\sqrt{3}X+Y)^2 + \frac{2}{3\sqrt{6}}(\sqrt{3}X+Y)+\frac{1}{9}+\sqrt{2}X+1=0\\
 &\Leftrightarrow\frac{1}{6}(\sqrt{3}X+Y)^2+\frac{4\sqrt{2}}{3}X+\frac{2}{3\sqrt{6}}Y+\frac{10}{9}=0\\
 &\Leftrightarrow(\sqrt{3}X+Y)^2+8\sqrt{2}X+\frac{2\sqrt{6}}{3}Y+\frac{20}{3}=0.
\end{align*}

On trouve déjà une conique du genre parabole. On pose maintenant $x'=\frac{1}{2}(\sqrt{3}X+Y)$ et $y'=\frac{1}{2}(-X+\sqrt{3}Y)$ correspondant aux formules de changement de repère
$\left\{
\begin{array}{l}
x'=\frac{1}{2}(\sqrt{3}X+Y)\\
\rule{0mm}{6mm}y'=\frac{1}{2}(-X+\sqrt{3}Y)
\end{array}
\right.$    ou encore  $\left\{
\begin{array}{l}
X=\frac{1}{2}(\sqrt{3}x'-y')\\
\rule{0mm}{6mm}Y=\frac{1}{2}(x'+\sqrt{3}y')
\end{array}
\right.$.

Dans le repère $(\Omega,x',y')$, une équation de la courbe $(\Gamma)$ est

\begin{align*}\ensuremath
(\sqrt{3}X+Y)^2+8\sqrt{2}X+\frac{2\sqrt{6}}{3}Y+\frac{20}{3}=0&\Leftrightarrow4x'^2+4\sqrt{2}(\sqrt{3}x'-y')+\frac{\sqrt{6}}{3}(x'+\sqrt{3}y')+\frac{20}{3}=0\\
 &\Leftrightarrow x'^2+\frac{13}{2\sqrt{6}}x'-\frac{3}{2\sqrt{2}}y'+\frac{5}{3}=0\\
 &\Leftrightarrow\left(x'+\frac{13}{4\sqrt{6}}\right)^2-\frac{3}{2\sqrt{2}}y'+\frac{5}{3}-\frac{169}{96}=0\Leftrightarrow\left(x'+\frac{13}{4\sqrt{6}}\right)^2=\frac{3}{2\sqrt{2}}\left(y'+\frac{1}{8\sqrt{2}}\right).
\end{align*}

$(\Gamma)$ est la parabole de paramètre $p=\frac{3}{4\sqrt{2}}$ et dont les éléments caractéristiques dans le repère $(\Omega,x',y')$ sont 

$S\left(-\frac{13}{4\sqrt{6}},-\frac{1}{8\sqrt{2}}\right)$, $F = S +\frac{p}{2}(0,1)=\left(-\frac{13}{4\sqrt{6}},-\frac{1}{8\sqrt{2}}\right)+\frac{3}{8\sqrt{2}}(0,1)=\left(-\frac{13}{4\sqrt{6}},\frac{1}{4\sqrt{2}}\right)$   
puis $K= S-\frac{p}{2}(0,1)=\left(-\frac{13}{4\sqrt{6}},-\frac{1}{8\sqrt{2}}\right)-\frac{3}{8\sqrt{2}}(0,1)=\left(-\frac{13}{4\sqrt{6}},-\frac{1}{2\sqrt{2}}\right)$ et  donc $(D)$ : $y'= -\frac{1}{2\sqrt{2}}$.

On repasse maintenant dans le repère $(\Omega,X,Y)$. $S$ a pour coordonnées $\frac{1}{2}\left(
\begin{array}{cc}
\sqrt{3}&-1\\
1&\sqrt{3}
\end{array}
\right)\left(
\begin{array}{c}
-\frac{13}{4\sqrt{6}}\\
\rule{0mm}{6mm}-\frac{1}{8\sqrt{2}}
\end{array}\right)=\left(
\begin{array}{c}
-\frac{25}{16\sqrt{2}}\\
\rule{0mm}{6mm}-\frac{29}{16\sqrt{6}}
\end{array}\right)$, $F$ a pour coordonnées $\frac{1}{2}\left(
\begin{array}{cc}
\sqrt{3}&-1\\
1&\sqrt{3}
\end{array}
\right)\left(
\begin{array}{c}
-\frac{13}{4\sqrt{6}}\\
\rule{0mm}{6mm}\frac{1}{4\sqrt{2}}
\end{array}\right)=\left(
\begin{array}{c}
-\frac{7}{4\sqrt{2}}\\
\rule{0mm}{6mm}-\frac{5}{4\sqrt{6}}
\end{array}\right)$  
puis $(D)$ a pour équation $-X+\sqrt{3}Y=-\frac{1}{\sqrt{2}}$.

On revient enfin au repère $(O,x,y,z)$.

Le point $S$ a pour coordonnées $\left(
\begin{array}{ccc}
\frac{1}{\sqrt{2}}&\frac{1}{\sqrt{6}}&\frac{1}{\sqrt{3}}\\
\rule[-4mm]{0mm}{10mm}-\frac{1}{\sqrt{2}}&\frac{1}{\sqrt{6}}&\frac{1}{\sqrt{3}}\\
0&-\frac{2}{\sqrt{6}}&\frac{1}{\sqrt{3}}
\end{array}
\right)\left(
\begin{array}{c}
-\frac{25}{16\sqrt{2}}\\
\rule{0mm}{6mm}-\frac{29}{16\sqrt{6}}\\
\rule{0mm}{6mm}0
\end{array}\right)+\left(
\begin{array}{c}
\frac{1}{3}\\
\rule[-4mm]{0mm}{10mm}\frac{1}{3}\\
\frac{1}{3}
\end{array}
\right)=\left(
\begin{array}{c}
-\frac{3}{4}\\
\rule[-4mm]{0mm}{10mm}\frac{13}{16}\\
\frac{15}{16}
\end{array}
\right)$ 
puis le point $F$ a pour coordonnées   $\left(
\begin{array}{ccc}
\frac{1}{\sqrt{2}}&\frac{1}{\sqrt{6}}&\frac{1}{\sqrt{3}}\\
\rule[-4mm]{0mm}{10mm}-\frac{1}{\sqrt{2}}&\frac{1}{\sqrt{6}}&\frac{1}{\sqrt{3}}\\
0&-\frac{2}{\sqrt{6}}&\frac{1}{\sqrt{3}}
\end{array}
\right)\left(
\begin{array}{c}
-\frac{7}{4\sqrt{2}}\\
\rule{0mm}{6mm}-\frac{5}{4\sqrt{6}}\\
\rule{0mm}{6mm}0
\end{array}\right)+\left(
\begin{array}{c}
\frac{1}{3}\\
\rule[-4mm]{0mm}{10mm}\frac{1}{3}\\
\frac{1}{3}
\end{array}
\right)=\left(
\begin{array}{c}
-\frac{3}{4}\\
\rule[-4mm]{0mm}{10mm}1\\
\frac{3}{4}
\end{array}
\right)$
et enfin 

$(D)$ :  $\left\{
\begin{array}{l}
-\frac{1}{\sqrt{2}}(x-y)+\frac{\sqrt{3}}{\sqrt{6}}(x+y-2z)=-\frac{1}{\sqrt{2}}\\
x+y+z-1=0
\end{array}
\right.$

\begin{center}
\shadowbox{
\begin{tabular}{c}
$(\Gamma)$ est la parabole de paramètre $p=\frac{3}{4\sqrt{2}}$, de sommet $S\left(-\frac{3}{4},\frac{13}{16},
\frac{15}{16}\right)$, de foyer $F\left(-\frac{3}{4},1,\frac{3}{4}\right)$\\
et de directrice $(D)$ : $\left\{
\begin{array}{l}
-y+z=\frac{1}{2}\\
x+y+z-1=0
\end{array}
\right.$.
\end{tabular}
}
\end{center}
}
}
