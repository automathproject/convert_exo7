\uuid{sGga}
\exo7id{5586}
\auteur{rouget}
\organisation{exo7}
\datecreate{2010-10-16}
\isIndication{false}
\isCorrection{true}
\chapitre{Application linéaire}
\sousChapitre{Image et noyau, théorème du rang}

\contenu{
\texte{
Soient $E$ un espace vectoriel et $f$ un endomorphisme de $E$. Pour $k\in\Nn$, on pose $N_k= \text{Ker}(f^k)$ et $I_k=\text{Im}(f^k)$ puis $N=\displaystyle\bigcup_{k\in\Nn}N_k$ et $I=\displaystyle\bigcap_{k\in\Nn}I_k$. ($N$ est le nilespace de $f$ et $I$ le c\oe ur de $f$)
}
\begin{enumerate}
    \item \question{\begin{enumerate}}
\reponse{\begin{enumerate}}
    \item \question{Montrer que les suites $(N_k)_{k\in\Nn}$ et $(I_k)_{k\in\Nn}$ sont respectivement croissante et décroissante pour l'inclusion.}
\reponse{Soient $k\in\Nn$ et $x\in E$. $x\in N_k\Rightarrow f^k(x)=0\Rightarrow f(f^k(x)) = 0\Rightarrow x\in N_{k+1}$. 

\begin{center}
\shadowbox{
$\forall k\in\Nn$, $N_k\subset N_{k+1}$.
}
\end{center}

Soient $k\in\Nn$ et $y\in I_{k+1}\Rightarrow\exists x\in E/\;y = f^{k+1}(x)\Rightarrow\exists x\in E/\;y = f^k(f(x))\Rightarrow y\in I_k$.

\begin{center}
\shadowbox{
$\forall k\in\Nn$, $I_{k+1}\subset I_{k}$.
}
\end{center}}
    \item \question{Montrer que $N$ et $I$ sont stables par $f$.}
\reponse{Soit $x\in N$. Il existe un entier $k$ tel que $x$ est dans $N_k$ ou encore tel que $f^k(x)=0$. Mais alors $f^k(f(x)) = f(f^k(x))=0$ et $f(x)$ est dans $N_k$ et donc dans $N$. Ainsi, $N$ est stable par $f$.

Soit $y\in I$. Alors, pour tout naturel $k$, il existe $x_k\in E$ tel que $y=f^k(x_k)$. Mais alors, pour tout entier $k$, $f(y)=f(f^k(x_k))=f^k(f(x))$ est dans $I_k$, et donc $f(y)$ est dans $I$. $I$ est stable par $f$.}
    \item \question{Montrer que $\forall k\in\Nn$, $(N_k =N_{k+1})\Rightarrow(N_{k+1}=N_{k+2})$.}
\reponse{Si $N_k=N_{k+1}$, on a déjà $N_{k+1}\subset N_{k+2}$. Montrons que $N_{k+2}\subset N_{k+1}$.

Soit $x\in N_{k+2}$. Alors $f^{k+1}(f(x))=0$ et donc $f(x)\in N_{k+1}= N_k$. Donc, $f^k(f(x))=0$ ou encore $x$ est dans $N_{k+1}$. On a montré que

\begin{center}
\shadowbox{
$\forall k\in\Nn$, $[(N_k = N_{k+1})\Rightarrow (N_{k+1}= N_{k+2})]$.
}
\end{center}}
\end{enumerate}
}
