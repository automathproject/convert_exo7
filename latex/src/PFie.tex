\uuid{PFie}
\exo7id{1877}
\auteur{maillot}
\organisation{exo7}
\datecreate{2001-09-01}
\isIndication{false}
\isCorrection{false}
\chapitre{Espace vectoriel normé}
\sousChapitre{Espace vectoriel normé}

\contenu{
\texte{
Soit $E$ l'espace vectoriel des fonctions continues de $[-1,1]$ à
valeurs dans $\R$. On définit une norme sur $E$ en posant
$$\|f\|_1=\int_{-1}^1 |f(t)| \,dt.$$ On va montrer que $E$ muni de
cette norme n'est pas complet. Pour cela, on définit une suite
$(f_n)_{n\in\N^*}$ par
\[f_n(t)=\begin{cases} -1 &\text{si\ } -1\le t \le -\frac1n\\
               nt &\text{si\ } -\frac1n\le t \le \frac1n\\
               1  &\text{si\ \ \ \ \ } \frac1n \le t\le 1.
          \end{cases}\]
}
\begin{enumerate}
    \item \question{Vérifier que $f_n\in E$ pour tout $n\ge 1$.}
    \item \question{Montrer que $$\|f_n-f_p\|\le \sup(\frac2n,\frac2p)$$ et en
déduire que $(f_n)$ est de Cauchy.}
    \item \question{Supposons qu'il existe une fonction $f\in E$ telle que $(f_n)$
converge vers $f$ dans $(E,\|\cdot\|_1)$. Montrer qu'alors on a
$$\lim_{n\rightarrow+\infty}\int_{-1}^{-\alpha} |f_n(t)-f(t)|\, dt=0
\qquad \text{et} \qquad \lim_{n\rightarrow+\infty}\int_{\alpha}^1
|f_n(t)-f(t)|\, dt=0$$ pour tout $0<\alpha<1$.}
    \item \question{Montrer qu'on a $$\lim_{n\rightarrow+\infty}\int_{-1}^{-\alpha}
|f_n(t)+1|\, dt=0
\qquad \text{et} \qquad \lim_{n\rightarrow+\infty}\int_{\alpha}^1
|f_n(t)-1|\, dt=0$$ pour tout $0<\alpha<1$.
En déduire que
\begin{align*}
&f(t)=-1\qquad &\forall t\in[-1,0[\\
&f(t)=1\qquad &\forall t\in ]0,1].
\end{align*}
Conclure.}
\end{enumerate}
}
