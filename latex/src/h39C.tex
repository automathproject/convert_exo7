\uuid{h39C}
\exo7id{5282}
\auteur{rouget}
\organisation{exo7}
\datecreate{2010-07-04}
\isIndication{false}
\isCorrection{true}
\chapitre{Dénombrement}
\sousChapitre{Autre}

\contenu{
\texte{
Quelle est la probabilité $p_n$ pour que dans un groupe de $n$ personnes choisies au hasard, deux personnes au moins aient le même anniversaire (on considèrera que l'année a toujours $365$ jours, tous équiprobables). Montrer que pour $n\geq23$, on a $p_n\geq\frac{1}{2}$.
}
\reponse{
Si $n\geq366$, on a clairement $p_n=1$ (Principe des tiroirs~:~si $366$ personnes sont à associer à $365$ dates d'anniversaire, alors $2$ personnes au moins sont à associer à la même date d'anniversaire).

Si $2\leq n\leq 365$, on a $p_n=1-q_n$ où $q_n$ est la probabilité que les dates d'anniversaire soient deux à deux distinctes. Il y a $(365)^n$ répartitions possibles des dates d'anniversaires (cas possibles) et parmi ces répartitions, il y en a $365.364.363....(365-n+1)$ telles que les dates d'anniversaire soient deux à deux distinctes. Finalement 

$$p_n=1-\frac{1}{(365)^n}365.364.363....(365-n+1)=1-\prod_{k=1}^{n-1}\frac{365-k}{365}=1-\prod_{k=1}^{n-1}(1-\frac{k}{365}).$$

Ensuite,

$$p_n\geq\frac{1}{2}\Leftrightarrow\prod_{k=1}^{n-1}(1-\frac{k}{365})\leq\frac{1}{2}\Leftrightarrow\sum_{k=1}^{n-1}\ln(1-\frac{k}{365}) \leq\ln\frac{1}{2}\Leftrightarrow\sum_{k=1}^{n-1}-\ln(1-\frac{k}{365})\geq\ln2.$$

Maintenant, soit $x\in[0,1[$. On a

$$-\ln(1-x)=\int_{0}^{x}\frac{1}{1-t}dt\geq\int_{0}^{x}\frac{1}{1-0}dt=x.$$

Pour $k$ élément de $\{1,...,n-1\}(\subset\{1,...,364\})$, $\frac{k}{365}$ est un réel élément de $[0,1[$.

En appliquant l'inégalité précédente, on obtient 

$$\sum_{k=1}^{n-1}-\ln(1-\frac{k}{365})\geq\sum_{k=1}^{n-1}\frac{k}{365}=\frac{n(n-1)}{730}.$$

Ainsi,

$$p_n\geq\frac{1}{2}\Leftarrow\frac{n(n-1)}{730}\geq\ln 2\Leftrightarrow n^2-n-730\ln2\geq0\Leftrightarrow n\geq\frac{1+\sqrt{1+2920\ln2}}{2}=22,99...\Leftrightarrow n\geq23.$$

Finalement, dans un groupe d'au moins $23$ personnes, il y a plus d'une chance sur deux que deux personnes au moins aient la même date d'anniversaire.
}
}
