\uuid{DuRa}
\exo7id{6750}
\auteur{queffelec}
\organisation{exo7}
\datecreate{2011-10-16}
\isIndication{false}
\isCorrection{false}
\chapitre{Théorème des résidus}
\sousChapitre{Théorème des résidus}

\contenu{
\texte{
Soit $P$ et $Q$ deux polyn\^omes tels que $\deg Q>\deg P$.
}
\begin{enumerate}
    \item \question{Exprimer $\sum\hbox{Res}({P\over Q})$ à l'aide des coefficients de $P$ et $Q$.}
    \item \question{Soit $P(z)=z^n+\sum_0^{n-1}a_kz^k$ un polyn\^ome dont toutes les racines sont
dans $D(0,R)$. Montrer que $f(x)=\displaystyle{1\over
2i\pi}\int_{\Gamma_R}{e^{xz}\over P(z)}\ dz$ est la solution de l'équation
différentielle d'ordre $n$, $y^{(n)}+ a_{n-1}y^{(n-1)}+\cdots+a_0y=0$ de
condition initiale $y^{(j)}(0)=0$ si $j<n-1$ et $y^{(n-1)}(0)=1$.}
\end{enumerate}
}
