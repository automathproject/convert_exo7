\uuid{I2ta}
\exo7id{2694}
\auteur{matexo1}
\organisation{exo7}
\datecreate{2002-02-01}
\isIndication{false}
\isCorrection{false}
\chapitre{Courbes planes}
\sousChapitre{Autre}

\contenu{
\texte{
Un segment AB de longueur $l$ se d{\'e}place dans le plan de fa\c{c}on que le point
A reste constamment sur l'axe ${\rm O}x$ et le point B sur l'axe ${\rm O}y$,
l'angle $\theta=(\vec{Ox},\vec {AB})$ variant de 0 {\`a} $ 2 \pi$. Soit M le point de AB tel que AM=$\alpha l$, $\alpha=C^{te}$, $0 \leq \alpha \leq 1$.
Calculer les coordonn{\'e}es de M en fonction de $\theta$ et d{\'e}terminer la courbe
qu'il d{\'e}crit. 
\begin{itemize}
\item A l'instant z{\'e}ro un oiseau s'envole d'un point A d'un
mouvement rectiligne uniforme de vitesse $\overrightarrow{v}$. Au m{\^e}me instant un
chasseur situ{\'e} au point B tire un coup de fusil en vue d'abattre l'oiseau. La
vitesse de la balle de fusil est en valeur absolue {\'e}gale {\`a} $u$. On suppose
{\'e}videmment que l'on a $u>\|\overrightarrow{v}\|$.
D{\'e}terminer la direction dans laquelle le chasseur doit tirer pour abattre
l'oiseau et l'instant $t_0$ de l'impact: on {\'e}crira deux {\'e}quations
d{\'e}terminant la vitesse vectorielle $\overrightarrow{u}$ de la balle de fusil
et l'instant $t_0$, et on en donnera les solutions. Donner l'expression de la
distance $d$ parcourue par l'oiseau entre les instants 0 et $t_0$.

Appliquer num{\'e}riquement les r{\'e}sultats pr{\'e}c{\'e}dents aux deux cas d{\'e}finis
par $A(0,0,a)$, $B(b,0,0)$, $\overrightarrow{v}=(0,v,0)$ avec~:
\item Envol {\`a} partir du repos~:
$a$=15m, $b$=10m, $v$=5m/s, $u$=300m/s 
\item Passage en plein vol~: 
$a$=20m, $b$=0, $v$=90km/h, $u$=300m/s
\end{itemize}
}
}
