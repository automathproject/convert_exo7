\uuid{4VVo}
\exo7id{4122}
\auteur{quercia}
\organisation{exo7}
\datecreate{2010-03-11}
\isIndication{false}
\isCorrection{true}
\chapitre{Equation différentielle}
\sousChapitre{Equations différentielles non linéaires}

\contenu{
\texte{
On consid{\`e}re l'{\'e}quationn diff{\'e}rentielle $(E) : x'=x^{2}-t$ et l'ensemble $D_{0}=\{(t,x)\,|\, x^{2}-t<0\}$.

Montrer que si $x$ est une solution de $(E)$ v{\'e}rifiant $(t_{0},x(t_{0}))\in D_{0}$, alors $x$ est d{\'e}finie sur
$[t_{0},+\infty[$ et la courbe int{\'e}grale reste dans $D_{0}$. En d{\'e}duire que $x(t)\mathop{\sim}\limits_{t\to +\infty} -\sqrt t$.
}
\reponse{
Supposons $t>t_{0}$ tel que $x^{2}(t)-t\ge0$. On peut alors poser $t_{1}=\min \{ t>t_{0}\,|\,x^{2}(t)-t>0\}$. On a alors 
$x^{2}(t_{1})-t_{1}=0$. Si $x(t_{1})=\sqrt{t_{1}}$, on {\'e}tudie la fonction $y(t)=x(t)-\sqrt t$. On a $y'(t_{1})=-\frac{1}{2\sqrt {t_{1}}}<0$. Cela
contredit le fait que, pour tout $t\in [t_{0},t_{1}[$, $y(t)<0$. De m{\^e}me si $x(t_{1})=-\sqrt{t_{1}}$, on {\'e}tudie la fonction $z(t)=x(t)+\sqrt t$ et on aboutit
{\`a} une contradiction. Par cons{\'e}quent la courbe int{\'e}grale reste dans $D_{0}$. Si la solution maximale ({\`a} droite) est d{\'e}finie sur $[t_{0},\beta[$, avec 
$\beta \in \R$, alors pour tout $t\in [t_{0},\beta[$, $-\beta\le x'(t)\le 0$. On en d{\'e}duit que $x'$ est int{\'e}grable sur $[t_{0},\beta[$ et donc que $x(t)$ admet
une limite finie quand $t$ tend vers $\beta$. On prolonge la fonction en $\beta$ et la fonction prolong{\'e}e v{\'e}rifie $(E)$ sur $[t_{0},\beta]$ ce qui est impossible.
On en d{\'e}duit que $\beta =+\infty$. On a, pour tout $t\ge t_{0}$, $x'(t)<0$, donc $x$ est d{\'e}croissante. Si $x(t)$ a une limite $\ell \in \R$ en $+\infty$
alors $x'(t)\mathop{\sim}\limits_{t\to +\infty}-t$, ce qui est impossible. Par cons{\'e}quent $x(t)\to -\infty$ (lorsque $t\to +\infty$). En particulier, pour $t$ assez
grand, $x(t)\le 0$. En d{\'e}rivant $(E)$ on a $x''(t)=2x(t)(x^{2}(t)-t)-1$. Si, {\`a} partir d'un certain rang, pour tout $t$, $x''(t)\ge 0$ alors $x'$ est croissante et 
major{\'e}e. Elle ne peut tendre que vers 
$0$ car sinon $x'(t) \sim \ell$, puis $x(t)\sim \ell t$ et $x'(t)\sim \ell ^{2} t^{2}$. Sinon il existe $t_{1}$ tel que $x''(t_{1})<0$. S'il existe $t_{2}>t_{1}$ tel
que $x''(t_{2})=0$ (avec $t_{2}$ minimal) alors $x'''(t_{2})=2x+\frac{1}{2x^{2}}$ qui est n{\'e}gatif pour $t$ assez grand. Ceci est impossible et donc
dans ce cas $x''(t)$ reste n{\'e}gatif lorsque $t$ tend vers $+\infty$, on a alors $0<t-x^{2}(t)\le \frac{-1}{2x(t)}\cdotp$ Par cons{\'e}quent 
$x^{2}(t)-t\to 0$ lorsque $t\to +\infty$, on en d{\'e}duit que $x(t)\mathop{\sim}\limits_{t\to +\infty}-\sqrt t$.
}
}
