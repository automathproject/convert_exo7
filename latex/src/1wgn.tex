\uuid{1wgn}
\exo7id{3903}
\auteur{quercia}
\organisation{exo7}
\datecreate{2010-03-11}
\isIndication{false}
\isCorrection{true}
\chapitre{Continuité, limite et étude de fonctions réelles}
\sousChapitre{Etude de fonctions}

\contenu{
\texte{
Soient $0 < a_1 < a_2 < \dots < a_p$ des réels fixés.
}
\begin{enumerate}
    \item \question{Montrer que pour tout réel $a > a_p$ il existe un unique réel $x_a > 0$
      solution de l'équation : $a_1^x + \dots + a_p^x = a^x$.}
\reponse{\'Etude de $x \mapsto \left(\frac{a_1}a\right)^x + \dots + \left(\frac{a_p}a\right)^x$.}
    \item \question{Pour $a < b$, comparer $x_a$ et $x_b$.}
\reponse{$x_a > x_b$.}
    \item \question{Chercher $\lim_{a\to+\infty} x_a$ puis
      $\lim_{a\to+\infty} x_a\ln a$}
\reponse{$x_a\to\ell$.
               Si $\ell > 0$, $a^{a_a} \to +\infty$, mais
               $a_1^{x_a} + \dots + a_p^{x_a} \to a_1^\ell + \dots + a_p^\ell$.

               Donc $\ell = 0$, et $x_a\ln a \to \ln p$.}
\end{enumerate}
}
