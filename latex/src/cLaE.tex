\uuid{cLaE}
\exo7id{4534}
\auteur{quercia}
\organisation{exo7}
\datecreate{2010-03-14}
\isIndication{false}
\isCorrection{true}
\chapitre{Suite et série de fonctions}
\sousChapitre{Autre}

\contenu{
\texte{
Soit $g(x) = \sum_{n=0}^\infty \frac{(-1)^n}{n!\,(x+n)}$.
}
\begin{enumerate}
    \item \question{Déterminer le domaine, $D$ de définition de $g$ et prouver que $g$ est de
    classe $\mathcal{C}^\infty$ sur $D$.}
    \item \question{Montrer que la quantité : $xg(x) - g(x+1)$ est constante sur $D$.}
    \item \question{Tracer la courbe représentative de $g$ sur $]0,+\infty[$.}
    \item \question{Donner un équivalent de $g(x)$ en $+\infty$ et en $0^+$.}
\reponse{
$xg(x) - g(x+1) = \frac1e$.
CSA $ \Rightarrow  g' < 0$. $g(x)\to +\infty$ lorsque $x\to0^+$,
                               $g(x)\to 0$ lorsque $x\to+\infty$.
$g(x) \sim \frac1x$ en $0^+$ et $g(x) \sim \frac1{ex}$ en
             $+\infty$.
}
\end{enumerate}
}
