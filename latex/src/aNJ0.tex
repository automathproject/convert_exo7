\uuid{aNJ0}
\exo7id{3877}
\auteur{quercia}
\organisation{exo7}
\datecreate{2010-03-11}
\isIndication{false}
\isCorrection{false}
\chapitre{Continuité, limite et étude de fonctions réelles}
\sousChapitre{Etude de fonctions}

\contenu{
\texte{
Soit $f : {\R} \to {\R}$ une fonction bornée.

On note ${\cal E} = \{g : \R \to \R \text{ croissantes tq } g \le f \}$,
et pour $x\in\R$ : $\tilde f(x) = \sup\{ g(x) \text{ tq } g \in {\cal E} \}$.
}
\begin{enumerate}
    \item \question{Montrer que $\tilde f \in {\cal E}$.}
    \item \question{On suppose $f$ continue. Montrer que $\tilde f$ est aussi continue.\par
      (S'il existe un point $x_0 \in \R$ tel que
      $\lim_{x\to x_0^-} \tilde f(x) < \lim_{x\to x_0^+} \tilde f(x)$,
      construire une fonction de $\cal E$ supérieure à~$\tilde f\,$)}
\end{enumerate}
}
