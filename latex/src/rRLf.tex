\uuid{rRLf}
\exo7id{7170}
\auteur{megy}
\organisation{exo7}
\datecreate{2017-07-26}
\isIndication{true}
\isCorrection{true}
\chapitre{Propriétés de R}
\sousChapitre{Autre}

\contenu{
\texte{
%[application directe de'AM>GM]
On considère deux réels positifs dont le produit vaut $100$. Leur somme a-t-elle une valeur minimale et si oui laquelle et dans quel(s) cas?
}
\indication{Utiliser l'inégalité arithmético-géométrique.}
\reponse{
Notons $a$ et $b$ les nombres de l'énoncé. On a $ab=100$. L'inégalité arithmético-géométrique fournit :
\[ \frac{a+b}{2}\geq \sqrt{ab}=10\]
avec égalité ssi $a=b$, donc la somme est supérieure à $20$, avec égalité ssi $a=b=10$.
}
}
