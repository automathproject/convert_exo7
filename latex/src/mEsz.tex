\uuid{mEsz}
\exo7id{5069}
\auteur{rouget}
\organisation{exo7}
\datecreate{2010-06-30}
\isIndication{false}
\isCorrection{true}
\chapitre{Nombres complexes}
\sousChapitre{Trigonométrie}

\contenu{
\texte{
Montrer que $\sum_{}^{}\cos\left(\pm a_1\pm a_2\pm...\pm
a_n\right)=2^n\cos a_1\cos a_2...\cos a_n$ (la somme comporte $2^n$ termes).
}
\reponse{
Pour $n$ naturel non nul, on pose $S_n=\sum_{}^{}e^{i(\pm a_1\pm...\pm a_n)}$.
\textbullet~$S_1=e^{ia_1}+e^{-ia_1}=2\cos a_1$
\textbullet~Soit $n\geq1$. Supposons que $S_n=2^n\cos a_1...\cos a_n$ alors

\begin{align*}
S_{n+1}&=\sum_{}^{}e^{i(\pm a_1\pm...\pm a_{n+1})}=e^{ia_{n+1}}\sum_{}^{}e^{i(\pm a_1\pm...\pm
a_n)}+e^{-ia_{n+1}}\sum_{}^{}e^{i(\pm a_1\pm...\pm a_n)}\\
 &=2\cos(a_{n+1})S_n=2^{n+1}\cos a_1...\cos a_{n+1}.
\end{align*}
On a montré par récurrence que : $\forall n\geq1,\;S_n=2^n\cos a_1...\cos a_n$.
Ensuite, pour $n\geq1$, $\sum_{}^{}\cos(\pm a_1\pm...\pm a_n)=\Re(S_n)=2^n\cos a_1...\cos a_n$ (et on obtient aussi
$\sum_{}^{}\sin(\pm a_1\pm...\pm a_n)=\Im(S_n)=0$).

\begin{center}
\shadowbox{
$\forall n\in\Nn^*$, $\sum_{}^{}\cos(\pm a_1\pm...\pm a_n)=2^n\cos a_1...\cos a_n$.
}
\end{center}
}
}
