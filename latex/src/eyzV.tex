\uuid{eyzV}
\exo7id{3267}
\auteur{quercia}
\organisation{exo7}
\datecreate{2010-03-08}
\isIndication{false}
\isCorrection{true}
\chapitre{Polynôme, fraction rationnelle}
\sousChapitre{Autre}

\contenu{
\texte{
Trouver $\lambda \in \R$ tel que $2X^3 + 5X^2 - X + \lambda$
ait une racine de module 1.
}
\reponse{
racine 1 : $\lambda = -6$.\par
 racine -1 : $\lambda = -4$.\par
 racine $\alpha \in \mathbb{U}\setminus\{\pm1\}$ : les autres sont $\frac 1\alpha$ et
 $-\lambda  \Rightarrow  \lambda = 6$, $\alpha = \frac{1+i\sqrt{15}}4$.
}
}
