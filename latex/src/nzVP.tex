\uuid{nzVP}
\exo7id{5480}
\auteur{rouget}
\organisation{exo7}
\datecreate{2010-07-10}
\isIndication{false}
\isCorrection{true}
\chapitre{Equation différentielle}
\sousChapitre{Résolution d'équation différentielle du deuxième ordre}

\contenu{
\texte{
On considère l'équation différentielle $(E)$~:~$ax^2y''+bxy'+cy= 0$ ($a$, $b$, $c$ réels, $a\neq0$)
pour $x\in]0,+\infty[$.
}
\begin{enumerate}
    \item \question{Soit $y$ une fonction deux fois dérivable sur $]0,+\infty[$. Pour $t\in\Rr$, on pose $z(t)=y(e^t)$. Vérifier
que $y$ est deux fois dérivable sur $]0,+\infty[$ si et seulement si $z$ est deux fois dérivable sur $\Rr$.}
\reponse{Supposons $y$ deux fois dérivable sur $]0,+\infty[$. La fonction $t\mapsto e^t$ est deux fois dérivable sur
$\Rr$ à valeurs dans $]0,+\infty[$ et la fonction $x\mapsto y(x)$ est deux fois dérivable sur $]0,+\infty[$. Donc,
puisque pour tout réel $t$, $z(t)=y(e^t)$, la fonction $z$ est deux fois dérivable sur $\Rr$ en tant que composée de
fonctions deux fois dérivables.
Réciproquement, supposons que $z$ est deux fois dérivable sur $\Rr$. La fonction $x\mapsto\ln x$ est deux fois
dérivable sur $]0,+\infty[$ à valeurs dans $\Rr$ et la fonction $t\mapsto z(t)$ est deux fois dérivable sur $\Rr$.
Donc, puisque pour tout réel strictement positif $x$, $y(x)=z(\ln x)$, la fonction $y$ est deux fois dérivable sur
$]0,+\infty[$.}
    \item \question{Effectuer le changement d'inconnue précédent dans l'équation différentielle $(E)$ et vérifier que la
résolution de $(E)$ se ramène à la résolution d'une équation linéaire du second ordre à coefficients constants.}
\reponse{Pour $t$ réel, posons donc $x=e^t$ puis, $z(t)=y(x)=y(e^t)$. Alors, $z'(t)=e^ty'(e^t)=xy'(x)$ puis
$z''(t)=e^ty'(e^t)+(e^t)^2y''(e^t)=xy'(x)+x^2y''(x)$. Donc, $xy'(x)=z'(t)$ et $x^2y''(x)=z''(t)-xy'(x)=z''(t)-z'(t)$.

Par suite,

$$ax^2y''(x)+bxy'(x)+cy(x)=a(z''(t)-z'(t))+bz'(t)+cz(t)=az''(t)+(b-a)z'(t)+cz(t).$$

Donc,

$$\forall x>0,\;ax^2y''(x)+bxy'(x)+cy(x)=0\Leftrightarrow\forall t\in\Rr,\;az''(t)+(b-a)z'(t)+cz(t)=0.$$}
    \item \question{Résoudre sur $]0,+\infty[$, l'équation différentielle $x^2y''-xy'+y=0$.}
\reponse{On applique le 2) avec $a=1$, $b=-1$ et $c=1$. L'équation à résoudre sur $\Rr$ est alors $z''-2z'+z=0$. Les
solutions de cette équation sur $\Rr$ sont les fonctions de la forme $t\mapsto(\lambda t+\mu)e^t$,
$(\lambda,\mu)\in\Rr^2$. Les solutions sur $]0,+\infty[$ de l'équation initiale sont donc les fonctions de la forme
$x\mapsto\lambda x\ln x+\mu x$, $(\lambda,\mu)\in\Rr^2$.}
\end{enumerate}
}
