\uuid{jxh8}
\exo7id{1143}
\auteur{barraud}
\organisation{exo7}
\datecreate{2003-09-01}
\isIndication{true}
\isCorrection{true}
\chapitre{Déterminant, système linéaire}
\sousChapitre{Calcul de déterminants}

\contenu{
\texte{
Soit $(a_{0},...,a_{n-1})\in\C^{n}$, $x\in\C$. Calculer
 $$
 \Delta_{n}=
 \left|
   \begin{matrix}
   x &  0    &        & a_{0}   \\
    -1 &\ddots &\ddots  &\vdots  \\
      &\ddots &x      & a_{n-2} \\
    0 &       & -1      & x+a_{n-1}
   \end{matrix}
\right|
 $$
}
\indication{Développer par rapport à la dernière colonne.}
\reponse{
Commençons par un travail préparatoire : le calcul du déterminant de taille $(n-1)\times (n-1)$ :
{\footnotesize$$\Gamma_k = 
\begin{array}{|cccc|cccc|} 
x  &        &        &   &&&&\\
-1 & x      &        &   &&&&\\
   & \ddots & \ddots &   &&&&\\
   &        & -1     & x &&&&\\ 
   \hline
&&&&  -1 & x      &       &   \\
&&&&     &\ddots & \ddots      &   \\
&&&&     & &\ddots & x  \\
&&&&     &       &     & -1 \\
\end{array}
$$}
où le bloc en haut à gauche est de taille $k\times k$.

On développe, en commençant par la première ligne, puis encore une fois par la première ligne,...
pour trouver que 
$$\Gamma_k = x^k\times (-1)^{n-1-k}$$

Autre méthode : on retrouve le même résultat en utilisant les déterminant par blocs :
$$\begin{array}{|c|c|} 
A & B \\
\hline
(0) & C \\ 
\end{array}
= \det A \times \det C$$

\bigskip


Revenons à l'exercice !

Contrairement à l'habitude on développe par rapport à la colonne qui a le moins de $0$.
En développant par rapport à la dernière colonne on obtient :
{\footnotesize
\begin{align*}
 \Delta_{n}
& =
   \begin{vmatrix}
   x &  0    &        & a_{0}   \\
    -1 &\ddots &\ddots  &\vdots  \\
      &\ddots &x      & a_{n-2} \\
    0 &       & -1      & x+a_{n-1}
   \end{vmatrix} \\
 & = (-1)^{n-1} a_0 
   \begin{vmatrix}
   -1 &  x    &        &    \\
     & -1 & x  &  \\
      & & \ddots & \ddots       \\
     &       &       & -1
   \end{vmatrix}
+ (-1)^{n} a_1   
\begin{vmatrix}
   x & &&&\\
   & -1 &  x    &        &    \\
   &  & -1 & x  &  \\
    &  & & \ddots & \ddots       \\
    & &       &       & -1
   \end{vmatrix} 
\\
& \quad + \cdots +
(-1)^{2n-3} a_{n-2}   
\begin{vmatrix}
  x &      &    & &      \\
    -1 &\ddots &  & & \\
     &\ddots &\ddots & &  \\
      && -1 & x    &   \\
    &&&& -1 \\
   \end{vmatrix}
+ (-1)^{2n-2}(x+a_{n-1})
\begin{vmatrix}
  x &      &    &       \\
    -1 &\ddots &  &  \\
     &\ddots &\ddots &   \\
    &  & -1 & x       \\
   \end{vmatrix} \\
 & = \sum_{k=0}^{n-2} (-1)^{n-1+k} a_ k \times \Gamma_k \quad  + \quad (-1)^{2n-2}(x+a_{n-1})\Gamma_{n-1} \\
 & = \sum_{k=0}^{n-2} (-1)^{n-1+k} a_ k \times x^k\times (-1)^{n-1-k} \quad + \quad (x+a_{n-1})x^{n-1}\\
 & = a_0+a_1x+a_2x^2+\cdots + a_{n-1}x^{n-1} + x^n\\
\end{align*}
}
}
}
