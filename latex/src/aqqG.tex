\uuid{aqqG}
\exo7id{1087}
\auteur{ridde}
\organisation{exo7}
\datecreate{1999-11-01}
\isIndication{false}
\isCorrection{true}
\chapitre{Matrice}
\sousChapitre{Matrice et application linéaire}

\contenu{
\texte{
Soit $\Rr^2$ muni de la base canonique $\mathcal{B}=(\vec{i}, \vec{j})$.
Soit $f : \Rr^2 \to \Rr^2$ la projection sur l'axe des abscisses $\Rr \vec{i}$ 
parall\`element à $\Rr (\vec{i} + \vec{j})$.
Déterminer $\textrm{Mat}_{\mathcal{B},\mathcal{B}}(f)$, la matrice de $f$ dans la base $(\vec{i}, \vec{j})$.

Même question avec $\textrm{Mat}_{\mathcal{B}',\mathcal{B}}(f)$ où $\mathcal{B'}$ est la base 
$(\vec{i} - \vec{j}, -2\vec{i}+3\vec{j})$ de $\Rr^2$.
Même question avec $\textrm{Mat}_{\mathcal{B}',\mathcal{B}'}(f)$.
}
\indication{$f$ est l'application qui à $\begin{pmatrix}x\\y\end{pmatrix}$ associe
 $\begin{pmatrix}x-y\\0\end{pmatrix}$.}
\reponse{
Calcul de $\textrm{Mat}(f,\mathcal{B},\mathcal{B})$.
Comme $\mathcal{B}=(\vec{i}, \vec{j})$, la matrice s'obtient en calculant $f(\vec{i})$ et $f(\vec{j})$ :
$$f(\vec{i})=f\begin{pmatrix}1\\0\end{pmatrix} = \begin{pmatrix}1\\0\end{pmatrix} = \vec{i}
\quad 
f(\vec{j})=f\begin{pmatrix}0\\1\end{pmatrix} = \begin{pmatrix}-1\\0\end{pmatrix} = -\vec{i}$$
donc
$$\textrm{Mat}(f,\mathcal{B},\mathcal{B}) = \begin{pmatrix} 1 & -1 \\ 0 & 0 \end{pmatrix}$$
On garde la même application linéaire mais la base de départ change (la base d'arrivée reste $\mathcal{B}$).
Si on note $\vec{u} = \vec{i}-\vec{j}$ et $\vec{v} = -2\vec{i}+3\vec{j}$, on a 
$\mathcal{B'}=(\vec{i} - \vec{j}, -2\vec{i}+3\vec{j}) = (\vec{u},\vec{v})$. On exprime 
$f(\vec{u})$ et $f(\vec{v})$ dans la base d'arrivée $\mathcal{B}$.
$$f(\vec{u})=f(\vec{i}- \vec{j})=f\begin{pmatrix}1\\-1\end{pmatrix} = \begin{pmatrix}2\\0\end{pmatrix}
\quad 
f(\vec{v})=f(-2\vec{i}+3\vec{j})=f\begin{pmatrix}-2\\3\end{pmatrix} = \begin{pmatrix}-5\\0\end{pmatrix}$$
donc
$$\textrm{Mat}(f,\mathcal{B}',\mathcal{B}) = \begin{pmatrix} 2 & -5 \\ 0 & 0 \end{pmatrix}$$
Toujours avec le même $f$ on prend $\mathcal{B}'$ comme base de départ et d'arrivée,
il s'agit donc d'exprimer $f(\vec{u})$ et $f(\vec{v})$ dans la base $\mathcal{B}'=(\vec{u},\vec{v})$.
Nous venons de calculer que 
$$f(\vec{u})=f(\vec{i}- \vec{j})=f\begin{pmatrix}1\\-1\end{pmatrix} = \begin{pmatrix}2\\0\end{pmatrix}=2\vec{i}
\quad 
f(\vec{v})=f(2\vec{i}+3\vec{j})=f\begin{pmatrix}-2\\3\end{pmatrix} = \begin{pmatrix}-5\\0\end{pmatrix}=-5\vec{i}$$
Mais il  nous faut obtenir une expression en fonction de la base $\mathcal{B}'$.
Remarquons que 
$$\left\{\begin{array}{lcr}
\vec{u} &=& \vec{i}-\vec{j} \\
\vec{v} &=& -2\vec{i}+3\vec{j} \\           
         \end{array}\right.
\implies
\left\{\begin{array}{lcr}
\vec{i} &=& 3\vec{u}+\vec{v} \\
\vec{j} &=& 2\vec{u}+\vec{v} \\           
         \end{array}\right.$$
Donc 
$$f(\vec{u})=f(\vec{i}- \vec{j})=2\vec{i}=6\vec{u}+2\vec{v} = \begin{pmatrix}6\\2\end{pmatrix}_{\mathcal{B}'}
\quad
f(\vec{v})=f(-2\vec{i}+3\vec{j})=-5\vec{i}=-15\vec{u}-5\vec{v} = \begin{pmatrix}-15\\-5\end{pmatrix}_{\mathcal{B}'}$$
Donc
$$\textrm{Mat}(f,\mathcal{B}',\mathcal{B}') = \begin{pmatrix} 6 & -15 \\ 2 & -5 \end{pmatrix}$$

Remarque :
$\begin{pmatrix}x\\y\end{pmatrix}_{\mathcal{B}'}$
désigne le vecteur $x \vec{u}+y\vec{v}$.
}
}
