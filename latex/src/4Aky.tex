\uuid{4Aky}
\exo7id{4506}
\auteur{quercia}
\organisation{exo7}
\datecreate{2010-03-14}
\isIndication{false}
\isCorrection{true}
\chapitre{Suite et série de fonctions}
\sousChapitre{Convergence simple, uniforme, normale}

\contenu{
\texte{

}
\begin{enumerate}
    \item \question{Déterminer la limite simple des fonctions
    $f_n : x  \mapsto \frac{x^ne^{-x}}{n!}$
    sur $\R^+$ et montrer qu'il y a convergence uniforme.
    (On admettra la formule de Stirling : $n! \sim n^ne^{-n}\sqrt{2\pi n}$)}
    \item \question{Calculer $\lim_{n\to\infty} \int_{t=0}^{+\infty} f_n(t)\,d t$.}
\reponse{
Intégrale constante = 1.
}
\end{enumerate}
}
