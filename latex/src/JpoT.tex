\uuid{JpoT}
\exo7id{7652}
\auteur{mourougane}
\organisation{exo7}
\datecreate{2021-08-11}
\isIndication{false}
\isCorrection{false}
\chapitre{Sous-variété}
\sousChapitre{Sous-variété}

\contenu{
\texte{
Le but de l'exercice est de montrer le théorème suivant.

\textbf{Théorème.} 
\emph{
Soit $I$ un intervalle et $\kappa~:~I\to\Rr$ une fonction de classe $\mathcal{C}^\infty$.
Alors, il existe une courbe plane $c~:~I\to\Rr^2$ paramétrée par la longueur d'arc
et de fonction courbure $\kappa$. 
Cette courbe est unique à composition au but par un déplacement près.
}
}
\begin{enumerate}
    \item \question{On considère le système d'équations différentielles linéaire du premier ordre
$$\frac{d}{dt}\left(\begin{array}{c}c\\v\\n\end{array}\right)
=\left(\begin{array}{ccc}0&1&0\\0&0&\kappa\\0&-\kappa&0\end{array}\right)
\left(\begin{array}{c}c\\v\\n\end{array}\right).$$
où les fonctions $c,v,n~:~I\to\Rr^2$ sont les inconnues.
 Montrer que ce système admet une unique solution $(c(t),v(t),n(t))$ 
avec pour valeur initiale un vecteur fixé $(c_0,v_0,n_0)$ 
avec $(v_0,n_0)$ base orthonormée directe.}
    \item \question{\'Ecrire un système d'équations différentielles linéaire du premier ordre
satisfait par le vecteur $(<v,v>,<n,n>,<v,n>)$.}
    \item \question{Montrer que pour les solutions obtenues précédemment, $(v(t),n(t))$
reste un repère orthonormé direct.}
    \item \question{En déduire que la courbe $c$ obtenue est paramétrée par la longueur d'arc
et que sa fonction courbure est $\kappa$.}
    \item \question{Conclure.}
\end{enumerate}
}
