\uuid{EfNA}
\exo7id{7164}
\auteur{megy}
\organisation{exo7}
\datecreate{2017-06-11}
\isIndication{false}
\isCorrection{true}
\chapitre{Nombres complexes}
\sousChapitre{Géométrie}

\contenu{
\texte{
Soit $ABC$ un triangle équilatéral direct, et $M$ un point. On note $A'$ (resp. $B'$ et $C'$) le symétrique orthogonal de $M$ par rapport à la droite $(BC)$ (resp. $(CA)$ et $(AB)$). Le but de l'exercice est de démontrer que $ABC$ et $A'B'C'$ ont le même centre de gravité.
}
\begin{enumerate}
    \item \question{\'Ecrire en coordonnée complexe (relativement à un repère que l'on choisira judicieusement) la réflexion $\sigma_{AB}$ par rapport à l'axe $(AB)$.}
\reponse{On a l'égalité d'angles de droites $(OA,BA)=-\pi/6$ (angle inscrit ou calcul en coordonnées). On en déduit que la réflexion d'axe $(AB)$ s'écrit $z\mapsto \alpha\bar z+\beta$, avec $\alpha=e^{-\pi/3}=-j$. Ensuite, l'image de l'origine $O$ par la réflexion $\sigma_{AB}$ est le point d'affixe $-j^2a$. Finalement, vis-à-vis du repère choisi, la réflexion $\sigma_{AB}$ est représentée en coordonnée complexe par:
\[ z\mapsto -j\bar z -j^2.\]}
    \item \question{Conclure.}
\reponse{On calcule de même que les réflexions suivant $(BC)$ et $(CA)$ s'écrivent $z\mapsto -\bar z -1$ et $z\mapsto -j^2 \bar z -j$.  Soit $m$ l'affixe de $M$. D'après ce qui précède, ses images par les trois réflexions sont:
\[ 
a'= -\bar m-1,\quad
b'=-j^2\bar m-j,\quad
c'=-j\bar m-j^2.
\]

La relation $1+j+j^2=0$ donne alors $\frac{1}{3}(a'+b'+c')=0$, ce qu'il fallait démontrer.}
\end{enumerate}
}
