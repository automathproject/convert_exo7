\uuid{czOI}
\exo7id{2033}
\auteur{liousse}
\organisation{exo7}
\datecreate{2003-10-01}
\isIndication{false}
\isCorrection{false}
\chapitre{Géométrie affine euclidienne}
\sousChapitre{Géométrie affine euclidienne de l'espace}

\contenu{
\texte{
On consid\`ere les droites et les plans suivants dont les \'equations sont donn\'ees dans le
rep\`ere $(O,{\buildrel\rightarrow \over i},{\buildrel\rightarrow \over j}
,{\buildrel\rightarrow \over k})$. 
Donner leurs \'equations dans le nouveau rep\`ere $(A,
{\buildrel\rightarrow \over {AB}}, {\buildrel\rightarrow \over {AC}}, 
{\buildrel\rightarrow \over {AD}})$, sachant que dans $(O,{\buildrel\rightarrow \over i},
{\buildrel\rightarrow \over j},{\buildrel\rightarrow \over k})$ les points $A, B, C$ et
$D$ ont pour coordonn\'ees respectives $A (4,-1,2)$, $B (2, -5, 4)$, $C(5, 0, -3)$, $D(1,-5,6)$.
}
\begin{enumerate}
    \item \question{$ P : x+y=1 $}
    \item \question{$ P : 2x -3y +4z -1=0$}
    \item \question{$ P : x-y+z+3=0  $}
    \item \question{$P : \left\{  
         \begin{array}{l}
           x = 2t +3s + 1 \\
           y = t -s +2 \\
           z = 4t - 2s - 3
         \end{array}
     \right. $}
    \item \question{$(D):\left\{ \begin{array}{l} x+y+z=1 \\ 2x-y+4z=3 \end{array}\right. $}
    \item \question{$(D):\left\{ \begin{array}{l} 3x-y-z=-1 \\ 4x-3y-z=-2 \end{array}\right. $}
\end{enumerate}
}
