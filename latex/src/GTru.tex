\uuid{GTru}
\exo7id{1549}
\auteur{barraud}
\organisation{exo7}
\datecreate{2003-09-01}
\isIndication{false}
\isCorrection{false}
\chapitre{Endomorphisme particulier}
\sousChapitre{Endomorphisme orthogonal}

\contenu{
\texte{

}
\begin{enumerate}
    \item \question{Soit $r$ un endomorphisme symétrique d'un espace euclidien $E$. On dit que $r$ est
positif, si toutes ses valeurs propres sont positives.

Montrer que si $r$ est défini positif, il existe un et un seul endomorphisme symétrique
$s$ positif tel que $s^{2}=r$. On appelle $s$ racine carrée positive de $r$.

On dit que $r$ est défini positif si et seulement si toutes ses racines sont strictement
positives. Montrer que si $r$ est défini positif, alors sa racine positive aussi.}
    \item \question{Soit $f$ un endomorphisme de $E$. Montrer que ${}^t{f}f$ est symétrique et positif.
Montrer que si en plus $f$ est bijective, ${}^t{f}f$ est défini positif.}
    \item \question{On suppose maintenant que $f$ est une bijection. Soit $s$ la racine carrée positive de
${}^t{f}f$. Montrer que $u=f\circ s^{-1}$ est une transformation orthogonale. En
déduire que tout endomorphisme bijectif de $E$ peut s mettre sous la forme :
$$
 f=u\circ s
$$
où $u$ et une transformation orthogonale, et $s$ est symétrique défini positif.

Montrer que cette décomposition, appelée décomposition polaire de $f$ est unique.}
    \item \question{Que se passe-t-il si $f$ n'est pas bijective ?}
\end{enumerate}
}
