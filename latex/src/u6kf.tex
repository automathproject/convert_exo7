\uuid{u6kf}
\exo7id{1097}
\auteur{cousquer}
\organisation{exo7}
\datecreate{2003-10-01}
\isIndication{false}
\isCorrection{true}
\chapitre{Matrice}
\sousChapitre{Matrice et application linéaire}

\contenu{
\texte{
Soient trois vecteurs $e_1,e_2,e_3$ formant une base de $\Rr^3$.
On note $\phi$ l'application linéaire définie par
$\phi(e_1)=e_3$, $\phi(e_2)=-e_1+e_2+e_3$ et $\phi(e_3)=e_3$.
}
\begin{enumerate}
    \item \question{Écrire la matrice $A$ de $\phi$ dans la base $(e_1,e_2,e_3)$.
Déterminer le noyau de cette application.}
\reponse{On note la base $\mathcal{B}=(e_1,e_2,e_3)$
  et $X=\begin{pmatrix}x\\y\\z\end{pmatrix}_{\mathcal{B}}= x e_1+y e_2+z e_3$.
  La matrice $A=\textrm{Mat}_{\mathcal{B}}(f)$ est composée des vecteurs colonnes $\phi(e_i)$,
on sait 
$$\phi(e_1)=e_3 = \begin{pmatrix}0\\0\\1\end{pmatrix}_{\mathcal{B}} \quad
\phi(e_2)=-e_1+e_2+e_3 = \begin{pmatrix}-1\\1\\1\end{pmatrix}_{\mathcal{B}} \quad 
\phi(e_3)=e_3 = \begin{pmatrix}0\\0\\1\end{pmatrix}_{\mathcal{B}} \quad
$$

$$\text{donc } \quad A=\begin{pmatrix}
0 & -1 & 0 \\
0 & 1  & 0 \\
1 & 1  & 1 \\    
\end{pmatrix}$$

Le noyau de $\phi$ (ou celui de $A$) est l'ensemble de $X=\begin{pmatrix}x\\y\\z\end{pmatrix}$
tel que $AX=0$.

$$AX=0 \iff \begin{pmatrix}
0 & -1 & 0 \\
0 & 1  & 0 \\
1 & 1  & 1 \\  
\end{pmatrix} \times \begin{pmatrix}x\\y\\z\end{pmatrix}
=\begin{pmatrix}0\\0\\0\end{pmatrix}
\iff \left\{
\begin{array}{rcl}
-y&=&0\\
y&=&0\\
x+y+z&=&0\\
\end{array}\right.
$$
Donc $\Ker \phi = \big\{ \begin{pmatrix}x \\ 0 \\-x\end{pmatrix}_{\mathcal{B}}  \in \Rr^3 \mid x\in \Rr \big\}= 
\textrm{Vect} \begin{pmatrix}1\\0\\-1\end{pmatrix}_{\mathcal{B}} = \textrm{Vect} (e_1-e_3)$.
Le noyau est donc de dimension $1$.}
    \item \question{On pose $f_1=e_1-e_3$, $f_2=e_1-e_2$,  $f_3=-e_1+e_2+e_3$.
Calculer $e_1,e_2,e_3$ en fonction de $f_1,f_2,f_3$.
Les vecteurs $f_1,f_2,f_3$ forment-ils une base de $\Rr^3$ ?}
\reponse{On applique le pivot de Gauss comme si c'était un système linéaire :
$$\left\{
\begin{array}{cccccclr}
e_1  & &     &-& e_3  &=& f_1 &_{L_1}\\
e_1  &-& e_2 & &      &=& f_2 &_{L_2}\\
-e_1 &+& e_2 &+& e_3  &=& f_3 &_{L_3}\\
\end{array}\right.
\iff  \left\{
\begin{array}{cccccclr}
e_1  & &     &-& e_3  &=& f_1 &\\
     &-& e_2 &+& e_3  &=& f_2-f_1 &_{L_2-L_1}\\
     & & e_2 & &      &=& f_3+f_1 &_{L_3+L_1}\\
\end{array}\right.
$$
On en déduit
$$\left\{
\begin{array}{rcl}
 e_1 &=& f_1+f_2+f_3 \\
 e_2 &=& f_1+f_3\\
 e_3 &=& f_2+f_3 \\
\end{array}\right.
$$

Donc tous les vecteurs de la base $\mathcal{B}=(e_1,e_2,e_3)$ s'expriment en fonction
de $(f_1,f_2,f_3)$, ainsi la famille $(f_1,f_2,f_3)$ est génératrice.
Comme elle a exactement $3$ éléments dans l'espace vectoriel $\Rr^3$ de dimension $3$ alors
$\mathcal{B}'=(f_1,f_2,f_3)$ est une base.}
    \item \question{Calculer $\phi(f_1), \phi(f_2), \phi(f_3)$ en fonction de $f_1,f_2,f_3$.
Écrire la matrice $B$ de $\phi$ dans la base $(f_1,f_2,f_3)$ et trouver la nature
de l'application $\phi$.}
\reponse{$$\phi(f_1)=\phi(e_1-e_3)=\phi(e_1)-\phi(e_3)=e_3-e_3=0$$

$$\phi(f_2)=\phi(e_1-e_2)= \phi(e_1)-\phi(e_2)=e_3 - (-e_1+e_2+e_3) = e_1-e_2 = f_2$$

$$\phi(f_3)=\phi(-e_1+e_2+e_3)=-\phi(e_1)+\phi(e_2)+\phi(e_3)=-e_1+e_2+e_3=f_3$$

Donc, dans la base $\mathcal{B}'=(f_1,f_2,f_3)$, nous avons
$$\phi(f_1)=0=\begin{pmatrix}0\\0\\0\end{pmatrix}_{\mathcal{B}'}\quad
\phi(f_2)=f_2=\begin{pmatrix}0\\1\\0\end{pmatrix}_{\mathcal{B}'}
\phi(f_3)=f_3=\begin{pmatrix}0\\0\\1\end{pmatrix}_{\mathcal{B}'}$$

Donc la matrice de $\phi$ dans la base $\mathcal{B}'$ est
$$B=\begin{pmatrix}
0 & 0 & 0 \\
0 & 1 & 0 \\
0 & 0 & 1 \\    
\end{pmatrix}$$

$\phi$ est la projection sur  $\textrm{Vect} (f_2,f_3)$ parallèlement à $\textrm{Vect} (f_1)$ (autrement dit
c'est la projection sur le plan d'équation $(x'=0)$, parallèlement à l'axe des $x'$, ceci dans la base $\mathcal{B}'$).}
    \item \question{On pose $P=\begin{pmatrix}1&1&-1\cr 0&-1&1\cr-1&0&1\cr\end{pmatrix}$. Vérifier que $P$ est
inversible et calculer $P^{-1}$. Quelle relation lie $A$, $B$, $P$ et $P^{-1}$ ?}
\reponse{$P$ est la matrice de passage de $\mathcal{B}$ vers $\mathcal{B}'$.
En effet la matrice de passage contient -en colonnes- les coordonnées des vecteurs
de la nouvelle base $\mathcal{B}'$ exprimés dans l'ancienne base $\mathcal{B}$.


Si un vecteur a pour coordonnées $X$ dans la base $\mathcal{B}$ et $X'$ dans la base $\mathcal{B}'$
alors $PX'=X$ (attention à l'ordre).
Et si $A$ est la matrice de $\phi$ dans la base $\mathcal{B}$ et $B$ est la matrice de $\phi$ dans la base
$\mathcal{B}'$ alors
$$B=P^{-1}AP$$
(Une matrice de passage entre deux bases est inversible.)

Ici on calcule l'inverse de $P$ :
$$P^{-1} = \begin{pmatrix}
1 & 1 & 0 \\
1 & 0 & 1 \\
1 & 1 & 1 \\    
\end{pmatrix}
\quad \text{ donc } \quad 
B=P^{-1}AP=\begin{pmatrix}
0 & 0 & 0 \\
0 & 1 & 0 \\
0 & 0 & 1 \\   
\end{pmatrix}
$$

On retrouve donc bien les mêmes résultats que précédemment.}
\end{enumerate}
}
