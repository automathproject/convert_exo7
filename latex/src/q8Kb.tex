\uuid{q8Kb}
\exo7id{6696}
\auteur{queffelec}
\organisation{exo7}
\datecreate{2011-10-16}
\isIndication{false}
\isCorrection{false}
\chapitre{Formule de Cauchy}
\sousChapitre{Formule de Cauchy}

\contenu{
\texte{
Soit $f$ une fonction holomorphe dans un ouvert connexe $\Omega$. On va montrer
l'équivalence entre 

(i) $f$ admet un logarithme holomorphe dans $\Omega$.

(ii)  $f$ admet des racines {\bf de tous ordres} holomorphes dans $\Omega$.

On a vu que (i) implique (ii).
Supposons maintenant que (ii) est vérifié : pour chaque $n$ on note $f_n$ la
fonction de $H(\Omega)$ telle que
$f_n^n(z)=f(z)$ si $z\in\Omega$.
}
\begin{enumerate}
    \item \question{Soit $a$ un zéro de $f$; que peut-on dire de la multiplicité de $a$ ? En
déduire que
$f$ ne s'annule pas sur
$\Omega$.}
    \item \question{Soit $\gamma$ une courbe fermée dans $\Omega$
de classe $C^1$ par morceaux. On pose $I={1\over 2\pi i}\int_\gamma
{f'(z)\over f(z)}\ dz$, et $I_n={1\over 2\pi i}\int_\gamma
{f_n'(z)\over f_n(z)}\ dz$. Montrer que $I$ et $I_n$ sont des entiers, $I=nI_n$,
puis que
$I=0$. Conclure.}
\end{enumerate}
}
