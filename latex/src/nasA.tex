\uuid{nasA}
\exo7id{5857}
\auteur{rouget}
\organisation{exo7}
\datecreate{2010-10-16}
\isIndication{false}
\isCorrection{true}
\chapitre{Topologie}
\sousChapitre{Application linéaire continue, norme matricielle}

\contenu{
\texte{
On munit $E=\mathcal{M}_n(\Rr)$ de la norme $N$ définie par $\forall A\in E$, $N(A)=\underset{1\leqslant i\leqslant n}{\text{Sup}}\left\{\sum_{j=1}^{n}|a_{i,j}|\right\}$ (on admet que $N$ est une norme sur $E$).

Soit $f$ l'application de $E$ dans $\Rr$ définie par $\forall A\in E$, $f(A)=\text{Tr}(A)$. Démontrer que l'application $f$ est continue sur $(E,N)$ et déterminer $|||f|||$.
}
\reponse{
L'application $f$ est linéaire de $(E,N)$ dans $(\Rr,|\;|)$. Soit $A=(a_{i,j})_{1\leqslant i,j\leqslant n}\in E$.

\begin{align*}\ensuremath
|f(A)|&=|\text{Tr}(A)|\leqslant\sum_{i=1}^{n}|a_{i,i}|\\
 &\leqslant\sum_{i=1}^{n}\left(\sum_{j=1}^{n}|a_{i,j}|\right)\leqslant\sum_{i=1}^{n}N(A)=nN(A).
\end{align*}
 

Ceci montre déjà que $f$ est continue sur $(E,N)$ et que $|||f|||\leqslant n$. De plus, si $A=I_n\neq0$, $ \frac{|f(A)|}{N(A)}= \frac{n}{1}=n$. Donc

\begin{center}
\shadowbox{
$f$ est continue sur $(E,N)$ et $|||f|||=n$.
}
\end{center}
}
}
