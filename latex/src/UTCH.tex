\uuid{UTCH}
\exo7id{6880}
\auteur{gammella}
\organisation{exo7}
\datecreate{2012-05-29}
\isIndication{false}
\isCorrection{true}
\chapitre{Analyse vectorielle}
\sousChapitre{Forme différentielle, champ de vecteurs, circulation}

\contenu{
\texte{
On donne le champ vectoriel
$$\vec{V}(x,y,z)=(y^2\cos x, 2y\sin x+e^{2z},2y e^{2z}).$$
}
\begin{enumerate}
    \item \question{Montrer que ce champ est un champ de gradient.}
    \item \question{Déterminer le potentiel $U(x,y,z)$ dont dérive ce champ sachant qu'il vaut $1$ à l'origine.}
    \item \question{Quelle est la circulation de ce champ
de $A(0,1,0)$ à $B( \frac{\pi}{2},3 ,0)$ ?}
\reponse{
On note $P(x,y,z)=y^2\cos x$, $Q(x,y,z)=2y\sin x+e^{2z}$ et
$R(x,y,z)=2ye^{2z}$. La forme
$\omega=Pdx+Qdy+Rdz$, naturellement associée au champ $\vec{V}(x,y,z)$,
est exacte puisqu'elle est définie sur $\Rr^3$ et 
\begin{enumerate}
$ \frac{\partial P}{\partial y}=  \frac{\partial Q}{\partial x}=2y\cos x$
$ \frac{\partial P}{\partial z}=  \frac{\partial R}{\partial x}= 0$
$ \frac{\partial Q}{\partial z}=  \frac{\partial R}{\partial y}=2e^{2z}.$
}
\end{enumerate}
}
