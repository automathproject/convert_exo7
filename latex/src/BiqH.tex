\uuid{BiqH}
\exo7id{2122}
\auteur{debes}
\organisation{exo7}
\datecreate{2008-02-12}
\isIndication{true}
\isCorrection{false}
\chapitre{Ordre d'un élément}
\sousChapitre{Ordre d'un élément}

\contenu{
\texte{
(a) Montrer que les seuls sous-groupes de $\Z $ sont de la forme $n\Z $ o\`u $n$
est un entier.
\smallskip

(b) Un \'el\'ement $x$ d'un groupe est dit d'ordre fini s'il existe un entier $k$ tel que
$x^k=e_G$. Montrer que $\{ k\in \Z \hskip 2pt |\hskip 2pt  x^k=e_G \} $ est alors un
sous-groupe non nul de $\Z$. On appelle ordre de $x$ le g\'en\'erateur positif de ce
sous-groupe.

\smallskip
(c) Soit $x$ un \'el\'ement d'un groupe $G$. Montrer que $x$ est d'ordre $d$ si et seulement
si le sous-groupe $< x >$ de $G$ engendr\'e par $x$ est d'ordre $d$.
}
\indication{Pour le (c), introduire le morphisme $\Z \rightarrow < x >$ qui associe $nx$ \`a tout
entier $n\in \Z$. Ce morphisme est surjectif et de noyau $d\Z$ o\`u $d$ est l'ordre de $x$.}
}
