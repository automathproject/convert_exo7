\uuid{LGKF}
\exo7id{3292}
\auteur{quercia}
\organisation{exo7}
\datecreate{2010-03-08}
\isIndication{false}
\isCorrection{true}
\chapitre{Polynôme, fraction rationnelle}
\sousChapitre{Fraction rationnelle}

\contenu{
\texte{
\item Soit $P\in\C[X]$ admettant deux racines distinctes et tel que $P''$
    divise $P$. Montrer que $P$ est {\`a} racines simples.

  \item Soit $P\in\R[X]$ admettant deux racines r{\'e}elles distinctes, et tel que
    $P''$ divise $P$. Montrer que $P$ est scind{\'e} sur~$\R$ et {\`a} racines simples.
}
\reponse{
On suppose $P\ne 0$.  Soient $\alpha_1,\dots,\alpha_p$ les
  racines de $P$ de multiplicit{\'e}s $m_1,\dots,m_p$ et
  $n=m_1+\dots+m_p=\deg(P)$.  On a $\frac{P'}P =
  \sum_i\frac{m_i}{X-\alpha_i}$ et $\sum_i\frac{-m_i}{(X-\alpha_i)^2}
  = \Bigl(\frac{P'}P\Bigr)' = \frac{P''}P - \Bigl(\frac{P'}P\Bigr)^2 =
  \frac{n(n-1)}{(X-\alpha)(X-\beta)} -
  \Bigl(\sum_i\frac{m_i}{X-\alpha_i}\Bigr)^2$ o{\`u} $\alpha,\beta$ sont
  les deux racines de~$P$ manquant dans~$P''$.  Si
  $\alpha_i\notin\{\alpha,\beta\}$ alors en comparant les termes en
  $1/(X-\alpha_i)^2$ des membres extr{\^e}mes on trouve $m_i=1$. De m{\^e}me
  si $\alpha_i=\alpha\ne\beta$ ou l'inverse. Reste le cas
  $\alpha_i=\alpha=\beta$ qui donne $-m_i = n(n-1) - m_i^2$ donc $m_i
  = n$ ce qui contredit l'hypoth{\`e}se ``$P$ a deux racines distinctes''.
Soient $\alpha_i<\alpha_j$ les deux plus petites racines r{\'e}elles
  de~$P$. Si $\alpha_i$ est aussi racine de $P''$ alors $P$ et $P''$
  changent de signe en $\alpha_i$ et, en rempla\c cant au besoin $P$
  par $-P$, $P$ est convexe positif sur $]-\infty,\alpha_i[$ et
  concave n{\'e}gatif sur $]\alpha_i,\alpha_j[$ ce qui est absurde.  Donc
  $\alpha_i\in\{\alpha,\beta\}$. De m{\^e}me pour la plus grande racine
  r{\'e}elle de~$P$, ce qui prouve que $\alpha$ et $\beta$ sont r{\'e}els. En
  identifiant les {\'e}l{\'e}ments de premi{\`e}re esp{\`e}ce dans les deux
  d{\'e}compositions de $P'/P$ on obtient~:
  $$\forall\ i\in{[[1,n]]},\ \sum_{j\ne i}\frac2{\alpha_i-\alpha_j} =
  \begin{cases}n(n-1)/(\alpha-\beta) & \text{ si } \alpha_i=\alpha,\cr
    n(n-1)/(\beta-\alpha) &\text{ si } \alpha_i=\beta,\cr 0 &\text{
      sinon.}\cr\end{cases}$$ En particulier pour
  $\alpha_i\notin\{\alpha,\beta\}$ on a~: $\sum_{j\ne
    i}\frac{\overline{\alpha_i}-\overline{\alpha_j}}{|\alpha_i-\alpha_j|^2}
  = 0$ ce qui signifie que $\overline{\alpha_i}$ est barycentre des
  $\overline{\alpha_j}$ avec des coefficients positifs, donc
  appartient {\`a} l'enveloppe convexe des $\overline{\alpha_j}$, $j\ne
  i$. Il en va de m{\^e}me sans les barres, et donc l'ensemble des racines
  de $P$ n'a pas d'autres points extr{\'e}maux que $\alpha$ et $\beta$~;
  il est inclus dans $[\alpha,\beta]$ donc dans~$\R$.
}
}
