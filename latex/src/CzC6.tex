\uuid{CzC6}
\exo7id{5682}
\auteur{rouget}
\organisation{exo7}
\datecreate{2010-10-16}
\isIndication{false}
\isCorrection{true}
\chapitre{Réduction d'endomorphisme, polynôme annulateur}
\sousChapitre{Diagonalisation}

\contenu{
\texte{
Soit $A=\left(
\begin{array}{cccc}
0&\ldots&0&a_1\\
\vdots& &\vdots&\vdots\\
0&\ldots&0&a_{n-1}\\
a_1&\ldots&a_{n-1}&a_n
\end{array}
\right)$ où $a_1$,..., $a_n$ sont $n$ nombres complexes ($n\geqslant2$). $A$ est-elle diagonalisable?
}
\reponse{
(Si les $a_k$ sont réels, la matrice $A$ est symétrique réelle et les redoublants savent que la matrice $A$ est diagonalisable.)

Si tous les $a_k$, $1\leqslant k\leqslant n-1$, sont nuls la matrice $A$ est diagonalisable car diagonale.

On suppose dorénavant que l'un au moins des $a_k$, $1\leqslant k\leqslant n-1$, est non nul. Dans ce cas, $\text{rg}A = 2$.

$0$ est valeur propre d'ordre $n-2$ au moins. Soient $\lambda$ et $\mu$ les deux dernières valeurs propres. On a

\begin{center}
$\lambda +\mu=\text{Tr}A=a_n$ et $\lambda^2 +\mu^2=\text{Tr}(A^2)=\sum_{k=1}^{n-1}a_k^2+\sum_{k=1}^{n}a_k^2 = 2\sum_{k=1}^{n-1}a_k^2 +a_n^2$.
\end{center}

$\lambda$ et $\mu$ sont solutions du système $\left\{
\begin{array}{l}
\lambda+\mu=a_n\\
\lambda^2+\mu^2=2\sum_{k=1}^{n-1}a_k^2 +a_n^2
\end{array}
\right.$ qui équivaut au système $\left\{
\begin{array}{l}
\lambda+\mu=a_n\\
\lambda^2+\mu^2=-\sum_{k=1}^{n-1}a_k^2
\end{array}
\right.$ $(S)$.

On a alors les situations suivantes :

\textbullet~Si $\lambda$ et $\mu$ sont distincts et non nuls, $A$ est diagonalisable car l'ordre de multiplicité de chaque valeur propre est égale à la dimension du sous-espace propre correspondant.

\textbullet~Si $\lambda$ ou $\mu$ est nul, $A$ n'est  pas diagonalisable car l'ordre de multiplicité de la valeur propre $0$ est différent de $n-2$, la dimension du noyau de $A$.

\textbullet~Si $\lambda=\mu\neq 0$, $A$ est diagonalisable si et seulement si $\text{rg}(A-\lambda I) = n-2$ mais on peut noter que si $\lambda$ n'est pas nul, on a toujours $\text{rg}(A-\lambda I)= n-1$ en considérant la matrice extraite formée des n-1 premières lignes et colonnes.

En résumé, la matrice $A$ est diagonalisable si et seulement si le système $(S)$ admet deux solutions distinctes et non nulles.

Mais $\lambda$ et $\mu$ sont solutions du système $(S)$ si et seulement si $\lambda$ et $\mu$ sont les racines de l'équation $(E)$ : $X^2 - a_nX- \sum_{k=1}^{n-1}a_k^2 = 0$. Par suite, $A$ est diagonalisable si et seulement si $\sum_{k=1}^{n-1}a_k^2 = 0$ et $\Delta= a_n^2+4\sum_{k=1}^{n-1}a_k^2\neq0$.
}
}
