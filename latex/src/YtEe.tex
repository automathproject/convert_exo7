\uuid{YtEe}
\exo7id{5843}
\auteur{rouget}
\organisation{exo7}
\datecreate{2010-10-16}
\isIndication{false}
\isCorrection{true}
\chapitre{Topologie}
\sousChapitre{Topologie de la droite réelle}

\contenu{
\texte{
Montrer qu'entre deux réels distincts, il existe un rationnel (ou encore montrer que $\Qq$ est dense dans $\Rr$).
}
\reponse{
\textbf{1ère solution.} \textbullet~Montrons qu'entre deux réels distincts, il existe un rationnel.

Soient $x$ et $y$ deux réels tels que $x < y$. Soient $d= y -x$ puis $n$ un entier naturel non nul tel que  $ \frac{1}{n}<d$ (par exemple, $n=E\left( \frac{1}{d}\right)+1$). Soient enfin $k = E(nx)$ et $r = \frac{k+1}{n}$. $r$ est un rationnel et de plus

\begin{center}
$x = \frac{nx}{n}< \frac{k+1}{n}=r\leqslant \frac{nx+1}{n}=x+ \frac{1}{n}<x+d=x+y-x = y$.
\end{center}

En  résumé, $\forall(x,y)\in\Rr^2$, $(x<y\Rightarrow\exists r\in\Qq/\;x<r<y)$. Ceci montre que $\Qq$ est dense dans $\Rr$.

\textbf{2ème solution.} On sait que tout réel est limite d'une suite de décimaux et en particulier tout réel est limite d'une suite de rationnels. Donc $\Qq$ est dense dans $\Rr$.

\begin{center}
\shadowbox{
$\Qq$ est dense dans $\Rr$.
}
\end{center}
}
}
