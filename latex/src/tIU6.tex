\uuid{tIU6}
\exo7id{7185}
\auteur{megy}
\organisation{exo7}
\datecreate{2017-07-26}
\isIndication{true}
\isCorrection{true}
\chapitre{Propriétés de R}
\sousChapitre{Autre}

\contenu{
\texte{
%[modifier pour se ramener à un cas simple]
Soient $a$, $b$ et $c$ des réels strictement positifs. Montrer que
\[ \frac{c}{a} + \frac{a}{b+c} +\frac{b}{c} \geq 2.
\]
}
\indication{Utiliser l'inégalité arithmético-géométrique.}
\reponse{
En appliquant l'inégalité arithmético-géométrique, on voit que les $a$ et $c$ se simplifient mais pas $b$ et $b+c$. L'idée est alors de modifier la forme de l'inégalité pour obtenir la simplification. Or, l'inégalité est équivalente à 
\[ \frac{c}{a} + \frac{a}{b+c} +\frac{b}{c}+1 \geq 3,
\]
c'est-à-dire à 
\[ \frac{c}{a} + \frac{a}{b+c} +\frac{b+c}{c} \geq 3.
\]
Sous cette forme, l'inégalité arithmético-géométrique donne le résultat:
\[ \frac{c}{a} + \frac{a}{b+c} +\frac{b+c}{c} \geq 
3\sqrt[3]{\frac{c}{a}  \frac{a}{b+c} \frac{b+c}{c}}
=3\sqrt[3]{1}=3.
\]
}
}
