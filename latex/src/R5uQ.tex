\uuid{R5uQ}
\exo7id{5328}
\auteur{rouget}
\organisation{exo7}
\datecreate{2010-07-04}
\isIndication{false}
\isCorrection{true}
\chapitre{Polynôme, fraction rationnelle}
\sousChapitre{Racine, décomposition en facteurs irréductibles}

\contenu{
\texte{
Déterminer $a\in\Cc$ tel que $P=X^5-209X+a$ admette deux zéros dont le produit vaut $1$.
}
\reponse{
$a$ est solution du problème si et seulement si $X^5-209X+a$ est divisible par un polynôme de la forme $X^2+\alpha X+1$. Mais 

$$X^5-209X+a=(X^2+\alpha X+1)(X^3-\alpha X^2+(\alpha^2-1)X-(\alpha^3-2\alpha))+(\alpha^4-3\alpha^2-208)X+a+(\alpha^3-2\alpha).$$

Donc a est solution $\Leftrightarrow\exists\alpha\in\Cc/\;\left\{
\begin{array}{l}
\alpha^4-3\alpha^2-208=0\\
a=-\alpha^3+2\alpha
\end{array}
\right.$. Mais, $\alpha^4-3\alpha^2-208=0\Leftrightarrow\alpha^2\in\{-13,16\}\Leftrightarrow\alpha\in\{-4,4,i\sqrt{13},-i\sqrt{13}\}$ et la deuxième équation fournit $a\in\{56,-56,15i\sqrt{13},-15i\sqrt{13}\}$.
}
}
