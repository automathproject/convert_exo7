\uuid{v78r}
\exo7id{6944}
\auteur{ruette}
\datecreate{2013-01-24}
\isIndication{false}
\isCorrection{true}
\chapitre{Loi, indépendance, loi conditionnelle}
\sousChapitre{Loi, indépendance, loi conditionnelle}

\contenu{
\texte{
Soit $(X_n)_{n\geq 1}$ une suite de variables aléatoires indépendantes de même loi de
Bernoulli de paramètre $p$. On définit la variable aléatoire $T_1$ à valeurs dans
$\Nn\cup\{+\infty\}$ par $T_1=\inf\{k>0 \mid X_k=1\}$, avec la
convention $\inf(\emptyset)=+\infty$.\\ \textit{(si on joue à pile $=0$ ou
 face $=1$,
$T_1$ est le temps nécessaire pour obtenir face une première fois)}
}
\begin{enumerate}
    \item \question{Montrer que $T_1$ est fini presque sûrement.}
\reponse{$\{\omega\mid T_1(\omega)=+\infty\}=\{\omega\mid X_n(\omega) =0
\mbox{ pour tout } n\geq 1\}\subset \{\omega\mid X_n(\omega) =0 \mbox{ pour }
1\leq n\leq N\}$. Donc
$\displaystyle
P(T_1=+\infty)\leq P(X_n=0\mbox{ pour } 1\leq n\leq N)=\prod_{n=1}^N P(X_n=0)
\ \mbox{ par indépendance}
$
d'où $P(T_1=+\infty)\leq (1-p)^N$. Or $1-p\in ]0,1[$ et $N$
est arbitrairement grand, donc $P(T_1=+\infty)=0$, autrement dit $T_1$
est fini presque sûrement.}
    \item \question{Déterminer la loi et l'espérance de $T_1$ \textit{(cette loi est appelée loi géométrique, $E(T_1)$ est le nombre
moyen de lancers qu'il faut effectuer pour obtenir face une première fois)}.}
\reponse{Pour tout $k\in\Nn^*$, $\{T_1=k\}=\{X_1=0, X_2=0,\ldots, X_{k-1}=0, X_k=1\}$ donc
par indépendance $P(T_1=k)=p(1-p)^{k-1}$.

$\displaystyle
E(T_1)=\sum_{k\geq 1} k P(T_1=k)=p\sum_{k\geq 1} k(1-p)^{k-1}$.
On reconnaît une série dérivée. Si on introduit la série
$\displaystyle f(x)=\sum_{k\geq 0} x^k$ (bien définie si $|x|<1$), alors 
$E(T_1)=p f'(1-p)$. Or
$f(x)=\frac{1}{1-x}$, donc $f'(x)=\frac{1}{(1-x)^2}$. D'où
$E(T_1)=p f'(1-p)=\frac{1}{p}$.}
    \item \question{Pour tout $n\geq 2$, on définit par récurrence
$T_n=\inf\{k>T_{n-1}\mid X_k=1\}$.\\
\textit{(si on joue à pile  ou face, $T_n$
est le temps nécessaire pour obtenir exactement $n$ fois face)}\\
Montrer que les variables aléatoires $T_1,\ (T_2-T_1), \ldots,\ (T_n-T_{n-1}),\ldots$
sont indépendantes et de même loi.}
\reponse{Montrons par récurrence que $T_n$ est finie presque sûrement
(hypothèse vérifiée par $T_1$ par a) et que les variables aléatoires
$T_1, (T_2-T_1),\ldots (T_n-T_{n-1}),\ldots$
ont la même loi que $T_1$ (hypothèse évidemment vérifiée par $T_1$).

Supposons que $T_{n-1}$ est finie presque sûrement et montrons que $T_n-T_{n-1}$
a la même loi que $T_1$ et que $T_n$ est finie
presque sûrement
Si $T_{n-1}=j$ alors $\{T_n=j+k\}=\{X_{j+1}=0,\ldots,X_{j+k-1}=0,
X_{j+k}=1\}$. Comme $T_{n-1}$ est finie presque sûrement, la formule des 
probabilités conditionnelles s'applique et donne :
\begin{eqnarray*}
P(T_n-T_{n-1}=k)&=&\sum_{j\in\Nn}P(T_n-T_{n-1}=k\mid T_{n-1}=j)
P(T_{n-1}=j)\\
&=&\sum_{j\in\Nn}P(X_{j+1}=0,\ldots,X_{j+k-1}=0,
X_{j+k}=1)P(T_{n-1}=j)\\
&=&\sum_{j\in\Nn}p(1-p)^{k-1}P(T_{n-1}=j)\ \mbox{ par indépendance}\\
&=&p(1-p)^{k-1} \sum_{j\in\Nn}P(T_{n-1}=j)
\end{eqnarray*}
Or 
$\displaystyle\quad \sum_{j\in\Nn}P(T_{n-1}=j)=P(T_{n-1}<+\infty)=1 \quad\mbox{(hypothèse
de récurrence pour $n-1$)},$ \\
donc
$P(T_n-T_{n-1}=k)=(1-p)^{k-1}p$. On vérifie que
$\displaystyle\quad \sum_{k\in\Nn}P(T_n-T_{n-1}=k)=1,$ 
donc $T_n-T_{n-1}<+\infty$
presque sûrement, et par conséquent $T_n<+\infty$ presque sûrement
(on peut aussi calculer directement $P(T_n-T_{n-1}=+\infty)=0$, comme
pour $T_1$). On vient de montrer que $T_n-T_{n-1}$ a la même loi que $T_1$.
Ceci termine la récurrence.

\medskip
Pour montrer l'indépendance, on regarde $\{T_1=k_1, T_2-T_1=k_2,\ldots, 
T_n-T_{n-1}=k_n\}$, ce qui décrit exactement les valeurs de $X_i$ pour
$0\leq i\leq k_1+k_2+\cdots+k_n $. C'est un peu lourd à écrire :
%\medskip
$$
\{T_1=k_1, T_2-T_1=k_2,\ldots, T_n-T_{n-1}=k_n\}=
\{\begin{array}[t]{l}
X_0=\cdots=X_{k_1-1}=0, X_{k_1}=1,\\ 
X_{k_1+1}=\cdots=X_{k_1+k_2-1}=0, X_{k_1+k_2}=1,\\
\cdots\\ 
X_{k_1+\cdots+k_{n-1}+1}=\cdots=X_{k_1+\cdots+k_n-1}=0,
X_{k_1+\cdots+k_n}=1\ \}\end{array}$$

\medskip
D'où par indépendance des $X_i$ :
\begin{eqnarray*}
P(T_1=k_1, T_2-T_1=k_2,\ldots, T_n-T_{n-1}=k_n)
&=&p^n(1-p)^{(k_1-1)+\cdots+(k_n-1)}\\
&=&P(T_1=k_1)P(T_2-T_1=k_2)\cdots P(T_n-T_{n-1}=k_n)
\end{eqnarray*}
Conclusion : les variables aléatoires $T_1, T_2-T_1,\ldots, T_n-T_{n-1}$ sont indépendantes.}
    \item \question{Quelle est la loi de $T_n$ ?}
\reponse{Ce qui précède permet de voir que, pour avoir
$T_n=k$, d'une part il faut que $X_k=1$, et d'autre part parmi les
$(X_i)_{1\leq i\leq k-1}$ il faut que $n-1$ soient égaux à $1$ et les autres
soient égaux à $0$. Il y a $C^{n-1}_{k-1}$ façons de choisir les $1$ (à 
condition bien sûr que $k\leq n$), et chaque
combinaison a la même probabilité, qui est $p^n(1-p)^{k-n}$. 
On peut en déduire que la loi de
$T_n$ est $P(T_n=k)=C^{n-1}_{k-1} p^n(1-p)^{k-n}$ si $k\geq n$ et
$P(T_n=k)=0$ si $k<n$.}
\end{enumerate}
}
