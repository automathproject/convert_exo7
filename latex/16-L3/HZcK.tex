\uuid{HZcK}
\exo7id{2681}
\auteur{matexo1}
\organisation{exo7}
\datecreate{2002-02-01}
\isIndication{false}
\isCorrection{true}
\chapitre{Fonction logarithme et fonction puissance}
\sousChapitre{Fonction logarithme et fonction puissance}

\contenu{
\texte{
Tout complexe $z$ qui n'est pas un r{\'e}el positif ou nul peut s'{\'e}crire sous la
forme $z = r e^{i \theta}$, avec $r > 0$ et~$\theta\in\left]0,2\pi \right[$. 
On d{\'e}finit une fonction $f$
sur $\Omega = \C\setminus[0,+\infty [$ par
$$ f(z) = P(r, \theta)+iQ(r,\theta) $$
o{\`u} $P$ et $Q$ sont des fonctions r{\'e}elles donn{\'e}es. Donner des conditions sur les
d{\'e}riv{\'e}es partielles de $P$, $Q$ pour que $f$ soit holomorphe sur $\Omega $.

En d{\'e}duire que la fonction
$$ \log z = \log r + i\theta $$
est holomorphe sur $\Omega $. Quelle est sa d{\'e}riv{\'e}e ?
}
\reponse{
En utilisant les relations de Cauchy, on trouve ais{\'e}ment
$$ {\partial P\over \partial r} = {1\over r}\,{\partial Q\over \partial \theta} \qquad{\rm et} \qquad
{\partial Q\over \partial r} = -{1\over r}\,{\partial P\over \partial \theta} $$
Ces conditions sont v{\'e}rifi{\'e}es par la fonction $\log$. Sa
d{\'e}riv{\'e}e est $1/z$.
}
}
