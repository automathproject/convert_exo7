\uuid{SdFM}
\exo7id{2870}
\auteur{burnol}
\datecreate{2009-12-15}
\isIndication{false}
\isCorrection{false}
\chapitre{Autre}
\sousChapitre{Autre}

\contenu{
\texte{
Le but de ce problème est d'établir les deux formules importantes:
\[ \forall z\in\Cc\setminus\Zz\quad \frac\pi{\sin \pi z} = \frac1{z} +
\lim_{\substack{N\to+\infty \\ M\to+\infty}} 
\sum_{\substack{-M\leq n \leq N \\ n\neq0}}
\frac{(-1)^n}{z-n} = \frac1z + \sum_{n=1}^\infty (-1)^n\frac{2z}{z^2-n^2}\] 
\[\forall z\in\Cc\qquad \sin(\pi z) = \lim_{N\to\infty} \pi z\prod_{n=1}^N (1 -
\frac{z^2}{n^2})\]
}
\begin{enumerate}
    \item \question{Montrer la convergence de la série
$\sum_{n=1}^\infty \frac{(-1)^n}{z-n}$ (regarder les sommes
partielles pour les indices pairs).}
    \item \question{On pose $f(w) =  \frac\pi{\sin \pi
w}$. Soit $z\notin\Zz$ fixé, soit $N>|z|-\frac12$ et ${\cal R}_N$ le carré
$\{|x|\leq N+\frac12$, $|y|\leq N+\frac12\}$, et $C_N =
\partial {\cal R}_N$ son
bord parcouru dans le sens direct. Exprimer $\frac1{2\pi i}
\int_{C_N} \frac{f(w)}{w-z}dw$ à l'aide du Théorème
des résidus.}
    \item \question{Montrer $\int_{C_N} \frac{f(w)}w dw = 0$ (on notera que $f$ est
impaire) et en déduire :
\[ \frac1{2\pi i} \int_{C_N} \frac{f(w)}{w-z}dw =
\frac1{2\pi i} \int_{C_N} \frac\pi{\sin \pi 
w}\frac{z}{w(w-z)}dw\]}
    \item \question{On rappelle l'identité $\sin(w) = \sin(x)\ch(y) + i
\cos(x)\sh(y)$ pour $w=x+iy$. Montrer $|\sin w|^2 = \sin^2
x + \sh^2 y$ ($x,y\in\Rr\dots$). En déduire $|\sin(\pi w)| = \ch(\pi
y)\geq 1$ sur les bords verticaux du carré et $|\sin(\pi w)|
\geq \sh(\pi(N+\frac12))\geq \sh(\pi\frac12) =
2.301\dots\geq1$ sur les bords horizontaux. 
Conclure la preuve de
\[ \frac\pi{\sin \pi z} = \frac1z + \sum_{n=1}^\infty
(-1)^n\frac{2z}{z^2-n^2}\]  
avec convergence uniforme pour $|z|$ borné.}
    \item \question{Reprendre la même technique et prouver:
\[ \forall z\in\Cc\setminus\Zz\quad\frac{\pi \cos(\pi
z)}{\sin(\pi z)} = \lim_{N\to\infty} 
\sum_{-N\leq n\leq N} \frac1{z-n} = \frac1z +
\sum_{n=1}^\infty \frac{2z}{z^2-n^2}\;,\]
avec convergence uniforme pour $|z|$ borné.}
    \item \question{On veut maintenant prouver: $\sin(\pi z) = \lim_{N\to\infty}
\pi z\prod_{n=1}^N (1 - 
\frac{z^2}{n^2})$
%
On fixe une fois pour toutes $R>0$, et on va
montrer la formule pour $|z|<R$. 
Soit $N$ avec $N>R$ et notons $f_N(z) =
\frac{\pi z}{\sin(\pi z)}\prod_{n=1}^N (1 - \frac{z^2}{n^2})$, prolongé
par continuité en les $n$, $|n|\leq N$. Montrer que $f_N$
est holomorphe et ne s'annule pas sur $D(0,R)$.}
    \item \question{Soit $\gamma:[0,1]\to\Cc^*$ le chemin $\gamma(t) =
f_N(tz)$. On a donc $\gamma(0) = 1$, $\gamma(1) = f_N(z)$,
et $\gamma(t)\neq0$ pour tout $t$. Par un théorème démontré
en cours (lequel?)
on a 
$\gamma(1) = \gamma(0)\exp\left(\int_\gamma \frac{dw}w\right) $.
En déduire $f_N(z) = \exp\left(\int_0^1 \frac{f_N'(tz)}{f_N(tz)}
zdt\right)$.}
    \item \question{Soit $\epsilon>0$. En utilisant la convergence uniforme pour
 $|z|$ borné
 du développement en fractions de $\pi\cotg(\pi z)$,
montrer que pour $N$ suffisamment grand, on a $|f_N'(w)|\leq
 \epsilon |f_N(w)|$ pour tout $w\in D(0,R)$, puis en déduire
\[ N\gg0\quad |z|<R \implies\quad |f_N(z)| \leq e^{\epsilon|z|} \leq
 e^{\epsilon R} \]}
    \item \question{En déduire $\lim_{N\to\infty} f_N(z) = 1$,
uniformément sur $D(0,R)$. Conclure la
preuve du produit infini de Euler pour $\sin(z)$.}
\end{enumerate}
}
