\uuid{JetL}
\exo7id{7532}
\auteur{mourougane}
\organisation{exo7}
\datecreate{2021-08-10}
\isIndication{false}
\isCorrection{false}
\chapitre{Fonction holomorphe}
\sousChapitre{Fonction holomorphe}

\contenu{
\texte{

}
\begin{enumerate}
    \item \question{Montrer qu'une fonction holomorphe sur un ouvert $D$ de $\Cc$ qui ne prend que des valeurs réelles
est localement constante.}
    \item \question{Que dire d'une fonction holomorphe sur un ouvert de $\Cc$ dont la partie réelle est constante ?}
    \item \question{Que dire d'une fonction holomorphe $f=u+iv$ sur un ouvert de $\Cc$ dont la conjuguée $\overline{f}:=u-iv$ 
est aussi holomorphe ?}
    \item \question{Montrer qu'une fonction holomorphe qui ne prend que des valeurs de module $1$ est localement constante.}
\indication{Pour la dernière question, on pourra montrer que $u\frac{\partial u}{\partial x}+v\frac{\partial v}{\partial x}=0$
puis que $u(u\frac{\partial u}{\partial x}+v\frac{\partial v}{\partial x})=\frac{\partial u}{\partial x}$.}
\end{enumerate}
}
