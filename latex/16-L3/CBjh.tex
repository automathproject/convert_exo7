\uuid{CBjh}
\exo7id{6742}
\auteur{queffelec}
\organisation{exo7}
\datecreate{2011-10-16}
\isIndication{false}
\isCorrection{false}
\chapitre{Théorème des résidus}
\sousChapitre{Théorème des résidus}

\contenu{
\texte{

}
\begin{enumerate}
    \item \question{Calculer les résidus aux différents p\^oles de la fonction $f(z)={z^2+z+1\over
z(z^2+1)^2};\ f(z)=\displaystyle{1\over 1+z+\cdots+z^{n-1}};\ f(z)={1\over
e^z-1}-{1\over z}$.}
    \item \question{Montrer que si $f(z)=(z-a)^{-n} g(z)$ où $g$ est holomorphe dans $\Omega$
ouvert contenant $a$, alors Res$(f,a)=\displaystyle{g^{(n-1)}(a)\over (n-1)!}$. 
Trouver les p\^oles et résidus des fonctions suivantes :
$\displaystyle{1-\cos z\over z^3},\ \displaystyle{e^{2z}\over (z-1)^3},\
\displaystyle{1\over (1+z^2)^n}\ $.}
\end{enumerate}
}
