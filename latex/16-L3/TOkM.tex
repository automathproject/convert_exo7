\uuid{TOkM}
\exo7id{7562}
\auteur{mourougane}
\organisation{exo7}
\datecreate{2021-08-10}
\isIndication{false}
\isCorrection{false}
\chapitre{Théorème des résidus}
\sousChapitre{Théorème des résidus}

\contenu{
\texte{

}
\begin{enumerate}
    \item \question{Soit $r>0$ et $D$ un voisinage ouvert de $\overline{\Delta_r}$.
 Soit $f : D\to\Cc$ une fonction holomorphe. Soit $a$ et $b$ deux points distincts dans $\Delta_r$.
 Calculer $$\int_{\partial\Delta_r}\frac{f(\zeta)}{(\zeta-a)(\zeta-b)}d\zeta.$$}
    \item \question{En déduire le théorème de Liouville : Toute fonction holomorphe bornée sur $\Cc$ est constante.}
\end{enumerate}
}
