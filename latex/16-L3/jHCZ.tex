\uuid{jHCZ}
\exo7id{6808}
\auteur{gijs}
\organisation{exo7}
\datecreate{2011-10-16}
\isIndication{false}
\isCorrection{false}
\chapitre{Théorème des résidus}
\sousChapitre{Théorème des résidus}

\contenu{
\texte{
Soit $\alpha$ un réel tel que $-1< \alpha < 2$, et soit
$f: \Cc\setminus\{it\mid t\in \,]-\infty,0]\,\}$ définie
par $$f(z) =   \frac{e^{iz} e^{\alpha\log z}}{1+z^2}\ ,$$ 
où $\log z$ désigne la branche uniforme
du logarithme complexe qui est réelle pour $z$ réel
strictement positif, avec $-\pi/2 < \arg z < 3\pi/2$.
}
\begin{enumerate}
    \item \question{Montrer que, pour tout $\theta$ tel que $0\le
\theta \le \pi/2$, on a~: $0\le 2\theta/\pi \le \sin
\theta \le \theta$.}
    \item \question{Soit $\gamma_\epsilon$ le demi-cercle de rayon
$\epsilon>0$, de centre 0, situé dans le demi-plan
$\Im\  z\ge 0$. Démontrer que
$ \lim_{\epsilon\downarrow 0} \int_{\gamma_\epsilon}
f(z)\,dz = 0$.}
    \item \question{Soit $\gamma_R$ le demi-cercle de rayon
$R>0$, de centre 0, situé dans le demi-plan $\Im 
z\ge 0$. Démontrer que $ \lim_{R\to \infty}
\int_{\gamma_R} f(z)\,dz = 0$.}
    \item \question{En intégrant $f$ sur le bord du domaine $\epsilon\le
 |z| \le R$, $0\le \arg(z) \le \pi$, déduire de ce
qui précède que l'intégrale $$I(\alpha) =
  \int_0^\infty \frac{x^\alpha
\cos(x-\frac{\alpha\pi}2)}{1+x^2}\,dx$$ est convergente et
en même temps calculer sa valeur.}
\end{enumerate}
}
