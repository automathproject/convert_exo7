\uuid{mJM5}
\exo7id{2809}
\auteur{burnol}
\datecreate{2009-12-15}
\isIndication{false}
\isCorrection{false}
\chapitre{Formule de Cauchy}
\sousChapitre{Formule de Cauchy}

\contenu{
\texte{
On veut exprimer le Laplacien avec les coordonnées
  polaires $r$ et $\theta$: autrement dit pour toute
  fonction deux fois différentiable $\Phi$ on veut calculer
  la fonction $\Delta(\Phi)$ à l'aide des opérateurs de
  dérivées partielles $\frac\partial{\partial r}$ et
  $\frac\partial{\partial
 \theta}$, lorsque l'on travaille sur un ouvert (ne
 contenant pas l'origine) dans lequel une détermination
 continue de l'argument $\theta$ est possible (par exemple
 sur $\Omega = \Cc\setminus]-\infty,0]$). Une méthode
 possible est d'utiliser les expressions obtenues dans
 l'exercice \ref{exo:CRpolaires} :
\[ \frac\partial{\partial x} =
 \cos(\theta)\frac\partial{\partial r} - \sin(\theta)\frac1r
 \frac\partial{\partial\theta}\qquad \frac\partial{\partial y} =
 \sin(\theta)\frac\partial{\partial r} + \cos(\theta)\frac1r
 \frac\partial{\partial\theta}\;,\]
et de calculer ensuite $(\frac\partial{\partial x})^2$ et
 $(\frac\partial{\partial y})^2$ puis de faire la
 somme. Mais cela donne des calculs un peu longs. Voici une
 ruse: en reprenant une formule déjà établie dans
 l'exercice \ref{exo:CRpolaires} montrer
\[ (x-iy)\left(\frac\partial{\partial x} +
 i\frac\partial{\partial y}\right) = r\frac\partial{\partial r} + 
 i\frac\partial{\partial \theta}\]
\[ (x+iy)\left(\frac\partial{\partial x} -
 i\frac\partial{\partial y}\right) = r\frac\partial{\partial r} - 
 i\frac\partial{\partial \theta}\]
On remarquera maintenant que l'opérateur différentiel
 $\frac\partial{\partial x} + 
 i\frac\partial{\partial y}$ appliqué à la fonction $x+iy$
 donne zéro. Donc (expliquer!):
\[ (x-iy)\left(\frac\partial{\partial x} +
 i\frac\partial{\partial y}\right)(x+iy)\left(\frac\partial{\partial x} -
 i\frac\partial{\partial y}\right) =
 (x-iy)(x+iy)\left(\frac\partial{\partial x} + 
 i\frac\partial{\partial y}\right)\left(\frac\partial{\partial x} -
 i\frac\partial{\partial y}\right)\]
Prouver alors en conclusion:
\[ \frac{\partial^2\Phi}{\partial x^2}
+\frac{\partial^2\Phi}{\partial y^2} = \frac{1}{r^2}\left(
 (r\frac\partial{\partial r})^2 + \frac{\partial^2}{\partial
 \theta^2}\right)\Phi
= \frac{\partial^2\Phi}{\partial r^2} + \frac1r
 \frac{\partial\Phi}{\partial
 r} +  \frac1{r^2}\frac{\partial^2\Phi}{\partial 
 \theta^2}.\]
}
}
