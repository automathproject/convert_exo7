\uuid{shC7}
\exo7id{2866}
\auteur{burnol}
\organisation{exo7}
\datecreate{2009-12-15}
\isIndication{false}
\isCorrection{false}
\chapitre{Théorème des résidus}
\sousChapitre{Théorème des résidus}

\contenu{
\texte{
Prouver, pour tout $x\in \Rr$: 
\[\frac1{2\pi}\int_\Rr e^{-i\xi x}(\pi e^{-|\xi|})d\xi =
\frac1{1+x^2}\;.\] Il suffit d'évaluer séparément
$\int_{-\infty}^0$ et $\int_0^\infty$ en utilisant le fait
que $\exp$ est sa propre primitive (ce calcul n'utilise donc
pas la notion de fonction analytique et le théorème des
résidus). On remarquera que l'on retombe sur la fonction
$1/(1+x^2)$, ce qui n'est pas un hasard (formule d'inversion
pour les transformations intégrales de Fourier).
}
}
