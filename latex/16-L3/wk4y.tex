\uuid{wk4y}
\exo7id{2859}
\auteur{burnol}
\datecreate{2009-12-15}
\isIndication{false}
\isCorrection{true}
\chapitre{Théorème des résidus}
\sousChapitre{Théorème des résidus}

\contenu{
\texte{
\label{ex:burnol2.2}
 Soit $f$ une fonction définie
et continue sur le domaine $\{\Im(z)>0, |z|>R\}$, ou
seulement sur une suite de demi-cercles $\{\Im(z)>0,
|z|=R_n\}$ de rayons tendant vers l'infini. On
suppose $\lim_{\substack{|z|\to\infty \\ \Im(z)>0}} |f(z)| = 0$
(ou $\displaystyle\lim_{n\to\infty}
\sup_{\Im(z)>0,\,|z|=R_n} |f(z)| = 0$.)  Montrer (on
utilisera  $\sin(\theta)\geq \frac2\pi \theta$ pour
$0\leq\theta\leq\frac\pi2$):
\[ \lim_{R\to\infty} \int_{z = R e^{i\theta},\,
0<\theta<\pi} f(z)e^{iz}\,dz = 0\qquad (\text{ou l'analogue avec
les\ }R_n)\]
}
\reponse{
Pour $\epsilon >0$ il existe $R> 0$ tel que $|f(z)|\leq \epsilon $ pour tout $|z|\geq R$, $\Im z \geq 0$. Si $C_R$ est le
demi-cercle sup\'erieur orient\'e alors
\begin{eqnarray*}
& &|\int _{C_R} f(z) e^{i z} \, dz | =|\int _0^\pi f(Re^{i\theta} ) e^{i(R e^{i\theta})} iR e^{i\theta } \, d\theta |
\leq \epsilon \int _0^\pi e^{-R \sin \theta } R \, d\theta\\
&& \qquad = 2\epsilon \int _0^{\frac{\pi}{2}} Re^{-R \sin \theta } \, d\theta
\leq 2\epsilon \int _0^{\frac{\pi}{2}} Re^{-R \frac{\theta}{2} }\, d\theta
= 4\epsilon (1-e^{-R \frac{\pi}{4}}) \leq 8 \epsilon .
\end{eqnarray*}
}
}
