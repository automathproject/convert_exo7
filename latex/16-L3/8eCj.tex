\uuid{8eCj}
\exo7id{7599}
\auteur{mourougane}
\datecreate{2021-08-10}
\isIndication{false}
\isCorrection{true}
\chapitre{Autre}
\sousChapitre{Autre}

\contenu{
\texte{

}
\begin{enumerate}
    \item \question{Paramétrer le cercle unité parcouru dans le sens trigonométrique.}
    \item \question{Calculer à l'aide du paramétrage précédent $\int_{\partial\Delta} \frac{\cosh z}{z} dz$.}
    \item \question{La fonction $\frac{\cosh(z)}{z}$ admet-elle une primitive sur $\Cc^\times$ ? Si oui, l'expliciter.

%}
    \item \question{Reprendre les questions précédentes avec la fonction $\frac{\sinh (z)}{z}$.}
\reponse{
L'application 
$$\begin{array}{cccc}
  \gamma :& [0,2\pi[ &\longrightarrow & \Cc\\ & \theta &\longmapsto& e^{i\theta}
 \end{array}$$
est un paramétrage du cercle unité du plan complexe euclidien parcouru dans le sens trigonométrique.
Comme le rayon de convergent de la série qui définit $\cosh$ est infini, 
la série de fonctions $\sum_{n=0}^{+\infty} \frac{z^{2n-1}}{(2n)!}$ est normalement donc uniformément convergente sur le cercle unité vers
la fonction $\frac{\cosh z}{z}$.
Par conséquent, 
\begin{eqnarray*}\int_{\partial\Delta} \frac{\cosh z}{z} dz&=&\sum_{n=0}^{+\infty} \int_{\partial\Delta}\frac{z^{2n-1}}{(2n)!}
=\sum_{n=0}^{+\infty} \int_0^{2\pi}\frac{(\gamma(\theta))^{2n-1}}{(2n)!}\gamma'(\theta)d\theta\\
&=&\sum_{n=0}^{+\infty} \int_0^{2\pi} i e^{i(2n)\theta} d\theta=2i\pi.
\end{eqnarray*}
Comme l'intégrale de la fonction $\frac{\cosh(z)}{z}$ sur le chemin fermé $\partial\Delta$ de $\Cc^\times$ n'est pas nulle,
la fonction $\frac{\cosh(z)}{z}$ n'admet pas de primitive sur $\Cc^\times$.
}
\end{enumerate}
}
