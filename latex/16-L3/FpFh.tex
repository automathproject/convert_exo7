\uuid{FpFh}
\exo7id{7627}
\auteur{mourougane}
\organisation{exo7}
\datecreate{2021-08-10}
\isIndication{false}
\isCorrection{true}
\chapitre{Autre}
\sousChapitre{Autre}

\contenu{
\texte{

}
\begin{enumerate}
    \item \question{Rappeler la définition de la branche principale du logarithme.}
\reponse{$$\begin{array}{cccc}
 \log :& \Cc-]-\infty,0]&\longrightarrow&\Cc\\
    &z         &\longmapsto& \ln_\Rr(r)+i\theta \ \ \substack{\textrm{ après avoir choisi l'écriture } z=re^{i\theta}\\ \textrm{ avec } r\in[0,+\infty[ \textrm{ et } \theta\in]-\pi,\pi[}
\end{array}$$

% $$\newfrac{\ln_\Rr(r)+i\theta}{\textrm{après avoir choisi l'écriture } z=re^{i\theta} \textrm{ avec } r\in[0,+\infty[ \textrm{ et } \theta\in]-\pi,\pi[}$$}
    \item \question{Rappeler la définition d'une primitive d'une fonction continue $f$ sur $\Cc$.}
\reponse{Une primitive d'une application continue $f$ sur $\Cc$ est une application holomorphe $F$ sur $\Cc$ dont la dérivée complexe $F'$ est l'application $f$.}
    \item \question{Donner si possible l'exemple d'une fonction continue non holomorphe $f$ sur $\Cc$ qui admet une primitive.}
\reponse{Comme la dérivée complexe d'une application holomorphe est holomorphe, il n'y a pas de tels exemples.}
    \item \question{Donner si possible l'exemple d'une fonction holomorphe non constante de $\Delta$ dans $\Cc$. Dans le cas contraire, démontrer la non-existence.}
\reponse{$$\begin{array}{cccc}
 Id :& \Delta &\longrightarrow&\Cc\\
    &z         &\longmapsto& z
\end{array}$$
est holomorphe non-constante.}
    \item \question{Donner si possible l'exemple d'une fonction holomorphe non constante de $\Cc$ dans $\Delta$. Dans le cas contraire, démontrer la non-existence.}
\reponse{Par le théorème de Liouville, toute application entière bornée est constante~: il n'y a donc pas de tels exemples.}
    \item \question{Donner si possible l'exemple d'une fonction holomorphe non constante de $\mathbb{H}$ dans $\Delta$. Dans le cas contraire, démontrer la non-existence.}
\reponse{$$\begin{array}{cccc}
 C :& \mathbb{H} &\longrightarrow&\Delta\\
    &z         &\longmapsto& \frac{z-i}{z+i}
\end{array}\textrm{ ou bien } 
\begin{array}{cccc}
 e :& \mathbb{H} &\longrightarrow&\Delta\\
    &z         &\longmapsto& \exp(2i\pi z)
\end{array}
$$}
\end{enumerate}
}
