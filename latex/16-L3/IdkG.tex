\uuid{IdkG}
\exo7id{7636}
\auteur{mourougane}
\datecreate{2021-08-10}
\isIndication{false}
\isCorrection{true}
\chapitre{Autre}
\sousChapitre{Autre}

\contenu{
\texte{
On considère le polynôme $P(z)=z^4+6z+3$.
}
\begin{enumerate}
    \item \question{Démontrer que $P$ a ses quatre racines dans le disque $\Delta_2$.}
\reponse{Sur le bord de $\Delta_2$
$$|P(z)-z^4|=|6z+3|\leq 6\times 2+3=15<16=2^4=|z|^4.$$
Par le théorème de Rouché, $P$ de degré $4$ a donc, comme $z\mapsto z^4$ ses quatre racines dans $\Delta_2$.}
    \item \question{Démontrer que $P$ n'admet qu'une racine dans $\Delta$.}
\reponse{Sur le bord de $\Delta$
$$|P(z)-(6z+3)|=|z^4|=1< 6-3\leq |6z|-3\leq |6z+3|.$$
Par le théorème de Rouché, le polynôme $P$ n'a, comme $z\mapsto 6z+3$ qu'une racine dans $\Delta$.}
    \item \question{Démontrer que $P$ n'admet pas de racine dans $\Delta_{\frac{1}{3}}$.}
\reponse{Sur le bord de $\Delta_{\frac{1}{3}}$
$$|P(z)-(6z+3)|=|z^4|=\frac{1}{81}<3-6/3=3-|6z|\leq |6z+3|.$$
Par le théorème de Rouché, le polynôme $P$ n'a, comme $z\mapsto 6z+3$ aucune racine dans $\Delta_{\frac{1}{3}}$.}
    \item \question{Soit $a$ la racine de $P$ dans le disque $\Delta$. Démontrer que
$$ a=\frac{1}{2i\pi}\int_{\partial\Delta} \frac{4z^3+6}{z^4+6z+3}zdz.$$}
\reponse{Par la question $2$, $z\mapsto \frac{4z^3+6}{z^4+6z+3}z$ est méromorphe sur $\Delta_{1+\epsilon}$ avec comme seul pôle un point $a$.
Le résidu en $a$ de $P'/P$ est la multiplicité $1$ de $a$ comme zéro de $P$.
Par conséquent, le résidu en $a$ de $\frac{4z^3+6}{z^4+6z+3}z$ est $a$.
La formule demandée résulte donc du théorème des résidus.}
\end{enumerate}
}
