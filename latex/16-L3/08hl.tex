\uuid{08hl}
\exo7id{7247}
\auteur{megy}
\organisation{exo7}
\datecreate{2021-03-06}
\isIndication{false}
\isCorrection{false}
\chapitre{Formule de Cauchy}
\sousChapitre{Formule de Cauchy}

\contenu{
\texte{
[Extension de \(\Gamma\)]
Notons \(U_{\Re>0}:=\left\{z\in \C\ ; \ \Re(z)>0\right\}\).
}
\begin{enumerate}
    \item \question{Montrer que pour tout \(z\in U_{\Re>0}\), on a 
\[\Gamma(z+1)=z\Gamma(z).\]}
    \item \question{En déduire que \(\Gamma(n)=(n-1)!\) pour tout \(n\in \N^*\).}
    \item \question{Utiliser la première question pour montrer que la fonction \(\Gamma\) s'étend en une fonction holomorphe sur \(\C\setminus \mathbb{Z}_-\).}
\end{enumerate}
}
