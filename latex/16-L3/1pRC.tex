\uuid{1pRC}
\exo7id{2826}
\auteur{burnol}
\datecreate{2009-12-15}
\isIndication{false}
\isCorrection{true}
\chapitre{Théorème des résidus}
\sousChapitre{Théorème des résidus}

\contenu{
\texte{
Soit $\phi(z) = \frac{4z + 3}{4 +
3z}$. Montrer: $\forall\theta\in\Rr\quad  |\phi(e^{i\theta})|
= 1$. En déduire $|z|<1 \implies |\phi(z)|<1$.
}
\reponse{
Soit $z=e^{i\theta}$. Alors $|4z+3|=|e^{i\theta } (4+3e^{-i\theta })| = |\overline{4+3e^{i\theta }}| =|4+3z|$.
Par le principe du maximum
$$|\Phi (z) | < 1=\sup_\theta |\Phi (e^{i\theta })| \quad \text{ pour tout } \;\; z \in D.$$
}
}
