\uuid{1cym}
\exo7id{2843}
\auteur{burnol}
\organisation{exo7}
\datecreate{2009-12-15}
\isIndication{false}
\isCorrection{false}
\chapitre{Théorème des résidus}
\sousChapitre{Théorème des résidus}

\contenu{
\texte{
Soit $\mathcal{R} = \{x_0\leq x\leq x_1, y_0\leq y\leq y_1\}$ un
  rectangle. En utilisant le théorème des résidus justifier
  la formule intégrale de Cauchy pour $z$ dans l'intérieur
  du rectangle et $f$ holomorphe sur le rectangle fermé:
\[ f(z) = \frac1{2\pi i}\int_{\partial\mathcal{R}}
\frac{f(w)}{w-z}dw\]
Démontrer ce résultat de manière plus
simple, directement à partir du théorème de Cauchy-Goursat pour les
fonctions holomorphes sur les rectangles, en utilisant la
fonction $w\mapsto (f(w)-f(z))/(w-z)$ (et aussi la notion
d'indice d'un lacet).  Dans le cas où $z$
est à l'\emph{extérieur} du rectangle $\mathcal{R}$, que vaut $\int_{\partial\mathcal{R}}
\frac{f(w)}{w-z}dw$?
}
}
