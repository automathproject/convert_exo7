\uuid{i6l5}
\exo7id{2678}
\auteur{matexo1}
\datecreate{2002-02-01}
\isIndication{false}
\isCorrection{true}
\chapitre{Théorème des résidus}
\sousChapitre{Théorème des résidus}

\contenu{
\texte{
En int{\'e}grant $e^{2iaz -z^2}$ le long du rectangle de sommets $0$, $R$, $R+ia$, $ia$, et
en faisant tendre $R$ vers $+\infty $, montrer que l'on a
$$ \int_0^{+\infty } e^{-x^2}\cos 2ax \,dx = {\sqrt \pi \over 2} e^{-a^2}.$$
(On admettra la formule $\int_0^{+\infty }e^{-x^2}\,dx = \sqrt \pi /2$.)
}
\reponse{
On int{\`e}gre $f(z) = e^{2iaz -z^2}$ sur le rectangle. Il n'y a aucun p{\^o}le, donc
l'int{\'e}grale sur le rectangle est nulle; l'int{\'e}grale sur le c{\^o}t{\'e} $[R, R+ia]$ tend vers
z{\'e}ro. On a donc
$$\begin{array}{ccc}
 \int_0^{+\infty } e^{2iax-x^2} \,dx &= \int_{[0,ia]} f(z)\,dz
 + \int_{[ia, ia+\infty ]}f(z)\,dz \cr
&=  \int_0^a  e^{-2ay+y^2}\,idy + \int_0^{+\infty } e^{(ia-x)(ia+x)}\,dx \cr
&= i\int_0^a e^{y^2-2ay}\, dy + e^{-a^2}{\sqrt \pi \over 2}
\end{array}$$
d'o{\`u} le r{\'e}sultat en ne conservant que la partie r{\'e}elle.
}
}
