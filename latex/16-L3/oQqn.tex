\uuid{oQqn}
\exo7id{2682}
\auteur{matexo1}
\datecreate{2002-02-01}
\isIndication{false}
\isCorrection{true}
\chapitre{Fonction holomorphe}
\sousChapitre{Fonction holomorphe}

\contenu{
\texte{
Soit $a$ un r{\'e}el dans $]0,1[$. On souhaite calculer l'int{\'e}grale
$$ I_a = \int_0^{+\infty } {x^{a-1} \over 1+x}\,dx $$

Pour cela, on d{\'e}finit une fonction $f$ sur
$\Omega =\C\setminus [0,+\infty [$ de la mani{\`e}re
suivante:
$$ f(z) = {\exp\left((a-1)\log z\right) \over 1+z} $$
o{\`u} la fonction $\log$ est d{\'e}finie comme dans l'exercice pr{\'e}c{\'e}dent.
\begin{itemize}
\item Montrer que $f$ est m{\'e}romorphe sur $\Omega $. Quels en sont les p{\^o}les ?

\item Soit $\varepsilon\in\left]0,1\right[$ et $R>1$ deux r{\'e}els.
On consid{\`e}re {\`a} pr{\'e}sent le chemin~$C$ 
r{\'e}union du segment $[\varepsilon, R]+0i$, du
cercle de rayon~$R$, parcouru positivement, du segment
$[R, \varepsilon]-0i$ et du cercle de rayon~$\varepsilon$
parcouru n{\'e}gativement.
Calculer $ \int_C f(z)\,dz $; en d{\'e}duire
la valeur de $I_a$.
\end{itemize}
}
\reponse{
La fonction $f$ est m{\'e}romorphe comme compos{\'e}e de fonctions m{\'e}romorphes. Son
unique p{\^o}le est $-1$. Le contour {\'e}tant enti{\`e}rement inclus dans le domaine $\Omega $, on a
$$ \int_C f(z) \,dz = 2i\pi  \mbox{\rm Res}(f, -1) = 2i\pi  \left[\exp((a-1)\log z\right]_{z=-1}
 = 2i\pi  e^{i(a-1)\pi } =-2i\pi  e^{ia\pi }$$

Notons  $\gamma$ le petit cercle
(orient{\'e} n{\'e}gativement), $\Gamma$ le grand cercle
 (orient{\'e} positivement) dans
$C$, et remarquons que sur le segment $[\varepsilon, R]+0i$,
l'argument de $z$ doit {\^e}tre pris nul, tandis qu'il faut le
prendre {\'e}gal {\`a} $2\pi $ sur le segment oppos{\'e}
$[R,\varepsilon]-0i$. On obtient
$$\int_C f(z) \,dz = \int_\varepsilon^R {x^{a-1}\over 1+x}\,dx
+\int_\Gamma f(z)\,dz +\int_R^\varepsilon {x^{a-1} e^{i2\pi
(a-1)}\over 1+x}\,dx +\int_\gamma f(z)\,dz$$

Lorsque $\varepsilon\to 0$ et $R\to 0$, $\int_\gamma f  \to 0$ et $\int_\Gamma f \to 0$ d'apr{\`e}s les
lemmes de Jordan, car $|zf(z)|\to 0$ quand $|z|\to 0$ ou $+\infty $. Finalement
$$ \int_C f(z)\,dz \to (1-e^{i2\pi a}) I $$
et donc
$$ I = {-2i\pi  e^{ia\pi }\over 1-e^{i2\pi a}} = {\pi \over \sin a\pi }.$$
}
}
