\uuid{vZHB}
\exo7id{7604}
\auteur{mourougane}
\organisation{exo7}
\datecreate{2021-08-10}
\isIndication{false}
\isCorrection{true}
\chapitre{Autre}
\sousChapitre{Autre}

\contenu{
\texte{

}
\begin{enumerate}
    \item \question{\'Enoncer le théorème des résidus.}
\reponse{Soit $D$ un ouvert étoilé de $\Cc$. Soit $\Gamma$ un chemin fermé dans $D$. Soit $C$ un ensemble fini de points de $D-\Gamma$.
Soit $f : D-C\to\Cc$ holomorphe.
Alors,
$$\sum_{c\in C\cap Int(\Gamma)} Ind_\Gamma(c) Res_c(f)=\frac{1}{2i\pi}\int_\Gamma f(z)dz.$$}
    \item \question{\`A l'aide du théorème des résidus, retrouver la formule de représentation de Cauchy pour les disques :
Soit $D$ un ouvert de $\Cc$ et $f : D\to \Cc$ une application holomorphe.
Alors, pour tout disque $\Delta_r(a)$ dont l'adhérence est incluse dans $D$,
$$\forall b\in\Delta_r(a), \ \ \ \frac{1}{2i\pi}\int_{\partial\Delta_r(a)}\frac{f(z)dz}{z-b}=f(b).$$}
\reponse{Soit $D$ un ouvert de $\Cc$ et $f : D\to \Cc$ une application holomorphe.
Soit $\Delta_r(a)$ un disque dont l'adhérence est incluse dans $D$.
Soit $b\in\Delta_r(a)$. 
Soit $R>0$ tel que $\overline{\Delta_r(a)}\subset \Delta_R(a)\subset D$.
L'ouvert $\Delta_R(a)$ est étoilé, le chemin $\partial\Delta_r(a)$ est inclus dans $\Delta_R(a)$
et ne contient pas $b$ ; l'application $f : \Delta_R(a)-\{b\}\to\Cc,\ \ z\mapsto \frac{f(z)}{z-b}$
est holomorphe : par le théorème des résidus, on en déduit
$$Ind_{\partial\Delta_r(c)}(b) Res_b(\frac{f(z)}{z-b})= Res_b(\frac{f(z)}{z-b})
=\frac{1}{2i\pi}\int_\Gamma \frac{f(z)}{z-b}dz.$$
Reste à noter que, en développant $f$ en série entière au voisinage de $b$, $Res_b(\frac{f(z)}{z-b})=f(b).$}
    \item \question{Montrer plus généralement : Soit $D$ un ouvert de $\Cc$ et $f : D\to \Cc$ une application holomorphe.
Alors, pour tout disque $\Delta_r(a)$ dont l'adhérence est incluse dans $D$,
$$\forall k\in\Nn,\ \ \ \forall b\in\Delta_r(a), \ \ \ \
\frac{1}{2i\pi}\int_{\partial\Delta_r(a)}\frac{f(z)dz}{(z-b)^{k+1}}=\frac{f^{(k)}(b)}{k!}.$$}
\reponse{On réitère le raisonnement précédent avec l'application 
$f_k : \Delta_R(a)-\{b\}\to\Cc, z\mapsto \frac{f(z)}{(z-b)^{k+1}}$.
Le résidu de $\frac{f(z)}{(z-b)^{k+1}}$ en $b$ est le coefficient de $(z-b)^k$ dans 
le développement en série entière de $f$ centré en $b$.
Comme ce développement est donné par le développement de Taylor,
le coefficient de $(z-b)^k$ est $\frac{f^{(k)}(b)}{k!}$.}
\end{enumerate}
}
