\uuid{AvR9}
\exo7id{2788}
\auteur{burnol}
\datecreate{2009-12-15}
\isIndication{false}
\isCorrection{true}
\chapitre{Fonction holomorphe}
\sousChapitre{Fonction holomorphe}

\contenu{
\texte{
\label{ex:burnol1.1.6}
  On se donne deux séries entières $f(z) = \sum_{n=0}^\infty
  a_n z^n$ et $g(z) = \sum_{n=0}^\infty
  b_n z^n$ de rayons de convergences $R_1$ et $R_2$ non
  nuls. En utilisant le théorème sur les séries doubles
  prouver $f(z)g(z) = \sum_{n=0}^\infty c_n z^n$ pour $|z|<R
  = \min(R_1,R_2)$ avec (formules dites de Cauchy):
\[ \forall n\in\Nn\qquad c_n = \sum_{j=0}^n a_j b_{n-j}\]
Le rayon de convergence de la série $\sum_{n=0}^\infty c_n
     z^n$ est-il toujours égal à 
$\min(R_1,R_2)$ ou peut-il être plus grand?
}
\reponse{
Prenons $r<\min(R_1,R_2)$. Alors, il existe $C>0$ et $0<\lambda <1$ tels que
$|a_n|r^n \leq C\lambda^n $ et $|b_n|r^n \leq C\lambda ^n$ (v\'erifiez-le !). D'o\`u
$$\sum_{j=0}^n |a_j|r^j |b_{n-j}|r^{n-j} \leq (n+1) C^2 \lambda^n\;, $$
ce qui permet d'affirmer, pour tout $z$ avec $|z|=r$:
$$\sum_{n=0}^\infty \big(\sum_{j=0}^n |a_j z^j||b_{n-j} z^{n-j}|\big) < \infty\;.$$
Par le th\'eor\`eme du cours sur les séries doubles (voir le polycopi\'e
2005/2006 de J.-F.~Burnol, Annexe 8.2), ceci signifie que la série double
$$\sum_{j=0}^\infty\sum_{k=0}^\infty (a_j z^j b_k z^k)$$
est absolument convergente. On peut donc d'après ce théorème affirmer:
$$f(z)g(z) = \sum_{j=0}^\infty a_jz^j \sum_{k=0}^\infty b_k z^k =
\sum_{n=0}^\infty \big(\sum_{j=0}^n (a_j z^j)(b_{n-j} z^{n-j})\big)\;.$$
Or, la série de droite est $\sum_{n=0}^\infty  c_n z^n$ avec $c_n = \sum_{j=0}^n
a_j b_{n-j}$. Au passage on obtient que le rayon de convergence de cette série
est au moins
égal à $r$. Comme $r<\min(R_1,R_2)$ est arbitraire, le rayon de convergence
est en fait au moins égal à $\min(R_1,R_2)$
(il peut \^etre plus grand comme on le voit par exemple avec $f(z) =
\frac1{1-z}$ et $g(z) = 1-z$, ou encore avec $f(z) = \frac{2-z}{(1-z)(3-z)}$
et $g(z) = \frac{1-z}{(2-z)(3-z)}$).
}
}
