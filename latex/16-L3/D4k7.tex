\uuid{D4k7}
\exo7id{2839}
\auteur{burnol}
\organisation{exo7}
\datecreate{2009-12-15}
\isIndication{false}
\isCorrection{false}
\chapitre{Théorème des résidus}
\sousChapitre{Théorème des résidus}

\contenu{
\texte{
On considère un lacet $\gamma:[a,b]
\to\Cc\setminus\{0\}$ (donc ne passant pas par
l'origine). On suppose qu'il n'existe qu'un nombre fini de
$t\in [a,b]$ avec $\gamma(t) \in \Delta = ]-\infty,0[$. On
les note $t_0<t_1<\dots <t_N$. Pour simplifier on supposera
que $\gamma(a)$ est sur $\Delta$, donc $t_0 =a$ et
$t_N=b$. Montrer que pour $t=t_j-\epsilon$, $\epsilon>0$
suffisamment petit, le signe $\mu_j$ de
$\Im(\gamma(t_j-\epsilon))$ ne dépend pas de $\epsilon$,
et de même pour le signe $\mu_j'$ de
$\Im(\gamma(t_j+\epsilon))$ (préciser ce que l'on fait
pour $j=0$ et $j=N$). 

 Si $\mu_j=+$ et $\mu_j' = -$ on dit
que $\gamma$ traverse $\Delta$ en $t=t_j$ dans le sens
direct, si $\mu_j=-$ et $\mu_j'=+$ on dit que $\gamma$
traverse $\Delta$ en $t=t_j$ dans le sens rétrograde. Sinon
on dit que $\gamma$ touche mais ne traverse pas $\Delta$. En
utilisant la relation entre la fonction $\mathrm{Log}(\gamma(t))$ et
la variation de l'argument de $\gamma(t)$ sur chaque
intervalle $]t_j,t_{j+1}[$, prouver $\Delta_{\gamma_j}
\arg(z) = \pi(\mu_{j+1} - \mu_j')$ avec $\gamma_j=\gamma$
restreint à $[t_j,t_{j+1}]$.

 En déduire que $\mathrm{Ind}(\gamma,0)$
est égal au nombre de valeurs de $t$ ($a$ et $b$ ne comptent
que pour un seul) pour lesquelles $\gamma$ traverse
$\Delta$, comptées positivement si la traversée est directe,
négativement si la traversée est rétrograde. 

Dans la
pratique vous pourrez  utiliser n'importe quelle demi-droite
issue de l'origine à la place de $\Delta$ à partir du moment
où elle n'intersecte le lacet $\gamma$ qu'en un nombre fini
de points (si on n'impose pas au lacet d'être régulier,
c'est-à-dire d'avoir un vecteur vitesse partout non nul,
alors il peut rester figé en un même point un certain temps,
et donc il faut modifier un petit peu la discussion
ci-dessus qui suppose qu'il n'y a qu'un nombre fini de
valeurs de $t$ pour lesquels $\gamma(t)$ est sur la
demi-droite).
}
}
