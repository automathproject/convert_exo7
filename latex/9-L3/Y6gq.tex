\uuid{Y6gq}
\exo7id{6426}
\auteur{potyag}
\organisation{exo7}
\datecreate{2011-10-16}
\isIndication{false}
\isCorrection{false}
\chapitre{Géométrie projective}
\sousChapitre{Géométrie projective}

\contenu{
\texte{
Rappelons qu'un cercle généralisé est soit un cercle
euclidien $\Sigma(z_o,r)=\{z\in \C\ \vert\ \vert z-z_0\vert=r\}$ soit une
droite à laquelle on ajoute le point $\{\infty\}$ (à l'aide la projection
stéréographique). On note $M=\{\frac{az+b}{cz+d}\ \vert\ a, b, c, d\in\C,\
ad-bc\not=0\}.$
}
\begin{enumerate}
    \item \question{Montrer que le groupe
  $PGL_2\C$ agit trois fois transitivement sur $\overline{\Cc}.$}
    \item \question{Vérifier que chaque cercle généralisé dans $\overline{\Cc}$ s'écrit sous la
  forme :
  $$A(x^2+y^2)+Bx+Cy+D=0,\ B^2+C^2 > 4 A D$$}
    \item \question{Soit $C_1\in \C$ un cercle généralisé, alors montrer que un
  sous-espace $C_2\subset \C$ est un cercle généralisé ssi il existe
  $\gamma\in M$ telle que $\gamma(C_1)=C_2.$}
    \item \question{Soit $K\subset \C$ un cercle généralisé et $f\in M(2)$ tel que $\displaystyle
f\vert_K\equiv {\rm id_K}$ alors montrer que soit $f\equiv {\rm id}$ soit $f$
est la réflexion par rapport à $K.$}
\end{enumerate}
}
