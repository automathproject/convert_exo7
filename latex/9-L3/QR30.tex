\uuid{QR30}
\exo7id{6429}
\auteur{potyag}
\organisation{exo7}
\datecreate{2011-10-16}
\isIndication{false}
\isCorrection{false}
\chapitre{Géométrie projective}
\sousChapitre{Géométrie projective}

\contenu{
\texte{

}
\begin{enumerate}
    \item \question{Montrer que chaque application $g\in M$ possède soit un point fixe
dans $\overline{\Cc}$ soit deux points fixes. Cette affirmation reste-t-elle vraie pour
les éléménts de $M(2)$ ?}
    \item \question{Notons ${\rm fix}(g)$ l'ensemble $\{x\in\overline{\Cc}\ \vert\ g(x)=x\}$ des points fixes
de $g.$ Montrer que si $\gamma=fgf^{-1}$ alors ${\rm fix} (\gamma)=f({\rm fix} (g).$}
    \item \question{Soient $C_i\ (i=1,2)$ deux cercles généralisés. Montrer que
  $\exists\ \gamma\in M\ :\ \tau_{C_1}=\gamma\tau_{C_2}\gamma^{-1},$ où $\tau_{C_i}$ désigne
  l'inversion par rapport à $C_i.$}
\end{enumerate}
}
