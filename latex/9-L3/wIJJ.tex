\uuid{wIJJ}
\exo7id{7753}
\auteur{mourougane}
\datecreate{2021-08-11}
\isIndication{false}
\isCorrection{false}
\chapitre{Géométrie projective}
\sousChapitre{Géométrie projective}

\contenu{
\texte{
Soit $\vec{V}$ un espace vectoriel et $q$ une forme quadratique sur
$\vec{V}$. On appelle quadrique projective associée à $q$ le sous
ensemble de $P:=P(\vec{V})$ défini par ($p$ est la projection
canonique $\vec{V}-\{0\}\to P$)
$$Q:=p\left( \{x\in\vec{V}-\{0\}/ q(x)=0\}\right).$$
}
\begin{enumerate}
    \item \question{On suppose $\dim P=1$. Montrer que si $Q$ contient trois points
 distincts, $Q=P$.}
    \item \question{On suppose $\dim P=2$. Montrer que si $Q$ contient une droite $d$,
 soit $Q=P$, soit il existe une droite $d'$ telle que $Q=d\cup d'$.}
    \item \question{Soit $d$ une droite de $P$. Montrer que si $d$ rencontre $Q$ en
 au moins trois points, $d$ est incluse dans $Q$. Montrer que si
 $k=\mathbb{C}$ alors $d$ rencontre $Q$.}
\end{enumerate}
}
