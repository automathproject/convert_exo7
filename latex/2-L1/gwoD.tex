\uuid{gwoD}
\exo7id{4259}
\auteur{quercia}
\organisation{exo7}
\datecreate{2010-03-12}
\isIndication{false}
\isCorrection{true}
\chapitre{Calcul d'intégrales}
\sousChapitre{Autre}

\contenu{
\texte{
Soit $f : {[a,b]} \to {\R^{+*}}$ continue.
}
\begin{enumerate}
    \item \question{Montrer qu'il existe une subdivision de $[a,b]$~:
    ${a=x_0 < x_1 < \dots < x_n = b}$ telle que~:

    $\forall\ k \in [[0,n-1]],\  \int_{t=x_k}^{x_{k+1}}f(t)\,d t =
        \frac1n  \int_{t=a}^{b}f(t)\,d t$.}
    \item \question{Étudier $\lim_{n\to\infty} \frac1n \sum_{k=0}^{n-1}f(x_k)$.}
\reponse{
Soit $F(x) =  \int_{t=a}^{x}f(t)\,d t$ et $G = F^{-1}$.
Alors $\frac1n \sum_{k=0}^{n-1}f(x_k) = \frac 1n \sum_{k=0}^{n-1} f\circ G(\frac kn)
\to  \int_{t=a}^{b}f^2(t)\,d t\Bigm/ \int_{t=a}^{b}f(t)\,d t$ lorsque $n\to\infty$.
}
\end{enumerate}
}
