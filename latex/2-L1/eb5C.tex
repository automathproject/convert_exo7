\uuid{eb5C}
\exo7id{5237}
\auteur{rouget}
\datecreate{2010-06-30}
\isIndication{false}
\isCorrection{true}
\chapitre{Suite}
\sousChapitre{Convergence}

\contenu{
\texte{
Etudier les deux suites $u_n=\sum_{k=0}^{n}\frac{1}{k!}$ et $v_n=u_n+\frac{1}{n.n!}$.
}
\reponse{
Il est immédiat que $u$ croit strictement et que $v-u$ est strictement positive et tend vers $0$.
De plus, pour $n$ entier naturel donné, 

$$v_{n+1}-v_n=\frac{1}{(n+1)!}+\frac{1}{(n+1)\times(n+1)!}-\frac{1}{n\times n!}=\frac{n(n+1)+n-(n+1)^2}{n(n+1)\times(n+1)!}=\frac{-1}{n(n+1)\times(n+1)!}< 0,$$
et la suite $v$ est strictement décroissante. Les suites $u$ et $v$ sont donc adjacentes et convergent vers une limite commune (à savoir $e$).
 

(Remarque. Dans ce cas, la convergence est très rapide. On a pour tout entier naturel non nul $n$,  $\sum_{k=0}^{n}\frac{1}{k!}<e<\sum_{k=0}^{n}\frac{1}{k!}+\frac{1}{n\times n!}$ et $n=5$ fournit par exemple $2,716...<e<2,718...$).
}
}
