\uuid{q7xP}
\exo7id{825}
\auteur{cousquer}
\datecreate{2003-10-01}
\isIndication{false}
\isCorrection{true}
\chapitre{Calcul d'intégrales}
\sousChapitre{Fraction rationnelle}

\contenu{
\texte{
Calculer les intégrales de fractions rationnelles suivantes.
}
\begin{enumerate}
    \item \question{$\displaystyle \int_0^1 \frac{dx}{x^2+2}$.}
\reponse{$\frac{1}{x^2+2}$ est un élément simple.
$\int_0^1\frac{dx}{x^2+2} = \frac{1}{\sqrt{2}} \arctan \frac{1}{\sqrt{2}}$.}
    \item \question{$\displaystyle \int_{-1/2}^{1/2} \frac{dx}{1-x^2}$.}
\reponse{Décomposition~: $\frac{1}{1-x^2} = \frac{1/2}{x+1} - \frac{1/2}{x-1}$. 
Intégrale~: $\int_{-1/2}^{1/2} \frac{dx}{1-x^2} = \ln 3$.}
    \item \question{$\displaystyle \int_2^3 \frac{2x+1}{x^2+x-3}\,dx$.}
\reponse{Pas besoin de décomposer la fraction rationnelle, car $2x+1$ est 
la dérivée de $x^2+x-3$~! $\int_2^3 \frac{2x+1}{x^2+x-3}\,dx = \ln 3$.}
    \item \question{$\displaystyle \int_0^2 \frac{x\,dx}{x^4+16}$.}
\reponse{On peut évidemment décomposer la fraction rationnelle en 
éléments simples~: 
$\frac{x}{x^4+16} = \frac{\sqrt{2}/8}{x^2-2x\sqrt{2}+4} - 
\frac{\sqrt{2}/8}{x^2+2x\sqrt{2}+4}$, 
mais il est bien plus simple de 
faire le changement de variables $x^2=u$. 
Alors 
$\int_0^2 \frac{x\,dx}{x^4+16} = 
\frac{1}{2}\int_0^4 \frac{du}{u^2+16} = \frac{\pi}{32}$.}
    \item \question{$\displaystyle \int_0^3 \frac{x^4+6x^3-5x^2+3x-7}{(x-4)^3}\,dx$.}
\reponse{La décomposition de $\frac{x^4+6x^3-5x^2+3x-7}{(x-4)^3}$ est
$x+18+\frac{163}{x-4}+\frac{507}{(x-4)^2}+\frac{565}{(x-4)^3}$~; les 
primitives sont 
$\frac{x^2}{2}+18x-\frac{1014x-3491}{2(x-4)^2}+163\ln\vert x-4\vert+C$.
Enfin,
$\int_0^3 \frac{x^4+6x^3-5x^2+3x-7}{(x-4)^3}\,dx =
\frac{5565}{32}-326\ln 2$.}
    \item \question{$\displaystyle \int_{-2}^0 \frac{dx}{x^3-7x+6}$.}
\reponse{Décomposition~: 
$\frac{1}{x^3-7x+6} = \frac{1}{20(x+3)} -\frac{1}{4(x-1)} 
+\frac{1}{5(x-2)}$. Primitives~:
$\frac{1}{20}\ln\Bigl\vert\frac{(x-2)^4(x+3)}{(x-1)^5}\Bigr\vert+C$, d'où
 $\int_{-2}^0 \frac{dx}{x^3-7x+6} = 
\frac{1}{10}\ln(27/4)$.}
    \item \question{$\displaystyle \int_{-1}^1 \frac{2x^4+3x^3+5x^2+17x+30}{x^3+8}\,dx$.}
\reponse{Décomposition~: 
$\frac{2x^4+3x^3+5x^2+17x+30}{x^3+8} = 
2x+3+\frac{2}{x+2}+\frac{3x-1}{x^2-2x+4}$. 
Les primitives sont~:
$x^2+3x+\ln(x+2)^2 + \frac{3}{2}\ln(x^2-2x+4) + 
\frac{2}{\sqrt{3}}\arctan\frac{x-1}{\sqrt{3}}+C$.
Intégrale~: 
$\int_{-1}^1 \frac{2x^4+3x^3+5x^2+17x+30}{x^3+8}\,dx = 
6+\frac{7\ln 3-3\ln 7}{2} + 
\frac{2}{\sqrt{3}}\arctan\frac{2}{\sqrt{3}}$.}
    \item \question{$\displaystyle \int_2^3 \frac{4x^2}{x^4-1}\,dx$.}
\reponse{Décomposition~: 
$\frac{4x^2}{x^4-1} = 
\frac{2}{x^2+1} - \frac{1}{x+1} +\frac{1}{x-1}$.
Primitives~:
$\ln\Bigl\vert \frac{x-1}{x+1}\Bigr\vert +2\arctan x +C$, d'où 
$\int_2^3 \frac{4x^2}{x^4-1}\,dx =
\ln\frac{3}{2}+2\arctan \frac{1}{7}$.}
    \item \question{$\displaystyle \int_{-1}^0 \frac{x^3+2x+1}{x^3-3x+2}\,dx$.}
\reponse{La décomposition est $\frac{x^3+2x+1}{x^3-3x+2} = 
1+\frac{4/3}{(x-1)^2} + \frac{11/9}{x-1} - \frac{11/9}{x+2}$. On 
trouve alors\linebreak %%%%
$\int_{-1}^0 \frac{x^3+2x+1}{x^3-3x+2}\,dx = 
\frac{5}{3}-\frac{22}{9}\ln 2$.}
    \item \question{$\displaystyle \int_1^2 \frac{2x^8+5x^6-12x^5+30x^4+36x^2+24}
    {x^4(x^2+2)^3}\,dx$.}
\reponse{La décomposition de 
$\frac{2x^8+5x^6-12x^5+30x^4+36x^2+24}{x^4(x^2+2)^3}$ est 
$\frac{3}{x^4} + \frac{2}{x^2+2} - \frac{6}{(x^2+2)^2} - 
\frac{12x-16}{(x^2+2)^3}$~; les primitives sont 
$-\frac{1}{x^3} + \frac{2x+3}{(x^2+2)^2} + \sqrt{2} 
\arctan\frac{x}{\sqrt{2}}+C$. 
Enfin 
$\int_{1}^2 \frac{2x^8+5x^6-12x^5+30x^4+36x^2+24}
    {x^4(x^2+2)^3}\,dx =
    \frac{37}{72} + 2\sqrt{2}\arctan\sqrt{2}-\frac{\pi}{\sqrt{2}}$.}
    \item \question{$\displaystyle \int_0^a \frac{-2x^2+6x+7}
    {x^4+5x^2+4}\,dx$ pour $a\in \mathbb{R}$. Y a-t-il une 
    limite quand $a\to+\infty$~?}
\reponse{Décomposition de la fraction rationnelle~: 
$\frac{-2x^2+6x+7}{x^4+5x^2+4} = 
\frac{2x+3}{x^2+1}-\frac{2x+5}{x^2+4}$. Primitives~:
$\ln\Bigl\vert \frac{x^2+1}{x^2+4}\Bigr\vert+3\arctan 
x-\frac{5}{2}\arctan\frac{x}{2}+C$. 
Alors
$\int_0^a \frac{-2x^2+6x+7}{x^4+5x^2+4}\,dx = 
\ln\Bigl\vert \frac{a^2+1}{a^2+4}\Bigr\vert+3\arctan 
a-\frac{5}{2}\arctan\frac{a}{2} + 2\ln 2$. Enfin
$\lim_{a\to +\infty} \int_0^a \frac{-2x^2+6x+7}{x^4+5x^2+4}\,dx =
\frac{\pi}{4}+2\ln 2$.}
    \item \question{$\displaystyle \int_0^2 \frac{dx}{x^4+1}$.}
\reponse{Pour factoriser le dénominateur, penser à 
faire $x^4+1=x^4+2x^2+1-2x^2$~; on trouve alors
$\frac{1}{x^4+1} = 
\frac{(x\sqrt{2}+2)/4}{x^2+x\sqrt {2}+1} - 
\frac{(x\sqrt{2}-2)/4}{x^2-x\sqrt{2}+1}$. Les primitives s'écrivent 
$$\textstyle \frac{1}{4\sqrt{2}} \ln \frac{x^2+x\sqrt{2}+1}{x^2-x\sqrt{2}+1} +
\frac{1}{2\sqrt{2}}\Bigl(\arctan(x\sqrt{2}+1)+\arctan(x\sqrt{2}-1)\Bigr)+C$$
ce qui donne 
$\int_0^2 \frac{dx}{x^4+1} = 
\frac{1}{4\sqrt{2}}\ln\frac{33+20\sqrt{2}}{17}
+\frac{1}{2\sqrt{2}}\Bigl(\pi-\arctan \frac{2\sqrt{2}}{3}\Bigr)$.}
\end{enumerate}
}
