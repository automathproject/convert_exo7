\uuid{WFj8}
\exo7id{677}
\auteur{bodin}
\datecreate{1998-09-01}
\isIndication{true}
\isCorrection{true}
\chapitre{Continuité, limite et étude de fonctions réelles}
\sousChapitre{Continuité : pratique}

\contenu{
\texte{
Les fonctions suivantes sont-elles prolongeables par continuit\'e sur
$\R$ ?
$$ a)\ f(x)=\sin x \cdot \sin \frac{1}{x}\ ;\ \ \
b)\ g(x)=\frac{1}{x}\ln\frac{e^x+e^{-x}}{2}\ ;$$
$$c)\ h(x)=\frac{1}{1-x}-\frac{2}{1-x^2}\ .$$
}
\indication{Oui pour le deux premi\`eres en posant $f(0)=0$, $g(0)=0$, non pour la troisi\`eme.}
\reponse{
La fonction est d\'efinie sur $\Rr^*$ t elle est continue sur $\Rr^*$. Il faut d\'eterminer un \'eventuel prolongement par continuit\'e en $x=0$, c'est-\`a-dire savoir si $f$ a une limite en $0$.
$$|f(x)| = |\sin x| |\sin 1/x| \leq |\sin x|.$$
Donc $f$ a une limite en $0$ qui vaut $0$. 
Donc en posant $f(0) = 0$, nous obtenons une fonction $f : \Rr \longrightarrow \Rr$ qui est continue.
La fonction $g$ est d\'efinie et continue sur $\Rr^*$. 
Etudions la situation en $0$. Il faut remarquer que $g$ est la taux d'accroissement en
$0$ de la fonction $k(x) = \ln \frac{e^x+e^{-x}}{2}$ : en effet $g(x)=\frac{k(x)-k(0)}{x-0}$.
Donc si $k$ est dérivable en $0$ alors la limite de $g$ en $0$ est \'egale \`a
la valeur de $k'$ en $0$. 

Or la fonction $k$ est dérivable sur $\Rr$ et $k'(x) = \frac{e^x-e^{-x}}{e^x+e^{-x}}$
donc $k'(0)=0$. Bilan : en posant $g(0)=0$ nous obtenons une fonction $g$ d\'efinie et continue sur $\Rr$.
$h$ est d\'efinie et continue sur $\Rr \setminus \{ -1,1 \}$.
$$h(x) = \frac{1}{1-x} - \frac{2}{1-x^2} = \frac{1+x-2}{(1-x)(1+x)}
= \frac{-1+x}{(1-x)(1+x)} = \frac{-1}{(1+x)}.$$
Donc $h$ a pour limite $-\frac 12$ quand $x$ tend vers $1$.
Et donc en posant $h(1) = -\frac 12$, nous d\'efinissons une fonction
continue sur $\Rr \setminus \{ -1 \}$.
En $-1$ la fonction $h$ ne peut \^etre prolong\'ee continuement,
car en $-1$, $h$ n'admet de limite finie.
}
}
