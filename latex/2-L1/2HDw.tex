\uuid{2HDw}
\exo7id{696}
\auteur{gourio}
\organisation{exo7}
\datecreate{2001-09-01}
\isIndication{false}
\isCorrection{false}
\chapitre{Continuité, limite et étude de fonctions réelles}
\sousChapitre{Fonction continue par morceaux}

\contenu{
\texte{
On dit qu'une suite $(f_{n})_{n\in \Nn}$ de fonctions d\'{e}finies sur
$I=[a,b]$ converge uniform\'{e}ment vers $f$ si :
$$\forall \epsilon >0,\exists \Nn\in \Nn,\forall n\geq \Nn,\forall x\in
I,\left| f_{n}(x)-f(x)\right| <\epsilon . $$

On suppose que $(f_{n})_{n\in \Nn}$ converge uniform\'{e}ment vers $f$
sur l'intervalle $[a,b]$, et que toutes les $f_{n}$ sont continues.\ Montrer
que $\forall x\in [a,b],$ la suite $(f_{n}(x))_{n\in \Nn}$ est
convergente, et donner sa limite. Montrer que $f$ est born\'{e}e et
continue.

On ne suppose plus que $(f_{n})_{n}$ converge uniform\'{e}ment mais
seulement point par point (ie, $\forall x\in [a,b],$ la suite $(f_{n}(x))_{n\in \Nn}$
 est convergente vers $f(x)$); de plus toutes les $f_{n}$
sont lipschitziennes de rapport $k$ ; montrer que $f$ est
lipschitzienne de rapport $k$ et qu'il y a converge uniforme.
}
}
