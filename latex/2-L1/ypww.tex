\uuid{ypww}
\exo7id{504}
\auteur{cousquer}
\organisation{exo7}
\datecreate{2003-10-01}
\isIndication{false}
\isCorrection{false}
\chapitre{Suite}
\sousChapitre{Convergence}

\contenu{
\texte{

}
\begin{enumerate}
    \item \question{Dessiner les suites suivantes~:
\begin{enumerate}}
    \item \question{$\displaystyle u_n = \frac{n^2-25}{2n^2+1}$\qquad
(prendre 2~cm comme unité sur $Oy$)}
    \item \question{$u_n= (-1)^n$}
    \item \question{$\displaystyle u_n =\frac{1}{n}\cos n \qquad
v_n=\frac{1}{n} \vert\cos n\vert$ \qquad($n$ en radians)}
    \item \question{$u_n = \cos n$}
    \item \question{$u_1=1$~; $u_2=2$~; $u_3=3$~; $u_4=-1$~; $u_n=2$ pour $n\ge 5$.}
    \item \question{$\displaystyle u_n=\frac{(-1)^n}{n^2+1}$ \qquad
(prendre 10 cm comme unité sur $Oy$)}
    \item \question{$\displaystyle u_n= \cos\frac{n\pi}{6}$}
    \item \question{$\displaystyle u_n = \sin\frac{1}{\sqrt{n}}$ \qquad
(prendre 1 cm comme unité sur $Oy$)}
    \item \question{$u_n = n^2+1$}
    \item \question{$\displaystyle u_n=\frac{1}{n+(-1)^n\sqrt{n}}$ \quad(pour $n\ge 2$)}
\end{enumerate}
}
