\uuid{K2A3}
\exo7id{646}
\auteur{ridde}
\organisation{exo7}
\datecreate{1999-11-01}
\isIndication{true}
\isCorrection{true}
\chapitre{Continuité, limite et étude de fonctions réelles}
\sousChapitre{Continuité : théorie}

\contenu{
\texte{
Soit $f : \Rr^ + \rightarrow \Rr$ continue admettant une limite finie en $ + \infty$.
Montrer que $f$ est born\'ee. Atteint-elle ses bornes ?
}
\indication{Il faut raisonner en deux temps : d'abord \'ecrire la d\'efinition de la limite en $+\infty$, en fixant par exemple $\epsilon =1$, cela donne une borne sur $[A,+\infty]$. Puis travailler sur $[0,A]$.}
\reponse{
Notons $\ell$ la limite de $f$ en $+\infty$:
$$\forall \epsilon > 0 \quad \exists A \in \Rr
\quad x>A \Rightarrow \ell - \epsilon \leq f(x) \leq \ell + \epsilon.$$
Fixons $\epsilon = +1$, nous obtenons un $A$ correspondant tel que
pour $x>A$, $\ell - 1 \leq f(x) \leq \ell +1$. Nous venons de montrer
que $f$ est born\'ee ``\`a l'infini''.
La fonction $f$ est continue sur l'intervalle ferm\'e born\'e $[0,A]$,
donc $f$ est born\'ee sur cet intervalle: il existe $m,M$ tels que
pour tout $x\in [0,A]$, $m \leq f(x) \leq M$.
En prenant $M' = \max (M,\ell+1)$, et $m' = \min(m,\ell-1)$ nous avons que pour tout $x\in \Rr$,
$m' \leq f(x) \leq M'$. Donc $f$ est born\'ee sur $\Rr$.

La fonction n'atteint pas n\'ecessairement ses bornes: regardez
$f(x) = \frac{1}{1+x}$.
}
}
