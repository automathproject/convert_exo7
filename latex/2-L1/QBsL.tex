\uuid{QBsL}
\exo7id{5397}
\auteur{rouget}
\organisation{exo7}
\datecreate{2010-07-06}
\isIndication{false}
\isCorrection{true}
\chapitre{Continuité, limite et étude de fonctions réelles}
\sousChapitre{Autre}

\contenu{
\texte{
Soit $f$ continue sur $\Rr^+$ telle que, pour tout réel positif $x$, on ait $f(x^2)=f(x)$. Montrer que $f$ est constante sur $\Rr^+$. Trouver un exemple où $f$ n'est pas constante.
}
\reponse{
Soit $x>0$. Pour tout naturel $n$, $f(x)=f(\sqrt{x})=f(x^{1/4})=...=f(x^{1/2^n})$. Or, à $x$ fixé, $\lim_{n\rightarrow +\infty}x^{1/2^n}=1$ et, $f$ étant continue en $1$, on a~:

$$\forall x>0,\;f(x)=\lim_{n\rightarrow +\infty}f(x^{1/2^n})=f(1).$$

$f$ est donc constante sur $]0,+\infty[$, puis sur $[0,+\infty[$ par continuité de $f$ en $0$.

Pour $x\geq0$, posons $f(x)=0$ si $x\neq1$ et $f(x)=1$ si $x=1$. Pour $x\geq0$, on a $x^2=1\Leftrightarrow x=1$. $f$ vérifie donc~:~$\forall x\geq0,\;f(x^2)=f(x)$, mais $f$ n'est pas constante sur $\Rr^+$.
}
}
