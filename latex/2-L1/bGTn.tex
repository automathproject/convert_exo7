\uuid{bGTn}
\exo7id{1937}
\auteur{gineste}
\datecreate{2001-11-01}
\isIndication{false}
\isCorrection{true}
\chapitre{Série numérique}
\sousChapitre{Série à  termes positifs}

\contenu{
\texte{
Étudier, suivant les valeurs de $p \in \Nn$, la nature de la série de terme général 
\[
u_n = \frac{1! + 2! + \cdots + n!}{(n+p)!} \cdotp
\]
}
\reponse{
\begin{itemize}
\item Pour $p = 0$:
\[ 
u_n = \frac{ 1! + 2! + \cdots + n! }{ n! } = 1 + \frac{ 1! + 2! + \cdots + (n-1)! }{ n! } > 1 
\]
$u_n$ ne tend pas vers $ 0 $ donc, $ \sum u_n $ diverge grossièrement pour $ p=0 $. 

\item Pour $p = 1$:
\[ 
u_n = \frac{1}{ (n+1)! } + \frac{2!}{ (n+1)! } + \cdots + \frac{ (n-1)! }{ (n+1)! } + \frac{n!}{ (n+1)! } 
\]
\[ 
u_n \geq \frac{n!}{ (n+1)! } = \frac{1}{ n + 1 }  \cdotp
\]
Or $\sum \frac{1}{n + 1}$ diverge, donc $\sum u_n$ diverge pour $p = 1$ .

\item Pour $p = 2$:
\[ u_n = \frac{1}{ (n+2)! } + \frac{2!}{ (n+2)! } + \cdots + \frac{ (n-1)! }{ (n+2)! } + \frac{n!}{ (n+2)! } 
\]
On serait tenté de dire que l'on a une somme de séries convergentes, donc $\sum u_n$ converge. 
Pas de chance, le nombre de terme croît en fonction de $n$, donc à l'infini, on en a une infinité 
et on ne peut rien conclure.
\[ 
u_n = \sum_{k=1}^{n} \frac{k!}{ (n + 2)!} = \sum_{k=1}^{n-1} \frac{k!}{(n + 2)!} + \frac{n!}{(n + 2)!}  
   \leq \frac{n (n - 1)!}{(n + 2)!} + \frac{n!}{(n + 2)!} \]
\[ u_n \leq 2\frac{n!}{(n + 2)!} = \frac{2}{(n + 1)(n + 2)} \sim \frac{2}{n^2} \]
Or $\sum \frac{1}{n^2}$ converge, donc $\sum u_n$ converge pour $p = 2$. 

\item Pour $p \geq 3$:
\[ 
u_n = \frac{ 1! + 2! + \cdots + n! }{( n + p)! } \leq \frac{ n \, n! }{ (n + p)! } = \frac{ n \, n! }{n! (n+1) \cdots (n+p) } 
\]
En simplifiant par $ n! $ et en posant $ u_n \leq \frac{ n }{ (n+1) \cdots (n+p)}$ et
\[ 
\frac{ n }{ (n+1) \cdots (n+p)} \thicksim  \frac{n}{n^p} 
= \frac{1}{n^{p - 1}} \text{ avec } p \geq 3
\] 
Or $\sum \frac{1}{n^{p - 1}}$ est une série de Riemann convergente car $p-1 \geq 2$, 
donc $\sum u_n $ converge pour $p \geq 3$. 

Note: on peut aussi remarquer que $u_n$ (quand $p\geq3$) est majoré par $u_n$ (quand $p=2$), or ce dernier est convergent.
\end{itemize}
\medskip

(\emph{Corrigé de Eugène Ndiaye})
}
}
