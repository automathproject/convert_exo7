\uuid{WtWI}
\exo7id{4473}
\auteur{quercia}
\datecreate{2010-03-14}
\isIndication{false}
\isCorrection{true}
\chapitre{Série numérique}
\sousChapitre{Autre}

\contenu{
\texte{

}
\begin{enumerate}
    \item \question{Soient $k,p \in \N$ avec $k \le p$.
    Montrer que $\sum_{n=k}^p \frac{{C_n^k}-C_n^{k+1}}{2^n}
    = \frac{C_{p+1}^{k+1}}{2^p}$.}
\reponse{Récurrence sur $p$.}
    \item \question{Soit $(u_n)$ une série convergente.
    On pose $v_n = \frac1{2^n} \sum_{p=0}^n C_n^pu_p$.
    Montrer que la série $(v_n)$ est convergente.}
\reponse{Transformation d'Abel et interversion de sommations :
         $\sum_{n=0}^p v_n
         = \sum_{k=0}^p \frac{C_{p+1}^{k+1}}{2^p}\sum_{n=0}^k u_n$.
         \par Théorème de Césaro $ \Rightarrow  \sum v_n = 2\sum u_n$ .}
\end{enumerate}
}
