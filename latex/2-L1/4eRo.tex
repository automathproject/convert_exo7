\uuid{4eRo}
\exo7id{505}
\auteur{bodin}
\organisation{exo7}
\datecreate{1998-09-01}
\isIndication{true}
\isCorrection{true}
\chapitre{Suite}
\sousChapitre{Convergence}

\contenu{
\texte{
Soit $(u_n)_{n\in\N}$ une suite de $\R$. Que pensez-vous des
propositions suivantes :
\par\noindent $\bullet$ Si $(u_{n})_n$ converge vers un r\'eel $\ell$ alors $(u_{2n})_n$ et $(u_{2n+1})_n$
convergent vers $\ell$.
\par\noindent  $\bullet$ Si $(u_{2n})_n$ et $(u_{2n+1})_n$ sont convergentes, il en est
de m\^{e}me de $(u_{n})_n$.
\par\noindent $\bullet$ Si $(u_{2n})_n$ et $(u_{2n+1})_n$ sont convergentes, de m\^{e}me
limite $\ell$, il en est de m\^{e}me de $(u_{n})_n$.
}
\indication{Dans l'ordre c'est vrai, faux et vrai. Lorsque c'est faux chercher un contre-exemple, lorsque c'est vrai il faut le prouver.}
\reponse{
Vrai. Toute sous-suite d'une suite convergente est convergente
et admet la m\^eme limite (c'est un résultat du cours).
Faux. Un contre-exemple est la suite $(u_n)_n$ d\'efinie
par $u_n = (-1)^n$. Alors $(u_{2n})_n$ est la suite constante
(donc convergente) de valeur $1$, et $(u_{2n+1})_n$ est constante
de valeur $-1$. Cependant la suite $(u_n)_n$ n'est pas
convergente.
Vrai.
La convergence de la suite $(u_n)_n$ vers $\ell$, que nous
souhaitons d\'emontrer, s'\'ecrit :
$$ \forall \epsilon > 0\  \ \exists N \in \Nn \ \text{\  tel que\ \  }
(n \geqslant N \Rightarrow |u_n-\ell| < \epsilon).$$ 
Fixons $\epsilon >
0$. Comme, par hypoth\`ese, la suite $(u_{2p})_p$  converge vers
$\ell$ alors il existe $N_1$ tel
$$ 2p \geqslant N_1 \Rightarrow |u_{2p}-\ell| < \epsilon.$$
Et de m\^eme, pour la suite $(u_{2p+1})_p$ il existe $N_2$ tel que
$$ 2p+1 \geqslant N_2 \Rightarrow |u_{2p+1}-\ell| < \epsilon.$$
Soit $N = \max(N_1,N_2)$, alors
$$n \geqslant N\Rightarrow |u_{n}-\ell| < \epsilon.$$
Ce qui prouve la convergence de $(u_n)_n$ vers $\ell$.
}
}
