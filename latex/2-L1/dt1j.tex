\uuid{dt1j}
\exo7id{1284}
\auteur{gourio}
\datecreate{2001-09-01}
\isIndication{false}
\isCorrection{false}
\chapitre{Calcul d'intégrales}
\sousChapitre{Intégrale impropre}

\contenu{
\texte{
Soit $f$ une application continue de ${\Rr}^{+}$ dans ${\Rr}$ et $F$ de
${\Rr}^{+*}$ dans ${\Rr}$ d\'{e}finie par :
$$\forall x\in {\Rr}^{+*},F(x)=\frac{1}{x}\int_{0}^{x}f(t)dt. $$
}
\begin{enumerate}
    \item \question{Montrer que si $f$ admet une limite $\ell$ en $+\infty $, alors $F$\ a aussi la
limite $\ell$ en $+\infty .$}
    \item \question{Donner un exemple o\`{u} $f$ n'a pas de limite en $+\infty $ mais o\`{u} $F$
tend vers $0$.}
    \item \question{Montrer que si $f\rightarrow \infty $ quand $x\rightarrow \infty $, alors
$F\rightarrow \infty $ quand $x\rightarrow \infty .$}
\end{enumerate}
}
