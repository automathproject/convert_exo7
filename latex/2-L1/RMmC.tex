\uuid{RMmC}
\exo7id{5465}
\auteur{rouget}
\organisation{exo7}
\datecreate{2010-07-10}
\isIndication{false}
\isCorrection{true}
\chapitre{Calcul d'intégrales}
\sousChapitre{Primitives diverses}

\contenu{
\texte{
Soit $f(t)=\frac{t^2}{e^t-1}$ si $t\neq0$ et $0$ si $t=0$.
}
\begin{enumerate}
    \item \question{Vérifier que $f$ est continue sur $\Rr$.}
\reponse{$f$ est continue sur $\Rr^*$ en tant que quotient de fonctions continues sur $\Rr^*$ dont le dénominateur ne s'annule pas sur $\Rr^*$. D'autre part, quand $t$ tend vers $0$, $f(t)\sim\frac{t^2}{t}=t$ et $\lim_{t\rightarrow 0,\;t\neq0}f(t)=0=f(0)$. Ainsi, $f$ est continue en $0$ et donc sur $\Rr$.}
    \item \question{Soit $F(x)=\int_{0}^{x}f(t)\;dt$. Montrer que $F$ a une limite réelle $\ell$ quand $x$ tend vers $+\infty$ puis que $\ell=2\lim_{n\rightarrow +\infty}\sum_{k=1}^{n}\frac{1}{k^3}$.}
\reponse{$f$ est continue sur $\Rr$ et donc $F$ est définie et de classe $C^1$ sur $\Rr$. De plus, $F'=f$ est positive sur $[0,+\infty[$, de sorte que $F$ est croissante sur $[0,+\infty[$. On en déduit que $F$ admet en $+\infty$ une limite dans $]-\infty,+\infty]$.

Vérifions alors que $F$ est majorée sur $\Rr$. On contate que $t^2.\frac{t^2}{e^t-1}$ tend vers $0$ quand $t$ tend vers $+\infty$, d'après un théorème de croissances comparées. Par suite, il existe un réel $A$ tel que pour $t\geq A$, 
$0\leq t^2.\frac{t^2}{e^t-1}\leq1$ ou encore $0\leq f(t)\leq\frac{1}{t^2}$. Pour $x\geq A$, on a alors

\begin{align*}\ensuremath
F(x)&=\int_{0}^{A}f(t)\;dt+\int_{A}^{x}\frac{t^2}{e^t-1}\;dt\leq\int_{0}^{A}f(t)\;dt+\int_{A}^{x}\frac{1}{t^2}\;dt\\
 &=\int_{0}^{A}f(t)\;dt+\frac{1}{A}-\frac{1}{x}\leq\int_{0}^{A}f(t)\;dt+\frac{1}{A}.
\end{align*}

$F$ est croissante et majorée et donc a une limite réelle $\ell$ quand $n$ tend vers $+\infty$.

Soit $n\in\Nn^*$. Pour $t\in]0,+\infty[$,

\begin{align*}
f(t)&=t^2e^{-t}\frac{1}{1-e^{-t}}=t^2e^{-t}(\sum_{k=0}^{n-1}(e^{-t})^k+\frac{(e^{-t})^{n}}{1-e^{-t}})\\
 &=\sum_{k=0}^{n-1}t^2e^{-(k+1)t}+\frac{t^2e^{-t}}{1-e^{-t}}e^{-nt}=\sum_{k=1}^{n}t^2e^{-kt}+f_n(t)\;(*),
\end{align*}

où $f_n(t)=\frac{t^2e^{-t}}{1-e^{-t}}e^{-nt}$ pour $t>0$. En posant de plus $f_n(0)=0$, d'une part, $f_n$ est continue sur $[0,+\infty[$ et d'autre part, l'égalité $(*)$ reste vraie quand $t=0$. En intégrant, on obtient

$$\forall x\in[0,+\infty[,\;\forall n\in\Nn^*,\;F(x)=\sum_{k=1}^{n}\int_{0}^{x}t^2e^{-kt}\;dt+\int_{0}^{x}f_n(t)\;dt\;(**).$$

Soient alors $k\in\Nn^*$ et $x\in[0,+\infty[$. Deux intégrations par parties fournissent~:

\begin{align*}\ensuremath
\int_{0}^{x}t^2e^{-kt}\;dt&=\left[-\frac{1}{k}t^2e^{-kt}\right]_{0}^{x}+\frac{2}{k}\int_{0}^{x}te^{-kt}\;dt
=-\frac{1}{k}x^2e^{-kx}+\frac{2}{k}(\left[-\frac{1}{k}te^{-kt}\right]_{0}^{x}+\frac{1}{k}\int_{0}^{x}e^{-kt}\;dt)\\
 &=-\frac{1}{k}x^2e^{-kx}-\frac{2}{k^2}xe^{-kx}-\frac{2}{k^3}e^{-kx}+\frac{2}{k^3}.
\end{align*}

Puisque $k>0$, quand $x$ tend vers $+\infty$, on obtient $\lim_{x\rightarrow +\infty}\int_{0}^{x}t^2e^{-kt}\;dt=\frac{2}{k^3}$. On fait alors tendre $x$ vers $+\infty$ dans $(**)$ et on obtient

$$\forall n\in\Nn^*,\;\ell-2\sum_{k=1}^{n}\frac{1}{k^3}=\lim_{x\rightarrow +\infty}\int_{0}^{x}f_n(t)\;dt\;(***).$$

Vérifions enfin que $\lim_{n\rightarrow +\infty}\left(\lim_{x\rightarrow +\infty}\int_{0}^{x}f_n(t)\;dt\right)=0$. La fonction $t\mapsto\frac{t^2e^{-t}}{1-e^{-t}}$ est continue sur $]0,+\infty[$, se prolonge par continuité en $0$ et a une limite réelle en $+\infty$. On en déduit que cette fonction est bornée sur $]0,+\infty[$. Soit $M$ un majorant de cette fonction sur $]0,+\infty[$. Pour $x\in[0,+\infty[$ et $n\in\Nn^*$, on a alors

$$0\leq\int_{0}^{x}f_n(t)\;dt\leq M\int_{0}^{x}e^{-nt}\;dt=\frac{M}{n}(1-e^{-nx}).$$

A $n\in\Nn^*$ fixé, on passe à la limite quand $n$ tend vers $+\infty$ et on obtient

$$0\leq\lim_{x\rightarrow +\infty}\int_{0}^{x}f_n(t)\;dt\leq\frac{M}{n},$$

puis on passe à la limite quand $n$ tend vers $+\infty$ et on obtient

$$\lim_{n\rightarrow +\infty}\left(\lim_{x\rightarrow +\infty}\int_{0}^{x}f_n(t)\;dt\right)=0.$$

Par passage à la limite quand $x$ tend vers $+\infty$ puis quand $n$ tend vers $+\infty$ dans $(***)$, on obtient enfin

$$\lim_{x\rightarrow +\infty}\int_{0}^{x}\frac{t^2}{e^t-1}\;dt=2\lim_{n\rightarrow +\infty}\sum_{k=1}^{n}\frac{1}{k^3}.$$}
\end{enumerate}
}
