\uuid{4pUD}
\exo7id{640}
\auteur{bodin}
\organisation{exo7}
\datecreate{1998-09-01}
\isIndication{false}
\isCorrection{false}
\chapitre{Continuité, limite et étude de fonctions réelles}
\sousChapitre{Continuité : théorie}

\contenu{
\texte{
Soit $f$ une fonction de $\lbrack a,b \rbrack$ dans $\lbrack a,b\rbrack$
telle que pour tout $x$ et $x'$ ($x\neq x'$) de $\lbrack a,b \rbrack$ on ait :
$\ \ \vert f(x)-f(x')\vert<\vert x-x'\vert.$
}
\begin{enumerate}
    \item \question{Montrer que $f$ est continue sur $\lbrack a,b\rbrack$.}
    \item \question{Montrer que l'\'equation $f(x)=x$ admet une et une seule solution
\mbox{dans $\lbrack a,b\rbrack$.}
 (On pourra introduire la fonction: $x\mapsto g(x)=f(x)-x$).}
\end{enumerate}
}
