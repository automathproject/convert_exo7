\uuid{9XZv}
\exo7id{2090}
\auteur{bodin}
\datecreate{2008-02-04}
\isIndication{false}
\isCorrection{true}
\chapitre{Calcul d'intégrales}
\sousChapitre{Polynôme en sin, cos ou en sh, ch}

\contenu{
\texte{
Calculer les primitives suivantes, en pr\'ecisant si n\'ecessaire les
intervalles de validit\'e des calculs:
$$
\begin{array}{llll}
\textbf{a)~} \displaystyle \int \sin ^8x\cos ^3x d x \quad & \textbf{b)~} \displaystyle \int
\cos^4x d x \quad  & \textbf{c)~} \displaystyle \int \cos ^{2003}x\sin x d x & \textbf{d)~}
\displaystyle \int \frac {1}{2+\sin x+\cos x}d x \\
\textbf{e)~}\displaystyle \int \frac 1{\sin x}d x & \textbf{f)~} \displaystyle \int \frac
1{\cos x}d x & \textbf{g)~} \displaystyle \int \frac{3-\sin x}{2\cos x+3\tan x}d x
\quad & \textbf{h)~} \displaystyle \int \frac 1{7+\tan x}d x
\end{array}$$
}
\reponse{
a-$\int \sin ^8x\cos ^3xdx=\frac 19\sin ^9x-\frac 1{11}\sin ^{11}x+c$ sur $%
\R$.

b-$\int \cos ^4xdx=\frac 1{32}\sin 4x+\frac 14\sin 2x+\frac 38x+c$ sur $\R$.

c-$\int \cos ^{2003}x\sin xdx=-\frac 1{2004}\cos ^{2004}x+c$ sur $\R$.

d-$\int \frac 1{\sin x}dx=\frac 12\ln \left| \frac{1-\cos x}{1+\cos x}%
\right| +c=\ln \left| \tan \frac x2\right| +c$ sur $\left] k\pi ,\left(
k+1\right) \pi \right[ $ (changement de variable $u=\cos x$ ou $u=\tan \frac
x2$).

e-$\int \frac 1{\cos x}dx=\frac 12\ln \left| \frac{1+\sin x}{1-\sin x}%
\right| +c=\ln \left| \tan \left( \frac x2+\frac \pi 4\right) \right| +c$
sur $\left] -\frac \pi 2+k\pi ,\frac \pi 2+k\pi \right[ $ (changement de
variable $u=\sin x$ ou $u=\tan \frac x2$).

f-$\int \frac{3-\sin x}{2\cos x+3\tan x}dx=-\frac 15\ln \left( 2-\sin
x\right) +\frac 7{10}\ln \left| 1+2\sin x\right| +c$ sur $\R\setminus
\left\{ \frac{2\pi }3\left[ 2\pi \right] ,-\frac{2\pi }3\left[ 2\pi \right]
\right\} $ (changement de variable $u=\sin x$).

g-$\int \frac 1{7+\tan x}dx=\frac 7{50}x+\frac 1{50}\ln \left| \tan
x+7\right| +\frac 1{50}\ln \left| \cos x\right| +c$ sur $\R\setminus\left\{ \arctan \left( -7\right) +k\pi\,,\,
\frac{\pi}{2}+k\pi\,, \;k\in
Z\right\} $ (changement de variable $u=\tan x$).

h-$\int \frac {1}{2+\sin
x+\cos x}dx=\sqrt{2}\Arctan\left(\frac{1+\tan(x/2)}{\sqrt{2}}\right)+c$ sur
$\R\setminus\{k\pi\,,\;k\in\Z\}$ (changement de variable $u=\tan
(x/2)$).
}
}
