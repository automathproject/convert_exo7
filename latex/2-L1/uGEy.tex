\uuid{uGEy}
\exo7id{5089}
\auteur{rouget}
\datecreate{2010-06-30}
\isIndication{false}
\isCorrection{true}
\chapitre{Fonctions circulaires et hyperboliques inverses}
\sousChapitre{Fonctions circulaires inverses}

\contenu{
\texte{
Calculer $\Arctan\frac{1}{2}+\Arctan\frac{1}{5}+\Arctan\frac{1}{8}$.
}
\reponse{
$0\leq\Arctan\frac{1}{2}+\Arctan\frac{1}{5}<\Arctan1+\Arctan1=\frac{\pi}{2}$ et

$$\tan\left(\Arctan\frac{1}{2}+\Arctan\frac{1}{5}\right)=\frac{\frac{1}{2}+\frac{1}{5}}{1-\frac{1}{2}\times\frac{1}{5}}
=\frac{7}{9}.$$
Comme $\Arctan\frac{1}{2}+\Arctan\frac{1}{5}\in[0,\frac{\pi}{2}[$, on a donc
$\Arctan\frac{1}{2}+\Arctan\frac{1}{5}=\Arctan\frac{7}{9}$. De même,
$\Arctan\frac{7}{9}+\Arctan\frac{1}{8}\in[0,\frac{\pi}{2}[$ et

$$\tan\left(\Arctan\frac{7}{9}+\Arctan\frac{1}{8}\right)=\frac{\frac{7}{9}+\frac{1}{8}}{1-\frac{7}{9}\times\frac{1}{8}}=\frac{65}{65}
=1,$$
et donc $\Arctan\frac{7}{9}+\Arctan\frac{1}{8}=\Arctan1=\frac{\pi}{4}$. Finalement,

\begin{center}
\shadowbox{
$\Arctan\frac{1}{2}+\Arctan\frac{1}{5}+\Arctan\frac{1}{8}=\frac{\pi}{4}.$
}
\end{center}
}
}
