\uuid{FLjr}
\exo7id{558}
\auteur{cousquer}
\datecreate{2003-10-01}
\isIndication{false}
\isCorrection{false}
\chapitre{Suite}
\sousChapitre{Suite définie par une relation de récurrence}

\contenu{
\texte{

}
\begin{enumerate}
    \item \question{Étudier dans $\mathbb{C}$ une suite $(u_n)$ telle que 
$\forall n \in\mathbb{N}^\ast,\;u_{n+1}=u_n^2$. Discuter suivant~$u_0$.}
    \item \question{On considère dans $\mathbb{C}$ une suite $(v_n)$ telle que 
$\forall n,\; v_{n+1}={1\over2}\bigl(v_n+{A\over v_n}\bigr)$
où $A$ est un nombre complexe non nul donné. Étudier l'existence et la
convergence de cette suite suivant les valeurs de $v_0$. On pourra noter $a$
une des racines carrées de $A$ et poser $w_n={v_n-a\over v_n+a}$.}
\end{enumerate}
}
