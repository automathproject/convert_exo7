\uuid{iYHW}
\exo7id{2087}
\auteur{bodin}
\organisation{exo7}
\datecreate{2008-02-04}
\isIndication{true}
\isCorrection{true}
\chapitre{Calcul d'intégrales}
\sousChapitre{Théorie}

\contenu{
\texte{
Soit $f:[0,1]\rightarrow {\R}$ une application strictement
croissante telle que $f(0)=0,\  f(1)=1$. Calculer :
$$\lim_{n\rightarrow \infty }\int_{0}^{1}f^{n}(t) d t. $$
}
\indication{Il s'agit de montrer que la limite vaut $0$. Pour un $\alpha >0$ fix\'e on s\'eparera l'int\'egrale en deux partie selon que $f$ est plus petit ou plus grand que $1-\alpha$.}
\reponse{
Soit $\alpha>0$ fix\'e. Soit $0<x_0<1$ tel que pour tout $x \in [0,x_0]$, $f(x) \leqslant 1-\alpha$. 
Ce $x_0$ existe bien car $f$ est strictement croissante
et $f(0)=0$, $f(1)=1$.
S\'eparons l'int\'egrale en deux :
\begin{align*}
\int_{0}^{1}f^{n}(t) dt 
  &= \int_{0}^{x_0}f^{n}(t) dt+\int_{x_0}^{1}f^{n}(t) dt \\
  &\leqslant \int_{0}^{x_0} (1-\alpha)^n dt + \int_{x_0}^{1} 1^n dt \\
  &\leqslant x_0(1-\alpha)^n+(1-x_0) \\
  &\leqslant (1-\alpha)^n+(1-x_0) \quad \text{ car } x_0 \leqslant 1 \\
\end{align*}

Soit maintenant donn\'e un $\epsilon >0$, on choisit $\alpha>0$ tel que 
$1-x_0 \leqslant \frac \epsilon 2$ (en remarquant que si $\alpha \to 0$ alors $x_0(\alpha) \to 1$),
puis il existe $n$ assez grand tel que $(1-\alpha)^n \leqslant \frac \epsilon 2$.
Donc pour tout $\epsilon>0$ il existe $n$ assez grand tel que $\int_{0}^{1}f^{n}(t) dt \leqslant \epsilon$ .
Donc $\int_{0}^{1}f^{n}(t) dt \to 0$.
}
}
