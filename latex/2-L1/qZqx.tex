\uuid{qZqx}
\exo7id{4489}
\auteur{quercia}
\datecreate{2010-03-14}
\isIndication{false}
\isCorrection{true}
\chapitre{Série numérique}
\sousChapitre{Familles sommables}

\contenu{
\texte{
Soit $(r_n)_{n\ge 1}$ une énumération des rationnels.
On note $I_n = {\Bigl]r_n-\frac1{n^2}, r_n+\frac1{n^2}\Bigr[}$, $E = \bigcup_{n=1}^\infty I_n$ et
$F = \R\setminus E$. Montrer que $F\ne \varnothing$ (ceci est choquant vu que les
éléments de~$F$ sont, par définition, "loin" de chaque rationnel, pourtant c'est vrai) .
}
\reponse{
Soit $[a,b]$ de longueur supérieure ou égale à $2\zeta(2)$ et
$F_n = [a,b]\setminus(I_1\cup\dots\cup I_n)$. Alors $(F_n)$ vérifie le
théorème des fermés emboités dans un compact.
}
}
