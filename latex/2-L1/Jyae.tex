\uuid{Jyae}
\exo7id{5439}
\auteur{rouget}
\datecreate{2010-07-10}
\isIndication{false}
\isCorrection{true}
\chapitre{Développement limité}
\sousChapitre{Calculs}

\contenu{
\texte{
Soit $f(x)=1+x+x^2+x^3\sin\frac{1}{x^2}$ si $x\neq0$ et $1$ si $x=0$.
}
\begin{enumerate}
    \item \question{Montrer que $f$ admet en $0$ un développement limité d'ordre $2$.}
\reponse{$x^3\sin\frac{1}{x^2}\underset{x\rightarrow0}{=}O(x^3)$ et en particulier $x^3\sin\frac{1}{x}\underset{x\rightarrow0}{=}o(x^2)$. Donc, en tenant compte de $f(0)=1$,

\begin{center}
$f(x)\underset{x\rightarrow0}{=}1+x+x^2+o(x^2)$.
\end{center}
$f$ admet en $0$ un développement limité d'ordre $2$.}
    \item \question{Montrer que $f$ est dérivable sur $\Rr$.}
\reponse{$f(x)\underset{x\rightarrow0}{=}1+x+o(x)$. Donc, $f$ admet en $0$ un développement limité d'ordre $1$. On en déduit que $f$ est continue et dérivable en $0$ avec $f(0)=f'(0)=1$. $f$ est d'autre part dérivable sur $\Rr^*$ en vertu de théorèmes généraux (et donc sur $\Rr$) et pour $x\neq 0$, $f'(x)=1+2x+3x^2\sin\frac{1}{x^2}-2\cos\frac{1}{x^2}$.}
    \item \question{Montrer que $f'$ n'admet en $0$ aucun développement limité d'aucun ordre que ce soit.}
\reponse{$f'$ est définie sur $\Rr$ mais n'a pas de limite en $0$. $f'$ n'admet donc même pas un développement limité d'ordre $0$ en $0$.}
\end{enumerate}
}
