\uuid{Xjzo}
\exo7id{4044}
\auteur{quercia}
\organisation{exo7}
\datecreate{2010-03-11}
\isIndication{false}
\isCorrection{true}
\chapitre{Développement limité}
\sousChapitre{Equivalents}

\contenu{
\texte{
Donner des équivalents simples pour les fonctions suivantes :
}
\begin{enumerate}
    \item \question{$2e^x - \sqrt{1+4x} - \sqrt{1+6x^2}$,            en $0$}
\reponse{Le dl à l'ordre $3$ en $0$ est 
$$2e^x - \sqrt{1+4x} - \sqrt{1+6x^2} = -\frac {11x^3}3 + o(x^3)$$
donc 
$$2e^x - \sqrt{1+4x} - \sqrt{1+6x^2} \sim -\frac {11x^3}3.$$}
    \item \question{$(\cos x)^{\sin x} - (\cos x)^{\tan x}$,         en $0$}
\reponse{De même 
$$(\cos x)^{\sin x} - (\cos x)^{\tan x} \sim \frac {x^5}4.$$}
    \item \question{$\arctan x + \arctan \frac 3x -\frac {2\pi}3$, en $\sqrt3$}
\reponse{On pose $h=x-\sqrt 3$ alors 
$$ \arctan x + \arctan \frac 3x -\frac {2\pi}3 = -\frac {h^2}{8\sqrt3} + o(h^2)$$
donc 
$$ \arctan x + \arctan \frac 3x -\frac {2\pi}3 \sim -\frac {(x-\sqrt 3)^2}{8\sqrt3}.$$}
    \item \question{$\sqrt{x^2+1} -2\sqrt[3]{x^3+x} + \sqrt[4]{x^4+x^2}$,  en $+\infty$}
\reponse{En $+\infty$
$$\sqrt{x^2+1} -2\sqrt[3]{x^3+x} + \sqrt[4]{x^4+x^2}= \frac 1{12x} + o(\frac 1x)$$
donc $$\sqrt{x^2+1} -2\sqrt[3]{x^3+x} + \sqrt[4]{x^4+x^2} \sim \frac 1{12x}.$$}
    \item \question{$\Argch\left(\frac1{\cos x}\right)$,            en $0$}
\reponse{Il faut distinguer les cas $x>0$ et $x<0$ pour trouver :
$$\Argch\left(\frac1{\cos x}\right) \sim |x|.$$}
\end{enumerate}
}
