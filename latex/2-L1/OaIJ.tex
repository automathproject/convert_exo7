\uuid{OaIJ}
\exo7id{4300}
\auteur{quercia}
\organisation{exo7}
\datecreate{2010-03-12}
\isIndication{false}
\isCorrection{false}
\chapitre{Calcul d'intégrales}
\sousChapitre{Intégrale impropre}

\contenu{
\texte{
Soit $f : {[0,+\infty[} \to \R$ continue telle que $ \int_{t=0}^{+\infty} f(t)\,d t$
converge.
}
\begin{enumerate}
    \item \question{Si $f(x) \to L$ lorsque $x\to{+\infty}$, combien vaut $L$ ?}
    \item \question{Donner un exemple où $f$ n'a pas de limite en $+\infty$.}
    \item \question{Si $f$ est décroissante, montrer que $xf(x) \to 0$ lorsque $x\to{+\infty}$.}
\end{enumerate}
}
