\uuid{yg4B}
\exo7id{3999}
\auteur{quercia}
\datecreate{2010-03-11}
\isIndication{false}
\isCorrection{true}
\chapitre{Dérivabilité des fonctions réelles}
\sousChapitre{Fonctions convexes}

\contenu{
\texte{
Soit $f : {\R} \to {\R}$ continue. On suppose que :
$\forall\ x \in \R,\
D^2f(x) = \lim_{h \to 0} \frac {f(x+h)+f(x-h)-2f(x)}{h^2}$
existe.
}
\begin{enumerate}
    \item \question{Si $f$ est de classe $\mathcal{C}^2$, calculer $D^2f(x)$.}
    \item \question{Soit $f$ quelconque et $a < b < c$ tels que
      $f(a) = f(b) = f(c)$.

      Montrer qu'il existe $x \in {]a,c[}$ tq $D^2f(x) \le 0$.
      


On suppose à présent que : $\forall\ x \in \R, D^2f(x) \ge 0$.}
    \item \question{Soient $a < b < c$ et
      $P$ le polynôme de degré inférieur ou égal à 2 coïncidant avec $f$
      aux points $a,b,c$. Montrer que $P'' \ge 0$.}
    \item \question{Calculer $P''$ en fonction de $a,b,c$ et $f(a),f(b),f(c)$.
      En déduire que $f$ est convexe.}
\reponse{
Prendre $x$ tel que $f(x)$ soit maximal.
}
\end{enumerate}
}
