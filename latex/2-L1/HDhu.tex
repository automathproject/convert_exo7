\uuid{HDhu}
\exo7id{715}
\auteur{bodin}
\organisation{exo7}
\datecreate{1998-09-01}
\isIndication{true}
\isCorrection{true}
\chapitre{Dérivabilité des fonctions réelles}
\sousChapitre{Théorème de Rolle et accroissements finis}

\contenu{
\texte{
Montrer que le polyn\^ome $P_n$ d\'efini par
$$P_n(t)=\left[ \left( 1-t^2 \right)^n \right]^{(n)}$$
est un polyn\^ome de degr\'e $n$ dont les racines sont
 r\'eelles, simples, et appartiennent \`a $[-1,1]$.
}
\indication{Il faut appliquer le th\'eor\`eme de Rolle une fois au polyn\^ome $(1-t^2)^n$,
puis deux fois \`a sa d\'eriv\'ee premi\`ere, puis trois fois \`a sa d\'eriv\'ee seconde,...}
\reponse{
$Q_n(t) = (1-t^2)^n$ est un polyn\^ome de degr\'e $2n$, on le d\'erive $n$ fois, on obtient un polyn\^ome de degr\'e $n$.
Les valeurs $-1$ et $+1$ sont des racines d'ordre $n$ de $Q_n$, donc
$Q_n(1)=Q_n'(1)=\ldots = Q_n^{(n-1)}(1)=0$. M\^eme chose en $-1$.
Enfin $Q(-1)=0=Q(+1)$ donc d'apr\`es le th\'eor\`eme de Rolle il existe $c \in ]-1,1[$ telle que $Q_n'(c)=0$. 

Donc $Q_n'(-1)=0$, $Q_n'(c)=0$, $Q_n'(-1)=0$. En appliquant le th\'eor\`eme de Rolle deux fois 
(sur $[-1,c]$ et sur $[c,+1]$), on obtient l'existence
de racines $d_1,d_2$ pour $Q_n''$, qui s'ajoutent aux racines $-1$ et $+1$.


On continue ainsi par r\'ecurrence. On obtient pour $Q_n^{(n-1)}$,
$n+1$ racines: $-1, e_1,\ldots, e_{n-1}, +1$. Nous appliquons le th\'eor\`eme de Rolle $n$ fois. Nous obtenons $n$ racines pour 
$P_n = Q_n^{(n)}$.  Comme un polyn\^ome de degr\'e $n$ a au plus
$n$ racines, nous avons obtenu toutes les racines. Par constructions ces racines sont r\'eelles distinctes, donc simples.
}
}
