\uuid{WmCh}
\exo7id{5399}
\auteur{rouget}
\datecreate{2010-07-06}
\isIndication{false}
\isCorrection{true}
\chapitre{Continuité, limite et étude de fonctions réelles}
\sousChapitre{Autre}

\contenu{
\texte{
Trouver tous les morphismes continus de $(\Rr,+)$.
}
\reponse{
Soit $f$ un morphisme de $(\Rr,+)$, c'est-à-dire que $f$ est une application de $\Rr$ dans $\Rr$ vérifiant 
$$\forall(x,y)\in\Rr^2,\;f(x+y)=f(x)+f(y).$$

On sait déjà $f(0)=f(0+0)=f(0)+f(0)$ et donc $f(0)=0$. Puis, pour $x$ réel donné, $f(-x)+f(x)=f(-x+x)=f(0)=0$ et donc, pour tout réel $x$, $f(-x)=-f(x)$ ($f$ est donc impaire). On a aussi $n\in\Nn^*$ et $x\in\Rr$, $f(nx)=f(x)+...+f(x)=nf(x)$. De ce qui précède, on déduit~:

$$\forall x\in\Rr,\;\forall n\in\Zz,\;f(nx)=nf(x).$$

Soit $a=f(1)$. D'après ce qui précède, $\forall n\in\Zz,\;f(n)=f(n.1)=nf(1)=an$.

Puis, pour $n\in\Nn^*$, $nf(\frac{1}{n})=f(n\frac{1}{n})=f(1)=a$ et donc $\forall n\in\Nn^*,\;f(\frac{1}{n})=a\frac{1}{n}$.

Puis, pour $p\in\Zz$ et $q\in\Nn^*$, $f(\frac{p}{q})=pf(\frac{1}{q})=pa\frac{1}{q}=a\frac{p}{q}$.

Finalement, 

$$\forall r\in\Qq,\;f(r)=ar.$$

Maintenant, si l'on n'a pas l'hypothèse de continuité, on ne peut aller plus loin. Supposons de plus que $f$ soit continue sur $\Rr$.

Soit $x$ un réel. Puisque $\Qq$ est dense dans $\Rr$, il existe une suite $(r_n)_{n\in\Nn}$ de rationnels, convergente de limite $x$.

$f$ étant continue en $x$, on a~:~

$$f(x)=f(\lim_{n\rightarrow +\infty}r_n)=\lim_{n\rightarrow +\infty}f(r_n)=\lim_{n\rightarrow +\infty}ar_n=ax.$$

Donc, si $f$ est un morphisme continu de $(\Rr,+)$, $f$ est une application linéaire de $\Rr$ dans $\Rr$. Réciproquement, les applications linéaires conviennent.
}
}
