\uuid{p2IX}
\exo7id{5087}
\auteur{rouget}
\datecreate{2010-06-30}
\isIndication{false}
\isCorrection{true}
\chapitre{Fonctions circulaires et hyperboliques inverses}
\sousChapitre{Fonctions circulaires inverses}

\contenu{
\texte{
Existence et calcul de $\int_{0}^{\sin^2x}\Arcsin\sqrt{t}\;dt+\int_{0}^{\cos^2x}\Arccos\sqrt{t}\;dt$.
}
\reponse{
Pour $x$ réel, on pose
$f(x)=\int_{0}^{\sin^2x}\Arcsin\sqrt{t}\;dt+\int_{0}^{\cos^2x}\Arccos\sqrt{t}\;dt$.\\
La fonction $t\mapsto\Arcsin\sqrt{t}$ est continue sur $[0,1]$. Donc, la fonction $y\mapsto\int_{0}^{y}\Arcsin\sqrt{t}\;dt$ est définie et dérivable sur $[0,1]$. De
plus, $x\mapsto\sin^2x$ est définie et dérivable sur $\Rr$ à valeurs dans 
$[0,1]$. Finalement, la fonction $x\mapsto\int_{0}^{\sin^2x}\Arcsin\sqrt{t}\;dt$ est définie
et dérivable sur $\Rr$.
De même, la fonction $t\mapsto\Arccos\sqrt{t}$ est continue sur $[0,1]$. Donc, la fonction $y\mapsto\int_{0}^{y}\Arccos\sqrt{t}\;dt$ est définie et dérivable sur $[0,1]$. De plus, la fonction $x\mapsto\cos^2x$ est définie
et dérivable sur $\Rr$, à valeurs dans $[0,1]$. Finalement, la fonction $x\mapsto\int_{0}^{\cos^2x}\Arccos\sqrt{t}\;dt$ est définie
et dérivable sur $\Rr$.
Donc, $f$ est définie et dérivable sur $\Rr$ et, pour tout réel $x$,

\begin{align*}
f'(x)&=2\sin x\cos x\Arcsin(\sqrt{\sin^2x})-2\sin x\cos x\Arccos(\sqrt{\cos^2x})\\
 &=2\sin x\cos x\left(\Arcsin(|\sin x|)-\Arccos(|\cos x|)\right).
\end{align*}
On note alors que $f$ est $\pi$-pérodique et paire. Pour $x$ élément de
$[0,\frac{\pi}{2}]$, $f'(x)=2\sin x\cos x(x-x)=0$. $f$ est donc constante sur $[0,\frac{\pi}{2}]$ et pour $x$ élément
de $[0,\frac{\pi}{2}]$,
$f(x)=f\left(\frac{\pi}{4}\right)=\int_{0}^{1/2}\Arcsin\sqrt{t}\;dt+\int_{0}^{1/2}\Arccos\sqrt{t}dt=\int_{0}^{1/2}\frac{\pi}{2}\;dt=
\frac{\pi}{4}$. Mais alors, par parité et $\pi$-périodicité,

\begin{center}
\shadowbox{
$\forall
x\in\Rr,\;\int_{0}^{\sin^2x}\Arcsin\sqrt{t}\;dt+\int_{0}^{\cos^2x}\Arccos\sqrt{t}\;dt=\frac{\pi}{4}.$
}
\end{center}
}
}
