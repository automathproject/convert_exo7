\uuid{IgRN}
\exo7id{813}
\auteur{bodin}
\organisation{exo7}
\datecreate{1998-09-01}
\isIndication{false}
\isCorrection{false}
\chapitre{Calcul d'intégrales}
\sousChapitre{Changement de variables}

\contenu{
\texte{
Soit $f:[a,b]\rightarrow \R$ une fonction strictement
croissante et contin\^ument d\'erivable.
On consid\`ere les deux int\'egrales $I_1=\int_a^b f(t)\, dt$ et
$I_2=\int_{f(a)}^{f(b)} f^{-1}(t)\, dt$.
}
\begin{enumerate}
    \item \question{Rappeler pourquoi $f$ admet une fonction r\'eciproque $f^{-1}$.}
    \item \question{Faire le changement de variable $t=f (u)$ dans l'int\'egrale $I_2$.}
    \item \question{Calculer $I_2$ en fonction de $I_1$.}
    \item \question{Faire un dessin faisant appara\^{\i}tre $f$ et $f^{-1}$,
et interpr\'eter ce r\'esultat g\'eom\'etriquement.}
\end{enumerate}
}
