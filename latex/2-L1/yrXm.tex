\uuid{yrXm}
\exo7id{739}
\auteur{bodin}
\datecreate{2001-11-01}
\isIndication{false}
\isCorrection{true}
\chapitre{Dérivabilité des fonctions réelles}
\sousChapitre{Autre}

\contenu{
\texte{
Soit $n \geq 2$ un entier fix\'e et $f:\R^{+} = [0,
+ \infty[ \longrightarrow \R$ la fonction d\'efinie par la formule
suivante:
$$ f (x) = \frac{1 + x^n}{(1 + x)^n}, \ \ x \geq 0.$$
}
\begin{enumerate}
    \item \question{\begin{enumerate}}
\reponse{\begin{enumerate}}
    \item \question{Montrer que $f$ est d\'erivable sur $\R^{+}$ et calculer $f' (x)$
pour $x \geq 0.$}
\reponse{Il est clair que la fonction $f$ est d\'erivable sur $\R^{+}$
puisque c'est une fonction rationnelle sans p\^ole dans cet
intervalle. De plus d'apr\`es la formule de la d\'eriv\'ee d'un
quotient, on obtient pour $x \geq 0$ :
$$ f' (x) = \frac{n (x^{n-1} - 1)}{(1 + x)^{n + 1}}.$$}
    \item \question{En \'etudiant le signe de $f' (x)$ sur $\R^{+},$ montrer que $f$
atteint un minimum sur $\R^{+}$ que l'on d\'eterminera.}
\reponse{Par l'expression pr\'ec\'edente $f' (x)$ est du signe de $x^{n - 1} - 1$ sur $\R^{+}.$ Par
cons\'equent on obtient: $ f' (x) \leq 0$ pour $0 \leq x \leq 1$
et $ f' (x) \geq 0$ pour $x \geq 1.$ Il en r\'esulte que $f$ est
d\'ecroissante sur $[0 , 1]$ et croissante sur $[1, + \infty[$ et
par suite $f$ atteint son minimum sur $\R^{+}$ au point $1$ et ce
minimum vaut $f (1) = 2^{1 - n}.$}
\end{enumerate}
}
