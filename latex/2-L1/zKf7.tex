\uuid{zKf7}
\exo7id{4007}
\auteur{quercia}
\datecreate{2010-03-11}
\isIndication{false}
\isCorrection{true}
\chapitre{Développement limité}
\sousChapitre{Formule de Taylor}

\contenu{
\texte{
Soit $f : {\R^+} \to \R$ de classe $\mathcal{C}^\infty$ telle que $f(0) = 1$,
et : $\forall\ x \ge \frac12,\ f(x) = 0$.
}
\begin{enumerate}
    \item \question{Montrer que $\sup\limits_{\R^+} \bigl|f^{(n)}\bigr| \ge 2^nn!$.}
\reponse{Formule de Taylor Lagrange entre $\frac 12$ et $0$.}
    \item \question{Montrer que pour $n \ge 1$, $\sup\limits_{\R^+} \bigl|f^{(n)}\bigr| > 2^nn!$.}
\reponse{Sinon, la fonction $g$ : $x \mapsto f(x) - (1-2x)^n$ est monotone
             sur $[0,\frac12]$ et nulle en $0$ et $\frac 12$, donc
             identiquement nulle. Impossible car $g^{(n)}(\frac12) \ne 0$.}
\end{enumerate}
}
