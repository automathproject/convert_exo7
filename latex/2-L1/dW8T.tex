\uuid{dW8T}
\exo7id{3918}
\auteur{quercia}
\organisation{exo7}
\datecreate{2010-03-11}
\isIndication{false}
\isCorrection{true}
\chapitre{Fonctions circulaires et hyperboliques inverses}
\sousChapitre{Fonctions circulaires inverses}

\contenu{
\texte{

}
\begin{enumerate}
    \item \question{Simplifier $\arctan \frac {1-x}{1+x}$.}
\reponse{$x  > -1  \Rightarrow  = \frac \pi4   - \arctan x$,  \qquad
$x  < -1  \Rightarrow  =-\frac {3\pi}4- \arctan x$.}
    \item \question{Simplifier $\arctan \sqrt{\frac {1-x}{1+x}}$.}
\reponse{$= \frac 12 \arccos x$.}
    \item \question{Simplifier $\arctan\frac{x-\sqrt{1-x^2}}{x+\sqrt{1-x^2}}$.}
\reponse{$-1 \le x < -\frac1{\sqrt2}  \Rightarrow  = \arcsin x + \frac{3\pi}4$,  \qquad
         $-\frac1{\sqrt2} < x \le 1   \Rightarrow  = \arcsin x - \frac{\pi}4$.}
    \item \question{Simplifier $\arctan\frac{\sqrt{x^2+1}-1}x + \arctan\bigl(\sqrt{1+x^2} - x\bigr)$.}
\reponse{$=\frac\pi4$.}
    \item \question{Simplifier $\arctan \frac1{2x^2} - \arctan\frac x{x-1} + \arctan\frac{x+1}x$.}
\reponse{$f(x)=0$ pour $x\in ]-\infty,0[$ ; \\
$f(x)=\pi$ pour $x\in ]0,1[$ ; \\
$f(x)=0$ pour $x\in ]1,+\infty[$.}
\end{enumerate}
}
