\uuid{5h9l}
\exo7id{5250}
\auteur{rouget}
\organisation{exo7}
\datecreate{2010-07-04}
\isIndication{false}
\isCorrection{true}
\chapitre{Suite}
\sousChapitre{Convergence}

\contenu{
\texte{
Soit $(u_n)$ une suite réelle non majorée. Montrer qu'il existe une suite extraite de $(u_n)$ tendant vers $+\infty$.
}
\reponse{
La suite $u$ n'est pas majorée. Donc, $\forall M\in\Rr,\;\exists n\in\Nn/\;u_n> M$. En particulier, $\exists n_0\in\Nn/\;u_{n_0}\geq0$.

Soit $k=0$. Supposons avoir construit des entiers $n_0$, $n_1$,..., $n_k$ tels que $n_0<n_1<...<n_k$ et $\forall i\in\{0,...,k\},\;u_{n_i}\geq i$.

On ne peut avoir~:~$\forall n>n_k,\;u_n<k+1$ car sinon la suite $u$ est majorée par le nombre 
$\mbox{Max}\{u_0,u_1,...,u_{n_k},k+1\})$. Par suite, $\exists n_{k+1}>n_k/\;u_{n_{k+1}}\geq k+1$.

On vient de construire par récurrence une suite $(u_{n_k})_{k\in\Nn}$ extraite de la suite $u$ telle que $\forall k\in\Nn,\;u_{n_k}\geq k$ et en particulier telle que $\lim_{k\rightarrow +\infty}u_{n_k}=+\infty$.
}
}
