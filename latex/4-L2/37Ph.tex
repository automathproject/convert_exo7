\uuid{37Ph}
\exo7id{3693}
\auteur{quercia}
\organisation{exo7}
\datecreate{2010-03-11}
\isIndication{false}
\isCorrection{true}
\chapitre{Espace euclidien, espace normé}
\sousChapitre{Produit scalaire, norme}

\contenu{
\texte{
Soit $E$ un espace euclidien de dimension $n>1$. Trouver toutes les fonctions $f$
de $E$ dans $\R$ continues telles que $u\perp v \Rightarrow f(u+v)=f(u)+f(v)$.
}
\reponse{
$f$ linéaire et $f = x \mapsto\|x\|^2$ conviennent et l'ensemble~$\cal E$
des fonctions $f$ vérifiant la propriété est stable par combinaison
linéaire donc toute fonction de la forme $x \mapsto\ell(x) + a\|x\|^2$
avec $\ell\in E^*$ et $a\in\R$ convient. On montre que ce sont les
seules~:
Soit $f\in{\cal E}$ l'on décompose en sa partie paire $f_p$
et sa partie impaire~$f_i$. Alors $f_p,f_i\in{\cal E}$.

Soient $x,y\in E$ avec $\|x\|=\|y\|$ et $x\perp y$.
On a $f_i(x\pm y) = f_i(x)\pm f_i(y)$ et $f_i(2x) = f_i(x+y) + f_i(x-y) = 2f_i(x)$.
Ensuite, $f_i(2x) + f_i(x) -f_i(y) = f_i(2x+y) + f_i(x-2y) = f_i(3x-y) = f_i(3x) - f_i(y)$
d'où $f_i(3x) = 3f_i(x)$ et de proche en proche $f_i(kx) = kf_i(x)$ pour $k\in\N$
puis pour $k\in\Z,\Q,\R$ successivement vu la continuité de~$f$.
En prenant une base $(e_1,\dots,e_n)$ orthonormale on a
$f(x_1e_1+\dots+x_ne_n) = x_1f(e_1)+\dots+x_nf(e_n)$ pour tous $x_1,\dots,x_n$
réels donc $f_i$ est linéaire.

Soient à présent $x,y\in E$ avec $\|x\|=\|y\|$
alors $f_p(x+y) + f_p(x-y) = f(2x)$ et $f_p(x+y) + f_p(y-x) = f_p(2y)$
d'où $f_p(2x) = f_p(2y)$. Ainsi $f_p$ est constante sur les sphères
de centre~$0$. On écrit $f_p(x) = \varphi(\|x\|^2)$ avec $\varphi : {\R^+} \to \R$
prolongée à $\R$ par imparité ($f_p(0) = 0$ de manière évidente)
et on a $\varphi(a^2+b^2) = f_p(ae_1+be_2) = f_p(ae_1)+f_p(be_2) = \varphi(a^2)+\varphi(b^2)$
d'où l'on conclut que $\varphi$ est linéaire.
}
}
