\uuid{AI8t}
\exo7id{3626}
\auteur{quercia}
\organisation{exo7}
\datecreate{2010-03-10}
\isIndication{false}
\isCorrection{true}
\chapitre{Endomorphisme particulier}
\sousChapitre{Autre}

\contenu{
\texte{
Dans $ K^3$ on considère les formes linéaires :
$f_1(\vec x) = x+2y+3z$,
$f_2(\vec x) = 2x+3y+4z$,
$f_3(\vec x) = 3x+4y+6z$.
}
\begin{enumerate}
    \item \question{Montrer que $(f_1,f_2,f_3)$ est une base de $( K^3)^*$.}
    \item \question{Trouver la base duale.}
\reponse{
$( (-2,0,1),\ (0,3,-2),\ (1,-2,1) )$.
}
\end{enumerate}
}
