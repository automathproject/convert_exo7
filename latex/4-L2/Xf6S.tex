\uuid{Xf6S}
\exo7id{6887}
\auteur{barraud}
\organisation{exo7}
\datecreate{2012-09-05}
\isIndication{false}
\isCorrection{true}
\chapitre{Déterminant, système linéaire}
\sousChapitre{Calcul de déterminants}

\contenu{
\texte{
Calculer les déterminants suivant~:
$$
 \left\vert
   \begin{matrix}
     a_1   &a_2   &\cdots&a_n    \\
     a_1   &a_1   &\ddots&\vdots \\
     \vdots&\ddots&\ddots&a_2    \\
     a_1   &\cdots&a_1   &a_1
   \end{matrix}
 \right\vert
\qquad
\left\vert
\begin{matrix}
  1    &      &        &1 \\
  1    &1      & (0)   & \\
       &\ddots &\ddots & \\  
  (0)  &       &1      &1
\end{matrix}
\right\vert
\qquad
\left\vert
\begin{matrix}
    a+b  &    a   & \cdots &  a       \\
     a   &   a+b  & \ddots & \vdots   \\
  \vdots & \ddots & \ddots &  a       \\
     a   & \cdots &    a   & a+b 
\end{matrix}
\right\vert
$$
}
\reponse{
On retire la première colonne à toutes les autres colonnes 
$$
\Delta_1 =   \begin{vmatrix}
     a_1   &a_2   &\cdots&a_n    \\
     a_1   &a_1   &\ddots&\vdots \\
     \vdots&\ddots&\ddots&a_2    \\
     a_1   &\cdots&a_1   &a_1
   \end{vmatrix}
 =  \begin{vmatrix}
     a_1   &a_2-a_1  & a_3-a_1 &\cdots&a_n -a_1   \\
     a_1   &0  & &\ddots&\vdots \\
     \vdots&\vdots &\ddots&\ddots&a_2-a_1    \\
     a_1   &0&\cdots&0   & 0
   \end{vmatrix}$$
On développe par rapport à la dernière ligne :
$$\Delta_1 = (-1)^{n-1}a_1 \begin{vmatrix}
     a_2-a_1  &  &\cdots&a_n -a_1   \\
     0  &\ddots &&\vdots \\
     \vdots &\ddots&\ddots&    \\
     0&\cdots&0   & a_2-a_1
   \end{vmatrix}
= (-1)^{n-1}a_1(a_2-a_1)^{n-1}
$$
Où l'on a reconnu le déterminant d'un matrice triangulaire supérieure.
Donc 
$$\Delta_1 = a_1(a_1-a_2)^{n-1}.$$
On va transformer la matrice correspondante en une matrice triangulaire supérieure,
on commence par remplacer la ligne $L_2$ par $L_2-L_1$ (on ne note que les coefficients non nuls) : 
$$\Delta_2 =   \begin{vmatrix}
  1 &      &    &   &+1 \\
  1 &1      && &  \\
   &1      &1  & &\\
   & &\ddots &\ddots & \\
  &  &       &1      &1
\end{vmatrix}=
 \begin{vmatrix}
  1 &      &    &   &+1 \\
  0 &1      && &  -1\\
   &1      &1  & &\\
   & &\ddots &\ddots & \\
  &  &       &1      &1
\end{vmatrix}
$$
Puis on remplace la ligne $L_3$ par $L_3-L_2$ (attention il s'agit de la nouvelle ligne $L_2$) et on 
continue ainsi de suite jusqu'à $L_{n-1} \leftarrow L_{n-1}-L_{n-2}$ ($n$ est la taille de la matrice sous-jacente) :
$$\Delta_2 =   \begin{vmatrix}
  1 &      &    &   &&+1 \\
  0 &1      && &  &-1\\
    & 0      &1  & & &+1\\
    & & 1      &1  & && \\
   & &&\ddots &\ddots & \\
  &  &       &&1      &1                 
               \end{vmatrix}
= \cdots = 
\begin{vmatrix}
  1 &      &    &   &&+1 \\
  0 &1      && &  &-1\\
    & 0      &1  & & &+1\\
    & & \ddots      &\ddots  & &\vdots& \\
   & &&0 &1 & (-1)^{n} \\
  &  &       &&1      &1                 
               \end{vmatrix}
$$
On fait attention pour le dernier remplacement $L_n \leftarrow L_n-L_{n-1}$ légèrement différent et qui conduit au déterminant d'une matrice triangulaire : :
$$\Delta_2 = \begin{vmatrix}
  1 &      &    &   &&+1 \\
  0 &1      && &  &-1\\
    & 0      &1  & & &+1\\
    & & \ddots      &\ddots  & &\vdots& \\
   & && &1 & (-1)^{n}\\
  &  &       &&0      &1 - (-1)^{n}                 
               \end{vmatrix}
= 1 - (-1)^{n}.
$$

En conclusion $\Delta_2 = \begin{cases}
                           0 & \text{ si $n$ est pair} \\
                           2 & \text{ si $n$ est impair} \\                           
                          \end{cases}$
On retire la colonne $C_1$ aux autres colonnes $C_i$ pour faire apparaître des $0$ :
$$\Delta_3 = \begin{vmatrix}
    a+b  &    a   & \cdots &  a       \\
     a   &   a+b  & \ddots & \vdots   \\
  \vdots & \ddots & \ddots &  a       \\
     a   & \cdots &    a   & a+b 
\end{vmatrix}
= \begin{vmatrix}
    a+b  &    -b   & & \cdots &  -b       \\
     a   &   b     &   0 & \cdots        &  0   \\
     a   &   0 &  \ddots & \ddots & \vdots   \\
  \vdots & \ddots & \ddots &  b & 0       \\
     a   &  0 & \cdots &    0   & b 
\end{vmatrix}$$
On remplace ensuite $L_1$ par $L_1+L_2+L_3+\cdots +L_n$
(ou ce qui revient au même : faites les opérations 
$L_1 \leftarrow L_1+L_2$ puis $L_1 \leftarrow L_1+L_3$,\ldots
chacune de ces opérations fait apparaître un $0$ sur la première ligne)
pour obtenir une matrice triangulaire inférieure :
$$\Delta_3 = \begin{vmatrix}
   na+b  &    0   & & \cdots &  0       \\
     a   &   b     &   0 & \cdots        &  0   \\
     a   &   0 &  \ddots & \ddots & \vdots   \\
  \vdots & \ddots & \ddots &  b & 0       \\
     a   &  0 & \cdots &    0   & b 
\end{vmatrix}=(na+b)b^{n-1}.$$
}
}
