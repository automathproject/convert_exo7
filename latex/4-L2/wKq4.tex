\uuid{wKq4}
\exo7id{2477}
\auteur{matexo1}
\organisation{exo7}
\datecreate{2002-02-01}
\isIndication{false}
\isCorrection{false}
\chapitre{Réduction d'endomorphisme, polynôme annulateur}
\sousChapitre{Diagonalisation}

\contenu{
\texte{
Soit $A$ une matrice carr\'ee r\'eelle d'ordre $n$ non nulle et
nilpotente.
}
\begin{enumerate}
    \item \question{Montrer que $I-A$ n'est pas diagonalisable.}
    \item \question{G\'en\'eraliser en montrant que si $B$ est une matrice
diagonalisable dont toutes les valeurs propres sont \'egales, alors
$B+A$ n'est pas diagonalisable.}
    \item \question{Montrer qu'il existe $A, B$ dans ${\mathcal M}_2(\R)$, $A \neq 0$
nilpotente et $B$ diagonalisable, telles que $A+B$ soit
diagonalisable.}
\end{enumerate}
}
