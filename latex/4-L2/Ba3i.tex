\uuid{Ba3i}
\exo7id{2570}
\auteur{delaunay}
\organisation{exo7}
\datecreate{2009-05-19}
\isIndication{false}
\isCorrection{true}
\chapitre{Réduction d'endomorphisme, polynôme annulateur}
\sousChapitre{Valeur propre, vecteur propre}

\contenu{
\texte{
Soit $A$ une matrice carr\'ee d'ordre $n$. On suppose que $A$ est inversible et que $\lambda\in\R$ est une valeur propre de $A$.
}
\begin{enumerate}
    \item \question{D\'emontrer que $\lambda\neq 0$.}
\reponse{{\it D\'emontrons que $\lambda\neq 0$}.
 Si $\lambda=0$ est valeur propre de $A$, alors $\ker A\neq\{0\}$, donc $A$ n'est pas injective et sa matrice ne peut pas \^etre inversible. Par
cons\'equent, $\lambda\neq 0$.}
    \item \question{D\'emontrer que si $\vec x$ est un vecteur propre de $A$ pour la valeur propre $\lambda$ alors il est vecteur propre de $A^{-1}$ de valeur propre
 $\lambda^{-1}$ .}
\reponse{{\it D\'emontrons que si $\vec x$ est un vecteur propre de $A$ pour la valeur propre $\lambda$ alors il est vecteur propre de $A^{-1}$ de valeur propre $\lambda^{-1}$ }.

Comme $A$ est inversible, on a $A\vec x=\lambda\vec x\iff A^{-1}(A\vec x)=A^{-1}(\lambda \vec x)\iff\vec x=\lambda A^{-1}\vec x $,
d'o\`u $A^{-1}\vec x=\lambda^{-1}\vec x$. Ce qui prouve que $\vec x$ est vecteur propre de $A^{-1}$ de valeur propre $\lambda^{-1}$.}
\end{enumerate}
}
