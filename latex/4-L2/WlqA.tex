\uuid{WlqA}
\exo7id{5662}
\auteur{rouget}
\organisation{exo7}
\datecreate{2010-10-16}
\isIndication{false}
\isCorrection{true}
\chapitre{Réduction d'endomorphisme, polynôme annulateur}
\sousChapitre{Autre}

\contenu{
\texte{
Montrer que toute matrice de trace nulle est semblable à une matrice de diagonale nulle.
}
\reponse{
On montre le résultat par récurrence sur $n\in\Nn^*$ le format de $A$.

\textbullet~C'est clair pour $n = 1$.

\textbullet~Soit $n\geqslant1$. Supposons que toute matrice de format $n$ et de trace nulle soit semblable à une matrice de diagonale nulle.

Soient $A$ une matrice carrée de format $n+1$ et de trace nulle puis $f$ l'endomorphisme de $\Kk^{n+1}$ de matrice $A$ dans la base canonique $(e_1,...,e_{n+1})$ de $\Kk^{n+1}$.

Si $f$ est une homothétie de rapport noté $k$, alors $0=\text{Tr}(f)=k(n+1)$ et donc $k = 0$ puis $f = 0$ puis $A = 0$. Dans ce cas, $A$ est effectivement semblable à une matrice de diagonale nulle.

Sinon $f$ n'est pas une homothétie et on sait qu'il existe un vecteur $u$ de $E$ tel que la famille $(u,f(u))$ soit libre (voir exercice \ref{ex:rou25}). On complète la famille libre $(u,f(u))$ en une base de $E$. Le coefficient ligne 1, colonne 1, de la matrice de $f$ dans cette base est nul. Plus précisément, $A$ est semblable à une matrice de la forme $\left(
\begin{array}{cccccc}
0&\times&\ldots& &\ldots&\times\\
1& & & & \\
0& & &  & \\
\vdots& & &A'& & \\
\vdots& \\
0&
\end{array}
\right)$.

Puis $\text{Tr}A'=\text{Tr}A= 0$ et par hypothèse de récurrence, $A'$ est semblable à une matrice $A_1$ de diagonale nulle ou encore il existe $A_1$ matrice carrée de format $n$ et de diagonale nulle et $Q\in GL_n(\Kk)$ telle que $Q^{-1}A'Q = A_1$.

Mais alors, si on pose $P=\left(
\begin{array}{cccc}
1&0&\ldots&0\\
0& & & \\
\vdots& &Q& \\
0& & & 
\end{array}
\right)$, $P$ est inversible car $\text{det}(P)=1\times\text{det}(Q)\neq0$ et un calcul par blocs montre que $P^{-1}=\left(
\begin{array}{cccc}
1&0&\ldots&0\\
0& & & \\
\vdots& &Q^{-1}& \\
0& & & 
\end{array}
\right)$ puis que $P^{-1}AP=\left(
\begin{array}{cccccc}
0&\times&\ldots& &\ldots&\times\\
\times& & & & \\
\vdots& & &  & \\
 & & &A_1& & \\
\vdots& \\
\times&
\end{array}
\right)$ est de diagonale nulle.
}
}
