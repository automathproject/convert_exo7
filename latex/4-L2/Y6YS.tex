\uuid{Y6YS}
\exo7id{5369}
\auteur{rouget}
\datecreate{2010-07-06}
\isIndication{false}
\isCorrection{true}
\chapitre{Déterminant, système linéaire}
\sousChapitre{Calcul de déterminants}

\contenu{
\texte{
Soit $A=(a_{i,j})_{1\leq i,j\leq n}$ et $B=(b_{i,j})_{1\leq i,j\leq n}$ avec $b_{i,j}=(-1)^{i+j}a_{i,j}$. Montrer que $\mbox{det}B=\mbox{det}A$.
}
\reponse{
\textbf{1ère solution.}

\begin{align*}\ensuremath
\mbox{det}B&=\sum_{\sigma\in S_n}\varepsilon(\sigma)(-1)^{1+\sigma(1)+2+\sigma(2)+...+n+\sigma(n)}a_{\sigma(1),1}a_{\sigma(2),2}...a_{\sigma(n),n}\\
 &=\sum_{\sigma\in S_n}\varepsilon(\sigma)a_{\sigma(1),1}a_{\sigma(2),2}...a_{\sigma(n),n}\;(\mbox{car}\;1+\sigma(1)+2+\sigma(2)+...+n+\sigma(n)=2(1+2+...+n)\in2\Nn)\\
 &=\mbox{det}A
\end{align*}
\textbf{2ème solution.} On multiplie par $-1$ les lignes $2$, $4$, $6$... puis les colonnes $2$, $4$, $6$...On obtient 
$\mbox{det}B=(-1)^{2p}\mbox{det}A=\mbox{det}A$ (où $p$ est le nombre de lignes ou de colonnes portant un numéro pair).
}
}
