\uuid{Kw75}
\exo7id{3819}
\auteur{quercia}
\datecreate{2010-03-11}
\isIndication{false}
\isCorrection{true}
\chapitre{Espace euclidien, espace normé}
\sousChapitre{Problèmes matriciels}

\contenu{
\texte{
Soit $A\in\mathcal{M}_n(\R)$ telle que $A^2 = -I_n$.
}
\begin{enumerate}
    \item \question{Montrer que $n$ est pair.}
\reponse{$\det(A)^2 = (-1)^n.$}
    \item \question{Montrer que $A$ est semblable à~$A' = \begin{pmatrix}0 &-I_{n/2}\cr I_{n/2}&0\cr\end{pmatrix}$.}
\reponse{$A$ est $\C$-diagonalisable (annulateur simple) et ses valeurs propres
sont $i,-i$ avec la même multiplicité ($A$ est réelle). La matrice $A'$ donnée a
les mêmes propriétés donc $A$ et $A'$ sont $\C$-semblables à la même matrice
diagonale, et donc $\C$-semblables l'une à l'autre. Comme la $\C$-similitude
entre matrices réelles est équivalente à la $\R$-similitude (résultat bien connu),
$A$ et $A'$ sont $\R$-semblables.}
    \item \question{On suppose $A\in{\cal O}(n)$. Montrer que $A$ est semblable à la matrice~$A'$
précédente avec une matrice de passage orthogonale.}
\reponse{Soit $e_1$ unitaire et $e'_1 = Ae_1$. Alors $e'_1$ est unitaire et
$Ae'_1 = -e_1$ d'où $(e_1\mid e'_1) = (Ae_1\mid Ae'_1) = -(e_1\mid e'_1) = 0$
donc $(e_1,e'_1)$ est une famille orthonormale. Si $F_1$ est le sous-espace vectoriel engendré
par $(e_1,e'_1)$ alors $F_1^\bot$ est stable par $A$ donc on peut construire
par récurrence une base orthonormale $(e_1,\dots,e_{n/2},e'_1,\dots,e'_{n/2})$
telle que $Ae_i = e'_i$ et $Ae'_i = -e_i$.}
\end{enumerate}
}
