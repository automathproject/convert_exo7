\uuid{0PDb}
\exo7id{3449}
\auteur{quercia}
\datecreate{2010-03-10}
\isIndication{false}
\isCorrection{true}
\chapitre{Déterminant, système linéaire}
\sousChapitre{Applications}

\contenu{
\texte{
Soit un déterminant symétrique réel d'ordre impair dont les coefficients
sont entiers, les diagonaux étant de plus pairs.
Montrer que ce déterminant est pair.
}
\reponse{
$\det(M) = \sum_{\sigma\in S_n}\varepsilon(\sigma)a_{1\sigma(1)}\dots a_{n\sigma(n)}$.
Soit $\sigma\in S_n$ telle que $\sigma\ne\sigma^{-1}$. Alors les termes associés
à $\sigma$ et $\sigma^{-1}$ sont égaux car $M$ est symétrique, donc la somme
de ces deux termes est paire. Soit $\sigma\in S_n$ telle que $\sigma=\sigma^{-1}$.
Alors comme $n$ est impair, il existe $i\in{[[1,n]]}$ tel que $\sigma(i)=i$
donc le terme associé à~$\sigma$ est pair.
}
}
