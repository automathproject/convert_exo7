\uuid{N7IZ}
\exo7id{3438}
\auteur{quercia}
\organisation{exo7}
\datecreate{2010-03-10}
\isIndication{false}
\isCorrection{true}
\chapitre{Déterminant, système linéaire}
\sousChapitre{Forme multilinéaire}

\contenu{
\texte{
Soit $E$ un ev de dimension $n$, $f \in \mathcal{L}(E)$,
et $\vec u_1 , \dots, \vec u_n$, $n$ vecteurs de $E$.
On note $\det$ le déterminant dans une base fixée de $E$.
Démontrer que :
$$ \det( f(\vec u_1), \vec u_2, \dots, \vec u_n )
 + \det( \vec u_1, f(\vec u_2), \vec u_3, \dots, \vec u_n )
 + \dots
 + \det( \vec u_1, \vec u_2, \dots, f(\vec u_n ))
 = 
 \det( \vec u_1, \vec u_2, \dots, \vec u_n ) \mathrm{tr}(f).
$$
}
\reponse{
Les deux membres sont $n$-linéaires alternés.
         On le vérifie sur la base du déterminant.
}
}
