\uuid{XcaD}
\exo7id{2600}
\auteur{delaunay}
\organisation{exo7}
\datecreate{2009-05-19}
\isIndication{false}
\isCorrection{true}
\chapitre{Réduction d'endomorphisme, polynôme annulateur}
\sousChapitre{Diagonalisation}

\contenu{
\texte{

}
\begin{enumerate}
    \item \question{Donner un exemple de matrice dans $M_2(\R)$, diagonalisable sur $\C$ mais non diagonalisable sur $\R$ (justifier).}
\reponse{{\it Donnons un exemple de matrice dans $M_2(\R)$, diagonalisable sur $\C$ mais non diagonalisable sur $\R$}. (2 points)

Soit $A$ la matrice $A=\begin{pmatrix}0&-1 \\  1&0\end{pmatrix}$. Son polyn\^ome caract\'eristique est \'egal \`a
$$P_A(X)=\begin{vmatrix}-X&-1 \\  1&-X\end{vmatrix}=X^2+1.$$
Le polyn\^ome caract\'eristique de $A$ admet deux racines complexes conjugu\'ees distinctes $i$ et $-i$ elle est donc diagonalisable sur $\C$ mais elle ne l'est pas sur $\R$.}
    \item \question{Donner un exemple de matrice dans $M_2(\R)$ non diagonalisable, ni sur $\C$, ni sur $\R$ (justifier).}
\reponse{{\it Donnons un exemple de matrice dans $M_2(\R)$ non diagonalisable, ni sur $\C$, ni sur $\R$.} (2 points) 
 
Soit $A$ la matrice $A=\begin{pmatrix}1&1 \\  0&1\end{pmatrix}$. Son polyn\^ome caract\'eristique est \'egal \`a
$$P_A(X)=\begin{vmatrix}1-X&1 \\  0&1-X\end{vmatrix}=(1-X)^2.$$
Le polyn\^ome caract\'eristique de $A$ admet une racine double $1$, la matrice $A$ admet l'unique valeur propre $1$, or, elle n'est pas \'egale \`a l'identit\'e, par cons\'equent, elle n'est diagonalisable, ni 
sur $\C$, ni sur $\R$.}
\end{enumerate}
}
