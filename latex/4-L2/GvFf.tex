\uuid{GvFf}
\exo7id{1389}
\auteur{legall}
\organisation{exo7}
\datecreate{1998-09-01}
\isIndication{false}
\isCorrection{true}
\chapitre{Groupe, anneau, corps}
\sousChapitre{Groupe, sous-groupe}

\contenu{
\texte{

}
\begin{enumerate}
    \item \question{Soit $  G  $ un groupe dans lequel tout \'el\'ement (distinct de
l'\'el\'ement neutre) est d'ordre $  2  .$ Montrer que $  G  $ est commutatif.}
\reponse{Notons d'abord que pour $x\in G$ $x^2=e$ et donc
$x^{-1}=x$. Soit maintenant $x,y \in G$. Alors $xy\in G$ et
$(xy)^2=e$ donc $xy=(xy)^{-1}$ et par suite $xy=y^{-1}x^{-1}=yx$
car $x$ et $y$ sont d'ordre $2$. Le produit de deux \'el\'ements
quelconques de $G$ commute donc $G$ est commutatif.}
    \item \question{Soit $  G  $ un groupe d'ordre pair. Montrer que $  G  $ contient au moins un \'el\'ement d'ordre $  2  .$}
\reponse{Notons $E$ l'ensemble des \'el\'ements d'ordre $2$.
$$E = \{ x\in G\ / \ x^2=e \text{ et } x \not=e \} =
\{ x\in G\ / \ x=x^{-1} \text{ et } x \not=e \}.$$ Par l'absurde
supposons que $H$ est l'ensemble vide. Alors quelque soit
$x\not=e$ dans $G$ $x\not= x^{-1}$. Donc nous pouvons d\'ecomposer
$G\setminus \{ e \}$ en deux ensembles disjoints $F = \{
x_1,\ldots,x_n\}$ et $F' =\{{x_1}^{-1}, \ldots,{x_n}^{-1} \}$ qui
sont de m\^eme cardinal $n$. Donc le cardinal de $G$ est $2n+1$
(le $+1$ provient de l'\'el\'ement neutre). Ce qui contredit
l'hypoth\`ese << $G$ d'orde pair >>.}
\end{enumerate}
}
