\uuid{nox5}
\exo7id{3500}
\auteur{quercia}
\organisation{exo7}
\datecreate{2010-03-10}
\isIndication{false}
\isCorrection{true}
\chapitre{Réduction d'endomorphisme, polynôme annulateur}
\sousChapitre{Valeur propre, vecteur propre}

\contenu{
\texte{
Soient $a_1,\dots,a_n \in \R$.

Chercher les valeurs et les vecteurs propres de la matrice
$A = \begin{pmatrix}    &        &        & a_1     \cr
                  &(0)     &        & \vdots  \cr
                  &        &        & a_{n-1} \cr
              a_1 &\dots   &a_{n-1} & a_n     \cr \end{pmatrix}$.
On distinguera les cas :
}
\begin{enumerate}
    \item \question{$(a_1,\dots,a_{n-1}) \ne (0,\dots,0)$.}
\reponse{$\mathrm{rg}(A) = 2  \Rightarrow  0$ est valeur propre d'ordre au moins $n-2$.
    $E_0 = \{ a_1x_1 + \dots + a_{n-1}x_{n-1} = x_n = 0 \}$.\par
    vp $\lambda \ne 0$ :
    $\lambda^2 - a_n\lambda - (a_1^2 + \dots + a_{n-1}^2) = 0$.
    Il y a deux racines distinctes,
    $E_\lambda = \text{vect}((a_1,\dots,a_{n-1},\lambda))$.}
    \item \question{$(a_1,\dots,a_{n-1}) = (0,\dots,0)$.}
\reponse{$A$ est diagonale. vp = 0 et $a_n$.}
\end{enumerate}
}
