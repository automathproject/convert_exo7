\uuid{QHyC}
\exo7id{2592}
\auteur{delaunay}
\datecreate{2009-05-19}
\isIndication{false}
\isCorrection{true}
\chapitre{Réduction d'endomorphisme, polynôme annulateur}
\sousChapitre{Diagonalisation}

\contenu{
\texte{
Soit $A$ la matrice de $M_3(\R)$ suivante :
$$A=\begin{pmatrix}0&1&0 \\  -4&4&0 \\ -2&1&2\end{pmatrix}.$$
}
\begin{enumerate}
    \item \question{La matrice $A$ est-elle diagonalisable ?}
\reponse{{\it La matrice $A$ est-elle diagonalisable ?}

Calculons son polyn\^ome caract\'eristique 
$$P_A(X)=\begin{vmatrix}-X&1&0 \\  -4&4-X&0 \\ -2&1&2-X\end{vmatrix}=(2-X)(X^2-4X+4)=(2-X)^3.$$
la matrice $A$ admet une unique valeur propre $2$, si elle \'etait diagonalisable, elle serait semblable \`a la matrice $2.I_3$, elle serait donc \'egale \`a $2I_3$ ce qui n'est pas le cas, elle n'est donc pas diagonalisable.}
    \item \question{Calculer $(A-2I_3)^2$, puis $(A-2I_3)^n$ pour tout $n\in\N$. En d\'eduire $A^n$.}
\reponse{{\it Calculons $(A-2I_3)^2$, puis $(A-2I_3)^n$ pour tout $n\in\N$}.

On a 
$$(A-2I_3)^2=\begin{pmatrix}-2&1&0 \\  -4&2&0 \\ -2&1&0\end{pmatrix}\begin{pmatrix}-2&1&0 \\  -4&2&0 \\ -2&1&0\end{pmatrix}=\begin{pmatrix}0&0&0 \\  0&0&0 \\ 0&0&0\end{pmatrix},$$
ainsi, $(A-2I_3)^0=I$, $(A-2I_3)^1=\begin{pmatrix}-2&1&0 \\  -4&2&0 \\ -2&1&0\end{pmatrix}$, et,  pour tout $n\geq2$, on a $(A-2I_3)^n=0$.

{\it On en d\'eduit $A^n$}

Notons $B=A-2I_3$, on a $A=A-2I_3+2I_3=B+2I_3$ avec $B^n=0$ pour $n\geq2$. Par ailleurs, les matrices $B$ et $2I_3$ commutent, ainsi
$$A^n=(B+2I_3)^n=\sum^n_{k=0}C_n^kB^k(2I_3)^{n-k}$$
o\`u les $C_n^k$ sont les coefficients du bin\^ome de Newton :
$$C_n^k={\frac{n!}{k!(n-k)!}}.$$ 
Or, pour $k\geq2$, on a $B^k=0$ d'o\`u pour $n\geq2$,
\begin{align*}
A^n&=C_n^0B^0(2I_3)^n+C_n^1B^1(2I_3)^{n-1} \\ 
&=2^nI_3+2^{n-1}nB \\ 
&=2^nI_3+2^{n-1}n(A-2I_3) \\ 
&=2^n(1-n)I_3+2^{n-1}nA.
\end{align*}}
\end{enumerate}
}
