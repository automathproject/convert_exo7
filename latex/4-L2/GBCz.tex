\uuid{GBCz}
\exo7id{5308}
\auteur{rouget}
\datecreate{2010-07-04}
\isIndication{false}
\isCorrection{true}
\chapitre{Arithmétique}
\sousChapitre{Arithmétique de Z}

\contenu{
\texte{
Soit $u_n=10...01_2$.($n$ chiffres égaux à $0$). Déterminer l'écriture binaire de~:
}
\begin{enumerate}
    \item \question{$u_n^2$,}
\reponse{$u_n^2=(2^{n+1}+1)^2=2^{2n+2}+2^{n+2}+1=10...010...01_2$ ($n-1$ puis $n+1$ chiffres $0$)}
    \item \question{$u_n^3$,}
\reponse{\begin{align*}\ensuremath
u_n^3&=(2^{n+1}+1)^3=2^{3n+3}+3.2^{2n+2}+3.2^{n+1}+1=2^{3n+3}+(2+1).2^{2n+2}+(2+1).2^{n+1}+1\\
 &=2^{3n+3}+2^{2n+3}+2^{2n+2}+2^{n+2}+2^{n+1}+1=10...0110...0110...01_2
\end{align*}

($n-1$ puis $n-1$ puis $n$ chiffres $0$)}
    \item \question{$u_n^3-u_n^2+u_n$.}
\reponse{\begin{align*}\ensuremath
u_n^3-u_n^2+u_n&=2^{3n+3}+3.2^{2n+2}+3.2^{n+1}+1-2^{2n+2}-2^{n+2}-1+2^{n+1}+1=2^{3n+3}+2^{2n+3}+2^{n+2}+1\\
 &=10...010...010...01
\end{align*}

($n-1$ puis $n$ puis $n+1$ chiffres $0$)}
\end{enumerate}
}
