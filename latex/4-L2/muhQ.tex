\uuid{muhQ}
\exo7id{1155}
\auteur{legall}
\organisation{exo7}
\datecreate{1998-09-01}
\isIndication{false}
\isCorrection{false}
\chapitre{Déterminant, système linéaire}
\sousChapitre{Autre}

\contenu{
\texte{
On consid\`ere les matrices~:


$$ I=\begin{pmatrix} 1 & 0 & 0  & 0  \cr
                                   0 & 1 & 0 &  0  \cr
                                    0 & 0 & 1 & 0 \cr
                                     0 & 0 & 0 &  1   \cr \end{pmatrix}  \hskip10mm
N=
\begin{pmatrix} 0 & 3 & 1  & 3  \cr
                                   0 & 0 & 0 &  1  \cr
                                    0 & 0 & 0 & -1 \cr
                                     0 & 0 & 0 &  0   \cr \end{pmatrix} \hskip5mm
A=I+N  .$$
}
\begin{enumerate}
    \item \question{Pour tout $ n \in \N ^* $ calculer $ \hbox{det}(A^n) .$}
    \item \question{Calculer $ N^2 $ et $ N^3 .$}
    \item \question{Pour tout $ n \in \N ^* $ donner le rang de $ N^n $ et celui de
$ A^n .$}
    \item \question{En utilisant 1., donner, en fonction de $ n \in \N ^* ,$
l'expression de la matrice $ M(n)=A^n .$}
    \item \question{Pour $ n \in \N ^* ,$ justifier la formule $ (A^n)^{-1}=M(-n)
.$ Expliquer et justifier l'\' ecriture~:
$ A^n=M(n) $ pour tout $ n \in \Z  .$}
\end{enumerate}
}
