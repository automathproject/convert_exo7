\uuid{UETo}
\exo7id{7364}
\auteur{mourougane}
\organisation{exo7}
\datecreate{2021-08-10}
\isIndication{false}
\isCorrection{true}
\chapitre{Groupe, anneau, corps}
\sousChapitre{Groupe, sous-groupe}

\contenu{
\texte{
Dans le groupe $\mathcal{S}_7$ des permutations de l'ensemble fini $\{1,2,\cdots,7\}$ on considère les deux permutations 
$$\alpha=(157)(43) \quad \text{ et } \quad \beta=(26).$$
}
\begin{enumerate}
    \item \question{Montrer que $\alpha\beta=\beta\alpha$.}
\reponse{Comme $\alpha$ et $\beta$ sont à support disjoint, ils commutent $\alpha\beta=\beta\alpha$}
    \item \question{Déterminer l'ordre de $\alpha$ et de $\beta$.}
\reponse{L'ordre de $\alpha$ donner comme produit de cycles à support disjoint est le $ppcm$ des ordres des cycles qui le composent, donc $ppcm(3,2)=6$. Celui de la transposition $\beta$ est $2$.}
    \item \question{Calculer $\alpha^{-1}$ et $\beta^{-1}$.}
\reponse{$\alpha^{-1}=(175)(43)$ et $\beta^{-1}=(26)$.}
    \item \question{Montrer que $S=\{\alpha^i\beta^j, 0\leq i \leq 5, 0\leq j\leq 1\}$ est un sous-groupe de $\mathcal{S}_7$.}
\reponse{Montrons que $S=\{\alpha^i\circ\beta^j, 0\leq i \leq 5, 0\leq j\leq 1\}$ est un sous-groupe de $\mathcal{S}_7$.


*stabilité par produit : Soit $\alpha^i\circ\beta^j$ et $\alpha^k\circ\beta^l$ deux éléments de $S$ où $0\leq i \leq 5, 0\leq j\leq 1$ et $0\leq k \leq 5, 0\leq l\leq 1$.
\begin{eqnarray*}
 (\alpha^i\circ\beta^j)\circ (\alpha^k\circ\beta^l)&=&\alpha^i\circ\beta^j\circ \alpha^k\circ\beta^l\\
&=&\alpha^i\alpha^k\beta^j\beta^l=\alpha^{i+k}\beta^{j+l}
\end{eqnarray*}
Comme $\alpha^6=Id$, en considérant le reste $0\leq r\leq 5$ de la division euclidienne de $i+k$ par $6$ et celui $0\leq s\leq 1$ de la division euclidienne de $j+l$ par $2$
on obtient $(\alpha^i\circ\beta^j)\circ (\alpha^k\circ\beta^l)=\alpha^r\beta^s$ qui appartient donc à $S$.


*stabilité par inversion :
Soit $\alpha^i\circ\beta^j$ dans $S$ où $0\leq i \leq 5, 0\leq j\leq 1$.
Comme $\alpha^6=Id$, $(\alpha^i)^{-1}=\alpha^{6-i}$. Comme $\beta^2=Id$, $\beta^{-1}=\beta$.
$$(\alpha^i\circ\beta^j)^{-1}=(\beta^j)^{-1}(\alpha^i)^{-1}=\beta^j\alpha^{6-i}=
\alpha^{6-i}\beta^j$$ qui appartient donc à $S$.}
    \item \question{Calculer l'ordre de $S$.}
\reponse{Soit $\alpha^i\circ\beta^j$ et $\alpha^k\circ\beta^l$ deux éléments de $S$ où $0\leq i \leq 5, 0\leq j\leq 1$ et $0\leq k \leq 5, 0\leq l\leq 1$.
Si $\alpha^i\circ\beta^j=\alpha^k\circ\beta^l$, alors $\alpha^i\alpha^{6-k}=\beta^l\beta^j$. Comme $\alpha$ et $\beta$ sont à support disjoint, ceci implique que $\alpha^i\alpha^{6-k}=Id$ et $\beta^l\beta^j=Id$,
soit en particulier $l=j$. On en déduit que $i=k$.
Donc tous les éléments de la liste $\{\alpha^i\circ\beta^j, 0\leq i \leq 5, 0\leq j\leq 1\}$ sont distincts.
L'ordre de $S$ est donc $6\times 2=12$.}
    \item \question{Montrer que tout sous-groupe de $\mathcal{S}_7$ qui contient $\alpha$ et $\beta$ contient $S$.}
\reponse{Soit $H$ un sous-groupe de $\mathcal{S}_7$ qui contient $\alpha$ et $\beta$.
Par stabilité par produit, il contient $\alpha^i$, $\beta^j$ et $\alpha^i\circ\beta^j$ pour tout $0\leq i \leq 5, 0\leq j\leq 1$. Il 
 contient donc $S$.}
    \item \question{Que peut-on déduire des questions 4. et 6. précédentes ?}
\reponse{On peut déduire des questions 4. et 6. précédentes que le sous-groupe de $\mathcal{S}_7$ engendré par $\alpha$ et $\beta$ est $S$.}
    \item \question{Déterminer l'ordre de $\alpha\beta$.}
\reponse{Comme $\alpha$ et $\beta$ sont à support disjoint, $\alpha\circ\beta=(157)(43)(26)$ est une écriture en produit de cycles à support disjoint. L'ordre de $\alpha\circ\beta$ est donc $ppcm(3,2,2)=6$.}
    \item \question{Le sous-groupe $S$ est-il cyclique ?}
\reponse{Une écriture en produit de cycles à support disjoint de $\alpha^i\circ\beta^j$ montre que son ordre est inférieur à $ppcm(3,2)=6$.
Par conséquent, le sous-groupe $S$ qui n'a pas d'éléments d'ordre 12 n'est pas cyclique.}
\end{enumerate}
}
