\uuid{ptOZ}
\exo7id{3161}
\auteur{quercia}
\organisation{exo7}
\datecreate{2010-03-08}
\isIndication{false}
\isCorrection{true}
\chapitre{Arithmétique}
\sousChapitre{Anneau Z/nZ, théorème chinois}

\contenu{
\texte{
On consid{\`e}re la suite $(x_n)$ {\`a} valeurs dans~$\Z/11\Z$ telle que pour tout~$n$
on ait $x_{n+3} = 4(x_{n+2}+x_{n+1}+x_n)$.
D{\'e}terminer les diff{\'e}rents comportements possibles de $(x_n)$.
}
\reponse{
L'{\'e}quation caract{\'e}ristique, $X^3=4(X^2+X+1)$ admet trois racines distinctes
dans~$\Z/11Z$~: $1,6,8$. Donc $x_n$ est de la forme~: $x_n = a + 6^nb + 8^nc$ avec $a,b,c\in\Z/11\Z$.
On a $6^{10} \equiv 8^{10} \equiv 1 (\mathrm{mod}\,{11})$, donc $(x_n)$ est p{\'e}riodique de p{\'e}riode divisant~$10$.
La plus petite p{\'e}riode est $1$ si $b=c=0$, $10$ sinon car les suites $(6^n)$ et $(8^n)$
ont $10$ comme plus petite p{\'e}riode modulo~$11$ et l'on a~:
$8(x_{n+1}-x_n) - 5(x_{n+2}-x_{n+1}) = 7\cdotp8^nc$ et
$7(x_{n+2}-x_{n+1}) - (x_{n+1}-x_n) = 7\cdotp6^nb$.
}
}
