\uuid{twHh}
\exo7id{5672}
\auteur{rouget}
\organisation{exo7}
\datecreate{2010-10-16}
\isIndication{false}
\isCorrection{true}
\chapitre{Réduction d'endomorphisme, polynôme annulateur}
\sousChapitre{Applications}

\contenu{
\texte{
Calculer $\left|
\begin{array}{cccc}
a&b&\ldots&b\\
b&a&\ddots&\vdots\\
\vdots&\ddots&\ddots&b\\
b&\ldots&b&a
\end{array}
\right|$.
}
\reponse{
Soit $A$ la matrice de l'énoncé. $\text{det}A$ est le produit des valeurs propres de $A$.

\textbullet~Si $b = 0$, $\text{det}A = a^n$.

\textbullet~Si $b\neq0$, $\text{rg}(A-(a-b)I) = 1$ ou encore $\text{dim}(\text{Ker}(A-(a-b)I))= n-1$. Par suite, $a-b$ est valeur propre d'ordre $n-1$ au moins. On obtient la valeur propre manquante $\lambda$ par la trace de $A$ : $(n-1)(a-b) +\lambda= na$ et donc $\lambda = a + (n-1)b$. Finalement $\text{det}A = (a-b)^{n-1}(a+(n-1)b)$ ce qui reste vrai quand $b = 0$.

\begin{center}
\shadowbox{
$\left|
\begin{array}{cccc}
a&b&\ldots&b\\
b&a&\ddots&\vdots\\
\vdots&\ddots&\ddots&b\\
b&\ldots&b&a
\end{array}
\right|=(a-b)^{n-1}(a+(n-1)b)$.
}
\end{center}
}
}
