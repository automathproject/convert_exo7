\uuid{qbUR}
\exo7id{1477}
\auteur{barraud}
\organisation{exo7}
\datecreate{2003-09-01}
\isIndication{false}
\isCorrection{false}
\chapitre{Espace euclidien, espace normé}
\sousChapitre{Projection, symétrie}

\contenu{
\texte{
Déterminer la matrice dans la base canonique de $\R^{3}$ de la projection
  orthogonale sur le plan d'équation $x+2y-3z=0$.

  En déduire la matrice de la symétrie orthogonale par rapport à ce plan.

  \medskip

  Dans un espace euclidien de dimension $n$, on considére un sous-espace
  $F$ de dimension $r$ et $(f_{1},...,f_{r})$ une base de orthonormée de
  cet espace. On not $p_{F}$ la projection orthogonale sur $F$, c'est à
  dire la projection sur $F$ associée à la décomposition $E=F\oplus
  F^{\bot}$. Montrer que :
  $$
  \forall v\in F, \qquad p_{F}(v)=<v,f_{1}>f_{1}
                                 +<v,f_{2}>f_{2}
                                 +\cdots
                                 +<v,f_{r}>f_{r}
  $$
}
}
