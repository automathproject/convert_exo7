\uuid{QRgR}
\exo7id{5489}
\auteur{rouget}
\datecreate{2010-07-10}
\isIndication{false}
\isCorrection{true}
\chapitre{Espace euclidien, espace normé}
\sousChapitre{Problèmes matriciels}

\contenu{
\texte{
$E=\Rr^3$ euclidien orienté rapporté à une base orthonormée directe $\mathcal{B}$.
Etudier les endomorphismes de matrice $A$ dans $\mathcal{B}$ suivants~:

$$\begin{array}{lll}
1)\;A=-\frac{1}{3}
\left(
\begin{array}{ccc}
-2&1&2\\
2&2&1\\
1&-2&2
\end{array}
\right)
&2/\;A=\frac{1}{4}
\left(
\begin{array}{ccc}
3&1&\sqrt{6}\\
1&3&-\sqrt{6}\\
-\sqrt{6}&\sqrt{6}&2
\end{array}
\right)&3/\;A=\frac{1}{9}
\left(
\begin{array}{ccc}
8&1&4\\
-4&4&7\\
1&8&-4
\end{array}
\right).
\end{array}
$$
}
\reponse{
$\|C_1\|=\|C_2\|=\frac{1}{3}\sqrt{4+4+1}=1$ et $C_1|C_2=\frac{1}{9}(-2+4-2)=0$. Enfin, 

$$C_1\wedge C_2=\frac{1}{9}\left(
\begin{array}{c}
-2\\
2\\
1
\end{array}
\right)\wedge\left(
\begin{array}{c}
1\\
2\\
-2
\end{array}
\right)
=\frac{1}{9}\left(
\begin{array}{c}
-6\\
-3\\
-6
\end{array}
\right)=-\frac{1}{3}\left(
\begin{array}{c}
2\\
1\\
2
\end{array}
\right)=C_3.$$ 
Donc, $A\in O_3^+(\Rr)$ et $f$ est une rotation (distincte de l'identité).
\textbf{Axe de $f$.} Soit $X\in\mathcal{M}_{3,1}(\Rr)$.

$$AX=X\Leftrightarrow\left\{
\begin{array}{l}
-x-y-2z=0\\
-2x-5y-z=0\\
-x+2y-5z=0
\end{array}
\right.\Leftrightarrow\left\{
\begin{array}{l}
z=-2x-5y\\
3x+9y=0\\
9x+27y=0
\end{array}
\right.\Leftrightarrow\left\{
\begin{array}{l}
x=-3y\\
z=y
\end{array}
\right..$$ 
L'axe $D$ de $f$ est $\mbox{Vect}(\overrightarrow{u})$ où $\overrightarrow{u}=(-3,1,1)$. $D$ est dorénavant orienté par $\overrightarrow{u}$.
\textbf{Angle de $f$.} Le vecteur $\overrightarrow{v}=\frac{1}{\sqrt{2}}(0,1,-1)$ est un vecteur unitaire orthogonal à l'axe. Donc,

$$\cos\theta=\overrightarrow{v}.f(\overrightarrow{v})=\frac{1}{\sqrt{2}}(0,1,-1).\frac{1}{\sqrt{2}}\frac{-1}{3}(-1,1,-4)=-\frac{1}{6}\times5
=-\frac{5}{6},$$
et donc, $\theta=\pm\Arccos(-\frac{5}{6})\;(2\pi)$. (Si on sait que $\text{Tr}(A)=2\cos\theta+1$, c'est plus court : $2\cos\theta+1=\frac{2}{3}-\frac{2}{3}-\frac{2}{3}$ fournit $\cos\theta=-\frac{5}{6}$).
Le signe de $\sin\theta$ est le signe de $[\overrightarrow{i},f(\overrightarrow{i}),\overrightarrow{u}]=\left|
\begin{array}{ccc}
1&\frac{2}{3}&-3\\
0&-\frac{2}{3}&1\\
0&-\frac{1}{3}&1\\
\end{array}
\right|=-\frac{1}{3}<0$. Donc, 

\begin{center}
\shadowbox{
$f$ est la rotation d'angle $-\Arccos(-\frac{5}{6})$ autour de $u=(-3,1,1)$.
}
\end{center}
$||C_1||=||C_2||=\frac{1}{4}\sqrt{9+1+6}=1$ et $C_1|C_2=\frac{1}{16}(3+3-6)=0$. Enfin, 

$$C_1\wedge C_2=\frac{1}{16}\left(
\begin{array}{c}
3\\
1\\
-\sqrt{6}
\end{array}
\right)\wedge\left(
\begin{array}{c}
1\\
3\\
\sqrt{6}
\end{array}
\right)
=\frac{1}{16}\left(
\begin{array}{c}
4\sqrt{6}\\
-4\sqrt{6}\\
8
\end{array}
\right)=\frac{1}{4}\left(
\begin{array}{c}
\sqrt{6}\\
-\sqrt{6}\\
2
\end{array}
\right)
=C_3.$$
Donc, $A\in O_3^+(\Rr)$ et $f$ est une rotation.
\textbf{Axe de $f$.} Soit $X\in\mathcal{M}_{3,1}(\Rr)$.

$$AX=X\Leftrightarrow\left\{
\begin{array}{l}
-x+y+\sqrt{6}z=0\\
x-y-\sqrt{6}z=0\\
-\sqrt{6}x+\sqrt{6}y-2z=0
\end{array}
\right.\Leftrightarrow x-y=\sqrt{6}z=\frac{2}{\sqrt{6}}z\Leftrightarrow x=y\;\mbox{et}\;z=0.$$ 
L'axe $D$ de $f$ est $\mbox{Vect}(\overrightarrow{u})$ où $\overrightarrow{u}=(1,1,0)$. $D$ est dorénavant orienté par $\overrightarrow{u}$.
\textbf{Angle de $f$.} $\overrightarrow{k}=[0,0,1)$ est un vecteur unitaire orthogonal à $\overrightarrow{u}$. Par suite,

$$\cos\theta=\overrightarrow{k}.f(\overrightarrow{k})=(0,0,1).\frac{1}{4}(\sqrt{6},-\sqrt{6},2)=\frac{1}{2},$$
et donc $\cos\theta=\pm\frac{\pi}{3}\;(2\pi)$. Le signe de $\sin\theta$ est le signe de $\left[\overrightarrow{i},f(\overrightarrow{i}),\overrightarrow{u}\right]
=\left|
\begin{array}{ccc}
1&3/4&1\\
0&1/4&1\\
0&-\sqrt{6}/4&0
\end{array}
\right|=\frac{1}{\sqrt{6}}>0$. Donc, 

\begin{center}
\shadowbox{
$f$ est la rotation d'angle $\frac{\pi}{3}$ autour de $\overrightarrow{u}=(1,1,0)$.
}
\end{center}
$||C_1||=||C_2||=\frac{1}{9}\sqrt{64+16+1}=1$ et $C_1|C_2=\frac{1}{81}(8-16+8)=0$. Enfin,

$$C_1\wedge C_2=\frac{1}{81}\left(
\begin{array}{c}
8\\
-4\\
1
\end{array}
\right)\wedge\left(
\begin{array}{c}
1\\
4\\
8
\end{array}
\right)=\frac{1}{81}\left(
\begin{array}{c}
-36\\
-63\\
36
\end{array}
\right)=-\frac{1}{9}\left(
\begin{array}{c}
4\\
7\\
-4
\end{array}
\right)=-C_3.$$
Donc, $A\in O_3^-(\Rr)$. $A$ n'est pas symétrique, et donc $f$ n'est pas une réflexion. $f$ est donc la composée commutative $s\circ r$ d'une rotation d'angle $\theta$ autour d'un certain vecteur unitaire $\overrightarrow{u}$ et de la réflexion de plan $\overrightarrow{u}^\bot$ où $\overrightarrow{u}$ et $\theta$ sont à déterminer.
\textbf{Axe de $r$.} L'axe de $r$ est $Ker(f+Id_E)$ (car $f\neq-Id_E$).

$$AX=-X\Leftrightarrow\left\{
\begin{array}{l}
17x+y+4z=0\\
-4x+13y+7z=0\\
x+8y+5z=0
\end{array}
\right.\Leftrightarrow\left\{
\begin{array}{l}
y=-17x-4z\\
-225x-45z=0\\
-135x-27z=0
\end{array}
\right.\Leftrightarrow\left\{
\begin{array}{l}
z=-5x\\
y=3x
\end{array}\right.$$
$\mbox{Ker}(f+Id_E)=\mbox{Vect}(\overrightarrow{u})=D$ où $u=(1,3,-5)$. $D$ est dorénavant orienté par $\overrightarrow{u}$.
$s$ est la réflexion par rapport au plan $P=u^\bot$ dont une équation est $x+3y-5z=0$.
On écrit alors la matrice $S$ de $s$ dans la base de départ. On calcule $S^{-1}A=SA$ qui est la matrice de $r$ et on termine comme en 1) et 2).
}
}
