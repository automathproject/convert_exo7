\uuid{7TZZ}
\exo7id{3740}
\auteur{quercia}
\organisation{exo7}
\datecreate{2010-03-11}
\isIndication{false}
\isCorrection{true}
\chapitre{Espace euclidien, espace normé}
\sousChapitre{Projection, symétrie}

\contenu{
\texte{
On définit l'application $\varphi : A\in\mathcal{M}_n(\R) \mapsto \sum_{i,j}a_{i,j}^2$.
Trouvez les matrices $P\in GL_n(\R)$ telles que
pour tout $A$ on ait $\varphi(P^{-1}AP)=\varphi(A)$.
}
\reponse{
$\varphi(A) = \mathrm{tr}(A\,{}^t\!A)$ donc pour toute matrice $P$
telle que $P\,{}^t\!P$ soit scalaire (non nulle) on a $\varphi(P^{-1}AP) = \varphi(A)$.
Ces matrices sont les matrices de la forme $P = \lambda M$ avec $M$ orthogonale
(matrices de similitude).

Réciproquement, soit $P$ telle que $\varphi(P^{-1}AP) = \varphi(A)$ et
$Q=P\,{}^t\!P$.

On a par polarisation~:
$\forall\ A,B,\ \mathrm{tr}(AQ\,{}^t\!B\,{}^t\!Q^{-1}) = \mathrm{tr}(A\,{}^t\!B)$ donc pour $B=QC$~:
$\forall\ A,C\ \mathrm{tr}(AQ{}^tC) = \mathrm{tr}(A{}^tCQ)$ ce qui implique~:
$\forall\ C,\ Q{}^tC = {}^tCQ$ et donc que $Q$ est scalaire.
}
}
