\uuid{hmj8}
\exo7id{3692}
\auteur{quercia}
\organisation{exo7}
\datecreate{2010-03-11}
\isIndication{false}
\isCorrection{true}
\chapitre{Espace euclidien, espace normé}
\sousChapitre{Produit scalaire, norme}

\contenu{
\texte{
Soit $E$ un espace euclidien et $u\in\mathcal{L}(E)$ tel que
$\forall\ x\in E,\ \|u(x)\| \le \|x\|$. Montrer que
$E = \mathrm{Ker}(u-\mathrm{id})\mathop\oplus\limits^\perp\Im(u-\mathrm{id})$.
}
\reponse{
Soient $x\in\mathrm{Ker}(u-\mathrm{id})$ et $y=u(z)-z\in\Im(u-\mathrm{id})$.
On a $y = u(z+\lambda x)-(z+\lambda x)$ d'où~:
$$\|z+\lambda x\|^2\ge \|u(z+\lambda x)\|^2 = \|z+\lambda x\|^2 + 2\lambda(x\mid y) + 2(z\mid y) + \|y\|^2.$$
En faisant tendre $\lambda$ vers $\pm\infty$ on obtient $(x\mid y) = 0$
et on conclut avec le théorème du rang.
}
}
