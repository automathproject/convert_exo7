\uuid{fall}
\exo7id{1507}
\auteur{barraud}
\organisation{exo7}
\datecreate{2003-09-01}
\isIndication{false}
\isCorrection{false}
\chapitre{Espace euclidien, espace normé}
\sousChapitre{Orthonormalisation}

\contenu{
\texte{
Soit $(E,<,>)$ un espace euclidien, $x_{0}$ un point de $E$ et $F$ un
sous espace vectoriel de $E$. On note $\pi$ la projection orthogonale de
$E$ sur $F$. On rappelle que pour $x\in E$, $\pi(x)$ est caractérisé par
les relations~: 
$$
\pi(x)\in F\qquad\text{ et }\qquad x-\pi(x)\in F^{\bot}
$$

Le but de cette partie est de montrer que la projection orthogonale de
$x_{0}$ sur $F$ est le point de $F$ le plus proche de $x_{0}$.
}
\begin{enumerate}
    \item \question{En utilisant que $x_{0}-y=(x_{0}-\pi(x_{0}))+(\pi(x_{0})-y)$, montrer
  que 
  $$
  \Vert x_{0}-y\Vert^{2}=\Vert x_{0}-\pi(x_{0})\Vert^{2}+\Vert
  y-\pi(x_{0})\Vert^{2}.
  $$}
    \item \question{En déduire que $\inf\limits_{y\in F}\Vert x_{0}-y\Vert^{2}=\Vert
  x_{0}-\pi(x_{0})\Vert^{2} $, c'est à dire que~:
  $$
  \forall y\in F,\ \Vert
  x_{0}-y\Vert^{2}\geq\Vert x_{0}-\pi(x_{0})\Vert^{2}
  $$
  A quelle condition a-t-on égalité dans la relation ci-dessus~?}
    \item \question{Soit $(e_{1},\dots,e_{k})$ une base orthonormée de $F$. Montrer que
  $\pi(x_{0})=\sum_{i=1}^{k}<e_{i},x_{0}>e_{i}$}
    \item \question{Déduire des deux questions précédentes que 
  $$
  \inf\limits_{y\in F}\Vert
  x_{0}-y\Vert^{2}=\Vert x_{0}-\sum_{i=1}^{k}<e_{i},x_{0}>e_{i}\Vert^{2}
  =\Vert x_{0}\Vert^{2}-\sum_{i=1}^{k}<e_{i},x_{0}>^{2}
  $$

\bigskip


{Application}~: Le but est maintenant de déterminer
$$
\alpha=\inf\limits_{a,b\in \R^{2}}\int_{-1}^{1} (e^{t}-at-b)^{2}dt.
$$ 

On considère à cet effet l'espace $F=\R_{1}[X]$, comme sous espace de
$E=F\oplus\R f_{0}$ où $f_{0}$ est la fonction définie par $f_{0}(t)=e^{t}$.
On admettra sans démonstration que $<f,g>=\int_{-1}^{1}f(t)g(t)dt$ est un
produit scalaire sur $E$.}
    \item \question{Donner une base orthonormée $(P_{1},P_{2})$ de $\R_{1}[X]$ pour ce produit
scalaire.}
    \item \question{Calculer $<f_{0},P_{1}>$, $<f_{0},P_{2}>$, et $\Vert f_{0}\Vert^{2}$. En
déduire que 
$$
 \alpha=\frac{e^{2}-e^{-2}}{2}-(2e^{-1})^{2}-\Big(\frac{e-e^{-1}}{2}\Big)^{2}.
$$}
    \item \question{Même question avec le calcul de $\alpha'=\inf\limits_{a,b\in
    \R^{2}}\int_{-1}^{1} (e^{t}-at^{2}-bt-c)^{2}dt. $~: commencer par
  chercher une base orthonormée de $\R_{2}[X]$ pour le même produit
  scalaire, et en déduire $\alpha'$.}
\end{enumerate}
}
