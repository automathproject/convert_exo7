\uuid{bVin}
\exo7id{1660}
\auteur{ortiz}
\organisation{exo7}
\datecreate{1999-04-01}
\isIndication{false}
\isCorrection{false}
\chapitre{Réduction d'endomorphisme, polynôme annulateur}
\sousChapitre{Diagonalisation}

\contenu{
\texte{
Les questions sont ind\'ependantes. $K$ d\'esigne
$\Rr$ ou $\Cc$, $E$ est un $K$-espace vectoriel de
dimension finie $n$, $\mathcal{B}=(e_1,...,e_n)$
est une base fix\'ee de $E$ et $f$ un
endomorphisme de $E$.
}
\begin{enumerate}
    \item \question{Quels sont les valeurs propres de l'endomorphisme nul de $E$ ?}
    \item \question{On suppose que la matrice de $f$ dans $\mathcal{B}$ est
     $M=\left(\begin{smallmatrix} 3&2&4\\-1&3&-1\\-2&-1&-3\end{smallmatrix}\right)$.
     \begin{enumerate}}
    \item \question{2 est-il valeur propre de $f$ ?}
    \item \question{Le vecteur $2e_1+e_2+e_3$ est-il un vecteur propre de $f$ ?}
\end{enumerate}
}
