\uuid{PFgW}
\exo7id{2584}
\auteur{delaunay}
\datecreate{2009-05-19}
\isIndication{false}
\isCorrection{true}
\chapitre{Réduction d'endomorphisme, polynôme annulateur}
\sousChapitre{Diagonalisation}

\contenu{
\texte{
Soit $A$ une matrice $2\times 2$ \`a coefficients r\'eels. On suppose que dans chaque colonne de $A$ la somme des coefficients est \'egale \`a $1$.
}
\begin{enumerate}
    \item \question{Soient $(x_1,x_2)$, $(y_1,y_2)$ deux vecteurs de $\R^2$, on suppose que  
$$A\begin{pmatrix}x_1 \\  x_2\end{pmatrix}=\begin{pmatrix}y_1 \\  y_2\end{pmatrix}$$ 
montrer qu'alors 
$$y_1+y_2=x_1+x_2.$$}
\reponse{Soient $(x_1,x_2)$, $(y_1,y_2)$ deux vecteurs de $\R^2$, on suppose que  
$$A\begin{pmatrix}x_1 \\  x_2\end{pmatrix}=\begin{pmatrix}y_1 \\  y_2\end{pmatrix}$$ 
montrons qu'alors 
$$y_1+y_2=x_1+x_2.$$

Compte tenu des hypoth\`eses, la matrice $A$ est de la forme
$$\begin{pmatrix}a&b \\ 1-a&1-b\end{pmatrix},$$
o\`u $a$ et $b$ sont des r\'eels. On a alors 
$$\begin{pmatrix}a&b \\ 1-a&1-b\end{pmatrix}\begin{pmatrix}x_1 \\  x_2\end{pmatrix}=\begin{pmatrix}y_1 \\  y_2\end{pmatrix}\iff
\left\{\begin{align*}ax_1+bx_2&=y_1 \\  (1-a)x_1+(1-b)x_2&=y_2\end{align*}\right.$$
ce qui implique $y_1+y_2=x_1+x_2$.}
    \item \question{Soit le vecteur $\varepsilon=(1,-1)$, montrer que c'est un vecteur propre de $A$.
On notera $\lambda$ sa valeur propre.}
\reponse{Montrons que le vecteur $\varepsilon=(1,-1)$ est un vecteur propre de $A$.

Si $A\varepsilon=\begin{pmatrix}y_1 \\  y_2\end{pmatrix}$, alors $y_1+y_2=0$ donc $y_2=-y_1$ et $A\varepsilon=y_1\varepsilon$, ce qui prouve que $\varepsilon$ est un vecteur propre. On peut aussi le voir de la mani\`ere suivante 
$$A\varepsilon=\begin{pmatrix}a&b \\ 1-a&1-b\end{pmatrix}\begin{pmatrix}1 \\  -1\end{pmatrix}=
\begin{pmatrix}a-b \\  b-a\end{pmatrix}=(a-b)\varepsilon.$$
On note $\lambda=(a-b)$ sa valeur propre.}
    \item \question{Montrer que si $v$ est un vecteur propre de $A$ non colin\'eaire \`a $\varepsilon$, alors la valeur propre
associ\'ee \`a $v$ est \'egale \`a $1$.}
\reponse{Montrons que si $v$ est un vecteur propre de $A$ non colin\'eaire \`a $\varepsilon$, alors la valeur propre associ\'ee \`a $v$ est \'egale \`a $1$.

Soit $v=(x_1,x_2)$ un vecteur propre de $A$ non colin\'eaire \`a $\varepsilon$, notons $\mu$ sa valeur propre, on a $Av=\mu v$, et, d'apr\`es la question $1)$, on a $$x_1+x_2=\mu x_1+\mu x_2=\mu(x_1+x_2)$$
ce qui implique $\mu=1$ car $v$ est suppos\'e non colin\'eaire \`a $\varepsilon$ donc $x_1+x_2\neq0$.}
    \item \question{Soit $e_1=(1,0)$. Montrer que la matrice, dans la base $(e_1,\varepsilon)$, de l'endomorphisme associ\'e \`a $A$  est de la forme
$$\begin{pmatrix}1&0  \\ \alpha&\lambda \end{pmatrix},$$
o\`u $\alpha\in\R$.

En d\'eduire que si $\lambda\neq 1$, alors $A$ est diagonalisable sur $\R$.}
\reponse{Soit $e_1=(1,0)$. Montrons que la matrice, dans la base $(e_1,\varepsilon)$, de l'endomorphisme associ\'e \`a $A$  est de la forme
$$\begin{pmatrix}1&0  \\ \alpha&\lambda \end{pmatrix},$$
o\`u $\alpha\in\R$.

Pour cela on \'ecrit $Ae_1$ et $A\varepsilon$ dans la base $(e_1,\varepsilon)$. 
On a d'une part $A\varepsilon=\lambda\varepsilon$ et, d'autre part,
$$\begin{pmatrix}a&b \\ 1-a&1-b\end{pmatrix}\begin{pmatrix}1 \\  0\end{pmatrix}=\begin{pmatrix}a \\  1-a\end{pmatrix}
=\begin{pmatrix}1 \\  0\end{pmatrix}+(a-1)\begin{pmatrix}1 \\  -1\end{pmatrix}.$$ 
D'o\`u la matrice dans la base $(e_1,\varepsilon)$
$$\begin{pmatrix}1&0 \\  \alpha&\lambda\end{pmatrix}$$
o\`u $\alpha=a-1$ et $\lambda=a-b$.

On en d\'eduit que si $\lambda\neq 1$, alors $A$ est diagonalisable sur $\R$.
Le polyn\^ome caract\'eristique de $A$ est \'egal \`a $(1-X)(\lambda-X)$, ainsi, si $\lambda\neq 1$, il admet deux racines distinctes ce qui prouve que $A$ est diagonalisable.}
\end{enumerate}
}
