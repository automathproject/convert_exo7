\uuid{p59L}
\exo7id{3733}
\auteur{quercia}
\organisation{exo7}
\datecreate{2010-03-11}
\isIndication{false}
\isCorrection{false}
\chapitre{Espace euclidien, espace normé}
\sousChapitre{Projection, symétrie}

\contenu{
\texte{
Soit $E$ un espace euclidien de dimension $n$.
Soit $f : E \to E$ une application non nécéssairement linéaire.
}
\begin{enumerate}
    \item \question{On suppose que $f$ conserve le produit scalaire.
    Démontrer que $f$ est linéaire.}
    \item \question{On suppose que $f$ conserve les distances, c'est à dire :
    $\forall\ \vec x,\vec y \in E,\ \| f(\vec x) - f(\vec y) \| = \|\vec x - \vec y\,\|$.
    Démontrer que $f = f(\vec0) + g$, avec $g \in {\cal O}(E)$.}
\end{enumerate}
}
