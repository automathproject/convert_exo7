\uuid{HBeI}
\exo7id{3566}
\auteur{quercia}
\datecreate{2010-03-10}
\isIndication{false}
\isCorrection{true}
\chapitre{Réduction d'endomorphisme, polynôme annulateur}
\sousChapitre{Polynôme annulateur}

\contenu{
\texte{
Soit $E$ un $ K$-ev de dimension~$n$.
Soit $u\in\mathcal{L}(E)$, $P$ son polynôme minimal et $p$ le plus petit exposant
de~$X$ dans l'écriture de~$P$.
}
\begin{enumerate}
    \item \question{Si $p=0$, que dire de~$u$~?}
\reponse{Que c'est un isomorphisme (et réciproquement).}
    \item \question{Si $p=1$, montrer que $E=\Im u \oplus \mathrm{Ker} u$.}
\reponse{Soit $Q(X) = P(X)/X$. On a $u\circ Q(u)=0$ et $X, Q$ sont
    premiers entre eux, d'où $E=\mathrm{Ker} u \oplus \mathrm{Ker} Q(u)$ et
    $\Im u \subset \mathrm{Ker} Q(u)$. On conclut avec le théorème du rang.}
    \item \question{Dans le cas général, montrer que $E = \mathrm{Ker} u^p \oplus \Im u^p$.}
\reponse{Même méthode.}
\end{enumerate}
}
