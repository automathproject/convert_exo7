\uuid{ZOld}
\exo7id{5634}
\auteur{rouget}
\organisation{exo7}
\datecreate{2010-10-16}
\isIndication{false}
\isCorrection{true}
\chapitre{Espace euclidien, espace normé}
\sousChapitre{Produit scalaire, norme}

\contenu{
\texte{
Rang du système de formes linéaires sur $\Rr^4$ 

\begin{center}
$\begin{array}{l}
f_1 = x_1 + 2x_2 - x_3 - 2x_4\\
f_2 = x_1 + x_2 + mx_3 + x_4\\
f_3 = x_1+ x_3 + (m+4)x_4\\
f_4=x_2 - 3x_3 - mx_4
\end{array}
$ ?
\end{center}
}
\reponse{
La matrice de la famille $(f_1,f_2,f_3,f_4)$ dans la base canonique du dual de $\Rr^4$ est $A =\left(
\begin{array}{cccc}
1&1&1&0\\
2&1&0&1\\
-1&m&1&-3\\
-2&1&m+4&-m
\end{array}
\right)$. La matrice $A$ a même rang que la matrice $\left(
\begin{array}{cccc}
1&0&0&0\\
2&-1&-2&1\\
-1&m+1&2&-3\\
-2&3&m+6&-m
\end{array}
\right)$ (pour $2\leqslant j\leqslant 3$, $C_j\leftarrow C_j-C_1$) puis que la matrice $\left(
\begin{array}{cccc}
1&0&0&0\\
2&-1&0&0\\
-1&m+1&-2m&m-2\\
-2&3&m&-m+3
\end{array}
\right)
$ ($C_3\leftarrow C_3-2C_2$ et $C_4\leftarrow C_4+C_2$)

\textbullet~Si $m = 0$, $A$ a même rang que la matrice $\left(
\begin{array}{cccc}
1&0&0\\
2&-1&0\\
-1&1&-2\\
-2&3&3
\end{array}
\right)
$et donc $\text{rg}(A)=3$.

\textbullet~Si $m\neq0$, $A$ a même rang que la matrice $\left(
\begin{array}{cccc}
1&0&0&0\\
2&-1&0&0\\
-1&m+1&-2&m-2\\
-2&3&1&-m+3
\end{array}
\right)
$ ($C_3\leftarrow\frac{1}{m}C_3$) puis que la matrice $\left(
\begin{array}{cccc}
1&0&0&0\\
2&-1&0&0\\
-1&m+1&-2&0\\
-2&3&1&-m+4
\end{array}
\right)
$ ($C_4\leftarrow2C_4+(m-2)C_3$)

Donc, si $m =4$, $\text{rg}(A)= 3$ et si $m$ n'est ni $0$ ni $4$, $\text{rg}(A)=4$.

\begin{center}
\shadowbox{
Si $m\notin\{0,4\}$, $\text{rg}(f_1,f_2,f_3,f_4)=4$ et si $m\in\{0,4\}$, $\text{rg}(f_1,f_2,f_3,f_4)=3$.
}
\end{center}
}
}
