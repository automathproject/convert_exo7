\uuid{unWM}
\exo7id{2291}
\auteur{barraud}
\datecreate{2008-04-24}
\isIndication{false}
\isCorrection{true}
\chapitre{Anneau, corps}
\sousChapitre{Anneau, corps}

\contenu{
\texte{
Soit $A$ un anneau, $B$ un sous-anneau de $A$,
$I$ un id\'eal de $A$.
}
\begin{enumerate}
    \item \question{Montrer que $B\cap I$ est un id\'eal de $B$,
$B+I=\{b+i\,|\, b\in B,\ i\in I\}$ est un sous-anneau
de l'anneau $A$ et $I$ est un id\'eal de ce sous-anneau.}
\reponse{Soit $J=B\cap I$. Soit $x,y\in J$, $a,b\in B$, alors $ax+by\in B$
    puisque $B$ est un sous-anneau de $A$. $ax+by\in I$ puisque $I$ est
    un idéal. On en déduit que $J$ est un idéal.

    $B+I$ est stable par addition (car $B$ et $I$ le sont). Soit
    $\alpha=a+x\in B+I$ et $\beta= b+y\in B+I$. Alors
    $\alpha\beta=(ab)+(ay+bx+xy)\in B+I$, donc $B+I$ est stable par
    multiplication. $1\in B+I$, donc $B+I$ est un sous anneau de $A$.
    $I\subset B+I$, et $I$ est absorbant pour la multiplication dans $A$,
    donc aussi dans $B$~: $I$est un idéal de $B+I$.}
    \item \question{Montrer que l'anneau quotient $B/(B\cap I)$  est isomorphe \`a
l'anneau quotient $(B+I)/I$. ({\it Indication :} Consid\'erer  le compos\'e de l'inclusion
$B\to B+I$ avec la projection canonique  $B+I \to (B+I)/I$.)}
\reponse{On a le diagramme (de morphismes d'anneaux) suivant~:
    $$
    \xymatrix{%
      B\ar[r]^{i} 
       \ar[d]^{{\pi_{0}}}
       \ar@(ur,ul)[rr]^{\phi}
      & B+I \ar[r]^{\pi} & (B+I)/I\\ 
      B/\ker\phi\ar[urr]_{\sim}
    }%
    $$
    Or, pour $x\in B$, on a~: $x\in\ker\phi \Leftrightarrow x=i(x)\in
    \ker\pi=I$. Donc $\ker\phi=B\cap I$, et par suite~:
    $$
    B/(B\cap I)\sim (B+I)/I.
    $$}
\end{enumerate}
}
