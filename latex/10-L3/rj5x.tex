\uuid{rj5x}
\exo7id{7735}
\auteur{mourougane}
\organisation{exo7}
\datecreate{2021-08-11}
\isIndication{false}
\isCorrection{false}
\chapitre{Forme bilinéaire}
\sousChapitre{Forme bilinéaire}

\contenu{
\texte{

}
\begin{enumerate}
    \item \question{Soit $E$ un espace muni d'une forme sesquilinéaire non dégénérée réflexive.
Soit $V$ un sous-espace de $E$. Supposons que le radical $\text{rad}(V)=V\cap V^\perp$ de $V$ est de dimension $2$, $\text{rad}(V)=\vec (N_1,N_2)$.
Soit $W$ un supplémentaire de $\text{rad}(V)$ dans $V$.
Calculer en détails le radical de $V'=\vec (N_1)\oplus W$.}
    \item \question{Soit $E$ un espace muni d'une forme alternée non-dégénérée $f$.
Soit $G$ le groupe (dit symplectique) des isométries de $(E,f)$.
Combien y a-t-il d'orbites dans l'action du groupe $G$ sur l'ensemble $P(E)$ des droites de $E$ ?}
    \item \question{Soit $E$ un espace muni d'une forme alternée non-dégénérée $f$. Quelles sont les restrictions possibles à équivalence près de $f$ sur les plans de $E$ ?}
    \item \question{Combien y a-t-il d'orbites dans l'action du groupe symplectique d'un espace vectoriel $E$ de dimension $6$ muni d'une forme alternée non-dégénérée sur l'ensemble des plans de $E$ ?}
\end{enumerate}
}
