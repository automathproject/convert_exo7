\uuid{Wd1H}
\exo7id{2270}
\auteur{barraud}
\organisation{exo7}
\datecreate{2008-04-24}
\isIndication{false}
\isCorrection{true}
\chapitre{Polynôme}
\sousChapitre{Polynôme}

\contenu{
\texte{
Pour quel $n$, $m$ dans $\Zz$ la fraction
$$
\frac{11n+2m}{18n+5m}
$$
est r\'eductible ?
}
\reponse{
Supposons que la fraction soit réductible. Alors, il existe
  $p,q,d\in\Zz$ tels que
  $$
  \begin{cases}
    11n+2m&=pd\\
    18n+5m&=qd\\
  \end{cases}
  $$
  On en déduit que 
  $$
  \begin{cases}
    19n&=5pd-2qd\\
    19m&=-18pd+1qd\\
  \end{cases}
  $$
  En particulier, $d|19n$ et $d|19m$. Si $d\neq19$, on a $\pgcd(n,m)\neq
  1$. Si $d=19$, alors 
  \begin{equation}\label{eq:n=5p-2q...}
  \begin{cases}
    n&=5p-2q\\
    m&=-18p+1q\\
  \end{cases}
  \end{equation}
  Réciproquement, si $\pgcd(n,m)\neq1$ ou si $n,m$ sont de la forme
  donnée par \eqref{eq:n=5p-2q...}, alors la fraction est réductible.
}
}
