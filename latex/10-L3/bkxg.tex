\uuid{bkxg}
\exo7id{7725}
\auteur{mourougane}
\organisation{exo7}
\datecreate{2021-08-11}
\isIndication{false}
\isCorrection{true}
\chapitre{Groupe fini}
\sousChapitre{Groupe fini}

\contenu{
\texte{

}
\begin{enumerate}
    \item \question{Déterminer le centre de $SL(4,\mathbb{F}_2)$ et calculer l'ordre de $PSL(4,\mathbb{F}_2)$.}
\reponse{Comme le seul élément non nul de $\mathbb{F}^2$ est $1$, $SL(4,\mathbb{F}^2)=GL(4,\mathbb{F}_2)$. Le centre de $GL(4,\mathbb{F}_2)$ composé des homothéties inversibles est donc réduit à l'identité ainsi que celui de $SL(4,\mathbb{F}_2)$. Ainsi, $|PSL(4,\mathbb{F}^2)|=|GL(4,\mathbb{F}^2)|=(2^4-1)(2^4-2)(2^4-2^2)(2^4-2^3)=15\times 14\times 12\times 8=7\times 5\times 3^2\times 2^6$.}
    \item \question{Dans $SL(4,\mathbb{F}_2)$ déterminer l'ordre de 
$$A=\left[\begin{array}{cccc}1&0&0&0\\0&1&0&0\\0&0&1&1\\0&0&0&1
    \end{array}\right] 
\textrm{ et de }
B=\left[\begin{array}{cccc}1&1&0&0\\0&1&0&0\\0&0&1&1\\0&0&0&1
    \end{array}\right]$$
Ces deux matrices sont-elles conjuguées dans $SL(4,\mathbb{F}_2)$ ?}
\reponse{Ces deux matrices sont d'ordre $2$.
 Elles ne sont pas conjuguées car la dimension des espaces propres de valeurs propres $1$ n'est pas la même pour les deux.}
    \item \question{Déterminer le centre de $SL(3,\mathbb{F}_4)$ et calculer l'ordre de $PSL(3,\mathbb{F}_4)$.}
\reponse{Tous les éléments de $\mathbb{F}^\times_4$ sont des racines $3$-ième d'unité par le théorème de Lagrange. Par conséquent toutes les homothéties inversibles sont dans le centre de $SL(3,\mathbb{F}_4)$.
 Donc, 
 $|PSL(3,\mathbb{F}^4)=GL(3,\mathbb{F}_4)/(3\times 3)=(4^3-1)(4^3-4)(4^3-4^2)/9
 =63\times 60\times 48/9=7\times 5\times 3^2\times 2^6$.}
    \item \question{Le but de cette question est de montrer que toutes les involutions de $PSL(3,\mathbb{F}_4)$ sont des applications projectives associées à des transvections.
Soit $F$ une homographie de $PSL(3,\mathbb{F}_4)$ involutive (i.e. $F^2=Id\in PSL(3,\mathbb{F}_4)$) et $f\in SL(3,\mathbb{F}_4)$ telle que $P(f)=F$. 
\begin{enumerate}}
\reponse{\begin{enumerate}}
    \item \question{Montrer que $P(f^3)=F$ et que $f^3$ est d'ordre $2$ dans $SL(3,\mathbb{F}_4)$. On notera $g=f^3$.}
\reponse{$P(f^3)=F^3=F$ car $F$ est une involution.
 $g=f^3$ n'est donc pas l'identité, mais $g^2=f^6=(f^2)^3=Id$ car $f^2$ est dans le centre de $SL(3,\mathbb{F}_4)$ qui est d'ordre $3$.}
    \item \question{Quelle est la caractéristique de $\mathbb{F}_4$ ?
Montrer que $(g-Id)^2=0$. Déterminer $\dim Ker(g-Id)$ et $\dim Im(g-Id)$.}
\reponse{Le corps $\mathbb{F}_4$ est de caractéristique $2$. $(g-Id)^2=g^2-2g+Id=0$. Par conséquent, $\dim Im(g-Id)\leq\dim Ker(g-Id)$ et $\dim Im(g-Id)+\dim Ker(g-Id)=3$. Comme $g\not=Id$, $\dim Ker(g-Id)=2$.}
    \item \question{Conclure.}
\reponse{Ainsi $g$ est un élément de $SL(3,\mathbb{F}_4)$ qui admet un plan de points fixes. C'est donc une transvection.}
\end{enumerate}
}
