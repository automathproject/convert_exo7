\uuid{VIXb}
\exo7id{2300}
\auteur{barraud}
\datecreate{2008-04-24}
\isIndication{false}
\isCorrection{true}
\chapitre{Anneau, corps}
\sousChapitre{Anneau, corps}

\contenu{
\texte{
Soient $A$ un anneau et $I$ et $J$ les  idéaux de $A$ tels que $I+J=(1)$. Démontrer que $I^n+J^m=(1)$ quels que soient entiers positifs non-nuls $n$ et $m$.
}
\reponse{
$1\in I+J$ donc $\exists(x,y)\in I\times J, 1=x+y$. En multipliant
  cette égalité par $x$, on obtient $x^{2}+xy=x$. On en déduit que $xy\in
  I$, donc $\forall p\in\Nn~; x^{p}y\in I^{p}$, et donc
  $\forall(p,q)\in\Nn^{2}, x^{p}y^{q}\in I^{p}$. Par symétrie, on a aussi
  $\forall(p,q)\in\Nn^{2}, x^{p}y^{q}\in J^{q}$.

  Soit maintenant $(m,n)\in\Nn^{2}$. Notons $N=2\sup(m,n)$. Alors
  $1=1^{N}=(x+y)^{N}=\sum_{p+q=N}C^{p}_{N}x^{p}y^{q}$. Comme~:
  $(p+q=2N)\Rightarrow (p\geq n \text{ ou }q\geq m)$, tous les termes de
  cette somme sont dans $I^{n}$ ou dans $J^{m}$, et donc $1\in I^{n}+J^{m}$
}
}
