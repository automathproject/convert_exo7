\uuid{3q8e}
\exo7id{2287}
\auteur{barraud}
\datecreate{2008-04-24}
\isIndication{false}
\isCorrection{true}
\chapitre{Anneau, corps}
\sousChapitre{Anneau, corps}

\contenu{
\texte{
Soit $A$ un anneau int\`egre. On appelle
{\it \'el\'ement premier} de $A$ un \'el\'ement qui engendre un id\'eal
principal premier.
}
\begin{enumerate}
    \item \question{Montrer que  un \'el\'ement premier est irr\'eductible.}
    \item \question{D'apr\`es le cours  tout  \'el\'ement irr\'eductible dans un anneau  
factoriel  est  premier. 
Montrer que dans un anneau factoriel, tout id\'eal premier
non nul contient un \'el\'ement irr\'eductible.}
    \item \question{Nous avons vu que l'\'el\'ement $3\in \Zz[\sqrt{-5}]$ 
est irr\'eductible. Montrer que $3$ n'est pas premier 
dans $\Zz[\sqrt{-5}]$.}
    \item \question{L'\'el\'ement  $2$ est-il irr\'eductible dans l'anneau 
$\Zz[\sqrt{-5}]$ ?}
\reponse{
\begin{itemize}
  \item 
    Si $x\in A$ est premier~: soit $a,b\in A$ tels que $ab=x$. Alors
    $ab\in(x)$ donc $a\in(x)$ ou $b\in(x)$. On en déduit que $a\sim x$ ou
    $b\sim x$. Donc $x$ est irréductible.

  \item
    $A$ est supposé factoriel. Soit $I$ un idéal premier. Soit $x\in I$
    et $x=p_{1}\dots p_{k}$ ``la'' factorisation de $x$ en produit
    d'irréductibles. Alors $(p_{1}\cdots p_{n-1})p_{n}\in I$ donc
    $(p_{1}\cdots p_{n-1})\in I$ ou $p_{n}\in I$. si $p_{n}$ in $I$, $I$
    contient un irréductible. Sinon, $(p_{1}\cdots p_{n-2})p_{n-1}\in I$.
    Par une récurrence finie, l'un au moins des $p_{i}\in I$, donc $I$
    contient un irréductible.

  \item
    Dans $\Zz[\sqrt{-5}]$, $9\in(3)$. Pourtant
    $9=(2+\sqrt{-5})(2-\sqrt{-5})$ et $(2\pm\sqrt{-5})\notin(3)$. Donc
    $(3)$ n'est pas premier.

  \item
    $2$ est irréductible~: $2=z_{1}z_{2}$ avec $z_{i}\in\Zz[\sqrt{-5}]$,
    alors $|z_{1}|^{2}|z_{2}|^{2}=4$, donc
    $\{|z_{1}|^{2},|z_{2}|^{2}\}=\{1,4\}$ ou $\{2,2\}$. Dans le premier
    cas, on a affaire à une factorisation triviale. Le second est
    impossible, puisque l'équation $a^{2}+5b^{2}=2$ n'a pas de solution
    entière $(a,b)$.

    Par ailleurs, $(1+\sqrt{-5})(1+\sqrt{-5})=6\in(2)$, mais
    $(1\pm\sqrt{-5})\notin(2)$ donc $2$ n'est pas premier dans
    $\Zz[\sqrt{-5}]$.
  \end{itemize}
}
\end{enumerate}
}
