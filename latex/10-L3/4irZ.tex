\uuid{4irZ}
\exo7id{6475}
\auteur{drutu}
\organisation{exo7}
\datecreate{2011-10-16}
\isIndication{false}
\isCorrection{false}
\chapitre{Groupe fini}
\sousChapitre{Groupe fini}

\contenu{
\texte{

}
\begin{enumerate}
    \item \question{Soit les applications $T:\R \cup\{ \infty \} \to \R \cup\{ \infty \} ,\; T(x)=x+4,\; T(\infty )=\infty $, et $g:\R \cup\{ \infty \} \to \R \cup\{ \infty \}$, 
$$
g(x)=\left\{ \begin{array}{ccc} 
          \frac{x}{1-2x} & \mbox{ si } & x\in \R\setminus \{ \frac{1}{2}\} \; ; \\
          \infty & \mbox{ si } & x= \frac{1}{2} \; ;\\ 
          -\frac{1}{2} & \mbox{ si } & x= \infty \;   
          \end{array}\right.
$$ 

Montrer que $T^k([-2,2])\subset ]-\infty ,\, -2]\cup [2,\, \infty [,\; \forall k\in \Z^* $ et que $g^m(]-\infty ,\, -1]\cup [1,\, \infty [)\subset [-1,\, 1],\; \forall m\in \Z^* $.}
    \item \question{Soit $G$ le groupe des applications bijectives $h:\R\cup \{ \infty \} \to \R\cup \{ \infty \}$, muni de l'opération de composition. Montrer que $G$ a une action naturelle sur $\R \cup\{ \infty \}$.}
    \item \question{Montrer que le sous-groupe $\Gamma $ de $G$ engendré par $T$ et $g$ est un groupe libre. {\it Indication }: Regarder les orbites des nombres dans l'intervalle $]1,2[$.}
\end{enumerate}
}
