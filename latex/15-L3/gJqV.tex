\uuid{gJqV}
\exo7id{2421}
\auteur{drutu}
\datecreate{2007-10-01}
\isIndication{false}
\isCorrection{true}
\chapitre{Espace topologique, espace métrique}
\sousChapitre{Espace topologique, espace métrique}

\contenu{
\texte{
Soit $X$ un espace muni d'une m\'etrique $\mathrm{dist} : X\times X \to
\R_+$.
}
\begin{enumerate}
    \item \question{Montrer que si $f:\R_+ \to R_+$ est une fonction
  croissante telle que $f(0)=0$ et $f(x+y)\leq f(x) +f(y)$ alors
  $\mathrm{dist}_f (x,y)=f(\mathrm{dist} (x,y))$ est une m\'etrique sur $X$.}
\reponse{On v\'erifie facilement les trois propri\'et\'es de
m\'etrique.}
    \item \question{Montrer que
$$
\mathrm{dist}'(x,y)=\frac{\mathrm{dist}(x,y)}{1+\mathrm{dist} (x,y)}\, ,\; \forall x,y,
$$ est une m\'etrique sur $X$.}
\reponse{Soit $f:\R_+ \to \R_+ \, ,\,
f(x)=\frac{x}{x+1}=1-\frac{1}{x+1}$. On a que $f(0)=0$ et
$f'(x)=\frac{1}{(x+1)^2}$,
  donc la fonction $f$ est croissante sur $\R_+$. L'in\'egalit\'e
  $f(x+y)\leq f(x)+f(y)$ pour $x,y\in \R_+$ est \'equivalente \`a $$\frac{1}{x+y+1}\geq \frac{1}{x+1}+\frac{1}{y+1}-1
  \Leftrightarrow \frac{1}{x+y+1}+1 \geq \frac{1}{x+1}+\frac{1}{y+1}\Leftrightarrow 1+x+y\leq (1+x)(1+y)\, .$$

La derni\`ere \'egalit\'e est \'evidemment v\'erifi\'ee pour
$x\geq 0\, ,\, y\geq 0\, $.}
    \item \question{Montrer que les m\'etriques $\mathrm{dist}$ et $\mathrm{dist}'$ sont
  topologiquement \'equivalentes.}
\reponse{D'apr\`es le cours, la m\'etrique $\mathrm{dist}$ et la m\'etrique
$\mathrm{dist}_2=\min (\mathrm{dist} ,1)$ sont topologiquement \'equivalentes.
Ainsi il suffit de montrer que $\mathrm{dist}_1$ et $\mathrm{dist}_2$ sont
topologiquement \'equivalentes.

Puisque $1+\mathrm{dist} \geq 1$, on a que $ \mathrm{dist}_1\leq \mathrm{dist}$. Aussi
$\mathrm{dist}_1\leq 1$, d'o\`u $\mathrm{dist}_1\leq \mathrm{dist}_2$.

La fonction $f$ \'etant croissante, pour tout $x,y$ on a que
$\mathrm{dist}_1(x,y)= f(\mathrm{dist} (x,y)) \geq f(\mathrm{dist}_2(x,y))$. D'autre part,
$\mathrm{dist}_2(x,y)\leq 1$ implique
$f(\mathrm{dist}_2(x,y))=\frac{\mathrm{dist}_2(x,y)}{1+\mathrm{dist}_2(x,y)}\geq
\frac{\mathrm{dist}_2(x,y)}{2}$.

On a obtenu que pour tout $x,y$,
$$\frac{\mathrm{dist}_2(x,y)}{2}\leq \mathrm{dist}_1(x,y)\leq \mathrm{dist}_2(x,y)\, .$$

Ainsi, les m\'etriques $\mathrm{dist}_1$ et $\mathrm{dist}_2$ sont \'equivalentes.}
\end{enumerate}
}
