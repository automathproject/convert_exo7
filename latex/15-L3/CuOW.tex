\uuid{CuOW}
\exo7id{2399}
\auteur{mayer}
\datecreate{2003-10-01}
\isIndication{true}
\isCorrection{true}
\chapitre{Espace métrique complet, espace de Banach}
\sousChapitre{Espace métrique complet, espace de Banach}

\contenu{
\texte{
Soit $X$ l'espace des suites r\'eelles nulles \`a partir d'un certain rang, et soit
$$\rho (x,y) =\sum _{k=1}^\infty 2^{-k}
\frac{|x_k-y_k|}{1+|x_k-y_k|}\quad  \text{pour} \; x,y\in X\; .$$
}
\begin{enumerate}
    \item \question{Montrer que $X$ n'est pas complet pour la m\'etrique $\rho $.}
    \item \question{Trouver un espace de suites $Y$ tel que $(Y, \rho )$ soit complet et
tel que $X$ soit dense dans $Y$.}
    \item \question{Que donne l'exercice si on remplace $\rho$ par la norme uniforme?}
\reponse{
Notons $x^p$ la suite 
$$x^p = (1,1,\ldots,1,1,0,0,0,\ldots).$$
(des $0$ à partir de la $p+1$-ième place et de $1$ avant.
Si $Y$ est l'espace de toute les suite, notons 
$$x^\infty = (1,1,1,1,\ldots).$$
La suite $x^\infty$ n'est pas dans $X$.
Par contre $x^p \rightarrow x^\infty$ pour la distance $\rho$.
En effet
$$\rho(x^p,x) = \sum_{k=p+1}^{+\infty} \frac{1}{2^k} \frac12 = \frac 1 {2^{p+1}} \rightarrow 0.$$
La suite $(x^p)$ est de Cauchy, mais ne converge pas dans $X$, donc $X$ n'est pas complet.
\begin{enumerate}
Soit $Y$ l'espace de toutes les suites. Alors $X$ est dense dans
dans $Y$ (pour la topologie définie par $\rho$), car toute suite $y = (y_1,y_2,\ldots)$ de $Y$ 
s'approche par une suite de suite $(x^p)$ obtenue en tronquant la suite $y$ : $x^1 = (y(1),0,0,\ldots)$, $x^2 = (y(1),y(2),0,0,\ldots)$,... 
En effet 
$$\rho(x^p,y)= \sum _{k=p+1}^\infty 2^{-k}
\frac{|x_k-y_k|}{1+|x_k-y_k|} \le \sum _{k=p+1}^\infty 2^{-k} = \frac{1}{2^p}$$
qui tend vers $0$ lorsque $p$ tend vers $+\infty$.
Soit $(x^n)_n$ une suite de Cauchy de $Y$.
Montrons que pour $k$ fixé alors $(x^n_k)_n$ est une suite de Cauchy de $\Rr$.
Prenons $\epsilon >0$, alors il existe $N$ tel que pour $p,q\ge N$ on ait
$\rho(x^p,x^q) \le \epsilon$.
$$\frac {1}{2^k}\frac{|x^p_k-x^q_k|}{1+|x^p_k-x^q_k|} \le \rho(x^p,x^q) \le \epsilon.$$
Posons la fonction $f(\alpha)=\frac{\alpha}{1+\alpha}$, $f$ est inversible pour $\alpha\ge 0$,
d'inverse $f^{-1}(\beta)= \frac{\beta}{1-\beta}$. Une étude de $f$ et de son inverse
montre que si $f(\alpha)\le \epsilon'\le 1$ alors $\alpha \le 2\epsilon'$.
Comme $k$ est fixé, posons $\epsilon = \frac{\epsilon'}{2^k}$ et $\alpha =|x^p_k-x^q_k|$  on a montrer :
$f(\alpha) \le \epsilon'$. Donc $\alpha \le 2\epsilon'$.
Récapitulons :
$$\forall \epsilon' >0 \quad \exists N \in \Nn \quad \forall p,q \ge N \quad
|x^p_k-x^q_k| < 2\epsilon',$$
donc la suite $(x^n_k)_n$ est de Cauchy dans $\Rr$, donc converge, nous notons $x^\infty_k$ sa limite.
Nous avons construit une suite $x^\infty=(x^\infty_1,x^\infty_2,\ldots)$. Comme $(x^n)$ est de Cauchy alors
$$\forall \epsilon >0 \quad \exists N \in \Nn \quad \forall p,q \ge N \quad
\rho(x^p,x^q) < \epsilon,$$
Lorsque l'on fixe $p$ et que l'on fait tendre $q$ vers $+\infty$ on a
$$\forall \epsilon >0 \quad \exists N \in \Nn \quad \forall p \ge N \quad
\rho(x^p,x^\infty) < \epsilon,$$
donc $(x^n)$ converge vers $x^\infty$ pour la distance $\rho$.
Bien évidemment $x^\infty \in Y$ donc
$(x^n)$ converge vers $x^\infty \in Y$ pour $\rho$. Donc
$(Y,\rho)$ est un espace complet.
}
\indication{\begin{enumerate}
  \item Prendre la suite $(x^p)$ définie par $x^p= (1,1,\ldots,1,1,0,0,0,\ldots).$
($(x^p)_{p\in\Nn}$ est donc une suite de suite).

  \item Prendre $Y$ l'espace de toutes les suites.

  \item Considérer $x^p= (1,\frac 12,\ldots,\frac1p,0,0,\ldots)$.
\end{enumerate}}
\end{enumerate}
}
