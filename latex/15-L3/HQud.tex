\uuid{HQud}
\exo7id{1872}
\auteur{roussel}
\datecreate{2001-09-01}
\isIndication{false}
\isCorrection{false}
\chapitre{Espace vectoriel normé}
\sousChapitre{Espace vectoriel normé}

\contenu{
\texte{
Lorsqu'un espace vectoriel $E$ est en outre muni d'une multiplication,
 l'application
 $\nolinebreak {N:\ E \rightarrow \mathbb{R}}$  est dite norme
multiplicative si:
\begin{itemize}
\item $N$ est une norme,
\item pour tous $A$ et $B$ dans $E$, $N(A.B)\leq N(A).N(B)$.
\end{itemize}
Soit $E=M_n(\mathbb{R})$, l'espace vectoriel des matrices carr\'ees \`a $n$
lignes et $n$ colonnes. $A\in E$ se  note
$A=(a_{i,j})_{1\leq i,j\leq n}$
}
\begin{enumerate}
    \item \question{Montrer que $\displaystyle N_{\infty }(A)=
\max_{1\leq i\leq n}\{\  \sum_{j=1}^n|a_{i,j}|\ \}$ d\'efinit une norme
multiplicative sur $E$.}
    \item \question{Montrer que $\displaystyle N_{\infty }(A)=\max_{\{ X\in \mathbb{R}^n,\ \|{X}\|_\infty=1\} }
\{\  \| {A.X}\|_\infty \}$.}
    \item \question{Soit $A\in M_n(\mathbb{R})$ telle que $\displaystyle \forall \ 1\leq i \leq n,\ |a_{i,i}|>
\sum_{j=1,j\neq i}^n|a_{i,j}|$ et $D$ la matrice diagonale form\'ee avec les
\'el\'ements diagonaux de $A$. Soit aussi $F$ un vecteur de $\mathbb{R}^n$.
On consid\`ere la suite des $X^{(p)}\in \mathbb{R}^n$ d\'efinie pour $p\geq 0$ par:
$$
  \left\{
 \begin{array}{lll}
X^{(0)} &=X_0\in \mathbb{R}^n &\\
X^{(p+1)} &=(I-D^{-1}A)X^{(p)}+D^{-1}F &\mbox{ pour p}\geq 0
\end{array}
\right.
$$
Montrer qu'elle est convergente et calculer sa limite.}
\end{enumerate}
}
