\uuid{hUYN}
\exo7id{6245}
\auteur{queffelec}
\organisation{exo7}
\datecreate{2011-10-16}
\isIndication{false}
\isCorrection{false}
\chapitre{Continuité, uniforme continuité}
\sousChapitre{Continuité, uniforme continuité}

\contenu{
\texte{
Soit $d_1$ et $d_2$ deux distances sur un espace $X$. On considère les
quatre assertions suivantes :

   (i) Les métriques sont topologiquement équivalentes.

  (ii) Les métriques sont uniformément équivalentes.
 
  (iii) Les métriques sont Lipschitz-équivalentes (il existe $A$ et $B$
constantes telles que $A\, d_1\leq d_2\leq B\, d_1$).

  (iv) $(X,d_1)$ et $(X,d_2)$ sont simultanément complets.

  
Etablir les implications entre ces propriétés et donner des contre-exemples
lorsque les implications  n'ont pas lieu.
}
}
