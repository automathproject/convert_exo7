\uuid{mIIm}
\exo7id{2388}
\auteur{mayer}
\organisation{exo7}
\datecreate{2003-10-01}
\isIndication{true}
\isCorrection{true}
\chapitre{Connexité}
\sousChapitre{Connexité}

\contenu{
\texte{
Dans $\Rr^2$, soit $B_a$ l'ensemble $\{a\}\times ]0,1]$ si $a$
est rationnel et $B_a=\{a\}\times [-1,0]$ si $a$ est irrationnel.
Montrer que $B = \bigcup _{a\in \Rr } B_a$ est une partie
connexe de $\Rr^2$.
}
\indication{Définir $g : \Rr \longrightarrow \{0,1\}$ tel que $g(x)$ prend  la valeur
qu'a $f$ sur $B_x$.
Montrer pour chaque points de $\Rr\setminus \Qq$, $g$ est constante dans un voisinage 
de ce point,
puis faire la m\^eme chose pour un point de $\Qq$. Conclure.}
\reponse{
L'ensemble $B$ est connexe si
et seulement si toute fonction continue $f:B\to \{0,1\}$ est
constante. Soit alors $f:B\to \{0,1\}$ une fonction continue et
montrons qu'elle est constante.
Remarquons que la restriction de $f$ \`a tout ensemble $B_a$ est constante ($B_a$ est connexe).

On définit $g : \Rr \longrightarrow \{0,1\}$ tel que $g(x)$ prend  la valeur
qu'a $f$ sur $B_x$.
Nous allons montrer que $g$ est localement constante (on ne sait pas si $g$ est continue).

\begin{itemize}
\item Soit $a \notin \Qq$ alors on a $(a,0)\in B$, $f$ est une fonction continue et
$\{f(a,0)\}$ est un ouvert de $\{0,1\}$, donc $f^{-1}(\{ f(a,0) \})$ est un ouvert de $B$.
Donc il existe $\epsilon >0$ tel que si
$(x,y) \in (]a-\epsilon,a+\epsilon[\times ]-\epsilon,\epsilon[)\times B$ alors
$f(x,y)=f(a,0)$. Alors pour $x\in ]a-\epsilon,a+\epsilon[$ on a $g(x)=g(a)$ :
si $x\notin \Qq$ alors $g(x)=f(x,0)=f(a,0)=g(a)$; et si $x\in \Qq$ alors
$g(x)= f(x,\frac \epsilon 2) = f(a,0)=g(a)$.
Donc $g$ est localement constante au voisinage des point irrationnels.

  \item Si $a\in \Qq$ et soit $b\in ]0,1]$ alors $f$ est continue en $(a,b)$ donc il existe
$\epsilon >0$ tel que pour tout $x\in ]a-\epsilon,a+\epsilon[\cap \Qq$,
$g(x)=f(x,b)=f(a,b)=g(a)$.
Si maintenant $x\in ]a-\epsilon,a+\epsilon[\cap (\Rr \setminus \Qq)$, on prend une 
suite $(x_n)$ de rationnels qui tendent vers $x$. 
Comme $f$ est continue alors
$g(a)=g(x_n)=f(x_n,b)$ tend vers $f(x,b)=g(x)$. Donc $g(a)=g(x)$.
Nous avons montrer que $g$ est localement constante au voisinage des point rationnels.

  \item Bilan : $g$ est localement constante sur $\Rr$.
\end{itemize}

Comme $\Rr$ est connexe, alors $g$ est constante sur $\Rr$. Donc
$f$ est constante sur $\Rr$. Ce qu'il fallait démontrer.
}
}
