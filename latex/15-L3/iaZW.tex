\uuid{iaZW}
\exo7id{2392}
\auteur{queffelec}
\datecreate{2003-10-01}
\isIndication{true}
\isCorrection{true}
\chapitre{Théorème de Baire}
\sousChapitre{Théorème de Baire}

\contenu{
\texte{
\`A l'aide du th\'eor\`eme de Baire, montrer qu'un ferm\'e d\'enombrable non vide $X$ de
$\Rr$ a au moins un point isol\'e.
\emph{Indication :} on pourra considérer $\omega_x = X \setminus \{x\}$.

Que peut-on dire de l'ensemble de Cantor ?
}
\indication{Raisonner par l'absurde et montrer que $\omega_x$ est un ouvert dense.}
\reponse{
Par l'absurde supposons que $X$ n'a aucun point isolé.
Comme $\{x\}$ est un fermé alors $\omega_x = X \setminus \{x\}$
est un ouvert (de $X$). De plus comme le point $x$ n'est pas isolé
alors $\omega_x$ est dense dans $X$.

Maintenant on peut appliquer le théorème de Baire à $X$ qui est un fermé de l'espace complet $\Rr$. Donc une intersection dénombrable d'ouverts denses dans $X$ est encore dense.
Mais ici nous obtenons une contradiction car les $\omega_x$ sont des ouverts denses, $X$ est dénombrable mais $$\bigcap_{x\in X} \omega_x = \varnothing.$$
Et l'ensemble vide n'est pas dense dans $X$ !!
Pour l'ensemble de Cantor aucun point n'est isolé, donc par la question précédente
l'ensemble de Cantor n'est pas dénombrable.
}
}
