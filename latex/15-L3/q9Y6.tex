\uuid{q9Y6}
\exo7id{2360}
\auteur{queffelec}
\datecreate{2003-10-01}
\isIndication{true}
\isCorrection{true}
\chapitre{Continuité, uniforme continuité}
\sousChapitre{Continuité, uniforme continuité}

\contenu{
\texte{
Soit
$f$ une fonction uniform\'ement continue sur
$\Rr$ telle que
$\int_0^\infty f(t) dt$ con\-verge.
Montrer que $f$ tend vers $0$ quand $x\to+\infty$. Retrouver ainsi le fait que
la fonction $\sin(x^2)$ n'est pas uniform\'ement continue.
}
\indication{\begin{enumerate}
  \item Par l'absurde, considérer $I(x) = \int_0^x f$. Trouver une suite $(p_n)$
telle que $(I(p_n))$ ne soit pas une suite de Cauchy.
  \item Pour montrer que cette intégrale converge utiliser le changement de variable $u=t^2$ puis faire une intégration par partie.
   \end{enumerate}}
\reponse{
Supposons que $f$ ne tende pas vers $0$. Soit $\epsilon >0$ fixé.
Pour tout $n\geq 0$, il existe $x_n \geq n$ tel que $|f(x_n)|>\epsilon$.
Sans perte de généralité nous supposons $f(x_n)>\epsilon$.
Appliquons l'uniforme continuité : soit $\epsilon' = \frac \epsilon 2$,
Il existe $\eta$ tel que pour  $|x_n-y|\le \eta$ on ait $|f(x_ n)-f(y)| < \epsilon'$. Donc pour un tel $y$, $f(y) > \frac \epsilon 2 >0$.
Donc $f$ est strictement positive sur $[x_n-\eta,x_n+\eta]$.
Notons alors $(p_n)$ définie par 
$p_{2n} = x_{n}-\eta$, $p_{2n+1} = x_{n}+\eta$.
Soit $I(x) = \int_0^x f$.
Alors $I(p_{2n+1})- I(p_{2n}) = \int_{x_n-\eta}^{x_n+\eta} f(t) dt \geq \frac \epsilon 2 \cdot 2\eta = \epsilon \eta$.
Donc la suite $(I(p_n))$ n'est pas de une suite de Cauchy, donc ne converge pas, donc la fonction $x\mapsto I(x)$ ne converge pas non plus, et donc 
$\int_0^\infty f(t) dt$ diverge.
Par le changement de variable $u=t^2$ puis une intégration par partie, on montre que l'intégrale $\int_0^\infty \sin(t^2) dt$ converge, mais comme $f(x)= \sin(x^2)$ ne tend pas vers $0$ alors
$f$ n'est pas uniformément continue sur $\Rr$.
}
}
