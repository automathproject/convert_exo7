\uuid{dOAU}
\exo7id{5959}
\auteur{tumpach}
\organisation{exo7}
\datecreate{2010-11-11}
\isIndication{false}
\isCorrection{true}
\chapitre{Intégrales multiples, théorème de Fubini}
\sousChapitre{Intégrales multiples, théorème de Fubini}

\contenu{
\texte{
Soient $f\in L^1(\mathbb{R}^n)$ et $g\in L^p(\mathbb{R}^n)$ avec
$1\leq p\leq +\infty$, o\`u $\mathbb{R}^n$ est muni de la mesure
de Lebesgue. Montrer que, pour presque tout $x\in \mathbb{R}^n$,
la fonction $y\mapsto f(x-y)\, g(y)$ est int\'egrable sur
$\mathbb{R}^n$ et que le \emph{ produit de convolution} de $f$ et
$g$ d\'efini par
$$
f*g(x) = \int_{\mathbb{R}^n} f(x-y) \,g(y)\,dy
$$
v\'erifie $f*g(x) = g*f(x)$ et
$$
\|f*g\|_{p} \leq \|f\|_{1}\,\|g\|_{p}.
$$
}
\reponse{
Pour $p = + \infty$, c'est clair.
Supposons que $p = 1$ et posons $F(x, y) =
f(x-y)\,g(y)$. Pour presque tout $y\in\mathbb{R}^n$, on a~:
$$
\int_{\mathbb{R}^n} |F(x, y)|\,dx = |g(y)|
\int_{\mathbb{R}^n}|f(x- y)|\,dx = |g(y)| \cdot\|f\|_1,
$$
et
$$
\int_{\mathbb{R}^n} dy \int_{\mathbb{R}^n}|F(x, y)|\,dx =
\|f\|_1\,\|g\|_1.
$$
D'apr\`es le th\'eor\`eme de Tonelli, $F\in
L^{1}\left(\mathbb{R}^n\times\mathbb{R}^n\right)$. D'apr\`es le
th\'eor\`eme de Fubini, on a
$$
\int_{\mathbb{R}^n} |F(x, y)|\,dy <+\infty \quad\quad\text{pour
presque tout }~x\in\mathbb{R}^n,
$$
et
$$
\int_{\mathbb{R}^n} dx \int_{\mathbb{R}^n}|F(x, y)|\,dy
\leq\|f\|_1\,\|g\|_1.
$$
Ainsi, $$\|f*g\|_{1} = \int_{\mathbb{R}^n} \,dx |f*g(x)|
=\int_{\mathbb{R}^n} \,dx \left|\int_{\mathbb{R}^{n}} F(x,
y)\,dy\right| \leq\|f\|_1\,\|g\|_1.
$$
Supposons que $1< p<+\infty$. Utilisons le cas
 pr\'ec\'edent, en faisant jouer ici à $g^p$ le rôle alors joué par $g$. Alors pour presque tout $x\in\mathbb{R}^n$ fix\'e, la
 fonction $y \mapsto |f(x-y)| |g(y)|^p$ est int\'egrable sur
 $\mathbb{R}^n$, i.e. la fonction $y \mapsto |f(x-y)|^{\frac{1}{p}}
 |g(y)|$ appartient \`a $L^p(\mathbb{R}^n)$. Soit $p'$ tel que
 $\frac{1}{p} + \frac{1}{p'} = 1$. La fonction $y\mapsto
 |f(x-y)|^{\frac{1}{p'}}$ appartient \`a $L^{p'}(\mathbb{R}^n)$
 car $f\in L^1(\mathbb{R}^n)$ et la mesure de Lebesgue est invariante par translation.
 D'apr\`es l'in\'egalit\'e de
 H\"older,
$$
|f(x-y)| \,|g(y)| = |f(x-y)|^{\frac{1}{p}}
\,|g(y)|\cdot|f(x-y)|^{\frac{1}{p'}}\in L^1(\mathbb{R}^n)
$$
et
$$
\int_{\mathbb{R}^n} |f(x-y)| \,|g(y)| \leq \left(
\int_{\mathbb{R}^n} |f(x-y)|
\,|g(y)|^p\right)^{\frac{1}{p}}\cdot\|f\|_1^{\frac{1}{p'}},
$$
ainsi
$$
|(f*g)(x)|^{p} \leq (|f|*|g|^p)(x)\cdot\|f\|_1^{\frac{p}{p'}}.
$$
D'apr\`es le cas pr\'ec\'edent, on voit que
$$
f*g\in L^p(\mathbb{R}^n) \quad \text{et}\quad \|f*g\|_p^{p} \leq
\|f\|_1\|g\|_{p}^{p}\cdot\|f\|_1^{\frac{p}{p'}},
$$
c'est-\`a-dire
$$
\|f*g\|_p \leq \|f\|_1\cdot\|g\|_p.
$$
}
}
