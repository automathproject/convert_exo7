\uuid{moE7}
\exo7id{5730}
\auteur{rouget}
\datecreate{2010-10-16}
\isIndication{false}
\isCorrection{true}
\chapitre{Suite et série de fonctions}
\sousChapitre{Continuité, dérivabilité}

\contenu{
\texte{
Soit $f(x) =\sum_{n=1}^{+\infty}\frac{x^n\sin(nx)}{n}$.
}
\begin{enumerate}
    \item \question{Montrer que $f$ est de classe $C^1$ sur $]-1,1[$.}
\reponse{Pour $x\in]-1,1[$ et $n$ entier naturel non nul, posons $f_n(x)=\frac{x^n\sin(nx)}{n}$. 

Soit $x\in]-1,1[$. Pour $n$ entier naturel non nul, $|f_n(x)|\leqslant|x|^n$. Or, la série géométrique de terme général $|x|^n$, $n\geqslant1$, est convergente et donc la série numérique de terme général $f_n(x)$ est absolument convergente et en particulier convergente. On en déduit que $f(x)$ existe.

\begin{center}
\shadowbox{
$f$ est définie sur $]-1,1[$.
}
\end{center}

Soit $a\in]0,1[$. Chaque $f_n$, $n\geqslant1$, est de classe $C^1$ sur $[-a,a]$ et pour $x\in[-a,a]$,

\begin{center}
$f_n'(x) =x^{n-1}\sin(nx)+x^n\cos(nx)$.
\end{center}

Pour $x\in[-a,a]$ et $n\in\Nn^*$, 

\begin{center}
$|f_n'(x)|\leqslant a^{n-1}+a^n\leqslant2a^{n-1}$.
\end{center} 

Puisque la série numérique de terme général $2a^{n-1}$, $n\geqslant1$, converge, la série de fonctions de terme général $f_n'$, $n\geqslant1$, est normalement et donc uniformément sur $[-a,a]$.

En résumé,

\textbullet~la série de fonctions de terme général $f_n$, $n\geqslant1$, converge simplement vers $f$ sur $[-a,a]$,

\textbullet~chaque fonction $f_n$, $n\geqslant1$, est de classe $C^1$ sur $[-a,a]$,

\textbullet~la série de fonctions de terme général $f_n'$ converge uniformément sur $[-a,a]$.

D'après un corollaire du théorème de dérivation terme à terme, $f$ est de classe $C^1$ sur $[-a,a]$ pour tout réel $a$ de $]0,1[$  et donc sur $]-1,1[$ et sa dérivée s'obtient par dérivation terme à terme. 

\begin{center}
\shadowbox{
$f$ est de classe $C^1$ sur $]-1,1[$ et $\forall x\in]-1,1[$, $f'(x)=\sum_{n=1}^{+\infty}(x^{n-1}\sin(nx)+x^n\cos(nx))$.
}
\end{center}}
    \item \question{Calculer $f'(x)$ et en déduire que $f(x) =\Arctan\left(\frac{x\sin x}{1-x\cos x}\right)$.}
\reponse{Ainsi, pour $x\in]-1,1[$

\begin{align*}\ensuremath
f'(x)&=\sum_{n=1}^{+\infty}(x^{n-1}\sin(nx)+x^n\cos(nx))=\text{Im}\left(\sum_{n=1}^{+\infty}x^{n-1}e^{inx}\right)+\text{Re}\left(\sum_{n=1}^{+\infty}x^{n}e^{inx}\right)\\
 &=\text{Im}\left(\frac{e^{ix}}{1-xe^{ix}}\right)+\text{Re}\left(\frac{xe^{ix}}{1-xe^{ix}}\right)=\text{Im}\left(\frac{e^{ix}(1-xe^{-ix})}{x^2-2x\cos x+1}\right) +\text{Re}\left(\frac{xe^{ix}(1-xe^{-ix})}{x^2-2x\cos x+1}\right)\\
  &=\frac{\sin x+x\cos x-x^2}{x^2-2x\cos x+1}  .
\end{align*}

Mais, pour $x\in]-1,1[$,

\begin{center}
$\left(\frac{x\sin x}{1-x\cos x}\right)'=\frac{(\sin x+x\cos x)(1-x\cos x)-x\sin x(-\cos x+x\sin x)}{(1-x\cos x)^2}=\frac{\sin x+x\cos x-x^2}{(1-x\cos x)^2}$.
\end{center}

et donc

\begin{align*}\ensuremath
\left(\Arctan\left(\frac{x\sin x}{1-x\cos x}\right)\right)'&=\frac{\sin x+x\cos x-x^2}{(1-x\cos x)^2}\times\frac{1}{1+\left(\frac{x\sin x}{1-x\cos x}\right)^2}=\frac{\sin x+x\cos x-x^2}{(1-x\cos x)^2+x^2\sin^2x}\\
 &=\frac{\sin x+x\cos x-x^2}{x^2-2x\cos x+1}= f'(x).
\end{align*}

Finalement, pour $x\in]-1,1[$,

\begin{center}
$f(x)=f(0)+\int_{0}^{x}f'(t)\;dt=0+\Arctan\left(\frac{x\sin x}{1-x\cos x}\right)-\Arctan(0)=\Arctan\left(\frac{x\sin x}{1-x\cos x}\right)$.
\end{center}

\begin{center}
\shadowbox{
$\forall x\in]-1,1[$, $\sum_{n=1}^{+\infty}\frac{x^n\sin(nx)}{n}=\Arctan\left(\frac{x\sin x}{1-x\cos x}\right)$.
}
\end{center}}
\end{enumerate}
}
