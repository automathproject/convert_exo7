\uuid{xvmB}
\exo7id{4745}
\auteur{quercia}
\organisation{exo7}
\datecreate{2010-03-16}
\isIndication{false}
\isCorrection{true}
\chapitre{Topologie}
\sousChapitre{Topologie des espaces vectoriels normés}

\contenu{
\texte{
Soit $f$ d{\'e}finie par $f(x)=\frac{x}{\max(1,\|x\|)}$. Montrer que $f$ est $2$-lipschitzienne.
}
\reponse{
Pour $\|x\| \le 1$ et $\|y\| \le 1$ on a $\|f(x)-f(y)\| = \|x-y\|$.

Pour $\|x\| \le 1< \|y\|$ on a
$\|f(x)-f(y)\| \le \|x-y\| + \Bigr\|y - \frac y{\|y\|}\Bigr\|
= \|x-y\| + \|y\| - 1 \le \|x-y\| + \|y\| - \|x\| \le 2\|x-y\|$.

Pour $1 < \|x\| \le \|y\|$ on a $\|f(x) - f(y)\| \le
\Bigl\|\frac x{\|x\|} - \frac y{\|x\|}\Bigr\| + 
\Bigl\|\frac y{\|x\|} - \frac y{\|y\|}\Bigr\| \le
\frac{\|x-y\| + \|y\| - \|x\|}{\|x\|} \le \frac{2\|x-y\|}{\|x\|}$.


Remarque~: dans le cas o{\`u} la norme est euclidienne, $f(x)$ est le projet{\'e}
de $x$ sur la boule unit{\'e}, c'est-{\`a}-dire le point de la boule unit{\'e} le plus
proche de~$x$. Dans ce cas, $f$ est $1$-lipschitzienne. Dans le cas d'une norme
non euclidienne on peut avoir $\|f(x)-f(y)\| > \|x-y\|$, par exemple
avec $x=(1,1)$ et $y=(\frac12,\frac32)$ dans $\R^2$ pour $\|\ \|_\infty$.
}
}
