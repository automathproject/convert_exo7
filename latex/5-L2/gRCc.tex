\uuid{gRCc}
\exo7id{4147}
\auteur{quercia}
\organisation{exo7}
\datecreate{2010-03-11}
\isIndication{false}
\isCorrection{true}
\chapitre{Fonction de plusieurs variables}
\sousChapitre{Dérivée partielle}

\contenu{
\texte{
\smallskip
Soit $f(x,y) = \frac{x^3y}{x^2+y^2}$ si $(x,y)\ne 0$ et $f(0,0)=0$.
}
\begin{enumerate}
    \item \question{\'Etudier la continuité de~$f$ et de ses dérivées partielles premières sur~$\R^2$.}
\reponse{$f$ est homogène de degré~$2$, $\frac{\partial f}{\partial x}$ et $\frac{\partial f}{\partial y}$ sont homogènes de degré~$1$, donc
    ces trois fonctions tendent vers~$0$ en~$(0,0)$. Ainsi $f$ est de classe~$\mathcal{C}^1$.}
    \item \question{Montrer que~$\frac{\partial^2 f}{\partial x \partial y}(0,0)\ne\frac{\partial^2 f}{\partial y \partial x}(0,0)$.}
\reponse{$\frac{\partial^2 f}{\partial x \partial y}(0,0)=0$, $\frac{\partial^2 f}{\partial y \partial x}(0,0)=1$.}
\end{enumerate}
}
