\uuid{1dXV}
\exo7id{4598}
\auteur{quercia}
\organisation{exo7}
\datecreate{2010-03-14}
\isIndication{false}
\isCorrection{true}
\chapitre{Série entière}
\sousChapitre{Equations différentielles}

\contenu{
\texte{

}
\begin{enumerate}
    \item \question{En utilisant la relation : $\tan' = 1 + \tan^2$, exprimer $\tan^{(n)}$
    en fonction de $\tan,\dots,\tan{(n-1)}$.
    En déduire que~: $\forall\ x \in {[0,\pi/2[}$, $\tan^{(n)}(x) \ge 0$.}
    \item \question{Montrer que la série de Taylor de $\tan$ en 0 converge sur $]-\pi/2,\pi/2[$.}
    \item \question{Soit $f$ la somme de la série précédente. Montrer que $f' = 1+f^2$ et en
    déduire que $f = \tan$.}
    \item \question{Prouver que le rayon de convergence est exactement $\pi/2$.}
\reponse{
$\tan^{(n)} =
             \sum_{k=0}^{n-1} C_{n-1}^k \tan^{(k)}\tan^{(n-1-k)}$.
Pour $0 \le x < \pi/2$ la série est à termes positifs et les
             sommes partielles sont majorées par $\tan x$.
             Pour $-\pi/2 < x \le 0$, il y a convergence absolue.
Si $R > \pi/2$, $f$ aurait une limite finie en $\pi/2$.
}
\end{enumerate}
}
