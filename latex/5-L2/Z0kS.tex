\uuid{Z0kS}
\exo7id{4589}
\auteur{quercia}
\datecreate{2010-03-14}
\isIndication{false}
\isCorrection{true}
\chapitre{Série entière}
\sousChapitre{Développement en série entière}

\contenu{
\texte{
Soit $q \in {]-1,1[}$ et $f(x) = \prod_{n=1}^\infty (1-q^nx)$.
}
\begin{enumerate}
    \item \question{Montrer que $f(x)$ existe pour tout $x \in \R$ et que $f$ est développable
    en série entière au voisinage de 0.
    On admettra que si une fonction $g$ est DSE alors $e^g$ l'est.}
\reponse{Pour $|x| < \frac1q$ : $\ln f(x) = \sum_{n=1}^\infty \ln(1-q^nx)
             = -\sum_{n=1}^\infty\sum_{k=1}^\infty \frac{q^{kn}x^k}k
             = -\sum_{k=1}^\infty \frac{q^kx^k}{k(1-q^k)}$,\par
             $f = e^{\ln f}$ est DSE par composition.}
    \item \question{A l'aide de la relation : $f(x) = (1-qx)f(qx)$, calculer les coefficients
    du développement de $f$ et le rayon de convergence.}
\reponse{$a_n = \frac{q^{n(n+1)/2}}{(q-1)\dots(q^n-1)}$, $R = \infty$.}
\end{enumerate}
}
