\uuid{p7Qw}
\exo7id{4380}
\auteur{quercia}
\datecreate{2010-03-12}
\isIndication{false}
\isCorrection{false}
\chapitre{Intégration}
\sousChapitre{Intégrale de Riemann dépendant d'un paramètre}

\contenu{
\texte{
Soit $P\in \C[X]$ de degré $n\ge 1$.
Le but de cet exercice est de prouver que $P$ admet une racine dans~$\C$.
On suppose au contraire que $P$ ne s'annule pas et on considère pour
$r \ge 0$, $\theta\in{[0,2\pi]}$~:
$f(r,\theta) = \frac{r^ne^{in\theta}}{P(re^{i\theta})}$
et $F(r) =  \int_{\theta=0}^{2\pi} f(r,\theta)\, d\theta$.
}
\begin{enumerate}
    \item \question{Montrer que $F$ est de classe $\mathcal{C}^1$ sur $[0,+\infty[$.}
    \item \question{Vérifier que $ir \frac{\partial f}{\partial r} = \frac{\partial f}{\partial \theta}$. En déduire que $F$ est constante.}
    \item \question{Obtenir une contradiction.}
\end{enumerate}
}
