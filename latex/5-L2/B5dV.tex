\uuid{B5dV}
\exo7id{4187}
\auteur{quercia}
\organisation{exo7}
\datecreate{2010-03-11}
\isIndication{false}
\isCorrection{true}
\chapitre{Fonction de plusieurs variables}
\sousChapitre{Dérivée partielle}

\contenu{
\texte{
Soient $f$ et $g$ deux fonctions de classe $\mathcal{C}^1$ sur $\R$ à valeurs
dans $\R$ vérifiant~: $\forall\ x\in\R,\ f'(x)\ge 1$ et $|g'(x)| < 1$.
Soit $\varphi$ définie sur $\R^2$ par $\varphi(x,y) = (f(x)+g(y), f(y) + g(x))$.
}
\begin{enumerate}
    \item \question{Montrer que $\varphi$ est un $\mathcal{C}^1$-difféomorphisme de $\R^2$ sur $\varphi(\R^2)$.}
\reponse{$\det(J_\varphi(x,y)) =  f'(x)f'(y) - g'(x)g'(y) > 0$ donc le théorème
    d'inversion locale s'applique, il suffit de vérifier l'injectivité de~$\varphi$.
    Si $\varphi(x,y) = \varphi(u,v)$ alors~:
    \begin{align*}&|x-u| \le |f(x)-f(u)| = |g(v)-g(y)| \le |v-y|\\
               &|v-y| \le |f(v)-f(y)| = |g(x)-g(u)| \le |x-u|\\ 
\end{align*}
    d'où $|x-u| \le |x-u|$ et il y a inégalité stricte si $v\ne y$ ce qui est
    absurde donc $v=y$ et de même $u=x$.}
    \item \question{On suppose qu'il existe $k\in{]0,1[}$ tel que $\forall\ x\in\R$, $|g'(x)| < k$~;
    montrer que $\varphi(\R^2) = \R^2$.}
\reponse{On a $\varphi(x,y) = (u,v)$ si et seulement si $x = f^{-1}(u-g(y))$ et
    $y = f^{-1}(v - g(f^{-1}(u-g(y)))) = h(y)$. $h$ est $k^2$-lipschitzienne
    donc le théorème du point fixe s'applique.}
\end{enumerate}
}
