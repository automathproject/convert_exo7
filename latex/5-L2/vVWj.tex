\uuid{vVWj}
\exo7id{4850}
\auteur{quercia}
\datecreate{2010-03-16}
\isIndication{false}
\isCorrection{true}
\chapitre{Topologie}
\sousChapitre{Espaces complets}

\contenu{
\texte{
Soit $E$ un espace vectoriel norm{\'e} complet et $(F_n)$
une suite de ferm{\'e}s de $E$ d'int{\'e}rieurs vides.
On pose $\smash{F = \bigcup_n F_n}$. Montrer que $\mathring F = \varnothing$.
}
\reponse{
Soit $a \in \mathring F$ et $B(a,r) \subset \bigcup_n F_n$ :
$B \setminus F_1$ est un ouvert non vide donc contient une boule
$B_1(a_1,r_1)$.
De m{\^e}me, $B_1 \setminus F_2$ contient une boule $B_2(a_2,r_2)$ etc.
On peut imposer $r_n \xrightarrow[n\to\infty]{} 0$, donc il existe $c \in \bigcap_n B_n$,
c.a.d. $c\in B$ mais pour tout $n$, $c \notin F_n$. Contradiction.
}
}
