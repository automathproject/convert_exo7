\uuid{s9lj}
\exo7id{4632}
\auteur{quercia}
\organisation{exo7}
\datecreate{2010-03-14}
\isIndication{false}
\isCorrection{true}
\chapitre{Série de Fourier}
\sousChapitre{Calcul de coefficients}

\contenu{
\texte{

}
\begin{enumerate}
    \item \question{Soit~$a\in\R$. Développer en série de Fourier la fonction $2\pi$-périodique
    valant~$e^{ax}$ sur~$]0,2\pi]$.

Soit~$a\in\R$. On pose~$I(a) =  \int_{u=0}^{+\infty}\frac{e^{-u}}{1-e^{-u}}\sin(au)\,d u$.}
\reponse{Si $a\ne 0$~: $S_f(x) = \sum_{n=-\infty}^{+\infty} \frac{e^{2\pi a}-1}{\strut2\pi(a-in)}e^{inx}
    = \frac{e^{2\pi a}-1}{2\pi a} + \sum_{n=1}^{+\infty}\frac{e^{2\pi a}-1}{\pi(a^2+n^2)}(a\cos(nx) - n\sin(nx))$.}
    \item \question{Exprimer $I(a)$ sous forme d'une série sans intégrale.}
\reponse{On peut supposer $a > 0$ car $I(-a) = -I(a)$ et $I(0) = 0$.
    On envisage d'intégrer terme à terme la
    relation~: $$\frac{e^{-u}}{1-e^{-u}}\sin(au) = \sum_{n=1}^{\infty}e^{-nu}\sin(au).$$
    On coupe l'intégrale $\int_0^{+\infty}$ en $\int_0^{\pi/a}+\int_{\pi/a}^{+\infty}$~:
    sur~$[0,\pi/a]$ le sinus est positif et le théorème de convergence monotone
    s'applique. Sur~$[\pi/a,+\infty[$ le théorème d'intégration terme
    à terme s'applique (série des normes~$1$ convergente) car
    $\int_{\pi/a}^{+\infty}|e^{-nu}\sin(au)|\,d u \le \int_{\pi/a}^{+\infty}e^{-nu}\,d u 
    = e^{-n\pi/a}/n$. Ainsi,
    $$I(a) = \sum_{n=1}^\infty  \int_{u=0}^{+\infty}e^{-nu}\sin(au)\,d u
           = \sum_{n=1}^\infty  \frac{a}{n^2+a^2}.$$}
    \item \question{Calculer $ \int_{u=0}^{+\infty}e^{-u}\sin(au)\,d u$.}
\reponse{Déjà fait, $ \int_{u=0}^{+\infty}e^{-u}\sin(au)\,d u =  \frac{a}{a^2+1}$.
    Il doit y avoir une autre méthode pour la question précédente ?!}
    \item \question{Conclure.}
\reponse{En comparant avec~{\bf 1)} pour~$x=0$ on obtient~:
    $I(a) = \frac\pi{\strut2}\frac{e^{2a\pi}+1}{e^{2a\pi}-1}-\frac1{\strut2a}$ pour $a>0$.}
\end{enumerate}
}
