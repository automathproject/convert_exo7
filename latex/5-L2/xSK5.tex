\uuid{xSK5}
\exo7id{4341}
\auteur{quercia}
\organisation{exo7}
\datecreate{2010-03-12}
\isIndication{false}
\isCorrection{false}
\chapitre{Intégration}
\sousChapitre{Intégrale de Riemann dépendant d'un paramètre}

\contenu{
\texte{
Soit $I(\alpha) =  \int_{x=0}^{+\infty} \frac{x^{\alpha-1}}{1+x}\,d x$.
Montrer que $I(\alpha)$ existe et définit une fonction de classe
$\mathcal{C}^1$ sur $]0,1[$. \'Ecrire $I(\alpha)$ comme somme d'une série.
}
}
