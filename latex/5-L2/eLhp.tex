\uuid{eLhp}
\exo7id{4667}
\auteur{quercia}
\organisation{exo7}
\datecreate{2010-03-14}
\isIndication{false}
\isCorrection{true}
\chapitre{Série de Fourier}
\sousChapitre{Autre}

\contenu{
\texte{
Soit $f$ à valeurs réelles, de classe $\mathcal{C}^{2}$, $2\pi$-p{\'e}riodique, de moyenne nulle. Montrer que $g=f+f''$ s'annule au moins quatre fois sur $[0,2\pi[$.
}
\reponse{
On a $c_0(g) = c_1(g) = c_{-1}(g) = 0$, donc $g$ est orthogonale à tout polynôme trigonométrique de degré
inférieur ou égal à~$1$. Si $g$ est de signe constant sur $]0,2\pi[$, on contredit $c_0(g) = 0$. Donc
$g$ a au moins une racine $a\in{]0,2\pi[}$. Si $g$ n'a pas d'autre racine
dans $]0,2\pi[$ alors $g$ est de signes constants opposés sur $]0,a[$ et $]a,2\pi[$.
Mais alors $g(t)(\cos(t-a/2)-\cos(a/2))$ est de signe constant sur la réunion de
ces intervalles, c'est absurde. Donc $g$ a une deuxième racine dans $]0,2\pi[$,
par exemple $b\in{]a,2\pi[}$. Si $g$ n'a pas d'autre racine sur $]0,2\pi[$
alors $g$ est de signes constants sur $]0,a[$, $]a,b[$ et $]b,2\pi[$ et les
signes alternent. On obtient une nouvelle contradiction car alors
$g(t)(\cos(t-(a+b)/2)-\cos((b-a)/2))$ est de signe constant sur la réunion de
ces intervalles. Ainsi $g$ admet une troisième racine, par exemple $c\in{]b,2\pi[}$.
Enfin, si l'on suppose que $g$ n'a pas d'autre racine sur $]0,2\pi[$ alors on
a $g(t) > 0$ sur $]0,a[$ et $]b,c[$ et $g(t) < 0$ sur $]a,b[$ et $]c,2\pi[$ ou
l'inverse. Dans les deux cas, on en déduit que $g(0) = g(2\pi) = 0$.
}
}
