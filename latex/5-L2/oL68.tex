\uuid{oL68}
\exo7id{4548}
\auteur{quercia}
\datecreate{2010-03-14}
\isIndication{false}
\isCorrection{true}
\chapitre{Suite et série de fonctions}
\sousChapitre{Autre}

\contenu{
\texte{
Soit $(p_n)$ une suite d'entiers naturels, strictement croissante
et telle que $p_n/n\to\infty$  lorsque $n\to\infty$. On pose pour $x\in{]-1,1[}$~:
$f(x)=\sum_{n=0}^\infty x^{p_n}$. Montrer que $(1-x)f(x)\to 0$ lorsque $x\to1^-$.
}
\reponse{
Pour $k$ fixé et $x\in{[0,1[}$ on a
$0\le f(x) \le\text{polynôme}(x) + \sum_{n=0}^\infty x^k = \text{polynôme}(x) + \frac1{1-x^k}$
et $\frac1{1-x^k}\sim\frac1{k(1-x)}$ au voisinage de~$1$ donc
$0\le f(x) \le \frac2{k(1-x)}$ pour $x$ suffisament proche de~$1$.
}
}
