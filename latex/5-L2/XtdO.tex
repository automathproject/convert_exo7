\uuid{XtdO}
\exo7id{4177}
\auteur{quercia}
\organisation{exo7}
\datecreate{2010-03-11}
\isIndication{false}
\isCorrection{false}
\chapitre{Fonction de plusieurs variables}
\sousChapitre{Dérivée partielle}

\contenu{
\texte{
Soit $E$ un espace vectoriel euclidien et $f : E \to E$ de classe
$\mathcal{C}^1$.
}
\begin{enumerate}
    \item \question{Montrer que $f$ est une application affine si et seulement
si sa différentielle est constante (c'est-à-dire $d f_x = d f_y$
pour tous $x,y$, égalité dans $\mathcal{L}(E)$).}
    \item \question{Soit $X$ un ensemble non vide quelconque et
$\varphi : {X^3}\to \R$ une application vérifiant~:
$$\forall\ x,y,z\in X,\ \varphi(x,y,z) = \varphi(y,x,z) = -\varphi(z,y,x).$$
Montrer que $\varphi=0$ (lemme des tresses).}
    \item \question{On suppose $f$ de classe $\mathcal{C}^2$. Montrer que $f$ est une isométrie
de~$E$ pour la distance euclidienne si et seulement si, pour tout
$x\in E$, $d f_x$ est une application orthogonale.}
\end{enumerate}
}
