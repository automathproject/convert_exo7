\uuid{RCXo}
\exo7id{4325}
\auteur{quercia}
\datecreate{2010-03-12}
\isIndication{false}
\isCorrection{true}
\chapitre{Intégration}
\sousChapitre{Intégrale de Riemann dépendant d'un paramètre}

\contenu{
\texte{

}
\begin{enumerate}
    \item \question{Soit $f : {[0,1]} \to \R$ continue.
    Montrer que $ \int_{t=0}^1 f(t^n)\,d t \to f(0)$  lorsque $n\to\infty$.}
\reponse{Couper en $ \int_{t=0}^{1-\varepsilon} +  \int_{t=1-\varepsilon}^1$}
    \item \question{Chercher un équivalent pour $n\to\infty$ de $ \int_{t=0}^1 \frac{t^n\,d t}{1+t^n}$.}
\reponse{$=\left[\frac{t\ln(1+t^n)}n\right]_{t=0}^1 - \frac1n \int_{t=0}^1 \ln(1+t^n)\,d t \sim \frac{\ln 2}n$.}
    \item \question{Chercher un équivalent pour $n\to\infty$ de $-1 +  \int_{t=0}^1 \sqrt{1+t^n}\,d t$.}
\reponse{$\frac1n \int_{t=0}^1  \frac{d t}{\sqrt{1+t}+1} = \frac{2\sqrt2-2+2\ln(2\sqrt2-2)}n$.}
\end{enumerate}
}
