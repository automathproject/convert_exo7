\uuid{rmMF}
\exo7id{4194}
\auteur{quercia}
\organisation{exo7}
\datecreate{2010-03-11}
\isIndication{false}
\isCorrection{true}
\chapitre{Fonction de plusieurs variables}
\sousChapitre{Extremums locaux}

\contenu{
\texte{
On considère un vrai triangle $ABC$ et $f$ la fonction définie par~:
$f(M) = d(M,AB)\times d(M,AC)\times d(M,BC)$.
Montrer que $f$ admet un maximum à l'intérieur du triangle~$ABC$,
et caractériser géométriquement le point~$M_0$ où~$f$ est maximale.
}
\reponse{
Existence d'un maximum par compacité.
Soient $x,y$ les coordonnées d'un point $M$ dans un repère orthonormé du plan et
$(u,v,w)$ les coordonnées barycentriques de $M$ par rapport à $A,B,C$ (avec $u+v+w=1$).
$u,v,w$ sont des fonctions affines de $x,y,z$ et $(AB)$ a pour équation
barycentrique $w=0$ d'où $d(M,AB) = \alpha |w|$ pour un certain réel $\alpha>0$.
De même pour $d(M,AC)$ et $d(M,BC)$ et $f(M) = \alpha\beta\gamma|u||v||w|$.
Lorsque $M$ varie dans le triangle, $(u,v,w)$ décrit tous les triplets de réels positifs de
somme~$1$ et on cherche le maximum du produit $uvw$, il est atteint quand $u,v,w$
sont égaux, c'est-à-dire au centre de gravité du triangle.
}
}
