\uuid{xjMD}
\exo7id{4099}
\auteur{quercia}
\organisation{exo7}
\datecreate{2010-03-11}
\isIndication{false}
\isCorrection{true}
\chapitre{Equation différentielle}
\sousChapitre{Equations différentielles linéaires}

\contenu{
\texte{
Soit $y : \R \to \R$ une solution non identiquement nulle de
$y'' + e^t y = 0$.
}
\begin{enumerate}
    \item \question{Montrer que l'ensemble des zéros de $y$ est infini dénombrable.}
    \item \question{On note $a_n$ le $n$ème zéro positif de $y$.
    En utilisant les fonctions
    $\begin{cases} \varphi(t) = \sin\Bigl(e^{a_n/2}(t-a_n)\Bigr) \cr
             \psi(t) = \sin\Bigl(e^{a_{n+1}/2}(t-a_n)\Bigr),\cr\end{cases}$ montrer que
    $\frac\pi{e^{a_{n+1}/2}} \le a_{n+1}-a_n \le \frac\pi{e^{a_n/2}}$.}
    \item \question{Donner un équivalent de $a_n$ quand $n\to\infty$.}
\reponse{
L'ensemble des zéros est localement fini d'après Cauchy-Lipchitz.

             Si $y$ ne s'annule pas sur $[a,+\infty[$, par exemple $y > 0$,
             alors $y$ est concave positive donc minorée, donc $y''\to-\infty$
             ce qui implique $y',y\to-\infty$, contradiction.
Soit $b_n = \frac\pi{e^{a_n/2}}$.
             Alors $b_{n+1} \le 2\ln\left(\frac {b_n}{b_{n+1}}\right) \le b_n$ et
             $b_n \to 0$ donc $b_n \sim b_{n+1} \sim 2\left(\frac{b_n}{b_{n+1}}-1\right)$.\par
             Alors $\frac 1{b_{n+1}} - \frac 1{b_n} \to \frac 12$,
             $b_n \sim \frac 1{2n}$ et $a_n \sim 2\ln n$.
}
\end{enumerate}
}
