\uuid{eXLA}
\exo7id{5737}
\auteur{rouget}
\datecreate{2010-10-16}
\isIndication{false}
\isCorrection{true}
\chapitre{Suite et série de fonctions}
\sousChapitre{Autre}

\contenu{
\texte{
Pour $x\in]-1,1[$, on pose $f(x)=\sum_{n=1}^{+\infty}x^{n^2}$. Trouver un équivalent simple de $f$ en $1$.
}
\reponse{
Soit $x\in]-1,1[$. Pour $n\in\Nn^*$, $\left|x^{n^2}\right|=|x|^{n^2}\leqslant|x|^n$. Puisque la série numérique de terme général $|x|^n$ converge, on en déduit que la série de terme général $x^{n^2}$ est absolument convergente et en particulier convergente. Donc, $f$ est bien définie sur $]-1,1[$.

Soit $x\in]0,1[$. La fonction $t\mapsto x^{t^2}=e^{t^2\ln x}$ est décroissante sur $[0,+\infty[$. Donc, $\forall k\in\Nn^*$, $\int_{k}^{k+1}x^{t^2}\;dt\leqslant x^{k^2}\leqslant\int_{k-1}^{k}x^{t^2}\;dt$. En sommant ces inégalités, on obtient

\begin{center}
$\forall x\in]0,1[$, $\int_{1}^{+\infty}x^{t^2}\;dt\leqslant f(x)\leqslant\int_{0}^{+\infty}x^{t^2}\;dt$\quad$(*)$.
\end{center}

Soit $x\in]0,1[$. En posant $u=t\sqrt{-\ln x}$, on obtient

\begin{center}
$\int_{0}^{+\infty}x^{t^2}\;dt=\int_{0}^{+\infty}e^{t^2\ln x}\;dt=\int_{0}^{+\infty}e^{-(t\sqrt{-\ln x})^2}\;dt=\frac{1}{\sqrt{-\ln x}}\int_{0}^{+\infty}e^{-u^2}\;du=\frac{\sqrt{\pi}}{2\sqrt{-\ln x}}$.
\end{center}

L'encadrement $(*)$ s'écrit alors

\begin{center}
$\forall x\in]0,1[$, $\frac{\sqrt{\pi}}{2\sqrt{-\ln x}}-\int_{0}^{1}x^{t^2}\;dt\leqslant f(x)\leqslant\frac{\sqrt{\pi}}{2\sqrt{-\ln x}}$.
\end{center}

Comme $\displaystyle\lim_{\substack{x\rightarrow1\\
x<1}}\frac{\sqrt{\pi}}{2\sqrt{-\ln x}}=+\infty$, on a montré que

\begin{center}
\shadowbox{
$\sum_{n=1}^{+\infty}x^{n^2}\underset{x\rightarrow1,\;x<1}{\sim}\frac{\sqrt{\pi}}{2\sqrt{-\ln x}}$.
}
\end{center}
}
}
