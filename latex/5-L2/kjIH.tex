\uuid{kjIH}
\exo7id{4365}
\auteur{quercia}
\datecreate{2010-03-12}
\isIndication{false}
\isCorrection{true}
\chapitre{Intégration}
\sousChapitre{Intégrale de Riemann dépendant d'un paramètre}

\contenu{
\texte{

}
\begin{enumerate}
    \item \question{Développer, pour tout $x>0$, $s(x) =  \int_{t=0}^{+\infty}\frac{\sin t}{e^{xt}-1}\,d t$
    en série de fractions rationnelles.}
\reponse{$s(x) =   \int_{t=0}^{+\infty} \sum_{k=1}^\infty\sin(t)e^{-kxt}\,d t$.

On a $|\sin(t)e^{-kxt}| \le te^{-kxt}$ et $ \int_{t=0}^{+\infty}te^{-kxt}\,d t = \frac 1{k^2}$
donc $\sum_{k=1}^{\infty} \int_{t=0}^{+\infty}|\sin(t)e^{-kxt}|\,d t$ converge
ce qui légitime l'interversion intégrale-série.
D'où $s(x) = \sum_{k=1}^\infty \int_{t=0}^{+\infty}\sin(t)e^{-kxt}\,d t
           = \sum_{k=1}^{\infty}\frac1{k^2x^2+1}$.}
    \item \question{Montrer qu'en $0^+$, $s(x)$ est équivalente à $\frac\pi{2x}$.}
\reponse{Sachant (?) que $ \int_{t=0}^{+\infty}\frac{\sin t}t\,d t = \frac\pi2$, on obtient~:

\begin{align*}
xs(x)-\frac\pi2
&=  \int_{t=0}^{+\infty}\Bigl(\frac{x\sin t}{e^{xt}-1}-\frac{\sin t}{t}\Bigr)\,d t\cr
&=  \int_{u=0}^{+\infty}\Bigl(\frac{1}{e^{u}-1}-\frac{1}{u}\Bigr)\sin\Bigl(\frac ux\Bigr)\,d u\cr
&= -x\Bigr[\underbrace{\Bigl(\frac{1}{e^{u}-1}-\frac{1}{u}\Bigr)}_{\to{\textstyle\frac12} \text{ si } u\to0^+}\cos\Bigl(\frac ux\Bigr)\Bigr]_{u=0}^{+\infty}
   +x \int_{u=0}^{+\infty}\underbrace{\Bigl(\frac{-e^u}{(e^{u}-1)^2}+\frac{1}{u^2}\Bigr)}_{\to{\textstyle\frac1{12}\text{ si } u\to0^+ }}\cos\Bigl(\frac ux\Bigr)\,d u\cr
&= x(\text{quantité bornée})\to 0 \text{ si } u\to0^+.\cr
\end{align*}}
\end{enumerate}
}
