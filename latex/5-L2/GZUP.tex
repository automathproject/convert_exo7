\uuid{GZUP}
\exo7id{2621}
\auteur{debievre}
\organisation{exo7}
\datecreate{2009-05-19}
\isIndication{true}
\isCorrection{true}
\chapitre{Fonction de plusieurs variables}
\sousChapitre{Limite}

\contenu{
\texte{
Pour chacune des suites $(u_n)_n$ dans le plan $\R^2$ ci-dessous, 
placer quelques-uns des points $u_n$ dans le plan 
et d\'ecrire qualitativement le comportement de la suite lorsque $n$ tend 
vers l'infini. \'Etudier ensuite 
la convergence de chacune des suites et d\'eterminer 
la limite le cas \'ech\'eant.
}
\begin{enumerate}
    \item \question{$u_n=(\frac{4n^2}{n^2+4n+3}, \cos\frac1n)$}
\reponse{$u_1=(\frac{1}{2}, \cos 1),\ u_2=(\frac{16}{15}, \cos \frac 12),\ 
\ldots \ ,
 u_{10}=(\frac{400}{143}, \cos \frac 1{10}),\ 
\ldots$}
    \item \question{$u_n=(\frac{n^2\arctan n}{n^2+1}, \sin(\frac{\pi}{4}\exp(-{\frac1n})))$}
\reponse{$u_1=(\frac{1}{2}\arctan 1,   \sin(\frac{\pi}{4\mathrm e     })),\ 
   u_2=(\frac{4}{5}\arctan 2,  \sin(\frac{\pi}{4\mathrm e^{1/2}})),\ \\
   u_3=(\frac{9}{10}\arctan 3, \sin(\frac{\pi}{4\mathrm e^{1/3}})),\ \ldots \ ,
u_{10}=(\frac{100}{101}\arctan (10), \sin(\frac{\pi}{4\mathrm e^{1/10}})),\ \ldots \ 
$}
    \item \question{$u_n=(\sinh n, \frac{\ln n}{n})$}
\reponse{$u_1=(\sinh 1, 0), \ 
   u_2=(\sinh 2, \frac{\ln 2}{2}), \ 
   u_3=(\sinh 3, \frac{\ln 3}{3}), \ \ldots \ ,\\ 
u_{10}=(\sinh 10, \frac{\ln 10}{10}),\ 
\ldots $}
    \item \question{$u_n=(a^n\cos (n\alpha), a^n\sin (n\alpha))$, en fonction de 
$a\in \R$, $a>0$ et $\alpha\in\R$.}
\reponse{$u_1=a^n   (\cos   (\alpha), \sin  (\alpha) ),\ 
   u_2=a^2   (\cos  (2\alpha), \sin  (2\alpha)),  \\
   u_3=a^3   (\cos  (3\alpha), \sin  (3\alpha)),\   \ldots \ , 
u_{10}=a^{10}(\cos (10\alpha), \sin (10\alpha)),\ \ldots
$}
\indication{Pour \'etablir ou r\'efuter
l'existences d'une limite  particuli\`ere dans le plan et pour ensuite
d\'eterminer une limite pourvu qu'elle existe, utiliser le fait que
pour que $\lim_{n \to \infty}(x_n,y_n)$ existe dans le plan $\R^2$ 
il faut et il suffit que
chacune des limites $\lim_{n \to \infty}x_n$ et $\lim_{n \to \infty}y_n$ existe
en tant que limite finie.}
\end{enumerate}
}
