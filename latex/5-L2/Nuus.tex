\uuid{Nuus}
\exo7id{5453}
\auteur{rouget}
\datecreate{2010-07-10}
\isIndication{false}
\isCorrection{true}
\chapitre{Intégration}
\sousChapitre{Intégrale de Riemann dépendant d'un paramètre}

\contenu{
\texte{
Soit $f$ une fonction continue sur $[a,b]$. Pour $x$ réel, on pose $F(x)=\int_{a}^{b}|t-x|f(t)\;dt$. Etudier la dérivabilité de $F$ sur $\Rr$.
}
\reponse{
Pour $x$ réel donné, la fonction $t\mapsto|t-x|f(t)$ est continue sur $[a,b]$ et donc $F(x)$ existe. Pour $x\leq a$,
$F(x)=\int_{a}^{b}(t-x)f(t)\;dt=-x\int_{a}^{b}f(t)\;dt+\int_{a}^{b}tf(t)\;dt$. $F$ est donc de classe $C^1$ sur $]-\infty,a]$ en tant que fonction affine et, pour $x<a$, $F'(x)=-\int_{a}^{b}f(t)\;dt$ (en particulier $F'_g(a)=-\int_{a}^{b}f(t)\;dt$)..

De même, pour $x\geq b$, $F(x)=x\int_{a}^{b}f(t)dt-\int_{a}^{b}tf(t)\;dt$. $F$ est donc de classe $C^1$ sur $[b,+\infty[$ en tant que fonction affine et, pour $x\geq b$, $F'(x)=\int_{a}^{b}f(t)\;dt$ (en particulier $F'_d(b)=\int_{a}^{b}f(t)\;dt$).

Enfin, si $a\leq x\leq b$, 

$$F(x)=\int_{a}^{x}(x-t)f(t)\;dt+\int_{x}^{b}(t-x)f(t)\;dt=x(\int_{a}^{x}f(t)\;dt-\int_{x}^{b}f(t)\;dt)-\int_{a}^{x}tf(t)\;dt+\int_{x}^{b}tf(t)\;dt.$$

$F$ est donc de classe $C^1$ sur $[a,b]$ et, pour $a\leq x\leq b$, 

\begin{align*}\ensuremath
F'(x)&=\int_{a}^{x}f(t)\;dt-\int_{x}^{b}f(t)\;dt+x(f(x)-(-f(x)))-xf(x)-xf(x)\\
 &=\int_{a}^{x}f(t)dt-\int_{x}^{b}f(t)\;dt.
\end{align*}

(et en particulier, $F'_d(a)=-\int_{a}^{b}f(t)\;dt=F'_g(a)$ et $F'_g(b)=\int_{a}^{b}f(t)\;dt=F'_d(b)$).

$F$ est continue $]-\infty,a]$, $[a,b]$ et $[b,+\infty[$ et donc sur $\Rr$. $F$ est de classe $C^1$ sur $]-\infty,a]$, $[a,b]$ et $[b,+\infty[$. De plus, $F'_g(a)=F'_d(a)$ et $F'_g(b)=F'_d(b)$. $F$ est donc de classe $C^1$ sur $\Rr$.
}
}
