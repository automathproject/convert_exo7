\uuid{beFb}
\exo7id{2619}
\auteur{debievre}
\organisation{exo7}
\datecreate{2009-05-19}
\isIndication{true}
\isCorrection{true}
\chapitre{Topologie}
\sousChapitre{Ouvert, fermé, intérieur, adhérence}

\contenu{
\texte{

}
\begin{enumerate}
    \item \question{Soit $(A_n)$ ($n\in\N$) 
une suite de parties ouvertes de $\R^2$. Est-ce que la
r\'eunion des $A_n$ est encore une partie ouverte? Et leur intersection?}
\reponse{La r\'eunion $\cup_n A_n$ d'une suite 
de parties ouvertes $A_n$ de $\R^2$
est bien une partie ouverte
de $\R^2$. En effet, soit $q$ un point de $\cup_n A_n$. Alors il existe
$n_0$ tel que $q$ appartienne \`a $A_{n_0}$. Puisque 
$A_{n_0}$ est ouvert, il existe un disque ouvert $D$
dans $A_{n_0}$ tel que $q$ appartienne \`a $D$. 
Par cons\'equent, il existe un disque ouvert $D$
dans $\cup_n A_n$ tel que $q$ appartienne \`a $D$. 

L'intersection $\cap_n A_n$  d'une suite 
de parties ouvertes $A_n$ de $\R^2$ n'est pas n\'ecessairement
ouverte. Par exemple, dans $\R$, l'intersection des intervalles ouverts
$]-1/n,1/n[$ est la partie $\{0\}\subseteq \R$
qui n'est pas ouverte.}
    \item \question{M\^eme question pour une famille de parties ferm\'ees.}
\reponse{La r\'eunion $\cup_n B_n$ d'une suite 
de parties ferm\'ees $B_n$ de $\R^2$
n'est pas n\'ecessairement
une partie  ferm\'ee de $\R^2$.
Car le compl\'ementaire $\mathcal C(\cup_n B_n)$ de $\cup_n B_n$
est l'intersection $\cap_n \mathcal CB_n$ des compl\'ementaires
et c'est donc l'intersection d'une suite $\left (\mathcal C B_n\right)$
de parties ouvertes  de $\R^2$ qui, d'apr\`es (1.), n'est pas 
 n\'ecessairement
une partie ouverte de $\R^2$.
De m\^eme,
l'intersection $\cap_n B_n$ d'une suite 
de parties ferm\'ees $B_n$ de $\R^2$
est bien une partie ferm\'ee
de $\R^2$. 
Car le compl\'ementaire $\mathcal C(\cap_n B_n)$ de $\cap_n B_n$
est la r\'eunion $\cup_n \mathcal CB_n$ des compl\'ementaires
et c'est donc la r\'eunion d'une suite  $\left(\mathcal C B_n\right)$
de parties ouvertes de $\R^2$ qui, d'apr\`es (1.),
est une partie ouverte de $\R^2$.}
\indication{Exploiter le fait que le compl\'ementaire d'un ouvert est ferm\'e
et que le compl\'ementaire d'un ferm\'e est ouvert.}
\end{enumerate}
}
