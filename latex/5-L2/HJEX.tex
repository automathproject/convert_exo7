\uuid{HJEX}
\exo7id{4087}
\auteur{quercia}
\organisation{exo7}
\datecreate{2010-03-11}
\isIndication{false}
\isCorrection{true}
\chapitre{Equation différentielle}
\sousChapitre{Equations différentielles linéaires}

\contenu{
\texte{
Soit $\lambda\in\R$ et $({\cal E})\Leftrightarrow \frac{d^2 u(x)}{d x^2} + (1-\lambda)x^2u(x) = 0$.
}
\begin{enumerate}
    \item \question{Montrer que les solutions de~$({\cal E})$ sont de la forme $H(x)e^{-x^2/2}$
    où $H$ est une fonction développable en série entière.}
\reponse{Il suffit de démontrer que les solutions de~$({\cal E})$ sont développables en série entière.
    La méthode des coefficients indéterminés donne $n(n-1)a_n = (\lambda-1)a_{n-4}$ si $n\ge 4$
    et $a_2 = a_3 = 0$ d'où $a_{4k} = \frac{(\lambda - 1)^ka_0}{\prod_{i=1}^k4i(4i-1)}$,
    $a_{4k+1} = \frac{(\lambda - 1)^ka_1}{\prod_{i=1}^k4i(4i+1)}$, $a_{4k+2} = a_{4k+3} = 0$.
    On obtient une série de rayon infini pour tout choix de~$a_0,a_1$ donc les solutions DSE
    forment un espace vectoriel de dimension~$2$ et on a ainsi trouvé toutes les solutions.}
    \item \question{Déterminer les valeurs de~$\lambda$ telles que $H$ soit une fonction
    polynomiale non nulle.}
\reponse{On doit avoir $H''(x) - 2xH'(x) + ((2-\lambda)x^2 - 1)H(x) = 0$.
    Si $H$ est une fonction polynomiale non nulle, en examinant les termes de
    plus haut degré on obtient une contradiction. Donc il n'existe pas de
    telle solution.}
\end{enumerate}
}
