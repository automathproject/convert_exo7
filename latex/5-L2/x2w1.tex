\uuid{x2w1}
\exo7id{4109}
\auteur{quercia}
\organisation{exo7}
\datecreate{2010-03-11}
\isIndication{false}
\isCorrection{true}
\chapitre{Equation différentielle}
\sousChapitre{Equations différentielles linéaires}

\contenu{
\texte{

}
\begin{enumerate}
    \item \question{Soient ${f,g}: \R \to \R$ continues et $k>0$, $t_0\in\R$
    tels que~: $\forall\ t\ge t_0,\ f(t)\le g(t) + k \int_{u=t_0}^t f(u)\,d u$.

    Montrer que~: $\forall\ t\ge t_0,\ f(t) \le g(t) + k \int_{u=t_0}^t e^{k(t-u)}g(u)\,d u$.}
\reponse{Poser $F(t) =  \int_{u=t_0}^t f(u)\,d u$ et résoudre l'inéquation
    différentielle $F'(t) \le g(t) + kF(t)$ par la formule de Duhamel.}
    \item \question{Soient ${A,B} : \R\to {\mathcal{M}_n(\R)}$ continues, $T>t_0$, $K>0$ et $\eta>0$
    tels que~: $\forall\ t\in[t_0,T],\ \|\kern-1.2pt|A(t)\|\kern-1.2pt|\le K$
    et $\|\kern-1.2pt|A(t)-B(t)\|\kern-1.2pt|\le\eta$.
    On note~$M_0$ (resp. $N_0$) la solution du problème de Cauchy~:
    $M(t_0)=I,\ M'(t) = A(t)M(t)$ (resp. $N(t_0)=I,\ N'(t) = B(t)N(t)$).
    Montrer que~: $\forall\ t\in[t_0,T],\ \|\kern-1.2pt|M_0(t)-N_0(t)\|\kern-1.2pt|
    \le e^{K(t-t_0)}(e^{\eta(t-t_0)}-1)$.}
\reponse{\begin{align*}
    M' = AM
    & \Rightarrow  \|\kern-1.2pt|M'(t)\|\kern-1.2pt| \le K\|\kern-1.2pt|M(t)\|\kern-1.2pt|\\
    & \Rightarrow  \|\kern-1.2pt|M(t)-I\|\kern-1.2pt| \le K \int_{u=t0}^t\|\kern-1.2pt|M(u)\|\kern-1.2pt|\,d u\\
    & \Rightarrow  \|\kern-1.2pt|M(t)\|\kern-1.2pt| \le 1 + K \int_{u=t0}^t\|\kern-1.2pt|M(u)\|\kern-1.2pt|\,d u\\
    & \Rightarrow  \|\kern-1.2pt|M(t)\|\kern-1.2pt| \le 1 + K \int_{u=t0}^te^{K(t-u)}\,d u = e^{K(t-t_0)}.\\
    \noalign{}
    (M-N)' &= (A-B)M + B(M-N)\\
    & \Rightarrow  \|\kern-1.2pt|(M-N)'(t)\|\kern-1.2pt| \le \eta e^{K(t-t_0)} + (K+\eta)\|\kern-1.2pt|(M-N)(t)\|\kern-1.2pt|\\
    & \Rightarrow  \|\kern-1.2pt|(M-N)(t)\|\kern-1.2pt| \le \frac\eta K (e^{K(t-t_0)}-1) + (K+\eta) \int_{u=t_0}^t\|\kern-1.2pt|(M-N)(u)\|\kern-1.2pt|\,d u\\
    & \Rightarrow  \|\kern-1.2pt|(M-N)(t)\|\kern-1.2pt| \le \underbrace{\frac\eta K (e^{K(t-t_0)}-1) + \frac{(K+\eta)\eta}K \int_{u=t_0}^te^{(K+\eta)(t-u)}(e^{K(u-t_0)}-1)\,d u}_{\textstyle=e^{K(t-t_0)}(e^{\eta(t-t_0)}-1)}\\
    \end{align*}}
    \item \question{On note $X_0$ (resp. $Y_0$) la solution du problème de Cauchy dans~$\R^n$~:
    $X(t_0) = \alpha,\ X'(t) = A(t)X(t)$ (resp. $Y(t_0) = \alpha,\ Y'(t) = B(t)Y(t)$)
    où $\alpha\in\R^n$. Quelle majoration a-t-on sur $\|X_0(t)-Y_0(t)\|$~?}
\reponse{$X_0(t) = M_0(t)\alpha$ et $Y_0(t) = N_0(t)\alpha$, d'où
    $\|X_0(t)-Y_0(t)\| \le e^{K(t-t_0)}(e^{\eta(t-t_0)}-1)\|\alpha\|$.}
\end{enumerate}
}
