\uuid{Gz7k}
\exo7id{2617}
\auteur{debievre}
\organisation{exo7}
\datecreate{2009-05-19}
\isIndication{true}
\isCorrection{true}
\chapitre{Topologie}
\sousChapitre{Ouvert, fermé, intérieur, adhérence}

\contenu{
\texte{
D\'eterminer si chacune des parties suivantes du plan
sont ouvertes ou
ferm\'ees, ou ni l'un ni l'autre. D\'eterminer chaque fois l'int\'erieur 
et l'adh\'erence.
}
\begin{enumerate}
    \item \question{$ A_1=\{ (x, y)\in\R^2 | x^2y^2 > 1\}$,}
\reponse{La partie $A_1$ est ouverte.
Car la courbe $x^2y^2 = 1$ a quatre branches,
les deux branches de $xy = 1$ et les deux branches de
$xy = -1$; ces quatre branches coupent le plan en cinq parties dont une
contient l'origine.
La courbe $x^2y^2 = 1$ \'etant une partie ferm\'ee,
le compl\'ementaire est un ouvert qui est r\'eunion de cinq ouverts.
La partie $A_1$ est la r\'eunion des quatre parties
qui ne contiennent pas l'origine.
Puisque $A_1$ est ouvert, $A_1$ co\"\i ncide avec son int\'erieur.
L'adh\'erence de $A_1$ est la r\'eunion de $A_1$ avec les quatre branches
de la courbe $x^2y^2 = 1$.}
    \item \question{$ A_2=\{ (x,y)\in\R^2 | x^2+y^2=1, y>0\}$.}
\reponse{La partie $A_2$ est le demi-cercle de rayon 1 ayant l'origine pour
centre constitu\'e des angles $0<\varphi<\pi$ en radians
et ce n'est ni ouvert ni ferm\'e.
La partie $A_2$ n'est pas ouverte car aucun disque de rayon positif n'est 
dans $A_2$; elle n'est pas ferm\'ee car les points $(\pm 1,0)$ sont
des points d'adh\'erence qui n'appartiennent pas \`a $A_2$.
L'adh\'erence de $A_2$ est la partie
\[
\{ (x,y)\in\R^2 | x^2+y^2=1, y\geq 0\}
\] 
du plan.}
\indication{Exploiter les propri\'et\'es g\'eom\'etriques
des parties du plan qui d\'efinissent $A_1$ et $A_2$. Par exemple,
une courbe qui est d\'efinie comme \'etant l'image r\'eciproque d'un point
relativement \`a une fonction continue est une partie ferm\'ee du plan.}
\end{enumerate}
}
