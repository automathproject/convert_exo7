\uuid{KunL}
\exo7id{5856}
\auteur{rouget}
\organisation{exo7}
\datecreate{2010-10-16}
\isIndication{false}
\isCorrection{true}
\chapitre{Topologie}
\sousChapitre{Application linéaire continue, norme matricielle}

\contenu{
\texte{
On munit $E=C^0([0,1],\Rr)$ de la norme $1$ définie par $\forall f\in E$, $\|f\|_1=\int_{0}^{1}|f(t)|\;dt$.

On pose $\begin{array}[t]{cccl}
T~:&E&\rightarrow&\rule{1.5mm}{0mm}E\\
 &f&\mapsto&\begin{array}[t]{cccc}Tf~:&[0,1]&\rightarrow&\Rr\\
  &x&\mapsto&\int_{0}^{x}f(t)\;dt
  \end{array}
 \end{array}$ et on admet que $T$ est un endomorphisme de $E$.
}
\begin{enumerate}
    \item \question{Démontrer que $T$ est continu sur $(E,\|\;\|_1)$ et déterminer $|||T|||$.}
\reponse{Soit $f\in E$.

\begin{align*}\ensuremath
\|Tf\|_1&=\int_{0}^{1}|Tf(x)|\;dx=\int_{0}^{1}\left|\int_{0}^{x}f(t)\;dt\right|dx\\
 &\leqslant\int_{0}^{1}\left(\int_{0}^{x}|f(t)|\;dt\right)dx\\
 &\leqslant\int_{0}^{1}\left(\int_{0}^{1}|f(t)|\;dt\right)dx=\int_{0}^{1}\|f\|_1\;dx=\|f\|_1.
\end{align*}

Ceci montre que $\forall f\in E\setminus\{0\}$, $ \frac{\|Tf\|_1}{\|f\|_1}\leqslant1$. Ceci montre que $T$ est continu sur $(E,\|\;\|_1)$ et que $|||T|||\leqslant1$.

Pour $n\in\Nn$ et $x\in[0,1]$, posons $f_n(x)=(1-x)^n$. Pour $n\in\Nn$,

\begin{center}
$\|f_n\|_1=\int_{0}^{1}(1-x)^n\;dx=\left[- \frac{(1-x)^{n+1}}{n+1}\right]_0^1= \frac{1}{n+1}$,
\end{center}

puis pour $x\in[0,1]$, $Tf_n(x)=\int_{0}^{x}(1-t)^n\;dt= \frac{1}{n+1}(1-(1-x)^{n+1})$ et donc

\begin{center}
$\|Tf_n\|_1=\int_{0}^{1}|Tf_n(x)|\;dx= \frac{1}{n+1}\int_{0}^{1}(1-(1-x)^{n+1})\;dx= \frac{1}{n+1}\left(1- \frac{1}{n+2}\right)= \frac{1}{n+2}$.
\end{center}

On en déduit que $\forall n\in\Nn$, $|||T|||\geqslant \frac{\|Tf_n\|_1}{\|f_n\|_1}= \frac{n+1}{n+2}$.

En résumé, $\forall n\in\Nn$, $ \frac{n+1}{n+2}\leqslant|||T|||\leqslant 1$ et donc $|||T|||=1$.

\begin{center}
\shadowbox{
$T$ est continu sur $(E,\|\;\|_1)$ et $|||T|||=1$.
}
\end{center}}
    \item \question{Vérifier que la borne supérieure n'est pas atteinte.}
\reponse{Supposons qu'il existe $f\in E\setminus\{0\}$ tel que $\|Tf\|_1=\|f\|_1$. On en déduit que chaque inégalité écrite au début de la question 1) est une égalité et en particulier $\int_{0}^{1}\left(\int_{0}^{x}|f(t)|\;dt\right)dx=\int_{0}^{1}\left(\int_{0}^{1}|f(t)|\;dt\right)dx$ ou encore $\int_{0}^{1}\left(\int_{0}^{1}|f(t)|\;dt-\int_{0}^{x}|f(t)|\;dt\right)dx=0$. Par suite, $\forall x\in[0,1]$, $\int_{0}^{1}|f(t)|\;dt-\int_{0}^{x}|f(t)|\;dt=0$ (fonction continue, positive, d'intégrale nulle) puis en dérivant la dernière inégalité, $\forall x\in[0,1]$, $|f(x)|=0$ et finalement $f=0$. Ceci est une contradiction et donc $|||T|||$ n'est pas atteinte.}
\end{enumerate}
}
