\uuid{8unT}
\exo7id{4126}
\auteur{quercia}
\datecreate{2010-03-11}
\isIndication{false}
\isCorrection{true}
\chapitre{Equation différentielle}
\sousChapitre{Equations différentielles non linéaires}

\contenu{
\texte{
Soit $F : {[a,b]\times \R} \to \R$ de classe $\mathcal{C}^1$ telle que pour tout
$x \in {[a,b]}$, l'application $y \mapsto F(x,y)$ est strictement croissante.
Montrer que pour tous $\alpha,\beta \in \R$, il existe au plus une solution à
l'équation :
$y'' = F(x,y)$ avec les conditions aux limites $y(a) = \alpha$, $y(b) = \beta$.
}
\reponse{
Soient $y,z$ deux solutions distinctes. D'après Cauchy-Lipschitz,
$y'(a) \ne z'(a)$, donc par exemple $y'(a) > z'(a)$.
Soit $c > a$ maximal tel que : $\forall\ x \in {]a,c[}$, $y(x) > z(x)$.
Donc $y-z$ est strictement positive convexe sur $[a,c]$, et s'annule en $a$ et $c$,
ce qui est impossible.
}
}
