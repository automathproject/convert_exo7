\uuid{eBSR}
\exo7id{6897}
\auteur{ruette}
\organisation{exo7}
\datecreate{2013-01-24}
\isIndication{false}
\isCorrection{true}
\chapitre{Probabilité discrète}
\sousChapitre{Probabilité conditionnelle}

\contenu{
\texte{
On a décelé dans une certaine population une probabilité de 0,01 pour
qu'un enfant soit atteint par une maladie M. La 
probabilité qu'un enfant qui n'est pas atteint par M ait une réaction 
négative à un test T est de 0,9. S'il est atteint par M, la probabilité 
qu'il ait une réaction positive au test est de 0,95.\\ 
Quelle est la probabilité  qu'un enfant pris
au hasard ait une réaction positive au test~?
Quelle est la probabilité
qu'un enfant pris au hasard et ayant une réaction 
positive soit atteint par M~?
}
\reponse{
On note $A$ l'événement ``l'enfant a la maladie M'', $A^c$ son
complémentaire (événement ``l'enfant n'a pas la maladie M''), $B$ l'événement
``l'enfant a une réaction positive au test'', $B^c$ son complémentaire
(événement
``l'enfant a une réaction négative au test''). D'après
l'énoncé, $P(A)=0,01$, $P(A^c)=0,99$, $P(B^c|A^c)=0,9$ et
$P(B|A)=0,95$. Donc,  
\begin{eqnarray*}P(B)&=&P(A\cap B)+P(A^c\cap
B)=P(B|A)P(A)+P(B|A^c)P(A^c)\\
&=&P(B|A)P(A)+(1-P(B^c|A^c))P(A^c)=0,0095+0,099=0,1085.
\end{eqnarray*}
La probabilité
qu'un enfant de moins de trois mois pris au hasard et ayant une réaction 
positive soit atteint par M est donnée par $P(A|B)=\frac{P(B\cap
A)}{P(B)}=\frac{P(B|A)P(A)}{P(B)}=\frac{95}{1085}\simeq 0,088$. Un tel test serait d'une utilité discutable.
}
}
