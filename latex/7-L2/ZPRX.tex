\uuid{ZPRX}
\exo7id{6003}
\auteur{quinio}
\datecreate{2011-05-20}
\isIndication{false}
\isCorrection{true}
\chapitre{Probabilité discrète}
\sousChapitre{Probabilité conditionnelle}

\contenu{
\texte{
Dans l'ancienne formule du Loto il fallait choisir 6 numéros parmi 49.
}
\begin{enumerate}
    \item \question{Combien y-a-t-il de grilles possibles ? En déduire la probabilité de
gagner en jouant une grille.}
\reponse{Combien de grilles ?
Il y en a $\binom{49}{6}=13\,983\,816$}
    \item \question{Quelle est la probabilité que la grille gagnante comporte 2
nombres consécutifs?}
\reponse{Combien de grilles avec 2 nombres consécutifs ?
Ce problème peut être résolu par astuce: considérer les numéros gagnants 
comme 6 places à <<choisir>> parmi 49.
En considérant des cloisons matérialisant les numéros gagnants,
c'est un problème de points et cloisons
Par exemple:
$$\mid \bullet \bullet \left\vert {}\right\vert \bullet
\left\vert \bullet \bullet \bullet \mid \bullet \bullet \right\vert$$
les gagnants sont: 1; 4; 5; 7; 11; 14.
Dans notre cas on ne veut pas de cloisons consécutives.
Les cinq cloisons séparent les numéros en 7 bo\^{\i}tes.
Les 5 bo\^{\i}tes intérieures étant non vides, on y met 5 points,
puis $38(=49-5-6)$ dans 7 bo\^{\i}tes.
Il y a $\frac{(38-1+7)!}{38!6!}=7.\,059\,1\times
10^{6}$ séquences ne comportant pas 2 nombres consécutifs.

D'où la probabilité d'avoir une grille
comportant 2 nombres consécutifs: $0.4952$.}
\end{enumerate}
}
