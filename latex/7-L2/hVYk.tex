\uuid{hVYk}
\exo7id{6002}
\auteur{quinio}
\organisation{exo7}
\datecreate{2011-05-20}
\isIndication{false}
\isCorrection{true}
\chapitre{Probabilité discrète}
\sousChapitre{Probabilité conditionnelle}

\contenu{
\texte{
Six couples sont réunis dans une soirée de réveillon. Une
fois les bises de bonne année échangées, on danse, de façon
conventionnelle: un homme avec une femme, mais pas forcément la sienne.
}
\begin{enumerate}
    \item \question{Quelle est la probabilité $P(A)$ pour que chacun des 6 hommes danse
avec son épouse légitime ?}
\reponse{L'univers des possibles est l'ensemble des couples possibles:
il y en a $6!=720$ (imaginez les dames assises et les hommes choisissant
leur partenaire). La probabilité $P(A)$ pour que chacun des $6$ hommes
danse avec son épouse légitime est, si chacun choisit au hasard, $\frac{1}{6!}$.}
    \item \question{Quelle est la probabilité $P(B)$ pour que André danse avec son 
épouse ?}
\reponse{André danse avec son épouse, les autres choisissent au hasard: il y
a $5!$ permutations pour ces derniers:
$P(B)=\frac{5!}{6!}=\frac{1}{6}$.}
    \item \question{Quelle est la probabilité $P(C)$ pour que André et René
dansent avec leur épouse ?}
\reponse{André et René dansent avec leur épouse, les $4$ autres
choisissent au hasard: il y a $4!$ permutations pour ces derniers:
$P(C)=\frac{4!}{6!}=\frac{1}{30}$.}
    \item \question{Quelle est la probabilité $P(D)$ pour que André ou René
danse(nt) avec leur épouse ?}
\reponse{André ou René dansent avec leur épouse, les $4$ autres font ce
qu'ils veulent. Considérons les événements $D_{1}:$ <<André
danse avec son épouse>> ; $D_{2}$ : <<René danse avec son épouse>>. 
Alors $D=D_{1}\cup D_{2}$ et
 $P(D_{1}\cup D_{2})=P(D_{1})+P(D_{2})-P(D_{1}\cap D_{2})=\frac{3}{10}$.}
\end{enumerate}
}
