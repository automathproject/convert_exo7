\uuid{RpZE}
\exo7id{6027}
\auteur{quinio}
\organisation{exo7}
\datecreate{2011-05-20}
\isIndication{false}
\isCorrection{true}
\chapitre{Statistique}
\sousChapitre{Tests d'hypothèses, intervalle de confiance}

\contenu{
\texte{
Un petit avion (liaison Saint Brieuc-Jersey) peut accueillir chaque jour 30
personnes; des statistiques montrent que 20\% des clients ayant réservé ne viennent pas.
Soit $X$ la variable aléatoire: <<nombre de clients qui
se présentent au comptoir parmi 30 personnes qui ont réservé>>.
}
\begin{enumerate}
    \item \question{Quelle est la loi de $X$ ? (on ne donnera que la forme générale);
quelle est son espérance, son écart-type ?}
\reponse{La loi de $X$ est la loi binomiale $n=30$, $p=0.2$.}
    \item \question{Donner un intervalle de confiance au seuil 95\%, permettant d'estimer le
nombre de clients à prévoir.}
\reponse{Un intervalle de confiance au seuil 95\%, permettant d'estimer le nombre
de clients à prévoir : c'est pour la fréquence: 0.657; 0.943.
Soit entre 20 et 28 personnes.
C'est une large fouchette due à $n$ petit.}
\end{enumerate}
}
