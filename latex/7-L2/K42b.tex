\uuid{K42b}
\exo7id{6028}
\auteur{quinio}
\organisation{exo7}
\datecreate{2011-05-20}
\isIndication{false}
\isCorrection{true}
\chapitre{Statistique}
\sousChapitre{Estimation}

\contenu{
\texte{
Le staff médical d'une grande entreprise fait ses
petites statistiques sur le taux de cholestérol de ses employés; les
observations sur 100 employés tirés au sort sont les suivantes.

\begin{tabular}{cc}
taux de cholestérol en cg:(centre classe) & effectif d'employés: \\
$120$ & $9$ \\ 
$160$ & $22$ \\ 
$200$ & $25$ \\ 
$240$ & $21$ \\ 
$280$ & $16$ \\ 
$320$ & $7$
\end{tabular}
}
\begin{enumerate}
    \item \question{Calculer la moyenne $m_{e}$ et l'écart-type $\sigma_{e}$ sur l'échantillon.}
    \item \question{Estimer la moyenne et l'écart-type pour le taux de cholestérol 
dans toute l'entreprise.}
    \item \question{Déterminer un intervalle de confiance pour la moyenne.}
    \item \question{Déterminer la taille minimum d'échantillon pour que
l'amplitude de l'intervalle de confiance soit inférieure à 10.}
\reponse{
On obtient, sur l'échantillon, la moyenne $m_{e}=214$, l'écart-type $\sigma _{e}=55.77$.
La moyenne sur l'entreprise est estimée par $m_{e}$.
L'écart-type est estimé par: $\widehat{\sigma _{e}}=\sqrt{\frac{100}{99}}55.77\simeq 56.05$.
On en déduit, au seuil 95\%, un intervalle de confiance pour la
moyenne :
$[m_{e} - y_{\alpha }\frac{\widehat{\sigma _{e}}}{\sqrt{n}};
m_{e} + y_{\alpha }\frac{\widehat{\sigma _{e}}}{\sqrt{n}}]=[203.01;224.99]$.
Ainsi le taux moyen de cholestérol est, à un seuil de confiance $95$\%, 
située entre $203$ et $225$ cg.
}
\end{enumerate}
}
