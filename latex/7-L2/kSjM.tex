\uuid{kSjM}
\exo7id{6000}
\auteur{quinio}
\organisation{exo7}
\datecreate{2011-05-20}
\isIndication{false}
\isCorrection{true}
\chapitre{Probabilité discrète}
\sousChapitre{Probabilité conditionnelle}

\contenu{
\texte{
On sait qu'à une date donnée, 3\% d'une population
est atteinte d'hépatite
On dispose de tests de dépistage de la maladie :
\begin{itemize}
\item Si la personne est malade, alors le test est positif avec une probabilité de 95\%.
\item Si la personne est saine, alors le test est positif avec une probabilité de 10\%.
\end{itemize}
}
\begin{enumerate}
    \item \question{Quelle est la probabilité pour une personne d'être malade
si son test est positif ?}
\reponse{La probabilité pour une personne d'être malade si son test est
positif est $P(M/T^{+})=P(T^{+}/M)P(M)/P(T^{+})$
or $P(T^{+})=P(T^{+}/M)P(M)+P(T^{+}/S)P(S)=0.95\cdot 0.03+0.1\cdot
0.97=0.125\,5$. 
D'où : $P(M/T^{+})=22.7\%$.}
    \item \question{Quelle est la probabilité pour une personne d'être saine si
son test est positif ?}
\reponse{La probabilité pour une personne d'être saine si son test est
positif est $P(S/T^{+})=1-P(M/T^{+})=77.3\%$.}
    \item \question{Quelle est la probabilité pour une personne d'être malade
si son test est négatif ?}
\reponse{La probabilité pour une personne d'être malade si son test est négatif 
est $P(M/T^{-})=0.0017$.}
    \item \question{Quelle est la probabilité pour une personne d'être saine si
son test est négatif ?}
\reponse{La probabilité pour une personne d'être saine si son test est négatif
est $1-P(M/T^{-})=0.998=99.8\%$.}
\end{enumerate}
}
