\uuid{x9Sr}
\exo7id{6927}
\auteur{ruette}
\organisation{exo7}
\datecreate{2013-01-24}
\isIndication{false}
\isCorrection{true}
\chapitre{Probabilité continue}
\sousChapitre{Loi faible des grands nombres}

\contenu{
\texte{
On suppose que le nombre de pièces sortant d'une
usine donnée en une journée est une variable aléatoire
d'espérance $50$.
}
\begin{enumerate}
    \item \question{Peut-on estimer la probabilité que la production de demain
dépasse 75 pièces~?}
\reponse{Inégalité de Markov : $P(X\ge 75)\le \frac{E(X)}{75}=\frac 23$.}
    \item \question{Que peut-on dire de plus sur cette probabilité si on sait que l'écart-type de la production quotidienne est
de $5$ pièces~?}
\reponse{Inégalité de Bienaymé-Tchebychev : $P(|X-50|\ge 25)\le \frac{\text{Var}(X)}{25^2}=
\frac{5^2}{25^2}=0,04$. Donc $P(X\ge 75)\le 0,04$.}
\end{enumerate}
}
