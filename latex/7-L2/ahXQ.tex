\uuid{ahXQ}
\exo7id{6009}
\auteur{quinio}
\datecreate{2011-05-20}
\isIndication{false}
\isCorrection{true}
\chapitre{Probabilité discrète}
\sousChapitre{Variable aléatoire discrète}

\contenu{
\texte{
Un candidat se présente à un concours où, cette fois, les
20 questions sont données sous forme de QCM. A chaque question, sont
proposées 4 réponses, une seule étant exacte. L'examinateur fait
le compte des réponses exactes données par les candidats.
Certains candidats répondent au hasard à chaque question; pour
ceux-la, définir une variable aléatoire associée à ce problème 
et donner sa loi de probabilité, son espérance.
}
\reponse{
Puisque les réponses sont données au hasard, chaque grille-réponses est en fait la 
répétition indépendante de $20$ épreuves aléatoires (il y a $4^{20}$ grilles-réponses). 
Pour chaque question la probabilité de succès est de $\frac{1}{4}$ et
l'examinateur fait le compte des succès: la variable aléatoire $X$,
nombre de bonnes réponses, obéit à une loi binomiale donc on a
directement les résultats.
Pour toute valeur de $k$ comprise entre $0$ et $20$:
$P[X=k]=C_{20}^{k}(\frac{1}{4})^{k}(1-\frac{1}{4})^{20-k}$, ce qui
donne la loi de cette variable aléatoire.

Quelle est l'espérance d'un candidat fumiste? C'est $E(X)=np=5$
}
}
