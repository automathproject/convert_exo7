\uuid{bDvu}
\exo7id{2174}
\auteur{debes}
\organisation{exo7}
\datecreate{2008-02-12}
\isIndication{false}
\isCorrection{true}
\chapitre{Action de groupe}
\sousChapitre{Action de groupe}

\contenu{
\texte{
Montrer que toute permutation d'ordre $10$ dans $S_8$ est
impaire.
}
\reponse{
L'ordre d'une permutation $\omega \in S_n$ est le $\mathrm{ppcm}$ des longueurs
des cycles de la d\'ecomposition de $\omega$ en cycles \`a supports disjoints. De plus, la
somme des longueurs de ces cycles (ceux de longueur $1$ y compris) vaut $n$. Pour une
permutation d'ordre $10$ dans $S_8$, il n'y a qu'un type possible: 5-2-1. La signature vaut
alors $(-1)^{5-1} (-1)^{2-1}=-1$.
}
}
