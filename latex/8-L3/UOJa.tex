\uuid{UOJa}
\exo7id{2206}
\auteur{debes}
\datecreate{2008-02-12}
\isIndication{false}
\isCorrection{true}
\chapitre{Théorème de Sylow}
\sousChapitre{Théorème de Sylow}

\contenu{
\texte{
Soient $p$ et $q$ deux nombres premiers et $G$ un groupe
d'ordre $p^2q$. On suppose que $p^2-1$ n'est pas divisible par $q$ et
que $q-1$ n'est pas divisible par $p$. Montrer que $G$ est ab\'elien.
}
\reponse{
Le nombre de $q$-Sylow d'un groupe $G$ d'ordre $p^2q$ est $\equiv 1\
[\hbox{\rm mod}\ q]$ et divise $p^2$. Ce ne peut \^etre ni $p$ ni $p^2$ car $p^2-1$
est suppos\'e non divisible par $q$; c'est donc $1$. De m\^eme le nombre de
$p$-Sylow est $\equiv 1\ [\hbox{\rm mod}\ p]$ et divise $q$ et ce ne peut \^etre
$q$ car $q-1$ est suppos\'e non divisible par $p$; c'est donc $1$. Ainsi il y a 
un unique $p$-Sylow $P$ d'ordre $p^2$, et donc ab\'elien, et un unique $q$-Sylow
$Q$ d'ordre $q$, et donc cyclique, tous deux n\'ecessairement distingu\'es. Il en
r\'esulte que tout \'el\'ement $x\in P$ commute avec tout \'el\'ement $y\in Q$:
en effet le commutateur $xyx^{-1}y^{-1} = (xyx^{-1})y^{-1}= x(yx^{-1}y^{-1})$ est
dans l'intersection $P\cap Q$ qui est le groupe trivial. Cela montre que le groupe
$PQ$ est ab\'elien; il est isomorphe au produit direct $P\times Q$ et est donc de
cardinal $|P|\hskip 2pt |Q| = p^2q=|G|$. D'o\`u finalement $G=PQ$ est ab\'elien.
}
}
