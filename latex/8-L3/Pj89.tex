\uuid{Pj89}
\exo7id{7741}
\auteur{mourougane}
\datecreate{2021-08-11}
\isIndication{false}
\isCorrection{true}
\chapitre{Théorème de Sylow}
\sousChapitre{Théorème de Sylow}

\contenu{
\texte{
Déterminer à isomorphisme près tous les groupes d'ordre $33$.
}
\reponse{
Les ordres des éléments de $G$ sont $3, 11$ ou $33$. Une application directe du
théorème de Sylow montre qu’on a un seul groupe d’ordre $11$ et un seul groupe d’ordre
$3$. Les éléments d’ordre $3$ et $11$ sont contenus dans ces deux groupes. On a au plus
$$1 + (3 - 1) + (11 - 1) = 1 + 2 + 10 = 13$$ éléments d’ordre 1, 3 ou $11$. Il existe donc un
élément d’ordre $33$ dans $G$ qui est donc cyclique isomorphe à $\mathbf{Z}/33\mathbf{Z}$.
}
}
