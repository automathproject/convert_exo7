\uuid{jNUi}
\exo7id{1442}
\auteur{legall}
\organisation{exo7}
\datecreate{1998-09-01}
\isIndication{false}
\isCorrection{true}
\chapitre{Groupe quotient, théorème de Lagrange}
\sousChapitre{Groupe quotient, théorème de Lagrange}

\contenu{
\texte{
Soit $  G  $ un groupe ; on note $  D(G)  $ le
groupe engendr\'e par les \'el\'ements de la forme
$ghg^{-1}h^{-1}$  ; $g,h\in G.$
}
\begin{enumerate}
    \item \question{Montrer que $  D(G)  $ est distingu\'e dans $  G  .$}
    \item \question{Montrer que $  G/D(G)  $ est commutatif~; plus g\'en\'eralement montrer qu'un sous-groupe distingu\'e
$  H  $ de $  G  $ contient $  D(G)  $ si et seulement si $  G/H  $ est commutatif.}
\reponse{
Il faut montrer que pour $x\in G$ et $y\in D(G)$, $xyx^{-1} \in D(G)$.
Commen\c{c}ons par montrer ceci pour $y$ un g\'en\'erateur de
$D(G)$. Si $y=ghg^{-1}h^{-1}$ avec $g,h\in G$. Nous remarquons que
:
$$xyx^{-1} = \left( x gh x^{-1}(gh)^{-1} \right) \left( gh g^{-1}h^{-1} \right)
\left( hg x (hg)^{-1}x^{-1} \right) $$ qui est un produit
d'\'el\'ements de $D(G)$. Donc $xyx^{-1}$ est un \'el\'ement de
$D(G)$.

Soit maintenant $y$ un \'el\'ement quelconque de $D(G)$, alors il
s'\'ecrit comme produit de g\'en\'erateurs :
$$y = y_1y_2\ldots y_n,\quad \text{ avec } y_i = g_ih_ig_i^{-1}h_i^{-1}.$$
\'Ecrivons $xyx^{-1} =
(xy_1x^{-1})(xy_2x^{-1})\ldots(xy_nx^{-1})$. Chaque $xy_ix^{-1}$
appartient \`a $D(G)$. Et donc $xyx^{-1}$. Donc $D(G)$ est un
sous-groupe distingu\'e de $G$.
Soit $\alpha,\beta \in G/D(G)$, alors il existe
$a,b \in G$ tels que $\overline{a}=\alpha$ et
$\overline{b}=\beta$. Nous savons que $aba^{-1}b^{-1}\in D(G)$ et
donc $\overline{aba^{-1}b^{-1}} = \epsilon$
 o\`u $\epsilon$ est l'\'el\'ement neutre de $G/D(G)$. Mais
$$\overline{aba^{-1}b^{-1}} = \overline{a}\overline{b}\overline{a^{-1}}\overline{b^{-1}}
= \overline{a}\overline{b}\overline{a}^{-1}\overline{b}^{-1} =
\alpha\beta\alpha^{-1}\beta^{-1}.$$ Donc
$\alpha\beta\alpha^{-1}\beta^{-1}=\epsilon$, autrement dit
$\alpha\beta= \beta\alpha$. Et ceci quelque soit $\alpha$ et
$\beta$, donc $G/D(G)$ est commutatif.
\bigskip
G\'en\'eralisation : si $H$ est un sous-groupe distingu\'e.
\begin{itemize}
[$\bullet$] Si $D(G) \subset H$ alors $G/D(G)$ est un sous-groupe de $G/H$ donc
$G/H$ est commutatif car $G/D(G)$ l'est.
[$\bullet$] Si $G/H$ est commutatif alors pour $g,h \in G$
la classe de $ghg^{-1}h{-1}$ dans $G/H$ v\'erifie :
$$\overline{ ghg^{-1}h{-1}} = \overline{g}\overline{h}\overline{g^{-1}}\overline{h^{-1}}
= \overline{g}\overline{ g^{-1}}\overline{h}\overline{h^{-1}} =
\epsilon.$$ Mais les \'el\'ements dont la classe dans $G/H$ est
l'\'el\'ement neutre sont exactement les \'el\'ements de $H$. Donc
$ghg^{-1}h^{-1}$ appartient \`a $H$. Ainsi  tous les
g\'en\'erateurs de $D(G)$ sont dans $H$ et donc $D(G) \subset H$.
\end{itemize}
}
\end{enumerate}
}
