\uuid{4tPm}
\exo7id{7765}
\auteur{mourougane}
\organisation{exo7}
\datecreate{2021-08-11}
\isIndication{false}
\isCorrection{false}
\chapitre{Groupe cyclique}
\sousChapitre{Groupe cyclique}

\contenu{
\texte{
\setcounter{MaxMatrixCols}{20}
}
\begin{enumerate}
    \item \question{On considère l'élément de $\mathcal{S}_8$ :
$$\sigma=\begin{pmatrix}1&2&3&4&5&6&7&8 \\ 5&6&4&1&2&8&3&7\end{pmatrix}.$$
Décomposer $\sigma$ en un produit de transpositions et calculer
sa signature. Peut-on écrire $\sigma$ comme produit de douze
transpositions ?}
    \item \question{Soit $\sigma$ l'élément de $\mathcal{S}_{11}$ :
$$\sigma=\begin{pmatrix}
1&2&3&4&5&6&7&8&9&10&11\\
10&7&9&11&2&1&3&5&8&4&6
\end{pmatrix}.$$
Décomposer $\sigma$ en un produit de cycles à support disjoints. Préciser 
l'ordre de $\sigma$, et la signature de $\sigma$. Calculer $\sigma^2$
et $\sigma^3$. \'Ecrire $\sigma^{-1}$ 
en un produit de cycles à support disjoints.}
\end{enumerate}
}
