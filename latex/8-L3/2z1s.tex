\uuid{2z1s}
\exo7id{2105}
\auteur{debes}
\datecreate{2008-02-12}
\isIndication{true}
\isCorrection{false}
\chapitre{Ordre d'un élément}
\sousChapitre{Ordre d'un élément}

\contenu{
\texte{
Soit $n$ et $p$ deux entiers, $p\leq n$. D\'emontrer, gr\^ace \`a un d\'enombrement, la
formule suivante:
$$\sum _{ 0\leq k\leq p} C_n ^k C_ {n-k}^{p-k} =2^p C_n^p$$
}
\indication{Compter, dans un ensemble $E$ \`a $n$ \'el\'ements, le nombre de parties \`a $p$
\'el\'ements obtenues en r\'eunissant une partie $X$ \`a $k$ \'el\'ements \`a une 
partie \`a $p-k$ \'el\'ements du compl\'ementaire de $X$ dans $E$, $k$ d\'ecrivant
$\{0,\ldots, p\}$.}
}
