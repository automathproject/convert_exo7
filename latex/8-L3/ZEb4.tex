\uuid{ZEb4}
\exo7id{2115}
\auteur{debes}
\datecreate{2008-02-12}
\isIndication{true}
\isCorrection{true}
\chapitre{Ordre d'un élément}
\sousChapitre{Ordre d'un élément}

\contenu{
\texte{
Soit $G$ un groupe d'ordre pair. Montrer qu'il existe un \'el\'ement $x\in
G$, $x \not= e$ tel que $x^2=e$.
}
\indication{Consid\'erer la partition de $G$ en sous-ensembles  du type $\{ x, x^{-1} \}$.}
\reponse{
En groupant chaque \'el\'ement $x\in G$ avec son inverse $x^{-1}$, on obtient une
partition de $G$ en sous-ensembles $\{y,y^{-1}\}$ qui ont deux \'el\'ements sauf
si $y=y^{-1}$, c'est-\`a-dire si $y^2=e$. L'\'el\'ement neutre $e$ est un tel
\'el\'ement $y$. Ce ne peut pas \^etre le seul, sinon $G$ serait d'ordre impair.
}
}
