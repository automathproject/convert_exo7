\uuid{GrCC}
\exo7id{6384}
\auteur{potyag}
\organisation{exo7}
\datecreate{2011-10-16}
\isIndication{false}
\isCorrection{false}
\chapitre{Action de groupe}
\sousChapitre{Action de groupe}

\contenu{
\texte{
Soient $G$ un groupe et $S$ un système de
générateurs de $G$ contenant avec chaque élément $s$ son
 inverse $s^{-1}$. Rappelons la construction du graphe de
Cayley $C(G,S)$. L'ensemble $V$ des sommets de $C(G,S)$
est en bijection  avec l'ensemble des éléments de $G$.
Deux sommets $g_1$ et $g_2$ sont joints par une arête si
$g^{-1}_1\cdot g_2 = s\in S$.
%
%(je ne crois pas  que ce soit un bonne idée de "noter" $s$,
%car cela donne plusieurs arêtes  avec le même nom)
%
%
La longueur de cette arête est déclarée par définition égale à $1$. Un
chemin $l\subset C(G,S)$ entre deux sommets $g$ et $h$ est une
succession finie d'arêtes $\{e_1,...,e_n\}$ joignant $g$ et $h$.
La longueur $|l|$ de $l$ vaut par définition $n$  : le nombre des
arêtes qui le constituent.
}
\begin{enumerate}
    \item \question{Montrer que la fonction $d:G \times G\mapsto \N$
  donnée par $$\displaystyle d(g,h)={\rm inf}\{{\rm longueurs\
  des\ chemins\ joignant\ } g {\rm\ et\ } h\}$$
  est une distance et qu'il existe un chemin $l\subset C(G,S)$ qui la
  réalise  c.-à.-d. $|l|=d(g,h)$.}
    \item \question{Pour chaque $g\in G$ posons $|g|=d(0,g)$. Montrer que
  $\displaystyle  |g|={\rm inf}\{k\ |\  g=s_{i_1}\cdot ... \cdot
  s_{i_k},\ s_{i_j}\in S\}$.}
    \item \question{Montrer que $G$ agit isométriquement sur les sommets de $C(G,S)$,
  c.-à.-d. $\ \forall g\in G\ d(g\gamma_1, g\gamma_2)=d(\gamma_1,\gamma_2)\
  {\rm o\grave u}\ \gamma_i\in V\ (i=1,2).$ En déduire que
  $d(f,h)=|f^{-1}\cdot h|\ (f,h\in V)$.}
    \item \question{Soit $F_2=<a,b>$ un groupe libre sur les générateurs
  $a$ et $b$ (voir l'exercice \ref{pot:exo9}). Donner un fragment
  (initial) de son graphe de Cayley $C(F_2, \{a,b\})$.}
    \item \question{Démontrer que
  le graphe de Cayley d'un groupe libre est toujours un arbre (un
  graphe sans lacet s'appelle arbre).}
\end{enumerate}
}
