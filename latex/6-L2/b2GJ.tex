\uuid{b2GJ}
\exo7id{7463}
\auteur{mourougane}
\datecreate{2021-08-10}
\isIndication{false}
\isCorrection{false}
\chapitre{Géométrie affine dans le plan et dans l'espace}
\sousChapitre{Géométrie affine dans le plan et dans l'espace}

\contenu{
\texte{
Soit $E$ un espace affine euclidien de dimension $3$ muni d'un repère cartésien
orthonormé $(O,i,j,k)$. On désigne par
$D$ la droite d'équation $(x=0,z=1)$ et par $D'$ la droite d'équation 
$(y=0,z=0)$. On note $S_D$ la
symétrie par rapport à la droite $D$ et $R_{\theta}$ la rotation 
d'axe $D'$ et d'angle $\theta$ (en
considérant la base $(j,k)$ comme directe). On pose $\varphi=S_D\circ 
R_{\theta}$.
}
\begin{enumerate}
    \item \question{{\'E}crire dans la base $(i,j,k)$ la matrice de 
$\overrightarrow {S_D}$, celle de
$\overrightarrow {R_{\theta}}$ et celle de $\overrightarrow{\varphi}$.
{\'E}crire les expressions analytiques de $S_D$ et de $R_{\theta}$ dans
le repère $(O,i,j,k)$.}
    \item \question{Montrer que $\varphi$ est une symétrie éventuellement 
glissée d'axe une droite $\Delta$.}
    \item \question{Pour tout point $M$ de $E$, prouver que les milieux de
$\bigl(M,s_\Delta (M)\bigr)$ et de $\bigl(M,\varphi(M)\bigr)$ 
sont sur $\Delta$.}
    \item \question{En utilisant le point $O$, montrer que $\Delta$ passe par le
point de coordonnées $(0,0,1)$.
et est contenue dans le plan affine d'équation $x=0$.}
    \item \question{Donner les composantes du vecteur de glissement de 
$\varphi$ en fonction de $\theta$.}
\end{enumerate}
}
