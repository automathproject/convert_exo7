\uuid{ol5R}
\exo7id{2011}
\auteur{liousse}
\datecreate{2003-10-01}
\isIndication{false}
\isCorrection{true}
\chapitre{Géométrie affine dans le plan et dans l'espace}
\sousChapitre{Géométrie affine dans le plan et dans l'espace}

\contenu{
\texte{

}
\begin{enumerate}
    \item \question{Trouver une équation du plan $(P)$ 
défini par les éléments suivants.


\begin{enumerate}}
\reponse{\begin{enumerate}}
    \item \question{$A$, $B$ et $C$ sont des points de $(P)$

\begin{enumerate}}
\reponse{Une équation d'un plan est $ax+by+cz+d=0$. 
Si un point appartient à un plan cela donne une condition linéaire sur $a,b,c,d$. 
Si l'on nous donne trois point cela donne un système linéaire de trois équations à trois inconnues
(car l'équation est unique à un facteur multplicatif non nul près). On trouve : 
    \begin{enumerate}}
    \item \question{$A(0,0,1)$, $B(1,0,0)$ et $C(0,1,0)$.}
\reponse{$x+y+z-1=0$}
    \item \question{$A(1,1,1)$, $B(2,0,1)$ et $C(-1,2,4)$.}
\reponse{$3x+3y+z-7=0$}
\end{enumerate}
}
