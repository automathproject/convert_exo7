\uuid{loRY}
\exo7id{5197}
\auteur{rouget}
\datecreate{2010-06-30}
\isIndication{false}
\isCorrection{true}
\chapitre{Géométrie affine euclidienne}
\sousChapitre{Géométrie affine euclidienne du plan}

\contenu{
\texte{
Déterminer le projeté orthogonal du point $M(x_0,y_0)$ sur la droite $(D)$ d'équation $x+3y-5=0$ ainsi que son symétrique orthogonal.
}
\reponse{
$(D)$ est une droite de vecteur normal $(1,3)$. Le projeté orthogonal $p(M_0)$ de $M_0$ sur $(D)$ est de la forme $M_0+\lambda.\vec{n}$ où $\lambda$ est un réel à déterminer. Le point $M_0+\lambda.\vec{n}$ a pour coordonnées $(x_0+\lambda,y_0+3\lambda)$.

$$M_0+\lambda.\vec{n}\in(D)\Leftrightarrow(x_0+\lambda)+3(y_0+3\lambda)-5=0\Leftrightarrow\lambda=\frac{-x_0-3y_0+5}{10}.$$

$p(M_0)$ a pour coordonnées $(x_0+\frac{-x_0-3y_0+5}{10},y_0+3\frac{-x_0-3y_0+5}{10})$ ou encore $(\frac{9x_0-3y_0+5}{10},\frac{-3x_0+y_0+15}{10})$.

Le symétrique orthogonal $s(M_0)$ vérifie~:~$s(M_0)=M_0+2\overrightarrow{M_0p(M_0)}$.

Ses coordonnées sont donc 
$(x_0+2(\frac{9x_0-3y_0+5}{10}-x_0),y0+2(\frac{-3x_0+y_0+15}{10}-y_0)$ ou encore

$(\frac{4x_0-3y_0+5}{5},\frac{-3x_0-4y_0+15}{5})$.

(Remarque. Si on n'avait pas déjà $p(M_0)$ on aurait cherché le symétrique sous la forme $M_0+\lambda.\vec{n}$, $\lambda$ étant entièrement déterminé par la condition~:~le milieu du segment $[M_0,s(M_0)]$ appartient à $(D)$.)
}
}
