\uuid{ULeB}
\exo7id{7451}
\auteur{mourougane}
\organisation{exo7}
\datecreate{2021-08-10}
\isIndication{false}
\isCorrection{false}
\chapitre{Géométrie affine dans le plan et dans l'espace}
\sousChapitre{Géométrie affine dans le plan et dans l'espace}

\contenu{
\texte{
On travaille dans un espace affine euclidien.
On rappelle que les homothéties-translations sont caractérisées par le
fait qu'elles transforment toute droite en une droite parallèle.
On appelle dilatation une application affine dont la partie linéaire
est une homothétie vectorielle.
}
\begin{enumerate}
    \item \question{Montrer que l'ensemble des dilatations coïncide avec
 l'ensemble des homothéties-translations.}
    \item \question{Montrer que l'ensemble des dilatations est un sous-groupe
 distingué du groupe des applications affines.}
    \item \question{On travaille maintenant dans le plan affine euclidien $\mathcal{P }$.
 Montrer qu'il existe exactement deux dilatations qui
 transforment un cercle donné en un cercle donné.}
\end{enumerate}
}
