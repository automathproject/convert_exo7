\uuid{Q8We}
\exo7id{5032}
\auteur{quercia}
\organisation{exo7}
\datecreate{2010-03-17}
\isIndication{false}
\isCorrection{false}
\chapitre{Courbes planes}
\sousChapitre{Propriétés métriques : longueur, courbure,...}

\contenu{
\texte{
Soit $\mathcal{C}$ la courbe d'équations paramétriques
$\begin{cases} x = a(t-\sin t) \cr y = a(1-\cos t) \cr \end{cases}$
pour $t \in {]0,2\pi[}$ (arche de cycloïde).
On note $S$ le point de paramètre $\pi$, et $D$ la tangente à $\mathcal{C}$ en $S$.

Soit $M \in \mathcal{C}\setminus\{S\}$, $I$ le point d'intersection de la normale à $\mathcal{C}$ en $M$ et
de $Ox$, et $J$ le point d'intersection de la tangente en $M$ avec $D$.
}
\begin{enumerate}
    \item \question{Faire un dessin.}
    \item \question{Montrer que $I$ et $J$ ont même abscisse.}
    \item \question{On prend $S$ comme origine des abscisses curvilignes. Trouver une relation
    entre $s$ et $\vec{MJ}$.}
\end{enumerate}
}
