\uuid{ciGf}
\exo7id{5502}
\auteur{rouget}
\organisation{exo7}
\datecreate{2010-07-10}
\isIndication{false}
\isCorrection{true}
\chapitre{Géométrie affine euclidienne}
\sousChapitre{Géométrie affine euclidienne de l'espace}

\contenu{
\texte{
Dans $E_3$ rapporté à un repère $(O,i,j,k)$, on donne~:
la droite $(D)$ dont un système d'équations paramétriques est $\left\{
\begin{array}{l}
x=2+3t\\
y=-t\\
z=1+t
\end{array}
\right.
$,
le plan $P$ dont un système d'équations paramétriques est $\left\{
\begin{array}{l}
x=1+2\lambda+\mu\\
y=-1-3\lambda+2\mu\\
z=1+\lambda
\end{array}
\right.
$,
le plan $P'$ dont un système d'équations paramétriques est $\left\{
\begin{array}{l}
x=-5-\nu\\
y=3+\nu+3\eta\\
z=\nu+\eta
\end{array}
\right.
$,
Etudier $D\cap P$ et $P\cap P'$
}
\reponse{
Les vecteurs $(2,-3,1)$ et $(1,2,0)$ ne sont pas colinéaires, de sorte que $(P)$ est bien un plan. Trouvons alors une équation cartésienne de $(P)$

\begin{align*}\ensuremath
M(x,y,z)\in(P)&\Leftrightarrow\exists(\lambda,\mu)\in\Rr^2/\;
\left\{
\begin{array}{l}
x=1+2\lambda+\mu\\
y=-1-3\lambda+2\mu\\
z=1+\lambda
\end{array}
\right.
\Leftrightarrow\exists(\lambda,\mu)\in\Rr^2/\;
\left\{
\begin{array}{l}
\lambda=z-1\\
x=1+2(z-1)+\mu\\
y=-1-3(z-1)+2\mu
\end{array}
\right.
\\
 &\Leftrightarrow\exists(\lambda,\mu)\in\Rr^2/\;
\left\{
\begin{array}{l}
\lambda=z-1\\
\mu=x-2z+1\\
y=-1-3(z-1)+2(x-2z+1)
\end{array}
\right.
\\
 &\Leftrightarrow
-2x+y+7z-4=0
\end{align*}
Soit alors $M(2+3t,-t,1+t)$, $t\in\Rr$, un point de $(D)$

\begin{align*}\ensuremath
M\in(P)&\Leftrightarrow-2(2+3t)+(-t)+7(1+t)-4=0
\Leftrightarrow
0\times t-1=0.
\end{align*}
Ce dernier système n'a pas de solution et donc $(D)\cap(P)=\varnothing$. La droite $(D)$ est strictement parallèle au plan $(P)$.

\begin{align*}\ensuremath
M(x,y,z)\in(P)\cap(P')&\Leftrightarrow\exists(\nu,\eta)\in\Rr^2/\;\left\{
\begin{array}{l}
x=-5-\nu\\
y=3+\nu+3\eta\\
z=\nu+\eta\\
-2x+y+7z-4=0
\end{array}
\right.\\
 &\Leftrightarrow\exists(\nu,\eta)\in\Rr^2/\;\left\{
\begin{array}{l}
x=-5-\nu\\
y=3+\nu+3\eta\\
z=\nu+\eta\\
-2(-5-\nu)+(3+\nu+3\eta)+7(\nu+\eta)-4=0
\end{array}
\right.\\
 &\Leftrightarrow\exists(\nu,\eta)\in\Rr^2/\;\left\{
\begin{array}{l}
\eta=-\nu-\frac{9}{10}\\
x=-5-\nu\\
y=3+\nu+3\left(-\nu-\frac{9}{10}\right)\\
z=\nu+\left(-\nu-\frac{9}{10}\right)
\end{array}
\right.
\Leftrightarrow
\exists\nu\in\Rr/\;
\left\{
\begin{array}{l}
x=-\nu-5\\
y=-2\nu+\frac{3}{10}\\
z=-\frac{9}{10}
\end{array}
\right.
\end{align*}
$(P)$ et $(P')$ sont donc sécants en la droite passant par le point $\left(-5,\frac{3}{10},-\frac{9}{10}\right)$ et de vecteur directeur $(1,2,0)$.
}
}
