\uuid{xCDk}
\exo7id{7504}
\auteur{mourougane}
\datecreate{2021-08-10}
\isIndication{false}
\isCorrection{false}
\chapitre{Géométrie affine euclidienne}
\sousChapitre{Géométrie affine euclidienne du plan}

\contenu{
\texte{
Dans le plan euclidien orienté muni d'un repère orthonormé direct $(O,\vec{\imath},\vec{\jmath})$, 
on considère la translation $t$ de vecteur $\vec{u}= \vec{\imath}+2\vec{\jmath}$
et la rotation $r$ de centre $A(-3,-1)$ et d'angle $+\pi/2$.
}
\begin{enumerate}
    \item \question{Montrer que $t\circ r$ a un unique point fixe.}
    \item \question{Décomposer la translation $t$ en produit $s_{d_2}\circ s_{d_1}$ de réflexions
    par rapport à des droites $d_1$ et $d_2$ que l'on décrira.}
    \item \question{Décomposer la rotation $r$ en produit $s_{d_4}\circ s_{d_3}$ de réflexions
    par rapport à des droites $d_3$ et $d_4$ que l'on décrira.}
    \item \question{Est-il possible de choisir $d_1=d_4$ dans les questions précédentes ?}
    \item \question{Déterminer par une méthode géométrique la nature et 
    les éléments caractéristiques de la composée $t\circ r$.}
\end{enumerate}
}
