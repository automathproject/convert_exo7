\uuid{Gbiq}
\exo7id{7053}
\auteur{megy}
\organisation{exo7}
\datecreate{2017-01-08}
\isIndication{true}
\isCorrection{false}
\chapitre{Géométrie affine euclidienne}
\sousChapitre{Géométrie affine euclidienne du plan}

\contenu{
\texte{
% source : wikipedia, Carrega
On donne deux points $O$ et $I$, avec $OI=1$. Un réel $r$ est constructible si on peut construire à la règle et au compas un point $M$ tel que $\overrightarrow{OM}=r\overrightarrow{OI}$. Le but de l'exercice est de montrer que l'ensemble des nombres constructibles est un sous-corps de $\R$ stable par racine carrée.
}
\begin{enumerate}
    \item \question{Construire sur la droite $(OI)$ des points $A$, $B$ et  $C$  tels que $OA = \frac{1}{\sqrt{2}}$,  $OB =\sqrt{2}$ et $OC =\sqrt{3}$.}
    \item \question{(Construction du produit et de l'inverse de deux nombres constructibles.) On donne deux points $A$ et $B$ alignés avec $O$. Construire sur la droite $(AB)$ des points $C$ et $D$ tel que $OC = OA\times OB$ et $OD = \frac{OA}{OB}$.}
    \item \question{(Construction de la racine carrée.) Soit $A$ un point sur la demi-droite $[OI)$. Soit $I'$ le symétrique de $I$ par rapport à $O$, soit $\mathcal C$ le cercle de diamètre $I'A$, et soit $F$ l'une des intersections du cercle $\mathcal C$ avec la perpendiculaire à $(OA)$ passant par $O$. Montrer que $OF = \sqrt{OA}$.}
\indication{\begin{enumerate}
\item Utiliser des triangles particuliers.
\item Utiliser le théorème de Thalès.
\end{enumerate}}
\end{enumerate}
}
