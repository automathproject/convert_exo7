\uuid{df0S}
\exo7id{4873}
\auteur{quercia}
\datecreate{2010-03-17}
\isIndication{false}
\isCorrection{true}
\chapitre{Géométrie affine dans le plan et dans l'espace}
\sousChapitre{Applications affines}

\contenu{
\texte{
On fixe un repère ${\cal R} = (O,\vec e_1, \vec e_2, \vec e_3)$ d'un espace
affine de dimension 3.
Reconaître l' application ayant l'expression  analytique  suivante :

   $$\begin{cases}x' =  3x + 4y + 2z - 4 \cr
            y' = -2x - 3y - 2z + 4 \cr
            z' =  4x + 8y + 5z - 8.\cr\end{cases}$$

   (chercher les points fixes de $f$ et étudier $\overrightarrow{MM'}$)
}
\reponse{
affinité de base ${\cal P} : x + 2y + z = 2$,
            de direction vect$(\vec e_1 - \vec e_2 + 2\vec e_3)$,
            de rapport 3.
}
}
