\uuid{00Fy}
\exo7id{7470}
\auteur{mourougane}
\organisation{exo7}
\datecreate{2021-08-10}
\isIndication{false}
\isCorrection{false}
\chapitre{Géométrie affine dans le plan et dans l'espace}
\sousChapitre{Géométrie affine dans le plan et dans l'espace}

\contenu{
\texte{
On considère le plan euclidien muni d'un repère orthonormé ($O, \overrightarrow {\imath},\overrightarrow{\jmath}$)  et la courbe $(C)$ d'équation

\begin{center}$ x^{2} - 2xy + y^{2} -6x - 10y + 9 = 0$ \end{center}
}
\begin{enumerate}
    \item \question{Montrer que $(C)$ est une parabole.}
    \item \question{Trouver un repère orthonormé ($S, \overrightarrow {u_{1}},\overrightarrow{u_{2}}$) tel que $(C)$ ait une équation de la forme $ x^{2} = 2py$ dans ce repère. 

 \emph{Indication.} On devra trouver que dans le repère orthonormé ($O, \overrightarrow {\imath},\overrightarrow{\jmath}$) le sommet a pour coordonnées $(0,1)$ et pour axe la droite $ y = x + 1$.}
\end{enumerate}
}
