\uuid{b3UC}
\exo7id{5002}
\auteur{quercia}
\datecreate{2010-03-17}
\isIndication{false}
\isCorrection{false}
\chapitre{Courbes planes}
\sousChapitre{Courbes définies par une condition}

\contenu{
\texte{
Soit $f:\R \to \R$ une fonction continue. On étudie les
courbes planes paramétrées par une abscisse curviligne, $s$, telles que la
courbure au point $M_s$ soit $c = f(s)$.
}
\begin{enumerate}
    \item \question{Montrer que si l'on impose la position de $M_0$ et la tangente en ce point,
    le problème admet une solution unique.}
    \item \question{Dans le cas général, démontrer que les courbes solutions se déduisent
    d'une courbe particulière en appliquant un déplacement du plan arbitraire.}
    \item \question{\'Etudier les équations : $c = {}$cste, $c = \frac 1s$
    (spirale logarithmique).}
\end{enumerate}
}
