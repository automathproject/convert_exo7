\uuid{IsM7}
\exo7id{7076}
\auteur{megy}
\organisation{exo7}
\datecreate{2017-01-11}
\isIndication{false}
\isCorrection{true}
\chapitre{Géométrie affine dans le plan et dans l'espace}
\sousChapitre{Propriétés des triangles}

\contenu{
\texte{
% Euclide, I prop. 17
% cercles inscrits et exinscrits
 On donne un cercle  $\mathcal C$ (de centre $O$), un point $M$ à l'extérieur du cercle, les deux tangentes $\mathcal D$ et $\mathcal D'$ à $\mathcal C$ passant par $M$. On notera $A$ et $B$ les points de tangence. 

Le cercle $\mathcal C$ coupe $(MO)$ en deux points $P$ et $Q$. D'autre part, soit $H$ l'intersection de la corde $[AB]$ avec $(OM)$. Montrer que les cercles de centres $P$ et $Q$ et passant par $H$ sont tangents à $\mathcal D$ et $\mathcal D'$.
}
\reponse{
Pour le cercle de centre $P$, il suffit de montrer que $P$ est le centre du cercle inscrit du triangle $MAB$. Pour cela, en notant $C$ le projeté orthogonal de $P$ sur $(MA)$, il suffit de montrer que $PC=PH$, ou de montrer que $AC=AH$. Or, on a $AC=AH = \cos(\widehat{AMO}) / OA$.

D'autres solutions sont possibles, par exemple avec des homothéties.
}
}
