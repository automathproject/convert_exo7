\uuid{43RE}
\exo7id{4888}
\auteur{quercia}
\datecreate{2010-03-17}
\isIndication{false}
\isCorrection{true}
\chapitre{Géométrie affine dans le plan et dans l'espace}
\sousChapitre{Barycentre}

\contenu{
\texte{
Soient $A_0,A_1,A_2$ trois points donnés. On considère la suite $(A_k)$
de points vérifiant la relation de récurrence~:
$$\forall\ k\ge3,\ A_k = \text{Bar}( A_{k-1}:1, A_{k-2},1, A_{k-3}:2 ).$$
\'Etudier la convergence de cette suite.
}
\reponse{
$A_k = \text{Bar}(A_0 : \alpha_k, A_1 : \beta_k, A_2 : \gamma_k)$
         où $\alpha_k$, $\beta_k$, $\gamma_k$ vérifient :
         $x_k = x_{k-1} + x_{k-2} + 2x_{k-3}$.

         Les racines de l'équation caractéristique sont $2,j,j^2$, donc
         $x_k \sim \lambda2^k$ avec $\lambda = \frac{x_0+x_1+x_2}7 = \frac 17$.

         Donc $\alpha_k \sim \beta_k \sim \gamma_k$, et $A_k \to G$,
         isobarycentre de $A_0A_1A_2$.
}
}
