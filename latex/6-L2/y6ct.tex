\uuid{y6ct}
\exo7id{4921}
\auteur{quercia}
\organisation{exo7}
\datecreate{2010-03-17}
\isIndication{false}
\isCorrection{true}
\chapitre{Conique}
\sousChapitre{Hyperbole}

\contenu{
\texte{
Soient $A,F$ deux points distincts, $D$ leur médiatrice, ${\cal H}$
l'hyperbole de foyer $F$, directrice $D$, excentricité 2, et ${\cal C}$
un cercle passant par $A$ et $F$, de centre $I$.
}
\begin{enumerate}
    \item \question{Pour $M \in {\cal C}$, montrer que $M \in {\cal H}
    \Leftrightarrow 3\overline{(\vec{IM},D)} \equiv \overline{(\vec{IF},D)}\ [2\pi]$.}
    \item \question{En déduire que si $I \notin (AF)$, ${\cal C}$ coupe ${\cal H}$ aux sommets
    d'un triangle équilatéral.}
\reponse{
$MH = \frac12MF  \Rightarrow $ $IMH$, $IMN$, et $INF$ sont semblables.
}
\end{enumerate}
}
