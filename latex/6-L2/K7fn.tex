\uuid{K7fn}
\exo7id{7121}
\auteur{megy}
\organisation{exo7}
\datecreate{2017-02-08}
\isIndication{false}
\isCorrection{true}
\chapitre{Géométrie affine euclidienne}
\sousChapitre{Géométrie affine euclidienne du plan}

\contenu{
\texte{
% angles inscrits, facile, application directe
Montrer qu'un trapèze est isocèle si et seulement s'il est inscriptible.
}
\reponse{
dans un trapèze, deux angles non adjacents à une même base sont supplémentaires, puisque les deux bases sont parallèles.
un quadrilatère non croisé est inscriptible ssi les angles opposés sont supplémentaires.
}
}
