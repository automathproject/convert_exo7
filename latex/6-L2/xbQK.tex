\uuid{xbQK}
\exo7id{7516}
\auteur{mourougane}
\organisation{exo7}
\datecreate{2021-08-10}
\isIndication{false}
\isCorrection{true}
\chapitre{Géométrie affine euclidienne}
\sousChapitre{Géométrie affine euclidienne de l'espace}

\contenu{
\texte{
Soient $E$ un espace affine de dimension 3, et $A,B,C,D$ un
tétraèdre de $E$. Montrer que les droites joignant les milieux des cotés
opposés du tétraèdre sont concourantes.
}
\reponse{
On considère l'isobarycentre $G$ des points $A,B,C,D$.
Par associativité, $G$ est le barycentre des barycentres $I$ de $(A,1),(B,1)$ 
et $J$ de $(C,1),(D,1)$ affecté des masses $(I,1+1)$ et $(J,1+1)$.
Par conséquent, $G$ est au milieu de $I$, $J$.
En particulier, les droites joignant les milieux des cotés
opposés du tétraèdre sont concourantes en $G$.
}
}
