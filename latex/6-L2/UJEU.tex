\uuid{UJEU}
\exo7id{4866}
\auteur{quercia}
\datecreate{2010-03-17}
\isIndication{false}
\isCorrection{true}
\chapitre{Géométrie affine dans le plan et dans l'espace}
\sousChapitre{Sous-espaces affines}

\contenu{
\texte{
Soit $(O,\vec i,\vec j)$ un repère de ${\cal E}_2$, et
$A : \left(\begin{smallmatrix} 1 \cr 0 \cr\end{smallmatrix}\right)$,
$B : \left(\begin{smallmatrix} 0 \cr 1 \cr\end{smallmatrix}\right)$,
$C : \left(\begin{smallmatrix} 0 \cr 2 \cr\end{smallmatrix}\right)$.

Pour $m \in \R$, on construit les droites $D : y = mx$ et $D' : y = -mx$,
puis $M \in D \cap (AB)$, et $M' \in D' \cap (AC)$
(si possible).

Montrer que la droite $(MM')$ passe par un point fixe (= indépendant de $m$).
}
\reponse{
$\left(\begin{smallmatrix} 0 \cr 4/3 \cr\end{smallmatrix}\right)$.
}
}
