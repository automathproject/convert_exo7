\uuid{pzhE}
\exo7id{1960}
\auteur{liousse}
\organisation{exo7}
\datecreate{2003-10-01}
\isIndication{true}
\isCorrection{true}
\chapitre{Géométrie affine dans le plan et dans l'espace}
\sousChapitre{Géométrie affine dans le plan et dans l'espace}

\contenu{
\texte{
On considère le triangle 
$ABC$ dont les c\^otés ont pour équations
$(AB) : x+2y=3 , (AC) : x+y=2 , (BC) : 2x+3y=4$.
}
\begin{enumerate}
    \item \question{Donner les coordonnées des points $A, B, C$.}
    \item \question{Donner les coordonnées des milieux $A', B', C'$ des segments $[BC]$, $[AC]$ et $[AB]$ respectivement.}
    \item \question{Donner une équation de chaque médiane et vérifier qu'elles sont concourantes.}
\reponse{
Le point $A$ est l'intersection des droites $(AB)$ et $(AC)$.
Les coordonnées $(x,y)$ de $A$ sont donc solutions du système :
$\left\{ \begin{array}{l}
x+2y=3 \\ x+y=2 \\ 
\end{array} \right.$ donné par 
les équations des deux droites.
La seule solution est $(x,y)=(1,1)$.
On a donc $A=(1,1)$. On fait de même pour obtenir le point $B=(-1,2)$ et $C=(2,0)$.
Notons $A'$ le milieu de $[BC]$ alors les coordonnées se trouvent par la formule suivante
$A'= (\frac{x_B+x_C}{2},\frac{y_B+y_C}{2}) = (\frac12,1)$.
De même on trouve $B'=(\frac32,\frac12)$ et $C'=(0,\frac32)$.
\begin{enumerate}
Les médianes ont pour équations : 
$(AA') : (y=1)$ ; $(BB') : (3x+5y=7)$ ; $(CC') : (3x+4y=6)$.
Vérifions que les trois médianes sont concourantes (ce qui est vrai quelque soit le triangle).
On calcule d'abord l'intersection $I=(AA')\cap (BB')$, les coordonnées du point $I$ d'intersection vérifient 
donc le système 
$\left\{ \begin{array}{l}
y=1 \\ 3x+5y=7 \\ 
\end{array} \right.$. On trouve $I=(\frac23,1)$.

Il ne reste plus qu'à vérifier que $I$ appartient à la droite $(CC')$ d'équation $3x+4y=6$.
En effet $3x_I+4y_I=6$ donc $I \in (CC')$.

Conclusion : les médianes sont concourantes au point $I=(\frac23,1)$.
}
\indication{Les médianes sont les droites $(AA')$, $(BB')$, $(CC')$.}
\end{enumerate}
}
