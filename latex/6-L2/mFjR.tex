\uuid{mFjR}
\exo7id{7120}
\auteur{megy}
\organisation{exo7}
\datecreate{2017-02-08}
\isIndication{true}
\isCorrection{true}
\chapitre{Géométrie affine euclidienne}
\sousChapitre{Géométrie affine euclidienne du plan}

\contenu{
\texte{
% angle au centre, inscrit. application directe
 Construire un octogone convexe régulier dont un des côtés est un segment $[AB]$ donné.
}
\indication{Il y a deux tels octogones. En notant $O$ le centre d'un tel octogone, on doit avoir $\widehat{AOB}=\pm \pi/4$.}
\reponse{
Construisons un triangle $AIB$ isocèle rectangle en $I$ et le cercle de centre $I$ et de rayon $IA$. Ce cercle intersecte la médiatrice de $[AB]$ en un point $O$ qui vérifie $\widehat{AOB}=\pm \pi/4$, par le théorème de l'angle au centre. C'est donc le centre d'un octogone appuyé sur $[AB]$. En traçant le cercle de centre $O$ et de rayon $OA$, on peut terminer la construction de cet octogone.
}
}
