\uuid{H3x1}
\exo7id{2041}
\auteur{liousse}
\datecreate{2003-10-01}
\isIndication{false}
\isCorrection{false}
\chapitre{Géométrie affine euclidienne}
\sousChapitre{Géométrie affine euclidienne du plan}

\contenu{
\texte{
Dans le plan cart\'esien identifi\'e \`a $\Cc$, un point $M$ est repr\'esent\'e par son affixe $z$.
}
\begin{enumerate}
    \item \question{Dessiner les ensembles suivants puis les exprimer en fonction de $(x,y)$ ($(z=x+iy)$) :

(i) $z+\overline{z}=1$ \hfill  (ii) $z-\overline{z}=i$ \hfill  (iii) $iz-i\overline{z}=1$}
    \item \question{Donner l'expression analytique en complexe des transformations suivantes, puis 
calculer l'image de $i$ par ces transformations :
\begin{enumerate}}
    \item \question{la rotation de centre $1+i$ et
d'angle $\pi\over 3$,}
    \item \question{la sym\'etrie d'axe la droite d'\'equation  $iz-i\overline{z}=1$,}
    \item \question{la compos\'ee des deux applications pr\'ec\'edentes.}
\end{enumerate}
}
