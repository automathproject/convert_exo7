\uuid{946O}
\exo7id{4859}
\auteur{quercia}
\datecreate{2010-03-17}
\isIndication{false}
\isCorrection{true}
\chapitre{Géométrie affine dans le plan et dans l'espace}
\sousChapitre{Sous-espaces affines}

\contenu{
\texte{
Soient $I,J,K$ trois points du plan. Montrer l'équivalence entre les trois
    propriétés~:
    
       a) $I$, $J$, $K$ sont alignés.\par
       b) Il existe $M$ tel que $\det(\vec{MI},\vec{MJ}) +  \det(\vec{MJ},\vec{MK}) +  \det(\vec{MK},\vec{MI}) = 0$.\par
       c) Pour tout point $M$, on a $\det(\vec{MI},\vec{MJ}) +  \det(\vec{MJ},\vec{MK}) +  \det(\vec{MK},\vec{MI}) = 0$.\par
}
\reponse{
$   \det(\vec{MI},\vec{MJ}) +  \det(\vec{MJ},\vec{MK}) +  \det(\vec{MK},\vec{MI}) = \det(\vec{IJ},\vec{IK})$.
}
}
