\uuid{3gjo}
\exo7id{1992}
\auteur{liousse}
\datecreate{2003-10-01}
\isIndication{false}
\isCorrection{false}
\chapitre{Géométrie affine dans le plan et dans l'espace}
\sousChapitre{Géométrie affine dans le plan et dans l'espace}

\contenu{
\texte{

}
\begin{enumerate}
    \item \question{On consid\`ere la famille des 
droites $D_{\lambda} : x + \lambda y + 1 =0$, o\`u $\lambda \in \Rr$.
\begin{enumerate}}
    \item \question{V\'erifier que ces droites passent toutes par un m\^eme point $A$ dont 
on donnera les coordonn\'ees.}
    \item \question{Parmi toutes ces droites, y en a-t-il une qui est verticale ? Si oui donner une \'equation de
cette droite.}
    \item \question{Parmi toutes ces droites, y en a-t-il une qui est horizontale ? Si oui donner une \'equation de
cette droite.}
    \item \question{Parmi toutes ces droites, y en a-t-il qui sont  parall\`eles, confondues  ou perpendiculaires \`a la droite 
$\Delta$  d'\'equation $2x-3y+1=0$ ? Si oui donner des \'equations
 de ces droites.}
\end{enumerate}
}
