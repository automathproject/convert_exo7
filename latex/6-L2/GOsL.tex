\uuid{GOsL}
\exo7id{1991}
\auteur{liousse}
\organisation{exo7}
\datecreate{2003-10-01}
\isIndication{false}
\isCorrection{false}
\chapitre{Géométrie affine dans le plan et dans l'espace}
\sousChapitre{Géométrie affine dans le plan et dans l'espace}

\contenu{
\texte{
On consid\`ere les droites $D : x+2y=5$ et $D' : 3x- y = 1$ et on note 
$A$ l'intersection des deux droites et $B$ le point de coordonn\'ees $(5,2)$.
}
\begin{enumerate}
    \item \question{Donner une \'equation cart\'esienne de la droite $(AB)$.}
    \item \question{Donner une \'equation cart\'esienne de la perpendiculaire \`a $D$ 
passant par $B$.}
    \item \question{Donner une \'equation cart\'esienne de la parall\`ele \`a $D'$ 
passant par $B$.}
    \item \question{Soit $C$ le point de coordonn\'ees $(2,-7)$). Donner une \'equation
cart\'esienne de la m\'ediatrice $\Delta$ du segment $[B,C]$. $\Delta$ est-elle parall\`ele \`a $D$ ? Et \`a $D'$ ?}
\end{enumerate}
}
