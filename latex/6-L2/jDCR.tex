\uuid{jDCR}
\exo7id{7502}
\auteur{mourougane}
\organisation{exo7}
\datecreate{2021-08-10}
\isIndication{false}
\isCorrection{false}
\chapitre{Géométrie affine euclidienne}
\sousChapitre{Géométrie affine euclidienne du plan}

\contenu{
\texte{
On munit le plan affine euclidien $(P,<,>)$ d'un repère orthonormé 
$(O,\vec{\imath},\vec{\jmath})$.
On considère les points $A(-1,2)$ et $B(5,4)$.
}
\begin{enumerate}
    \item \question{Déterminer les coordonnées du barycentre $G$ des points massiques 
    $A(-3)$ et $B(1)$.}
    \item \question{Calculer $-3GA^ 2+GB^2$.}
    \item \question{Démontrer que pour tout point $M$ du plan, on a $$-3MA^2+MB^2=-2MG^2 -3GA^ 2+GB^2.$$}
    \item \question{Déterminer l'ensemble $\mathcal L$ des points $M$ du plan tels que $-3MA^2+MB^2=50$.}
\end{enumerate}
}
