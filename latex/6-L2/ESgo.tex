\uuid{ESgo}
\exo7id{6877}
\auteur{gammella}
\organisation{exo7}
\datecreate{2012-05-29}
\isIndication{false}
\isCorrection{true}
\chapitre{Analyse vectorielle}
\sousChapitre{Forme différentielle, champ de vecteurs, circulation}

\contenu{
\texte{
Quel est le champ vectoriel qui dérive du potentiel
$$U(x,y,z)=1+x+xy+xyz ?$$
}
\reponse{
Le champ vectoriel qui dérive du potentiel $U$ est
$$\vec{\mathrm{grad}}(U)= (  \frac{\partial u}{\partial x},   \frac{\partial u}{\partial y},  \frac{\partial u}{\partial z}).$$ 
Il s'agit donc du champ vectoriel de composantes :
$$\vec{\mathrm{grad}}(U)=(1+y+yz, x+xz, xy).$$
}
}
