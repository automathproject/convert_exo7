\uuid{EVCx}
\exo7id{5542}
\auteur{rouget}
\datecreate{2010-07-15}
\isIndication{false}
\isCorrection{true}
\chapitre{Conique}
\sousChapitre{Conique}

\contenu{
\texte{
Le plan est rapporté à un repère orthonormé $\mathcal{R}=(0,\overrightarrow{i},\overrightarrow{j})$.
Nature et éléments caractéristiques de la courbe dont une équation en repère orthonormé est
}
\begin{enumerate}
    \item \question{$y=\frac{1}{x}$,}
\reponse{On note $\mathcal{H}$ l'hyperbole considérée. On tourne de $\frac{\pi}{4}$. Pour cela, on pose $\left\{
\begin{array}{l}
x=\frac{1}{\sqrt{2}}(X-Y)\\
y=\frac{1}{\sqrt{2}}(X+Y)
\end{array}
\right.$. On a alors

$$y=\frac{1}{x}\Leftrightarrow xy=1\Leftrightarrow \frac{1}{2}(X-Y)(X+Y)=1\Leftrightarrow\frac{X^2}{\left(\sqrt{2}\right)^2}-\frac{Y^2}{\left(\sqrt{2}\right)^2}=1.$$
Ainsi, si $\mathcal{R}$ est le repère orthonormé initial $\left(O,\overrightarrow{i},\overrightarrow{j}\right)$ et $\mathcal{R}'$ est le repère
$\left(O,\overrightarrow{I},\overrightarrow{J}\right)$ où $\overrightarrow{I}=\frac{1}{\sqrt{2}}(\overrightarrow{i}+\overrightarrow{j})$ et
$\overrightarrow{J}=\frac{1}{\sqrt{2}}(-\overrightarrow{i}+\overrightarrow{j})$, une équation de $\mathcal{H}$ dans $\mathcal{R}$ est $xy=1$ et une
équation de $\mathcal{H}$ dans $\mathcal{R}'$ est $\frac{X^2}{(\sqrt{2})^2}-\frac{Y^2}{(\sqrt{2})^2}=1$. On obtient
$a=b=\sqrt{2}$, $c=\sqrt{a^2+b^2}=2$ et $e=\frac{c}{a}=\sqrt{2}$.
Les formules de changement de repère s'écrivent $\left\{
\begin{array}{l}
x=\frac{1}{\sqrt{2}}(X-Y)\\
y=\frac{1}{\sqrt{2}}(X+Y)
\end{array}
\right.$ et les formules inverses s'écrivent

$\left\{
\begin{array}{l}
X=\frac{1}{\sqrt{2}}(x+y)\\
y=\frac{1}{\sqrt{2}}(-x+y)
\end{array}
\right.$ (dans ce qui suit, les coordonnées d'un point dans $\mathcal{R}'$ seront notées avec $\mathcal{R}'$ en indice
alors que les coordonnées dans $\mathcal{R}$ seront notées sans écrire $\mathcal{R}$ en indice).

$\shadowbox{\mbox{Centre}\;O(0,0)}$.

Asymptotes~:~bien sûr, les axes $(Ox)$ et $(Oy)$.

Axe focal~:~l'axe $(OX)$ ou encore la droite d'équation $y=x$ (dans $\mathcal{R}$).

Sommets~:~$A(\sqrt{2},0)_{\mathcal{R}'}$, $A'(-\sqrt{2},0)_{\mathcal{R}'}$ et donc
$\shadowbox{\mbox{Sommets}\;A(1,1)\;\mbox{et}\;A'(-1,-1)}$.

Foyers~:~$F(2,0)_{\mathcal{R}'}$, $F'(-2,0)_{\mathcal{R}'}$ et
donc \shadowbox{Foyers\;$F(\sqrt{2},\sqrt{2})$\; et \;$F'(-\sqrt{2},-\sqrt{2})$.}

Directrices~:~les droites d'équations $X=\pm\frac{a}{e}=\pm1$ et donc dans $\mathcal{R}$, les droites d'équations
respectives $x+y=\pm\sqrt{2}$.

%41x^2-24xy+34y^2-106x+92y+74=0}
    \item \question{$41x^2-24xy+34y^2-106x+92y+74=0$,}
\reponse{Le discriminant de cette conique vaut $41\times34-12^2=1250>0$. Il s'agit donc d'une conique du genre ellipse.
On pose $\left\{
\begin{array}{l}
x=\cos(\theta)X-\sin(\theta)Y\\
y=\sin(\theta)X+\cos(\theta)Y
\end{array}
\right.$ et on détermine $\theta$ (ou plutôt $\cos\theta$ et $\sin\theta$) de sorte que le terme en $XY$ disparaisse.
Mais, le coefficient de $XY$ dans

$$41x^2-24xy+34y^2=41(\cos(\theta)X-\sin(\theta)Y)^2-24(\cos(\theta)X-\sin(\theta)Y)(\sin(\theta)X+\cos(\theta)Y)+
34(\sin(\theta)X+\cos(\theta)Y)^2,$$
vaut

$$-82\cos\theta\sin\theta-24(\cos^2\theta-\sin^2\theta)+68\cos\theta\sin\theta=-24(\cos^2\theta-\sin^2\theta)
-14\cos\theta\sin\theta.$$
Ce coefficient est nul si et seulement si $-12\cos^2\theta+12\sin^2\theta-7\cos\theta\sin\theta=0$ ou encore, après division
par $\cos^2\theta$, $12\tan^2\theta-7\tan\theta-12=0$. On peut alors prendre $\tan\theta=\frac{4}{3}$, puis on peut prendre
$\cos\theta=\frac{1}{\sqrt{1+\tan^2\theta}}=\frac{3}{5}$ et
$\sin\theta=\cos\theta\tan\theta=\frac{3}{5}\frac{4}{3}=\frac{4}{5}$.

Posons donc $\left\{
\begin{array}{l}
x=\frac{3X-4Y}{5}\\
y=\frac{4X+3Y}{5}
\end{array}\right.$ $(*)$. On a alors

\begin{align*}\ensuremath
41x^2-24xy+34y^2-106x+92y+74&=\frac{1}{25}(41(3X-4Y)^2-24(3X-4Y)(4X+3Y)+34(4X+3Y)^2\\
 &\;-530(3X-4Y)+460(4X+3Y)+1850)\\
 &=\frac{1}{25}(625X^2+1250Y^2+250X+3500Y+1850)\\
 &=25\left(X^2+2Y^2+\frac{2}{5}X+\frac{28}{5}Y+\frac{74}{25}\right)
\end{align*}
Une équation de la courbe dans le repère défini par $(*)$ est donc
$X^2+2Y^2+\frac{2}{5}X+\frac{28}{5}Y+\frac{74}{25}=0$. Ensuite,

\begin{center}
$X^2+2Y^2+\frac{2}{5}X+\frac{28}{5}Y+\frac{74}{25}=0\Leftrightarrow\left(X+\frac{1}{5}\right)^2+2\left(Y+\frac{7}{5}\right)^2=1\Leftrightarrow\left(X+\frac{1}{5}\right)^2+\frac{\left(Y+\frac{7}{5}\right)^2}{\left(\sqrt{2}\right)^2}=1$.
\end{center}
$\mathcal{C}$ est une ellipse. On trouve $a=1$, $b=\sqrt{2}$, $c=1$ $e=\frac{1}{\sqrt{2}}$ puis
Centre $\Omega\left(1,-1\right)$. Axe focal : $3x+4y+1=0$ et axe non focal : $-4x+3y+7=0$.

Sommets : $A\left(\frac{8}{5},-\frac{1}{5}\right)$, $A'\left(\frac{2}{5},-\frac{9}{5}\right)$, $B\left(1-\frac{4\sqrt{2}}{5},-1+\frac{3\sqrt{2}}{5}\right)$ et $B'\left(1+\frac{4\sqrt{2}}{5},-1-\frac{3\sqrt{2}}{5}\right)$.

Foyers : $F\left(\frac{1}{5},-\frac{2}{5}\right)$ et $F'\left(\frac{9}{5},-\frac{8}{5}\right)$. Directrices : $4x-3y+3=0$ et $4x-3y+17=0$.}
    \item \question{$x^2+2xy+y^2+3x-2y+1=0$,}
\reponse{$x^2+2xy+y^2=(x+y)^2$. On pose donc $\left\{
\begin{array}{l}
X=\frac{1}{\sqrt{2}}(x-y)\\
Y=\frac{1}{\sqrt{2}}(x+y)
\end{array}
\right.$ ou encore $\left\{
\begin{array}{l}
x=\frac{1}{\sqrt{2}}(X+Y)\\
y=\frac{1}{\sqrt{2}}(-X+Y)
\end{array}
\right.$.

\begin{align*}\ensuremath
x^2+2xy+y^2+3x-2y+1=0&\Leftrightarrow 2Y^2+\frac{3}{\sqrt{2}}(X+Y)-\frac{2}{\sqrt{2}}(-X+Y)+1=0\\
 &\Leftrightarrow 2\left(Y+\frac{1}{4\sqrt{2}}\right)^2+\frac{5}{\sqrt{2}}X+\frac{15}{16}=0\Leftrightarrow\left(Y+\frac{1}{4\sqrt{2}}\right)^2=-\frac{5}{2\sqrt{2}}\left(X+\frac{3}{8\sqrt{2}}\right).
\end{align*}
$\mathcal{C}$ est une parabole de paramètre $p=\frac{5}{4\sqrt{2}}$.

Sommet : $S\left(-\frac{5}{16},\frac{1}{16}\right)$. Axe focal : $x+y+\frac{1}{4}=0$.

Foyer : $F\left(-\frac{5}{8},\frac{3}{8}\right)$. Directrice : $x-y-\frac{1}{4}=0$.}
    \item \question{$(x-y+1)^2+(x+y-1)^2=0$,}
\reponse{$\mathcal{C}$ est le point d'intersection des droites d'équation $x-y+1=0$ et $x+y-1=0$ c'est-à-dire le point de coordonnées $(0,1)$.}
    \item \question{$x^2+y^2-3x-y+3=0$,}
\reponse{$x^2+y^2-3x-y+3=\left(x-\frac{3}{2}\right)^2+\left(y-\frac{1}{2}\right)^2+\frac{1}{2}>0$ et donc $\mathcal{C}$ est vide.}
    \item \question{$x(x-1)+(y-2)(y-3)=0$,}
\reponse{$x(x-1)+(y-2)(y-3)=0$ est une équation du cercle de diamètre $[AB]$ où $A(0,2)$ et $B(1,3)$.}
    \item \question{$(x+y+1)(x-y+3)=3$,}
\reponse{Si on pose $\left\{
\begin{array}{l}
X=x+y+1\\
Y=x-y+3
\end{array}
\right.$, on effectue un changement de repère non orthonormé. Dans le nouveau repère, $\mathcal{C}$ admet pour équation cartésienne $XY=3$ et donc $\mathcal{C}$ est une hyperbole. Avec le changement de repère effectué, on obtient directement les éléments affines de cette hyperbole mais pas ses éléments métriques : hyperbole d'asymptotes les droites d'équations $x+y+1=0$ et $x-y+3=0$ et donc de centre $(-2,1)$.
Pour obtenir l'axe focal, l'excentricité, les foyers et les directrices il faut faire un changement de repère orthonormé.}
    \item \question{$(2x+y-1)^2-3(x+y)=0$.}
\reponse{Si on pose $\left\{
\begin{array}{l}
X=2x+y+1\\
Y=3x+3y
\end{array}
\right.$, $\mathcal{C}$ admet pour équation cartésienne dans le nouveau repère $Y=X^2$ et donc $\mathcal{C}$ est une parabole. Pour obtenir ces éléments métriques, il faut un changement de repère orthonormé.}
\end{enumerate}
}
