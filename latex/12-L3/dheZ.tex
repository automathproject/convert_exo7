\uuid{dheZ}
\exo7id{6297}
\auteur{queffelec}
\organisation{exo7}
\datecreate{2011-10-16}
\isIndication{false}
\isCorrection{false}
\chapitre{Autre}
\sousChapitre{Autre}

\contenu{
\texte{
Soient $\rho, \theta, \varphi$ les coordonnées
sphériques dans $\Rr^3$. On pose $\sin \varphi = t$.
Montrer que, pour qu'une fonction de la forme $f(x,y,z) =
\rho^n P(t)$, où $n$ est un entier $\geq 0$, soit
harmonique, il faut et il suffit que la fonction $t \mapsto
P(t)$ soit solution de l'équation différentielle (dite
\emph{de Legendre}) : $$(1 - t^2) P''(t) - 2t
P'(t) + n(n + 1) P(t) = 0.\eqno (D_n)$$
Pour $n = 0,1,2,3,4,5$, vérifiez, en le calculant par la
méthode des coefficients indéter\-minés, qu'il y a un
polyn\^ome $P_n(t)$, et un seul, de degré $n$, solution
de $(D_n)$, et tel que $P_n(1) = 1$. [Remarque : ce fait
vaut pour tout $n$ ; les polyn\^omes $P_n$ s'appellent
polyn\^omes de Legendre].
}
}
