\uuid{xyAq}
\exo7id{7671}
\auteur{mourougane}
\datecreate{2021-08-11}
\isIndication{false}
\isCorrection{false}
\chapitre{Sous-variété}
\sousChapitre{Sous-variété}

\contenu{
\texte{
Soit $S_1$ et $S_2$ deux sous-surfaces différentiables de $\Rr^3$.
}
\begin{enumerate}
    \item \question{Soit $g~:\Rr^3\to\Rr^3$ une application de classe $\mathcal{C}^\infty$.
On suppose que $g(S_1)\subset S_2$. On note $f : S_1\to S_2$ la restriction de $g$ à $S_1$.
Montrer $f$ est une application différentiable.}
    \item \question{Soit $f : S_1\to S_2$ une application différentiable et $p$ un point de $S_1$.
Montrer qu'il existe un voisinage de $p$ dans $\Rr^3$ sur lequel $f$ se prolonge en une application $\mathcal{C}^\infty$ 
à valeurs dans $\Rr^3$.}
    \item \question{Montrer que la composée de deux applications différentiables entre sous-surfaces différentiables de $\Rr^3$
est différentiable et expliciter la différentielle de la composée.}
\end{enumerate}
}
