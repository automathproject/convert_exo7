\uuid{wXwy}
\exo7id{7669}
\auteur{mourougane}
\organisation{exo7}
\datecreate{2021-08-11}
\isIndication{false}
\isCorrection{false}
\chapitre{Espace tangent, application linéaire tangente}
\sousChapitre{Espace tangent, application linéaire tangente}

\contenu{
\texte{
Dans tout l'exercice, $r$ et $h$ parcourent $]0,+\infty[$.
}
\begin{enumerate}
    \item \question{Déterminer le volume $V(r,h)$ d'un cylindre plein de hauteur $h$ et de rayon $r$.}
    \item \question{Déterminer l'aire $A(r,h)$ d'une casserole de hauteur $h$ et de rayon $r$.}
    \item \question{Montrer que le sous-ensemble de $\Rr^3$ avec coordonnées $(r,h,v)$ 
d'équation $1=V(r,h)$ est une surface régulière.}
    \item \question{Déterminer le gradient de la fonction $\alpha : (r,h,v)\mapsto A(r,h)$ 
et celui de la fonction $\nu :(r,h,v)\mapsto V(r,h)-1$}
    \item \question{Soit $(r_0,h_0,v_0)$ un minimum de la fonction $\alpha$ sur la surface d'équation $1=V(r,h)$.
Comparer $grad _{(r_0,h_0,v_0)}\alpha$ et $grad _{(r_0,h_0,v_0)}\nu$.}
    \item \question{Déterminer le minimum de $A(r,h)$ sur la surface d'équation
$V(r,h)=1$. Interpréter ce résultat.}
\end{enumerate}
}
