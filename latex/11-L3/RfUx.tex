\uuid{RfUx}
\exo7id{7703}
\auteur{mourougane}
\organisation{exo7}
\datecreate{2021-08-11}
\isIndication{false}
\isCorrection{true}
\chapitre{Sous-variété}
\sousChapitre{Sous-variété}

\contenu{
\texte{
On considère l'application 
$$\begin{array}{ccc}
 F:]-\pi,\pi[\times ]-\pi,\pi[ &\to&\Rr ^3\\ 
\left(\begin{array}{c}\varphi\\ \theta\end{array}\right)
&\mapsto& 
\begin{pmatrix}(2+\cos(\varphi))\cos(\theta)\\ (2+\cos(\varphi))\sin(\theta)\\ r\sin (\varphi)\end{pmatrix}.
\end{array}$$
}
\begin{enumerate}
    \item \question{Montrer qu'on définit une métrique riemannienne sur $Im(F)=T$
en posant $g(\frac{\partial F}{\partial \varphi})=g(\frac{\partial F}{\partial \theta})=1$
et $g(\frac{\partial F}{\partial \varphi},\frac{\partial F}{\partial \theta})=0$.}
\reponse{Comme $F $ est un paramétrage de la surface régulière $T$, les vecteurs 
 $X_\varphi=\frac{\partial F}{\partial \varphi}$ et $X_\theta=\frac{\partial F}{\partial \theta}$
 sont indépendants et forment une base de l'espace tangent à $T$ en $F(\varphi,\theta)$.
 La matrice imposée est bien celle d'un produit scalaire. Comme elle est constante,
 elle dépend de fa\c con $\mathcal{C}^\infty$ de $(\varphi,\theta)$.}
    \item \question{Déterminer la courbure de Gauss $K$ de $T$ avec la métrique $g$.}
\reponse{Comme la matrice de la métrique riemannienne est constante, les symboles de Christoffel
 sont nuls en tous points et le tenseur de courbure est identiquement nul.
 On en déduit que la courbure de Gauss est identiquement nulle.}
    \item \question{Calculer $\int_T K(m)d\sigma(m)$.}
\reponse{D'après la question précédente $\int_T K(m)d\sigma(m)=0$}
\end{enumerate}
}
