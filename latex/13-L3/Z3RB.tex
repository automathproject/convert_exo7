\uuid{Z3RB}
\exo7id{6326}
\auteur{queffelec}
\datecreate{2011-10-16}
\isIndication{false}
\isCorrection{false}
\chapitre{Théorème de Cauchy-Lipschitz}
\sousChapitre{Théorème de Cauchy-Lipschitz}

\contenu{
\texte{
On considère  l'équation différentielle (de Ricatti) sur $\Rr$ :
$$({\cal E})\quad\quad x'(t)=a(t)x^2(t)+b(t)x(t)+c(t),$$
où $a,b,c\in C(\Rr,R)$. Soit $x_i, 1\leq i\leq4$, quatre solutions distinctes
définies sur $I$. On pose $B=\displaystyle{x_3-x_1\over x_3-x_2}
{x_4-x_2\over x_4-x_1}$.
}
\begin{enumerate}
    \item \question{Montrer que $B$ est bien défini sur $I$.}
    \item \question{Montrer que $B$ est une fonction constante sur $I$ (utiliser la dérivée
logarith\-mique).}
\end{enumerate}
}
