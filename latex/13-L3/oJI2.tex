\uuid{oJI2}
\exo7id{2560}
\auteur{tahani}
\organisation{exo7}
\datecreate{2009-04-01}
\isIndication{false}
\isCorrection{true}
\chapitre{Solution maximale}
\sousChapitre{Solution maximale}

\contenu{
\texte{
Soit $f:\mathbb{R}^2
\rightarrow \mathbb{R}$ donn\'ee par
$f(t,x)=4\frac{t^3x}{t^4+x^2}$ si $(t,x)\neq (0,0)$ et $f(0,0)=0$.
On s'interesse \`a l'\'equation diff\'erentielle
$$x'(t)=f(t,x(t)).$$
}
\begin{enumerate}
    \item \question{L'application $f$, est-elle continue ? est-elle localement
lipschitzienne par rapport à sa seconde variable ? Que peut-on en
d\'eduire pour l'\'equation $(2)$ ?}
    \item \question{Soit $\varphi$ une
solution de $(2)$ qui est d\'efinie sur un intervalle $I$ ne
contenant pas $0$. On d\'efinit une application $psi$ par
$\varphi(t)=t^2\psi(t), t\in I$. D\'eterminer une \'equation
diff\'erentielle $(E)$ telle que $\psi$ soit solution de cette
\'equation, puis r\'esoudre cette \'equation $(E)$.}
    \item \question{Que
peut-on en d\'eduire pour l'existence et l'unicit\'e de
l'\'equation diff\'erentielle $(2)$ avec donn\'ee initiale
$(t_0,x_0)=(0,0)$}
\reponse{
$|f(t,x)|=|2t|.|\frac{2t^2x=}{(t^2)^2+x^2}| \leq 2|t|
\rightarrow_{(t,x) \rightarrow 0}0=f(0,0)$. $f$ est donc continue
en $(0,0)$. $f$ n'est pas localement lipschitzienne au voisinage
de $(0,0)$ car sinon il existerait $k, \alpha, \beta \in
\mathbb{R}$ tels que $t\in ]-\alpha,\alpha[$, $x\in
]-\beta,\beta[$ et
$$|f(t,x)-f(t,0)| \leq k|x-0|$$
D'o\`u $\frac{4t^3x}{t^4+x^2} \leq kx \Rightarrow
\frac{4t^3}{t^4+x^2} \leq k \rightarrow \frac{4}{t} \leq k,
\forall t \in ]0,\alpha[$ ce qui est absurde. Nous ne pouvons pas
appliquer Cauchy-Lipschitz.
$(\varphi,I)$ solution de $(2)$
avec $0\not \in I$, $$\psi(t)=t^{-2}\varphi(t) \Rightarrow
\psi'(t)=t^{-2}\varphi'(t)-2t^{-3}\varphi(t)$$
$$\psi'(t)=4t^{-2}\frac{t^3\varphi(t)}{t^4+\varphi^2(t)}-2t^{-1}\psi(t)$$
d'o\`u en exprimant tout en fonction de $\psi$:
$$\frac{\psi'(t)(1+\psi^2(t))}{\psi(t)(1-\psi(t))(1+\psi(t))}=\frac{2}{t}$$

Or $\frac{1+\psi^2(t)}{\psi(t)(1-\psi(t))(1+\psi(t))}=
\frac{1}{\psi(t)}+\frac{1}{1-\psi(t)}-\frac{1}{1+\psi(t)}$ d'o\`u
$$\psi'(t)(\frac{1}{\psi(t)}+\frac{1}{1-\psi(t)}-\frac{1}{1+\psi(t)})=\frac{2}{t}$$
En int\'egrant par rapport \`a $t$ on obtient:
$$ln|\frac{\psi(t)}{1-\psi^2(t)}|=ln(t^2)+c$$
d'o\`u $$\frac{\psi(t)}{1-\psi(t)}=ct^2.$$ $\psi(t)$ v\'erifie est
donc une racine de l'\'equation $$ct^2\psi^2(t)+\psi(t)-ct^2=0$$
et donc $$\psi(t)=\frac{-1 \pm \sqrt{1+4c^2t^4}}{2ct^2}$$ d'o\`u
$\varphi=\frac{-1 \pm \sqrt{1+4c^2t^4}}{2c}$.
}
\end{enumerate}
}
