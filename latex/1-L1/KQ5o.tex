\uuid{KQ5o}
\exo7id{3188}
\auteur{quercia}
\organisation{exo7}
\datecreate{2010-03-08}
\isIndication{false}
\isCorrection{true}
\chapitre{Polynôme, fraction rationnelle}
\sousChapitre{Autre}

\contenu{
\texte{
D{\'e}terminer tous les polyn{\^o}mes $P$ tels que $P(\C)\subset\R$
puis tels que $P(\Q )\subset\Q$ et enfin tels que $P(\Q )=\Q$.
}
\reponse{
Tout polyn{\^o}me {\`a} coefficients complexes non constant est surjectif sur~$\C$
donc $P(\C)\subset\R \Leftrightarrow P = a$ (constante r{\'e}elle).

On a par interpolation de Lagrange~: $P(\Q)\subset\Q \Leftrightarrow P\in\Q[X]$.

Montrons que $P(\Q)=\Q \Leftrightarrow P=aX+b$ avec $a\in\Q^*$, $b\in\Q$~: la condition
est clairement suffisante. Pour prouver qu'elle est n{\'e}cessaire, consid{\'e}rons
un polyn{\^o}me {\'e}ventuel $P$ de degr{\'e} $n\ge 2$ tel que $P(\Q)=\Q$.
On sait d{\'e}j{\`a} que $P$ est {\`a} coefficients rationnels, donc on peut
l'{\'e}crire sous la forme~: $P = \frac{a_0+a_1X+\dots+a_nX^n}d$
avec $a_i\in\Z$, $a_n\ne 0$ et $d\in\N^*$. Soit $\pi$ un nombre premier ne divisant
ni $a_n$ ni $d$, et $x=p/q$ (forme irr{\'e}ductible) un rationnel tel que
$P(x) = 1/\pi$. On a donc~:
$\pi(a_0q^n+\dots+a_np^n) = dq^n$ ce qui implique que $\pi$ divise $q$.
Il vient alors~: $a_np^n = dq^n/\pi - a_0q^n - \dots -a_{n-1}qp^{n-1}$ ce qui
est impossible puisque $\pi$ est facteur du second membre ($n\ge 2$) mais
pas du premier ($p\wedge q = 1$).
}
}
