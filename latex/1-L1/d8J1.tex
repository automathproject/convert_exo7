\uuid{d8J1}
\exo7id{3389}
\auteur{quercia}
\datecreate{2010-03-09}
\isIndication{false}
\isCorrection{true}
\chapitre{Matrice}
\sousChapitre{Autre}

\contenu{
\texte{
Soit $A = (a_{ij}) \in \mathcal{M}_n(\R)$ avec : $\forall\ i,j,\ a_{ij} > 0$.
On munit $\mathcal{M}_{n,1}(\R)$ de la relation d'ordre :
$$(X \ge Y) \iff (\forall\ i,\ x_i \ge y_i),$$
et on pose pour $X \in \mathcal{M}_{n,1}(\R)$, $X \ge 0$, $X\ne 0$ :
$$\left\{
\begin{array}{lll} R(X)    &=& \text{sup}\{ r \ge 0 \text{ tq } AX \ge rX \},    \cr
         \hfil R &=& \text{sup}\{R(X) \text{ tq } X \ge 0, X \ne 0 \}. \cr
         \end{array}\right.$$
}
\begin{enumerate}
    \item \question{Montrer que $R$ est fini et qu'il existe $X_0 \in \R^n$ tel que
    $R(X_0) = R$.}
\reponse{Compacité.}
    \item \question{Montrer que toutes les coordonnées de $X_0$ sont strictement positives.}
\reponse{Si $x_1 = 0$, on pose $Y = \begin{pmatrix}\alpha \cr x_2 \cr \vdots \cr x_n \cr\end{pmatrix}$ :

    $R(Y) \ge \text{min}\Big( a_{11} + \frac {a_{12}x_2 + \dots + a_{1n}x_n}{\alpha},
    \frac {\alpha a_{21}}{x_2} + R(X_0), \dots, \frac {\alpha a_{n1}}{x_n} + R(X_0) \Big)
    > R(X_0)$ pour $\alpha > 0$ assez petit.}
    \item \question{On pose $AX_0 = RX_0 + Y$. Montrer que $Y = 0$.}
\reponse{Si $y_1 > 0$, on pose $X = X_0 + \begin{pmatrix}\alpha\cr0\cr\vdots\cr0\cr\end{pmatrix}$ :

    $AX - RX = Y + \alpha\begin{pmatrix}a_{11}-R\cr a_{21}\cr\vdots\cr a_{n1}\cr\end{pmatrix}$,
    donc pour $\alpha > 0$ assez petit, $R(X) > R$.}
    \item \question{Soit $\lambda$ une valeur propre complexe de $A$.
    Montrer que $|\lambda| \le R$, et $(|\lambda| = R) \Leftrightarrow (\lambda = R)$.}
\reponse{Inégalité triangulaire.}
\end{enumerate}
}
