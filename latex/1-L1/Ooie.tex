\uuid{Ooie}
\exo7id{5129}
\auteur{rouget}
\datecreate{2010-06-30}
\isIndication{false}
\isCorrection{true}
\chapitre{Nombres complexes}
\sousChapitre{Forme cartésienne, forme polaire}

\contenu{
\texte{
Forme trigonométrique de $\frac{1+\cos\theta-i\sin\theta}{1-\cos\theta+i\sin\theta}$ et de
$\frac{1+e^{i\theta}}{1-e^{i\theta}}$.
}
\reponse{
Soit $\theta\in\Rr$.

$$1-\cos\theta+i\sin\theta=0\Leftrightarrow\cos\theta=1\;\mbox{et}\;\sin\theta=0\Leftrightarrow\theta\in0+2\pi\Zz.$$
Donc, $\frac{1+\cos\theta-i\sin\theta}{1-\cos\theta+i\sin\theta}$ existe pour $\theta\notin 2\pi\Zz$. Pour un tel
$\theta$,

$$\frac{1+\cos\theta-i\sin\theta}{1-\cos\theta+i\sin\theta}=\frac{2\cos^2\frac{\theta}{2}-2i\sin\frac{\theta}{2}\cos
\frac{\theta}{2}}{2\sin^2\frac{\theta}{2}+2i\sin\frac{\theta}{2}\cos\frac{\theta}{2}}
=\frac{\cos\frac{\theta}{2}}{\sin\frac{\theta}{2}}\frac{\cos(\theta/2)-i\sin(\theta/2)}{\sin(\theta/2)+i\cos(\theta/2)}
=\frac{\cos\frac{\theta}{2}}{\sin\frac{\theta}{2}}\frac{e^{-i\theta/2}}{e^{i(\pi-\theta)/2}}=-i\cotan\left(\frac{\theta}{2}\right).$$
\textbf{- 1er cas.}
$\cotan\frac{\theta}{2}>0\Leftrightarrow\frac{\theta}{2}\in\displaystyle\bigcup_{k\in\Zz}]k\pi,\frac{\pi}{2}+k\pi[\Leftrightarrow\theta\in
\displaystyle\bigcup_{k\in\Zz}]2k\pi,\pi+2k\pi[$.
Dans ce cas, la forme trigonométrique de $\frac{1+\cos\theta-i\sin\theta}{1-\cos\theta+i\sin\theta}$ est 
$\cotan(\frac{\theta}{2})e^{-i\pi/2}$ (module$=\cotan(\frac{\theta}{2})$ et argument$=-\frac{\pi}{2}\;(2\pi)$).

$$\frac{1+\cos\theta-i\sin\theta}{1-\cos\theta+i\sin\theta}=\left[\cotan\left(\frac{\theta}{2}\right),-\frac{\pi}{2}\right].$$
\textbf{- 2ème cas.} $\cotan\frac{\theta}{2}<0\Leftrightarrow\theta\in
\displaystyle\bigcup_{k\in\Zz}]\pi+2k\pi,2(k+1)\pi[$.
Dans ce cas,

$$\frac{1+\cos\theta-i\sin\theta}{1-\cos\theta+i\sin\theta}=-\cotan(\frac{\theta}{2}).e^{i\pi/2}
=|\cotan(\frac{\theta}{2})|e^{i\pi/2},$$
et donc,

$$\frac{1+\cos\theta-i\sin\theta}{1-\cos\theta+i\sin\theta}=\left[-\cotan\left(\frac{\theta}{2}\right),\frac{\pi}{2}\right].$$
\textbf{- 3ème cas.} $\cotan\frac{\theta}{2}=0\Leftrightarrow\theta\in\pi+2\pi\Zz$. Dans ce cas, on
a $\frac{1+\cos\theta-i\sin\theta}{1-\cos\theta+i\sin\theta}=0$.
Pour $\theta\notin2\pi\Zz$, on a

$$\frac{1+e^{i\theta}}{1-e^{i\theta}}=\frac{e^{i\theta/2}(e^{-i\theta/2}+e^{i\theta/2})}{e^{i\theta/2}(e^{-i\theta/2}-
e^{i\theta/2})}=\frac{2\cos\frac{\theta}{2}}{-2i\sin\frac{\theta}{2}}=i\cotan\frac{\theta}{2}.$$
Si $\theta\in\displaystyle\bigcup_{k\in\Zz}]2k\pi,\pi+2k\pi[$,
$\frac{1+e^{i\theta}}{1-e^{i\theta}}=[\cotan\frac{\theta}{2},\frac{\pi}{2}]$.
Si $\theta\in\displaystyle\bigcup_{k\in\Zz}]\pi+2k\pi,2(k+1)\pi[$,
$\frac{1+e^{i\theta}}{1-e^{i\theta}}=[-\cotan\frac{\theta}{2},-\frac{\pi}{2}]$.
Si $\theta\in\pi+2\pi\Zz$, $\frac{1+e^{i\theta}}{1-e^{i\theta}}=0$.
}
}
