\uuid{N8I9}
\exo7id{5114}
\auteur{rouget}
\organisation{exo7}
\datecreate{2010-06-30}
\isIndication{false}
\isCorrection{true}
\chapitre{Injection, surjection, bijection}
\sousChapitre{Injection, surjection}

\contenu{
\texte{
Montrer que les assertions suivantes sont équivalentes ($f$ est une application d'un ensemble $E$ dans lui-même)~:
}
\begin{enumerate}
    \item \question{$f$ est injective.}
    \item \question{$\forall X\in\mathcal{P}(E),\;f^{-1}(f(X))=X$.}
    \item \question{$\forall(X,Y)\in\mathcal{P}(E)^2,\;f(X\cap Y)=f(X)\cap f(Y)$.}
    \item \question{$\forall(X,Y)\in\mathcal{P}(E)^2,\;X\cap Y=\varnothing\Rightarrow f(X)\cap f(Y)=\varnothing$.}
    \item \question{$\forall(X,Y)\in\mathcal{P}(E)^2,\;Y\subset X\Rightarrow f(X\setminus Y)=f(X)\setminus f(Y)$.}
\reponse{
\textbf{1) $\Rightarrow$ 2)} Soit $X\in\mathcal{P}(E)$. On a toujours $X\subset f^{-1}(f(X))$. (En effet, pou $x\in E$, $x\in
X\Rightarrow f(x)\in f(X)\Rightarrow x\in f^{-1}(f(X))$).
Réciproquement, soit $x\in E$.

\begin{align*}
x\in f^{-1}(f(X))&\Rightarrow f(x)\in f(X)\Rightarrow\exists x'\in X/\;f(x)=f(x')\Rightarrow\exists x'\in
X/\;x=x'\;(\mbox{puisque}\;f\;\mbox{est injective})\\
 &\Rightarrow x\in X.
\end{align*}
Finalement, $f^{-1}(f(X))\subset X$ et donc $f^{-1}(f(X))=X$.
\textbf{2) $\Rightarrow$ 1)} Soit $x\in X$. Par hypothése, $f^{-1}\{f(x)\}=f^{-1}(f(\{x\}))=\{x\}$ ce qui signifie que
$f(x)$ a un et un seul antécédent à savoir $x$. Par suite, tout élément de l'ensemble d'arrivée a au plus un antécédent
par $f$ et $f$ est injective.

\textbf{1) $\Rightarrow$ 3)} Soit $(X,Y)\in(\mathcal{P}(E))^2$. On a toujours $f(X\cap Y)\subset f(X)\cap
f(Y)$ ($X\cap Y\subset X\Rightarrow f(X\cap Y)\subset f(X)$ et de même, $f(X\cap Y)\subset 
f(Y)$ et finalement, $f(X\cap Y)\subset f(X)\cap f(Y)$).
Réciproquement, soit $y\in F$. $y\in f(X)\cap f(Y)\Rightarrow\exists(x,x')\in X\times Y/\;y=f(x)=f(x')$. Mais alors,
puisque $f$ est injective, $x=x'\in X\cap Y$ puis $y=f(x)\in f(X\cap Y)$. Finalement, $f(X\cap Y)=f(X)\cap f(Y)$.

\textbf{3) $\Rightarrow$ 4)} Soit $(X,Y)\in(\mathcal{P}(E))^2$. $X\cap Y=\varnothing\Rightarrow f(X)\cap f(Y)=f(X\cap
Y)=f(\varnothing)=\varnothing$.

\textbf{4) $\Rightarrow$ 5)} Soit $(X,Y)\in(\mathcal{P}(E))^2$ tel que $Y\subset X$.
Puisque $X\setminus Y\subset X$, on a $f(X\setminus Y)\subset f(X)$. Mais, puisque $Y\cap(X\setminus Y)=\varnothing$, par
hypothèse $f(X\setminus Y)\cap f(Y)=\varnothing$. Finalement,$f(X\setminus Y)\subset f(X)\setminus f(Y)$.
Inversement, si $f(X)\setminus f(Y)=\varnothing$, l'inclusion contraire est immédiate et si $f(X)\setminus
f(Y)\neq\varnothing$, un élémént de $f(X)\setminus f(Y)$ est l'image d'un certain élément de $X$ qui ne peut être dans $Y$
et donc est l'image d'un élément de $X\setminus Y$ ce qui montre que $f(X)\setminus f(Y)\subset f(X\setminus Y)$ et
finalement que $f(X)\setminus f(Y)=f(X\setminus Y)$.

\textbf{5) $\Rightarrow$ 1)} Soit $(x_1,x_2)\in E^2$ tel que $x_1\neq x_2$. Posons $X=\{x_1,x_2\}$ et $Y=\{x_2\}$.
On a
donc $Y\subset X$. Par hypothèse $f(X\setminus Y)=f(X)\setminus f(Y)$ ce qui fournit
$f(\{x_1\})=f(\{x_1,x_2\})\setminus f(\{x_2\})$ ou encore, $\{f(x_1)\}=\{f(x_1),f(x_2)\}\setminus\{f(x_2)\}$.
Maintenant, si $f(x_1)=f(x_2)$ alors $\{f(x_1),f(x_2)\}\setminus\{f(x_2)\}=\varnothing$ (et pas $\{f(x_1)\}$). Donc
$f(x_1)\neq f(x_2)$.
On a montré que $f$ est injective.
}
\end{enumerate}
}
