\uuid{WEp4}
\exo7id{941}
\auteur{legall}
\datecreate{1998-09-01}
\isIndication{false}
\isCorrection{true}
\chapitre{Application linéaire}
\sousChapitre{Image et noyau, théorème du rang}

\contenu{
\texte{
Soient  $E$  un espace vectoriel et  $\varphi $
une application lin\' eaire de  $E$  dans  $E$. On suppose que
$\hbox {Ker } (\varphi ) \cap \hbox{Im }(\varphi )=\{ 0\}$.
Montrer que, si  $x \not \in \hbox {Ker } (\varphi )$  alors, pour
tout  $n\in { \Nn }  : \varphi ^n(x)\not = 0$.
}
\reponse{
Montrons ceci par r\'ecurence : Pour $n=1$, l'assertion est triviale
: $x\notin \ker \varphi \Rightarrow \varphi(x) \neq 0$. Supposons
que si $x \notin \ker \varphi$ alors  $\varphi^{n-1}(x) \neq 0$,
 ($n \geqslant 2$).
Fixons $x \notin \ker \varphi$,
 Alors par hypoth\`eses de r\'ecurrence
$\varphi^{n-1}(x) \neq 0$, mais $\varphi^{n-1}(x) = \varphi(
\varphi^{n-2}(x)) \in \mathop{\mathrm{Im}}\nolimits \varphi$ donc $\varphi^{n-1}(x) \notin
\ker \varphi$ gr\^ace \`a l'hypoth\`ese sur $\varphi$. Ainsi
$\varphi(\varphi^{n-1}(x)) \neq 0$, soit $\varphi^{n}(x)\neq 0$.
Ce qui termine la r\'ecurrence.
}
}
