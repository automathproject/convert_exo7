\uuid{rfjJ}
\exo7id{400}
\auteur{bodin}
\datecreate{1998-09-01}
\isIndication{false}
\isCorrection{false}
\chapitre{Polynôme, fraction rationnelle}
\sousChapitre{Racine, décomposition en facteurs irréductibles}

\contenu{
\texte{
Soit $P=(X^2-X+1)^2+1$.
}
\begin{enumerate}
    \item \question{V\'erifier que $i $ est racine de $P$.}
    \item \question{En d\'eduire alors la d\'ecomposition en produit de facteurs
     irr\'eductibles de $P$ sur ${\R}[X]$}
    \item \question{Factoriser sur ${\C}[X]$ et sur ${\R}[X]$ les polyn\^omes suivants en
produit de polyn\^omes irr\'eductibles :
    $P=X^4+X^2+1,\  Q=X^{2n}+1,\ R=X^6-X^5+X^4-X^3+X^2-X+1$,
 $S=X^5-13X^4+67X^3-171 X^2+216X-108$ (on cherchera les racines doubles de $S$).}
\end{enumerate}
}
