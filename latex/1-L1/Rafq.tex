\uuid{Rafq}
\exo7id{1061}
\auteur{ridde}
\organisation{exo7}
\datecreate{1999-11-01}
\isIndication{true}
\isCorrection{true}
\chapitre{Matrice}
\sousChapitre{Propriétés élémentaires, généralités}

\contenu{
\texte{
Soit $A(\theta) = \begin{pmatrix} \cos \theta & -\sin \theta \\ \sin \theta
& \cos \theta \end{pmatrix}$ pour $\theta \in \Rr$. Calculer $A(\theta) \times A(\theta')$ et $\big(A(\theta)\big)^n$ pour
$n \ge 1$.
}
\indication{Il faut connaître les formules de $\cos(\theta+\theta')$ et $\sin(\theta+\theta')$.}
\reponse{
\begin{align*}
A(\theta)\times A(\theta')  
  & =  \begin{pmatrix} \cos \theta & -\sin \theta \\ \sin \theta
& \cos \theta \end{pmatrix} \times \begin{pmatrix} \cos \theta' & -\sin \theta' \\ \sin \theta'
& \cos \theta' \end{pmatrix} \\
  & = \begin{pmatrix} 
\cos \theta\cos \theta' -\sin \theta \sin\theta' & - \cos \theta \sin \theta' - \sin \theta \cos \theta' \\
 \sin \theta \cos \theta'+\cos \theta \sin \theta'   & -\sin \theta \sin\theta'+\cos \theta\cos \theta'  \\
 \end{pmatrix} \\
  & =
\begin{pmatrix} \cos (\theta+\theta') & -\sin (\theta+\theta')  \\ \sin  (\theta+\theta')
& \cos  (\theta+\theta') \end{pmatrix} \\
  & = A(\theta+\theta') \\
\end{align*}

Bilan : $A(\theta)\times A(\theta') = A(\theta+\theta')$.

\bigskip

Nous allons montrer par récurrence sur $n\ge 1$ que $\big(A(\theta)\big)^n = A(n\theta)$.

\begin{itemize}
  \item C'est bien sûr vrai pour $n=1$.
  \item Fixons $n\ge 1$ et supposons que $\big(A(\theta)\big)^n = A(n\theta)$ alors
$$\big(A(\theta)\big)^{n+1} = \big(A(\theta)\big)^n \times A(\theta) = A(n\theta) \times A(\theta) = A(n\theta+\theta) = A((n+1)\theta)$$
  \item C'est donc vrai pour tout $n\ge 1$.
\end{itemize}


Remarques :
\begin{itemize}
  \item On aurait aussi la formule $A(\theta')\times A(\theta) = A(\theta+\theta') = A(\theta)\times A(\theta')$.
Les matrices $A(\theta)$ et $A(\theta')$ commutent.

  \item En fait il n'est pas plus difficile de montrer que $\big(A(\theta)\big)^{-1}=A(-\theta)$.
On sait aussi que par définition $\big(A(\theta)\big)^{0}=I$. Et on en déduit que pour $n\in \Zz$ on a 
$\big(A(\theta)\big)^n = A(n\theta)$.

  \item En terme géométrique $A(\theta)$ est la matrice de la rotation d'angle $\theta$ (centrée à l'origine).
On vient de montrer que si l'on compose un rotation d'angle $\theta$ avec un rotation d'angle $\theta'$
alors on obtient une rotation d'angle $\theta+\theta'$.
\end{itemize}
}
}
