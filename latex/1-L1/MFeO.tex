\uuid{MFeO}
\exo7id{36}
\auteur{cousquer}
\organisation{exo7}
\datecreate{2003-10-01}
\isIndication{false}
\isCorrection{true}
\chapitre{Nombres complexes}
\sousChapitre{Racine carrée, équation du second degré}

\contenu{
\texte{
R\'esoudre dans $\mathbb{C}$ les \'equations suivantes :
}
\begin{enumerate}
    \item \question{$z^2-(11-5i)z+24-27i=0$.}
\reponse{$\Delta =-2i$ dont les racines carr\'ees sont $1-i$ et $-1+i$, d'o\`u les
racines $z_1=5-2i$ et $z_2=6-3i$.}
    \item \question{$z^3+3z-2i=0$.}
\reponse{Une racine ``\'evidente'' $z_1=i$, d'o\`u la r\'esolution compl\`ete en effectuant
la division par $z-i$. On trouve $z_2=i$
et $z_3=-2i$.}
\end{enumerate}
}
