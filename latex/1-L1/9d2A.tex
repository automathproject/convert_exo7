\uuid{9d2A}
\exo7id{307}
\auteur{cousquer}
\datecreate{2003-10-01}
\isIndication{false}
\isCorrection{false}
\chapitre{Arithmétique dans Z}
\sousChapitre{Pgcd, ppcm, algorithme d'Euclide}

\contenu{
\texte{
Soit $m$ et $n$ deux entiers positifs.
}
\begin{enumerate}
    \item \question{Si $\mbox{pgcd}(m,4)=2$ et $\mbox{pgcd}(n,4)=2$, montrer que $\mbox{pgcd}(m+n,4)=4$.}
    \item \question{Montrer que pour chaque entier $n$, $6$ divise $n^3-n$.}
    \item \question{Montrer que pour chaque entier $n$, $30$ divise $n^5-n$.}
    \item \question{Montrer que si $m$ et $n$ sont des entiers impairs, 
$m^2+n^2$ est pair mais non divisible par~$4$.}
    \item \question{Montrer que le produit de quatre entiers consécutifs est 
divisible par~$24$.}
    \item \question{Montrer que si  $\mbox{pgcd}(a,b)=1$, alors
\begin{itemize}}
    \item \question{$\mbox{pgcd}(a+b,a-b)\in \{1,2\}$,}
    \item \question{$\mbox{pgcd}(2a+b,a+2b)\in \{1,3\}$,}
    \item \question{$\mbox{pgcd}(a^2+b^2,a+b)\in \{1,2\}$,}
    \item \question{$\mbox{pgcd}(a+b,a^2-3ab+b^2)\in \{1,5\}$.
\end{itemize}}
\end{enumerate}
}
