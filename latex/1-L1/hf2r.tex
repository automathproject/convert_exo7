\uuid{hf2r}
\exo7id{5173}
\auteur{rouget}
\datecreate{2010-06-30}
\isIndication{false}
\isCorrection{true}
\chapitre{Espace vectoriel}
\sousChapitre{Définition, sous-espace}

\contenu{
\texte{
Dans $E=\Rr^4$, on considère $V=\{(x,y,z,t)\in E/\;x-2y=0\;\mbox{et}\;y-2z=0\}$ et $W=\{(x,y,z,t)\in E/\;x+z=y+t\}$.
}
\begin{enumerate}
    \item \question{Montrer que $V$ et $W$ sont des sous espaces vectoriels de $E$.}
\reponse{Pour $(x,y,z,t)\in\Rr^4$, on pose $f((x,y,z,t))=x-2y$, $g((x,y,z,t))=y-2z$ et $h((x,y,z,t))=x-y+z-t$. $f$,
$g$ et $h$ sont des formes linéaires sur $\Rr^4$. Donc, $V=\mbox{Ker}f\cap\mbox{Ker}g$ est un sous-espace vectoriel de
$\Rr^4$ en tant qu'intersection de sous-espaces vectoriels de $\Rr^4$ et $W=\mbox{Ker}h$ est un sous-espace vectoriel
de $\Rr^4$.}
    \item \question{Donner une base de $V$, $W$ et $V\cap W$.}
\reponse{Soit $(x,y,z,t)\in\Rr^4$.

\begin{align*}
(x,y,z,t)\in V\Leftrightarrow
\left\{
\begin{array}{l}
x=2y\\
y=2z
\end{array}
\right.\Leftrightarrow
\Leftrightarrow
\left\{
\begin{array}{l}
x=4z\\
y=2z
\end{array}
\right.
.
\end{align*}

Donc, $V=\{(4z,2z,z,t),\;(z,t)\in\Rr^2\}=\mbox{Vect}(e_1,e_2)$ où $e_1=(4,2,1,0)$ et $e_2=(0,0,0,1)$. Montrons alors
que $(e_1,e_2)$ est libre. Soit $(z,t)\in\Rr^2$.

$$ze_1+te_2=0\Rightarrow(4z,2z,z,t)=(0,0,0,0)\Rightarrow z=t=0.$$

Donc, $(e_1,e_2)$ est une base de $V$.

Pour $(x,y,z,t)\in\Rr^4$, $(x,y,z,t)\in W\Leftrightarrow t=x-y+z$. Donc,
$W=\{(x,y,z,x-y+z),\;(x,y,z)\in\Rr^3\}=\mbox{Vect}(e_1',e_2',e_3')$ où $e_1'=(1,0,0,1)$, $e_2'=(0,1,0,-1)$ et
$e_3'=(0,0,1,1)$.

Montrons alors que $(e_1',e_2',e_3')$ est libre. Soit $(x,y,z)\in\Rr^3$.

$$xe_1'+ye_2'+ze_3'=0\Rightarrow(x,y,z,x-y+z)=(0,0,0,0)\Rightarrow x=y=z=0.$$

Donc, $(e_1',e_2',e_3')$ est une base de $W$.
Soit $(x,y,z,t)\in\Rr^4$.

$$(x,y,z,t)\in V\cap W\Leftrightarrow
\left\{
\begin{array}{l}
x=2y\\
y=2z\\
x-y+z-t=0
\end{array}
\right.\Leftrightarrow
\left\{
\begin{array}{l}
x=4z\\
y=2z\\
t=3z
\end{array}
\right..$$

Donc, $V\cap W=\{(4z,2z,z,3z),\;z\in\Rr\}=\mbox{Vect}(e)$ où $e=(4,2,1,3)$. De plus, $e$ étant non nul, la famille
$(e)$ est libre et est donc une base de $V\cap W$.}
    \item \question{Montrer que $E=V+W$.}
\reponse{Soit $u=(x,y,z,t)$ un vecteur de $\Rr^4$.

On cherche $v=(4\alpha,2\alpha,\alpha,\beta)\in V$ et $w=(a,b,c,a-b+c)\in W$ tels que $u=v+w$.

$$u=v+w\Leftrightarrow\left\{
\begin{array}{l}
4\alpha+a=x\\
2\alpha+b=y\\
\alpha+c=z\\
\beta+a-b+c=t
\end{array}
\right.\Leftrightarrow
\left\{
\begin{array}{l}
a=x-4\alpha\\
b=y-2\alpha\\
c=z-\alpha\\
\beta=-x+y-z+t-3\alpha
\end{array}
\right..$$

et $\alpha=0$, $\beta=-x+y-z+t$, $a=x$, $b=y$ et $c=z$ conviennent. Donc, $\forall u\in\Rr^4,\;\exists(v,w)\in
V\times W/\;u=v+w$. On a montré que $\Rr^4=V+W$.}
\end{enumerate}
}
