\uuid{hgSB}
\exo7id{3343}
\auteur{quercia}
\organisation{exo7}
\datecreate{2010-03-09}
\isIndication{false}
\isCorrection{true}
\chapitre{Application linéaire}
\sousChapitre{Morphismes particuliers}

\contenu{
\texte{
Un endomorphisme $f \in \mathcal{L}(E)$ est dit {\it nilpotent\/} s'il existe
$p \in \N$ tel que $f^p = 0$. Dans ce cas, {\it l'indice\/} de $f$ est le
plus petit entier $p$ tel que $f^p = 0$.
On considère $f\in\mathcal{L}(E)$ nilpotent d'indice $p$.
}
\begin{enumerate}
    \item \question{Soit $\vec u \in E \setminus \mathrm{Ker} f^{p-1}$. Montrer que la famille
    $\bigl(\vec u, f(\vec u), \dots, f^{p-1}(\vec u) \bigr)$ est libre.}
    \item \question{En déduire que si $E$ est de dimension finie $n$, alors $f^n = 0$.}
    \item \question{Soit $g \in GL(E)$ tel que $f\circ g = g\circ f$.
    Montrer que $f + g \in GL(E) \dots$
  \begin{enumerate}}
    \item \question{en dimension finie.}
    \item \question{pour $E$ quelconque.}
\reponse{
3. (a) $(f+g)(\vec x) = \vec0  \Rightarrow  f^k(\vec x) + g\circ f^{k-1}(\vec x) = \vec0$. \par
             Pour $k = p$   : $f^{k-1}(\vec x) = \vec0$, puis
             pour $k = p-1$ : $f^{k-2}(\vec x) = \vec 0$, etc, jusqu'à $\vec x = \vec 0$.\\
    3. (b) Même principe sur l'équation : $(f+g)(\vec x) = \vec y$.\par
             On obtient : $(f+g)^{-1} = g^{-1} \circ \bigl( \mathrm{id} - g^{-1}\circ f
             + g^{-2}\circ f^2 - \dots + (-1)^{p-1}g^{1-p}\circ f^{p-1} \bigr)$.
}
\end{enumerate}
}
