\uuid{1VJf}
\exo7id{5258}
\auteur{rouget}
\organisation{exo7}
\datecreate{2010-07-04}
\isIndication{false}
\isCorrection{true}
\chapitre{Matrice}
\sousChapitre{Propriétés élémentaires, généralités}

\contenu{
\texte{
Pour $x$ réel, on pose~:

$$A(x)=
\left(
\begin{array}{cc}
\ch x&\sh x\\
\sh x&\ch x
\end{array}
\right)
.$$

Déterminer $(A(x))^n$ pour $x$ réel et $n$ entier relatif.
}
\reponse{
Soient $x$ et $y$ deux réels.

\begin{align*}\ensuremath
A(x)A(y)&=\left(
\begin{array}{cc}
\ch x&\sh x\\
\sh x&\ch x
\end{array}
\right)\left(
\begin{array}{cc}
\ch y&\sh y\\
\sh y&\ch y
\end{array}
\right)=\left(
\begin{array}{cc}
\ch x\ch y+\sh x\sh y&\sh x\ch y+\ch x\sh y\\
\sh x\ch y+\ch x\sh y&\ch x\ch y+\sh x\sh y
\end{array}
\right)\\
 &=\left(
\begin{array}{cc}
\ch(x+y)&\sh(x+y)\\
\sh(x+y)&\ch(x+y)
\end{array}
\right).
\end{align*}

En particulier,

$$A(x)A(-x)=A(-x)A(x)=A(0)=\left(
\begin{array}{cc}
1&0\\
0&1
\end{array}
\right)=I_2,$$

et $A(x)$ est inversible d'inverse $A(-x)$.

On a aussi, pour $n$ entier naturel non nul donné~:

$$(A(x))^n=A(x)A(x)...A(x)=A(x+x...+x)=A(nx),$$

ce qui reste clair pour $n=0$ car $A(x)^0=I_2=A(0)$. Enfin, $(A(x))^{-n}=(A(x)^{-1})^n=A(-x)^n=A(-nx)$. Finalement, 

$$\forall n\in\Zz,\;(A(x))^n=A(nx)=\left(
\begin{array}{cc}
\ch(nx)&\sh(nx)\\
\sh(nx)&\ch(nx)
\end{array}
\right).$$
}
}
