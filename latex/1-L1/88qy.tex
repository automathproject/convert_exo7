\uuid{88qy}
\exo7id{3316}
\auteur{quercia}
\datecreate{2010-03-09}
\isIndication{false}
\isCorrection{true}
\chapitre{Application linéaire}
\sousChapitre{Image et noyau, théorème du rang}

\contenu{
\texte{
Soit $u\in\mathcal{L}(E)$. On pose pour $v\in\mathcal{L}(E)$~: $\varphi(v) = v\circ u - u\circ v$,
et on note $c_i = \mathrm{Ker}\varphi^i$ ($c_0 = \{0\}$, $c_1$ est le commutant
de~$u$, $c_2$ est l'ensemble des $v$ tels que $v\circ u - u\circ v$ commute
avec $u$,\dots).
}
\begin{enumerate}
    \item \question{Calculer $\varphi(v\circ w)$ en fonction de $v,w,\varphi(v)$ et $\varphi(w)$.}
\reponse{$\varphi(v\circ w) = \varphi(v)\circ w + v\circ \varphi(w)$.}
    \item \question{Montrer que $c = \bigcup_{i\in\N}c_i$ est une sous-algèbre de~$\mathcal{L}(E)$.}
\reponse{Par récurrence $\varphi^n(v\circ w) = \sum_{k=0}^n C_n^k\varphi^k(v)\circ\varphi^{n-k}(w)$
donc si $v\in c_p$ et $w\in c_q$ alors $v\circ w\in c_{p+q-1}$.}
\end{enumerate}
}
