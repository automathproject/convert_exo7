\uuid{CUpO}
\exo7id{3342}
\auteur{quercia}
\datecreate{2010-03-09}
\isIndication{false}
\isCorrection{true}
\chapitre{Application linéaire}
\sousChapitre{Morphismes particuliers}

\contenu{
\texte{

}
\begin{enumerate}
    \item \question{Soit $E$ un $ K$-ev et $f,g \in \mathcal{L}(E)$ tels que :
    $\begin{cases} f^2 = 0 \cr f\circ g + g\circ f = \mathrm{id}_E.\end{cases}$ \par
    Montrer que $\mathrm{Ker} f = \Im f$.}
    \item \question{Réciproquement, soit $f \in \mathcal{L}(E)$ tel que $\mathrm{Ker} f = \Im f$, et $F$ un supplémentaire de $\mathrm{Ker} f$.
   Montrer que
  \begin{enumerate}}
    \item \question{$f^2 = 0$.}
    \item \question{$\forall\ \vec x \in E$, il existe $\vec y,\vec z \in F$ uniques tels
       que $\vec x = \vec y + f(\vec z)$.}
    \item \question{Il existe $g \in \mathcal{L}(E)$ tel que $f\circ g + g\circ f = \mathrm{id}_E$.}
\reponse{
2. (c) $g(\vec x) = \vec z$.
}
\end{enumerate}
}
