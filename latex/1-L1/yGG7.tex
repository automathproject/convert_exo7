\uuid{yGG7}
\exo7id{67}
\auteur{cousquer}
\datecreate{2003-10-01}
\isIndication{false}
\isCorrection{true}
\chapitre{Nombres complexes}
\sousChapitre{Géométrie}

\contenu{
\texte{

}
\begin{enumerate}
    \item \question{Soit $A$, $B$, $C$ trois points du plan complexe dont les affixes sont
respectivement $a$, $b$, $c$. On suppose que $a+jb+j^2c=0$ ; montrer que $ABC$
est un triangle \'equilat\'eral ($j$ et $j^2$ sont les racines cubiques complexes
de $1$ --- plus pr\'ecis\'ement $j=\frac{-1+i\sqrt3}{2}$). R\'eciproque ?}
\reponse{R\'eciproque : $a+jb+j^2c=0$ ou $a+j^2b+jc=0$ (cela d\'epend de
l'orientation du triangle).}
    \item \question{$ABC$ \'etant un triangle \'equilat\'eral direct du plan complexe, on construit
les triangles \'equilat\'eraux directs $BOD$ et $OCE$, ce qui d\'etermine les points
$D$ et $E$ ($O$ est l'origine du plan complexe). Quelle est la nature du
quadrilat\`ere $ADOE$ ? Comparer les triangles $OBC$, $DBA$ et $EAC$.}
\reponse{$ADOE$ est un parall\'elogramme. Les trois triangles $OBC$, $DBA$ et $EAC$
sont directement isom\'etriques, ce qui d'ailleurs se v\'erifie imm\'ediatement \`a
l'aide de rotations.}
\end{enumerate}
}
