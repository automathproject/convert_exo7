\uuid{KnPI}
\exo7id{62}
\auteur{cousquer}
\datecreate{2003-10-01}
\isIndication{false}
\isCorrection{false}
\chapitre{Nombres complexes}
\sousChapitre{Géométrie}

\contenu{
\texte{
Le plan $P$ est rapporté à un repère orthonormé et identifié à l'ensemble $\C$ des nombres complexes par
$$
M(x,y) \mapsto x+iy = z,
$$
où $z$ est appelé l'affixe de $M.$
Soit $f : P \mathrm{rg} P$ qui à tout point $M$ d'affixe $z$ associe $M'$ d'affixe $z' = \frac{z-i}{z+i}.$
}
\begin{enumerate}
    \item \question{Sur quel sous ensemble de $P$, $f$ est-elle définie?}
    \item \question{Calculer $|z'|$ pour $z$ affixe d'un point $M$ situé dans le demi plan ouvert 
$$
H := \{ M(x,y) \in P \mid y > 0.\}?
$$}
    \item \question{En déduire l'image par $f$ de $H.$}
\end{enumerate}
}
