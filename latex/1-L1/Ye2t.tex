\uuid{Ye2t}
\exo7id{349}
\auteur{gourio}
\datecreate{2001-09-01}
\isIndication{true}
\isCorrection{true}
\chapitre{Arithmétique dans Z}
\sousChapitre{Nombres premiers, nombres premiers entre eux}

\contenu{
\texte{
Soit $a\in \Nn  $ tel que $a^{n}+1 $ soit premier, montrer que $\exists
 k\in \Nn,n=2^{k}.$
Que penser de la conjecture : $\forall  n\in \Nn,2^{2^{n}}+1$ est
premier ?
}
\indication{Raisonner par contraposition (ou par l'absurde) : supposer que $n$ n'est pas de la forme $2^k$,
alors $n$ admet un facteur irr\'eductible $p>2$.
Utiliser aussi $x^p+1 = (x+1)(1-x+x^2-x^3+\ldots+x^{p-1})$ avec $x$ bien choisi.}
\reponse{
Supposons que $a^n + 1$ est premier. Nous allons montrer la contrapos\'ee. Supposons
que $n$ n'est pas de la forme $2^k$, c'est-\`a-dire que $n=p\times q$ avec
$p$ un nombre premier $>2$ et $q\in\Nn$.
Nous utilisons la formule
$$x^p+1 = (x+1)(1-x+x^2-x^3+\ldots+x^{p-1})$$
avec $x = a^q$ :
$$a^n+1 = a^{pq}+1 = (a^q)^p+1 = (a^q+1)(1-a^q+(a^q)^2+\cdots+(a^q)^{p-1}).$$
Donc $a^q+1$ divise $a^n+1$ et comme $1 < a^q+1 < a^{n}+1$ alors $a^n+1$ n'est pas premier.
Par contraposition si $a^n+1$ est premier alor $n = 2^k$.
Cette conjecture est fausse, mais pas facile \`a v\'erifier
sans une bonne calculette ! En effet pour $n=5$ nous obtenons :
$$2^{2^5} + 1 = 4294967297 = 641 \times 6700417.$$
}
}
