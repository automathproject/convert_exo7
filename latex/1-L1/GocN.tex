\uuid{GocN}
\exo7id{5192}
\auteur{rouget}
\datecreate{2010-06-30}
\isIndication{false}
\isCorrection{true}
\chapitre{Application linéaire}
\sousChapitre{Image et noyau, théorème du rang}

\contenu{
\texte{
\label{exo:suprou10}
Soient $\Kk$ un sous-corps de $\Cc$, $E$ un $\Kk$-espace vectoriel de dimension finie $n$ et $f$ un endomorphisme de $E$ non injectif. Pour $k$
entier naturel donné, on pose $N_k=\mbox{Ker}f^k$ et $I_k=\mbox{Im}f^k$ (avec la convention $f^0=Id_E$).
}
\begin{enumerate}
    \item \question{Montrer que~:~$\forall k\in\Nn,\;(N_k\subset N_{k+1}\;\mbox{et}\;I_{k+1}\subset I_k)$.}
\reponse{Soient $k$ un entier naturel et $x$ un élément de $E$.

$$x\in N_k\Rightarrow f^k(x)=0\Rightarrow f(f^k(x))=f(0)=0\Rightarrow f^{k+1}(x)=0\Rightarrow x\in N_{k+1}.$$
On a montré que~:~$\forall k\in\Nn,\;N_k\subset N_{k+1}$. Ensuite,

$$x\in I_{k+1}\Rightarrow\exists y\in E/\;x=f^{k+1}(y)\Rightarrow\exists z(=f(y))\in E/\;x=f^k(z)\Rightarrow x\in I_k.$$
On a montré que~:~$\forall k\in\Nn,\;I_{k+1}\subset I_k$.}
    \item \question{\begin{enumerate}}
\reponse{\begin{enumerate}}
    \item \question{Montrer que~:~$(\forall k\in\Nn,\;(N_k=N_{k+1}\Rightarrow N_{k+1}=N_{k+2})$.}
\reponse{Soit $k$ un entier naturel. Supposons que $N_k=N_{k+1}$.

On a déjà $N_{k+1}\subset N_{k+2}$. Montrons que $N_{k+2}\subset N_{k+1}$.

Soit $x$ un élément de $E$.

\begin{align*}
x\in N_{k+2}&\Rightarrow f^{k+2}(x)=0\Rightarrow f^{k+1}(f(x))=0\Rightarrow f(x)\in N_{k+1}=N_k
\Rightarrow f^k(f(x))=0\\
 &\Rightarrow f^{k+1}(x)=0\Rightarrow x\in N_{k+1}.
\end{align*}}
    \item \question{Montrer que~:~$\exists p\in\Nn/\;\forall k\in\Nn,\;(k<p\Rightarrow N_k\neq N_{k+1}\;\mbox{et}\;k\geq
p\Rightarrow N_k=N_{k+1})$.}
\reponse{On a $\{0\}=N_0\subset N_1\subset N_2...$ Supposons que chacune de ces inclusions soient strictes. Alors,
$0=\mbox{dim }N_0<\mbox{dim }N_1<\mbox{dim }N_2$... Donc $\mbox{dim }N_1\geq1$, $\mbox{dim }N_2\geq2$ et par une récurrence
facile, $\forall k\in\Nn,\;\mbox{dim }N_k\geq k$. En particulier, $\mbox{dim }N_{n+1}\geq n+1>n=\mbox{dim }E$, ce qui est
impossible. Donc, il existe $k$ entier naturel tel que $N_k=N_{k+1}$.

Soit $p$ le plus petit de ces entiers $k$ (l'existence de $p$ est démontrée proprement de la façon suivante~:~si
$K=\in\{k\in\Nn/\;N_k=N_{k+1}\}$, $K$ est une partie non vide de $\Nn$ et admet donc un plus petit élément). On note
que puisque $f$ est non injectif, $\{0\}=N_0\underset{\neq}{\subset}N_1$ et donc $p\in\Nn^*$. Par définition de $p$,
pour $k<p$, $N_k\underset{\neq}{\subset}N_{k+1}$ et, d'après le a) et puisque $N_p=N_{p+1}$, on montre par récurrence
que pour $k=p$, on a $N_k=N_p$.}
    \item \question{Montrer que $p\leq n$.}
\reponse{$0<\mbox{dim }N_1<...<\mbox{dim }N_p$ montre que pour $k\leq p$, on a $\mbox{dim }N_k=k$ et en particulier $p
\leq\mbox{dim }N_p=n$.}
\end{enumerate}
}
