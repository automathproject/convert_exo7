\uuid{MrZq}
\exo7id{5073}
\auteur{rouget}
\organisation{exo7}
\datecreate{2010-06-30}
\isIndication{false}
\isCorrection{true}
\chapitre{Nombres complexes}
\sousChapitre{Trigonométrie}

\contenu{
\texte{
On veut calculer $S=\tan9^\circ-\tan27^\circ-\tan63^\circ+\tan81^\circ$.\\
}
\begin{enumerate}
    \item \question{Calculer $\tan(5x)$ en fonction de $\tan x$.}
\reponse{Pour $x\notin\frac{\pi}{10}+\frac{\pi}{5}\Zz$,

$$\tan(5x)=\frac{\Im((e^{ix})^5)}{\Re((e^{ix})^5)}=\frac{5\cos^4x\sin
x-10\cos^2x\sin^3x+\sin^5x}{\cos^5x-10\cos^3x\sin^2x+5\cos x\sin^4x}=\frac{5\tan
x-10\tan^3x+\tan^5x}{1-10\tan^2x+5\tan^4x},$$
après division du numérateur et du dénominateur par le réel non nul $\cos^5x$.

\begin{center}
\shadowbox{
$\forall x\in\Rr\setminus\left(\frac{\pi}{10}+\frac{\pi}{5}\Zz\right),\;\tan(5x)=\frac{5\tan
x-10\tan^3x+\tan^5x}{1-10\tan^2x+5\tan^4x}.$
}
\end{center}}
    \item \question{En déduire un polynôme de degré $4$ dont les racines sont $\tan9^\circ$, $-\tan27^\circ$, $-\tan63^\circ$ et
$\tan81^\circ$ puis la valeur de $S$.}
\reponse{$9^\circ$, $-27^\circ$, $-63^\circ$ et $81^\circ$ vérifient
$\tan(5\times9^\circ)=\tan(5\times(-27^\circ))=\tan(5\times(-63^\circ))=\tan(5\times81^\circ)=1$. On résoud donc l'équation~:

$$\tan(5x)=1\Leftrightarrow5x\in\left(\frac{\pi}{4}+\pi\Zz\right)\Leftrightarrow x\in\left(\frac{\pi}{20}+\frac{\pi}{5}\Zz\right).$$
Les solutions, exprimées en degrés et éléments de $]-90^\circ,90^\circ[$, sont $-63^\circ$, $-27^\circ$, $9^\circ$,
$45^\circ$ et $81^\circ$. Ainsi, les cinq nombres $\tan(-63^\circ)$, $\tan(-27^\circ)$, $\tan(9^\circ)$,
$\tan(45^\circ)$ et $\tan(81^\circ)$ sont deux à deux distincts et solutions de l'équation
$\frac{5X-10X^3+X^5}{1-10X^2+5X^4}=1$ qui s'écrit encore~:

$$X^5-5X^4-10X^3+10X^2+5X-1=0.$$

Le polynôme $X^5-5x^4-10X^3+10X^2+5X-1$ admet déjà $\tan(45^\circ)=1$ pour racine et on a

$$X^5-5X^4-10X^3+10X^2+5X-1=(X-1)(X^4-4X^3-14X^2-4X+1).$$
Les quatre nombres $\tan(-63^\circ)$, $\tan(-27^\circ)$, $\tan(9^\circ)$ et $\tan(81^\circ)$ sont ainsi les racines du
polynôme $X^4-4X^3-14X^2-4X+1$. Ce dernier peut donc encore s'écrire
$(X-\tan(9^\circ))(X+\tan(27^\circ))(X+\tan(63^\circ))(X-\tan(81^\circ))$. L'opposé du coefficient de $X^3$ à savoir $4$
vaut donc également $\tan(9^\circ)-\tan(27^\circ)-\tan(63^\circ)+\tan(81^\circ)$ et on a montré que~:

\begin{center}
\shadowbox{
$\tan(9^\circ)-\tan(27^\circ)-\tan(63^\circ)+\tan(81^\circ)=4.$
}
\end{center}}
\end{enumerate}
}
