\uuid{PGSw}
\exo7id{5175}
\auteur{rouget}
\organisation{exo7}
\datecreate{2010-06-30}
\isIndication{false}
\isCorrection{true}
\chapitre{Espace vectoriel}
\sousChapitre{Définition, sous-espace}

\contenu{
\texte{
Soit $C$ l'ensemble des applications de $\Rr$ dans $\Rr$, croissantes sur $\Rr$.
}
\begin{enumerate}
    \item \question{$C$ est-il un espace vectoriel (pour les opérations usuelles)~?}
\reponse{$C$ contient l'identité de $\Rr$, mais ne contient pas son opposé. Donc, $C$ n'est pas un espace vectoriel.}
    \item \question{Montrer que $V=\{f\in\Rr^\Rr/\;\exists(g,h)\in C^2\;\mbox{tel que}\;f= g-h\}$ est un $\Rr$-espace
vectoriel.}
\reponse{Montrons que $V$ est un sous-espace vectoriel de l'espace vectoriel des applications de $\Rr$ dans $\Rr$.
$V$ est déjà non vide car contient la fonction nulle $(0=0-0)$.

Soit $(f_1,f_2)\in V^2$. Il existe $(g_1,g_2,h_1,h_2)\in C^4$ tel que $f_1=g_1-h_1$ et $f_2=g_2-h_2$. Mais alors,
$f_1+f_2=(g_1+g_2)-(h_1+h_2)$. Or, une somme de fonctions croissantes sur $\Rr$ est croissante sur $\Rr$, et 
donc, $g_1+g_2$ et $h_1+h_2$ sont des éléments de $C$ ou encore $f_1+f_2$ est dans $V$.

Soit $f\in V$ et $\lambda\in\Rr$. Il existe $(g,h)\in V^2$ tel que $f=g-h$ et donc $\lambda f=\lambda g-\lambda h$.

Si $\lambda\geq 0$, $\lambda g$ et $\lambda h$ sont croissantes sur $\Rr$ et $\lambda f$ est dans $V$.

Si $\lambda<0$, on écrit $\lambda f=(-\lambda h)-(-\lambda g)$, et puisque $-\lambda g$ et $-\lambda h$ sont 
croissantes sur $\Rr$, $\lambda f$ est encore dans $V$. $V$ est donc un sous-espace vectoriel de l'espace vectoriel
des applications de $\Rr$ dans $\Rr$.}
\end{enumerate}
}
