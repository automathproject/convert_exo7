\uuid{7yzD}
\exo7id{222}
\auteur{bodin}
\datecreate{1998-09-01}
\isIndication{true}
\isCorrection{true}
\chapitre{Dénombrement}
\sousChapitre{Binôme de Newton et combinaison}

\contenu{
\texte{
En utilisant la formule du bin\^ome, d\'emontrer
que :
}
\begin{enumerate}
    \item \question{$2^n+1$ est divisible par $3$ si et seulement si $n$ est impair ;}
    \item \question{$3^{2n+1}+2^{4n+2}$ est divisible par $7$.}
\reponse{
L'astuce consiste \`a \'ecrire $2=3-1$ (!)
$$2^{n} = (3-1)^{n} = 3\times p+ (-1)^{n}$$
O\`u $3\times p$ ($p\in\Zz$) repr\'esente les $n$ premiers termes
de $\sum_{k=0}^{n}C^k_{n}3^k(-1)^{n-k}$ et $(-1)^{n}$ est le
dernier terme. 
Donc $2^n -(-1)^n=3p$. 
Si $n$ est impair l'\'egalit\'e s'\'ecrit
$2^n+1=3p$ et donc $2^n+1$ est divisible par 3.
Si $n$ est pair $2^n-1=3p$ donc $2^n+1=3p+2$ qui n'est pas
divisible par $3$.


Pour l'autre assertion regarder $3=7-4$.
}
\indication{Commencer par $2^{n} = (3-1)^{n}$.}
\end{enumerate}
}
