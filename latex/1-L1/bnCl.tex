\uuid{bnCl}
\exo7id{7146}
\auteur{megy}
\organisation{exo7}
\datecreate{2017-05-01}
\isIndication{false}
\isCorrection{false}
\chapitre{Nombres complexes}
\sousChapitre{Géométrie}

\contenu{
\texte{
\'Ecrire en coordonnée complexe :
}
\begin{enumerate}
    \item \question{la rotation d'angle $\pi/4$ et de centre d'affixe $2+3i$;}
\reponse{Par le cours, la rotation d'angle $\theta$ et de centre d'affixe $\omega$ est représentée par l'application 
\[\C\to \C, \: z\mapsto e^{i\theta}(z-\omega)+\omega.\]

La rotation d'angle $\pi/4$ et de centre d'affixe $2+3i$ est donc représentée par l'application
\[ \C\to \C,\: z\mapsto e^{i\pi/4}(z-(2+3i)) + 2+3i = e^{i\pi/4}z +\frac{4 + \sqrt 2 + i(6-5\sqrt 2)}{2}.\]}
    \item \question{la réflexion d'axe d'équation  $y=2x+1$.}
\reponse{La réflexion d'axe d'équation  $y=2x+1$ est représentée par une application de la forme
\[ \C\to \C, z\mapsto a\overline z+b,\]
où $a \in \C^*$ et $b\in \C$ sont des paramètres à déterminer. On sait que $a$ est le nombre complexe de module un dont l'argument est le double de l'angle entre l'axe des abscisses et l'axe de la réflexion. Donc dans ce cas, $a=e^{2i \arctan(2)}$. On peut aussi calculer la forme algébrique de $a$ en écrivant $a=\frac{(1+2i)^2}{|(1+2i)^2|} = \frac{-3+4i}{5}$.

Calculons $b$. On peut par exemple injecter un affixe particulier $z$ et son image $z'$ dans l'équation $z'=\frac{-3+4i}{5} \overline z+b$ et obtenir $b$. Le plus simple est de prendre l'affixe d'un point sur l'axe de la réflexion, qui est donc un point fixe de la réflexion. Prenons par exemple $z=i=z'$. On obtient l'égalité $i =\frac{3-4i}{5} i+b$, d'où $b=i\left(1-\frac{3-4i}{5}\right) = i\frac{2+4i}{5} = \frac{-4+2i}{5}$.

Finalement, la réflexion d'axe d'équation  $y=2x+1$ est représentée par l'application:
\[ \C\to \C, z\mapsto \frac{(-3+4i)\overline z-4+2i}{5}.\]}
\end{enumerate}
}
