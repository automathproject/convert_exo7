\uuid{Ug8O}
\exo7id{10}
\auteur{bodin}
\datecreate{1998-09-01}
\isIndication{false}
\isCorrection{true}
\chapitre{Nombres complexes}
\sousChapitre{Forme cartésienne, forme polaire}

\contenu{
\texte{
\'Etablir les \'egalit\'es suivantes :
}
\begin{enumerate}
    \item \question{$(\cos (\pi/7) + i \sin (\pi/7))(\frac{1-i\sqrt{3}}{2})(1+i) = \sqrt{2}(\cos(5\pi/84)+i\sin(5\pi/84)),$}
    \item \question{$(1-i)(\cos (\pi/5) + i \sin (\pi/5))(\sqrt{3}-i) = 2\sqrt{2}(\cos(13\pi/60)-i\sin(13\pi/60)),$}
    \item \question{$\frac {\sqrt{2}(\cos (\pi/12) + i \sin (\pi/12))}{1+i} = \frac {\sqrt{3}-i}{2}.$}
\reponse{
Il s'agit juste d'appliquer la formule de Moivre :
$$e^{i\theta} = \cos \theta +i \sin \theta ;$$
ainsi que les formules sur les produits de puissances :
$$e^{ia}e^{ib} = e^{i(a+b)}\text{ et  } e^{ia} / e^{ib} = e^{i(a-b)}.$$
}
\end{enumerate}
}
