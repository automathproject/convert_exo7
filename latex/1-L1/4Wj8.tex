\uuid{4Wj8}
\exo7id{3380}
\auteur{quercia}
\datecreate{2010-03-09}
\isIndication{true}
\isCorrection{true}
\chapitre{Matrice}
\sousChapitre{Inverse, méthode de Gauss}

\contenu{
\texte{
Soit $M \in \mathcal{M}_n(\R)$ antisymétrique.
}
\begin{enumerate}
    \item \question{Montrer que $I+M$ est inversible (si $(I+M)X = 0$, calculer ${}^t\!(MX)(MX)$).}
    \item \question{Soit $A = (I-M)(I+M)^{-1}$. Montrer que ${}^t\!A = A^{-1}$.}
\reponse{
Nous allons montrer que $I+M$ est inversible en montrant que si un vecteur $X$ vérifie
$(I+M)X = 0$ alors $X=0$.

Nous allons estimer ${}^t\!(MX)(MX)$ de deux façons.
D'une part c'est un produit de la forme ${}^t\!Y Y = \| Y\|^2$ et donc ${}^t\!(MX)(MX) \ge 0$.

D'autre part :
\begin{align*}
{}^t\!(MX)(MX) 
 & = {}^t\!(MX) (-X) \quad \text{ car } (I+M)X = 0 \text{ donc } MX = -X \\
 & = {}^t\!X {}^t\!M (-X)    \quad \text{ car } {}^t\!(AB)={}^t\! B {}^t\!A \\
 & = {}^t\!X (-M) (-X)    \quad \text{ car } {}^t\!M=-M \\
 & = {}^t\!X MX \\
 & = {}^t\!X (-X) \\
 & = -{}^t\!XX \\
 & = - \|X\|^2 \\
\end{align*}
Qui est donc négatif.

Seule possibilité $\|X\|^2=0$ donc $X=0$ (= le vecteur nul) et donc $I+M$ inversible.
\begin{enumerate}
Calculons $A^{-1}$.
$$A^{-1} =\big( (I-M)\times(I+M)^{-1} \big)^{-1} = \big( (I+M)^{-1} \big)^{-1} \times(I-M)^{-1} = (I+M)\times(I-M)^{-1}$$
(n'oubliez pas que $(AB)^{-1}=B^{-1}A^{-1}$).
Calculons ${}^t\!A$.
\begin{align*}
{}^t\!A 
  & = {}^t\!\big( (I-M) \times (I+M)^{-1} \big) \\
  & = {}^t\!\big( (I+M)^{-1} \big) \times {}^t\!(I-M) \qquad \text{ car } {}^t\!(AB)={}^t\!B{}^t\!A \\
  & = \big( {}^t\!(I+M) \big)^{-1}\times{}^t\!(I-M) \qquad \text{ car } {}^t\!(A^{-1})=\big({}^tA\big)^{-1} \\
  & = \big( I+{}^t\!M) \big)^{-1}\times(I-{}^t\!M) \qquad \text{ car } {}^t\!(A+B)={}^t\!A+{}^t\!B \\
  & = (I-M) ^{-1}\times (I+M) \qquad \text{ car ici } {}^{t}\!{M} = -M \\  
\end{align*}
Montrons que $I+M$ et $(I-M) ^{-1}$ commutent.

Tout d'abord $I+M$ et $I-M$ commutent car $(I+M)(I-M) = I-M^2 = (I-M)(I+M)$.
Maintenant nous avons le petit résultat suivant :

\textbf{Lemme.} Si $AB=BA$ alors $AB^{-1}=B^{-1}A$.

Pour la preuve on écrit :
$$AB=BA \Rightarrow B^{-1}(AB)B^{-1}=B^{-1}(BA)B^{-1} \Rightarrow B^{-1}A=AB^{-1}.$$

En appliquant ceci à $I+M$ et $I-M$ on trouve $(I+M)\times (I-M) ^{-1}= (I-M) ^{-1}\times(I+M)$
et donc $A^{-1}={}^t\!A$.
}
\indication{$M$ antisymétrique signifie ${}^{t}\!{M}=-M$.
\begin{enumerate}
  \item Si $Y$ est un vecteur alors ${}^t\!Y Y = \| Y\|^2$ est un réel positif ou nul.
  \item $I-M$ et $(I+M)^{-1}$ commutent.
\end{enumerate}}
\end{enumerate}
}
