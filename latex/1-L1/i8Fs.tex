\uuid{i8Fs}
\exo7id{6869}
\auteur{chataur}
\datecreate{2012-05-13}
\isIndication{true}
\isCorrection{true}
\chapitre{Espace vectoriel}
\sousChapitre{Définition, sous-espace}

\contenu{
\texte{

}
\begin{enumerate}
    \item \question{D\'ecrire les sous-espaces vectoriels de $\Rr$ ;  puis de $\Rr^2$ et $\Rr^3$.}
\reponse{L'espace vectoriel $\mathbb{R}$ a deux sous-espaces : celui formé du vecteur nul $\{0\}$  et $\Rr$ lui-m\^eme.
\\
L'espace vectoriel $\mathbb{R}^2$ a trois types de sous-espaces: $\{0\}$, 
une infinit\'e de sous-espaces de dimension $1$ (ce sont les droites vectorielles) et $\Rr^2$ lui-m\^eme.
\\
Enfin, l'espace $\mathbb{R}^3$ a quatre types de sous-espaces: le vecteur nul, les droites vectorielles, 
les plans vectoriels et lui-m\^eme.}
    \item \question{Dans $\mathbb{R}^3$ donner un exemple de deux sous-espaces dont l'union 
n'est pas un sous-espace vectoriel.}
\reponse{On consid\`ere deux droites vectorielles de $\mathbb{R}^3$ dont des vecteurs directeurs 
$u$ et $v$ ne sont pas colin\'eaires alors le vecteur $u+v$ n'appartient 
\`a aucune de ces deux droites, l'union de celles-ci n'est pas un espace vectoriel.}
\indication{\begin{enumerate}
  \item Discuter suivant la dimension des sous-espaces.
  \item Penser aux droites vectorielles.
\end{enumerate}}
\end{enumerate}
}
