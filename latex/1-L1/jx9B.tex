\uuid{jx9B}
\exo7id{3037}
\auteur{quercia}
\organisation{exo7}
\datecreate{2010-03-08}
\isIndication{false}
\isCorrection{false}
\chapitre{Logique, ensemble, raisonnement}
\sousChapitre{Relation d'équivalence, relation d'ordre}

\contenu{
\texte{
Soit $X$ un ensemble et $E = \R^X$. On ordonne $E$ par :
$f \le g \iff \forall\ x \in X,\ f(x) \le g(x)$.
}
\begin{enumerate}
    \item \question{V{\'e}rifier que c'est une relation d'ordre.}
    \item \question{L'ordre est-il total ?}
    \item \question{Comparer les {\'e}nonc{\'e}s : {\it``$f$ est major{\'e}e''}, et {\it``$\{f\}$ est major{\'e}''}.}
    \item \question{Soit $(f_i)_{i\in I}$ une famille major{\'e}e de fonctions de $E$. Montrer
     qu'elle admet une borne sup{\'e}rieure.}
\end{enumerate}
}
