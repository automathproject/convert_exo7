\uuid{tEm0}
\exo7id{7021}
\auteur{megy}
\datecreate{2016-08-25}
\isIndication{false}
\isCorrection{false}
\chapitre{Logique, ensemble, raisonnement}
\sousChapitre{Récurrence}

\contenu{
\texte{
%(Bac Antilles-Guyane 2005)
Définissons une suite par $u_0=1$ et pour tout $n \in\N$, $u_{n+1} = \frac12 u_n+n-1$.
}
\begin{enumerate}
    \item \question{Démontrer que pour tout $n\geq 3$, $u_n$ est positif. En déduire que pour tout $n\geq 4$, on a $u_n\geq n-2$. En déduire la limite de la suite.}
    \item \question{Définissons maintenant la suite $v_n=4u_n-8n+24$. Montrer que la suite $(v_n)$ est une suite géométrique, donner son premier terme et sa raison. Montrer que pour tout $n\in \N, u_n = 7\left(\frac{1}{2}\right)^n+2n-6$. Remarquer que $u_n$ est la somme d'une suite géométrique et d'une suite arithmétique dont on précisera les raisons et les premiers termes. En déduire une formule pour la quantité $u_0+u_1+...+u_n$ en fonction de l'entier $n$.}
\end{enumerate}
}
