\uuid{fSY6}
\exo7id{5184}
\auteur{rouget}
\organisation{exo7}
\datecreate{2010-06-30}
\isIndication{false}
\isCorrection{true}
\chapitre{Espace vectoriel}
\sousChapitre{Dimension}

\contenu{
\texte{
Soit $\Kk$ un sous-corps de $\Cc$, $E$ un $\Kk$-espace vectoriel de dimension finie $n\geq2$. Soient $H_1$ et $H_2$ deux hyperplans de $E$.
Déterminer $\mbox{dim}_\Kk(H_1\cap H_2)$. Interprétez le résultat quand $n=2$ ou $n=3$.
}
\reponse{
On a $H_1\subset H_1+H_2$ et donc $\mbox{dim }(H_1+H_2)\geq n-1$ ou encore $\mbox{dim }(H_1+H_2)\in\{n-1,n\}$. Donc

$$\mbox{dim }(H_1\cap H_2)=\mbox{dim }H_1+\mbox{dim }H_2-\mbox{dim }(H_1+H_2)=
\left\{
\begin{array}{l}
(n-1)+(n-1)-(n-1)=n-1\\
\quad\mbox{ou}\\
(n-1)+(n-1)-n=n-2
\end{array}
\right..$$
Maintenant, si $\mbox{dim }(H_1+H_2)=n-1=\mbox{dim }H_1=\mbox{dim }H_2$, alors $H_1=H_1+H_2=H_2$ et donc en particulier,
$H_1=H_2$. Réciproquement, si $H_1=H_2$ alors $H_1+H_2=H_1$ et $\mbox{dim }(H_1+H_2)=n-1$.
En résumé, si $H_1$ et $H_2$ sont deux hyperplans distincts, $\mbox{dim }(H_1\cap H_2)=n-2$ et bien sûr, si 
$H_1=H_2$, alors $\mbox{dim }(H_1\cap H_2)=n-1$.
Si $n=2$, les hyperplans sont des droites vectorielles et l'intersection de deux droites vectorielles distinctes du
plan vectoriel est de dimension $0$, c'est-à-dire réduite au vecteur nul.
Si $n=3$, les hyperplans sont des plans vectoriels et l'intersection de deux plans vectoriels distincts de l'espace de
dimension $3$ est une droite vectorielle.
}
}
