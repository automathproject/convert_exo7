\uuid{2sAS}
\exo7id{5077}
\auteur{rouget}
\organisation{exo7}
\datecreate{2010-06-30}
\isIndication{false}
\isCorrection{true}
\chapitre{Nombres complexes}
\sousChapitre{Trigonométrie}

\contenu{
\texte{
Calculer $I=\int_{\pi/6}^{\pi/3}\cos^4x\sin^6x\;dx$ et
$J=\int_{\pi/6}^{\pi/3}\cos^4x\sin^7x\;dx$.
}
\reponse{
Pour $x$ réel , on a~:
\begin{align*}
\cos^4x\sin^6x&=\left(\frac{1}{2}(e^{ix}+e^{-ix})\right)^4\left(\frac{1}{2i}(e^{ix}-e^{-ix})\right)^6\\
 &=-\frac{1}{2^{10}}(e^{4ix}+4e^{2ix}+6+4e^{-2ix}+e^{-4ix})
(e^{6ix}-6e^{4ix}+15e^{2ix}-20+15e^{-2ix}-6e^{-4ix}+e^{-6ix})\\
 &= -\frac{1}{2^{10}}(e^{10ix}-2e^{8ix}-3e^{6ix}+8e^{4ix}+2e^{2ix}-12
+2e^{-2ix}+8e^{-4ix}-3e^{-6ix}-2e^{-8ix}+e^{-10ix})\\
 &=-\frac{1}{2^9}(\cos10x-2\cos8x-3\cos6x+8\cos4x+2\cos2x-6)\\
 &=-\frac{1}{512}(\cos10x-2\cos8x-3\cos6x+8\cos4x+2\cos2x-6)
\end{align*}
(Remarque. La fonction proposée était paire et l'absence de sinus était donc prévisible. Cette remarque guidait aussi
les calculs intermédiaires~:~les coefficients de $e^{-2ix}$, $e^{-4ix}$,... étaient les mêmes que ceux
de $e^{2ix}$, $e^{4ix}$,...) Par suite,

\begin{align*}
I&=-\frac{1}{512}\left(\left[\frac{\sin10x}{10}-\frac{\sin8x}{4}-\frac{\sin6x}{2}+2\sin4x+\sin2x\right]_{\pi/6}^{\pi/3}
-6\left(\frac{\pi}{3}-\frac{\pi}{6}\right)\right)\\
 &=-\frac{1}{512}\left(\frac{1}{10}(-\frac{\sqrt{3}}{2}+\frac{\sqrt{3}}{2})-\frac{1}{4}(\frac{\sqrt{3}}{2}+\frac{\sqrt{3}}{2})-\frac{1}{2}(0-0)+2(-\frac{\sqrt{3}}{2}-\frac{\sqrt{3}}{2})+(\frac{\sqrt{3}}{2}-\frac{\sqrt{3}}{2})-\pi\right)\\
 &=-\frac{1}{512}\left(-\frac{\sqrt{3}}{4}-2\sqrt{3}-\pi\right)=\frac{9\sqrt{3}+4\pi}{2048}.
\end{align*}
Pour $x$ réel, on a

\begin{align*}
\cos^4x\sin^7x&=\cos^4x\sin^6x\sin x=\cos^4x(1-\cos^2x)^3\sin x\\
 &=\cos^4x\sin x-3\cos^6x\sin x+3\cos^8x\sin x-\cos^{10}x\sin x.
\end{align*}

Par suite,

\begin{align*}
J&=\left[-\frac{\cos^5x}{5}+\frac{3\cos^7x}{7}-\frac{\cos^9x}{3}+\frac{\cos^{11}x}{11}\right]_{\pi/6}^{\pi/3}\\
 &=-\frac{1}{5}\times\frac{1-9\sqrt{3}}{32}+\frac{3}{7}\times\frac{1-27\sqrt{3}}{128}-\frac{1}{3}\times\frac{1-81\sqrt
{3}}{512}+\frac{1}{11}\times\frac{1-243\sqrt{3}}{2048}\\
 &=\frac{1}{2^{11}\times3\times5\times7\times11}(-14784(1-9\sqrt{3})+7920(1-27\sqrt{3})-154
0(1-81\sqrt{3})+105(1-243\sqrt{3}))\\
 &=\frac{1}{{2365440}}(-{8299}+{18441}\sqrt{3}).
\end{align*}
}
}
