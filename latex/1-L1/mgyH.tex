\uuid{mgyH}
\exo7id{7202}
\auteur{megy}
\datecreate{2019-07-23}
\isIndication{false}
\isCorrection{false}
\chapitre{Logique, ensemble, raisonnement}
\sousChapitre{Relation d'équivalence, relation d'ordre}

\contenu{
\texte{
Soit $f : E\to F$, soit $\equiv_f$ la relation d'équivalence sur $E$ dont les classes d'équivalence sont les fibres de $f$, et soit $Q = E/\equiv_f$ l'ensemble quotient.
}
\begin{enumerate}
    \item \question{Montrer que $f$ passe au quotient en une application $\bar f : Q\to F$ qui est injective.}
    \item \question{Montrer qu'une relation d'équivalence $\mathcal R$ sur $E$ est plus fine que $\equiv_f$ si et seulement si $f$ passe au quotient par $\mathcal R$.}
    \item \question{En déduire quelles sont les relations d'équivalence les plus et moins fines telles que $f$ passe au quotient par $\mathcal R$.}
\end{enumerate}
}
