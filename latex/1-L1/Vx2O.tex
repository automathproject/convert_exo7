\uuid{Vx2O}
\exo7id{5187}
\auteur{rouget}
\organisation{exo7}
\datecreate{2010-06-30}
\isIndication{false}
\isCorrection{true}
\chapitre{Application linéaire}
\sousChapitre{Image et noyau, théorème du rang}

\contenu{
\texte{
Soient $(e_i)_{1\leq i\leq 4}$ la base canonique de $\Rr^4$ et $f$ l'endomorphisme de $\Rr^4$ défini
par~:~$f(e_1)=2e_1+e_3$, $f(e_2)=-e_2+e_4$, $f(e_3)=e_1+2e_3$ et $f(e_4)=e_2-e_4$. Déterminer $\mbox{Ker}f$ et
$\mbox{Im}f$.
}
\reponse{
Soit $u=(x,y,z,t)=xe_1+ye_2+ze_3+te_4\in\Rr^4$. Alors,

\begin{align*}
f(u)&=xf(e_1)+yf(e_2)+zf(e_3)+tf(e_4)=x(2e_1+e_3)+y(-e_2+e_4)+z(e_1+2e_3)+t(e_2-e_4)\\
 &=(2x+z)e_1+(-y+t)e_2+(x+2z)e_3+(y-t)e_4.
\end{align*}
Par suite,

$$u\in\mbox{Ker }f\Leftrightarrow
\left\{
\begin{array}{l}
2x+z=0\\
-y+t=0\\
x+2z=0\\
y-t=0
\end{array}
\right.\Leftrightarrow
\left\{
\begin{array}{l}
x=z=0\\
y=t
\end{array}
\right..$$
Donc, $\mbox{Ker }f=\{(0,y,0,y),\;y\in\Rr\}=\mbox{Vect}((0,1,0,1))$.

\begin{center}
\shadowbox{
$\mbox{Ker }f=\mbox{Vect}((0,1,0,1))$.
}
\end{center}

Soit $u'=(x',y',z',t')\in\Rr^4$.

\begin{align*}
u'= (x',y',z',t')\in\mbox{Im }f&\Leftrightarrow\exists(x,y,z,t)\in\Rr^4/\;\left\{
\begin{array}{l}
2x+z=x'\\
-y+t=y'\\
x+2z=z'\\
y-t=t'
\end{array}
\right.\Leftrightarrow\exists(x,y,z,t)\in\Rr^4/\;\left\{
\begin{array}{l}
x=\frac{1}{3}(2x'-z')\\
z=\frac{1}{3}(-x'+2z')\\
t=y+y'\\
y'+t'=0
\end{array}
\right.\\
 &\Leftrightarrow y'=-t'
\end{align*}
(si $y'\neq-t'$, le système ci-dessus, d'inconnues $x$, $y$, $z$ et $t$, n'a pas de solution et si $y'=-t'$, le système
ci-dessus admet au moins une solution comme par exemple
$(x,y,z,t)=\left(\frac{1}{3}(2x'-z'),0,\frac{1}{3}(-x'+2z'),y')\right)$.
Donc,
$\mbox{Im }f=\{(x,y,z,t)\in\Rr^4/\;y+t=0\}=\{(x,y,z,-y)/(x,y,z)\in\Rr^3\}=\{xe_1+y(e_2-e_4)+ze_3,\;(x,y,z)\in\Rr^4\}=
\mbox{Vect}(e_1,e_2-e_4,e_3)$.

\begin{center}
\shadowbox{
$\mbox{Im }f==\{(x,y,z,-y)/(x,y,z)\in\Rr^3\}=\mbox{Vect}(e_1,e_2-e_4,e_3)$.
}
\end{center}

\textbf{Autre solution} pour la détermination de $\mbox{Im }f$.
$\mbox{Im }f=\mbox{Vect}(f(e_1),f(e_2),f(e_3),f(e_4))=\mbox{Vect}(2e_1+e_3,-e_2+e_4,e_1+2e_3,e_2-e_4)
=\mbox{Vect}(2e_1+e_3,e_1+2e_3,e_2-e_4)$. Mais d'autre part, d'après le théorème du rang,
$\mbox{dim }(\mbox{Im }f)=4-1=3$. Donc, $(2e_1+e_3,e_1+2e_3,e_2-e_4)$ est une base de $\mbox{Im }f$.
}
}
