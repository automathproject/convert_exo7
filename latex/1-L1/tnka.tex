\uuid{tnka}
\exo7id{3339}
\auteur{quercia}
\datecreate{2010-03-09}
\isIndication{false}
\isCorrection{false}
\chapitre{Application linéaire}
\sousChapitre{Morphismes particuliers}

\contenu{
\texte{
Soit $E$ un $ K$-ev de dimension finie.
Le centre de $\mathcal{L}(E)$ est :
$Z = \{f \in \mathcal{L}(E) \text{ tq } \forall\ g \in \mathcal{L}(E),\ f\circ g = g\circ f\}$.
}
\begin{enumerate}
    \item \question{Soit $f \in \mathcal{L}(E)$ et $\vec x \in E$. Si $(\vec x,f(\vec x))$ est libre,
    montrer qu'il existe $g \in \mathcal{L}(E)$ telle que $g(\vec x) = \vec x$ et
    $g\circ f(\vec x) = -f(\vec x)$.}
    \item \question{En déduire que $Z$ est l'ensemble des homothéties.}
    \item \question{Déterminer $Z' =
    \{f \in \mathcal{L}(E) \text{ tq } \forall\ g \in GL(E),\ f\circ g = g\circ f\}.$}
\end{enumerate}
}
