\uuid{8CiV}
\exo7id{2468}
\auteur{matexo1}
\datecreate{2002-02-01}
\isIndication{false}
\isCorrection{false}
\chapitre{Réduction d'endomorphisme, polynôme annulateur}
\sousChapitre{Diagonalisation}

\contenu{
\texte{
Soit $\mathbb K$ le corps des r\'eels ou des
complexes, et $u$ l'endomorphisme de $\mathbb K^3$ ayant pour matrice
$$A = \left( \begin{array}{ccc} 0&-2&0\\ 1&0&-1\\ 0&2&0 \end{array} \right).$$
\'Etudier, dans les deux cas $\mathbb K = \R$ et $\mathbb K = \C$, si $u$ est
diagonalisable. En donner une forme diagonalis\'ee dans une base
dont on donnera la matrice de passage par rapport \`a la base
canonique.
}
}
