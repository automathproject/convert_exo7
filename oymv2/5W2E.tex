\uuid{5W2E}
\exo7id{2064}
\auteur{ridde}
\datecreate{1999-11-01}
\isIndication{false}
\isCorrection{false}
\chapitre{Courbes planes}
\sousChapitre{Propriétés métriques : longueur, courbure,...}

\contenu{
\texte{
Soit $M (s)$ un arc $C^2$ bir\'egulier param\'etr\'e par une abscisse curviligne.
Soit $\mathcal{R}$ le rep\`ere de Fr\'enet $ (M (0), \vec{t} (0), \vec{n} (0))$.
On note $ (X (s), Y (s))$ les coordonn\'ees dans ce rep\`ere d'un point $M (s)$ de la courbe.
}
\begin{enumerate}
    \item \question{Montrer que si $R_{0}$ est le rayon de courbure en $M (0)$ alors
$R_0 = \lim\limits_{s \rightarrow 0} \dfrac{X^2 (s)}{2Y (s)}$.}
    \item \question{En d\'eduire le rayon de courbure au point $\theta  = 0$ de la courbe
$\rho (\theta) = 1 + 2\cos (\frac{\theta}2)$.}
\end{enumerate}
}
