\uuid{2141}
\auteur{debes}
\datecreate{2008-02-12}
\isIndication{false}
\isCorrection{true}
\chapitre{Sous-groupe distingué}
\sousChapitre{Sous-groupe distingué}

\contenu{
\texte{
Soit $G$ un groupe et $\simeq $ une relation d'\'equivalence sur $G$. On suppose que
cette relation est compatible avec la loi de groupe, c'est-\`a-dire que
$$\forall x,y \in G \quad \forall x',y' \in G \quad  x\simeq x' \quad {\rm et}\quad 
y\simeq y' \quad {\rm alors} \quad  xy\simeq x'y' $$
Montrer que la classe $H$ de l'\'el\'ement neutre $1$ est un sous-groupe distingu\'e de
$G$ et que  $$\forall x, x' \in G \quad x\simeq x' \quad \hbox{\rm est\quad
\'equivalent \quad  \`a} \quad x'x^{-1} \in H$$
}
\reponse{
Etant donn\'es $y,z\in H$, on a $y\simeq 1$ et $z\simeq 1$. La
compatibilit\'e de la loi donne d'une part $yz \simeq 1$, soit $yz \in H$, et d'autre
part $y y^{-1} \simeq y^{-1}$ soit $y^{-1}\in H$. Cela montre que $H$ est un
sous-groupe de $G$. Pour tout $x\in G$, on a aussi $xyx^{-1} \simeq x 1 x^{-1}=1$ et
donc $xyx^{-1}\in H$. Le sous-groupe $H$ est donc distingu\'e.

De plus, pour $x,x^\prime \in G$, si $x\simeq x^\prime$, alors par compatibilit\'e de
la loi, on a $x^\prime x^{-1} \simeq x x^{-1} = 1$, c'est-\`a-dire $x^\prime x^{-1}\in
H$. R\'eciproquement, si $x^\prime x^{-1}\in H$, alors $x^\prime x^{-1} \simeq 1$,
et donc, par compatibilit\'e de la loi, $x\simeq x^\prime$.
}
}
