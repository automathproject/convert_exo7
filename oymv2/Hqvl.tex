\uuid{Hqvl}
\exo7id{5116}
\auteur{rouget}
\datecreate{2010-06-30}
\isIndication{false}
\isCorrection{true}
\chapitre{Arithmétique dans Z}
\sousChapitre{Divisibilité, division euclidienne}

\contenu{
\texte{
Pour $n\geq1$, on pose $H_n=\sum_{k=1}^{n}\frac{1}{k}$. Montrer que, pour $n\geq2$, $H_n$ n'est jamais un entier
(indication~:~montrer par récurrence que $H_n$ est le quotient d'un entier impair par un entier pair en distingant
les cas où $n$ est pair et $n$ est impair).
}
\reponse{
Montrons par récurrence que, pour $n\geq2$, $H_n$ peut s'écrire sous la forme $\frac{p_n}{q_n}$ où $q_n$ est un
entier pair et $p_n$ est un entier impair (la fraction précédente n'étant pas nécessairement irréductible mais à coup
sûr pas un entier).
Pour $n=2$, $H_2=\frac{3}{2}$ et $H_2$ est bien du type annoncé.
Soit $n\geq2$. Supposons que pour tout entier $k$ tel que $2\leq k\leq n$, on ait $H_k=\frac{p_k}{q_k}$ où $p_k$ est un
entier impair et $q_k$ est un entier pair et montrons que $H_{n+1}=\frac{p_{n+1}}{q_{n+1}}$ où $p_{n+1}$ est un 
entier impair et $q_{n+1}$ est un entier pair.
(Recherche. L'idée
$H_{n+1}=\frac{p_n}{q_n}+\frac{1}{n+1}=\frac{(n+1)p_n+q_n}{(n+1)q_n}$ ne marche à coup sur que si $(n+1)p_n+q_n$ est
impair ce qui est assuré si $n+1$ est impair et donc $n$ pair)
\begin{itemize}
\item[\textbf{1er cas.}] Si $n$ est pair, on peut poser $n=2k$ où $k\in\Nn^*$. Dans ce cas,
$H_{n+1}=\frac{(2k+1)p_n+q_n}{(2k+1)q_n}$ et $H_{n+1}$ est bien le quotient d'un entier impair par un
entier pair.

\item[\textbf{2ème cas.}] Si $n$ est impair, on pose $n=2k-1$ où $k\geq2$ (de sorte que $2k-1\geq3$).

\begin{align*}
H_{n+1}&=\sum_{i=1}^{2k}\frac{1}{i}=\sum_{i=1}^{k}\frac{1}{2i}+\sum_{i=0}^{k-1}\frac{1}{2i+1}\;\\
 &(\mbox{en séparant les
fractions de dénominateurs pairs des fractions de dénominateurs impairs})\\
 &=\frac{1}{2}\sum_{i=1}^{k}\frac{1}{i}+\sum_{i=0}^{k-1}\frac{1}{2i+1}=\frac{1}{2}H_k+\sum_{i=0}^{k-1}\frac{1}{2i+1}.
\end{align*}

Maintenant, en réduisant au même dénominateur et puisque un produit de nombres impairs est impair, on voit 
que $\sum_{i=0}^{k-1}\frac{1}{2i+1}$ est du type $\frac{K}{2K'+1}$ où $K$ et $K'$ sont des entiers. Ensuite, puisque
$2\leq k\leq2k-1=n$, par hypothèse de récurrence, $H_k=\frac{p_k}{q_k}$ où $p_k$ est un entier impair et $q_k$ un
entier pair. Après réduction au même dénominateur, on obtient

$$H_{n+1}=\frac{p_k}{2q_k}+\frac{K}{2K'+1}=\frac{(2K'+1)p_k+2Kq_k}{2q_k(2K'+1)}.$$

$2Kq_k$ est un entier pair et $(2K'+1)p_k$ est un entier impair en tant que produit de deux nombres impairs. Donc le
numérateur est bien un entier impair et puisque $2qk(2K'+1)$ est un entier pair, $H_{n+1}$ est bien dans tous les cas de
la forme désirée.
\end{itemize}
On a montré par récurrence que pour tout entier naturel $n\geq2$, $H_n$ est le quotient d'un entier impair par un entier pair
et donc n'est pas un entier.
}
}
