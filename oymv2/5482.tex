\uuid{CTqp}
\exo7id{5482}
\auteur{rouget}
\datecreate{2010-07-10}
\isIndication{false}
\isCorrection{true}
\chapitre{Espace euclidien, espace normé}
\sousChapitre{Produit scalaire, norme}

\contenu{
\texte{
Pour $A=(a_{i,j})_{1\leq i,j\leq n}\in\mathcal{M}_n(\Rr)$, $N(A)=\mbox{Tr}(^tAA)$. Montrer que $N$ est une norme vérifiant de plus $N(AB)\leq N(A)N(B)$ pour toutes matrices carrées $A$ et $B$. $N$ est-elle associée à un produit scalaire~?
}
\reponse{
Posons $\varphi~:~(A,B)\mapsto\mbox{Tr}({^t}AB)$. Montrons que $\varphi$ est un produit scalaire sur $\mathcal{M}_n(\Rr)$. 
\textbf{1ère solution.} \textbullet~$\varphi$ est symétrique. En effet, pour $(A,B)\in(\mathcal{M}_n(\Rr))^2$,

$$\varphi(A,B)=\mbox{Tr}(^{t}AB)=\mbox{Tr}({^t}({^t}AB))=\mbox{Tr}({^t}BA)=\varphi(B,A).$$

\textbullet~$\varphi$ est bilinéaire par linéarité de la trace et de la transposition.
\textbullet~Si $A=(a_{i,j})_{1\leq i,j\leq n}\in\mathcal{M}_n(\Rr)\setminus\{0\}$, alors

$$\varphi(A,A)=\sum_{i=1}^{n}\left(\sum_{j=1}^{n}a_{i,j}a_{i,j}\right)=\sum_{i,j}^{}a_{i,j}^2>0$$
car au moins un des réels de cette somme est strictement positif. $\varphi$ est donc définie, positive.

\textbf{2ème solution.} Posons $A=(a_{i,j})$ et $B=(b_{i,j})$. On a

\begin{center}
$\text{Tr}({^t}AB)=\sum_{j=1}^{n}\left(\sum_{i=1}^{n}a_{i,j}b_{i,j}\right)=\sum_{1\leq i,j\leq n}^{}a_{i,j}b_{i,j}$.
\end{center}
Ainsi, $\varphi$ est le produit scalaire canonique sur $\mathcal{M}_n(\Rr)$ et en particulier, $\varphi$ est un produit scalaire sur $\mathcal{M}_n(\Rr)$.
$N$ n'est autre que la norme associée au produit scalaire $\varphi$ (et en particulier, $N$ est une norme).
Soit $(A,B)\in(\mathcal{M}_n(\Rr))^2$.

\begin{align*}\ensuremath
N(AB)^2&=\sum_{i,j}^{}\left(\sum_{k=1}^{n}a_{i,k}b_{k,j}\right)^2\\
 &\leq\sum_{i,j}^{}\left(\sum_{k=1}^{n}a_{i,k}^2\right)\left(\sum_{l=1}^{n}b_{l,j}^2\right)\;(\mbox{d'après l'inégalité de \textsc{Cauchy}-\textsc{Schwarz}})\\
 &=\sum_{i,j,k,l}^{}a_{i,k}^2b_{l,j}^2=\left(\sum_{i,k}^{}a_{i,k}^2\right)\left(\sum_{l,j}^{}b_{l,j}^2\right)=N(A)^2N(B)^2,
\end{align*}
et donc,

\begin{center}
\shadowbox{
$\forall(A,B)\in(\mathcal{M}_n(\Rr))^2,\;N(AB)\leq N(A)N(B)$.
}
\end{center}
}
}
