\uuid{2620}
\auteur{debievre}
\datecreate{2009-05-19}

\contenu{
\texte{
Soit $A=\{(t,\sin\frac1t)\in\R^2 ; t>0 \}$. Montrer que $A$
n'est ni ouvert ni ferm\'e. D\'eterminer l'adh\'erence $\overline A$
de $A$.
}
\indication{Distinguer la partie triviale de l'exercice de la partie 
non triviale. 
Dans cet exercice,  le seul point d\'elicat est pour le param\`etre $t$ proche de $0$.}
\reponse{
La partie $A$ du plan n'est pas ouverte puisqu'elle
ne contient aucun disque ouvert.
Cette partie $A$ n'est pas ferm\'ee non plus: L'origine est un point 
d'adh\'erence: Quel que soit le disque ouvert $B$ centr\'e \`a l'origine,
il existe un point de $B$ qui appartient \`a $A$.
Mais l'origine n'appartient pas \`a $A$ d'o\`u $A$ n'est pas ferm\'e.
L'adh\'erence $\overline A$ de $A$ est la r\'eunion
$A \cup (\{0\}\times [-1,1])$. Car quelle que soit
la suite $(x_n)$ de points de $A \cup (\{0\}\times [-1,1])$
telle que cette suite converge dans le plan,
la limite appartient \`a $A \cup (\{0\}\times [-1,1])$.
}
}
