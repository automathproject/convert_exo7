\uuid{CLeh}
\exo7id{7203}
\auteur{megy}
\datecreate{2019-07-23}
\isIndication{false}
\isCorrection{false}
\chapitre{Logique, ensemble, raisonnement}
\sousChapitre{Relation d'équivalence, relation d'ordre}

\contenu{
\texte{
(Coégalisateur) 
Soient $A$ et $B$ deux ensembles et $f$ et $g$ deux applications entre $A$ et $B$. On définit sur $B$ la relation binaire suivante : $\mathcal R$ est la relation d'équivalence la plus fine telle que $\forall a\in A, f(a)\mathcal R g(a)$. Le \emph{coégalisateur de $f$ et $g$} est par définition l'ensemble quotient $C = B/\mathcal R$. On note $\pi : B \to C$ la surjection canonique sur le quotient. On a alors $\pi\circ f = \pi \circ g$.

Montrer que $C$ et $\pi$ vérifient la propriété suivante (dite \emph{propriété universelle du coégalisateur}):


Pour tout ensemble $X$ et application $\phi : B \to X$ vérifiant $\phi\circ f = \phi\circ g$,  il existe une unique application $h : C\to X$ telle que $\phi = h \circ \pi $.
}
}
