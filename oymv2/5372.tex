\uuid{5372}
\auteur{rouget}
\datecreate{2010-07-06}

\contenu{
\texte{
Calculer $\mbox{det}(\mbox{com}A)$ en fonction de $\mbox{det}A$ puis étudier le rang de $\mbox{com}A$ en fonction du rang de $A$.
}
\reponse{
On a toujours $A\times{^t}\mbox{com}A=(\mbox{det}A)I_n$ et donc 

$$(\mbox{det}A)(\mbox{det}(\mbox{com}A))=(\mbox{det}A)(\mbox{det}(^t\mbox{com}A))=\mbox{det}(\mbox{det}A\;I_n)=(\mbox{det}A)^n.$$
\textbullet~Si $\mbox{det}A\neq0$, on obtient $\mbox{det}(\mbox{com}A)=(\mbox{det}A)^{n-1}$.
\textbullet~Si $\mbox{det}A=0$, alors $A{^t\mbox{com}A}=0$ et $\mbox{com}A$ n'est pas inversible car sinon, $A=0$ puis $\mbox{com}A=0$ ce qui est absurde. Donc, $\mbox{det}(\mbox{com}A)=0$. Ainsi, dans tous les cas,

\begin{center}
\shadowbox{
$\forall A\in M_n(\Cc),\;\mbox{det}(\mbox{com}A)=(\mbox{det}A)^{n-1}$.
}
\end{center}
\textbullet~Si $\mbox{rg}A=n$, alors  $\mbox{com}A\in GL_n(\Kk)$ (car $\mbox{det}(\mbox{com}A)\neq0$) et $\mbox{rg}(\mbox{com}A)=n$.
\textbullet~Si $\mbox{rg}A\leq n-2$, alors tous les mineurs de format $n-1$ sont nuls et $\mbox{com}A=0$. Dans ce cas, $\mbox{rg}(\mbox{com}A)=0$.
\textbullet~Si $\mbox{rg}A=n-1$, il existe un mineur de format $n-1$ non nul et $\mbox{com}A\neq0$. Dans ce cas, $1\leq\mbox{rg}(\mbox{com}A)\leq n-1$.
Plus précisément, 
$$A{^t\mbox{com}A}=0\Rightarrow\mbox{com}A{^tA}=0\Rightarrow\mbox{Im}({^tA})\subset\mbox{Ker}(\mbox{com}A)
\Rightarrow\mbox{dim}(\mbox{Ker}(\mbox{com}A))\geq\mbox{rg}({^tA})=\mbox{rg}A=n-1\Rightarrow\mbox{rg}(\mbox{com}A)\leq1,$$
et finalement si $\mbox{rg}A=n-1$, $\mbox{rg}(\mbox{com}A)=1$.
}
}
