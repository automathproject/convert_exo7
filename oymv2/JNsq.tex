\uuid{JNsq}
\exo7id{1281}
\auteur{ridde}
\datecreate{1999-11-01}
\isIndication{false}
\isCorrection{false}
\chapitre{Calcul d'intégrales}
\sousChapitre{Intégrale impropre}

\contenu{
\texte{

}
\begin{enumerate}
    \item \question{Montrer que $\forall x>-1 \, \,   \ln (1 + x)\leq x.$\\}
    \item \question{Soit $n \in \Nn^{*}.$ Montrer que $\forall x \in [0, n]\, \,   (1-\frac{x}{n})^{n}\leq
e^{-x} \leq (1 + \frac{x}{n})^{-n}.$\\}
    \item \question{En d\'eduire que\\
$\displaystyle{\int _{0}^{\sqrt {n}}\!\left (1-{\frac {{t}^{2}}{n}}\right )^{n}{dt}} \leq
\displaystyle{\int _{0}^{\sqrt {n}}\!{e^{-{t}^{2}}}{dt}} \leq
\displaystyle{\int _{0}^{\sqrt {n}}\!\frac{1}{\left (1+{\frac {{t}^{2}}{n}}\right )^n}{dt}}. $ \\

Rappel (int\'egrales de Wallis) : $I_{n} =
\displaystyle{\int _{0}^{{\frac {\pi }{2}}}\!\left (\cos(\theta)\right )^{n}{d\theta}} \sim
\sqrt {\frac{\pi}{2n}}. $\\}
    \item \question{Montrer que $\displaystyle{\int _{0}^{\infty } \frac 1{(1+{u}^{2})^n}{du}}$
existe et vaut $I_{2n-2}. $\\}
    \item \question{Montrer que $\displaystyle{\int _{0}^{\infty }\!{e^{-{x}^{2}}}{dx}} $ existe et vaut
 $\frac{\sqrt\pi}{2} $.}
\end{enumerate}
}
