\uuid{hFby}
\exo7id{5744}
\auteur{rouget}
\datecreate{2010-10-16}
\isIndication{false}
\isCorrection{true}
\chapitre{Suite et série de fonctions}
\sousChapitre{Suites et séries d'intégrales}

\contenu{
\texte{
Montrer que pour tout réel $x$, $\int_{0}^{+\infty}\frac{\cos(xt)}{\ch t}\;dt = 2\sum_{n=0}^{+\infty}(-1)^n\frac{2n+1}{(2n+1)^2+x^2}$
}
\reponse{
\textbf{Existence de l'intégrale.} Soit $x\in\Rr$. La fonction $f~:~t\mapsto\frac{\cos(xt)}{\ch t}$ est continue sur $[0,+\infty[$. De plus, pour tout réel positif $t$, $|f(t)|\leqslant\frac{1}{\ch t}$ et donc $|f(t)|\underset{t\rightarrow+\infty}{\sim}\frac{2}{e^t}\underset{t\rightarrow+\infty}{=}o\left(\frac{1}{t^2}\right)$. On en déduit que la fonction $f$ est intégrable sur  $[0,+\infty[$.

\begin{center}
Pour tout réel $x$, $\int_{0}^{+\infty}\frac{\cos(xt)}{\ch t}\;dt$ existe.
\end{center}

\textbf{Convergence de la série.} Soit $x\in\Rr$. Pour $n\in\Nn$, posons $u_n(x)=\frac{2n+1}{(2n+1)^2+x^2}$. Pour $n\in\Nn$,

\begin{align*}\ensuremath
u_n(x)-u_{n+1}(x)&=\frac{2n+1}{(2n+1)^2+x^2}-\frac{2n+3}{(2n+3)^2+x^2}=\frac{(2n+1)((2n+3)^2+x^2)-(2n+3)((2n+1)^2+x^2)}{((2n+1)^2+x^2)((2n+3)^2+x^2)}\\
 &=\frac{2(2n+1)(2n+3)-2x^2}{((2n+1)^2+x^2)((2n+3)^2+x^2)}.
\end{align*}

Puisque le numérateur de cette dernière expression tend vers $+\infty$ quand $n$ tend vers $+\infty$, cette expression est positive pour $n$ grand. On en déduit que la suite $(u_n(x))$ décroît à partir d'un certain rang. D'autre part, $\lim_{n \rightarrow +\infty}u_n(x)=0$.

On en déduit que la série de terme général $(-1)^nu_n(x)$ converge en vertu du critère spécial aux séries alternées.

\begin{center}
Pour tout réel $x$, la série de terme général $(-1)^n\frac{2n+1}{(2n+1)^2+x^2}$ converge.
\end{center}

\textbf{Egalité de l'intégrale et de la somme de la série.} Soit $n\in\Nn$. Pour $t\in]0,+\infty[$, on a $e^{-t}\in]0,1[$ et donc

\begin{align*}\ensuremath
\frac{\cos(xt)}{\ch t}&=\frac{2\cos(xt)e^{-t}}{1+e^{-2t}}=2\cos(xt)e^{-t}\sum_{k=0}^{n}(-1)^ke^{-2kt}+(-1)^{n+1}\frac{\cos(xt)e^{-(2n+3)t}}{1+e^{-2t}}\\
 &=2\sum_{k=0}^{n}(-1)^k\cos(xt)e^{-(2k+1)t}+(-1)^{n+1}\frac{\cos(xt)e^{-(2n+3)t}}{1+e^{-2t}}.
\end{align*}

Maintenant, pour chaque $k\in\Nn$, la fonction $t\mapsto\cos(xt)e^{-(2k+1)t}$ est intégrable sur $[0,+\infty[$ car continue sur $[0,+\infty[$ et négligeable devant $\frac{1}{t^2}$ quand $t$ tend vers $+\infty$. On en déduit encore que la fonction $t\mapsto (-1)^{n+1}\frac{\cos(xt)e^{-(2n+3)t}}{1+e^{-2t}}$ est intégrable sur $[0,+\infty[$ puis que 

\begin{center}
$\forall n\in\Nn$, $\int_{0}^{+\infty}\frac{\cos(xt)}{\ch t}\;dt=2\sum_{k=0}^{n}(-1)^k\int_{0}^{+\infty}\cos(xt)e^{-(2k+1)t}\;dt+(-1)^{n+1}\int_{0}^{+\infty}\frac{\cos(xt)e^{-(2n+3)t}}{1+e^{-2t}}\;dt$.
\end{center}

Ensuite, $\left|\int_{0}^{+\infty}(-1)^{n+1}\frac{\cos(xt)e^{-(2n+3)t}}{1+e^{-2t}}\;dt\right|\leqslant\int_{0}^{+\infty}e^{-(2n+3)t}\;dt=\frac{1}{2n+3}$, et donc $\lim_{n \rightarrow +\infty}(-1)^{n+1}\frac{\cos(xt)e^{-(2n+3)t}}{1+e^{-2t}}\;dt=0$ puis

\begin{center}
$\int_{0}^{+\infty}\frac{\cos(xt)}{\ch t}\;dt=2\sum_{n=0}^{+\infty}(-1)^n\int_{0}^{+\infty}\cos(xt)e^{-(2n+1)t}\;dt$.
\end{center}

Soit $n\in\Nn$.

\begin{align*}\ensuremath
\int_{0}^{+\infty}\cos(xt)e^{-(2n+1)t}\;dt&=\text{Re}\left(\int_{0}^{+\infty}e^{ixt}e^{-(2n+1)t}\;dt\right)=\text{Re}\left(\int_{0}^{+\infty}e^{(-(2n+1)+ix)t}\;dt\right)\\
 &=\text{Re}\left(\left[\frac{e^{(-(2n+1)+ix)t}}{-(2n+1)+ix}\right]_0^{+\infty}\right)=\text{Re}\left(\frac{1}{(2n+1)-ix}\left(1-\lim_{t \rightarrow +\infty}e^{(-(2n+1)+ix)t}\right)\right)\\
 &=\text{Re}\left(\frac{1}{(2n+1)-ix}\right)\;(\text{car}\;\left|e^{(-(2n+1)+ix)t}\right|=e^{-(2n+1)t}\underset{n\rightarrow+\infty}{\rightarrow}0)\\
 &=\text{Re}\left(\frac{2n+1+ix}{(2n+1)^2+x^2}\right)=\frac{2n+1}{(2n+1)^2+x^2}.
\end{align*}

On a enfin montré que

\begin{center}
\shadowbox{
$\int_{0}^{+\infty}\frac{\cos(xt)}{\ch t}\;dt = 2\sum_{n=0}^{+\infty}(-1)^n\frac{2n+1}{(2n+1)^2+x^2}$.
}
\end{center}
}
}
