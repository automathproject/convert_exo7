\uuid{TKIQ}
\exo7id{900}
\auteur{liousse}
\datecreate{2003-10-01}
\isIndication{true}
\isCorrection{true}
\chapitre{Espace vectoriel}
\sousChapitre{Système de vecteurs}

\contenu{
\texte{
Soient dans $\Rr^4$ les
vecteurs $v_1=(1,2,3,4)$ et $v_2=(1,-2,3,-4)$. Peut-on
d\'eterminer $x$ et $y$ pour que $(x,1,y,1) \in \text{Vect}\{ v_1, v_2 \}$ ? 
Et pour que $(x,1,1,y) \in \text{Vect}\{v_1,v_2 \}$ ?
}
\indication{On ne peut pas pour le premier, mais on peut pour le second.}
\reponse{
\begin{align*}
& (x,1,y,1) \in \text{Vect}\{v_1,v_2\} \\ 
\iff& \exists \lambda,\mu\in \Rr \qquad (x,1,y,1) = \lambda(1,2,3,4)+\mu(1,-2,3,-4) \\
\iff& \exists \lambda,\mu\in \Rr \qquad (x,1,y,1) = (\lambda,2\lambda,3\lambda,4\lambda)+(\mu,-2\mu,3\mu,-4\mu) \\
\iff& \exists \lambda,\mu\in \Rr \qquad (x,1,y,1) = (\lambda+\mu,2\lambda-2\mu,3\lambda+3\mu,4\lambda-4\mu) \\
\implies& \exists \lambda,\mu\in \Rr \qquad 1 = 2(\lambda-\mu) \text{ et } 1=4(\lambda-\mu) \\
\implies& \exists \lambda,\mu\in \Rr \qquad \lambda-\mu= \frac 12 \text{ et } \lambda-\mu=\frac14 \\
\end{align*}
Ce qui est impossible (quelque soient $x,y$). Donc on ne peut pas trouver de tels $x,y$.
On fait le m\^eme raisonnement : 
\begin{align*}
& (x,1,1,y) \in \text{Vect}\{v_1,v_2\} \\ 
iff&\ \exists \lambda,\mu\in \Rr \qquad (x,1,1,y) = (\lambda+\mu,2\lambda-2\mu,3\lambda+3\mu,4\lambda-4\mu) \\
\iff& \exists \lambda,\mu\in \Rr \qquad 
\begin{cases}
  x &= \lambda + \mu \\
  1  &= 2\lambda -2\mu \\
  1 &= 3\lambda +3\mu \\
  y &= 4\lambda -4 \mu \\
 \end{cases} \\
\iff& \exists \lambda,\mu\in \Rr \qquad 
\begin{cases}
  \lambda &= \frac 5{12} \\
  \mu &=  -\frac1{12} \\
  x &= \frac 13 \\
  y &=  2 \\  
\end{cases}.\\
\end{align*}  
Donc le seul vecteur $(x,1,1,y)$ qui convienne est $(\frac13,1,1,2)$.
}
}
