\uuid{pxB2}
\exo7id{6014}
\auteur{quinio}
\datecreate{2011-05-20}
\isIndication{false}
\isCorrection{true}
\chapitre{Probabilité discrète}
\sousChapitre{Lois de distributions}

\contenu{
\texte{
Une usine fabrique des billes de diamètre 8mm. Les erreurs d'usinage
provoquent des variations de diamètre.
On estime, sur les données antérieures, que l'erreur est une
variable aléatoire qui obeit à une loi normale les paramètres étant :
moyenne: $0$mm, écart-type: $0.02$mm. 
On rejette les pièces dont le diamètre n'est pas compris entre $7.97$mm et $8.03$mm.
Quelle est la proportion de billes rejetées?
}
\reponse{
La probabilité qu'une bille soit rejetée est, en notant $D$ la
variable aléatoire <<diamètre>>, $p=1-P[7.97\leq D\leq 8.03]$.
Or $P[7.97\leq D\leq 8.03]=P[-\frac{0.03}{0.02}\leq \frac{D-8}{0.02}\leq 
\frac{0.03}{0.02}]=F(1.5)-F(-1.5)=0.866\,4$.
La proportion de billes rejetées est donc $p=13.4\%$.
}
}
