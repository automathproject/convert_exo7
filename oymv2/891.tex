\uuid{891}
\auteur{cousquer}
\datecreate{2003-10-01}

\contenu{
\texte{
Dire si les objets suivants sont des espaces vectoriels~:
}
\begin{enumerate}
    \item \question{L'ensemble des fonctions réelles sur
$\lbrack0,1 \rbrack$, continues, positives ou nulles, pour
l'addition et le produit par un réel.}
    \item \question{L'ensemble des fonctions réelles sur ${ \bf R}$~ vérifiant
$\lim_{x \to+\infty} f(x)=0$ pour les mêmes opérations.}
    \item \question{L'ensemble des solutions $(x_1,x_2,x_3)$ du système~:
$\left\{\begin{array}{rcl}
    2x_1-x_2+x_3 &=&0 \\
    x_1-4x_2+7x_3 &=&0 \\ 
    x_1+3x_2-6x_3 &=&0.
\end{array}\right.$}
    \item \question{L'ensemble des fonctions continues sur $\lbrack0,1 \rbrack$
vérifiant $f(1/2)=0$.}
    \item \question{L'ensemble $\mathbb{R}_+^*$ pour les opérations $x \oplus y=xy$ et 
$\lambda\cdot x=x^{\lambda}$, $(\lambda\in\mathbb{R})$.}
    \item \question{L'ensemble des fonctions impaires sur $\mathbb{R}$.}
    \item \question{L'ensemble des fonctions sur $\lbrack a,b \rbrack$ continues, vérifiant
$f(a)=7f(b)+\int_a^b t^3f(t)\,dt$.}
    \item \question{L'ensemble des fonctions sur $\mathbb{R}$ qui sont nulle en~$1$
ou nulle en~$4$.}
    \item \question{L'ensemble des fonctions sur $\mathbb{R}$ qui peuvent s'écrire comme somme
d'une fonction nulle en~$1$ et d'une fonction nulle en~$4$. Identifier cet
ensemble.}
    \item \question{L'ensemble des polynômes de degré exactement~$n$.}
    \item \question{L'ensemble des fonctions de classe $C^2$ vérifiant $f''+\omega^2f=0$.}
    \item \question{L'ensemble des fonctions sur $\mathbb{R}$ telles que $f(3)=7$.}
    \item \question{L'ensemble des primitives de la fonction $xe^x$ sur $\mathbb{R}$.}
    \item \question{L'ensemble des nombres complexes d'argument $\pi/4+k\pi$,
$(k\in\mathbb{Z})$.}
    \item \question{L'ensemble des points $(x,y)$ de $\mathbb{R}^2$,  vérifiant
$\sin(x+y)=0$.}
    \item \question{L'ensemble des vecteurs $(x,y,z)$ de $\mathbb{R}^3$ orthogonaux
au vecteur $(-1,3,-2)$.}
    \item \question{L'ensemble des fonctions continues sur $\lbrack0,1 \rbrack$
vérifiant $\int_0^1\sin x f(x)\,dx=0$.}
    \item \question{L'ensemble des polynômes ne comportant pas de terme de degré~$7$.}
    \item \question{L'ensemble des fonctions paires sur $\mathbb{R}$.}
\end{enumerate}
}
