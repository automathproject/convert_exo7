\uuid{27}
\auteur{bodin}
\datecreate{1998-09-01}

\contenu{
\texte{
Calculer les racines carr\'ees de $1,\ i,\ 3+4i,\
8-6i,$ et $7+24i$.
}
\indication{Pour $z= a+ib$ on cherche $\omega = \alpha + i \beta$ tel que  $(\alpha +i \beta)^2 = a+ib$.
Développez et indentifiez. Utilisez aussi que $|\omega|^2 = |z|$.}
\reponse{
\textbf{Racines carr\'ees.} Soit $z= a+ib$ un nombre complexe avec
$a,b \in \Rr$ ; nous cherchons les complexes $\omega \in \Cc$ tels
que $\omega^2 = z$. \'Ecrivons $\omega = \alpha + i \beta$. Nous
raisonnons par \'equivalence :
\begin{align*}
\omega^2 = z    & \Leftrightarrow (\alpha +i \beta)^2 = a+ib \\
        & \Leftrightarrow \alpha^2-\beta^2+2i\alpha\beta = a +i b \\
\intertext{Soit en identifiant les parties r\'eelles entre elles ainsi que les parties imaginaires :}\\
        & \Leftrightarrow
              \begin{cases}
                 \alpha^2-\beta^2 = a \\
                 2\alpha\beta = b
              \end{cases} \\
\intertext{Sans changer l'\'equivalence nous rajoutons la
condition $|\omega|^2 = |z|$.} & \Leftrightarrow
  \begin{cases}
     \alpha^2 + \beta^2 = \sqrt{a^2+b^2} \\
     \alpha^2-\beta^2 = a \\
     2\alpha\beta = b
  \end{cases} \\
\intertext{Par somme et différence des deux premières lignes :}
& \Leftrightarrow
  \begin{cases}
     \alpha^2  =\frac{a+ \sqrt{a^2+b^2}}{2} \\
     \beta^2 =  \frac{-a+ \sqrt{a^2+b^2}}{2} \\
     2\alpha\beta = b
  \end{cases} \\
& \Leftrightarrow
  \begin{cases}
     \alpha  =\pm \sqrt{\frac{a+ \sqrt{a^2+b^2}}{2}} \\
     \beta = \pm \sqrt{ \frac{-a+ \sqrt{a^2+b^2}}{2}} \\
     \alpha\beta \text{ est du m\^eme signe que } b
  \end{cases} \\
\end{align*}
Cela donne deux couples $(\alpha,\beta)$ de solutions et donc deux
racines carr\'ees (opposées) $\omega = \alpha + i\beta$ de $z$.

En pratique on r\'ep\`ete facilement ce raisonnement, par exemple
pour  $z = 8-6i$,
\begin{align*}
\omega^2 = z
&\Leftrightarrow (\alpha +i \beta)^2 = 8-6i\\
&\Leftrightarrow \alpha^2-\beta^2+2i\alpha\beta = 8-6i\\
&\Leftrightarrow
  \begin{cases}
     \alpha^2-\beta^2 = 8\\
     2\alpha\beta = -6
  \end{cases} \\
&\Leftrightarrow
  \begin{cases}
     \alpha^2 + \beta^2 = \sqrt{8^2+(-6)^2} = 10 \text{ le module de $z$} \\
     \alpha^2-\beta^2 = 8 \\
     2\alpha\beta = -6
  \end{cases}\\
&\Leftrightarrow
  \begin{cases}
     2\alpha^2  =18 \\
     \beta^2 = 1 \\
     2\alpha\beta = -6
  \end{cases}\\
&\Leftrightarrow
  \begin{cases}
     \alpha  =\pm \sqrt{9} = \pm 3 \\
     \beta = \pm 1 \\
     \alpha \text{ et } \beta \text{ de signes oppos\'es}
  \end{cases} \\
&\Leftrightarrow
  \begin{cases}
     \ \ & \alpha  = 3 \text{ et } \beta = - 1\\
     \text{ou}&\\
    \ \ &\alpha  = -3 \text{ et } \beta = +1 \\
\end{cases} \\
\end{align*}

Les racines de $ z = 8-6i$ sont donc $\omega_1 = 3-i$ et $\omega_2 = -\omega_1 =
-3+i$.

Pour les autres : 
\begin{itemize}
 \item Les racines carrées de $1$ sont : $+1$ et $-1$.
 \item  Les racines carrées de $i$ sont : $\frac{\sqrt 2}{2}(1+i)$ et $-\frac{\sqrt 2}{2}(1+i)$.
 \item  Les racines carrées de $3+4i$ sont : $2+i$ et $-2-i$. 
 \item  Les racines carrées de $7+24i$ sont : $4+3i$ et $-4-3i$. 
\end{itemize}
}
}
