\uuid{dJoJ}
\exo7id{6687}
\auteur{queffelec}
\datecreate{2011-10-16}
\isIndication{false}
\isCorrection{false}
\chapitre{Formule de Cauchy}
\sousChapitre{Formule de Cauchy}

\contenu{
\texte{

}
\begin{enumerate}
    \item \question{Soit $P$ un polynôme qui ne s'annule pas sur le cercle $\vert z\vert=1$.
Montrer que le nombre de zéros de $P$ à l'intérieur du cercle unité est
$${1\over 2\pi}\left[ \mathrm{Arg}{P(e^{i\theta })}\right] _0^{2\pi}$$
(variation d'une détermination continue de l'argument sur le cercle
unité). On utilisera le théorème de d'Alembert pour factoriser
$P$ en facteurs de degré 1, puis on considèrera l'indice de chacune des
racines par rapport au cercle.}
    \item \question{Soit $P$ un polynôme n'ayant aucun zéro sur le cercle $\vert z\vert=1$
et ayant exatement $k$ racines (comptées avec multiplicité) à l'intérieur du
cercle unité. Montrer que la fonction
$$\theta \mapsto\Re P(e^{i\theta })$$
s'annule au moins $2k$ fois pour $\theta \in [0,2\pi]$ (indication : étudier
les zéros de la fonction $\cos{\mathrm{Arg}{P(e^{i\theta })}}$, où $\mathrm{Arg}$ est
une détermination continue de l'argument).}
\end{enumerate}
}
