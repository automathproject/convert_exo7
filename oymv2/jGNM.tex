\uuid{jGNM}
\exo7id{6045}
\auteur{queffelec}
\datecreate{2011-10-16}
\isIndication{false}
\isCorrection{false}
\chapitre{Espace topologique, espace métrique}
\sousChapitre{Espace topologique, espace métrique}

\contenu{
\texte{
Dans un espace topologique, on définit la frontière d'une partie $A$
comme étant $\partial A =\overline A \ \backslash \stackrel{\circ}{A}$.
}
\begin{enumerate}
    \item \question{Montrer que $\partial A =\partial (A^c) $ et que $A=\partial A
\Longleftrightarrow$ $A$ fermé d'intérieur vide.}
    \item \question{Montrer que $\partial (\overline A) $ et $\partial (\stackrel{\circ}{A}) $
sont toutes deux incluses dans
$\partial A $, et donner un exemple où ces inclusions sont strictes.}
    \item \question{Montrer que $\partial(A\cup B)\subset \partial A \cup \partial B $, et que
l'inclusion peut
être stricte; montrer qu'il y a égalité lorsque $\overline A\cap \overline B
=\emptyset$ (établir $\stackrel{\circ}{A\cup B}\subset\stackrel{\circ}{A} \cup
\stackrel{\circ}{B}$).

Montrer que $\stackrel{\circ}{A\cup B}=\stackrel{\circ}{A} \cup
\stackrel{\circ}{B}$ reste vrai lorsque $\partial A \cap \partial B =\emptyset$
(raisonner par l'absurde).}
\end{enumerate}
}
