\uuid{7690}
\auteur{mourougane}
\datecreate{2021-08-11}

\contenu{
\texte{
Soit $S$ la surface de $\Rr^3$ d'équation $2(2z^2+y^2)+x=0$.
}
\begin{enumerate}
    \item \question{La surface $S$ est-elle régulière ?}
\reponse{La surface $S$ est le graphe de la fonction $\mathcal{C}^\infty$ $f~:~(y,z)\mapsto -2(2z^2+y^2)$.
 Elle est donc régulière.}
    \item \question{Paramétrer la surface $S$ (de manière polynomiale) en prenant les paramètres $u$ et $v$
 parmi les variables $x$, $y$ et $z$.}
\reponse{La surface $S$ est paramétrée par $F(u,v)=(-2(u^2+2v^2),u,v)$.}
    \item \question{Déterminer une base de l'espace tangent à la surface $S$ en $A(-6,1,-1)$.}
\reponse{Le point $A$ est obtenu au point de paramètre $(1,-1)$
 On calcule $\frac{\partial F}{\partial u}=(-4u,1,0)=(-4,1,0)$ 
 et $\frac{\partial F}{\partial v}=(-8v,0,1)=(8,0,1)$.
 Ces deux vecteurs forment une base de l'espace tangent à $S$ au point de paramètre $(u,v)$.}
    \item \question{Calculer un vecteur normal à la surface $S$ en $A(-6,1,-1)$.}
\reponse{Comme $S$ est la ligne de niveau $0$ de la fonction $\varphi (x,y,z)=2(2z^2+y^2)+x$ sur $\Rr^3$,
 on calcule $(grad \varphi)_{A}=(1,4y,8z)=(1,4,-8)=:n$.
 Ce vecteur est bien orthogonal au plan tangent obtenu à la question 3.}
    \item \question{Le vecteur $V = (27,-29,-1)$ appartient-il au plan tangent à $S$ en $A (-6,1,1)$ ?}
\reponse{Comme $<V,n>=<(27,-29,-1),(1,4,-8)>=-81\not=0$, le vecteur $V$ n'est pas dans le plan tangent à $S$ en $A$.}
\end{enumerate}
}
