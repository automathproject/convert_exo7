\uuid{4910}
\auteur{quercia}
\datecreate{2010-03-17}
\isIndication{false}
\isCorrection{true}
\chapitre{Conique}
\sousChapitre{Parabole}

\contenu{
\texte{
Soit~${\cal P}$ la parabole d'équation $y^2 = 2px$ et $M_0=(x_0,y_0)\in{\cal P}$.
}
\begin{enumerate}
    \item \question{Discuter l'existence et le nombre de points $M\in{\cal P}$ distincts de~$M_0$
    tels que la normale à~${\cal P}$ en~$M$ passe par~$M_0$.}
\reponse{$M = (2pt^2,2pt) \Rightarrow 2t^2+2tt_0+1=0$. Il y a deux solutions si
         $|t_0| >\sqrt2$, une seule si $|t_0|=\sqrt2$ et aucune si
         $|t_0| <\sqrt2$.}
    \item \question{Dans le cas où il y a deux solutions, $M_1$ et $M_2$, trouver le lieu
    géométrique du centre de gravité du triangle $M_0M_1M_2$.}
\reponse{$t_1+t_2 = -t_0$, $t_1^2+t_2^2=t_0^2-1$.
         Centre~: $(4pt_0^2-2p,0)$ (1/2-droite).}
\end{enumerate}
}
