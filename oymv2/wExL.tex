\uuid{wExL}
\exo7id{7350}
\auteur{mourougane}
\datecreate{2021-08-10}
\isIndication{false}
\isCorrection{false}
\chapitre{Groupe, anneau, corps}
\sousChapitre{Autre}

\contenu{
\texte{
Considérons le groupe $(\Zz , +)$.
Le sous-ensemble $4\Zz$ est par définition l'ensemble des multiples
entiers de $4$, autrement dit
$$4\Zz =\{p\in \Zz , \exists m\in \Zz \ \ p=4m\}.$$
}
\begin{enumerate}
    \item \question{Les sous-ensembles $4\Zz$ et $6\Zz$ sont-ils stables pour la loi $+$ ? Sont-ils alors des sous-groupes ?}
    \item \question{Le sous-ensemble $4\Zz \cap 6\Zz$ est-il un sous-groupe ?
(indication~: on pourra énumérer ses premiers éléments positifs)}
    \item \question{Le sous-ensemble $2\Zz \cup 3\Zz$ est-il un sous-groupe ?}
\end{enumerate}
}
