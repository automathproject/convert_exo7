\uuid{71JP}
\exo7id{5786}
\auteur{rouget}
\datecreate{2010-10-16}
\isIndication{false}
\isCorrection{true}
\chapitre{Espace euclidien, espace normé}
\sousChapitre{Problèmes matriciels}

\contenu{
\texte{
Montrer que la matrice de \textsc{Hilbert} $H_n =\left(\frac{1}{i+j-1}\right)_{1\leqslant i,j\leqslant n}$ est définie positive.
}
\reponse{
La matrice $H_n$ est symétrique réelle. Soit $X=(x_i)_{1\leqslant i\leqslant n}\in\mathcal{M}_{n,1}(\Rr)$.

\begin{align*}\ensuremath
{^t}X H_n X&=\sum_{1\leqslant i,j\leqslant n}^{}\frac{x_ix_j}{i+j-1}=\sum_{1\leqslant i,j\leqslant n}^{}x_ix_j\int_{0}^{1}t^{i+j-2}\;dt =\int_{0}^{1}\left(\sum_{1\leqslant i,j\leqslant n}^{}x_ix_j t^{i+j-2}\right)dt\\
 &=\int_{0}^{1}\left(\sum_{i=1}^{n}x_it^{i-1}\right)^2dt\geqslant0 .
\end{align*}

De plus, si $X\neq0$, le polynôme $\sum_{i=1}^{n}x_iY^{i-1}$ n'est pas le polynôme nul et donc, puisqu'un polynôme non nul admet un nombre fini de racines, la fonction $t\mapsto\left(\sum_{i=1}^{n}x_it^{i-1}\right)^2$. Ainsi, la fonction $t\mapsto\left(\sum_{i=1}^{n}x_it^{i-1}\right)^2$ est continue positive et non nulle sur $[0,1]$ et on en déduit que $\int_{0}^{1}\left(\sum_{i=1}^{n}x_it^{i-1}\right)^2\;dt>0$. On a montré que $\forall X\in\mathcal{M}_{n,1}(\Rr)\setminus\{0\}$, ${^t}X H_n X > 0$ et donc que

\begin{center}
\shadowbox{
la matrice $H_n$ est symétrique définie positive.
}
\end{center}
}
}
