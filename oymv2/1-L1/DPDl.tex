\uuid{DPDl}
\exo7id{124}
\auteur{bodin}
\datecreate{1998-09-01}
\isIndication{false}
\isCorrection{true}
\chapitre{Logique, ensemble, raisonnement}
\sousChapitre{Ensemble}

\contenu{
\texte{
Soient $E$ et $F$ deux ensembles, $f:E\rightarrow F$. D\'emontrer que :\\
$\forall A,B \in \mathcal{P}(E) \quad (A\subset B)\Rightarrow (f(A)\subset f(B))$,\\
$\forall A,B \in \mathcal{P}(E) \quad f(A\cap B)\subset f(A)\cap f(B)$,\\
$\forall A,B \in \mathcal{P}(E) \quad f(A\cup B) = f(A)\cup f(B)$,\\
$\forall A,B \in \mathcal{P}(F) \quad f^{-1}(A\cup B) = f^{-1}(A)\cup f^{-1}(B)$,\\
$\forall A \in \mathcal{P}(F) \quad f^{-1}(F\setminus A)=E\setminus f^{-1}(A)$.
}
\reponse{
Montrons quelques assertions.

$f(A\cap B) \subset f(A)\cap f(B)$.\\
Si $y\in f(A\cap B)$, il existe $x\in A\cap B$ tel que $y=f(x)$,
or $x\in A$ donc $y=f(x) \in f(A)$ et de m\^eme $x\in B$ donc
$y\in f(B)$. D'o\`u $y\in f(A)\cap f(B)$. Tout \'el\'ement de
$f(A\cap B)$ est un \'el\'ement de $f(A)\cap f(B)$ donc $f(A\cap
B) \subset f(A)\cap f(B)$.

Remarque : l'inclusion r\'eciproque est fausse. Exercice : trouver
un contre-exemple.

\bigskip

$f^{-1}(F\setminus A) = E\setminus f^{-1}(A)$.\\
\begin{align*}
x\in f^{-1}(F\setminus A) &\Leftrightarrow f(x) \in F\setminus A\\
&\Leftrightarrow f(x) \notin A\\
&\Leftrightarrow x \notin f^{-1}(A) \quad \text{ car } f^{-1}(A) = \{ x\in E \ / \ f(x) \in A \}\\
&\Leftrightarrow x\in E\setminus f^{-1}(A)\\
\end{align*}
}
}
