\uuid{9S9O}
\exo7id{3361}
\auteur{quercia}
\datecreate{2010-03-09}
\isIndication{false}
\isCorrection{false}
\chapitre{Matrice}
\sousChapitre{Inverse, méthode de Gauss}

\contenu{
\texte{
Soit $M \in \mathcal{M}_n(\Q)$. Comparer les énoncés :
\par {\bf1:} $M$ est inversible dans $\mathcal{M}_n(\Q)$.
\par {\bf2:} $M$ est inversible dans $\mathcal{M}_n(\C)$.
}
}
