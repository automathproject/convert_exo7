\uuid{cPQN}
\exo7id{986}
\auteur{cousquer}
\datecreate{2003-10-01}
\isIndication{false}
\isCorrection{false}
\chapitre{Espace vectoriel}
\sousChapitre{Base}

\contenu{
\texte{
Étudier l'indépendance linéaire des listes de vecteurs suivantes,
et trouver à chaque fois une base du sous-espace engendré.
}
\begin{enumerate}
    \item \question{$(1,0,1)$, $(0,2,2)$, $(3,7,1)$ dans $\mathbb{R}^3$.}
    \item \question{$(1,0,0)$, $(0,1,1)$, $(1,1,1)$ dans $\mathbb{R}^3$.}
    \item \question{$(1,2,1,2,1)$, $(2,1,2,1,2)$, $(1,0,1,1,0)$, $(0,1,0,0,1)$ dans $\mathbb{R}^5$.}
    \item \question{$(2,4,3,-1,-2,1)$, $(1,1,2,1,3,1)$, $(0,-1,0,3,6,2)$ dans $\mathbb{R}^6$.}
    \item \question{$(2,1,3,-1,4,-1)$, $(-1,1,-2,2,-3,3)$, $(1,5,0,4,-1,7)$ dans $\mathbb{R}^6$.}
\end{enumerate}
}
