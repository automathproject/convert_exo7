\uuid{AY3t}
\exo7id{199}
\auteur{bodin}
\datecreate{1998-09-01}
\isIndication{true}
\isCorrection{true}
\chapitre{Injection, surjection, bijection}
\sousChapitre{Bijection}

\contenu{
\texte{
Soit $f: [0,1] \rightarrow [0,1]$ telle que
$$f(x) =    \begin{cases}
            x     & \text{si}\ x\in[0,1]\cap\Qq,\\
            1-x  & \text{sinon.}
        \end{cases}  $$
D\'emontrer que $f \circ f= id$.
}
\indication{$id$ est l'application identité définie par $id(x)=x$ pour tout $x\in[0,1]$.
Donc $f \circ f= id$ signifie $f\circ f(c) = x$ pour tout $x\in[0,1]$.}
\reponse{
Soit $x\in [0,1]\cap\Qq$ alors $f(x) = x$ donc $f \circ f (x) = f(x) = x$.
Soit $x\notin [0,1]\cap\Qq$ alors $f(x) = 1-x$ donc $f \circ f (x) = f(1-x)$,
mais $1-x \notin [0,1]\cap\Qq$ (vérifiez-le !) donc  $f \circ f (x) = f(1-x) = 1 - (1-x) = x$.
Donc pour tout $x \in [0,1]$ on a  $f \circ f (x) = x$. Et donc $f \circ f= id$.
}
}
