\uuid{JJJg}
\exo7id{80}
\auteur{cousquer}
\datecreate{2003-10-01}
\isIndication{true}
\isCorrection{true}
\chapitre{Nombres complexes}
\sousChapitre{Trigonométrie}

\contenu{
\texte{
En utilisant les nombres complexes, calculer $\cos 5\theta$ et
$\sin5\theta$ en fonction de $\cos\theta$ et $\sin\theta$.
}
\indication{Appliquer deux fois la formule de Moivre en remarquant
$e^{i5\theta}=(e^{i\theta})^5$.}
\reponse{
Nous avons par la formule de Moivre
$$\cos5\theta+i\sin5\theta=e^{i5\theta}=(e^{i\theta})^5=
(\cos\theta+i\sin\theta)^5.$$
On d\'eveloppe ce dernier produit, puis on identifie parties r\'eelles
et parties imaginaires. On obtient :
\begin{eqnarray*}
\cos 5\theta & = & \cos^5\theta -10\cos^3\theta \sin^2\theta
                    +5\cos\theta \sin^4\theta\\
\sin 5\theta & = & 5\cos^4\theta \sin \theta -10\cos^2\theta \sin^3\theta
                   +\sin^5\theta 
\end{eqnarray*}

Remarque : Gr\^ace \`a la formule $\cos^2\theta +\sin^2\theta =1$, on pourrait
 continuer les
calculs et exprimer $\cos 5\theta $ en fonction de $\cos\theta $, et
$\sin 5\theta $ en fonction de $\sin\theta $.
}
}
