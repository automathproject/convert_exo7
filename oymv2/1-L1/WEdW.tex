\uuid{WEdW}
\exo7id{56}
\auteur{bodin}
\datecreate{1998-09-01}
\isIndication{false}
\isCorrection{true}
\chapitre{Nombres complexes}
\sousChapitre{Racine n-ieme}

\contenu{
\texte{

}
\begin{enumerate}
    \item \question{Soient $z_1$, $z_2$, $z_3$ trois nombres complexes distincts
        ayant le m\^eme cube.

        Exprimer $z_2$ et $z_3$ en fonction de $z_1$.}
\reponse{$z_{1}\neq0$ car sinon on aurait $z_{1}=z_{2}=z_{3}=0$. Ainsi
$(\frac{z_{2}}{z_{1}})^3=(\frac{z_{3}}{z_{1}})^3=1$. Comme les
trois nombres $1,(\frac{z_{2}}{z_{1}})$ et $(\frac{z_{3}}{z_{1}})$
sont distincts on en d\'eduit que ce sont les trois racines
cubiques de 1. Ces racines sont $1, j=e^{\frac{2i\pi}{3}}$ et $
j^2=e^{-\frac{2i\pi}{3}}$. A une permutation pr\`es des indices 2
et 3 on a donc :
$$
  z_{2}=jz_{1} \qquad\text{ et }\qquad z_{3}=j^2z_{1}.
$$}
    \item \question{Donner, sous forme polaire, les solutions dans $\Cc$ de :
        $$ z^6+(7-i)z^3 -8 -8i = 0.$$
        (Indication : poser $Z=z^3$ ; calculer $(9+i)^2$)}
\reponse{Soit $z\in\C$. On a les \'equivalences suivantes :
\begin{align*}
z^6+(7-i)z^3-8-8i=0 &\Leftrightarrow z^3\text{ est solution de }Z^2+(7-i)Z-8-8i=0\\
\intertext{ Etudions l'\'equation $Z^2+(7-i)Z-8-8i=0$.
$\Delta=(7-i)^2+4(8+8i)=80+18i=(9+i)^2$. Les solutions sont donc
$-8$ et $1+i$. Nous pouvons reprendre notre suite d'\'equivalences
: }
z^6+(7-i)z^3-8-8i=0 &\Leftrightarrow z^3\in\{-8,1+i\} \\
  &\Leftrightarrow z^3=(-2)^3 \quad\text{ ou }\quad z^3=(\sqrt[6]{2}e^{i\frac{\pi}{12}})^3\\
  &\Leftrightarrow z\in\{-2,
              -2e^{\frac{2i\pi}{3}},
              -2e^{-\frac{2i\pi}{3}}\}
         \text{ ou }
        z\in\{\sqrt[6]{2}e^{i\frac{\pi}{12}},
              \sqrt[6]{2}e^{i\frac{9\pi}{12}},
              \sqrt[6]{2}e^{i\frac{17\pi}{12}}\}\\
  &\Leftrightarrow z\in\{-2,
               2e^{\frac{i\pi}{3}},
               2e^{-\frac{i\pi}{3}},
              \sqrt[6]{2}e^{i\frac{\pi}{12}},
              \sqrt[6]{2}e^{i\frac{3\pi}{4}},
              \sqrt[6]{2}e^{i\frac{17\pi}{12}}\}.
\end{align*}
L'ensemble des solutions est donc :
           $$\{-2,
               2e^{\frac{i\pi}{3}},
               2e^{-\frac{i\pi}{3}},
              \sqrt[6]{2}e^{i\frac{\pi}{12}},
              \sqrt[6]{2}e^{i\frac{3\pi}{4}},
              \sqrt[6]{2}e^{i\frac{17\pi}{12}}\}.$$}
\end{enumerate}
}
