\uuid{pnMd}
\exo7id{5140}
\auteur{rouget}
\datecreate{2010-06-30}
\isIndication{false}
\isCorrection{true}
\chapitre{Dénombrement}
\sousChapitre{Binôme de Newton et combinaison}

\contenu{
\texte{
Soit $(n,a,b)\in\Nn^*\times]0,+\infty[\times]0,+\infty[$. Quel est le plus grand terme du développement de $(a+b)^n$~?
}
\reponse{
Soit $n$ un entier naturel non nul. Le terme général du développement de $(a+b)^n$ est $u_k=\binom{n}{k}a^kb^{n-k}$, $0\leq
k\leq n$. Pour $0\leq k\leq n-1$, on a~:

$$\frac{u_{k+1}}{u_k}=\frac{\binom{n}{k+1}a^{k+1}b^{n-k-1}}{\binom{n}{k}a^kb^{n-k}}=\frac{n-k}{k+1}\frac{a}{b}.$$

Par suite,

$$\frac{u_{k+1}}{u_k}>1\Leftrightarrow\frac{n-k}{k+1}\frac{a}{b}>1\Leftrightarrow(n-k)a>(k+1)b\Leftrightarrow k<\frac{na-b}{a+b}.$$

\begin{itemize}
\item[1er cas.] Si $\frac{na-b}{a+b}>n-1$ (ce qui équivaut à $n<\frac{a}{b}$), alors la suite $(u_k)_{0\leq k\leq n}$
est strictement croissante et le plus grand terme est le dernier~:~$a^n$.

\item[2ème cas.] Si $\frac{na-b}{a+b}\leq0$ (ce qui équivaut à $n\leq\frac{b}{a}$), alors la suite $(u_k)_{0\leq k\leq
n}$ est strictement décroissante et le plus grand terme est le premier~:~$b^n$.

\item[3ème cas.] Si $0<\frac{na-b}{a+b}\leq n-1$. Dans ce cas, la suite est strictement croissante puis éventuellement
momentanément constante, suivant que $\frac{na-b}{a+b}$ soit un entier ou non, puis strictement décroissante (on dit
que la suite u est unimodale).

Si $\frac{na-b}{a+b}\notin\Nn$, on pose $k=E(\frac{na-b}{a+b})+1$, la suite $u$ croit strictement jusqu'à ce rang
puis redécroit strictement. Le plus grand des termes est celui d'indice $k$, atteint une et une seule fois.

Si $\frac{na-b}{a+b}\in\Nn$, le plus grand des termes est atteint deux fois à l'indice $k$ et à l'indice $k+1$.
\end{itemize}
}
}
