\uuid{6Rwu}
\exo7id{5265}
\auteur{rouget}
\datecreate{2010-07-04}
\isIndication{false}
\isCorrection{true}
\chapitre{Matrice}
\sousChapitre{Autre}

\contenu{
\texte{
Soient $I=\left(
\begin{array}{cc}
1&0\\
0&1
\end{array}
\right)$ et $J=\left(
\begin{array}{cc}
1&1\\
0&1
\end{array}
\right)$ puis $E=\{M(x,y)=xI+yJ,\;(x,y)\in\Rr^2\}$.
}
\begin{enumerate}
    \item \question{Montrer que $(E,+,.)$ est un sous-espace vectoriel de $\mathcal{M}_2(\Rr)$. Déterminer une base de $E$ et sa dimension.}
    \item \question{Montrer que $(E,+,\times)$ est un anneau commutatif.}
    \item \question{Quels sont les inversibles de $E$~?}
    \item \question{Résoudre dans $E$ les équations suivantes~:

$$a)\;X^2=I\quad b)\;X^2=0\quad c)\;X^2 = X.$$}
    \item \question{Calculer $(M(x,y))^n$ pour $n$ entier naturel non nul.}
\reponse{
$E=\mbox{Vect}(I,J)$. Donc, $E$ est un sous-espace vectoriel de $\mathcal{M}_2(\Rr)$. La famille $(I,J)$ est clairement libre et donc est une base de $E$. Par suite, $\mbox{dim}E=2$.
$J^2=\left(
\begin{array}{cc}
1&1\\
0&1
\end{array}
\right)
\left(
\begin{array}{cc}
1&1\\
0&1
\end{array}
\right)=\left(
\begin{array}{cc}
1&2\\
0&1
\end{array}
\right)=2J-I$. Plus généralement, pour $(x,y,x',y')\in\Rr^4$,

$$M(x,y)M(x',y')=(xI+yJ)(x'I+y'J)=xx'I+(xy'+yx')J+yy'J^2=(xx'-yy')I+(xy'+yx'+2yy')J\;(*).$$

Montrons alors que $(E,+,\times)$ est un sous-anneau de $(\mathcal{M}_2(\Rr),+,\times)$.

$E$ contient $I=1.I+0.J$. $(E,+)$ est un sous-groupe de $(\mathcal{M}_2(\Rr),+)$ et, d'après $(*)$, $E$ est stable pour $\times$. Donc, $(E,+,\times)$ est un sous-anneau de $(\mathcal{M}_2(\Rr),+,\times)$.
Soit $((x,y),(x',y'))\in(\Rr^2)^2$.

$$M(x,y)M(x',y')=I\Leftrightarrow(xx'-yy')I+(xy'+yx'+2yy')J=I\Leftrightarrow
\left\{
\begin{array}{l}
xx'-yy'=1\\
yx'+(x+2y)y'=0
\end{array}
\right..$$

Le déterminant de ce dernier système d'inconnues $x'$ et $y'$ vaut $x(x+2y)+y^2=x^2+2xy+y^2=(x+y)^2$. Si $y\neq-x$, ce système admet un et seule couple solution. Par suite, si $y\neq -x$, il existe $(x',y')\in\Rr^2$ tel que $M(x,y)M(x',y')=I$. Dans ce cas, la matrice $M(x,y)$ est inversible dans $E$.

Si $y=-x$, le système s'écrit $\left\{
\begin{array}{l}
x(x'+y')=1\\
-x(x'+y')=0
\end{array}
\right.$ et n'a clairement pas de solution.
\begin{enumerate}
Soit $(x,y)\in\Rr^2$.

$$M(x,y)^2=I\Leftrightarrow\left\{
\begin{array}{l}
x^2-y^2=1\\
2y(x+y)=0
\end{array}
\right.\Leftrightarrow\left\{
\begin{array}{l}
y=0\\
x^2=1
\end{array}
\right.
\;\mbox{ou}
\left\{
\begin{array}{l}
x^2-y^2=1\\
x+y=0
\end{array}
\right.
\Leftrightarrow\left\{
\begin{array}{l}
y=0\\
x=1
\end{array}
\right.
\;\mbox{ou}\;\left\{
\begin{array}{l}
y=0\\
x=-1
\end{array}
\right..$$

Dans $E$, l'équation $X^2=I$ admet exactement deux solutions à savoir $I$ et $-I$.
Soit $(x,y)\in\Rr^2$.

$$M(x,y)^2=0\Leftrightarrow\left\{
\begin{array}{l}
x^2-y^2=0\\
2y(x+y)=0
\end{array}
\right.\Leftrightarrow\left\{
\begin{array}{l}
y=0\\
x^2=0
\end{array}
\right.
\;\mbox{ou}
\left\{
\begin{array}{l}
y=-x\\
0=0
\end{array}
\right.
\Leftrightarrow y=-x.$$

Dans $E$, l'équation $X^2=0$ admet pour solutions les matrices de la forme $\lambda(J-I)=\left(
\begin{array}{cc}
0&\lambda\\
0&0
\end{array}
\right)$, $\lambda\in\Rr$.
Soit $(x,y)\in\Rr^2$.

\begin{align*}\ensuremath
M(x,y)^2=M(x,y)&\Leftrightarrow\left\{
\begin{array}{l}
x^2-y^2=x\\
2y(x+y)=y
\end{array}
\right.
\Leftrightarrow\left\{
\begin{array}{l}
x^2-y^2=x\\
y(2x+2y-1)=0
\end{array}
\right.
\\
 &
\Leftrightarrow\left\{
\begin{array}{l}
y=0\\
x^2=x
\end{array}
\right.
\;\mbox{ou}\;
\left\{
\begin{array}{l}
y=-x+\frac{1}{2}\\
x^2-(-x+\frac{1}{2})^2=x
\end{array}
\right.
\\
 &\Leftrightarrow\left\{
\begin{array}{l}
y=0\\
x=0
\end{array}
\right.
\;\mbox{ou}\;\left\{
\begin{array}{l}
y=0\\
x=1
\end{array}
\right.
\;\mbox{ou}\;\left\{
\begin{array}{l}
\frac{1}{4}=0\\
y=-x+\frac{1}{2}
\end{array}
\right.\Leftrightarrow\left\{
\begin{array}{l}
y=0\\
x=0
\end{array}
\right.
\;\mbox{ou}\;\left\{
\begin{array}{l}
y=0\\
x=1
\end{array}
\right.
.
\end{align*}

Dans $E$, l'équation $X^2=X$ admet exactement deux solutions à savoir $0$ et $I$.
}
\end{enumerate}
}
