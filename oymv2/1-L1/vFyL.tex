\uuid{vFyL}
\exo7id{2913}
\auteur{quercia}
\datecreate{2010-03-08}
\isIndication{false}
\isCorrection{true}
\chapitre{Dénombrement}
\sousChapitre{Cardinal}

\contenu{
\texte{
On doit placer autour d'une table ronde un groupe de $2n$ personnes, $n$ hommes
et $n$ femmes, qui constituent $n$ couples.
Combien existe-t-il de dispositions $\ldots$
}
\begin{enumerate}
    \item \question{au total ?}
\reponse{$(2n)!$.}
    \item \question{en respectant l'alternance des sexes ?}
\reponse{$2(n!)^2$.}
    \item \question{sans s{\'e}parer les couples ?}
\reponse{$2^{n+1}\times n!$.}
    \item \question{en remplissant les deux conditions pr{\'e}c{\'e}dentes ?}
\reponse{$4\times n!$.}
\end{enumerate}
}
