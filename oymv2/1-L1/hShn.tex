\uuid{hShn}
\exo7id{5276}
\auteur{rouget}
\datecreate{2010-07-04}
\isIndication{false}
\isCorrection{true}
\chapitre{Matrice}
\sousChapitre{Inverse, méthode de Gauss}

\contenu{
\texte{
Soit $A=(a_{i,j})_{1\leq i,j\leq n+1}$ définie par $a_{i,j}=0$ si $i>j$ et $a_{i,j}=C_{j-1}^{i-1}$ si $i\leq j$.

Montrer que $A$ est inversible et déterminer son inverse. (Indication~:~considérer l'endomorphisme de $\Rr_n[X]$ qui à un polynôme $P$ associe le polynôme $P(X+1)$).
}
\reponse{
Soit $f$ l'endomorphisme de $\Rr_n[X]$ qui, à un polynôme $P$ de degré inférieur ou égal à $n$, associe le polynôme $P(X+1)$.

Par la formule du binôme de \textsc{Newton}, on voit que $A$ est la matrice de $f$ dans la base canonique $(1,X,...,X^n)$ de $\Rr_n[X]$. $f$ est clairement un automorphisme de $\Rr_n[X]$, sa réciproque étant l'application qui, à un polynôme $P$ associe le polynôme $P(X-1)$.

$A$ est donc inversible et $A^{-1}=(b_{i,j})_{0\leq i,j\leq n}$ où $b_{i,j}=0$ si $i>j$ et $b_{i,j}=(-1)^{i+j}C_{j}^{i}$ si $i\leq j$.
}
}
