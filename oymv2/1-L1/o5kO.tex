\uuid{o5kO}
\exo7id{3192}
\auteur{quercia}
\datecreate{2010-03-08}
\isIndication{false}
\isCorrection{true}
\chapitre{Polynôme, fraction rationnelle}
\sousChapitre{Autre}

\contenu{
\texte{
Soit~$P\in\R[X]$. Montrer~:

($\forall\ x\ge 0,\ P(x) > 0$) $\Leftrightarrow$ ($\exists\ \ell \in\N$  tq $(X+1)^\ell P(X)$ est {\`a} coefficients strictement positifs).
}
\reponse{
Le sens $\Leftarrow$ est trivial. Pour le sens $ \Rightarrow $, il suffit de v{\'e}rifier
la propri{\'e}t{\'e} lorsque $P$ est irr{\'e}ductible, strictement positif sur~$\R^+$, et
le seul cas non trivial est celui o{\`u} $P$ est de la forme~:
$P = (X-a)^2 + b^2$ avec $a>0$, $b>0$. Dans ce cas, le coefficient de~$X^k$
dans $(X+1)^\ell P(X)$ est~:
$C_\ell^k(a^2+b^2) - 2aC_\ell^{k-1} + C_\ell^{k-2}$, en convenant que $C_x^y$
vaut~$0$ si l'on n'a pas $0\le y\le x$. En mettant ce qui peut l'{\^e}tre en facteur
et en ordonnant le reste suivant les puissances de~$k$, on
est rammen{\'e} {\`a} montrer que la quantit{\'e}~:
$$k^2(a^2+b^2+2a+1) - k((a^2+b^2)(2\ell+3) + 2a(\ell+2) + 1) + \ell^2(a^2+b^2)$$
est strictement positive pour tout $k\in[[0,\ell+2]]$ si~$\ell$ est choisi convenablement.
Or le discriminant par rapport {\`a}~$k$ est {\'e}quivalent {\`a} $-4\ell^2(2a+1)$ lorsque
$\ell$ tend vers $+\infty$ donc un tel choix de~$\ell$ est possible.
}
}
