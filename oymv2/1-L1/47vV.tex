\uuid{47vV}
\exo7id{151}
\auteur{bodin}
\datecreate{1998-09-01}
\isIndication{true}
\isCorrection{true}
\chapitre{Logique, ensemble, raisonnement}
\sousChapitre{Absurde et contraposée}

\contenu{
\texte{

}
\begin{enumerate}
    \item \question{Soit $p_{1},p_{2},\ldots ,p_{r}$, $r$ nombres premiers. Montrer que l'entier
$N=p_{1}p_{2}\ldots p_{r}+1$ n'est divisible par aucun des entiers $p_{i}$.}
\reponse{Montrons en fait la contrapos\'ee.

S'il existe $i$ tel que $p_i$ divise $N=p_1p_2 \ldots p_r +1$ ($i$
est fix\'e) alors il existe $k \in \Zz$ tel que $N = kp_i$ donc
$$p_i(k-p_1p_2\ldots p_{i-1}p_{i+1}\ldots p_r) = 1$$
 soit $p_iq = 1$ (avec
$q = k-p_1p_2\ldots p_{i-1}p_{i+1}\ldots p_r $ un nombre entier).
Donc $p_i \in \Zz$ et $1/p_i = q \in \Zz$,  alors $p_i$ vaut $1$
ou $-1$. Et donc $p_i$ n'est pas un nombre premier.

Conclusion : par contraposition il est vrai que $N$ n'est
divisible par aucun des $p_i$}
    \item \question{Utiliser la question pr\'ec\'edente pour montrer par l'absurde qu'il existe une infinit\'e de
nombres premiers.}
\reponse{Raisonnons par l'absurde : s'il n'existe qu'un nombre fini $r$ de nombres premiers $p_1,\ldots,p_r$
alors $N=p_1p_2 \ldots p_r +1$ est un nombre premier car divisible
par aucun nombre premier autre que lui m\^eme (c'est le 1.).

 Mais $N$ est strictement sup\'erieur \`a tous les $p_i$. Conclusion on a construit un
nombre premier $N$ diff\'erent des $p_i$, il y a donc au moins
$r+1$ nombres premiers, ce qui est absurde.}
\indication{Pour la premi\`ere question vous pouvez raisonner par contraposition ou par l'absurde.}
\end{enumerate}
}
