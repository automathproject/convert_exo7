\uuid{5Uro}
\exo7id{5259}
\auteur{rouget}
\datecreate{2010-07-04}
\isIndication{false}
\isCorrection{true}
\chapitre{Matrice}
\sousChapitre{Changement de base, matrice de passage}

\contenu{
\texte{
Soit $u$ l'endomorphisme de $\Rr^3$ dont la matrice dans la base canonique $(i,j,k)$ de $\Rr^3$ est~:

$$M=\left(
\begin{array}{ccc}
0&1&0\\
0&0&1\\
1&-3&3
\end{array}
\right)
.$$
}
\begin{enumerate}
    \item \question{Montrer que $u$ est un automorphisme de $\Rr^3$ et déterminer $u^{-1}$.}
\reponse{$\mbox{rg}u=\mbox{rg}(u(i),u(j),u(k))=rg(u(j),u(k),u(i))$. La matrice de cette dernière famille dans la base $(i,j,k)$ est $\left(
\begin{array}{ccc}
1&0&0\\
0&1&0\\
-3&3&1
\end{array}
\right)$. Cette dernière famille est de rang $3$. Donc, $\mbox{rg}u=3$ et $u$ est bien un automorphisme de $\Rr^3$. Posons $e_1=u(i)$, $e_2=u(j)$ et $e_3=u(k)$.

$$\left\{
\begin{array}{l}
e_1=k\\
e_2=i-3k\\
e_3=j+3k
\end{array}\right.
\Leftrightarrow\left\{
\begin{array}{l}
k=e_1\\
i=3e_1+e_2\\
j=-3e_1+e_3
\end{array}\right.
\Leftrightarrow
\left\{
\begin{array}{l}
u^{-1}(k)=i\\
u^{-1}(i)=3i+j\\
u^{-1}(j)=-3i+k
\end{array}\right.
$$

et 

$$A^{-1}=\mbox{Mat}_{\mathcal{B}}(u^{-1})=\left(
\begin{array}{ccc}
3&-3&1\\
1&0&0\\
0&1&0
\end{array}
\right).$$}
    \item \question{Déterminer une base $(e_1,e_2,e_3)$ de $\Rr^3$ telle que $u(e_1)=e_1$, $u(e_2)=e_1+e_2$ et $u(e_3)=e_2+e_3$.}
\reponse{(Questions 2) et 3)). Posons $e_1=xi+yj+zk$ ($e_1$, $e_2$ et $e_3$ désignent d'autres vecteurs que ceux du 1)).

$$u(e_1)=e_1\Leftrightarrow(u-Id)(e_1)=0\Leftrightarrow\left(
\begin{array}{ccc}
-1&1&0\\
0&-1&1\\
1&-3&2
\end{array}
\right)\left(
\begin{array}{c}
x\\
y\\
z
\end{array}
\right)=\left(
\begin{array}{c}
0\\
0\\
0
\end{array}
\right)\Leftrightarrow
\left\{
\begin{array}{l}
-x+y=0\\
-y+z=0\\
x-3y+2z=0
\end{array}
\right.\Leftrightarrow x=y=z.$$

On prend $e_1=i+j+k$.

Posons $e_2=xi+yj+zk$.

$$u(e_2)=e_1+e_2\Leftrightarrow(u-Id)(e_2)=e_1\Leftrightarrow
\left\{
\begin{array}{l}
-x+y=1\\
-y+z=1\\
x-3y+2z=1
\end{array}
\right.\Leftrightarrow y=x+1\;\mbox{et}\;z=x+2.$$

On prend $e_2=j+2k$.

Posons $e_3=xi+yj+zk$.

$$u(e_3)=e_2+e_3\Leftrightarrow(u-Id)(e_3)=e_2\Leftrightarrow
\left\{
\begin{array}{l}
-x+y=0\\
-y+z=1\\
x-3y+2z=2
\end{array}
\right.\Leftrightarrow y=x\;\mbox{et}\;z=x+1.$$

On prend $e_3=k$.

La matrice de la famille $(e_1,e_2,e_3)$ dans la base $(i,j,k)$ est $P=
\left(
\begin{array}{ccc}
1&0&0\\
1&1&0\\
1&2&1
\end{array}
\right)$. Cette matrice est de rang $3$ et est donc inversible. Par suite $(e_1,e_2,e_3)$ est une base de $\Rr^3$. Enfin,

$$\left\{
\begin{array}{l}
e_1=i+j+k\\
e_2=j+2k\\
e_3=k
\end{array}
\right.
\Leftrightarrow
\left\{
\begin{array}{l}
k=e_3\\
j=e_2-2e_3\\
i=e_1-e_2+e_3
\end{array}
\right.
,$$

et 

$$P^{-1}=\left(
\begin{array}{ccc}
1&0&0\\
-1&1&0\\
1&-2&1
\end{array}
\right).$$}
    \item \question{Déterminer $P$ la matrice de passage de $(i,j,k)$ à $(e_1,e_2,e_3)$ ainsi que $P^{-1}$.}
\reponse{Voir question précédente.}
    \item \question{En déduire $u^n(i)$, $u^n(j)$ et $u^n(k)$ pour $n$ entier relatif.}
\reponse{Soit $T$ est la matrice de $u$ dans la base $(e_1,e_2,e_3)$. $T=
\left(
\begin{array}{ccc}
1&1&0\\
0&1&1\\
0&0&1
\end{array}
\right)$. Les formules de changement de bases s'écrivent $T=P^{-1}AP$ ou encore $A=PTP^{-1}$. Par suite, pour tout relatif $n$, $A^n=PT^nP^{-1}$.

Posons $N=\left(
\begin{array}{ccc}
0&1&0\\
0&0&1\\
0&0&0
\end{array}
\right)
$. On a $N^2=\left(
\begin{array}{ccc}
0&0&1\\
0&0&0\\
0&0&0
\end{array}
\right)$  puis $N^3=0$.

Donc, pour $n$  entier naturel supérieur ou égal à $2$ donné, puisque $I$ et $N$ commutent, la formule du binôme de \textsc{Newton} fournit 

$$T^n=(I+N)^n=I+nN+\frac{n(n-1)}{2}N^2=\left(
\begin{array}{ccc}
1&n&n(n-1)/2\\
0&1&n\\
0&0&1
\end{array}
\right).$$

Cette formule reste claire pour $n=0$ et $n=1$.
Pour $n=-1$, $(I+N)(I-N+N^2)=I+N^3=I$ et donc

$$T^{-1}=(I+N)^{-1}=I-N+N^2=\left(
\begin{array}{ccc}
1&-1&1\\
0&1&-1\\
0&0&1
\end{array}
\right)=\left(
\begin{array}{ccc}
1&-1&\frac{(-1)(-1-1)}{2}\\
0&1&-1\\
0&0&1
\end{array}
\right),$$

et la formule reste vraie pour $n=-1$. Enfin, pour $n$ entier naturel non nul donné, $T^{-n}=(I+nN+\frac{n(n-1)}{2}N^2)^{-1}$ mais $(I+nN+\frac{n(n-1)}{2}N^2)(I-nN+\frac{-n(-n-1)}{2}N^2)=I$ et donc
$T^{-n}=I-nN+\frac{-n(-n-1)}{2}N^2$. Finalement, 

$$\forall n\in\Zz,\;T^n=I+nN+\frac{n(n-1)}{2}N^2=\left(
\begin{array}{ccc}
1&n&n(n-1)/2\\
0&1&n\\
0&0&1
\end{array}
\right).$$

Puis 

\begin{align*}\ensuremath
A^n&=PT^nP^{-1}=\left(
\begin{array}{ccc}
1&0&0\\
1&1&0\\
1&2&1
\end{array}
\right)\left(
\begin{array}{ccc}
1&n&n(n-1)/2\\
0&1&n\\
0&0&1
\end{array}
\right)\left(
\begin{array}{ccc}
1&0&0\\
-1&1&0\\
1&-2&1
\end{array}
\right)\\
 &=\left(
\begin{array}{ccc}
1&n&n(n-1)/2\\
1&n+1&n(n+1)/2\\
1&n+2&(n+1)(n+2)/2
\end{array}
\right)\left(
\begin{array}{ccc}
1&0&0\\
-1&1&0\\
1&-2&1
\end{array}
\right)
\\
 &=\left(
\begin{array}{ccc}
(n-1)(n-2)/2&-n(n-2)&n(n-1)/2\\
n(n-1)/2&-(n-1)(n+1)&n(n+1)/2\\
n(n+1)/2&-n(n+2)&(n+1)(n+2)/2
\end{array}
\right)
\end{align*}

ce qui fournit $u^n(i)$, $u^n(j)$ et $u^n(k)$.}
\end{enumerate}
}
