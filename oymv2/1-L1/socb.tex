\uuid{socb}
\exo7id{330}
\auteur{cousquer}
\datecreate{2003-10-01}
\isIndication{false}
\isCorrection{false}
\chapitre{Arithmétique dans Z}
\sousChapitre{Pgcd, ppcm, algorithme d'Euclide}

\contenu{
\texte{
Dans un pays nommé ASU, dont l'unité monétaire est le rallod, 
la banque nationale émet seulement des 
billets de 95 rallods et des pièces de 14 rallods.
}
\begin{enumerate}
    \item \question{Montrer qu'il est possible de payer n'importe quelle somme 
entière (à condition bien sûr que les deux parties disposent chacune 
d'assez de pièces et de billets).}
    \item \question{On suppose que vous devez payer une somme~$S$, que vous avez une 
quantité illimitée de pièces et de billets, mais que votre créancier 
ne puisse pas rendre la monnaie. Ainsi, il est possible de payer si 
$S=14$, mais pas si $S=13$ ou si $S=15$\dots{} 
Montrer qu'il est toujours possible de payer si $S$ est assez grande.
Quelle est la plus grande valeur de~$S$ telle qu'il soit impossible 
de payer~$S$~?}
\end{enumerate}
}
