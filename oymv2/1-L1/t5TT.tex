\uuid{t5TT}
\exo7id{955}
\auteur{cousquer}
\datecreate{2003-10-01}
\isIndication{false}
\isCorrection{false}
\chapitre{Application linéaire}
\sousChapitre{Image et noyau, théorème du rang}

\contenu{
\texte{
$E$ étant un espace vectoriel de dimension $n$ sur $\mathbb{R}$, $f$ une
application linéaire de~$E$ dans~$E$, construire dans
les trois cas suivants deux applications linéaires bijectives $u$ et $v$ de~$E$
dans~$E$ telles que $f=u-v$.
\begin{itemize}
\item $f$ est bijective.
\item $\mathrm{Ker} f+\mathrm{Im} f=E$.
\item $f$ est quelconque.
\end{itemize}
}
}
