\uuid{MSSM}
\exo7id{5604}
\auteur{rouget}
\datecreate{2010-10-16}
\isIndication{false}
\isCorrection{true}
\chapitre{Matrice}
\sousChapitre{Inverse, méthode de Gauss}

\contenu{
\texte{
Soit $A=(a_{i,j})_{1\leqslant i,j\leqslant n}$ définie par $a_{i,j}=1$ si $i=j$, $j$ si $i=j-1$ et $0$ sinon.
Montrer que $A$ est inversible et calculer $A^{-1}$.
}
\reponse{
$A= \left(
\begin{array}{cccccc}
1&2&0&\ldots&\ldots&0\\
0&1&3&\ddots& &\vdots\\
\vdots&\ddots&\ddots&\ddots&\ddots&\vdots\\
 & & & &\ddots&0\\
\vdots& & &\ddots&\ddots&n-1\\
0&\ldots& &\ldots&0&1
\end{array}
\right)= I + N$ où $N=\left(
\begin{array}{cccccc}
0&2&0&\ldots&\ldots&0\\
0&0&3&\ddots& &\vdots\\
\vdots&\ddots&\ddots&\ddots&\ddots&\vdots\\
 & & & &\ddots&0\\
\vdots& & &\ddots&\ddots&n-1\\
0&\ldots& &\ldots&0&0
\end{array}
\right)$.

$N$ est nilpotente et donc $N^n = 0$. Par suite, 

\begin{center}
$I=I-(-N)^n=(I+N)(I -N + ... +(-N)^{n-1})$.
\end{center}

Ainsi $A$ est inversible à gauche et donc inversible, d'inverse $I -N + ... +(-N)^{n-1}$.

Calcul de $N^p$ pour $1\leqslant p\leqslant n$.

\begin{center}
$N^2 =\left(\sum_{j=2}^{n}jE_{j-1,j}\right)^2=\sum_{2\leqslant j,k\leqslant n}^{}jkE_{j-1,j}E_{k-1,k}=\sum_{j=2}^{n-1}j(j+1)E_{j-1,j}E_{j,j+1}=\sum_{j=3}^{n}j(j-1)jE_{j-2,j}$.
\end{center}

c'est-à-dire $N^2=\left(
\begin{array}{cccccc}
0&0&2\times3&0&\ldots&0\\
\vdots& &\ddots&3\times4&\ddots&\vdots\\
 & & &\ddots&\ddots&0\\
 & & & &\ddots&(n-1)n\\
\vdots& & & & &0\\
0&\ldots& &\ldots&0&0
\end{array}
\right)$.

Ensuite, $N^3=\left(\sum_{j=3}^{n}(j-1)jE_{j-2,j}\right)\left(\sum_{k=2}^{n}kE_{k-1,k}\right)) =\sum_{j=4}^{n}j(j-1)(j-2)E_{j-3,j}$.

Supposons que pour $p$ donné dans $\llbracket1,n-1\rrbracket$, $N^p=\sum_{j=p+1}^{n}j(j-1)...(j-p+1)E_{j-p,j}$.

Alors $N^{p+1}=\left(\sum_{j=p+1}^{n}j(j-1)...(j-p+1)E_{j-p,j}\right)\left(\sum_{k=2}^{n}kE_{k-1,k}\right)=\sum_{j=p+2}^{n}j(j-1)...(j-p)E_{j-p-1,j}$. Ainsi 

\begin{center}
\shadowbox{
$A^{-1}= (a_{i,j})_{1\leqslant i,j\leqslant n}$ où $a_{i,j}= 0$ si $i > j$, $1$ si $i = j$ et $(-1)^{i+j-2}\prod_{k=0}^{j-i-1}(j-k)$ sinon.
}
\end{center}
}
}
