\uuid{OoBa}
\exo7id{5623}
\auteur{rouget}
\datecreate{2010-10-16}
\isIndication{false}
\isCorrection{true}
\chapitre{Matrice}
\sousChapitre{Noyau, image}

\contenu{
\texte{
Soit $H$ un élément de $\mathcal{M}_n(\Cc)$ tel que $\forall A\in\mathcal{M}_n(\Cc),\;\exists\lambda_A\in\Cc/\;HAH=\lambda_AH$. Montrer que $\text{rg}H\leqslant 1$.
}
\reponse{
Soit $r$ le rang de $H$. Il existe deux matrices carrées inversibles $P$ et $Q$ de format $n$ telles que $H =PJ_rQ$ où $J_r=\left(
\begin{array}{cc}
I_r&0\\
0&0
\end{array}
\right)$. L'égalité $HAH =\lambda_AH$ s'écrit après simplifications $J_rQAPJ_r=\lambda_AJ_r$.
Maintenant , quand $A$ décrit $\mathcal{M}_n(\Kk)$, la matrice $B=QAP$ décrit également $\mathcal{M}_n(\Kk)$ (par exemple, l'application qui à $A$ associe $QAP$ est une permutation de $\mathcal{M}_n(\Kk)$ de réciproque l'application qui à $A$ associe $Q^{-1}AP^{-1}$).

L'énoncé s'écrit maintenant de manière plus simple : montrons que $(\forall B\in\mathcal{M}_n(\Kk),\;\exists\lambda_B\in\Kk/\;J_rBJ_r =\lambda_BJ_r)\Rightarrow r\leqslant 1$.

Un calcul par blocs fournit en posant $B=\left(
\begin{array}{cc}
B_1&B_3\\
B_2&B_4
\end{array}
\right)$

\begin{center}
$J_rBJ_r=\left(
\begin{array}{cc}
I_r&0\\
0&0
\end{array}
\right)\left(
\begin{array}{cc}
B_1&B_3\\
B_2&B_4
\end{array}
\right)\left(
\begin{array}{cc}
I_r&0\\
0&0
\end{array}
\right)=\left(
\begin{array}{cc}
B_1&0\\
B_2&0
\end{array}
\right)\left(
\begin{array}{cc}
I_r&0\\
0&0
\end{array}
\right)=\left(
\begin{array}{cc}
B_1&0\\
0&0
\end{array}
\right)$
\end{center} 

Mais si $r\geqslant 2$, il existe des matrices carrées $B_1$ de format $r$ qui ne sont pas des matrices scalaires et donc telles que $B_1$ n'est pas colinéaire à $I_r$. Donc $r\leqslant 1$.
}
}
