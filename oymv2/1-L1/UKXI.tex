\uuid{UKXI}
\exo7id{269}
\auteur{cousquer}
\datecreate{2003-10-01}
\isIndication{false}
\isCorrection{false}
\chapitre{Arithmétique dans Z}
\sousChapitre{Divisibilité, division euclidienne}

\contenu{
\texte{
Soit $n\in\mathbb{N}^*$. Montrer que parmi les trois entiers $n.(n+1)$,
$n.(n+2)$ et $(n+1).(n+2)$, il y en a exactement deux qui sont divisibles par $3$.
}
}
