\uuid{GZJC}
\exo7id{3186}
\auteur{quercia}
\datecreate{2010-03-08}
\isIndication{false}
\isCorrection{true}
\chapitre{Polynôme, fraction rationnelle}
\sousChapitre{Autre}

\contenu{
\texte{

}
\begin{enumerate}
    \item \question{Montrer que pour tout entier $n \in \N$ il existe un unique polyn{\^o}me
     $P_n \in {\Z[X]}$ v{\'e}rifiant :
     $$\forall\ z \in \C^*,\ z^n + z^{-n} = P_n(z + z^{-1}).$$}
    \item \question{D{\'e}terminer le degr{\'e}, le coefficient dominant, et les racines de $P_n$.}
    \item \question{Pour $P \in {\C[X]}$, on note $\tilde P$ le polyn{\^o}me tel que :
     $$\forall\ z \in \C^*,\ P(z) + P(z^{-1}) = \tilde P(z + z^{-1}).$$

     {\'E}tudier l'application $P  \mapsto \tilde P$.}
\reponse{
$P_0(u) = 2$, $P_1(u) = u$, $P_{n+1}(u) = uP_n(u) - P_{n-1}(u)$.
$u_k = 2\cos\Bigl(\frac {(2k+1)\pi}{2n}\Bigr),\ k = 0, \dots, n-1$.
}
\end{enumerate}
}
