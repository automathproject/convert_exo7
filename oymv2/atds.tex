\uuid{atds}
\exo7id{4625}
\auteur{quercia}
\datecreate{2010-03-14}
\isIndication{false}
\isCorrection{true}
\chapitre{Série de Fourier}
\sousChapitre{Calcul de coefficients}

\contenu{
\texte{
Soit $a\in {]-1,1[}$ et $g$ : $x \mapsto\frac{1-a\cos x}{ 1-2a\cos x +a^2}$.
}
\begin{enumerate}
    \item \question{Prouver~: $\forall\ x\in\R,\ g(x)=\sum_{n=0}^{\infty} a^n\cos nx$.}
\reponse{$\sum_{n=0}^\infty a^n\cos nx = \Re\Bigl(\sum_{n=0}^\infty (ae^{ix})^n\Bigr)
    =\Re\Bigl(\frac1{1-ae^{ix}}\Bigr) = g(x)$.}
    \item \question{Quel est le mode de convergence de la série~?}
\reponse{Il y a convergence normale.}
    \item \question{Soit $f : \R \to \C$ continue par morceaux et $2\pi$-périodique.
    Montrer que $h$ : $x \mapsto \int_{t=0}^{2\pi}g(x-t)f(t)\,d t$
    est somme d'une série trigonométrique uniformément convergente.
    Que peut-on déduire pour~$h$~?\vrule height 12pt width 0pt}
\reponse{\begin{align*} h(x)
    &=  \int_{t=0}^{2\pi}\sum_{n=0}^\infty a^n\cos(nx-nt)f(t)\,d t\cr
    &= \sum_{n=0}^\infty \int_{t=0}^{2\pi} a^n\cos(nx-nt)f(t)\,d t\cr
    &= \sum_{n=0}^\infty\Bigl(\cos(nx) \int_{t=0}^{2\pi} a^n\cos(nt)f(t)\,d t + \sin(nx) \int_{t=0}^{2\pi} a^n\sin(nt)f(t)\,d t\Bigr)\cr
    &= \sum_{n=0}^\infty(\pi a^na_n(f)\cos(nx) + \pi a^nb_n(f)\sin(nx)).\cr
    \end{align*}
    Il y a convergence normale car $|a|<1$ et les coefficients de Fourier
    de~$f$ sont bornés.
    On en déduit que $h$ est continue, puis que les coefficients de Fourier de~$h$
    sont $a^na_n(f)$ et $a^nb_n(f)$.}
    \item \question{Soit $\lambda\in\R$. Trouver toutes les fonctions $f : \R \to \C$
    continues par morceaux et $2\pi$-périodiques telles que~:
    $\forall\ x\in\R,\ f(x)=\lambda \int_{t=0}^{2\pi}g(x-t)f(t)\,d t +\sum_{n=1}^{\infty}\frac{\cos nx}{n^2}$.}
\reponse{Les coefficients de Fourier des deux membres doivent être égaux, ce qui
    donne~: $a_n(f)= \frac1{n^2(1-\pi\lambda a^n)}$ et $b_n(f) = 0$ si pour
    tout~$n\in\N^*$ on a $\pi\lambda a^n\ne 1$ (sinon il n'y a pas de solution),
    et $a_0(f) = 0$ si $2\pi\lambda\ne1$, $a_0(f)$ quelconque sinon.
    Réciproquement, en posant ${f(x) = [a_0/2] + \sum_{n=1}^\infty\frac{\cos nx}{n^2(1-\pi\lambda a^n)}}$
    on définit $f$, $2\pi$-périodique continue (la série converge normalement),
    solution de l'équation par égalité des coefficients de Fourier de chaque membre.}
\end{enumerate}
}
