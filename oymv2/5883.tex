\uuid{5883}
\auteur{rouget}
\datecreate{2010-10-16}
\isIndication{false}
\isCorrection{true}
\chapitre{Equation différentielle}
\sousChapitre{Equations différentielles linéaires}

\contenu{
\texte{
Trouver les fonctions $f$ dérivables sur $\Rr$ vérifiant $\forall x\in\Rr$, $f'(x) + f(-x) = e^x$.
}
\reponse{
Soit $f$ une éventuelle solution. $f$ est dérivable sur $\Rr$ et pour tout réel $x$, $f'(x)=-f(-x)+e^{x}$. On en déduit que $f'$ est dérivable sur $\Rr$ ou encore que $f$ est deux fois dérivable sur $\Rr$. En dérivant l'égalité initiale, on obtient pour tout réel $x$

\begin{center}
$f''(x)=f'(-x)+e^{x}=-f(x)+e^{-x}+e^x$,
\end{center}

et donc $f$ est solution sur $\Rr$ de l'équation différentielle $y''+y=2\ch(x)$. Par suite, il existe $(\lambda,\mu)\in\Rr^2$ tel que $\forall x\in\Rr$, $f(x)=\ch(x)+\lambda\cos(x)+\mu\sin(x)$.

Réciproquement, soit $f$ une telle fonction. $f$ est dérivable sur $\Rr$ et pour tout réel $x$,

\begin{center}
$f'(x)+f(-x)=(\sh(x)-\lambda\sin(x)+\mu\cos(x))+(\ch(x)+\lambda\cos(x)-\mu\sin(x))=e^x+(\lambda+\mu)(\cos(x)-\sin(x))$,
\end{center}

et $f$ est solution si et seulement si $\lambda+\mu=0$.

Les fonctions solutions sont les fonctions de la forme $x\mapsto\ch(x)+\lambda(\cos(x)-\sin(x))$, $\lambda\in\Rr$.
}
}
