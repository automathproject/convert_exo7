\uuid{zdl1}
\exo7id{3970}
\auteur{quercia}
\datecreate{2010-03-11}
\isIndication{false}
\isCorrection{true}
\chapitre{Dérivabilité des fonctions réelles}
\sousChapitre{Autre}

\contenu{
\texte{
Soit $f : \R \to \R$ une fonction de classe $\mathcal{C}^n$.
On pose $g(x) = f(x^2)$.
}
\begin{enumerate}
    \item \question{Montrer qu'il existe des entiers $a_{n,k}$ tels que :
    $\forall\ x,\ g^{(n)}(x) = \sum_{k= [(n+1)/2]}^n a_{n,k}f^{(k)}(x^2)(2x)^{2k-n}$.}
\reponse{$a_{n+1,k} = a_{n,k-1} + 2(2k-n)a_{n,k}$.}
    \item \question{Calculer $a_{n,k}$ en fonction de $n$ et $k$.}
\reponse{$a_{n,k} = \frac {n!}{(n-k)!(2k-n)!}$.}
\end{enumerate}
}
