\uuid{270}
\auteur{cousquer}
\datecreate{2003-10-01}
\isIndication{false}
\isCorrection{true}
\chapitre{Arithmétique dans Z}
\sousChapitre{Divisibilité, division euclidienne}

\contenu{
\texte{

}
\begin{enumerate}
    \item \question{Pour tout couple de nombres r\'eels $(x,y)$ montrer, par r\'ecurrence, que
pour tout $n \in \mathbb{N}^{*}$ on a la relation
$$(*)\; x^{n}-y^{n} = (x-y).\sum_{k=0}^{n-1}x^{k}y^{n-1-k}.$$
Indication: on pourra \'ecrire de deux mani\`eres diff\'erentes la quantit\'e
$y(x^{n}-y^{n}) + (x-y)x^{n}.$}
    \item \question{\label{div} Soit $(a,b,p)$ des entiers \'el\'ements de $\mathbb{N}$. En utilisant
la formule $(*),$ montrer que s'il existe un entier $l \in \mathbb{N}$ tel que $b = a+pl,$
alors pour tout $n \in \mathbb{N}^{*},$ il existe un entier $m \in \mathbb{N}$ tel que $b^{n} =
a^{n}+pm.$}
    \item \question{\label{divv} Soient $a,b,p$ des entiers \'el\'ements de $\mathbb{N}$, en
utilisant la question \ref{div}, montrer que si $a-b$ est divisible par $p,$
$$\sum_{k=0}^{p-1}a^{k}b^{p-k-1}$$
est aussi divisible par $p.$
En d\'eduire, \`a l'aide de la question \ref{div} et de la formule $(*),$ que si $a-b$
est divisible par $p^{n}$ i.e. il existe un entier $l \in \mathbb{N}$ tel que $a-b =
l.p^{n},$ alors $a^{p}-b^{p}$ est divisible par $p^{n+1}.$}
\reponse{
Pour \ref{div}. Si $p$ divise $b-a$ alors $p$ divise aussi $b^n-a^n$
d'apr\`es la formule $(*)$.


Pour \ref{divv}. On utilise le r\'esultat de la question pr\'ec\'edente
avec $n=p-k-1$ pour \'ecrire $b^{p-k-1}$ en fonction de $a^{p-k-1}$
modulo $p$ dans $$\sum_{k=0}^{p-1}a^{k}b^{p-k-1}.$$ 
On peut alors conclure.
}
\end{enumerate}
}
