\uuid{2GMl}
\exo7id{7267}
\auteur{mourougane}
\datecreate{2021-08-10}
\isIndication{false}
\isCorrection{false}
\chapitre{Géométrie affine euclidienne}
\sousChapitre{Géométrie affine euclidienne du plan}

\contenu{
\texte{
Soit \(ABC\) un triangle. Notons \(\Delta_A\), \(\Delta_B\) et 
\(\Delta_C\) les bissectrices des angles en \(A\), \(B\), \(C\) 
respectivement. On note aussi \(a\), \(b\), \(c\) les longueurs 
\(BC\), \(CA\) et \(AB\) respectivement, et \(\alpha\), \(\beta\), 
\(\gamma\) les mesures des angles (non orientés) \(\widehat{CAB}\), 
\(\widehat{ABC}\) et \(\widehat{BCA}\) respectivement.
}
\begin{enumerate}
    \item \question{Montrer que si \(\Delta_A\) est parallèle à \(\Delta_B\), 
alors le triangle \(ABC\) est plat. On supposera dans la suite de 
l'exercice que ce triangle n'est pas plat.}
    \item \question{Montrer que pour tout point \(P\) de \(\Delta_A\), la distance 
de \(P\) à la droite \((AB)\) est égale à la distance de \(P\) à 
la droite \((AC)\).}
    \item \question{En déduire que le point d'intersection de \(\Delta_A\) et de 
\(\Delta_B\), que l'on notera \(\Omega\), est équidistant de 
\((AB)\), \((BC)\) et \((CA)\). On admettra que cela permet de démontrer 
que \(\Omega \in \Delta_C\).}
    \item \question{On note \(A'\) le projeté orthogonal de \(\Omega\) sur la droite 
\((BC)\), c'est-à-dire l'unique point \(A' \in (BC)\) tel que 
\((\Omega A') \perp (BC)\). De même, on note \(B'\) le projeté orthogonal 
de \(\Omega\) sur \((CA)\) et \(C'\) le projeté orthogonal de \(\Omega\) 
sur \((AB)\). Montrer que \(AB' = AC'\).}
    \item \question{On admet que \(A' \in [BC]\), \(B' \in [CA]\), \(C' \in [AB]\). 
Montrer que
\[
2 AB' + a = 2 AB' + A'B + A'C = 2 AB' + BC' + CB' = b + c.
\]}
    \item \question{On note \(p = \frac{a + b + c}{2}\). Montrer que \(AB' = p - a\).}
    \item \question{On note \(r = \Omega A'\). Montrer que
\[
r = (p - a) \tan\frac{\alpha}{2}
 = (p - b) \tan\frac{\beta}{2}
 = (p - c) \tan\frac{\gamma}{2}.
\]}
    \item \question{En utilisant les formules d'addition pour \(\sin\) et \(\cos\), montrer que
\[
\tan\left(\frac{\alpha}{2} + \frac{\beta}{2}\right)
= \frac{\tan\frac{\alpha}{2} + \tan\frac{\beta}{2}}{1 - \tan\frac{\alpha}{2} \tan\frac{\beta}{2}}
\quad\text{et}\quad
\tan\left(\frac{\pi}{2} - \frac{\gamma}{2}\right)
= \frac{1}{\tan\frac{\gamma}{2}}.
\]}
    \item \question{Montrer que
\(\frac{\alpha}{2} + \frac{\beta}{2} + \frac{\gamma}{2} = \frac{\pi}{2}\).
En déduire que
\[
\frac{1}{\tan\frac{\alpha}{2}} + \frac{1}{\tan\frac{\beta}{2}} + \frac{1}{\tan\frac{\gamma}{2}}
= \frac{1}{\tan\frac{\alpha}{2} \tan\frac{\beta}{2} \tan\frac{\gamma}{2}}.
\]}
    \item \question{En déduire que \((p - a) (p - b) (p - c) = r^2 p\), puis que l'aire 
du triangle \(ABC\) est
\[
\sqrt{p (p - a) (p - b) (p - c)}.
\]}
\end{enumerate}
}
