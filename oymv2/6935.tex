\uuid{6935}
\auteur{ruette}
\datecreate{2013-01-24}

\contenu{
\texte{
A l'issue d'une expérience de 1000 tirages, un générateur
de chiffres aléatoires a donné les résultats suivants :
\begin{center}
\begin{tabular}
[c]{|c|llllllllll|}%
\hline
chiffre & 1 & 2 & 3 & 4 & 5 & 6 & 7 & 8 & 9 & 0\\
nombre d'apparitions & 87 & 103 & 90 & 110 & 81 & 108 & 85 & 123 & 90
& 123\\
\hline
\end{tabular}
\end{center}
 Tester au niveau 5\% l'hypothèse selon laquelle le générateur 
simule de façon satisfaisante un tirage uniforme sur 
les entiers $\{0,\ldots, 9\}$.
}
\reponse{
Notons $X$ le résultat d'un tirage  d'un entier entre $0$
et $9$ à l'aide de ce générateur et $p=(p_0,\ldots,p_9)$ sa loi. On cherche à tester
l'hypothèse  $H_0$ : ``$p$ est la loi $\mathcal{U}_{\{0,\ldots,9\}}$''
contre l'hypothèse $H_1$ : ``$p$ n'est pas la loi
$\mathcal{U}_{\{0,\ldots,9\}}$'' au niveau $5\%$.
Notons $N^{(i)}$ le nombre d'apparitions du chiffre $i$ sur $1000$ 
tirages de chiffres à l'aide de ce générateur et
$Z=\sum_{i=0}^{9}\frac{(N^{(i)}-100)^{2}}{100}$.\\ On rejette
l'hypothèse $H_0$ au niveau $5\%$ si la valeur observée $z_{obs}$
de $Z$ est telle que $P_{H_{0}}(Z\geq z_{obs})\leq 0,05$. Comme pour tout $i\in\{0,\ldots,9\}$,  $1000p_i=100$ est
suffisamment grand,  on peut approximer la fonction de répartition de la  loi de
$Z$ sous $H_0$ par celle du $\chi^{2}$ à $9$ degrés de
liberté. Donc, on peut approximer $P_{H_{0}}(Z\geq z_{obs})$ par
$1-F_{\chi^{2}_{9}}(z_{obs})$. A partir des valeurs observées pour les variables aléatoires
$N^{(i)}$ qui sont données dans le tableau, on obtient
$z_{obs}\simeq 21,86$. D'après la table de valeurs numériques de
la loi du $\chi^2$, %\versun{donnée à l'annexe \ref{tablechi}},
$F_{\chi^{2}_{9}}(z_{obs})$ est compris entre $0,99$ et
$0,995$. Donc, au niveau $5\%$, on rejette l'hypothèse $H_0$. On
rejette encore l'hypothèse $H_0$ au niveau $1\%$ mais pas au
niveau $0,5\%$.
}
}
