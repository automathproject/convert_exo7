\uuid{zgZv}
\exo7id{1410}
\auteur{legall}
\datecreate{1998-09-01}
\isIndication{false}
\isCorrection{false}
\chapitre{Groupe, anneau, corps}
\sousChapitre{Groupe de permutation}

\contenu{
\texte{
Soit $  G  $ un groupe d'ordre $  2n  $ et $  H  $ un sous-groupe de $  G  $ d'ordre $  n
$ ($  H  $ est donc d'indice deux dans $  G  $).
}
\begin{enumerate}
    \item \question{Montrer que si $  g \in G  $ et $  g\not \in H  ,$ on a $  H\cap gH=\emptyset   $ puis que $  G=H\cup gH  .$}
    \item \question{En d\'eduire que pour tout $  g \in G  ,  g^2\in H  .$}
    \item \question{On suppose d\'esormais $  G=\mathcal{A} _4  $ le groupe des permutations paires de l'ensemble
$ \{1,2,3,4 \}   .$ Soit $  \sigma =(a,b,c)  $ un $  3$-cycle. Montrer que $  \sigma  $ peut
s'\'ecrire comme le carr\'e d'une permutation paire c'est \`a dire qu'il existe $  \varphi \in \mathcal{A}_4
$ telle que $  \varphi ^2=\sigma   .$ En d\'eduire que $  \mathcal{A} _4  $ ne poss\`ede pas de sous-groupe d'ordre $  6 .$}
\end{enumerate}
}
