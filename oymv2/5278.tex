\uuid{UUCh}
\exo7id{5278}
\auteur{rouget}
\datecreate{2010-07-04}
\isIndication{false}
\isCorrection{true}
\chapitre{Dénombrement}
\sousChapitre{Binôme de Newton et combinaison}

\contenu{
\texte{

}
\begin{enumerate}
    \item \question{(***) Trouver une démonstration combinatoire de l'identité $\sum_{}^{}C_n^{2k}=\sum_{}^{}C_n^{2k+1}$ ou encore démontrer directement qu'un ensemble à $n$ éléments contient autant de parties de cardinal pair que de parties de cardinal impair.}
\reponse{Soit $E$ un ensemble à $n$ éléments, $n\geq 1$, et $a$ un élément fixé de $E$. Soit $\begin{array}[t]{cccc}
f~:&\mathcal{P}(E)&\rightarrow&\mathcal{P}(E)\\
 &A&\mapsto&\left\{
 \begin{array}{l}
 A\setminus\{a\}\;\mbox{si}\;a\in A\\
 A\cup\{a\}\;\mbox{si}\;a\notin A
 \end{array}
 \right.
\end{array}$.

Montrons que $f$ est involutive (et donc bijective). Soit $A$ un élément de $\mathcal{P}(E)$.

Si $a\notin A$, $f(A)=A\cup\{a\}$ et donc, puisque $a\in A\cup\{a\}$, $f(f(A))=(A\cup\{a\})\setminus\{a\}=A$.

Si $a\in A$, $f(A)=A\setminus\{a\}$ et $f(f(A))=(A\setminus\{a\})\cup\{a\}=A$.

Ainsi, $\forall A\in\mathcal{P}(E),\;f\circ f(A)=A$ ou encore, $f\circ f=Id_{\mathcal{P}(E)}$.

Maintenant clairement, en notant $\mathcal{P}_p(E)$ (resp. $\mathcal{P}_i(E)$) l'ensemble des parties de $E$ de cardinal pair (resp. impair), $f(\mathcal{P}_p(E))\subset\mathcal{P}_i(E)$ et $f(\mathcal{P}_i(E))\subset\mathcal{P}_p(E)$. Donc, puisque $f$ est bijective 
$$\mbox{card}(\mathcal{P}_p(E))=\mbox{card}(f(P_p(E))\leq\mbox{card}\mathcal{P}_i(E)$$

et de même $\mbox{card}(\mathcal{P}_i(E))\leq\mbox{card}\mathcal{P}_p(E)$. Finalement, $\mbox{card}(\mathcal{P}_i(E))=\mbox{card}\mathcal{P}_p(E)$.}
    \item \question{(****) Trouver une démonstration combinatoire de l'identité $kC_n^k=nC_{n-1}^{k-1}$.}
\reponse{Soient $E=\{a_1,...,a_n\}$ un ensemble à $n$ éléments et $a$ un élément fixé de $E$. Soit $k\in\{1,...,n-1\}$.

Il y a  $C_{n-1}^{k-1}$ parties à $k$ éléments qui contiennent $a$. Donc, $nC_{n-1}^{k-1}(=C_{n-1}^{k-1}+...+C_{n-1}^{k-1})$ est donc la somme du nombre de parties à $k$ éléments qui contiennent $a_1$ et du nombre de parties à $k$ éléments qui contiennent $a_2$ ... et du nombre de parties à $k$ éléments qui contiennent $a_n$.

Dans cette dernière somme, chaque partie à $k$ éléments de $E$ a été comptée plusieurs fois et toutes les parties à $k$ éléments (en nombre égal à $C_n^k$) ont été comptés un même nombre de fois. Combien de fois a été comptée $\{a_1,a_2...a_k\}$~?~Cette partie a été comptée une fois en tant que partie contenant $a_1$, une fois en tant que partie contenant $a_2$... et une fois comme partie contenant $a_k$ et donc a été comptée $k$ fois.

Conclusion~:~$kC_n^k=nC_{n-1}^{k-1}$.}
    \item \question{(****) Trouver une démonstration combinatoire de l'identité $C_{2n}^n=\sum_{k=0}^{n}(C_{n}^k)^2$.}
\reponse{Soit $E=\{a_1,...,a_n,b_1,...,b_n\}$ un ensemble à $2n$ éléments. Il y a $C_{2n}^n$ parties à $n$ éléments de $E$. Une telle partie a $k$ éléments dans $\{a_1,...,a_n\}$ et $n-k$ dans $\{b_1,...,b_n\}$ pour un certain $k$ de $\{0,...,n\}$. Il y a $C_n^k$ choix possibles de $k$ éléments dans $\{a_1,...,a_n\}$ et $C_n^{n-k}$ choix possibles de $n-k$ éléments dans $\{b_1,...,b_n\}$ pour $k$ donné dans $\{0,...,n\}$ et quand $k$ varie de $0$ à $n$, on obtient~:

$$C_{2n}^n=\sum_{k=0}^{n}C_n^kC_n^{n-k}=\sum_{k=0}^{n}(C_n^k)^2.$$}
\end{enumerate}
}
