\uuid{2464}
\auteur{matexo1}
\datecreate{2002-02-01}
\isIndication{false}
\isCorrection{false}
\chapitre{Déterminant, système linéaire}
\sousChapitre{Système linéaire, rang}

\contenu{
\texte{
Soit $a, b$ deux r\'eels diff\'erents. Montrer que le
syst\`eme lin\'eaire 
$$\left\{\
\begin{array}{rcrcrcccrcc}
 x_1   &+& x_2  &+&  x_3  &+&\ldots&+&  x_n &=&1\\
 bx_1  &+& ax_2 &+& ax_3  &+&\ldots&+& ax_n &=&c_1\\
 bx_1  &+& bx_2 &+& ax_3  &+&\ldots&+& ax_n &=&c_2\\
\ldots  & &\ldots & &\ldots  & &\ldots& &\ldots & &\ldots\\
 bx_1  &+& bx_2 &+& bx_3  &+&\ldots&+& ax_n &=&c_n
\end{array}\right.$$
admet une solution unique que l'on calculera.
}
}
