\uuid{734}
\auteur{gourio}
\datecreate{2001-09-01}

\contenu{
\texte{
Quel est le lieu des points d'inflexion (puis des extr\'{e}mums locaux) de $
f_{\lambda } $ quand $\lambda $ d\'{e}crit ${\Rr}, $ o\`{u} :
$$f_{\lambda }:x \mapsto \lambda e^{x}+x^{2}. $$
}
\indication{On distinguera les cas $\lambda \geq 0$ et $\lambda < 0$.
Pour le cas $\lambda < 0$ on considerera des sous-cas.}
\reponse{
$f_\lambda'(x) = \lambda e^x+2x$, $f_\lambda''(x) = \lambda e^x+2$.
Les points d'inflexion sont les racines de $f_\lambda''$,
donc si $\lambda \geq 0$ il n'y a pas de point d'inflexion,
si $\lambda < 0$ alors il y a un point d'inflexion 
en $x_\lambda = \ln(-2/\lambda)$.
Si $\lambda \geq 0$ alors $f_\lambda''$ est toujours strictement positive, donc $f_\lambda'$ est strictement croissante, en $-\infty$
la limite de $f_\lambda'$ est $-\infty$, en $+\infty$ la limite de $f_\lambda'$ est $+\infty$, donc il existe
un unique r\'eel $y_\lambda$ tel que $f'_\lambda(y_\lambda)=0$. 
$f_\lambda$ est d\'ecroissante sur $]-\infty,y_\lambda]$ et
croissante sur $[y_\lambda, +\infty[$. Et en $y_\lambda$
nous avons un minimum absolu.
Nous supposons $\lambda < 0$. Alors $f_\lambda''$ s'annule 
seulement en $x_\lambda$. $f_\lambda'$ est croissante sur $]-\infty,x_\lambda]$ et d\'ecroissante sur $[x_\lambda, +\infty[$.
Donc $f_\lambda'$ a des racines si et seulement si
$f'(x_\lambda) \geq 0$. Or $f'(x_\lambda) = -2+2x_\lambda$.
  \begin{enumerate}
Si $\lambda = -2/e$ alors $f_\lambda'(x_\lambda) = 0$.
  Comme $f_\lambda''(x_\lambda)=0$ et $f_\lambda'''$ ne s'annule pas alors $x_\lambda$ est un point d'inflexion qui 
n'est pas un extremum local.
Si $\lambda>-2/e$ alors $f_\lambda'(x_\lambda) <0$
  donc $f_\lambda'$ est n\'egative donc $f$ est strictement d\'ecroissante.
  Il n'y a pas d'extremum local.
Si $-2/e<\lambda<0$ alors $f_\lambda'(x_\lambda) > 0$.
  Donc $f_\lambda'$ s'annule en deux points,
  une fois sur $]-\infty,x_\lambda[$ et une sur $]x_\lambda,+\infty[$.
  Ce sont des extremums locaux (minimum et maximum respectivement).
}
}
