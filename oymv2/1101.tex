\uuid{1101}
\auteur{ridde}
\datecreate{1999-11-01}

\contenu{
\texte{
% anti diagonal dots
\def\Ddots{\mathinner{\mkern2mu\raise1pt\hbox{.}\mkern2mu
\newline \raise4pt\hbox{.}\mkern2mu\raise7pt\hbox{.}\mkern1mu}}

Soit $A = \begin{pmatrix} 
0&&\dots&0&1 \\ 
\vdots&&&1&0 \\
 & & \Ddots && \\
0&1& & &\vdots \\ 
1&0&&\dots&0
\end{pmatrix}$. 
En utilisant l'application linéaire associée de 
$\mathcal{L} (\Rr^n,\Rr^n)$, calculer $A^p$ pour $p \in \Zz$.
}
\reponse{
Nous associons à la matrice $A$ son application linéaire naturelle $f$.
Si $\mathcal{B}=(e_1,e_2,\ldots,e_n)$ est la base canonique de $\Rr^n$
alors $f(e_1)$ est donné par le premier vecteur colonne, $f(e_2)$ par le deuxième, etc.
Donc ici 
$$f(e_1)=\begin{pmatrix}0\\ \vdots \\ 0 \\ 0 \\ 1 \end{pmatrix}=e_n, \
f(e_2)= \begin{pmatrix}0\\ \vdots \\ 0 \\ 1 \\0 \end{pmatrix}=e_{n-1},...  \quad \text{ et en général }
f(e_i) = e_{n+1-i}$$ 
Calculons ce que vaut la composition $f\circ f$.
Comme une application linéaire est déterminée par
 les images des éléments d'une base alors
on calcule $f\circ f(e_i)$, $i=1,\ldots,n$ en appliquant deux fois la formule précédente :
$$f\circ f(e_i) = f\big( f(e_i) \big) =  f(e_{n+1-i})=e_{n+1-(n+1-i)}=e_i$$
Comme $f\circ f$ laisse invariant tous les vecteurs de la base alors
$f\circ f (x)=x$ pour tout $x\in \Rr^n$. Donc $f\circ f=\mathrm{id}$.

On en déduit $f^{-1}=f$ et que la composition itérée vérifie $f^{p}=\mathrm{id}$ si $p$ est pair
et $f^{p}=f$ si $p$ est impair.
Conclusion : $A^p=I$ si $p$ est pair et $A^p=A$ si $p$ est impair.
}
}
