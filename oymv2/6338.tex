\uuid{6338}
\auteur{queffelec}
\datecreate{2011-10-16}
\isIndication{false}
\isCorrection{false}
\chapitre{Système linéaire à  coefficients constants}
\sousChapitre{Système linéaire à  coefficients constants}

\contenu{
\texte{
Soit $A$ un opérateur de $\Rr^n$ et $x'=Ax$ le système associé.
}
\begin{enumerate}
    \item \question{On suppose que $A$ laisse un sous-espace $E$ invariant; montrer que si
$\varphi$ est une solution de condition initiale $\varphi(t_0)\in E$ alors
$\varphi(t)\in E$ pour tout $t\in \Rr$.}
    \item \question{Que peut-on dire des solutions du
système si $A$ est  nilpotente;?}
    \item \question{On suppose que $A$ a une valeur propre de partie réelle $<0$; montrer qu'il
existe au moins une solution $\varphi$ telle que
$\lim_{t\to+\infty}\varphi(t)=0$.}
    \item \question{A quelles conditions le système n'a-t-il que des solutions bornées ?}
\end{enumerate}
}
