\uuid{639}
\auteur{bodin}
\datecreate{1998-09-01}

\contenu{
\texte{
Soit $I$ un intervalle ouvert de $\Rr$, $f$ et $g$ deux fonctions d\'efinies sur $I$.
}
\begin{enumerate}
    \item \question{Soit $a\in I$. Donner une raison pour laquelle :
$$\left( \lim_{x\rightarrow a}f(x)=f(a) \right) \Rightarrow
\left( \lim_{x\rightarrow a} |f(x)|=|f(a)| \right). $$}
\reponse{On a pour tout $x,y\in\R$ $|x-y|\geq \big| |x|-|y|\big|$
(c'est la deuxi\`eme formulation de l'in\'egalit\'e triangulaire).
Donc pour tout $x\in I$ :$ \big| |f(x)|-|f(a)| \big| \leq |f(x)-f(a)| $.
L'implication annonc\'ee r\'esulte alors imm\'ediatement de la
d\'efinition de l'assertion $\lim_{x\to a} f(x)=f(a). $}
    \item \question{On suppose que $f$ et $g$ sont continues sur $I$. En utilisant
l'implication d\'emontr\'ee ci-dessus, la relation $\sup(f,g)=\frac{1}{2}(f+g+|f-g|)$,
et les propri\'et\'es des fonctions continues, montrer que la fonction $\sup(f,g)$
est continue sur $I$.}
\reponse{Si $f,g$ sont continues
alors $\alpha f+\beta g$ est continue sur $I$, pour tout
$\alpha,\beta\in\R$. Donc les fonctions $f+g$ et $f-g$ sont
continues sur $I$. L'implication de $1.$ prouve alors que $|f-g|$
est continue sur $I$, et finalement on peut
conclure :\\
La fonction $\sup (f,g) = \frac{1}{2}(f+g+|f-g|)$ est continue sur
$I$.}
\indication{\begin{enumerate}
    \item On pourra utiliser la variante de l'in\'egalit\'e triangulaire $|x-y|\geq \big| |x|-|y| \big|$.
    \item Utiliser la premi\`ere question pour montrer que $|f-g|$ est continue.
\end{enumerate}}
\end{enumerate}
}
