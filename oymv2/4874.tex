\uuid{4874}
\auteur{quercia}
\datecreate{2010-03-17}
\isIndication{false}
\isCorrection{true}
\chapitre{Géométrie affine dans le plan et dans l'espace}
\sousChapitre{Applications affines}

\contenu{
\texte{
Dans un espace affine ${\cal E}$, on considère quatre points $A,B,C,D$.
\'Etudier l'existence d'une application affine $f$ telle que
$f(A) = B$, $f(B) = C$, $f(C) = D$, $f(D) = A$.
}
\reponse{
si trois points sont non alignés, $ABCD$ doit être un parallélogramme.

si deux points sont distincts et $A,B,C,D$ sont alignés, on doit avoir
$A=C$, $B=D$.
}
}
