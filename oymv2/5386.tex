\uuid{5386}
\auteur{rouget}
\datecreate{2010-07-06}
\isIndication{false}
\isCorrection{true}
\chapitre{Continuité, limite et étude de fonctions réelles}
\sousChapitre{Continuité : théorie}

\contenu{
\texte{
Montrer en revenant à la définition que $f(x)=\frac{3x-1}{x-5}$ est continue en tout point de $\Rr\setminus\{5\}$.
}
\reponse{
Soit $x_0\in\Rr\setminus\{5\}$. Pour $x\neq5$, 

$$|f(x)-f(x_0)|=\left|\frac{3x-1}{x-5}-\frac{3x_0-1}{x_0-5}\right|=\frac{14|x-x_0|}{|x-5|.|x_0-5|}.$$

Puis, pour $x\in]x_0-\frac{|x_0-5|}{2},x_0+\frac{|x_0-5|}{2}[$, on a $|x-5|>\frac{|x_0-5|}{2}[(>0)$, et donc,

$$\forall x\in]x_0-\frac{|x_0-5|}{2},x_0+\frac{|x_0-5|}{2}[,\;|f(x)-f(x_0)|=\frac{28}{(x_0-5)^2}|x-x_0|.$$

Soient $\varepsilon>0$ puis $\alpha=\mbox{Min}\{\frac{|x_0-5|}{2},\frac{(x_0-5)^2\varepsilon}{28}\}(>0)$.

$$|x-x_0|<\alpha\Rightarrow|f(x)-f(x_0)|\leq\frac{28}{(x_0-5)^2}|x-x_0|<\frac{28}{(x_0-5)^2}\frac{(x_0-5)^2\varepsilon}{28}=\varepsilon.$$

On a monté que $\forall\varepsilon>0,\;\exists\alpha>0/\;(\forall x\in\Rr\setminus\{5\},\;|x-x_0|<\alpha\Rightarrow|f(x)-f(x_0)|<\varepsilon)$. $f$ est donc continue sur $\Rr\setminus\{5\}$.
}
}
