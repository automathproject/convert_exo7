\uuid{834}
\auteur{ridde}
\datecreate{1999-11-01}
\isIndication{false}
\isCorrection{false}
\chapitre{Calcul d'intégrales}
\sousChapitre{Primitives diverses}

\contenu{
\texte{
D\'eterminer les intervalles d'\'etude et calculer les primitives des fonctions :
$$ \text{ch}x \sin (2x)$$
$$ \frac 1{\sqrt{2 + \tan ^2 x}}$$
$$ (x^2 + 2x + 2) \cos (2x)$$
$$ x^2 \cos x \text{ et } x^2 \sin x \text{ en utilisant les complexes}$$
$$\frac 1 { (x^2-1)^3} \text{ et } \frac 1{ (x^2-1)^2}$$
$$ \frac {\sqrt{1 + x}} {x \sqrt{1-x}}$$
}
}
