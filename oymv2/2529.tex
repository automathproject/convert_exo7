\uuid{2529}
\auteur{tahani}
\datecreate{2009-04-01}
\isIndication{false}
\isCorrection{true}
\chapitre{Difféomorphisme, théorème d'inversion locale et des fonctions implicites}
\sousChapitre{Difféomorphisme, théorème d'inversion locale et des fonctions implicites}

\contenu{
\texte{
Soit $\varphi$
l'application de $\mathbb{R}^2$ dans $\mathbb{R}^2$ d\'efinie pas
$$\varphi(x,y)=(\sin(y/2)-x,\sin(x/2)-y).$$
}
\begin{enumerate}
    \item \question{Justifier que $\varphi$ est de classe $C^1$, calculer sa
diff\'erentielle et voir que $D\varphi(x,y)$ est inversible pour
tout $(x,y) \in \mathbb{R}^2$.}
\reponse{$\varphi$ a des coordonn\'ees de classe $C^1$, elle l'est
donc aussi. On a
$$Jac(\varphi)(x,y)=\left (\begin{array}{cc}
-1 & 1/2 \cos(y/2) \\
1/2 \cos(x/2) & -1
\end{array}\right )$$
On a  $det(Jac(\varphi)(x,y))=1-1/4\cos(x/2)\cos(y/2) \geq 3/4 >
0$. Par cons\'equent la jacobienne est inversible et
$D\varphi(x,y) \in Isom(\mathbb{R}^2,
\mathbb{R}^2)=GL(\mathbb{R}^2$.}
    \item \question{Montrer que $\varphi$ est un
$C^1$-diff\'eomorphisme de $\mathbb{R}^2$ sur
$\varphi(\mathbb{R}^2)$ et justifier que $\varphi(\mathbb{R}^2)$
est un ouvert.}
\reponse{D'apr\`es le th\'eor\`eme de
l'inverse local, Il suffit de montrer que $\varphi$ est injective.
Supposons $\varphi(x_1,y_1)=\varphi(x_2,y_2)$, alors
$\sin(y_1/2)-x_1=\sin(y_2/2)-x_2$ et
$\sin(x_1/2)-y_1=\sin(x_2/2)-y_2$. D'o\`u
$\sin(y_1/2)-\sin(y_2/2)=x_1-x_2$ et
$\sin(x_1/2)-\sin(x_2/2)=y_1-y_2$. Or, $\forall a,b \in
\mathbb{R}, |\sin a -\sin b| \leq |a-b|$ (cons\'equence des
accroissements finis appliqu\'e \`a $\sin x$). Donc $|x_1-x_2|\leq
|y_1/2-y_2/2|$ et $|y_1-y_2| \leq |x_1/2-x_2/2|$ d'o\`u
$|x_1-x_2|\leq 1/4 |x_1-x_2| \Rightarrow x_1=x_2$  et $y_1=y_2$.
$\varphi: U \rightarrow F$ est injective. L'ensemble $f(U)$ est
ouvert car il est r\'eunion d'ouverts (d'apr\`es thm inverse
local). C'est un diff\'eomorphisme en $U$ et $\varphi(U)$.}
    \item \question{Montrer que $\varphi^{-1}$ est lipschitzienne
(on prendra comme norme sur $\mathbb{R}^2$: $\|(x,y)\|=|x|+|y|$).}
\reponse{Soient $(X_1,y_1), (X_2,Y_2) \in \varphi(\mathbb{R}^2)$ avec
$\varphi(x_1,y_1)=(X_1,Y_1)$ et $\varphi(x_2,y_2)=(X_2,Y_2)$ ou
encore $\varphi^{-1}(X_1,Y_1)=(x_1,y_1)$ et
$\varphi^{-1}(X_2,Y_2)=(x_2,y_2)$. On a
$$||\varphi^{-1}(X_1,Y_1)-\varphi^{-1}(X_2,Y_2)||=||(x_1,y_1)-(x_2,y_2)||=||(x_1-x_2,y_1,y_2)||=|x_1-x_2|+|y_1-y_2|$$
Or $\sin(y_1/2)-x_1=X_1$, $\sin(y_2/2)-x_2=X_2$,
$\sin(x_1/2)-y_1=Y_1$ et $\sin(x_2/2)-y_2=Y_2$. Par cons\'equent
$$x_1-x_2=\sin(y_1/2)-X_1-\sin(y_2/2)+X_2$$
$$y_1-y_2=\sin(x_1/2)-Y_1-\sin(x_2/2)+Y_2.$$
D'o\`u
$$|x_1-x_2|+|y_1-y_2| \leq |X_2-X_1|+|\sin(y_1/2)-\sin(y_2/2)|+
|Y_2-Y_1|+ |\sin(x_1/2)-\sin(x_2/2)|$$ $$ \leq
|X_2-X_1|+1/2|y_1-y_2|+|Y_2-Y_1|+1/2|x_1-x_2|$$ d'où
$$|x_1-x_2|+|y_1-y_2| \leq 2(|X_2-X_1|+|Y_2-Y_1|) \leq 2
||(X_1,Y_1)-(X_2,Y_2)||.$$ Donc $\varphi^{-1}$ est lipschitzienne.}
    \item \question{En d\'eduire que $\varphi$ est un diff\'eomorphisme de
$\mathbb{R}^2$ sur $\mathbb{R}^2$}
\reponse{Soit $(X_n,Y_n)$ une suite de cauchy dans
$\varphi(\mathbb{R}^2)$,  ($(X_n,Y_n)=\varphi(x_n,y_n)$;
$(x_n,y_n)=\varphi^{-1}(X_n,Y_n)$). Pour tout $\epsilon > 0,
\exists n \in \mathbb{N}, p,q \geq n \Rightarrow
||(X_p,Y_p)-(X_q,Y_q)|| < \epsilon.$ Par cons\'equent, $\forall
\epsilon > 0$, $\exists n \in \mathbb{N}; p,q \geq n \Rightarrow
||(x_p,y_p)-(x_q,y_q)||< 2 \epsilon$. La suite $(x_n,y_n)$ est
alors de cauchy dans $\mathbb{R}^2$, qui est complet. Par
cons\'equent elle converge. Soit $(x,y)$ sa limite. Comme
$\varphi$ est continue et que $\lim_{n}(x_n,y_n)=(x,y)$ alors
$\lim_{n} \varphi(x_n,y_n)=\varphi(x,y)$. La suite $(X_n,Y_n)$ est
une suite de Cauchy de $\mathbb{R}^2$. Elle converge. Soit $(X,Y)$
sa limite, alors $(X,Y)=\varphi(x,y)$ car $(X_n,Y_n) \rightarrow
(X,Y)$ et $\varphi(x_n,y_n) \rightarrow \varphi(x,y)$. Donc $(X,Y)
\in \varphi(\mathbb{R}^2)$. $\varphi(\mathbb{R}^2)$ est alors
complet et donc ferm\'e. Comme $\varphi(\mathbb{R}^2$ est un
ouvert ferm\'e et non vide (il contient $(0,0)=\varphi(0,0)$) dans
le connexe $\mathbb{R}^2$, on a
$\varphi(\mathbb{R}^2)=\mathbb{R}^2$.}
    \item \question{Calculer
$D\varphi^{-1}(p)$ o\`u $p=(1-\pi/2, \sqrt{2}/2-\pi)$.}
\reponse{$p=(1-\pi/2,
\sqrt{2}/2 - \pi)=\varphi(\pi/2,\pi)=\varphi(q)$ o\`u
$q=(\pi/2,\pi)$. $\varphi: E \rightarrow F$ est un
$C^1$-diff\'eormorphisme donc $\varphi^{-1} \circ \varphi=Id$ et
donc $$Id=D(\varphi^{-1} \circ
\varphi)(q)=D\varphi^{-1}(\varphi(q))\circ D\varphi(q).$$ Or
$D\varphi^{-1}(\varphi(q))=(D\varphi(q))^{-1}$ et donc $Jac
\varphi^{-1}(p)= (Jac \varphi (\pi/2,\pi))^{-1}$. Or
$$Jac \varphi(x,y)= \left (
\begin{array}{cc}
-1 & 1/2 \cos(y/2) \\
1/2 \cos(x/2) & -1
\end{array} \right ).$$
D`o\`u $$Jac \varphi(\pi/2,\pi)= \left (
\begin{array}{cc}
-1 &0\\
\sqrt{2}/4 & -1
\end{array} \right )$$ et donc
$$Jac \varphi^{-1}(p)=
 \left (
\begin{array}{cc}
-1 &0\\
-\sqrt{2}/4 & -1
\end{array} \right ).$$}
\end{enumerate}
}
