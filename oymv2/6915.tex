\uuid{6915}
\auteur{ruette}
\datecreate{2013-01-24}

\contenu{
\texte{
Dans une pile de $n$ ($n \geq 2$) feuilles dactylographiées, se trouvent les
deux lettres que l'on doit envoyer. On enlève une par une les
feuilles du paquet jusqu'à ce que l'une des lettres à envoyer se
trouve sur le dessus du paquet. On note $X_1$ la variable
aléatoire donnant le nombre de feuilles enlevées. On recommence
l'opération jusqu'à trouver la deuxième lettre et on note  $X_2$
la variable aléatoire donnant le nombre de feuilles
qu'il a fallu retirer du paquet après avoir trouvé la première lettre et
avant que la deuxième lettre soit
sur le dessus du paquet. Sans information supplémentaire, on peut
supposer que toutes les positions possibles pour les deux lettres sont
équiprobables.
}
\begin{enumerate}
    \item \question{Décrire l'ensemble $\Omega$ des résultats possibles pour cette
expérience aléatoire et la probabilité $P$ mise sur $\Omega$.}
    \item \question{Déterminer la loi du couple $(X_1,X_2)$ puis la loi de $X_1$ et de
$X_2$.}
    \item \question{Calculer la probabilité de l'événement ``$X_1=X_2$''.}
    \item \question{On note $Z=X_1+X_2+2$.  Que représente la variable aléatoire $Z$~?
Déterminer sa loi.}
\end{enumerate}
}
