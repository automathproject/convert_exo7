\uuid{1819}
\auteur{drutu}
\datecreate{2003-10-01}

\contenu{
\texte{

}
\begin{enumerate}
    \item \question{Calculer la d\'eriv\'ee de la fonction 
$F(x,y)=x^2-xy-2y^2$ 
au point $P(1,2)$ dans une direction formant avec l'axe $Ox$ un angle de 
$\frac{\pi}{3}$.}
    \item \question{Calculer la d\'eriv\'ee de la fonction $F(x,y)=x^3-2x^2y+xy^2+1$ au point 
$P(1,2)$ dans la direction joignant ce point au point $M(4,6)$.}
    \item \question{Calculer la d\'eriv\'ee de la fonction $F(x,y)=\ln \sqrt{x^2+y^2}$ au point 
$P(1,1)$ suivant la bissectrice du premier quadrant.}
\end{enumerate}
}
