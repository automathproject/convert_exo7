\uuid{SqEQ}
\exo7id{4213}
\auteur{quercia}
\datecreate{2010-03-11}
\isIndication{false}
\isCorrection{true}
\chapitre{Equation différentielle}
\sousChapitre{Equations aux dérivées partielles}

\contenu{
\texte{
Soit $f : {\R^2} \to \R$ de classe $\mathcal{C}^2$.
On pose $g(x,y) = f(2x+y,2x-y)$.
}
\begin{enumerate}
    \item \question{Calculer les dérivées partielles secondes de $g$ en fonction de celles
    de $f$.}
\reponse{$\frac{\partial^2 g}{\partial x^2} = 4\frac{\partial^2 f}{\partial x^2} + 8\frac{\partial^2 f}{\partial x \partial y} + 4\frac{\partial^2 f}{\partial y^2}$\par
             $\frac{\partial^2 g}{\partial y^2} =  \frac{\partial^2 f}{\partial x^2} - 2\frac{\partial^2 f}{\partial x \partial y} +  \frac{\partial^2 f}{\partial y^2}$\par
             $\frac{\partial^2 g}{\partial x \partial y} = 2\frac{\partial^2 f}{\partial x^2} - 2\frac{\partial^2 f}{\partial y^2}$.}
    \item \question{Trouver $f$ telle que $\frac{\partial^2 g}{\partial x^2} - 4\frac{\partial^2 g}{\partial y^2} = 1$.}
\reponse{$f(x,y) = \frac{xy}{16} + h(x) + k(y)$.}
\end{enumerate}
}
