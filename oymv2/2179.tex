\uuid{2179}
\auteur{debes}
\datecreate{2008-02-12}
\isIndication{true}
\isCorrection{false}
\chapitre{Action de groupe}
\sousChapitre{Action de groupe}

\contenu{
\texte{
Soit $G$ un groupe et $H$ un sous-groupe d'indice fini $n$. Montrer que
l'intersection $H^\prime$ des conjugu\'es de $H$ par les \'el\'ements de $G$ est un
sous-groupe distingu\'e de $G$ et d'indice fini dans $G$. Montrer que c'est le plus
grand sous-groupe distingu\'e de $G$ contenu dans $H$.
}
\indication{Le seul point non imm\'ediat est que $H^\prime$ est d'indice fini dans $G$. Pour cela
consid\'erer le morphisme de $G$ \`a valeurs dans le groupe des permutations des
classes \`a gauche de $G$ modulo $H$, qui \`a $g\in G$ associe la permutation
$aH\rightarrow gaH$ et montrer que le noyau de ce morphisme est le groupe $H^\prime$.}
}
