\uuid{5191}
\auteur{rouget}
\datecreate{2010-06-30}
\isIndication{false}
\isCorrection{true}
\chapitre{Application linéaire}
\sousChapitre{Image et noyau, théorème du rang}

\contenu{
\texte{
Soient $\Kk$ un sous-corps de $\Cc$ et $E$ un $\Kk$-espace vectoriel de dimension finie $n$.
}
\begin{enumerate}
    \item \question{Montrer que, pour tout endomorphisme $f$ de $\Rr^2$, on a~:

\begin{center}
$(\mbox{Ker}f=\mbox{Im}f)\Leftrightarrow(f^2=0\;\mbox{et}\;n=2\mbox{rg}f)\Leftrightarrow(f^2=0\;\mbox{et}\;
\exists g\in\mathcal{L}(E)/\;f\circ g+g\circ f=Id_E)$.
\end{center}}
\reponse{\textbullet~\textbf{(1)$\Rightarrow$(2).} Si $\mbox{Ker }f=\mbox{Im }f$, alors pour tout élément $x$ de $E$, $f(x)$ est dans
$\mbox{Im }f=\mbox{Ker }f$ et donc $f(f(x))=0$. Par suite, $f^2=0$. De plus, d'après le théorème du rang,
$n=\mbox{dim }(\mbox{Ker }f)+\mbox{rg }f=2\;\mbox{rg }f$ ce qui montre que $n$ est nécessairement pair et que
$\mbox{rg }f=\frac{n}{2}$.
\textbullet~\textbf{(2)$\Rightarrow$(3).} Si $f^2=0$ et $n=2\;\mbox{rg }f\;(\in2\Nn)$, cherchons un endomorphisme $g$ de $E$ tel que
$f\circ g+g\circ f=Id_E$.
Posons $r=\mbox{rg }f$ et donc $n=2r$, puis $F=\mbox{Ker }f=\mbox{Im }f$ ($\mbox{dim }F=r$).

Soit $G$ un supplémentaire de $F$ dans $E$ ($\mbox{dim }G=r$).
Soit $(e_1',...,e_r')$ une base de $G$. Pour $i\in\llbracket1,r\rrbracket$, on pose $e_i=f(e_i')$.
Montrons que la famille $(e_1,...,e_r)$ est libre. Soit $(\lambda_1,...,\lambda_r)\in\Rr^r$.

$$\sum_{i=1}^{r}\lambda_ie_i=0\Rightarrow f\left(\sum_{i=1}^{r}\lambda
e_i'\right)=0\Rightarrow\sum_{i=1}^{r}\lambda_ie_i'\in\mbox{Ker }f\cap G=\{0\}\Rightarrow\forall
i\in\{1,...,r\},\;\lambda_i=0,$$
car la famille $(e_i')_{1\leq i\leq r}$ est libre. $(e_1,...,e_r)$ est une famille libre de $F=\mbox{Im }f$ de cardinal
$r$ et donc une base de $F=\mbox{Ker }f=\mbox{Im }f$. Au passage, puisque $E=F\oplus G$, $(e_1,...,e_r,e_1',...,e_r')$ est
une base de $E$.
Soit alors $g$ l'endomorphisme de $E$ défini par les égalités~:~$\forall i\in\llbracket1,r\rrbracket,\;g(e_i)=e_i'$ et
$g(e_i')=e_i$ ($g$ est entièrement déterminé par les images des vecteurs d'une base de $E$). Pour $i$ élément de
$\llbracket1,r\rrbracket$, on a alors~:~

$$(f\circ g+g\circ f)(e_i)=f(e_i')+g(0)=e_i+0= e_i,$$

et

$$(f\circ g+g\circ f)(e_i')=f(e_i)+g(e_i)=0+e_i'=e_i'.$$
Ainsi, les endomorphismes $f\circ g+g\circ f$ et $Id_E$ coïncident sur une base de $E$ et donc $f\circ g+g\circ f=Id_E$.
\textbullet~\textbf{(3)$\Rightarrow$(1).} Supposons que $f^2=0$ et qu'il existe $g\in\mathcal{L}(E)$ tel que $f\circ g+g\circ
f=Id_E$.
Comme $f^2=0$, on a déjà $\mbox{Im }f\subset\mbox{Ker }f$. D'autre part, si $x$ est un élément de $\mbox{Ker }f$, alors
$x=f(g(x))+g(f(x))=f(g(x))\in\mbox{Im }f$ et on a aussi $\mbox{Ker }f\subset\mbox{Im }f$. Finalement,
$\mbox{Ker }f=\mbox{Im }f$.}
    \item \question{On suppose $\mbox{Ker}f=\mbox{Im}f$. Montrer qu'il existe une base $(u_1,...,u_p,v_1,...,v_p)$ de $E$ telle
que~:

\begin{center}
$\forall i\in\{1,...,p\},\;f(u_i)=0\;\mbox{et}\;f(v_i)=u_i$.
\end{center}}
\reponse{L'existence d'une base $(e_1,...,e_p,e_1',...,e_p')$ de $E$ vérifiant les conditions de l'énoncé a été établie
au passage (avec $p=r=\mbox{rg }f$).}
\end{enumerate}
}
