\uuid{0Je4}
\exo7id{5939}
\auteur{tumpach}
\datecreate{2010-11-11}
\isIndication{false}
\isCorrection{true}
\chapitre{Lemme de Fatou, convergence monotone}
\sousChapitre{Lemme de Fatou, convergence monotone}

\contenu{
\texte{

}
\begin{enumerate}
    \item \question{Soit $\{g_{n}\}_{n\in\mathbb{N}}$ une suite dans
$\mathcal{M}^{+}(\Omega, \Sigma)$. Montrer que
$$
\int_{\Omega} \left(\sum_{n=1}^{+\infty} g_{n}\right)\,d\mu =
\sum_{n=1}^{+\infty}\int_{\Omega} g_{n} \,d\mu.
$$}
\reponse{Soit $\{g_{n}\}_{n\in\mathbb{N}}$ une suite dans
$\mathcal{M}^{+}(\Omega, \Sigma)$. Alors $f_{k} = \sum_{n=1}^{k}
g_{n}$ est une suite croissante de $\mathcal{M}^{+}(\Omega,
\Sigma)$. D'apr\`es le th\'eor\`eme de convergence monotone
$$
\int_{\Omega} \left(\sum_{n=1}^{+\infty} g_{n}\right)\,d\mu =
\sum_{n=1}^{+\infty}\int_{\Omega} g_{n} \,d\mu.
$$}
    \item \question{Montrer que
$$
\int_{0}^{+\infty} \frac{x^{s-1}}{e^x -1}\,dx = \Gamma(s)
\zeta(s),
$$
o\`u $\Gamma$ est la fonction d'Euler et o\`u $\zeta(s) =
\sum_{n=1}^{+\infty} n^{-s}$. (On pourra consid\'erer les
fonctions $g_{n}(x) = x^{s-1} e^{-nx} \mathbf{1}_{[0, +\infty)}$.)}
\reponse{Posons $g_{n}(x) = x^{s-1} e^{-nx} \mathbf{1}_{[0, +\infty)}$. Les
$g_{n}$ appartiennent \`a $\mathcal{M}^{+}(\Omega, \Sigma)$ pour
tout $n\in\mathbb{N}$. D'apr\`es la question pr\'ec\'edente,
$$
\int_{\Omega} \left(\sum_{n=1}^{+\infty} g_{n}\right)\,d\mu =
\sum_{n=1}^{+\infty}\int_{\Omega} g_{n} \,d\mu.
$$
Or d'une part,
$$
\int_{\Omega}g_{n} \,d\mu = \int_{\Omega} x^{s-1} e^{-nx}
\mathbf{1}_{[0,+\infty)}\, dx = \frac{1}{n^s}\int_{0}^{+\infty} y^{s-1}
e^{-y}\,dy = \frac{1}{n^s}\Gamma(s),
$$
donc
$$
\sum_{n=1}^{+\infty}\int_{\Omega} g_{n} \,d\mu =
\sum_{n=1}^{\infty}\frac{1}{n^s}\Gamma(s) = \zeta(s) \Gamma(s).
$$
D'autre part,
$$
\int_{\Omega} \left(\sum_{n=1}^{+\infty} g_{n}\right)\,d\mu =
\int_{0}^{+\infty} x^{s-1} \sum_{n=1}^{+\infty} e^{-nx}\,dx =
\int_{0}^{+\infty} \frac{x^{s-1}}{e^x -1}\,dx,
$$
d'o\`u l'\'egalit\'e cherch\'ee.}
\end{enumerate}
}
