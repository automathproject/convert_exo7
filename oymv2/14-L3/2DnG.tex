\uuid{2DnG}
\exo7id{5960}
\auteur{tumpach}
\datecreate{2010-11-11}
\isIndication{false}
\isCorrection{true}
\chapitre{Intégrales multiples, théorème de Fubini}
\sousChapitre{Intégrales multiples, théorème de Fubini}

\contenu{
\texte{
Soient $a, b>0$, et $f$ et $g$ les fonctions d\'efinies sur
$\mathbb{R}^n$ par $f(x) = e^{-\frac{a |x|^2}{2}}$ et $g(x) =
e^{-\frac{b |x|^2}{2}}$. Calculer $f*g(x)$.
}
\reponse{
Soient $a, b>0$, et $f$ et $g$ les fonctions d\'efinies sur
$\mathbb{R}^n$ par $f(x) = e^{-\frac{a |x|^2}{2}}$ et $g(x) =
e^{-\frac{b |x|^2}{2}}$. On a
\begin{eqnarray*}
f*g(x)  =  \int_{\mathbb{R}^n} f(x-y)\,g(y)\,dy =
\int_{\mathbb{R}^n} e^{-\left(\frac{a|x-y|^2 +
b|y|^2}{2}\right)}\, dy
\end{eqnarray*}
Or
\begin{eqnarray*}
a|x-y|^2 + b|y|^2 &=& \sum_{i=1}^n a x_i^2 + (a + b) y_i^2 -2a x_i
y_i\\
& = & \sum_{i=1}^n a x_i^2 + (a + b)\left(y_i - \frac{a}{a+b} x_i
\right)^2 - (a+b) \left(\frac{a x_i}{a+b} \right)^2\\
& = & \sum_{i=1}^n \left(a - \frac{a^2}{a+b}\right) x_i^2 + (a +
b)\left(y_i
- \frac{a}{a+b} x_i \right)^2\\
& = & \frac{ab}{a + b}|x|^2 +
(a+b)\left|y-\frac{a}{a+b}x\right|^2.
\end{eqnarray*}
Ainsi
\begin{eqnarray*}
f*g(x)  & = & e^{-\frac{ab}{a +
b}\frac{|x|^2}{2}}\int_{\mathbb{R}^n}
e^{-\frac{(a+b)}{2}\left|y-\frac{a}{a+b}x\right|^2}\,dy =
e^{-\frac{ab}{a + b}\frac{|x|^2}{2}}
 \int_{\mathbb{R}^n} e^{-\frac{(a + b)}{2} |z|^2}\,dz
 \end{eqnarray*}
car la mesure de Lebesgue est invariante par translation. En
utilisant $\int_{\mathbb{R}} e^{-t^2} \,dt = \sqrt{\pi}$, on
obtient alors~:
$$
f*g(x) = \left(\frac{2\pi}{a + b}\right)^{\frac{n}{2}}
e^{-\frac{ab}{a + b}\frac{|x|^2}{2}}.
$$
}
}
