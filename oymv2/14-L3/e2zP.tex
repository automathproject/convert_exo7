\uuid{e2zP}
\exo7id{5964}
\auteur{tumpach}
\datecreate{2010-11-11}
\isIndication{false}
\isCorrection{true}
\chapitre{Espace L^p}
\sousChapitre{Espace Lp}

\contenu{
\texte{
Soit $\Omega$ un sous-ensemble de $\mathbb{R}^n$ dont la mesure de
Lebesgue   est \emph{finie}~: $\mu(\Omega)< +\infty$. Pour tout
$1\leq p <+\infty$, on note $L^p(\Omega)$ l'espace des fonctions
$f~:\Omega \rightarrow \mathbb{C}$ telles que $\|f\|_{p} :=
\left(\int_{\Omega} |f|^{p}(x)\,dx\right)^{\frac{1}{p}}<+\infty$
modulo l'\'equivalence $f\sim g \Leftrightarrow f-g = 0 ~\mu-p.p$.
L'espace des fonctions essentiellement born\'ees sera not\'e
$L^{\infty}(\Omega).$
}
\begin{enumerate}
    \item \question{Montrer que si $q\leq p$, alors $L^{p}(\Omega) \subset
L^{q}(\Omega)$. En particulier, pour $1<q<2<p$, on a~:
$$
L^{\infty}(\Omega) \subset L^{p}(\Omega) \subset L^{2}(\Omega)
\subset L^{q}(\Omega) \subset L^{1}(\Omega).
$$}
\reponse{Si $f\in L^{\infty}(\Omega)$, alors
$$
\| f \|_p^p = \int_{\Omega} |f|^{p}(x)\,dx \leq \|f\|_{\infty}^{p}
\mu(\Omega)<+\infty,
$$
ainsi $L^{\infty}(\Omega) \subset L^{p}(\Omega)$ pour tout $p$ et
$\|f\|_{p} \leq
\|f\|_{\infty}\left(\mu(\Omega)\right)^{\frac{1}{p}}$. Montrons
que si $q\leq p$, alors $L^{p}(\Omega) \subset L^{q}(\Omega)$.
Soit $f\in L^{p}(\Omega)$, on a par exemple~:
\begin{eqnarray*}
\| f \|_q^q = \int_{\Omega}|f|^{q}(x)\,dx &=& \int_{\{|f|\geq1\}}|f|^{q}(x)\,dx
+
\int_{\{|f|<1\}} |f|^{q}(x)\,dx \\
& \leq & \int_{\{|f|\geq1\}}|f|^{p}(x)\,dx + \int_{\{|f|<1\}}
1\,dx\\
& \leq & \|f\|_{p}^{p} + \mu(\Omega)<+\infty.
\end{eqnarray*}
Ou encore, en utilisant l'in\'egalit\'e de H\"older pour les
r\'eels conjugu\'es $r = \frac{p}{q} > 1$ et $r' =
\frac{p}{p-q}$~:
\begin{eqnarray*}
\| f \|_q^q =  \int_{\Omega}|f|^{q}(x)\,dx &=& \left( \int_{\Omega}|f|^{q\cdot
\frac{p}{q}}(x)\,dx\right)^{\frac{q}{p}}
\left(\int_{\Omega} 1^{\frac{p}{p-q}}(x)\,dx\right)^{\frac{p-q}{p}} \\
& = & \|f\|_{p}^{q}~\mu(\Omega)^{\frac{p-q}{p}},
\end{eqnarray*}
ce qui implique~:
$$
\|f\|_{q} \leq \|f\|_{p} \mu(\Omega)^{\frac{p-q}{qp}}.
$$
 En conclusion, pour $1<q<2<p$~:
$$
L^{\infty}(\Omega) \subset L^{p}(\Omega) \subset L^{2}(\Omega)
\subset L^{q}(\Omega) \subset L^{1}(\Omega).
$$}
    \item \question{Soit $\mathcal{B}^{n}(0, 1)$ la boule unit\'e centr\'ee en
$0$ de $\mathbb{R}^{n}$. En consid\'erant les fonctions
$$
f_{\alpha}(x)= |x|^{-\alpha}
$$
montrer que pour $q<p$, l'inclusion $L^{p}(\mathcal{B}^{n}(0,1))
\subset L^{q}(\mathcal{B}^{n}(0,1))$ est stricte.}
\reponse{Montrons que pour $q<p$, l'inclusion
$L^{p}(\mathcal{B}^{n}(0,1)) \subset
L^{q}(\mathcal{B}^{n}(0,1))$ est stricte. La fonction
$f_{\alpha}$ appartient \`a $L^{\infty}(\mathcal{B}^{n}(0,1))$
si et seulement $\alpha \leq 0$, et \`a
$L^{p}(\mathcal{B}^{n}(0,1))$ avec $p<+\infty$ si et seulement
si
$$p\alpha - n +1 < 1 \Leftrightarrow \alpha < \frac{n}{p}
$$
Soit $1\leq q<p$, alors
$f_{\frac{1}{2}\left(\frac{n}{p}+\frac{n}{q}\right)}$ appartient
\`a $L^{q}(\mathcal{B}^{n}(0,1))\setminus
L^{p}(\mathcal{B}^{n}(0,1))$. En particulier,
$f_{\frac{1}{2}\left(\frac{n}{p}+\frac{n}{q}\right)}$ appartient
\`a $L^{q}(\mathcal{B}^{n}(0,1))\setminus
L^{\infty}(\mathcal{B}^{n}(0,1))$.}
\end{enumerate}
}
