\uuid{mLR7}
\exo7id{6966}
\auteur{blanc-centi}
\datecreate{2014-04-08}
\isIndication{true}
\isCorrection{true}
\chapitre{Polynôme, fraction rationnelle}
\sousChapitre{Fraction rationnelle}

\contenu{
\texte{
Soit $n\in\Nn^*$ et $P(X)=c(X-a_1)\cdots(X-a_n)$ 
(où les $a_i$ sont des nombres complexes et où $c\not=0$).
}
\begin{enumerate}
    \item \question{Exprimer à l'aide de $P$ et de ses dérivées les sommes suivantes:
$$\sum_{k=1}^n\frac{1}{X-a_k}\quad\quad
\sum_{k=1}^n\frac{1}{(X-a_k)^2}\quad\quad
\sum_{\substack{1\le k,\ell\le n \\ k\not= \ell}}\frac{1}{(X-a_k)(X-a_\ell)}$$}
    \item \question{Montrer que si $z$ est racine de $P'$ mais pas de $P$, alors il existe 
$\lambda_1,\hdots,\lambda_n$ des réels positifs ou nuls tels que 
$\sum_{k=1}^n\lambda_k=1$ et $z=\sum_{k=1}^n\lambda_ka_k$. 
Si toutes les racines de $P$ sont réelles, que peut-on en déduire sur les racines de $P'$ ?}
\reponse{
\begin{enumerate}
Puisque $P(X)=c(X-a_1)\cdots(X-a_n)$ : 
\begin{eqnarray*}
P'(X)=&c(X-a_2)\cdots(X-a_n)+c(X-a_1)(X-a_3)\cdots(X-a_n)\\
 &+\cdots+c(X-a_1)\cdots(X-a_{k-1})(X-a_{k+1})\cdots(X-a_n)\\
 &\ \ \ \ \ \ +\cdots+c(X-a_1)\cdots(X-a_{n-1})
\end{eqnarray*}
La dérivée est donc la somme des termes de la forme : $\frac{c(X-a_1)\cdots(X-a_n)}{X-a_k} = \frac{P(X)}{X-a_k}$.

Ainsi 
$$P'(X) = \frac{P(X)}{X-a_1}+ \cdots + \frac{P(X)}{X-a_k}+ \cdots + \frac{P(X)}{X-a_n}.$$
Donc :
$$\frac{P'}{P}=\sum_{k=1}^n\frac{1}{X-a_k}$$
Puisque $\sum_{k=1}^n\frac{1}{(X-a_k)^2}$ est la dérivée de $-\sum_{k=1}^n\frac{1}{X-a_k}$, on obtient
  par dérivation de $-\frac{P'}{P}$ :
$$\frac{P'^2-PP''}{P^2}=\sum_{k=1}^n\frac{1}{(X-a_k)^2}$$
On a remarqué que la dérivée de $P'$ est la somme de facteurs $c(X-a_1)\cdots(X-a_n)$
avec un des facteurs en moins, donc de la forme $\frac{c(X-a_1)\cdots(X-a_n)}{X-a_k} = \frac{P}{X-a_k}$. 
De même $P''$ est la somme de facteurs $c(X-a_1)\cdots(X-a_n)$
avec deux facteurs en moins, c'est-à-dire de la forme $\frac{c(X-a_1)\cdots(X-a_n)}{(X-a_k)(X-a_\ell)} = \frac{P}{(X-a_k)(X-a_\ell)}$ :
$$P'' = \sum_{\substack{1\le k,\ell\le n \\ k\not= \ell}}\frac{P}{(X-a_k)(X-a_\ell)} \quad \text{ donc } \quad
\frac{P''}{P} = \sum_{\substack{1\le k,\ell\le n \\ k\not= \ell}}\frac{1}{(X-a_k)(X-a_\ell)}$$

% \begin{eqnarray*}
% \sum_{k\not=l}\frac{1}{(X-a_k)(X-a_l)}&=&\sum_{k=1}^n\frac{1}{X-a_k}\left(\sum_{l\not= k}\frac{1}{X-a_l}\right)\\
%  &=&\sum_{k=1}^n\frac{1}{X-a_k}\left(\sum_{l=1}^n\frac{1}{X-a_l}-\frac{1}{X-a_k}\right)\\
%  &=&\sum_{k=1}^n\frac{1}{X-a_k}\frac{P'}{P}-\sum_{k=1}^n\frac{1}{(X-a_k)^2}\\
%  &=&\frac{P'^2}{P^2}-\frac{P'^2-PP''}{P^2}=\frac{P''}{P}
% \end{eqnarray*}
}
\indication{Considérer $P'/P$ et sa dérivée, et enfin $P''/P$.}
\end{enumerate}
}
