\uuid{4771}
\auteur{quercia}
\datecreate{2010-03-16}
\isIndication{false}
\isCorrection{true}
\chapitre{Topologie}
\sousChapitre{Topologie des espaces vectoriels normés}

\contenu{
\texte{

}
\begin{enumerate}
    \item \question{Soit $E$ un espace pr{\'e}hilbertien r{\'e}el et $u_1,\dots,u_n$ des {\'e}l{\'e}ments de~$E$.
    Calculer $\sum_\sigma\bigl\|\sum_{i=1}^n\sigma(i)u_i\bigr\|^2$ o{\`u} $\sigma$
    parcourt l'ensemble des fonctions de~$[[1,n]]$ dans $\{-1,1\}$.}
\reponse{$2^n\sum_{i=1}^n\|u_i\|^2$.}
    \item \question{On se place dans l'ensemble des fonctions continues de~$[0,1]$ dans~$\R$.
    Montrer que la norme infinie n'est {\'e}quivalente {\`a} aucune norme euclidienne.}
\reponse{Supposons qu'il existe une norme euclidienne $\|\ \|$ et deux r{\'e}els
    $\alpha,\beta>0$ tels que $\alpha\|u\|_\infty\le\|u\|\le\beta\|u\|_\infty$
    pour tout $u\in\mathcal{C}([0,1],\R)$.
    On pose $u(x) = 1-2|x|$ pour $x\in[\frac{-1}2,\frac12]$ et $u(x) = 0$
    sinon. Soit $n\in\N$ et pour $1\le i\le n$~: $u_i(x) = u((n+1)x-i)$.
    Alors $\sum_\sigma\bigl\|\sum_{i=1}^n\sigma(i)u_i\bigr\|^2 \le 2^n\beta^2$
    et $2^n\sum_{i=1}^n\|u_i\|^2 \ge 2^nn\alpha^2$ donc ces deux sommes ne peuvent
    rester {\'e}gales quand $n\to\infty$.}
    \item \question{M{\^e}me question avec la norme $\|\ \|_p$, $p\in{[1,+\infty[}\setminus\{2\}$.}
\reponse{M{\^e}me construction. On trouve
    $$\sum_\sigma\bigl\|\sum_{i=1}^n\sigma(i)u_i\bigr\|^2 \le 2^n\beta^2\|u\|_p^2\Bigl(\frac{n}{n+1}\Bigr)^{2/p}$$
    et $2^n\sum_{i=1}^n\|u_i\|^2 \ge 2^n\alpha^2\|u\|_p^2\frac{n}{(n+1)^{2/p}}$.}
\end{enumerate}
}
