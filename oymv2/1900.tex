\uuid{1900}
\auteur{legall}
\datecreate{2003-10-01}

\contenu{
\texte{
\ \par
\begin{center}{\large \bf I Pr\'eliminaires}
\end{center}
}
\begin{enumerate}
    \item \question{Soit $\mathcal{P} $ l'espace vectoriel des fonctions polynomiales de 
$[0,1]$ \`a valeurs dans $\Rr .$ Montrer que $\mathcal{P} $ est de dimension 
infinie.}
\reponse{Soit $\mathcal{P}$ l'espace vectoriel des fonctions polynomiales. 
Supposons $\mathcal{P}$ de dimension finie $n$. Notons $f_k$ la fonction 
$x\mapsto x^k$. Alors la famille $\{ f_0, \cdots ,
f_n \}$ qui compte $n+1$ \'el\'ements est li\'ee, donc il existe 
$a_0, \cdots , a_n$ des scalaires non tous nuls tels que, pour tout
$x\in \Rr $ on ait $a_0+a_1x+\cdots a_n x^n=0.$ Il en r\'esulte que 
le polyn\^ome non nul \`a coefficients r\'eels $a_0+a_1X+\cdots 
a_nX^n$ a une infinit\'e de racines, ce qui est
absurde.}
    \item \question{Soit $X$ une partie born\'ee de $\Rr .$ Montrer que $\sup 
(X)=\sup \bar{X}.$}
\reponse{Posons $M=\sup (\bar{X}).$ On doit v\'erifier que, {\em i)} 
pour tout $x\in X , x\leq M$ et {\em ii)} pour tout $\epsilon >0$ 
il existe $x\in X$ tel que
$M-\epsilon \leq x.$
Comme $X\subset \bar{X}$ et, pour tout $x\in \bar{X} , x\leq M$ la 
propri\'et\'e {\em i)} est v\'erifi\'ee par $M.$ Soit maintenant 
$\epsilon >0$. Il existe
$x\in \bar{X}$ tel que $M-\dfrac{\epsilon }{2}<x$. Comme 
$x\in\bar{X},$ il existe aussi $y\in X$ tel que $\vert x-y\vert 
<\dfrac{\epsilon }{2}$. Donc $ M-\epsilon< y$
et $M$ satisfait \`a {\em ii)}.
\vskip1mm
{\em Remarque : } on note \'egalement que $\sup (X)\in \bar{X}$. En 
effet, pour tout $n\in \Nn ,$ choisissons un \'el\'ement $x_n\in X$ 
tel que $x_n\geq \sup (x)-\dfrac{1}{n}.$
Alors la suite $(x_n)_{n\in \Nn}$ constitu\'ee d'\'el\'ements de $X$ 
converge dans $\Rr$ vers $\sup (X)$ qui appartient donc \`a $ 
\bar{X}.$ On peut bien s\^ur en d\'eduire la
propri\'et\'e {\em ii)} de $M$.}
\end{enumerate}
}
