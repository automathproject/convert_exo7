\uuid{1OZm}
\exo7id{5340}
\auteur{rouget}
\datecreate{2010-07-04}
\isIndication{false}
\isCorrection{true}
\chapitre{Polynôme, fraction rationnelle}
\sousChapitre{Fraction rationnelle}

\contenu{
\texte{
Calculer la dérivée $n$-ième de $\frac{1}{X^2+1}$.
}
\reponse{
Soit $n\in\Nn^*$. $\frac{1}{X^2+1}=\frac{1}{2i}(\frac{1}{X-i}-\frac{1}{X+i})$. Donc,

\begin{align*}\ensuremath
\left(\frac{1}{X^2+1}\right)^{(n)}&=\frac{1}{2i}(\left(\frac{1}{X-i}\right)^{(n)}-\left(\frac{1}{X+i}\right)^{(n)})=\frac{1}{2i}(\frac{(-1)(-2)...(-n)}{(X-i)^{n+1}}-\frac{(-1)(-2)...(-n)}{(X+i)^{n+1}})\\
 &=(-1)^n.n!\mbox{Im}(\frac{1}{(X-i)^{n+1}})=(-1)^n.n!\mbox{Im}(\frac{(X+i)^{n+1}}{(X^2+1)^{n+1}})
=\frac{(-1)^n.n!\sum_{}^{}C_{n+1}^{2k+1}(-1)^kX^{2n-2k}}{(X^2+1)^{n+1}}
\end{align*}
}
}
