\uuid{7105}
\auteur{megy}
\datecreate{2017-01-21}
\isIndication{false}
\isCorrection{true}
\chapitre{Géométrie affine euclidienne}
\sousChapitre{Géométrie affine euclidienne du plan}

\contenu{
\texte{
% source : Debart "rotation au collège"
% tags: collège, rotation

Sur les côtés $[AB]$ et $[BC]$ d'un carré direct $ABCD$, on place des points $M$ et $N$vérifiant $AM = BN$. Soit $H$ le point d'intersection des droites $(AN)$ et $(CM)$. Montrer que $H$ est l'orthocentre du triangle $DMN$.
}
\reponse{
La rotation de centre $O$ (le centre du carré) et d'angle $\pi/2$ envoie le triangle $DAM$ sur $ABN$. On en déduit que $(DM)\bot (AN)$ et donc que $(AN)$ est une hauteur de $DMN$. On procède de même pour la deuxième hauteur.
}
}
