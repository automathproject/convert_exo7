\uuid{2586}
\auteur{delaunay}
\datecreate{2009-05-19}

\contenu{
\texte{
\centerline{\bf I}

Soit $\alpha\in\R$ et $A_\alpha\in M_3(\R)$ la matrice suivante
$$A_\alpha=\begin{pmatrix}-1 & 0 & \alpha+1 \\  1&-2&0 \\ -1&1&\alpha\end{pmatrix}$$
{\it Premi\`ere partie} :
}
\begin{enumerate}
    \item \question{Factoriser le polyn\^ome caract\'eristique $P_{A_\alpha}(X)$ en produit de facteurs du premier degr\'e.}
\reponse{{\it Factorisons le polyn\^ome caract\'eristique $P_{A_\alpha}(X)$ en produit de facteurs du premier degr\'e.}

On a
\begin{align*}
P_{A_\alpha}(X)
&=\begin{vmatrix}-1-X & 0 & \alpha+1 \\  1&-2-X&0 \\ -1&1&\alpha-X\end{vmatrix}
=\begin{vmatrix}-1-X & 0 & \alpha+1 \\  -1-X&-2-X&0 \\ 0&1&\alpha-X\end{vmatrix} \\ 
&=\begin{vmatrix}-1-X & 0 & \alpha+1 \\  0&-2-X&-\alpha-1 \\ -1&1&\alpha-X\end{vmatrix} \\ 
&=(-1-X)[(-2-X)(\alpha-X)+\alpha+1] \\ 
&=-(X+1)[X^2+(2-\alpha)X+1-\alpha].
\end{align*}
Factorisons le polyn\^ome $X^2+(2-\alpha)X+1-\alpha$, son discriminant est \'egal \`a
$$\Delta=(2-\alpha)^2-4(1-\alpha)=\alpha^2.$$
On a donc $\sqrt{\Delta}=|\alpha|$, ce qui nous donne les deux racines 
$$\lambda_1={\frac{\alpha-2-\alpha}{2}}=-1\ \ {\hbox{et}}\ \ 
\lambda_2={\frac{\alpha-2+\alpha}{2}}=\alpha-1.$$
Le polyn\^ome caract\'eristique $P_{A_\alpha}(X)$ se factorise donc en
$$P_{A_\alpha}(X)=-(X+1)^2(X-\alpha+1).$$}
    \item \question{D\'eterminer selon la valeur du param\`etre $\alpha$ les valeurs propres distinctes de $A_\alpha$ et leur multiplicit\'e.}
\reponse{{\it D\'eterminons selon la valeur du param\`etre $\alpha$ les valeurs propres distinctes de $A_\alpha$ et leur multiplicit\'e.}

Les valeurs propres de $A_\alpha$ sont les racines du polyn\^ome caract\'eristique $P_{A_\alpha}$, ainsi,

- si $\alpha=0$, la matrice $A_\alpha$ admet une valeur propre triple $\lambda=-1$,

- si $\alpha\neq0$, la matrice $A_\alpha$ admet deux valeurs propres distinctes
$\lambda_1=-1$ valeur propre double et $\lambda_2=\alpha-1$, valeur propre simple.}
    \item \question{D\'eterminer les valeurs de $\alpha$ pour lesquelles la matrice $A_\alpha$ est diagonalisable.}
\reponse{{\it D\'eterminons les valeurs de $\alpha$ pour lesquelles la matrice $A_\alpha$ est diagonalisable.}

Il est clair que dans le cas $\alpha=0$, la matrice n'est pas diagonalisable, en effet si elle l'\'etait, il existerait une matrice inversible $P$ telle que $A_\alpha=P(-I)P^{-1}=-I$, ce qui n'est pas le cas.

Si $\alpha\neq0$, la matrice $A_\alpha$ est diagonalisable si le sous-espace propre associ\'e \`a la valeur propre $-1$ est de dimension $2$. D\'eterminons ce sous-espace propre.

$$E_{-1}=\ker(A_\alpha+I)=\left\{(x,y,z)\in\R^3,\ \begin{pmatrix}-1 & 0 & \alpha+1 \\  1&-2&0 \\ -1&1&\alpha\end{pmatrix}\begin{pmatrix}x \\  y \\  z\end{pmatrix}=\begin{pmatrix}-x \\  -y \\  -z\end{pmatrix}\right\}$$
ainsi,
$$(x,y,z)\in E_{-1}\iff\left\{\begin{align*}-x+(\alpha+1)z&=-x \\  x-2y&=-y \\  -x+y+\alpha z&=-z\end{align*}\right.
\iff\left\{\begin{align*}(\alpha+1)z&=0 \\  x-y=0\end{align*}\right.$$
Il faut distinguer les cas $\alpha=-1$ et $\alpha\neq -1$.

- Si $\alpha=-1$, le sous-espace $E_{-1}$ est le plan vectoriel d'\'equation $x=y$, dans ce cas la matrice $A_\alpha$ est diagonalisable.

- Si $\alpha\neq -1$, le sous-espace $E_{-1}$ est la droite vectorielle engendr\'ee par le vecteur
$(1,1,0)$, dans ce cas la matrice $A_\alpha$ n'est pas diagonalisable.}
    \item \question{D\'eterminer selon la valeur de $\alpha$ le polyn\^ome minimal de $A_\alpha$.}
\reponse{{\it D\'eterminons selon la valeur de $\alpha$ le polyn\^ome minimal de $A_\alpha$.}

 Notons $Q_A$ le polyn\^ome minimal de $A_\alpha$. On sait que la matrice $A_\alpha$ est diagonalisable sur $\R$ si et seulement si son polyn\^ome minimal a toutes ses racines dans $\R$ et que celles-ci sont simples. Or, nous venons de d\'emontrer que $A_\alpha$ est diagonalisable sur $\R$ si et seulement $\alpha=-1$, on a donc 

- Si $\alpha=-1$, $A_\alpha$ est diagonalisable, donc $Q_A(X)=(X+1)(X-\alpha+1)=(X+1)(X+2)$.

- Si $\alpha\neq -1$, $A_\alpha$ n'est pas diagonalisable, donc $Q_A(X)=P_A(X)=(X+1)^2(X-\alpha+1)$.}
\end{enumerate}
}
