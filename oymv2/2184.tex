\uuid{2184}
\auteur{debes}
\datecreate{2008-02-12}
\isIndication{true}
\isCorrection{true}
\chapitre{Action de groupe}
\sousChapitre{Action de groupe}

\contenu{
\texte{
\label{ex:deb84}
Soit $G$ un groupe fini et $X$ un $G$-ensemble transitif. On
dira que $X$ est
{\it imprimitif } si $X$ admet une partition $X=\bigcup _{1\leq i\leq r}
X_i$ telle que tout \'el\'ement
$g$ de $G$ respecte cette partition, i.e. envoie un sous-ensemble $X_i$
sur un sous-ensemble $X_k$
(\'eventuellement $k=i$) et telle que $2\leq r$ et les parties $X_i$ ne
sont pas r\'eduites \`a un
\'el\'ement. Dans le cas contraire on dit que $X$ est {\it primitif}.
\smallskip

(a) Montrer que dans la d\'ecomposition pr\'ec\'edente, si elle existe,
tous les sous-ensembles $X_i$ ont
m\^eme nombre $m$ d'\'el\'ements.
\smallskip

(b) Soit $H$ un sous-groupe de $G$. Montrer que $G/H$ est imprimitif si et
seulement s'il existe un
sous-groupe propre $K$ de $G$ diff\'erent de $H$ tel que $H\subset K
\subset G$ (on regardera la
partition de $G/H$ en classes modulo $K$).
\smallskip

(c) D\'eduire de ce qui pr\'ec\`ede que $X$ est primitif si et seulement si
le fixateur d'un \'el\'ement $x$
de $X$ est maximal parmi les sous-groupes propres de $G$.
\smallskip

(d) On suppose ici que $X$ est primitif et que $H$ est un sous-groupe
distingu\'e de $G$ dont l'action
n'est pas triviale sur $X$. Montrer qu'alors $H$ agit transitivement sur $X$.
}
\indication{Question (d): Si $K$ le fixateur d'un \'el\'ement $x\in X$, alors $K$ est un sous-groupe
propre maximal de $G$ et $X$ est isomorphe \`a $G/\cdot K$ en tant que $G$-ensemble. D\'eduire
du fait que $H$ n'est pas contenu dans $K$ que $HK=G$ et que
$H/\cdot H\cap K\simeq G/\cdot K$.}
\reponse{
(a) Pour $1\leq i,j \leq r$ quelconques et $x_i,x_j \in X_i \times X_j$, il existe $g\in G$
tel que $g\cdot x_i=x_j$ (par transitivit\'e de $G$). On a alors $g\cdot X_i = X_j$. En
particulier
$\hbox{\rm card}(X_i)= \hbox{\rm card}(g\cdot X_i)=\hbox{\rm card}(X_j)$.
\medskip

(b) Si l'action de $G$ sur $G/\cdot H$ est imprimitive, le sous-ensemble $K=\{g\in G\hskip 2pt
|\hskip 2pt g\cdot X_1 = X_1\}$, o\`u $X_1$ est par exemple celui des sous-ensembles
$X_i\subset X$ qui contient la classe neutre $H$ de $G/\cdot H$, est un sous-groupe propre de
$G$ ($K\not=G$ car
$G$ agissant transitivement, il existe $g\in G$ tel que $(g\cdot X_1) \cap X_2\not=\emptyset$)
et contenant strictement $H$ (car encore par transitivit\'e, il existe $g\in G$ tel que
$g\cdot H$ soit un \'el\'ement de $X_1$ (ce qui assure que $g\in K$) mais diff\'erent de $H$
(ce qui assure que $g\notin H$)). 

\hskip 5mm Inversement, si un tel sous-groupe $K$ de $G$ existe, la relation ``$gH \sim g^\prime H$
si $(g^\prime)^{-1} g\in K$'' est bien d\'efinie sur $G/\cdot H$ (la d\'efinition ne
d\'epend pas des repr\'esentants dans $G$ des classes $gH$ et $g^\prime H$) et est une
relation d'\'equivalence (imm\'ediat). La partition associ\'ee de $G/\cdot H$ en classes
d'\'equivalence v\'erifie les conditions de la d\'efinition d'imprimitivit\'e (pour l'action
de $G$ sur $G/\cdot H$): la partition est non triviale car $K$ est strictement contenu entre
$H$ et $K$; et si $(\gamma H)K$ est une de ces classes d'\'equivalence et
$g\in G$, alors $g\cdot (\gamma H)K$ est la classe $(g\gamma H)K$: l'action de $G$ permute
bien les classes constituant la partition de $X$.
\medskip

(c) D'apr\`es l'exercice \ref{ex:deb83}, les ensembles $X$ et $G/\cdot G(x)$ sont isomorphes comme
$G$-ensembles. L'action de $G$ sur $X$ est primitive si et seulement si celle de $G$ sur
$G/\cdot G(x)$ l'est, ce qui, d'apr\`es la question pr\'ec\'edente, \'equivaut \`a dire que le
fixateur $G(x)$ est maximal parmi les sous-groupes de $G$.
\medskip 

(d) Soient $x\in X$ et $G(x)$ son fixateur. Le sous-groupe $H$ \'etant distingu\'e dans
$G$, l'ensemble $HG(x)$ est un sous-groupe; c'est le sous-groupe engendr\'e par $H$ et
$G(x)$. De plus, l'action de $H$ sur $G$ n'\'etant pas triviale, $H$ n'est pas contenu dans
$G(x)$ et par cons\'equent $HG(x)$ contient strictement $G(x)$. D'apr\`es la question (c), il
en r\'esulte que $HG(x) = G$. On v\'erifie sans peine que l'application $H/\cdot (H\cap G(x))
\rightarrow (HG(x))/\cdot G(x)$ qui \`a toute classe $h(H\cap G(x))$ associe la classe $hG(x)$
est une bijection (ce qui g\'en\'eralise le th\'eor\`eme d'isomorphisme
$HK/K\simeq H/(H\cap K)$ qui est vrai sous l'hypoth\`ese suppl\'ementaire ``$K$
distingu\'e'' (qui assure que les ensembles $HK/K$ et $H/(H\cap K)$ sont des groupes et non
de simples ensembles comme ici)). On obtient donc que les ensembles $H/\cdot (H\cap
G(x))$ et $G/\cdot G(x)$ sont isomorphes comme $G$-ensembles (la compatibilit\'e des
actions est imm\'ediate). Or ces deux ensembles sont en bijection avec les orbites de $x$
sous $H$ et sous $G$ respectivement. Conclusion: l'action de $H$ est, comme celle de $G$,
transitive sur l'ensemble $X$.
}
}
