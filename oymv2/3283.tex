\uuid{3283}
\auteur{quercia}
\datecreate{2010-03-08}

\contenu{
\texte{
Soit $F(X) = \frac 1{R(X)} = \frac 1{(X-a)^2Q(X)}$ avec $Q(a) \ne 0$.
Chercher la partie polaire de $F$ en $a$ en fonction de $Q$ puis en fonction
de~$R$.
}
\reponse{
$\frac 1{Q(a)(X-a)^2}    - \frac {Q'(a)}{Q^2(a)(X-a)} =
\frac 2{R''(a)(X-a)^2} - \frac {2R'''(a)}{3{R''}^2(a)(X-a)}$.
}
}
