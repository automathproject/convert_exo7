\uuid{3791}
\auteur{quercia}
\datecreate{2010-03-11}
\isIndication{false}
\isCorrection{true}
\chapitre{Espace euclidien, espace normé}
\sousChapitre{Endomorphismes auto-adjoints}

\contenu{
\texte{
Soit $E$ un espace euclidien et $f\in\mathcal{L}(E)$.
}
\begin{enumerate}
    \item \question{Montrer~: $\mathrm{Ker} f = \Im f  \Rightarrow  f + f^* \in GL(E)$.}
\reponse{Si $f(x) + f^*(x) = 0$ alors
$f(x) \in \Im f \cap \Im f^* = \Im f \cap (\mathrm{Ker} f)^\bot
                             = \Im f \cap (\Im f)^\bot$
donc $f(x) = f^*(x) = 0$ et
$x\in \mathrm{Ker} f \cap \mathrm{Ker} f^* = \mathrm{Ker} f \cap (\mathrm{Ker} f)^\bot$.}
    \item \question{Montrer la réciproque lorsque l'on a $f^2 = 0$.}
\reponse{$f^2 = 0  \Rightarrow  \Im f \subset \mathrm{Ker} f$.\par
$f+f^* \in GL(E)  \Rightarrow  \Im f + \Im f^* = \Im f + (\mathrm{Ker} f)^\bot = E                  \Rightarrow  \dim \Im f \ge \dim \mathrm{Ker} f$.}
\end{enumerate}
}
