\uuid{824}
\auteur{cousquer}
\datecreate{2003-10-01}
\isIndication{false}
\isCorrection{true}
\chapitre{Calcul d'intégrales}
\sousChapitre{Fraction rationnelle}

\contenu{
\texte{
Décomposer les fractions rationnelles suivantes~; en
calculer les primitives.
}
\begin{enumerate}
    \item \question{$\displaystyle{1 \over a^2+x^2}$.}
\reponse{${1 \over x^2+a^2}$ est un élément simple.
Primitives~: ${1 \over a} \arctan({x \over a})+k$.}
    \item \question{${1 \over{(1+x^2)}^2}$.}
\reponse{${1 \over{(1+x^2)}^2}$ est un élément simple.
Primitives~: ${1 \over2} \arctan x + {x \over2(1+x^2)}+k$.}
    \item \question{$\displaystyle{x^3 \over x^2-4}$.}
\reponse{${x^3 \over x^2-4}=x+{2 \over x-2}+{2 \over x+2}$.
Primitives~: ${x^2 \over2}+ \ln(x^2-4)^2 +k$.}
    \item \question{$\displaystyle{4x \over{(x-2)}^2}$.}
\reponse{${4x \over{(x-2)}^2}={4 \over x-2}+{8 \over{(x-2)}^2}$.
Primitives~: $4 \ln\vert x-2\vert-{8 \over x-2}+k$.}
    \item \question{$\displaystyle{1 \over x^2+x+1}$.}
\reponse{${1 \over x^2+x+1}$ est un élément simple.
Primitives~: ${2 \over\sqrt3} \arctan{(2x+1) \over\sqrt3}+k$.}
    \item \question{$\displaystyle{1 \over{(t^2+2t-1)}^2}$.}
\reponse{$ {1 \over{(t^2+2t-1)}^2} = {1 \over8{(t+1+ \sqrt2)}^2 }+
{ \sqrt2 \over16(t+1+\sqrt2)} + {1 \over8{(t+1- \sqrt2)}^2}+
{ - \sqrt2 \over16(t+1- \sqrt2)}$.\newline
Primitives~: $-{t+1 \over4(t^2+2t-1)} +{\sqrt2 \over16} \ln
\left\vert{t+1+ \sqrt2 \over t+1- \sqrt2 } \right\vert+ k$.}
    \item \question{$\displaystyle{3t+1 \over{(t^2-2t+10)}^2}$.}
\reponse{$ {3t+1 \over{(t^2-2t+10)}^2}$ est un élément simple.\newline
Primitives~: $-{3 \over2(t^2-2t+10)} +{2(t-1) \over9(t^2-2t+10)}
+{2 \over27} \arctan({t-1 \over3}) +k $.}
    \item \question{$\displaystyle{3t+1 \over{t^2-2t+10}}$.}
\reponse{$ {3t+1 \over t^2-2t+10}$ est un élément simple.
Primitives~: ${3 \over2} \ln(t^2-2t+10) +{4 \over3} \arctan({t-1
\over3})+k$.}
    \item \question{$\displaystyle{ 1 \over{t^3+1}}$.}
\reponse{${1 \over t^3+1}={1 \over3(t+1)}-{t-2 \over3(t^2-t+1)}$.
Primitives~: $ {1 \over3}\ln\vert t+1 \vert-{1 \over6} \ln(
t^2-t+1) + {1 \over\sqrt3} \arctan({2t-1 \over\sqrt3}) +k$.}
    \item \question{$\displaystyle{x^3+2 \over{(x+1)}^2}$.}
\reponse{$ {x^3+2 \over{(x+1)}^2}= x-2+{3 \over x+1}+{1 \over{(x+1)}^2}$.
Primitives~: ${x^2 \over2}-2x+3 \ln\vert x+1 \vert-{1 \over x+1} +k$.}
    \item \question{$\displaystyle{x+1 \over{x{(x-2)}^2}}$.}
\reponse{${x+1 \over x{(x-2)}^2}= {1 \over4x}- {1 \over4(x-2)}
 +{3 \over2{(x-2)}^2}$.
Primitives~: $ {1 \over4} \ln\vert x \vert-{1 \over4} \ln\vert x-2 \vert
-{3 \over2(x-2)} +k$.}
    \item \question{$\displaystyle{(x^2-1)(x^3+3) \over2x+2x^2}$.}
\reponse{${(x^2-1)(x^3+3) \over2x+2x^2}= {1 \over2} (x^3-x^2+3) -{3 \over2x}
$.
Primitives~: ${x^4 \over8}-{x^3 \over6}+{3x \over2}-{3 \over2}\ln\vert x
\vert+k$.}
    \item \question{$\displaystyle{x^2 \over{{(x^2+3)}^3 (x+1)}}$.}
\reponse{${x^2 \over{(x^2+3)}^3(x+1)}={1 \over4^3(x+1)}+
{1-x \over4^3 (x^2+3)} +
{1-x \over4^2 {(x^2+3)}^2}
-{3(1-x) \over4{(x^2+3)}^3}$.\newline
Primitives~: $-{x+3 \over4^2{(x^2+3)}^2}-{2x-3 \over3.2^5(x^2+3)}
-{1 \over2^7}\ln(x^2+3) - {1 \over3 \sqrt3  \, 2^6} \arctan({x \over\sqrt3})
+{1 \over4^3}\ln\vert x+1 \vert+k$.}
    \item \question{$\displaystyle{x^7+x^3-4x-1 \over x{(x^2+1)}^2}$.}
\reponse{${x^7+x^3-4x-1 \over x{(x^2+1)}^2}= x^2-2-{1 \over x} +{x+4 \over x^2+1}+
{x-6 \over{(x^2+1)}^2}$.\newline
Primitives~: ${x^3 \over3}-2x -\ln\vert x \vert+{1 \over2}\ln
(1+x^2) +\arctan x -{6x+1 \over2(x^2+1)} +k$.}
    \item \question{$\displaystyle{3x^4-9x^3+12x^2-11x+7\over(x-1)^3(x^2+1)}$.}
\reponse{${3x^4-9x^3+12x^2-11x+7\over(x-1)^3(x^2+1)}={1\over
(x-1)^3}-{2\over(x-1)^2}+{3\over x-1}-{1\over x^2+1}$.\newline
Primitives~: $-{1/2\over(x-1)^2}+{2\over x-1}+3\ln\vert
x-1\vert-\arctan x +k$.}
\end{enumerate}
}
