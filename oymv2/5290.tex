\uuid{5290}
\auteur{rouget}
\datecreate{2010-07-04}

\contenu{
\texte{
Soit $P_n^k$ le nombre de partitions d'un ensemble à $n$ éléments en $k$ classes.

Montrer que $P_n^k=P_{n-1}^{k-1}+kP_{n-1}^k$ pour $2\leq k\leq n-1$.

Dresser un tableau pour $1\leq k,n\leq 5$.

Calculer en fonction de $P_n^k$ le nombre de surjections d'un ensemble à $n$ éléments sur un ensemble à $p$ éléments.
}
\reponse{
Soient $n$ et $k$ des entiers naturels tels que $2\leq k\leq n-1$.

Soit $E$ un ensemble à $n$ éléments et $a$ un élément fixé de $E$.

Il y a $P_n^k$ partitions de $E$ en $k$ classes. Parmi ces partitions, il y a celles dans lesquelles $a$ est dans un singleton. Elles s'identifient aux partitions en $k-1$ classes de $E\setminus\{a\}$ et sont au nombre de $P_{n-1}^{k-1}$. Il y a ensuite les partitions dans lesquelles $a$ est élément d'une partie de cardinal au moins $2$. Une telle partition est obtenue en partitionnant $E\setminus\{a\}$ en $k$ classes puis en adjoignant à l'une de ces $k$ classes au choix l'élément $a$. Il y a $kP_{n-1}{k}$ telles partitions. Au total, $P_n^k=P_{n-1}^{k-1}+kP_{n-1}^k$.

Valeurs de $P_n^k$ pour $1\leq k,n\leq 5$.

$$\begin{array}{|c|c|c|c|c|c|}
\hline
n\;/\;k&1&2&3&4&5\\
\hline
1&1&0&0&0&0\\
\hline
2&1&1&0&0&0\\
\hline
3&1&3&1&0&0\\
\hline
4&1&7&6&1&0\\
\hline
5&1&15&25&10&1\\
\hline
\end{array}$$
Exprimons maintenant en fonction des $P_n^k$, le nombre de surjections d'un ensemble à $n$ éléments dans un ensemble à $p$ éléments.

Si $p>n$, il n'y a pas de surjections de $E_n$ dans $E_p$ (où $E_n$ et $E_p$ désignent des ensembles à $n$ et $p$ éléments respectivement).

On suppose dorénavant $p\leq n$. La donnée d'une surjection $f$ de $E_n$ sur $E_p$ équivaut à la donnée d'une partition de l'ensemble $E_n$ en $p$ classes (chaque élément d'une même classe ayant même image par $f$) puis d'une bijection de l'ensemble des parties de la partition vers $E_p$.

Au total, il y a donc $p!P_n^k$ surjections d'un ensemble à $n$ éléments dans un ensemble à $p$ éléments pour $1\leq p\leq n$.
}
}
