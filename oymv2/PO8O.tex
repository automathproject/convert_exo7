\uuid{PO8O}
\exo7id{7373}
\auteur{mourougane}
\datecreate{2021-08-10}
\isIndication{false}
\isCorrection{true}
\chapitre{Groupe, anneau, corps}
\sousChapitre{Anneau}

\contenu{
\texte{
Voici la table d'un groupe $G$.
Quel est l'ordre de $G$ ? Le groupe $G$ est-il nécessairement commutatif ?
 Compléter la table en énonçant précisément les propriétés utilisées.
$$\begin{array}{|c|c|c|c|c|c|}\hline
 * & a & b 	& c &	d & e\\ \hline
 a & 	 & 	& d &	 & c \\ \hline
 b & e &	&a  &	 & d \\ \hline
 c &	 &a 	 &  &  &	 \\ \hline
 d &	 &	 &  &d &e \\ \hline
 e &	 &	 &b &	 & a \\ \hline
\end{array}
$$
}
\reponse{
Comme il s'agit d'un groupe d'ordre $5$, il est cyclique donc commutatif.
La table se complète donc en 
$$\begin{array}{|c|c|c|c|c|c|}\hline
 * & a & b 	& c &	d & e\\ \hline
 a & 	 & e	& d &	 & c \\ \hline
 b & e &	&a  &	 & d \\ \hline
 c &	d &a 	 &  &  &b \\ \hline
 d &	 &	 &  &d &e \\ \hline
 e & c & d	 &b &	 e & a \\ \hline
\end{array}
$$
Comme $de=e$, $d$ est l'élément neutre.
La table se complète donc en 
$$\begin{array}{|c|c|c|c|c|c|}\hline
 * & a & b 	& c &	d & e\\ \hline
 a & 	 & e	& d &	 a & c \\ \hline
 b & e &	&a  &	b & d \\ \hline
 c &	d &a 	 &  & c &b \\ \hline
 d &	a &b	 & c &d &e \\ \hline
 e & c & d	 &b &	 e & a \\ \hline
\end{array}
$$
Par la propriété de carré latin, 
on peut terminer le tableau
$$\begin{array}{|c|c|c|c|c|c|}\hline
 * & a & b 	& c &	d & e\\ \hline
 a & 	b & e	& d &	 a & c \\ \hline
 b & e &	c&a  &	b & d \\ \hline
 c &	d &a 	 & e & c &b \\ \hline
 d &	a &b	 & c &d &e \\ \hline
 e & c & d	 &b &	 e & a \\ \hline
\end{array}
$$
}
}
