\uuid{tGz0}
\exo7id{1895}
\auteur{legall}
\datecreate{2003-10-01}
\isIndication{false}
\isCorrection{false}
\chapitre{Espace vectoriel normé}
\sousChapitre{Espace vectoriel normé}

\contenu{
\texte{
On munit 
$C[0,1]$, l'espace vectoriel des fonctions continues sur $[0,1]$ \`a 
valeurs r\'eelles
de la norme $\Vert f\Vert _{\infty }= \displaystyle{ \sup _{x\in 
[0,1]}\vert f(x)\vert }.$
}
\begin{enumerate}
    \item \question{Soit $\varphi : C[0,1]\rightarrow \Rr $
une application lin\'eaire. On pose $N(\varphi ) =\displaystyle{ \sup 
_{f\in C[0,1]; \Vert f \Vert _{\infty}=1}\vert \varphi (f)\vert } .$
Montrer que $\varphi $ est continue si et seulement si $N(\varphi ) $ est fini.}
    \item \question{Calculer $N(\psi )$ lorsque  $\psi (f) = \displaystyle{\int _0 
^1f(t)dt }.$}
    \item \question{Posons, pour toute fonction $f\in C[0,1]$ : $\varphi (f) = 
\displaystyle{\int _0 ^\frac{1}{2}f(t)dt -\int _\frac{1}{2}^1f(t)dt}.$
Montrer que $N(\varphi )=1$.}
\end{enumerate}
}
