\uuid{646Z}
\exo7id{2401}
\auteur{mayer}
\datecreate{2003-10-01}
\isIndication{false}
\isCorrection{true}
\chapitre{Espace métrique complet, espace de Banach}
\sousChapitre{Espace métrique complet, espace de Banach}

\contenu{
\texte{
Soient $E,F$ des espaces norm\'es et $A_n,A \in {\cal L}(E,F)$.
Montrer l'\'equivalence entre:
}
\begin{enumerate}
    \item \question{$A_n \to A$ dans ${\cal L}(E,F)$.}
\reponse{(1) $\Rightarrow$ (2). Supposons que $A_n$ converge vers $A$ dans 
${\cal L}(E,F)$. Soit $M\subset E$ une partie bornée, notons $M$ sa borne
(c'est-à-dire pour tout $x\in M$, $\|x\|\le B$).
Alors 
\begin{align*}
 & \forall \epsilon >0\quad \exists N\in \Nn \quad \forall n\ge N \quad \|A_n-A\| \le \frac \epsilon B \\
 \Rightarrow \qquad & \forall \epsilon >0\quad \exists N\in \Nn \quad \forall n\ge N \quad \forall x \in M \quad \|A_n(x)-A(x)\| \le \frac {\epsilon\|x\|} B \\
\Rightarrow \qquad & \forall \epsilon >0\quad \exists N\in \Nn \quad \forall n\ge N \quad \forall x \in M \quad \|A_n(x)-A(x)\| \le \epsilon \\
\end{align*}
Ce qui exactement la confergence uniforme de $A_n$ vers $A$ sur $M$.}
    \item \question{Pour toute partie born\'ee $M\subset E$, la suite $A_n x$
converge uniform\'ement vers $Ax$, $x\in M$.}
\reponse{(2) $\Rightarrow$ (1). Par définition de la norme d'un opérateur nous avons $\| A_n - A\| = \sup_{\|x\|=1} \|A_n(x)-A(x)\|$.
Prenons comme partie bornée la sphère unité : $M = S(0,1) =\{x\in E \mid \|x\|=1\}$.
Alors :
\begin{align*}
 &\forall \epsilon >0\quad \exists N\in \Nn \quad \forall n\ge N \quad \forall x \in S(0,1) \in \|A_n(x)-A(x)\| \le \epsilon \\
 \Rightarrow \qquad & \forall \epsilon >0\quad \exists N\in \Nn \quad \forall n\ge N \quad \|A_n-A\| \le \epsilon \\
\end{align*}
Donc $\|A_n-A\|$ tend vers $0$.}
\end{enumerate}
}
