\uuid{buUX}
\exo7id{2415}
\auteur{mayer}
\datecreate{2003-10-01}
\isIndication{true}
\isCorrection{true}
\chapitre{Théorème de Stone-Weirstrass, théorème d'Ascoli}
\sousChapitre{Théorème de Stone-Weirstrass, théorème d'Ascoli}

\contenu{
\texte{
Soient $(E,d)$ un espace m\'etrique et ${\cal H}$ une famille \'equicontinue d'appli\-cations
de $E$ dans $\Rr$. \'Etablir:
}
\begin{enumerate}
    \item \question{L'ensemble $A$ des $x\in E$ pour lesquels ${\cal H} (x)$ est born\'e est ouvert et ferm\'e.}
    \item \question{Si $E$ est compact et connexe et si ${\cal H} (x_0)$ est born\'e pour un point quelconque $x_0 \in E$,
alors ${\cal H}$ est relativement compact dans ${\cal C} (E, \Rr )$.}
\reponse{
\begin{enumerate}
Montrons que $A$ est ouvert. Soit $x\in A$, alors $\mathcal{H}(x) = \{f(x) \mid f\in \mathcal{H}\}$ est bornée, notons $M$ une borne. \'Ecrivons l'équicontinuité pour $\epsilon = 1$, il existe $\eta >0$ tel que
$$\forall f \in \mathcal{H} \quad \forall y\in E \quad (\|x-y\| < \eta \Rightarrow  |f(x)-f(y)| < 1).$$
Or si $|f(x)-f(y)| < 1$ alors $|f(y)| < |f(x)| + 1 \le M+1$. On a donc montré
$$\forall f \in \mathcal{H} \quad \forall y\in E \quad (y\in B(x,\eta) \Rightarrow  |f(y)| < M+1).$$
Donc $B(x,\eta) \subset A$. Donc $A$ est ouvert.
Montrons que $A$ est fermé. Soit $(x_n)$ une suite d'éléments de $A$
qui converge vers $x\in E$. On reprend $\epsilon = 1$ et on obtient un $\eta$ par équicontinuité. Comme $x_n\rightarrow x$ alors il existe $N$ tel que $\|x_N-x\| < \eta$. Donc pour tout $f$ dans $\mathcal{H}$, $|f(x)-f(x_N)| < 1$ ; donc
$|f(x)| < |f(x_N)|+1$. Or $x_N \in A$, il existe $M$ tel  $|f(x_N)|$ soit bornée par $M$ pour tout $f$ dans $\mathcal{H}$.
Donc pour tout $f\in \mathcal{H}$, $|f(x)| < M+1$. Donc $x\in A$. Donc $A$ est fermé.
}
\indication{\begin{enumerate}
  \item Pour ouvert et fermé, écrire l'\'equicontinuité pour $\epsilon=1$ en un point $x$ (à fixer).

  \item Ascoli...
\end{enumerate}}
\end{enumerate}
}
