\uuid{g4zK}
\exo7id{229}
\auteur{cousquer}
\datecreate{2003-10-01}
\isIndication{true}
\isCorrection{true}
\chapitre{Dénombrement}
\sousChapitre{Binôme de Newton et combinaison}

\contenu{
\texte{
Calculer le module et l'argument de $(1+i)^n$.
En d\'eduire les valeurs de
\begin{eqnarray*}
S_1 & = & 1-C_n^2+C_n^4-C_n^6+\cdots\\
S_2 & = & C_n^1-C_n^3+C_n^5-\cdots
\end{eqnarray*}
}
\indication{$1+i= \frac{\sqrt2}2e^{\frac{2i\pi}{4}}$}
\reponse{
$A=(1+i)^n$ a pour module $2^{n/2}$ et pour argument $n{\pi \over 4}$
(et $B$ est son conjugu\'e).
On en tire gr\^ace \`a la formule du bin\^ome, et en s\'eparant partie r\'eelle 
et partie imaginaire :
$S_1 = 2^{n/2}\cos n{\pi \over4}$ et 
et $S_2 = 2^{n/2}\sin n{\pi \over4}$. 
On a aussi 
$S_1={A+B\over 2}$ et $S_2={B-A\over 2}i$.
}
}
