\uuid{7728}
\auteur{mourougane}
\datecreate{2021-08-11}

\contenu{
\texte{

}
\begin{enumerate}
    \item \question{Soit $E$ un espace muni d'une forme alternée non-dégénérée $f$.
Soit $G$ le groupe (dit symplectique) des isométries de $(E,f)$.
Combien y a-t-il d'orbites dans l'action du groupe $G$ sur l'ensemble $P(E)$ des droites de $E$ ?}
\reponse{La restriction d'une forme alternée sur une droite est nulle.
 Les droites sont donc toutes isométriques. Par le théorème de Witt, toute isométrie entre deux droites se prolonge à l'espace $E$ non dégénéré en un élément de $G$. Par conséquent, les droites sont toutes dans la même orbite pour l'action de $G$.}
    \item \question{Soit $E$ un espace muni d'une forme alternée non-dégénérée $f$. Quelles sont les restrictions possibles à équivalence près de $f$ sur les plans de $E$ ?}
\reponse{Les formes alternées sont classifiées à équivalence près par leur rang qui est toujours pair. Par conséquent, les restrictions à un plan sont donc la forme nulle ou une forme symplectique de rang $2$.}
    \item \question{Combien y a-t-il d'orbites dans l'action du groupe symplectique d'un espace vectoriel $E$ de dimension~$6$ muni d'une forme alternée non-dégénérée sur l'ensemble des plans de $E$ ?}
\reponse{Par la question précédente et par le théorème de Witt, 
 on conclut comme dans la première question qu'il y a deux orbites sur l'ensemble des plans de $E$.}
\end{enumerate}
}
