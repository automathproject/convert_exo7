\uuid{7377}
\auteur{mourougane}
\datecreate{2021-08-10}
\isIndication{false}
\isCorrection{true}
\chapitre{Groupe, anneau, corps}
\sousChapitre{Autre}

\contenu{
\texte{

}
\begin{enumerate}
    \item \question{Factoriser $221$ en produit de nombres premiers.}
\reponse{$221=13\times17$}
    \item \question{Déterminer l'ensemble des solutions entières de l'équation
$$(X-3)(X-5)=0\pmod{221}.$$}
\reponse{Par le théorème chinois,
\begin{eqnarray*}
 (X-3)(X-5)=0\pmod{221}&\iff& \left\{\begin{array}{c}(X-3)(X-5)=0\pmod{13} \\ (X-3)(X-5)=0\pmod{17}.\end{array}\right.
\end{eqnarray*}
Puisque $13$ et $17$ sont premiers, par le lemme d'Euclide,
\begin{eqnarray*}
\lefteqn{\left\{\begin{array}{c}(X-3)(X-5)=0\pmod{13} \\ (X-3)(X-5)=0\pmod{17}.\end{array}\right.}\\
&\iff&\left\{\begin{array}{c} (X=3\pmod{13}\quad\text{ou}\quad X=5\pmod{13})\\
\quad \text{ et } \quad (X=3\pmod{17}\quad\text{ou}\quad X=5\pmod{17})\end{array}\right.\\
&\iff& \left\{\begin{array}{ccc}
    X=3\pmod{13}&\quad \text{ et } \quad &X=3\pmod{17}\\ &\quad\text{ou}\quad&\\
 X=3\pmod{13}&\quad \text{ et } \quad &X=5\pmod{17}\\ &\quad\text{ou}\quad&\\
 X=5\pmod{13}&\quad \text{ et } \quad &X=3\pmod{17}\\ &\quad\text{ou}\quad&\\
 X=5\pmod{13}&\quad \text{ et } \quad &X=5\pmod{17}\\ 
    \end{array}\right.\\
&\iff&X=3\pmod{221}\quad\text{ou}\quad X=107\pmod{221}\\
&&\quad\text{ou}\quad X=122\pmod{221}\quad\text{ou}\quad X=5\pmod{221}.
\end{eqnarray*}
Les solutions entières de l'équation
$(X-3)(X-5)=0\pmod{221}$ sont les entiers congrus à $3$, $107$, $122$ ou $5$ modulo $221$.}
\end{enumerate}
}
