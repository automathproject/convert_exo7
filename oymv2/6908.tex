\uuid{6908}
\auteur{ruette}
\datecreate{2013-01-24}

\contenu{
\texte{
Dans une dictature militaire, le dictateur veut augmenter le nombre de
naissances de garçons. Il impose la règle suivante : 
si une femme donne naissance à une fille, elle doit continuer à faire
des enfants ; si elle donne naissance à un garçon, elle doit arrêter
d'avoir des enfants. On suppose que chaque femme a au moins un enfant et 
pas plus de 5 enfants.
}
\begin{enumerate}
    \item \question{Soit $X$ le nombre de filles par femme. Quelle est la loi de $X$ ?}
\reponse{$P(X=k)=0,5^k 0,5= 0,5^{k+1}$ si $0\le k\le 4$, $P(X=5)=0,5^5$.}
    \item \question{Quel est le nombre moyen de filles d'une femme ? Le nombre moyen 
de garçons ?
Cette règle est-elle efficace pour augmenter le nombre de garçons ?}
\reponse{$E(X)=\sum_{0\le k\le 5} kP(X=k)= \frac 14+ 2\frac 18+3\frac 1{16}+4\frac{1}{32}+5\frac{1}{32}=0,96875$.

$Y=$ nombre de garçons. Il y a exactement 1 garçon sauf s'il y a 5 filles.
$P(Y=0)=0,5^5=1/32$. $E(Y)=P(Y=1)\times 1=0,96875$.  $E(X)=E(Y)$, donc pas
efficace.}
\end{enumerate}
}
