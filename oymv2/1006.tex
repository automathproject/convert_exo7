\uuid{1006}
\auteur{legall}
\datecreate{1998-09-01}
\isIndication{true}
\isCorrection{true}
\chapitre{Espace vectoriel}
\sousChapitre{Base}

\contenu{
\texte{

}
\begin{enumerate}
    \item \question{Montrer que les vecteurs  $v_1 =(1,-1,i)$, $v_2=(-1,i,1)$, $v_3 =(i,1,-1)$ forment une base de $\Cc^3$.}
\reponse{C'est bien une base. Comme nous avons trois vecteurs et nous souhaitons 
montrer qu'ils forment un base d'un espace vectoriel de dimension $3$, 
il suffit de montrer que soit la famille est libre, soit elle est génératrice
(ces conditions sont équivalentes pour $n$ vecteurs dans un espace vectoriel de dimension $n$).

Il est plus simple de montrer que la famille est libre. Soit une combinaison linéaire nulle
$a v_1+b v_2+c v_3=0$  il faut montrer que $a=b=c=0$. Mais attention ici le corps de base est $K=\Cc$
donc $a,b,c$ sont des nombres complexes.

\begin{align*}
     &  a v_1+b v_2+c v_3=0 \\
\iff & a (1,-1,i) + b (-1,i,1) + c (i,1,-1) = (0,0,0) \\
\iff & (a-b+ic,-a+ib+c,ia+b-c)=(0,0,0) \\
\iff & \begin{cases}
        a-b+ic  = 0 \\
        -a+ib+c = 0  \\
        ia+b-c = 0 \\
       \end{cases} \\
\iff & \cdots \text{ on résout le système } \\
\iff & a=0, b=0, c=0 \\
\end{align*}

La famille $(v_1,v_2,v_3)$ est libre, donc aussi génératrice ; c'est donc une base de $\Cc^3$.}
    \item \question{Calculer les coordonnées de $v = (1+i,1-i,i)$ dans cette base.}
\reponse{On cherche $a,b,c \in \Cc$ tels que $a v_1+b v_2+c v_3=v$.
Il s'agit donc de r\'esoudre le syst\`eme :
$$ \begin{cases} 
a  -b +ic = 1+i \\ 
-a + ib  +c = 1-i \\ 
ia+b-c = i \\
\end{cases}$$
On trouve $a=0$, $b=\frac12(1-i)$, $c=\frac12(1-3i)$.
Nous avons donc $v = \frac12(1-i) v_2 + \frac12(1-3i) v_3$ et ainsi
 les coordonn\'ees de $v$ dans la base $(v_1,v_2,v_3)$ sont
$(0,\frac12(1-i),\frac12(1-3i))$.}
\indication{Il n'y a aucune difficulté. C'est comme dans $\Rr^3$ sauf qu'ici les coefficients sont des nombres complexes.}
\end{enumerate}
}
