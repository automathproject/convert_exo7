\uuid{Qsth}
\exo7id{2275}
\auteur{barraud}
\datecreate{2008-04-24}
\isIndication{false}
\isCorrection{true}
\chapitre{Polynôme}
\sousChapitre{Polynôme}

\contenu{
\texte{
Montrer que $f$ est irr\'eductible dans 
$\Qq[x]$ :
}
\begin{enumerate}
    \item \question{$f=x^4-8x^3+12x^2-6x+2$;}
\reponse{$P$ est primitif, $2$ divise tous les coefficients de $P$ sauf le
    dominant, et $4$ ne divise pas le terme constant~: d'après le critère
    d'Eisenstein, on en déduit que $P$ est irréductible dans $\Zz[x]$
    (puis dans $\Qq[x]$ car il est unitaire...).}
    \item \question{$f=x^5-12x^3+36x-12$;}
\reponse{On peut appliquer le même critère, avec $3$ cette fois.}
    \item \question{$f=x^4-x^3+2x+1$;}
\reponse{$f$ est primitif, et sa réduction modulo $2$ est irréductible. Donc
    $f$ est irréductible dans $\Zz[x]$.}
    \item \question{$f=x^{p-1}+\dots+x+1$, o\`u $p$ est premier.}
\reponse{$f(x+1)=\sum_{k=1}^{p}C_{p}^{k}x^{k-1}$. Or $p|\frac{p!}{k!(p-k)!}$
    (car $p$ apparaît au numérateur, tandis que tous les facteurs du
    dénominateur sont $<p$~; comme $p$ est premier, ils sont donc
    premiers avec $p$). De plus $C_{p}^{1}=p$, donc $p^{2}$ ne divise pas
    le terme constant de $f(x+1)$. D'après le critère d'Eisenstein,
    $f(x+1)$ est irréductible, et donc $f$ aussi.}
\end{enumerate}
}
