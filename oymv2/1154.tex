\uuid{2pLt}
\exo7id{1154}
\auteur{legall}
\datecreate{1998-09-01}
\isIndication{false}
\isCorrection{false}
\chapitre{Déterminant, système linéaire}
\sousChapitre{Autre}

\contenu{
\texte{
Une matrice carr\'ee $ A=(a_{ij})_{i,j\in \{ 1,\ldots, n
\} }\in M_n({\Rr}) $ est dite triangulaire
 sup\'erieure lorsque pour tout $ i>j $~: $ a_{ij}=0 .$
}
\begin{enumerate}
    \item \question{Montrer que le produit de deux matrices triangulaires sup\'erieures est
une matrice triangulaire sup\'erieure.}
    \item \question{D\'emontrer que $ \hbox{det}(A)=a_{11}\cdots a_{nn} .$}
    \item \question{Soit $ E $ un espace vectoriel, $ \epsilon =\{ e_1 , \ldots, e_n\}  $
une base de $ E $ et $ \varphi \in \mathcal{L} (E) .$ On note $ E_i $
l'espace vectoriel engendr\'e par $ \{ e_1, \ldots, e_i\}  $, pour tout $ 1\leq i\leq
n .$ Montrer que $ \hbox{Mat} (\varphi , \epsilon ) $ est triangulaire sup\'erieure
si et seulement si $ \varphi (E_i )\subset E_i  $ pour tout $ 1\leq i\leq n .$}
    \item \question{D\'emontrer que l'inverse d'une matrice triangulaire sup\'erieure est
une matrice triangulaire sup\'erieure.}
\end{enumerate}
}
