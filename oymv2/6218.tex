\uuid{UnJu}
\exo7id{6218}
\auteur{queffelec}
\datecreate{2011-10-16}
\isIndication{false}
\isCorrection{false}
\chapitre{Espace métrique complet, espace de Banach}
\sousChapitre{Espace métrique complet, espace de Banach}

\contenu{
\texte{
Soit $(X,d)$ un espace métrique, et $(x_n)$ une suite de Cauchy dans $X$.
Vérifier :
}
\begin{enumerate}
    \item \question{La suite $(x_n)$ est bornée même si la métrique est non
bornée, mais il existe des suites bornées dont aucune sous-suite n'est de Cauchy.}
    \item \question{Si $(x_n)$ contient une sous-suite convergente, elle est convergente.}
    \item \question{Soit $(\epsilon_k)$ une suite quelconque de réels $>0$; il existe une
sous-suite $(x_{n_k})$ de $(x_n)$ telle que $d(x_{n_k},x_{n_{k+1}})\leq
\epsilon_k.$}
    \item \question{Soit $(y_n)$ une suite quelconque de $X$. Si $\sum_1^\infty
d(y_n,y_{n+1})<\infty$, la suite $(y_n)$ est de Cauchy. Réciproque ?}
    \item \question{On suppose cette fois la distance $d$ ultramétrique. Dans ce cas $(y_n)$
est de Cauchy si et seulement si $d(y_n,y_{n+1})$ tend vers $0$.}
\end{enumerate}
}
