\uuid{2553}
\auteur{tahani}
\datecreate{2009-04-01}

\contenu{
\texte{
Calculez $D^2f(x)$ dans les
cas suivants:
}
\begin{enumerate}
    \item \question{$f\in L(E,G)$ continue}
\reponse{Calculons l'accroissement:
$$f(x+h)-f(x)=f(x)+f(h)-f(x)=f(h)+0.$$
Or, par d\'efinition $f(h)$ est lin\'eaire en $h$, continue et
$0=o(||h||)$. Par cons\'equent $f$ est diff\'erentiable et
$$Df(x)=f, \mbox{ ou encore } Df(x).h=f(h).$$
On remarque que $Df$ est l'application constante que à $x\in E$
associe l'application lin\'eaire $f$. Par cons\'equent, $Df$ est
diff\'erentiable et sa diff\'erentielle est nulle: $$D^2f=0.$$}
    \item \question{$f: E \times F \rightarrow G$, bilinéaire continue.}
\reponse{Calculons $$f((x,y)+(h,k))-f(x,y)=f(x+h,y+k)-f(x,y)=
f(x,y+k)+f(h,y+k)-f(x,y)=$$
$$f(x,y)+f(x,k)+f(h,y)+f(h,k)-f(x,y)=f(x,k)+f(h,y)+f(h,k).$$
L'application qui \`a $(x,y)$ associe l'application lin\'eaire
$Df(x,y)(h,k)=f(x,k)+f(h,y)$ est donc candidate pour être la
diff\'erentielle de $f$. V\'erifions qu'elle est bien continue et
que $f(h,k)=o(||(h,k)||)$. Nous rappelons qu'une application
bilin\'eaire $f(x,y)$ est continue s'il existe $M>0$ tel que
$$\forall (x,y) \in E^2; ||f(x,y)||\leq M||x||||y||.$$
On a $$||Df(x,y)(h,k)||=||f(x,k)+f(h,y)|| \leq
||f(x,k)||+||f(h,y)||\leq$$ $$M||x||.||k||+M||h||.||y|| \leq M
(||x||+||y||)\max(||k||,||h||)\leq M(||x||+||y||)||(h,k)||.$$ Par
cons\'equent, $Df(x,y)$ est continue et a une norme inf\'erieure
\`a $M(||x||+||y||)$. De plus
$$||f(h,k)||\leq M||h||.||k|| \leq ||(h,k)||.\epsilon(h,k)$$ o\`u
$\epsilon$ tend vers zero quand $(h,k)$ tend vers zero car
$$\epsilon(h,k)=\frac{||h||.||k||}{\sup(||h||,||k||)}$$ ce qui
fini de montrer que $f$ est diff\'erentiable et que sa
diff\'erentielle est d\'efinie par
$$Df(x,y).(h,k)=f(x,k)+f(h,y).$$
En remarquant que $Df$ est lin\'eaire par rapport \`a $(x,y)$,
d'apr\`es la premi\`ere question, on d\'eduit que sa
diff\'erentielle est $$D^2f(x,y)[(h,k),(u,v)]=f(u,k)+f(h,v).$$}
    \item \question{$f:M_n(\mathbb{R}) \rightarrow M_n(\mathbb{R})$, $f(A)=A^2$}
\reponse{$$f(A+h)-f(A)=(A+h)^2-A^2=Ah+hA+h^2$$ avec $Ah+hA$
lin\'eaire en $h$ (et en $A$) et $||h^2|| \leq ||h||^2=o(||h||)$.
Par cons\'equent $f$ est diff\'erentiable et sa diff\'erentielle
est $Df(A).h=Ah+hA$. Comme $Df(A)$ est lin\'aire par rapport \`a
$A$, sa diff\'erentielle en $A$ est l'application bilin\'eaire
$$D^2f(A)[H,K]=KH+HK.$$}
\end{enumerate}
}
