\uuid{3390}
\auteur{quercia}
\datecreate{2010-03-09}
\isIndication{false}
\isCorrection{true}
\chapitre{Matrice}
\sousChapitre{Autre}

\contenu{
\texte{
Soit $E = \{ \text{matrices de } \mathcal{M}_n(\R) \text{ antisymétriques}\}$ et
$f : E \to  E,  M \mapsto {^t\!AM+MA}$ où $A \in \mathcal{M}_n(\R)$.
}
\begin{enumerate}
    \item \question{Montrer que $f$ est un endomorphisme.}
    \item \question{Quelle est la trace de $f$ ?}
\reponse{
La base canonique de $E$ est $(F_{ij} = E_{ij}-E_{ji})_{1\le i<j \le n}$ où $(E_{ij})$
est la base canonique de $\mathcal{M}_n(\R)$ :
Si $M \in E$, la coordonnée de $M$ suivant $F_{ij}$ est le coefficient d'indices
$i,j$ de $M$. En particulier, en notant $A = (a_{ij})$, la coordonnée de
$f(F_{ij})$ suivant $F_{ij}$ est $a_{ii}+a_{jj}$, donc :
$$\mathrm{tr} f = \sum_{i,j} (a_{ii} + a_{jj}) = (n-1)\mathrm{tr} A.$$
}
\end{enumerate}
}
