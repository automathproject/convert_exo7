\uuid{249}
\auteur{bodin}
\datecreate{1998-09-01}
\isIndication{true}
\isCorrection{true}
\chapitre{Arithmétique dans Z}
\sousChapitre{Divisibilité, division euclidienne}

\contenu{
\texte{
Combien $15!$ admet-il de diviseurs ?
}
\indication{Il ne faut surtout pas chercher \`a calculer $15!=1\times2\times3\times4\times\cdots\times15$, mais profiter du fait
qu'il est d\'ej\`a ``presque'' factoris\'e.}
\reponse{
\'Ecrivons la d\'ecomposition de  $15 !=1.2.3.4\ldots15$ en facteurs premiers. $15 !  = 2^{11}.3^6.5^3.7^2 .11.13$.
Un diviseur de $15 !$ s'\'ecrit $d = 2^{\alpha}.3^\beta.5^\gamma.7^\delta .11^\epsilon.13^\eta$
avec $0 \leq \alpha \leq 11$, $0 \leq \beta \leq 6$, $0 \leq \gamma \leq 3$, $0 \leq \delta \leq 2$,
$0 \leq \epsilon \leq 1$, $0 \leq \eta \leq 1$. De plus tout nombre $d$ de cette forme est un diviseur de $15 !$.
Le nombre de diviseurs est donc $(11+1)(6+1)(3+1)(2+1)(1+1)(1+1) = 4032$.
}
}
