\uuid{2626}
\auteur{debievre}
\datecreate{2009-05-19}

\contenu{
\texte{
Soit $f\colon\R^3 \rightarrow \R$ une
fonction de classe $C^1$
et soit  $g\colon \R^3 \rightarrow \R$
la fonction d\'efinie par
\[
g(x,y,z) = f(x-y,y-z,z-x). 
\]
Montrer que
\begin{equation}
\frac{\partial g}{\partial x}  + \frac{\partial g}{\partial y}  + \frac{\partial g}{\partial z} = 0.
\label{ex3}
\end{equation}
}
\indication{Calculer m\`ene \`a la v\'erit\'e.}
\reponse{
\begin{align*}
\frac{\partial g}{\partial x}
&=
\frac{\partial f}{\partial x}-\frac{\partial f}{\partial z}
\\
\frac{\partial g}{\partial y}
&=
\frac{\partial f}{\partial y}-\frac{\partial f}{\partial x}
\\
\frac{\partial g}{\partial z}
&=
\frac{\partial f}{\partial z}-\frac{\partial f}{\partial y}
\end{align*}
d'o\`u \eqref{ex3}.
}
}
