\uuid{coIi}
\exo7id{3835}
\auteur{quercia}
\datecreate{2010-03-11}
\isIndication{false}
\isCorrection{true}
\chapitre{Espace euclidien, espace normé}
\sousChapitre{Espaces vectoriels hermitiens}

\contenu{
\texte{

}
\begin{enumerate}
    \item \question{Montrer qu'il existe des polynômes $P_0,\dots,P_n \in \R_n[X]$ tels que :
$\forall\ i,j\le n,\  \int_{t=0}^{+\infty} e^{-t}t^iP_j(t)\,d t = \delta_{ij}$.}
    \item \question{Montrer qu'il n'existe pas de suite de polynômes $(P_0,\dots,P_n,\dots)$ telle que :
$\forall\ i,j\in\N,\  \int_{t=0}^{+\infty} e^{-t}t^iP_j(t)\,d t = \delta_{ij}$.}
\reponse{
Soit $P_0 = Q_0'$. Par IPP on obtient $Q_0$ est orthogonal à la
famille $(jX^{j-1}-X^j)_{j\ge 1}$ qui est une base de $\R[X]$ donc $Q_0 = 0 = P_0$
et $ \int_{t=0}^{+\infty} e^{-t}t^0P_0(t)\,d t \ne \delta_{0,0}$.
}
\end{enumerate}
}
