\uuid{4096}
\auteur{quercia}
\datecreate{2010-03-11}

\contenu{
\texte{
Soient $r$ et $q$ deux fonctions continues définies sur~$I={[a,b]}$ telles que~:
$\forall\ x\in I,\ r(x)\ge q(x)$.
On considère les équations différentielles~:
$$(E_1)\Leftrightarrow y'' + qy = 0,
  \qquad (E_2)\Leftrightarrow z'' + rz = 0.$$
}
\begin{enumerate}
    \item \question{Soit~$y$ une solution de~$(E_1)$ et $x_0,x_1$ deux zéros consécutifs
    de~$y$. $y'(x_0)$ et $y'(x_1)$ peuvent-ils être nuls~? Que dire de leurs signes~?}
\reponse{On suppose $y\ne 0$ sinon $y$ n'a pas de zéros consécutifs.
    Comme $y(x_0) = 0$, on a $y'(x_0)\ne 0$ sinon $y=0$. Ceci implique que chaque
    zéro de~$y$ est isolé, donc la notion de zéros consécutifs est pertinente.
    Enfin, $y'(x_0)$ et $y'(x_1)$ sont de signes opposés sinon il existe un
    autre zéro dans~$]x_0,x_1[$.}
    \item \question{Soit~$z$ une solution de~$(E_2)$. On considère
    $W(x) = \begin{vmatrix}y(x)&z(x)\cr y'(x)&z'(x)\cr\end{vmatrix}$. Calculer $W'(x)$
    et $W(x_1)-W(x_0)$.}
\reponse{$W' = (q-r)yz$. $W(x_1)-W(x_0) = y'(x_0)z(x_0) - y'(x_1)z(x_1)$ (non simplifiable).}
    \item \question{Montrer que $z$ a un zéro dans~$]x_0,x_1[$ ou $z(x_0)=z(x_1)=0$.}
\reponse{Si $z$ ne s'annule pas dans $]x_0,x_1[$ alors $W'$ est de signe constant
    sur cet intervalle. L'examen des différents cas possibles de signe apporte une
    contradiction entre les signes de~$W'$ et de~$W(x_1)-W(x_0)$ si $z(x_0)\ne 0$
    ou $z(x_1) \ne 0$.}
    \item \question{Soit~$u$ une solution de~$(E_1)$. Montrer que $u$ est soit proportionnelle
    à~$y$, soit admet un unique zéro dans~$]x_0,x_1[$.}
\reponse{On prend $r=q$, $z=u$. Si $u(x_0)\ne 0$
    alors $u$ admet un zéro dans~$]x_0,x_1[$ et en permutant les rôles de~$u$ et~$y$,
    le prochain zéro éventuel de~$u$ vient après $y_1$.
    Sinon, $u = \frac{u'(x_0)}{y'(x_0)}y$.}
\end{enumerate}
}
