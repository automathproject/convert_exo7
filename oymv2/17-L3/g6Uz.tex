\uuid{g6Uz}
\exo7id{2215}
\auteur{matos}
\datecreate{2008-04-23}
\isIndication{false}
\isCorrection{false}
\chapitre{Autre}
\sousChapitre{Autre}

\contenu{
\texte{
Soient
\begin{itemize}
\item $H=$span$\{v_1, \cdots , v_r\}$ le sous--espace vectoriel de $\Rr^n$ engendr\'e par les vecteurs $\{v_i\}$ suppos\'es ind\'ependants;
\item $V=\left[ \begin{array}{cccc}
v_1&v_2 & \cdots & v_r\end{array}\right]$ la matrice de type $n\times r$ dont les colonnes sont les composantes des $v_i$ dans la base canonique $\epsilon = (e_1, \cdots , e_n)$
\end{itemize}
Pour tout $x\in \Rr^n$ on d\'esigne par $y$ sa projection orthogonale sur $H$ et par $X$ et $Y$ les matrices colonnnes des composantes de $x$ et $y$ dans la base $\epsilon$.
On pose
$$y=\sum_{i=1}^r \alpha_i v_i .$$
}
\begin{enumerate}
    \item \question{Montrer que la matrice $G=G(v_1, \cdots ,v_r)=V^TV$ est inversible.}
    \item \question{Montrer que les $\alpha_i$ v\'erifient le syst\`eme
$$G\left( \begin{array}{c}\alpha_1 \\ \vdots \\ \alpha_r\end{array}
\right) = V^TX$$}
    \item \question{En d\'eduire que $Y=VG^{-1}V^TX= PX$ avec $P=VG^{-1}V^T$ ($P$ est donc la matrice de la projection orthogonale de $\Rr^n$ sur $H$}
    \item \question{\emph{Application}: on consid\`ere $n=3, v_1=e_1, v_2=e_1+e_2+e_3$. D\'eterminer la projection orthogonale sur $H=$ span $\{v_1, v_2\}$ de $x=2e_1-e_2+e_3$.}
    \item \question{Quelle est la matrice de la projection orthogonale sur $H=$span $\{v\}$?}
    \item \question{Montrer que, pour $x\in\Rr^n$
$$d^2(x,H)= \frac{\mbox{det} G(x,v_1,\cdots ,v_r)}{\mbox{det} G(v_1, \cdots , v_r)}$$}
\end{enumerate}
}
