\uuid{8fHj}
\exo7id{2216}
\auteur{matos}
\datecreate{2008-04-23}
\isIndication{false}
\isCorrection{true}
\chapitre{Autre}
\sousChapitre{Autre}

\contenu{
\texte{
Soit $A\in\Rr^{m\times n}$ de rang $r\leq p=\min (m,n)$. On consid\`ere la d\'ecomposition en valeurs singuli\`eres de $A$
 $$U^TAV = \mbox{diag} (\sigma_1, \cdots , \sigma_p )$$
o\`u les $\sigma_i$ sont les valeurs singuli\`eres de $A$
}
\begin{enumerate}
    \item \question{Montrer que Im$(A)=$ span$\{u^1, u^2, \cdots , u^r\}$ et 
 Ker$(A)=$ span$\{v^{r+1}, \cdots , v^n\}$.}
    \item \question{Montrer que Im$(^TA)=$ span$\{v^1, u^2, \cdots , v^r\}$ et 
 Ker$(A^T)=$ span$\{u^{r+1}, \cdots , u^m\}$.}
    \item \question{D\'eterminer les matrices des projections orthogonales sur Im$(A)$, Ker$(A)$, Im$(A^T)$, Ker$(A^T)$ \`a l'aide de $U$ et $V$.}
    \item \question{\emph{Application:} calculer la d\'ecomposition en valeurs singuli\`eres de la matrice
$$A=\left(\begin{array}{cc}1&1\\
2&1\\
-1&1\end{array}\right)$$
et les matrices correspondantes aux projections orthogonales de l'exercice pr\'ec\'edent.}
\reponse{
4.On calcule
$$A^TA=\left(\begin{array}{cc}6&2\\2& 3\end{array}\right)$$ et ses valeurs propres
$$\mbox{det} \left(\begin{array}{cc}6-\lambda &2\\2& 3-\lambda\end{array}\right)=0\Leftrightarrow \lambda_1=7=\mu_1^2, \lambda_2=2=\mu_2^2$$
On calcule ensuite les vecteurs propres associ\'es \`a ces valeurs propres
$$\left(\begin{array}{cc}6&2\\2& 3\end{array}\right)\left(\begin{array}{c}x_1\\x_2\end{array}\right) =\lambda \left(\begin{array}{c}x_1\\x_2\end{array}\right)$$
et la matrice $V$ est la matrice dont les colonnes sont
$$v_1=(2/\sqrt{5}, 1/\sqrt{5})^T,\quad v_2=(1/\sqrt{5}, -2/\sqrt{5})^T$$
les colonnes de $U$ sont alors donn\'ees par
$$u_1=Av_1/\mu_1=1/(\sqrt{7}\sqrt{5})(3,5,-1)^T ,\quad u_2=Av_2/\mu_2=1/(\sqrt{2}\sqrt{5})(-1,0,-3)^T$$
quant \`a $u_3$ il est choisi orthogonal \`a $u_1$ et $u_2$ et de norme 1.
}
\end{enumerate}
}
