\uuid{G9B5}
\exo7id{1163}
\auteur{barraud}
\datecreate{2003-09-01}
\isIndication{false}
\isCorrection{true}
\chapitre{Déterminant, système linéaire}
\sousChapitre{Système linéaire, rang}

\contenu{
\texte{
Résoudre les systèmes suivants
  $$
  \left\{
    \begin{array}{rcrcrcl}
         x &+&    y &-&    z &=&  0  \\
         x &-&    y & &      &=&  0  \\
         x &+&    4y &+&   z &=&  0  
    \end{array}
  \right.  
 \qquad
\left\{
    \begin{array}{rcrcrcl}
       x &+&  y &+& 2z &=& 5  \\
       x &-&  y &-&  z &=& 1  \\
       x & &    &+&  z &=& 3    
    \end{array}
  \right. 
\qquad 
  \left\{
    \begin{array}{rcrcrcl}
      3x &-& y  &+& 2z &=& a \\
      -x &+& 2y &-& 3z &=& b \\
       x &+& 2y &+&  z &=& c 
    \end{array}
  \right.
  $$
}
\reponse{
Remarquons que comme le système est homogène (c'est-à-dire les coefficients du second membre sont
nuls) alors $(0,0,0)$ est une solution du système. Voyons s'il y en a d'autres. 
Nous faisons semblant de ne pas voir que la seconde ligne implique $x=y$ et 
que le système est en fait très simple à résoudre. 
Nous allons appliquer le pivot de Gauss en faisant les opérations suivantes sur les lignes 
$L_2 \leftarrow L_2-L_1$ et $L_3 \leftarrow L_3-L_1$ :
$$  \left\{
    \begin{array}{rcrcrcl}
         x &+&    y &-&    z &=&  0  \\
         x &-&    y & &      &=&  0  \\
         x &+&    4y &+&    z &=&  0  
    \end{array}
  \right.
\iff\left\{
    \begin{array}{rcrcrcl}
         x &+&    y &-&    z &=&  0  \\
           &-&   2y &+&    z &=&  0  \\
           & &   3y &+&    2z &=&  0  
    \end{array}
  \right.
$$
On fait maintenant $L_3 \leftarrow 2L_3+3L_2$ pour obtenir :
$$\left\{
    \begin{array}{rcrcrcl}
         x &+&    y &-&    z &=&  0  \\
           &-&   2y &+&    z &=&  0  \\
           & &    &&       7z &=&  0  
    \end{array}
  \right.
$$

En partant de la dernière ligne on trouve $z=0$, puis en remontant $y=0$, puis $x=0$.
Conclusion l'unique solution de ce système est $(0,0,0)$.
On applique le pivot de Gauss $L_2 \leftarrow L_2-L_1$ et $L_3 \leftarrow L_3-L_1$ :
$$ \left\{
    \begin{array}{rcrcrcl}
       x &+&  y &+& 2z &=& 5  \\
       x &-&  y &-&  z &=& 1  \\
       x & &    &+&  z &=& 3    
    \end{array}
  \right. 
\iff
 \left\{
    \begin{array}{rcrcrcl}
       x &+&  y &+&  2z &=& 5  \\
        &-&  2y &-&  3z &=& -4  \\
        &-&  y  &-&  z  &=& -2    
    \end{array}
  \right. 
$$
Puis $L_3\leftarrow 2L_3-L_2$ pour obtenir un système équivalent qui est triangulaire donc facile à résoudre :
$$ \left\{
    \begin{array}{rcrcrcl}
       x &+&  y &+&  2z &=& 5  \\
        &-&  2y &-&  3z &=& -4  \\
        &&    &&  z  &=& 0    
    \end{array}
  \right. 
\iff \left\{
    \begin{array}{rcl}
        x &=& 3 \\
        y &=& 2 \\
        z &=& 0    
    \end{array} \right.
$$

On n'oublie pas de vérifier que c'est une solution du système initial.
On fait les opérations $L_2\leftarrow 3L_2+L_1$ et $L_3\leftarrow 3L_3-L_1$ pour obtenir :
$$\left\{
    \begin{array}{rcrcrcl}
      3x &-& y  &+& 2z &=& a \\
      -x &+& 2y &-& 3z &=& b \\
       x &+& 2y &+&  z &=& c 
    \end{array}
  \right.
\iff
\left\{
    \begin{array}{rcrcrcl}
      3x &-& y  &+& 2z &=& a \\
         & & 5y &-& 7z &=& 3b+a \\
         & & 7y &+&  z &=& 3c-a 
    \end{array}
  \right.
  $$
Puis on fait $L_3\leftarrow 5L_3-7L_2$, ce qui donne un système triangulaire :
$$\left\{
    \begin{array}{rcrcrcl}
      3x &-& y  &+& 2z &=& a \\
         & & 5y &-& 7z &=& 3b+a \\
         & &    & & 54z &=& 5(3c-a)-7(3b+a) 
    \end{array}
  \right.
  $$
En partant de la fin on en déduit : $z= \frac{1}{54}(-12a-21b+15c)$ puis en remontant cela donne
$$\left\{
    \begin{array}{rcl}
         x &=& \frac{1}{18}(8a+5b-c) \\
         y &=& \frac{1}{18}(-2a+b+7c) \\
         z &=& \frac{1}{18}(-4a-7b+5c)
    \end{array}
  \right.
  $$
}
}
