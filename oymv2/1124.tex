\uuid{1124}
\auteur{liousse}
\datecreate{2003-10-01}
\isIndication{false}
\isCorrection{false}
\chapitre{Déterminant, système linéaire}
\sousChapitre{Calcul de déterminants}

\contenu{
\texte{
On note $a,$ $b,$ $c$ des r\'eels. 
Calculer les d\'eterminants suivants.
$$D_1=\left\vert\begin{array}{cccc}1&0&0&1\\0&1&0&0\\1&0&1&1\\
2&3&1&1\end{array}\right\vert\hbox{, }
D_2=\left\vert\begin{array}{cccc}a+b+c&b&b&b\\c&a+b+c&b&b\\c&c&a+b+c&b\\
c&c&c&a+b+c\end{array}\right\vert\hbox{, }
D_3=\left\vert\begin{array}{ccccc}1&0&3&0&0\\0&1&0&3&0\\a&0&a&0&3\\
b&a&0&a&0\\0&b&0&0&a\end{array}\right\vert$$
G\'en\'eraliser le calcul de $D_2$ \`a un d\'eterminant $n\times n$ 
du m\^eme type.
}
}
