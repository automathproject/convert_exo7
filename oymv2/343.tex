\uuid{OUwa}
\exo7id{343}
\auteur{cousquer}
\datecreate{2003-10-01}
\isIndication{false}
\isCorrection{false}
\chapitre{Arithmétique dans Z}
\sousChapitre{Nombres premiers, nombres premiers entre eux}

\contenu{
\texte{

}
\begin{enumerate}
    \item \question{Soit $p \in \mathbb{Z}$ un nombre premier. 
Montrer que si $a\in \mathbb{Z}$ n'est pas congru à $0$ modulo $p$
 alors $p $ ne divise pas $a$ et donc $\mbox{pgcd}(a,p)=1$.}
    \item \question{Soit $a\in \mathbb{Z}$ non congru à $0$ modulo $p$ avec $p$ premier. 
 Montrer en utilisant
 le a) qu'il existe $u \in \mathbb{Z}$ non congru à $0$ modulo $p$ vérifiant
 $au \equiv 1[p]$. (Remarquer que cela donne un inverse de $a$ modulo $p$).}
    \item \question{Montrer que si $p$ n'est pas premier, 
il existe des éléments $a,u\in \mathbb{Z}$ non nuls modulo $p$
tels que $au\equiv 0[p]$.}
\end{enumerate}
}
