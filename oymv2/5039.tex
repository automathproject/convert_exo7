\uuid{SjMT}
\exo7id{5039}
\auteur{quercia}
\datecreate{2010-03-17}
\isIndication{false}
\isCorrection{true}
\chapitre{Courbes planes}
\sousChapitre{Propriétés métriques : longueur, courbure,...}

\contenu{
\texte{
Soit $\mathcal{C}$ le cercle d'équation $x^2+y^2-2Rx = 0$ et $\Delta$ une droite
variable passant par $O$.
}
\begin{enumerate}
    \item \question{Chercher l'équation de la parabole $\cal P$ d'axe parallèle à $\Delta$,
    passant par $O$, dont $\mathcal{C}$ est le cercle osculateur en~$O$.}
\reponse{Soit $\theta$ l'angle polaire de $D$. Dans le repère
    $(O,\vec u_\theta, \vec v_\theta)$, $\cal P$ a pour équation :
    $Y = aX^2 + bX$.

    On veut que $\cal P$ soit tangente à $Oy$, soit $b = -\tan\theta$ et que
    le rayon de courbure soit $R$, soit $a = \frac1{2R\cos^3\theta}$.

    \'Equation dans $0xy$ : $x^2\sin^2\theta - 2xy\cos\theta\sin\theta
                             + y^2\cos^2\theta - 2Rx\cos^2\theta = 0$.}
    \item \question{Quelle est l'enveloppe des paraboles précédentes ?}
\reponse{$O$.}
\end{enumerate}
}
