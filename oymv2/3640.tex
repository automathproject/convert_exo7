\uuid{3640}
\auteur{quercia}
\datecreate{2010-03-10}
\isIndication{false}
\isCorrection{true}
\chapitre{Endomorphisme particulier}
\sousChapitre{Autre}

\contenu{
\texte{
On note $f_n(x) = \cos nx$ et $g_n(x) = \sin nx$ ($x \in \R$, $n \in \N$).

Soit $E_n$ l'espace engendré par la famille
${\cal F}_n = (f_0,\dots,f_n,g_1,\dots,g_n)$.
}
\begin{enumerate}
    \item \question{Montrer que pour $k \ge 1$, $(f_k,g_k)$ est libre.}
    \item \question{Soit $\varphi : {E_n} \to {E_n}, f \mapsto {f''}$.
    Chercher les sous-espaces propres de $\varphi$.
    En déduire que ${\cal F}_n$ est libre.}
    \item \question{On note $a_k = \frac{2k\pi}{2n+1}$ et
    ${\varphi_k} : {E_n} \to \R, f \mapsto {f(a_k).}$

    Montrer que $(\varphi_0,\dots,\varphi_{2n})$ est une base de $E_n^*$.
    (On utilisera la fonction $f : x  \mapsto \prod_{k=1}^n (\cos x - \cos a_k)$)}
    \item \question{Soit $N \in \N^*$. On note $b_k = \frac{2k\pi}N$ et
    ${\psi_k} : {E_n}\to \R, f \mapsto {f(b_k).}$
    Montrer que $(\psi_0,\dots,\psi_{N-1})$ est libre si et seulement si
    $N \le 2n+1$, et engendre $E_n^*$ si et seulement si $N \ge 2n+1$.}
\reponse{
3. Rmq : coefficients de Fourrier :
    $\alpha_p = \frac 2{2n+1}\sum_{k=0}^{2n} f(a_k)\cos(pa_k)$ et
    $\beta_p = \frac 2{2n+1}\sum_{k=0}^{2n} f(a_k)\sin(pa_k)$.
}
\end{enumerate}
}
