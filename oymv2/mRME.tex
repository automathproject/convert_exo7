\uuid{mRME}
\exo7id{5310}
\auteur{rouget}
\datecreate{2010-07-04}
\isIndication{false}
\isCorrection{true}
\chapitre{Arithmétique}
\sousChapitre{Arithmétique de Z}

\contenu{
\texte{

}
\begin{enumerate}
    \item \question{(Formule de \textsc{Legendre}) Soit $n$ un entier naturel supérieur ou égal à $2$ et $p$ un nombre premier. Etablir que l'exposant de $p$ dans la décomposition de $n!$ en facteurs premiers est

$$E(\frac{n}{p})+E(\frac{n}{p^2})+E(\frac{n}{p^3})+...$$}
\reponse{(Formule de \textsc{Legendre}) Soit $n$ un entier naturel supérieur ou égal à $2$.

Si $p$ est un nombre premier qui divise $n!=1.2...n$, alors $p$ est un facteur premier de l'un des entiers $2$,..., $n$ et en particulier, $p\leq n$. Réciproquement, il est clair que si $p$ est un nombre premier tel que $p\leq n$, $p$ divise $n!$. Les facteurs premiers de $n!$ sont donc les nombres premiers inférieurs ou égaux à $n$.

Soit donc $p$ un nombre premier tel que $p\leq n$. Pour trouver l'exposant de $p$ dans la décomposition primaire de $n!$, on compte $1$ pour chaque multiple de $p$ inférieur ou égal à $n$, on rajoute $1$ pour chaque multiple de $p^2$ inférieur ou égal à $n$, on rajoute encore $1$ pour chaque multiple de $p^3$ inférieur ou égal à $n$... et on s'arrête quand l'exposant $k$ vérifie $p^k>n$.

$$n\geq p^k\Leftrightarrow\ln n\geq k\ln p\Leftrightarrow k\leq\frac{\ln n}{\ln p},$$

(car $\ln p>0$). Donc, si $k\geq E(\frac{\ln n}{\ln p})+1$, alors $p^k>n$.

Dit autrement, l'exposant de $p$ est la somme du nombre de multiples de $p$ inférieurs ou égaux à $n$, du nombre de multiples de $p^2$ inférieurs ou égaux à $n$, du nombre de multiple de $p^3$ inférieurs ou égaux à $n$... et du nombre de multiples de $p^{E(\ln n/\ln p)}$.

Soit $k$ un entier tel que $1\leq k\leq E(\frac{\ln n}{\ln p})$ et $K$ un entier naturel.

$$1\leq K.p^k\leq n\Leftrightarrow \frac{1}{p^k}\leq K\leq\frac{n}{p^k}\Leftrightarrow 1\leq K\leq E(\frac{n}{p^k}).$$

Il y a donc $E(\frac{n}{p^k})$ multiples de $p^k$ compris au sens large entre $1$ et $n$. On a montré que l'exposant de $p$ dans la décomposition de $n!$ en facteurs premiers est

$$E(\frac{n}{p})+E(\frac{n}{p^2})+E(\frac{n}{p^3})+...$$}
    \item \question{Par combien de $0$ se termine l'écriture en base $10$ de $1000!$~?}
\reponse{L'exposant de $5$ dans la décomposition primaire de $1000!$ est

$$E(\frac{1000}{5})+E(\frac{1000}{5^2})+E(\frac{1000}{5^3})+E(\frac{1000}{5^4})=200+40+8+1=249.$$

L'exposant de $2$ est évidemment supérieur (il y a déjà au moins $500$ nombres pairs entre $1$ et $1000$). Donc, la plus grande puissance de $10$ divisant $1000!$ est encore la plus grande puissance de $5$ divisant $1000!$, à savoir $249$. L'écriture en base $10$ de $1000!$ se termine par $249$ zéros.}
\end{enumerate}
}
