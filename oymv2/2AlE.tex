\uuid{2AlE}
\exo7id{4368}
\auteur{quercia}
\datecreate{2010-03-12}
\isIndication{false}
\isCorrection{true}
\chapitre{Intégration}
\sousChapitre{Intégrale de Riemann dépendant d'un paramètre}

\contenu{
\texte{
Soit $\alpha>0$.
}
\begin{enumerate}
    \item \question{Montrer que $f : x \mapsto e^{-\alpha x} \int_{\theta=0}^\pi \cos(x\sin\theta)\, d\theta$ est intégrable
    sur $\R^+$.}
    \item \question{Calculer $I =  \int_{x=0}^{+\infty}f(x)\,d x$.
    Indication~: écrire $I = \lim_{a\to+\infty} \int_{x=0}^a f(x)\,d x$.}
\reponse{
Théorème de Fubini~:
   $ \int_{x=0}^{+\infty} f(x)\,d x =  \int_{\theta=0}^\pi \int_{x=0}^{+\infty}\Re(e^{(-\alpha+i\sin\theta)x})\,d x\, d\theta
   =  \int_{\theta=0}^\pi\frac{\alpha\, d\theta}{\alpha^2+\sin^2\theta}=\frac\pi{\sqrt{1+\alpha^2}}$

   (couper en $\theta=\pi/2$ et poser $u=\tan\theta$).
}
\end{enumerate}
}
