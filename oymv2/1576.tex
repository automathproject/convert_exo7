\uuid{1576}
\auteur{legall}
\datecreate{1998-09-01}
\isIndication{false}
\isCorrection{false}
\chapitre{Réduction d'endomorphisme, polynôme annulateur}
\sousChapitre{Polynôme annulateur}

\contenu{
\texte{
Soit $  E  $ un $  \mathbb{K} $-espace vectoriel de dimension finie
$  n  $ et $  f
\in \mathcal{L} (E)  $ tel que $  \hbox{rg}(f-id)=1  .$ On note $  H = \hbox{Ker}(f-id)  .$
}
\begin{enumerate}
    \item \question{Soit $  \{ e_1, \cdots , e_{n-1}\}   $ une base de $  H   $ et $  e_n \notin H  .$ Montrer que $  \{ e_1, \ldots , e_{n}\}   $ est une base de $  E   $ et donner l'allure de la matrice de $  f   $ dans cette base.}
    \item \question{Montrer que le polyn\^ome $  (X-1)(X-\hbox{det}(f))  $ annule $  f  .$ Donner une condition n\'ec\'essaire et
suffisante pour que $  f  $ soit diagonalisable.}
\end{enumerate}
}
