\uuid{Scdk}
\exo7id{5645}
\auteur{rouget}
\datecreate{2010-10-16}
\isIndication{false}
\isCorrection{true}
\chapitre{Déterminant, système linéaire}
\sousChapitre{Calcul de déterminants}

\contenu{
\texte{

}
\begin{enumerate}
    \item \question{Soient $a_{i,j}$, $1\leqslant i,j\leqslant n$, $n^2$ fonctions dérivables sur $\Rr$ à valeurs dans $\Cc$. Soit $d=\text{det}(a_{i,j})_{1\leqslant i,j\leqslant n}$.

Montrer que $d$ est dérivable sur $\Rr$ et calculer $d'$.}
\reponse{$d=\sum_{\sigma\in  S_n}^{}\varepsilon(\sigma)a_{\sigma(1),1}...a_{\sigma(n),n}$ est dérivable sur $\Rr$ en tant que combinaison linéaire de produits de fonctions dérivables sur $\Rr$ et de plus

\begin{align*}\ensuremath
d'&=\sum_{\sigma\in  S_n}^{}\varepsilon(\sigma)(a_{\sigma(1),1}...a_{\sigma(n),n})'=\sum_{\sigma\in  S_n}^{}\varepsilon(\sigma)\sum_{i=1}^{n}a_{\sigma(1),1}...a_{\sigma(i),i}'\ldots a_{\sigma(n),n}=\sum_{i=1}^{n}\sum_{\sigma\in  S_n}^{}\varepsilon(\sigma)a_{\sigma(1),1}...a_{\sigma(i),i}'\ldots a_{\sigma(n),n}\\
   &=\sum_{i=1}^{n}\text{det}(C_1,...,C_i',...,C_n)\;(\text{où}\;C_1,...,C_n\;\text{sont les colonnes de la matrice}).
\end{align*}}
    \item \question{Application : calculer $d_n(x)=\left|
\begin{array}{cccc}
x+1&1&\ldots&1\\
1&\ddots&\ddots&\vdots\\
\vdots&\ddots&\ddots&1\\
1&\ldots&1&x+1
\end{array}
\right|$.}
\reponse{\textbf{1 ère solution.} D'après ce qui précède, la fonction $d_n$ est dérivable sur $\Rr$ et pour $n\geqslant 2$ et $x$ réel, on a

\begin{align*}\ensuremath
d_n'(x)&=\sum_{i=1}^{n}\left|
\begin{array}{ccccccccc}
x+1&1&\ldots&1&0&1&\ldots&\ldots&1\\
1&\ddots&\ddots&\vdots&\vdots&\vdots\\
\vdots&\ddots&\ddots&1&\vdots\\
 & &\ddots&x+1&0&\vdots\\
\vdots& & &1&1&1\\
 & & &\vdots&0&x+1&\ddots\\
 & & & &\vdots&1&\ddots&\ddots&\vdots\\
\vdots& & &\vdots&\vdots&\vdots&\ddots&\ddots&1\\
1&\ldots&\ldots&1&0&1&\ldots&1&x+1
\end{array}
\right|(\text{la colonne particulière est la colonne}\;i)\\
 &=\sum_{i=1}^{n}d_{n-1}(x)(\text{en développant le}\;i\text{-ème déterminant par rapport à sa}\;i\text{-ème colonne})\\
 &=nd_{n-1}(x).
\end{align*}

En résumé, $\forall n\geqslant 2$, $\forall x\in\Rr$, $d_n(x)=nd_{n-1}(x)$. D'autre part $\forall x\in\Rr$, $d_1(x)=x+1$ et $\forall n\geqslant 2$, $d_n(0) = 0$ (déterminant ayant deux colonnes identiques).

Montrons alors par récurrence que 
$\forall n\geqslant 1$, $\forall x\in\Rr$, $d_n(x) =x^n+nx^{n-1}$.

\textbullet~C'est vrai pour $n=1$.

\textbullet~Soit $n\geqslant1$. Supposons que $\forall n\geqslant 1$, $\forall x\in\Rr$, $d_n(x) =x^n+nx^{n-1}$. Alors, pour $x\in\Rr$,

\begin{center}
$d_{n+1}(x)=d_{n+1}(0)+\int_{0}^{x}d_{n+1}'(t)\;dt=(n+1)\int_{0}^{x}d_n(t\;dt)=x^{n+1}+(n+1)x^n$.
\end{center}

On a montré que

\begin{center}
\shadowbox{
$\forall n\geqslant 1$, $\forall x\in\Rr$, $d_n(x)=x^n+nx^{n-1}$.
}
\end{center}

\textbf{2 ème solution.} $d_n$ est clairement un polynôme de degré $n$ unitaire.
Pour $n\geqslant 2$, puisque dn(0) = 0 et que $d_n'= nd_{n-1}$, $0$ est racine de $d_n$, $d_n'$, ..., $d_n^{(n-2)}$ et est donc racine d'ordre $n-1$ au moins de $d_n$. 
Enfin, $d_n(-n)=0$ car la somme des colonnes du déterminant obtenu est nulle.
Finalement $\forall n\geqslant2$, $\forall x\in\Rr$, $d_n(x) = x^{n-1}(x+n)$ ce qui reste vrai pour $n=1$.

Une variante peut être obtenue avec des connaissances sur la réduction.}
\end{enumerate}
}
