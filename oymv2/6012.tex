\uuid{6012}
\auteur{quinio}
\datecreate{2011-05-20}

\contenu{
\texte{
Un industriel doit vérifier l'état de marche de ses machines et en
remplacer certaines le cas échéant. D'après des statistiques précédentes, 
il évalue à 30\% la probabilité pour une
machine de tomber en panne en 5 ans; parmi ces dernières, 
la probabilité de devenir hors d'usage suite à une panne plus grave est 
évaluée à 75\%; cette probabilité est de 40\% pour une machine
n'ayant jamais eu de panne.
}
\begin{enumerate}
    \item \question{Quelle est la probabilité pour une machine donnée de plus de cinq
ans d'être hors d'usage ?}
\reponse{30\% est la probabilité de l'événement Panne, noté $Pa$;
la probabilité pour une machine donnée de plus de cinq ans, d'être hors d'usage est
$P(HU)=P(HU/Pa) P(Pa)+P(HU/nonPa) P(nonPa)=0.3\cdot 0.75+0.4\cdot
0.7=0.505$.}
    \item \question{Quelle est la probabilité pour une machine hors d'usage de n'avoir
jamais eu de panne auparavant ?}
\reponse{La probabilité pour une machine hors d'usage de n'avoir jamais eu de
panne auparavant est 
$P(\text{non}\,Pa/HU)=P(HU/\text{non}\,Pa) P(\text{non}\,Pa)/P(HU)=0.4\cdot 0.7/
0.505\,=0.554\,46$.}
    \item \question{Soit $X$ la variable aléatoire <<nombre de machines
qui tombent en panne au bout de 5 ans, parmi 10 machines choisies au
hasard>>. Quelle est la loi de probabilité de $X$, (on
donnera le type de loi et les formules de calcul), son espérance, sa
variance et son écart-type ?}
\reponse{La loi de probabilité de $X$ est une loi binomiale, $n= 10$, $p=0.3$, espérance $3$.}
    \item \question{Calculer $P[X=5]$.}
\reponse{$P[X=5]=\binom{10}{5}(0.3)^{5}(0.7)^{5}=0.102\,92$}
\end{enumerate}
}
