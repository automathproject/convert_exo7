\uuid{VmGq}
\exo7id{5189}
\auteur{rouget}
\datecreate{2010-06-30}
\isIndication{false}
\isCorrection{true}
\chapitre{Application linéaire}
\sousChapitre{Définition}

\contenu{
\texte{
Soit $f\in\mathcal{L}(\Rr^2)$. Pour $(x,y)\in\Rr^2$, on pose $f((x,y))=(x',y')$.
}
\begin{enumerate}
    \item \question{Rappeler l'écriture générale de $(x',y')$ en fonction de $(x,y)$.}
\reponse{Pour $(x,y)\in\Rr^2$, posons $f((x,y))=(x',y')$.

\begin{center}
$f\in\mathcal{L}(\Rr^2)\Leftrightarrow\exists(\alpha,\beta,\gamma,\delta)\in\Rr^4/\;\forall(x,y)\in\Rr^2,\;\left\{
\begin{array}{l}
x'=\alpha x+\gamma y\\
y'=\beta x+\delta y
\end{array}
\right.$.
\end{center}}
    \item \question{Si on pose $z=x+iy$ et $z'=x'+iy'$ (où $i^2=-1$), montrer que~:~$\exists(a,b)\in\Cc^2/\;\forall
z\in\Cc,\;z'=az+b{\bar z}$.}
\reponse{Avec les notations précédentes,

\begin{align*}
z'&=x'+iy'=(\alpha x+\gamma y)+i(\beta x+\delta y)
=(\alpha\frac{z+{\bar z}}{2}+\gamma\frac{z-{\bar z}}{2i})+i(\beta\frac{z+{\bar z}}{2}+\delta\frac{z-{\bar z}}{2i})\\
 &=\left(\frac{\alpha+\delta}{2}+i\frac{\beta-\gamma}{2}\right)z+\left(\frac{\alpha-\delta}{2}+i\frac{\beta+\gamma}{2}\right){\bar z}
=az+b{\bar z}
\end{align*}
où $a=\frac{\alpha+\delta}{2}+i\frac{\beta-\gamma}{2}$ et $b=\frac{\alpha-\delta}{2}+i\frac{\beta+\gamma}{2}$.}
    \item \question{Réciproquement, montrer que l'expression ci-dessus définit un unique endomorphisme de
$\Rr^2$ (en clair, l'expression complexe d'un endomorphisme de $\Rr^2$ est $z'=az+b{\bar z}$).}
\reponse{Réciproquement, si $z'=az+b{\bar z}$, en posant $a=a_1+ia_2$ et $b=b_1+ib_2$ où $(a_1,a_2,b_1,b_2)\in\Rr^4$, on
obtient~:

$$x'+iy'=(a_1+ia_2)(x+iy)+(b_1+ib_2)(x-iy)=(a_1+b_1)x+(-a_2+b_2)y+i\left((a_2+b_2)x+(a_1-b_1)y\right)$$
et donc,

$$\left\{
\begin{array}{l}
x'=(a_1+b_1)x+(b_2-a_2)y\\
y'=(a_2+b_2)x+(a_1-b_1)y
\end{array}
\right..$$}
\end{enumerate}
}
