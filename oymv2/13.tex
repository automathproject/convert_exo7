\uuid{13}
\auteur{bodin}
\datecreate{1998-09-01}
\isIndication{true}
\isCorrection{true}
\chapitre{Nombres complexes}
\sousChapitre{Forme cartésienne, forme polaire}

\contenu{
\texte{
D\'eterminer le module et l'argument des nombres
complexes :
$$e^{e^{i\alpha}} \quad \text{ et } \quad e^{i\theta}+e^{2i\theta}.$$
}
\indication{Pour calculer un somme du type
$e^{iu}+e^{iv}$ il est souvent utile de factoriser par
$e^{i\frac{u+v}{2}}$.}
\reponse{
D'apr\`es la formule de Moivre pour $e^{i\alpha}$ nous avons :
$$e^{e^{i\alpha}} = e^{\cos \alpha + i\sin \alpha}
= e^{\cos \alpha}e^{i\sin \alpha}.$$ Or $e^{\cos \alpha} > 0$ donc
l'\'ecriture pr\'ec\'edente est bien de la forme
``module-argument''.

\bigskip

De fa\c{c}on g\'en\'erale pour calculer une somme du type
$e^{iu}+e^{iv}$ il est souvent utile de factoriser par
$e^{i\frac{u+v}{2}}$. En effet
\begin{align*}
e^{iu}+e^{iv} &= e^{i\frac{u+v}{2}}\left( e^{i
\frac{u-v}{2}}+ e^{-i \frac{u-v}{2}}\right) \\
&= e^{i\frac{u+v}{2}} 2 \cos  \frac{u-v}{2} \\
&=  2 \cos  \frac{u-v}{2} e^{i\frac{u+v}{2}}.\\
\end{align*}
Ce qui est proche de l'\'ecriture en coordon\'ees polaires.

Pour le cas qui nous concerne :
$$z = e^{i\theta} + e^{2i\theta}
= e^{\frac{3i\theta}{2}} \left[ e^{-\frac{i\theta}{2}} +
e^{\frac{i\theta}{2}} \right] = 2\cos \frac{\theta}{2}
e^{\frac{3i\theta}{2}}.$$ Attention le module dans une
d\'ecomposion en forme polaire doit \^etre positif ! Donc si $\cos
\frac{\theta}{2}  \geq 0$  alors $2\cos\frac{\theta}{2}$ est le module de $z$
et $3\theta/2$ est son argument ; par contre si $\cos \frac{\theta}{2}  <
0$ le module est $2|\cos\frac{\theta}{2}|$ et l'argument $3\theta/2+\pi$ (le
$+\pi$ compense le changement de signe car $e^{i\pi} = -1$).
}
}
