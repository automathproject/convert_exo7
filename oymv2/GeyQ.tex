\uuid{GeyQ}
\exo7id{5667}
\auteur{exo7}
\datecreate{2010-10-16}
\isIndication{false}
\isCorrection{true}
\chapitre{Réduction d'endomorphisme, polynôme annulateur}
\sousChapitre{Diagonalisation}

\contenu{
\texte{
\newcommand{\adots}{\mathinner{\mkern2mu\raise 1pt\hbox{.}\mkern 3mu\raise
4pt\hbox{.}\mkern1mu\raise 7pt\hbox{{.}}}}


Soit $A=\left(
\begin{array}{cccc}
0&\ldots&0&1\\
\vdots&\adots&\adots&0\\
0& \adots & \adots &\vdots\\
1&0&\ldots&0
\end{array}
\right)$

 Montrer que $A$ est diagonalisable.
}
\reponse{
Soit $f$ l'endomorphisme de $\Rr^n$ de matrice $A$ dans la base canonique $(e_1,\ldots,e_n)$ de $\Rr^n$. $\forall i\in\llbracket1,n\rrbracket$, $f(e_{i})=e_{n+1-i}$ et donc $\forall i\in\llbracket1,n\rrbracket$, $f^2(e_{i})=e_i$. Donc $f$ est une symétrie distincte de l'identité et en particulier $\text{Sp}A=\{-1,1\}$ et $f$ est diagonalisable. On en déduit que $A$ est diagonalisable dans $\Rr$.
}
}
