\uuid{jJoS}
\exo7id{7097}
\auteur{megy}
\datecreate{2017-01-21}
\isIndication{true}
\isCorrection{true}
\chapitre{Géométrie affine euclidienne}
\sousChapitre{Géométrie affine euclidienne du plan}

\contenu{
\texte{
% translations
Soit $\mathcal C$ un cercle et $D$ une droite. Construire une droite parallèle à $D$ coupant le cercle $\mathcal C$ en deux points situés à une distance $a$ donnée (inférieure au diamètre).
}
\indication{Considérer la translation $\tau$ de distance $a$ suivant la direction de la droite.}
\reponse{
Appliquer la translation au cercle. (Si on n'a pas donné le centre du cercle, commencer par construire le centre.)

Les points d'intersection des deux cercles fournissent les (ou la) solutions du problème.
}
}
