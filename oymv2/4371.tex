\uuid{4371}
\auteur{quercia}
\datecreate{2010-03-12}
\isIndication{false}
\isCorrection{true}
\chapitre{Intégration}
\sousChapitre{Intégrale de Riemann dépendant d'un paramètre}

\contenu{
\texte{
Soit $f$ : $x \mapsto  \int_{t=0}^{+\infty}\frac{d t}{t^{x+1}+t+1}$.
Déterminer son domaine de définition~; étudier sa continuité et sa monotonie.
Calculer $ \int_{t=1}^{+\infty}\frac{d t }{t^{x+1}+t}$ et en déduire des
équivalents et les limites de~$f$ en~$0$ et en~$+\infty$.
}
\reponse{
$D_f = {]0,+\infty[}$. Il y a domination locale, donc $f$ est continue.

De même, pour $x>0$ on a $f'(x) =  \int_{t=0}^{+\infty}\frac{-\ln(t)t^{x+1}}{(t^{x+1}+t+1)^2}\,d t$.
En coupant l'intégrale en~$1$ et en posant $u=1/t$ dans l'intégrale sur $[1,+\infty[$
il vient~: $f'(x) =  \int_{t=0}^1\ln(t)t^{x+1}\Bigl(\frac1{(t+t^{x+1}+t^{x+2})^2} - \frac1{(t^{x+1}+t+1)^2}\Bigr)\,d t < 0$
car $\ln(t)<0$ et $t^{x+2}<1$ si $t\in{]0,1[}$. Donc $f$ est strictement décroissante sur~$]0,+\infty[$.

$ \int_{t=1}^{+\infty}\frac{d t }{t^{x+1}+t}
=  \int_{t=1}^{+\infty}\sum_{k=0}^\infty\frac{(-1)^k}{t^{(k+1)x+1}}\,d t =$
(domination du reste avec le CSA)
$= \sum_{k=0}^\infty\frac{(-1)^k}{(k+1)x}= \frac{\ln 2}x$.

$\Bigl|f(x)- \int_{t=1}^{+\infty}\frac{1}{t^{x+1}+t}\Bigr|
= \int_{t=0}^1\frac{d t}{t^{x+1}+t+1} +  \int_{t=1}^{+\infty}\frac{d t}{(t^{x+1}+t)(t^{x+1}+t+1)}
\le  \int_{t=0}^1\frac{d t}{t+1} +  \int_{t=1}^{+\infty}\frac{d t}{t(t+1)} = 2\ln 2$ 
donc $f(x) = \frac{\ln 2}x + O_{x\to0^+}(1)$.

Pour $x\to+\infty$, on a avec le TCM séparément sur $[0,1]$ et sur $[1,+\infty[$~:
lorsque $x\to+\infty$, $f(x)\to \int_{t=0}^1\frac{d t}{t+1} = \ln 2$.
}
}
