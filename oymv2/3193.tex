\uuid{3193}
\auteur{quercia}
\datecreate{2010-03-08}
\isIndication{false}
\isCorrection{true}
\chapitre{Polynôme, fraction rationnelle}
\sousChapitre{Autre}

\contenu{
\texte{
Soit $P\in\Z[X]$ non constant et $E$ l'ensemble des diviseurs premiers d'au moins un $P(n)$, $n\in\Z$.
Montrer que~$E$ est infini.
}
\reponse{
On suppose $E$ fini et on montre que~$P$ est constant~: il existe~$a\in\Z$
tel que~$P(a)\ne 0$. Soit $N = \prod_{p\in E}p^{1+v_p(P(a))}$.
Alors pour tout~$k\in\Z$, $P(a+kN)\equiv P(a) (\mathrm{mod}\, N)$ (formule de Taylor),
donc $v_p(P(a+kN)) = v_p(P(a))$ pour tous~$k\in\Z$ et~$p\in E$.
Comme $P(a+kN)$ est produit d'{\'e}l{\'e}ments de~$E$, on en d{\'e}duit que $P(a+kN) = \pm P(a)$
pour tout~$k$, donc $P$ prend une infinit{\'e} de fois la m{\^e}me valeur.
}
}
