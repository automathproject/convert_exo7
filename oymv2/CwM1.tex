\uuid{CwM1}
\exo7id{239}
\auteur{cousquer}
\datecreate{2003-10-01}
\isIndication{true}
\isCorrection{true}
\chapitre{Dénombrement}
\sousChapitre{Cardinal}

\contenu{
\texte{
On considère les mains de $5$~cartes que l'on peut extraire d'un jeu de
$52$~cartes.
}
\begin{enumerate}
    \item \question{Combien y a-t-il de mains différentes~?}
\reponse{Il s'agit donc de choisir $5$ cartes parmi $52$ : il y a donc $C_{52}^5$ mains différentes.
Ceci peut être calculé : $C_{52}^5 = \frac{52\cdot51\cdot50\cdot49\cdot48}{5!} = 2598960$.}
    \item \question{Combien y a-t-il de mains comprenant exactement un as~?}
\reponse{Il y a $4$ choix pour l'as (l'as de pique ou l'as de c{\oe}ur ou ...), puis il faut choisir les $4$ cartes restantes
parmi $48$ cartes (on ne peut pas rechoisir un as). Bilan $4 \times C_{48}^4$ mains comprenant exactement un as.}
    \item \question{Combien y a-t-il de mains comprenant au moins un valet~?}
\reponse{Il est beaucoup plus facile de compter d'abord les mains qui ne contiennent aucun valet :
il faut choisir $5$ cartes parmi $48$ (on exclut les valets) ; il y a donc $C_{48}^5$ mains ne contenant aucun valet.
Les autres mains sont les mains qui contiennent au moins un valet : il y en a donc $C_{52}^5 - C_{48}^5$.}
    \item \question{Combien y a-t-il de mains comprenant (à la fois) au moins un roi et au
moins une dame~?}
\reponse{Nous allons d'abord compter le nombre de mains que ne contiennent pas de roi ou pas de dame.
Le nombre de mains qui ne contiennent pas de roi est $C_{48}^5$ (comme la question 3.). Le nombre de mains qui ne contiennent
pas de dame est aussi $C_{48}^5$. 
Le nombre de mains ne contenant pas de roi ou pas de dame \emph{n'est pas} $C_{48}^5 + C_{48}^5$,
car on aurait compté deux fois les mains ne contenant ni roi, ni dame (il y a $C_{44}^5$ telles mains).
Le nombre de mains ne contenant pas de roi ou pas de dame est donc : $2C_{48}^5-C_{44}^5$ (on retire une fois les mains comptées deux fois !).
Ce que nous cherchons ce sont toutes les autres mains : celles qui contiennent au moins un roi et au moins une dame.
Leur nombre est donc : $C_{52}^5 - 2C_{48}^5 + C_{44}^5$.}
\indication{Petits rappels : dans un jeu de $52$ cartes il y a $4$ ``couleurs'' (pique, c{\oe}ur, carreau, trèfle)
et $13$ ``valeurs'' ($1 =$ As, $2,3,\ldots,10$, Valet, Dame, Roi).
Une ``main'' c'est juste choisir $5$ cartes parmi les $52$, l'ordre du choix n'important pas.}
\end{enumerate}
}
