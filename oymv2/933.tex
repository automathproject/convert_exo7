\uuid{933}
\auteur{liousse}
\datecreate{2003-10-01}
\isIndication{false}
\isCorrection{false}
\chapitre{Application linéaire}
\sousChapitre{Image et noyau, théorème du rang}

\contenu{
\texte{
Soient :
$E$, $F$ et $G$ trois sous espaces vectoriels de $\mathbb R^N$,
 $f$ une application lin\'eaire de $E$ dans $F$ et 
 $g$ une application lin\'eaire de $F$ dans $G$.
On rappelle que  $g\circ f$ est l'application de $E$ dans $G$ d\'efinie par $
g\circ f(v) = g(f(v))$, pour tout vecteur $v$ de $E$.
}
\begin{enumerate}
    \item \question{Montrer que $g\circ f$ est une application lin\'eaire.}
    \item \question{Montrer que $f\big({\text Ker}(g\circ f) \big) = {\text Ker} g \cap {\text Im} f$.}
\end{enumerate}
}
