\uuid{808}
\auteur{cousquer}
\datecreate{2003-10-01}
\isIndication{false}
\isCorrection{true}
\chapitre{Calcul d'intégrales}
\sousChapitre{Longueur, aire, volume}

\contenu{
\texte{
On appelle \emph{cycloïde} la courbe décrite par un point d'un
cercle de rayon~$R$, lié à ce cercle, quand celui-ci roule sans glisser
sur une droite en restant dans plan fixe.
Montrer que dans un repère bien choisi,
la cycloïde admet la représentation paramétrique~:
$\left\lbrace
\begin{array}{l}
    x = R(t-\sin t)  \\
    y = R(1-\cos t)
\end{array}\right.$
Représenter la cycloïde et calculer~: la longueur $L$ d'une arche,
l'aire $A$ de la surface $S$ comprise entre cette arche et la droite fixe
$(Ox)$,
les volumes $V_1$ et $V_2$ obtenus par révolution de $S$ autour de $Ox$
et $Oy$ respectivement,
les aires $A_1$ et $A_2$ obtenues par révolution d'une arche
de la cycloïde  autour de $Ox$ et $Oy$ respectivement.
}
\reponse{
$\displaystyle L=8R,\quad A=3\pi R^2,\quad V_1=5\pi^2R^3,\quad
V_2=6\pi^3R^3,\quad A_1=\frac{64\pi R^2}{3},\quad
A_2=16\pi^2R^2$.
}
}
