\uuid{S1Oy}
\exo7id{7773}
\auteur{mourougane}
\datecreate{2021-08-11}
\isIndication{false}
\isCorrection{false}
\chapitre{Action de groupe}
\sousChapitre{Action de groupe}

\contenu{
\texte{
Soit $G$ un groupe d'ordre $8$.
}
\begin{enumerate}
    \item \question{Enumérer quatre groupes d'ordre $8$, deux à deux non isomorphes, et même $5$ si possible.}
    \item \question{On suppose que tous les éléments de $G$ sont d'ordre $2$.
Montrer que $G$ est abélien. Soit $a$ et $b$ deux éléments non neutres distincts de $G$. Montrer que $\{e,a,b,ab\}$ est un sous-groupe d'ordre $4$ de $G$. 
Déterminer un isomorphisme de $G$ avec un groupe connu.}
    \item \question{On suppose que $G$ admet un élément $a$ d'ordre $4$. Soit $b$ un élément hors du sous-groupe engendré par $a$. Montrer que $\langle a\rangle$ est distingué et que $b^2$ appartient à $\langle a\rangle$.
\begin{enumerate}}
    \item \question{Quel est l'ordre de $b$ si $b^2=a$ ou si $b^2=a^3$ ? Conclure dans ce cas.}
    \item \question{Si $b^2=e$, montrer que $G$ est un produit semi-direct et en déduire un isomorphisme avec un groupe connu.}
    \item \question{Si tous les éléments hors de $<a>$ ont un carré égal à $a^2$, établir la liste des éléments et la table de multiplication de $G$ à l'aide seulement de $a$ et $b$.}
\end{enumerate}
}
