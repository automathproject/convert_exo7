\uuid{89}
\auteur{cousquer}
\datecreate{2003-10-01}
\isIndication{false}
\isCorrection{true}
\chapitre{Nombres complexes}
\sousChapitre{Trigonométrie}

\contenu{
\texte{
Résoudre dans $\mathbb{R}$ les équations suivantes:
(donner les valeurs des solutions appartenant à 
$\left]-\pi, \pi\right]$ et les placer sur le cercle trigonométrique).
}
\begin{enumerate}
    \item \question{$ \sin\left(5x\right)=\sin\left(\frac{2\pi}{3}+x\right)$,}
\reponse{$\sin\left(5x\right)=\sin\left(\frac{2\pi}{3}+x\right)$
ssi $x=\pi/6+k\pi/2$ ou $x=\pi/18+k\pi/3$, avec $k\in\mathbb{Z}$.}
    \item \question{$  \sin\left(2x-\frac{\pi}{3}\right)=\cos\left(\frac{x}{3}\right)$,}
\reponse{$
  \sin\left(2x-\frac{\pi}{3}\right)=\cos\left(\frac{x}{3}\right)$ ssi
$x=5\pi/14+6k\pi/7$ ou $x=\pi/2+6k\pi/5$, avec $k\in\mathbb{Z}$.}
    \item \question{$\cos\left(3x\right)=\sin\left(x\right)$.}
\reponse{$\cos\left(3x\right)=\sin\left(x\right)$ ssi $x=\pi/8+k\pi/2$ ou
  $x=-\pi/4+k\pi$, avec $k\in\mathbb{Z}$.}
\end{enumerate}
}
