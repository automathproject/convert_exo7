\uuid{AibI}
\exo7id{5318}
\auteur{rouget}
\datecreate{2010-07-04}
\isIndication{false}
\isCorrection{true}
\chapitre{Polynôme, fraction rationnelle}
\sousChapitre{Racine, décomposition en facteurs irréductibles}

\contenu{
\texte{
Pour quelles valeurs de l'entier naturel $n$ le polynôme $(X+1)^n-X^n-1$ est-il divisible par $X^2+X+1$~?
}
\reponse{
Soit $n\in\Nn$.

\begin{align*}\ensuremath
(X+1)^n-X^n-1\;\mbox{est divisible par}\;X^2+X+1&\Leftrightarrow j\;\mbox{et}\;j^2\;\mbox{sont racines de}\;(X+1)^n-X^n-1\\
 &\Leftrightarrow j\;\mbox{est racine de}\;(X+1)^n-X^n-1\\
 &(\mbox{car}\;(X+1)^n-X^{n-1}\;\mbox{est dans}\;\Rr[X])\\
 &\Leftrightarrow(j+1)^n-j^n-1=0\Leftrightarrow(-j^2)^n-j^n-1=0.
\end{align*}

Si $n\in6\Zz$, $(-j^2)^n-j^n-1=-3\neq0$.
 
Si $n\in1+6\Zz$, $(-j^2)^n-j^n-1=-j^2-j-1=0$.
 
Si $n\in2+6\Zz$, $(-j^2)^n-j^n-1=j-j^2-1=2j\neq0$.
 
Si $n\in3+6\Zz$, $(-j^2)^n-j^n-1=-3\neq0$.
 
Si $n\in4+6\Zz$, $(-j^2)^n-j^n-1=j^2-j-1=2j^2\neq0$.
 
Si $n\in5+6\Zz$, $(-j^2)^n-j^n-1=-j-j^2-1=0$.

En résumé, $(X+1)^n-X^n-1$ est divisible par $X^2+X+1$ si et seulement si $n$ est dans $(1+6\Zz)\cup(5+6\Zz)$.
}
}
