\uuid{GGKD}
\exo7id{5329}
\auteur{rouget}
\datecreate{2010-07-04}
\isIndication{false}
\isCorrection{true}
\chapitre{Polynôme, fraction rationnelle}
\sousChapitre{Racine, décomposition en facteurs irréductibles}

\contenu{
\texte{
Soit $(a_k)_{1\leq k\leq 5}$ la famille des racines de $P=X^5+2X^4-X-1$. Calculer $\sum_{k=1}^{5}\frac{a_k+2}{a_k-1}$.
}
\reponse{
On note que $P(1)=1\neq0$ et donc que l'expression proposée a bien un sens.

$$\sum_{k=1}^{5}\frac{a_k+2}{a_k-1}=\sum_{k=1}^{5}(1+\frac{3}{a_k-1})=5-3\sum_{k=1}^{5}\frac{1}{1-a_k}=5-3\frac{P'(1)}{P(1)}=5-3\frac{12}{1}=-31.$$
}
}
