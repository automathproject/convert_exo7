\uuid{5438}
\auteur{rouget}
\datecreate{2010-07-06}
\isIndication{false}
\isCorrection{true}
\chapitre{Développement limité}
\sousChapitre{Calculs}

\contenu{
\texte{

}
\begin{enumerate}
    \item \question{Montrer que l'équation $x+\ln x=k$ admet, pour $k$ réel donné, une unique solution dans $]0,+\infty[$, notée $x_k$.}
\reponse{Pour $x>0$, posons $f(x)=x+\ln x$. $f$ est continue sur $]0,+\infty[$, strictement croissante sur $]0,+\infty[$ en tant que somme de deux fonctions continues et strictement croissantes sur $]0,+\infty[$. $f$ réalise donc une bijection de $]0,+\infty[$ sur $f\left(]0,+\infty[\right)=\left]\lim_{x\rightarrow 0,\;x>0}f(x),\lim_{x\rightarrow +\infty}f(x)\right[=]-\infty,+\infty[=\Rr$. En particulier,

\begin{center}
\shadowbox{
$\forall k\in\Rr,\;\exists!x_k\in]0,+\infty[/\;f(x_k)=k.$
}
\end{center}}
    \item \question{Montrer que, quand $k$ tend vers $+\infty$, on a~:~$x_k=ak+b\ln k+c\frac{\ln k}{k}+o\left(\frac{\ln k}{k}\right)$ où $a$, $b$ et $c$ sont des constantes à déterminer.}
\reponse{$f\left(\frac{k}{2}\right)=\frac{k}{2}+\ln\frac{k}{2}<k$ pour $k$ suffisament grand (car $k-(\frac{k}{2}+\ln\frac{k}{2})=\frac{k}{2}-\ln\frac{k}{2}\underset{k\rightarrow+\infty}{\rightarrow}+\infty$ d'après les théorèmes de croissances comparées). Donc, pour $k$ suffisament grand, $f\left(\frac{k}{2}\right)<f(x_k)$. Puisque $f$ est strictement croissante sur $]0,+\infty[$, on en déduit que $x_k>\frac{k}{2}$ pour $k$ suffisament grand et donc que $\lim_{k\rightarrow +\infty}x_k=+\infty$. Mais alors, $k=x_k+\ln x_k\sim x_k$ et donc, quand $k$ tend vers $+\infty$,

\begin{center}
\shadowbox{
$x_k\underset{k\rightarrow+\infty}{=}k+o(k).$
}
\end{center}
Posons $y_k=x_k-k$. On a $y_k=o(k)$ et de plus $y_k+\ln(y_k+k)=0$ ce qui s'écrit~:

$$y_k=-\ln(k+y_k)=-\ln(k+o(k))=-\ln k+\ln(1+o(1))=-\ln k+o(1).$$
Donc,

\begin{center}
\shadowbox{
$x_k\underset{k\rightarrow+\infty}{=}k-\ln k+o(1).$
}
\end{center}
Posons $z_k=y_k+\ln k=x_k-k+\ln k$. Alors, $z_k=o(1)$ et $-\ln k+z_k=-\ln(k-\ln k+z_k)$. Par suite,

$$z_k=\ln k-\ln(k-\ln k+o(1))=-\ln\left(1-\frac{\ln k}{k}+o\left(\frac{\ln k}{k}\right)\right)=\frac{\ln k}{k}+o\left(\frac{\ln k}{k}\right).$$

Finalement,

\begin{center}
\shadowbox{
$x_k\underset{k\rightarrow+\infty}{=}k-\ln k+\frac{\ln k}{k}+o\left(\frac{\ln k}{k}\right).$
}
\end{center}}
\end{enumerate}
}
