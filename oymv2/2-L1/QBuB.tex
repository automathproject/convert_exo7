\uuid{QBuB}
\exo7id{682}
\auteur{bodin}
\datecreate{1998-09-01}
\isIndication{false}
\isCorrection{false}
\chapitre{Continuité, limite et étude de fonctions réelles}
\sousChapitre{Continuité : pratique}

\contenu{
\texte{
Soit $f : [0,1]\longrightarrow \R$ d\'efinie par $f(0)=0$,
$f(x)=1/2-x$ si $x\in ]0,1/2[$, $f(1/2)=1/2$, $f(x)=3/2-x$ si $x\in ]1/2,1[$ et $f(1)=1$.
}
\begin{enumerate}
    \item \question{Tracer le graphe de $f$. \'Etudier sa continuit\'e.\\}
    \item \question{D\'emontrer que $f$ est une bijection de $[0,1]$ sur $[0,1]$.\\}
    \item \question{D\'emontrer que pour tout $x\in [0,1]$, on a
$ f(x)= \frac{1}{2}-x+\frac{1}{2}E(2x)-\frac{1}{2}E(1-2x)$.}
\end{enumerate}
}
