\uuid{XGV3}
\exo7id{4297}
\auteur{quercia}
\datecreate{2010-03-12}
\isIndication{false}
\isCorrection{true}
\chapitre{Calcul d'intégrales}
\sousChapitre{Intégrale impropre}

\contenu{
\texte{
Soit $f:\R \to \R$ continue par morceaux telle que $ \int_{t=-\infty}^{+\infty}|f(t)|\,d t$ converge.
On pose $F(x) = \frac12 \int_{t=x-1}^{x+1}f(t)\,d t$.

Montrer que $ \int_{t=-\infty}^{+\infty}F(t)\,d t =  \int_{t=-\infty}^{+\infty}f(t)\,d t$.

Démontrer le même résultat en supposant seulement la convergence de
$ \int_{t=-\infty}^{+\infty}f(t)\,d t$.
}
\reponse{
$ \int_{t=a}^b F(t)\,d t =  \int_{u=a-1}^{a+1} \frac{u-(a-1)}2f(u)\,d u
                        +  \int_{u=a+1}^{b-1} f(u)\,d u
                        +  \int_{u=b-1}^{b+1} \frac{b+1-u}2f(u)\,d u$.\par
$\phantom{ \int_{t=a}^b F(t)\,d t} =
\varphi(a+1) - \frac12 \int_{u=a-1}^{a+1}\varphi(u)\,d u
+  \int_{u=a+1}^{b-1} f(u)\,d u
+ \frac12 \int_{u=b-1}^{b+1}\varphi(u)\,d u - \varphi(b-1)$
où $\varphi$ est une primitive de $f$.
}
}
