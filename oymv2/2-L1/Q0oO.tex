\uuid{Q0oO}
\exo7id{3951}
\auteur{quercia}
\datecreate{2010-03-11}
\isIndication{false}
\isCorrection{false}
\chapitre{Dérivabilité des fonctions réelles}
\sousChapitre{Autre}

\contenu{
\texte{
Soit $f : {[a,b]} \to \R$ de classe $\mathcal{C}^2$.
}
\begin{enumerate}
    \item \question{On suppose que $f(a) = f(b) = 0$. Soit $c \in {]a,b[}$.
      Montrer qu'il existe $d \in {]a,b[}$ tel que :
      $$f(c) = -\frac {(c-a)(b-c)}2 f''(d).$$
      (Considérer
      $g(t) = f(t) + \lambda (t-a)(b-t)$ où $\lambda$ est choisi de
      sorte que $g(c) = 0$)}
    \item \question{Cas général : Soit $c \in {]a,b[}$.
      Montrer qu'il existe $d \in {]a,b[}$ tel que :
      $$f(c) = \frac {b-c}{b-a} f(a) + \frac {c-a}{b-a} f(b) -\frac {(c-a)(b-c)}2 f''(d).$$}
\end{enumerate}
}
