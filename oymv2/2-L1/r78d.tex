\uuid{r78d}
\exo7id{2330}
\auteur{matexo1}
\datecreate{2002-02-01}
\isIndication{false}
\isCorrection{false}
\chapitre{Calcul d'intégrales}
\sousChapitre{Intégrale impropre}

\contenu{
\texte{
{\bf Fonction Gamma - }
Pour tout $x>0$, on pose
$$ \Gamma (x) = \int_0^{+\infty} {t^{x-1} e^{-t}} dt $$
(on admettra que l'int\'egrale converge). Montrer que 
$\Gamma(x+1) = x \Gamma(x)$. Calculer la valeur de 
$\Gamma(1)$. En d\'eduire celle de $\Gamma(n)$, pour tout
entier $n>0$.

\item Soit $a>0$ un r\'eel, et $n>0$ un entier. Montrer que
$$ \int {dt \over (x^2 + a^2)^n} = {1 \over a^{2n-1}}
	\int \cos^{2n-2}\theta d\theta \qquad \mbox{\rm o\`u}\qquad
\theta = \arctan{x \over a}. $$
En d\'eduire la primitive de $\displaystyle{ x+4 \over (x^2+2x+2)^3 }$.

\item Soient $x$ et $y$ deux r\'eels v\'erifiant $1>y>x>0$. Calculer
$$ \lim_{x\to  0 \atop y\to  1} 
 \int_x^y {\ln t \over (1+t)\sqrt{1-t^2}} dt. $$

\item Soit $f$ une fonction continue et positive sur $[0, +\infty[$.
On pose pour tout $x>0$ et tout entier $n>0$
$$ u_n(x) = \left[ \int_0^x f(t)^n\ dt \right]^{1/n} $$
et
$$ M(x) = \sup_{t \in [0, x]} \left| f(t) \right|. $$
}
\begin{enumerate}
    \item \question{Montrer que $u_n(x)  \leq M(x) x^{1/n}$.}
    \item \question{En utilisant la continuit\'e de $f$, montrer que, quel que 
soit $\varepsilon > 0$, il existe $\delta > 0$ tel que 
$u_n(x) \geq \delta^{1/n} [ M(x)-\varepsilon ]$.}
    \item \question{En d\'eduire que 
$$\lim_{n \to +\infty} u_n(x) = M(x).$$}
\end{enumerate}
}
