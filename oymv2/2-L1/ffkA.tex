\uuid{ffkA}
\exo7id{725}
\auteur{bodin}
\datecreate{1998-09-01}
\isIndication{true}
\isCorrection{true}
\chapitre{Dérivabilité des fonctions réelles}
\sousChapitre{Théorème de Rolle et accroissements finis}

\contenu{
\texte{
Par application du th\'eor\`eme des accroissements finis \`a
$f(x) = \ln x$ sur $[n,n+1]$
montrer que
$$ S_n = \sum_{k=1}^n \frac{1}{k}$$
tend vers l'infini quand $n$ tend vers l'infini.
}
\indication{Une fois le th\'eor\`eme des accroissements finis utilis\'e vous obtenez une somme t\'el\'escopique.}
\reponse{
Le th\'eor\`eme des accroissements finis donne : 
$\ln(n+1)-\ln(n) = \frac1{c_n} (n+1-n) = \frac1{c_n}$, avec $c_n \in [n,n+1]$. Or $c_n \geq n$ donc $\frac 1n \geq \frac 1{c_n}$.
Donc :
$$S_n=\sum_{k=1}^n \frac 1k \geq \sum_{k=1}^n \frac 1{c_k}
= \sum_{k=1}^n \ln(k+1)-\ln(k) = \ln(n+1).$$
La derni\`ere \'egalit\'e s'obtient car la somme est t\'el\'escopique et $\ln 1 = 0$.
Donc $S_n \geq \ln(n+1)$, donc $S_n \rightarrow +\infty$.
}
}
