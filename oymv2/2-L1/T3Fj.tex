\uuid{T3Fj}
\exo7id{805}
\auteur{cousquer}
\datecreate{2003-10-01}
\isIndication{false}
\isCorrection{true}
\chapitre{Calcul d'intégrales}
\sousChapitre{Longueur, aire, volume}

\contenu{
\texte{
Construire la courbe paramétrée
$C\left\lbrace\renewcommand{\arraystretch}{1.2}
\begin{array}{l}
    x = \frac{\cos t}{1+\lambda\cos t}  \\
    y = \frac{\sin t}{1+\lambda\cos t}
\end{array}\right.$
où $\lambda$ est un
paramètre appartenant à $[0,1\mathclose[$. \\Calculer l'aire $S$ limitée par $C$
de deux façons~:
\begin{itemize}
\item En se ramenant au calcul de $\int_0^{2\pi}\frac{dt}{(1+\lambda\cos t)^2}$.

\item En reconnaissant la nature géométrique de $C$.
\end{itemize}
}
\reponse{
$\displaystyle S=\frac{\pi}{(1-\lambda^2)^{3/2}}$.
}
}
