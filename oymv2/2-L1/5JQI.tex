\uuid{5JQI}
\exo7id{4232}
\auteur{quercia}
\datecreate{2010-03-12}
\isIndication{false}
\isCorrection{true}
\chapitre{Calcul d'intégrales}
\sousChapitre{Théorie}

\contenu{
\texte{
Soient $a,b \in \R$. \'Etudier la convergence des suites $(a_n)$, $(b_n)$
définies par :
$$a_0=a,\ b_0=b,
  \qquad a_{n+1} = \frac12 \int_{x=-1}^1 \min(x,b_n)\,d x,\
                       b_{n+1} = \frac12 \int_{x=-1}^1 \max(x,a_n)\,d x.$$
}
\reponse{
$a_{n+1} = \begin{cases} b_n &\text{ si } b_n < -1\cr
                   -(b_n-1)^2/4 &\text{ si } -1\le b_n\le1\cr
                   0 &\text{ si } b_n > 1,\cr\end{cases}$  

$b_{n+1} = \begin{cases} 0 &\text{ si } a_n < -1\cr
                   (a_n+1)^2/4 &\text{ si } -1\le a_n\le1\cr
                   a_n &\text{ si } a_n > 1.\cr\end{cases}$
\par
Donc $a_{n+1} = f(a_{n-1})$, $b_{n+1} = g(b_{n-1})$.
Point fixe : $a_n \to \sqrt8-3$, $b_n\to 3-\sqrt8$.
}
}
