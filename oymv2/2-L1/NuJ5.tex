\uuid{NuJ5}
\exo7id{776}
\auteur{ridde}
\datecreate{1999-11-01}
\isIndication{true}
\isCorrection{true}
\chapitre{Fonctions circulaires et hyperboliques inverses}
\sousChapitre{Fonctions hyperboliques et hyperboliques inverses}

\contenu{
\texte{
R\'esoudre l'\'equation $x^y = y^x$ o\`u $x$ et $y$ sont des entiers positifs non nuls.
}
\indication{Montrer que l'\'equation $x^y=y^x$ est \'equivalente \`a $\frac{\ln x}{x}= \frac{\ln y}{y}$,
puis \'etudier la fonction $x \mapsto \frac{\ln x}{x}$.}
\reponse{
$$x^y=y^x \Leftrightarrow e^{y\ln x}= e^{x\ln y}
\Leftrightarrow {y\ln x}= {x\ln y}\Leftrightarrow \frac{\ln x}{x}= \frac{\ln y}{y}$$
(la fonction exponentielle est bijective).
Etudions la fonction $f(x) = \frac{\ln x}{x}$ sur $[1,+\infty[$.
$$f'(x)= \frac{1-\ln x}{x^2},$$
donc $f$ est croissante sur $[1,e]$ et d\'ecroissante
sur $[e,+\infty[$. Donc pour $z \in ]0,f(e)[=]0,1/e[$, l'\'equation $f(x)=z$ a exactement deux solutions, une 
dans $]1,e[$ et une dans $]e,+\infty[$.

Revenons \`a l'\'equation $x^y=y^x$ \'equivalente \`a $f(x)=f(y)$.
Prenons $y$ un entier, nous allons distinguer trois cas :
$y=1$, $y=2$ et $y \geq 3$.
Si $y=1$ alors $f(y)=z=0$ on doit
donc r\'esoudre $f(x)=0$ et alors $x=1$.
Si $y=2$ alors il faut r\'esoudre l'\'equation $f(x) = \frac{\ln 2}{2} \in ]0,1/e[$.
Alors d'apr\`es l'\'etude pr\'ec\'edente, il existe deux solutions
une sur $]0,e[$ qui est $x=2$ (!) et une sur $]e,+\infty[$
qui est $4$, en effet $\frac {\ln 4}4= \frac {\ln 2}2$.
Nous avons pour l'instant les solutions correspondant à $2^2=2^2$ et $2^4=4^2$.

Si $y \geq 3$ alors $y> e$ donc il y a une solution $x$ 
de l'\'equation $f(x)=f(y)$ dans $]e,+\infty[$ qui est $x=y$,
et une solution $x$ dans l'intervalle $]1,e[$.
Mais comme $x$ est un entier alors $x=2$ (c'est le seul entier appartenant à $]1,e[$)
c'est un cas que nous avons d\'ej\`a \'etudi\'e conduisant à $4^2 = 2^4$.

Conclusion : les couples d'entiers qui v\'erifient l'\'equation $x^y=y^x$
sont les couples $(x,y=x)$ et les couples $(2,4)$ et $(4,2)$.
}
}
