\uuid{F10v}
\exo7id{5098}
\auteur{rouget}
\datecreate{2010-06-30}
\isIndication{false}
\isCorrection{true}
\chapitre{Continuité, limite et étude de fonctions réelles}
\sousChapitre{Etude de fonctions}

\contenu{
\texte{
Trouver la plus grande valeur de $\sqrt[n]{n}$, $n\in\Nn^*$.
}
\reponse{
Pour $n\in\Nn^*$, posons $u_n=\sqrt[n]{n}$ puis, pour $x$ réel strictement positif, $f(x)=x^{1/x}$
de sorte que pour tout naturel non nul $n$, on a $u_n=f(n)$.
$f$ est définie sur $]0,+\infty[$ et pour $x>0$, $f(x)=e^{\ln x/x}$. $f$ est dérivable sur $]0,+\infty[$ et pour
$x>0$,

$$f'(x)=\frac{1-\ln x}{x^2}e^{\ln x/x}.$$
Pour $x>0$, $f'(x)$ est du signe de $1-\ln x$ et donc $f'$ est strictement positive sur $]0,e[$ et strictement négative
sur $]e,+\infty[$. $f$ est donc strictement croissante sur $]0,e]$ et strictement décroissante sur $[e,+\infty[$. En
particulier, pour $n\geq3$,

$$u_n=f(n)\leq f(3)=u_3=\sqrt[3]{3}.$$
Comme $u_2=\sqrt{2}>1=u_1$, on a donc $\mbox{Max}\{u_n,\;n\in\Nn^*\}=\mbox{Max}\{\sqrt{2},\sqrt[3]{3}\}$. Enfin,
$\sqrt{2}=1,41...<1,44..=\sqrt[3]{3}$ (on peut aussi constater que $(\sqrt{2})^6=8<9=(\sqrt[3]{3})^6$). Finalement,

\begin{center}
\shadowbox{
$\text{Max}\left\{\sqrt[n]{n},\;n\in\Nn^*\right\}=\sqrt[3]{3}=1,44...$
}
\end{center}
}
}
