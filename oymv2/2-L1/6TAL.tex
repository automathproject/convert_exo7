\uuid{6TAL}
\exo7id{4250}
\auteur{quercia}
\datecreate{2010-03-12}
\isIndication{false}
\isCorrection{false}
\chapitre{Calcul d'intégrales}
\sousChapitre{Autre}

\contenu{
\texte{
Soit $f:\R \to \R$ de classe $\mathcal{C}^{n+p}$ telle que
$f(0) = f'(0) = \dots = f^{(n-1)}(0) = 0$.

On pose $g(x) = \frac {f(x)}{x^n}$ pour $x \ne 0$ et
$g(0) = \frac {f^{(n)}(0)}{n!}$.
}
\begin{enumerate}
    \item \question{\'Ecrire $g(x)$ sous forme d'une intégrale.}
    \item \question{En déduire que $g$ est de classe $\mathcal{C}^p$ et
    $|g^{(p)}(x)| \le \frac {p!}{(p+n)!}{\sup\{|f^{(n+p)}(tx)| \text{ tel que } 0 \le t \le 1 \}}$.}
\end{enumerate}
}
