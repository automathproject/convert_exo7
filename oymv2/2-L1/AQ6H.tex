\uuid{AQ6H}
\exo7id{4445}
\auteur{quercia}
\datecreate{2010-03-14}
\isIndication{false}
\isCorrection{true}
\chapitre{Série numérique}
\sousChapitre{Autre}

\contenu{
\texte{
Soit $u_{n,k}$ le reste de la division du $n$ par $k$.
Quelle est la limite de $\frac1n\sum_{k=1}^n\frac{u_{n,k}}k$~?
}
\reponse{
$\frac{u_{n,k}}k = \frac nk - \bigl[\frac nk\bigr]$, donc
$v_n = \frac1n\sum_{k=1}^n\frac{u_{n,k}}k$ est une somme de Riemann pour
l'intégrale $I =  \int_{t=0}^1 \Bigl(\frac1t - \bigl[\frac1t\bigr]\Bigr)\,d t$.
La fonction $\varphi$ : $t \mapsto\frac1t - \bigl[\frac1t\bigr]$ est Riemann-intégrable
sur $[0,1]$, donc $v_n\to I$ lorsque $n\to\infty$.

Calcul de~$I$~: $I_n =  \int_{t=1/n}^1 \Bigl(\frac1t - \bigl[\frac1t\bigr]\Bigr)\,d t
                     = \ln n - \sum_{k=1}^n \int_{t=\frac1{k+1}}^{\frac1k}k\,d t
                     = \ln n - \sum_{k=1}^n\frac1{k+1} \to 1-\gamma = I$ lorsque $n\to\infty$.
}
}
