\uuid{NY5H}
\exo7id{5692}
\auteur{rouget}
\datecreate{2010-10-16}
\isIndication{false}
\isCorrection{true}
\chapitre{Série numérique}
\sousChapitre{Autre}

\contenu{
\texte{
Soit $(u_n)_{n\in\Nn}$ une suite décroissante de nombres réels strictement positifs telle que la série de terme général $u_n$ converge. Montrer que $u_n\underset{n\rightarrow+\infty}{=}o\left(\frac{1}{n}\right)$. Trouver un exemple de suite $(u_n)_{n\in\Nn}$ de réels strictement positifs telle que la série de terme général $u_n$ converge mais telle que la suite de terme général $nu_n$ ne tende pas vers $0$.
}
\reponse{
Il faut vérifier que $nu_n\underset{n\rightarrow+\infty}{\rightarrow}0$. Pour $n\in\Nn$, posons $S_n =\sum_{k=0}^{n}u_k$. Pour $n\in\Nn$, on a

\begin{align*}\ensuremath
0<(2n)u_{2n}&=2(\underbrace{u_{2n}+\ldots+u_{2n}}_{n})\leqslant2\sum_{k=n+1}^{2n}u_k\;(\text{car la suite}\;u\;\text{est décroissante})\\
 &= 2(S_{2n} - S_n).
\end{align*}

Puisque la série de terme général $u_n$ converge, $\lim_{n \rightarrow +\infty}2(S_{2n} - S_n)=0$ et donc $\lim_{n \rightarrow +\infty}(2n)u_{2n}=0$.

Ensuite, $0 < (2n+1)u_{2n+1}\leqslant(2n+1)u_{2n}=(2n)u_{2n}+u_{2n}\underset{n\rightarrow+\infty}{\rightarrow}0$. Donc les suites des termes de rangs pairs et impairs extraites de la suite $(nu_n)_{n\in\Nn}$ convergent et ont même limite à savoir $0$. On en déduit que $\lim_{n \rightarrow +\infty}nu_n=0$ ou encore que $u_n\underset{n\rightarrow+\infty}{=}o\left(\frac{1}{n}\right)$.

Contre exemple avec $u$ non monotone. Pour $n\in\Nn$, on pose $u_n=\left\{
\begin{array}{l}
0\;\text{si}\;n=0\\
\rule[-4mm]{0mm}{10mm}\frac{1}{n}\;\text{si}\;n\;\text{est un carré parfait non nul}\\
0\;\text{sinon}
\end{array}
\right.$. La suite $u$ est positive et $\sum_{n=0}^{+\infty}u_n=\sum_{p=1}^{+\infty}\frac{1}{p^2}<+\infty$. Pourtant, $p^2u_{p^2}=1\underset{p\rightarrow+\infty}{\rightarrow}1$ et la suite $(nu_n)$ admet une suite extraite convergeant vers $1$. On a donc pas $\lim_{n \rightarrow +\infty}nu_n=0$.
}
}
