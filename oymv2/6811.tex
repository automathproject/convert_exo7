\uuid{jxnR}
\exo7id{6811}
\auteur{gijs}
\datecreate{2011-10-16}
\isIndication{false}
\isCorrection{false}
\chapitre{Théorème des résidus}
\sousChapitre{Théorème des résidus}

\contenu{
\texte{
Soit $a$ un réel tel que $0 \le a < 1$.
}
\begin{enumerate}
    \item \question{Démontrer que les intégrales $%  
I(a) =
\int_0^{+\infty} \frac{\sinh(ax)}{\sinh x} \,dx$ et
$%  
J(a) = \int_0^{+\infty} \frac{\cosh(ax)}{\cosh x}
\,dx$ sont convergentes ($\sinh$ et $\cosh$ désignent
les sinus et cosinus hyperboliques).}
    \item \question{Soit $\epsilon$ et $R$ des réels tels que $0<\epsilon <
\frac\pi2 < R$, soit $f(z)$ la fonction $f(z) =
\frac{e^{az}}{e^z - e^{-z}}$ et soit
$K_{\epsilon,R} \subset \Cc$ le compact obtenu en \^otant
la demi-boule ouverte de centre 0 et de rayon $\epsilon$ du
rectangle de sommets $R$, $R + i\frac\pi2$, $-R +
i\frac\pi2$, $-R$.
  \begin{enumerate}}
    \item \question{Démontrer que ${ \lim_{R\to
+\infty}}\int_\gamma f(z)\,dz = 0$ lorsque $\gamma$ est
(i) le c\^oté du rectangle joignant $R$ à $R +
i\frac\pi2$ et (ii) le c\^oté du rectangle joignant
$-R + i\frac\pi2$ à $-R$.}
    \item \question{Calculer le résidu de $f$ en 0.}
    \item \question{Du calcul de l'intégrale de $f$ le long du bord 
orienté de $K_{\epsilon,R}$ et de sa limite quand
$\epsilon
\to 0$ et
$R
\to +\infty$, déduire $J(a)$ et $I(a)$.}
\end{enumerate}
}
