\uuid{ogqd}
\exo7id{2120}
\auteur{debes}
\datecreate{2008-02-12}
\isIndication{false}
\isCorrection{true}
\chapitre{Ordre d'un élément}
\sousChapitre{Ordre d'un élément}

\contenu{
\texte{
Soit $G$ un groupe et $H,K$ deux sous-groupes de $G$.\smallskip

(a) Montrer que $H\cup K$ est un sous-groupe de $G$ si et seulement si $H<K$ ou $K<H$.
\smallskip

(b) Montrer qu'un groupe ne peut \^etre la r\'eunion de deux sous-groupes propres.
}
\reponse{
(a) Supposons que $H \cup K$ soit un sous-groupe de $G$ et que $H$ ne soit pas inclus dans
$K$, c'est-\`a-dire, qu'il existe $h\in H$ tel que $h\notin K$. Montrons que $K\subset H$.
Soit $k\in K$ quelconque. On a $hk \in H\cup K$. Mais $hk\notin K$ car sinon $h=(hk)k^{-1} 
\in K$. D'o\`u $hk\in H$ et donc $k=h^{-1}(hk)\in H$.
\smallskip

(b) d\'ecoule imm\'ediatement de (a).
}
}
