\uuid{rTQd}
\exo7id{5828}
\auteur{rouget}
\datecreate{2010-10-16}
\isIndication{false}
\isCorrection{true}
\chapitre{Conique}
\sousChapitre{Quadrique}

\contenu{
\texte{
Former l'équation de la surface de révolution $(\mathcal{S})$ engendrée par la rotation de la droite $(\mathcal{D})$ $\left\{
\begin{array}{l}
x=z+2\\
y=2z+1
\end{array}
\right.$ autour de la droite $(\Delta)$ d'équations $x = y = z$. Quelle surface obtient-on ?
}
\reponse{
Soit $A(a,b,c)$ un point quelconque de l'espace $E_3$.

Déterminons un système d'équation du cercle $(C_A)$ d'axe $(\Delta)$ d'équations $x=y=z$ passant par $A$.

Ce cercle est par exemple l'intersection du plan passant par $A$ de vecteur normal $(1,1,1)$ et de la sphère de centre $O$ et de rayon $OA$.

Un système d'équations de $(C_A)$ est $\left\{
\begin{array}{l}
x+y+z=a+b+c\\\
x^2+y^2+z^2=a^2+b^2+c^2
\end{array}
\right.$.

Déterminons alors une équation cartésienne de la surface $\mathcal{S}$. Une condition nécessaire et suffisante pour qu'un point $M(x,y,z)$ soit un point de $(\mathcal{S})$ est $(C_M)\cap(\mathcal{D})\neq\varnothing$. Donc

\begin{align*}
M\in(\mathcal{S})&\Leftrightarrow\exists(\alpha,\beta,\gamma)\in\Rr^3/\;\left\{
\begin{array}{l}
x+y+z=\alpha+\beta+\gamma\\
x^2+y^2+z^2=\alpha^2+\beta^2+\gamma^2\\
\alpha=\gamma+2\\
\beta=2\gamma+1
\end{array}
\right.\Leftrightarrow\exists(\alpha,\beta,\gamma)\in\Rr^3/\;\left\{
\begin{array}{l}
\alpha=\gamma+2\\
\beta=2\gamma+1\\
x+y+z=\gamma+2+2\gamma+1+\gamma\\
x^2+y^2+z^2=\alpha^2+\beta^2+\gamma^2
\end{array}
\right. \\
&\Leftrightarrow\exists(\alpha,\beta,\gamma)\in\Rr^3/\;\left\{
\begin{array}{l}
\alpha=\gamma+2\\
\beta=2\gamma+1\\
\gamma=\frac{1}{4}(x+y+z-3)\\
x^2+y^2+z^2=\alpha^2+\beta^2+\gamma^2
\end{array}
\right.\\
 &\Leftrightarrow x^2+y^2+z^2 =\left(\frac{1}{4}(x+y+z-3)+2\right)^2 +\left(\frac{2}{4}(x+y+z-3)+1\right)^2 +\left(\frac{1}{4}(x+y+z-3)\right)^2\\
 &\Leftrightarrow16(x^2+y^2+z^2) =(x+y+z+5)^2+4(x+y+z-1)^2+(x+y+z-3)^2 \\
 &\Leftrightarrow16(x^2+y^2+z^2) = 6(x+y+z)^2-2(x+y+z) + 38\\
 &\Leftrightarrow5(x^2+y^2+z^2) -6(xy+yz+zx)-(x+y+z)-19 = 0.
\end{align*}

\begin{center}
\shadowbox{
Une équation cartésienne de $(\mathcal{S})$ est $5(x^2+y^2+z^2) -6(xy+yz+zx)-(x+y+z)-19 = 0$.
}
\end{center}

La matrice de la forme quadratique $(x,y,z)\mapsto5(x^2+y^2+z^2) -6(xy+yz+zx)$ dans la base canonique de $\Rr^3$ est $\left(
\begin{array}{ccc}
5&-3&-3\\
-3&5&-3\\
-3&-3&5
\end{array}
\right)$. Ses valeurs propres sont $8$, valeur propre d'ordre $2$ associée au plan d'équation $x+y+z=0$ et $-1$ valeur propre d'ordre $1$ associé à la droite d'équation. Dans le repère $\left(O,\overrightarrow{e_1},\overrightarrow{e_2},\overrightarrow{e_3}\right)$ où $\overrightarrow{e_1}=\frac{1}{\sqrt{2}}(1,-1,0)$, $\overrightarrow{e_2}=\frac{1}{\sqrt{6}}(1,1,-2)$ et $\overrightarrow{e_3}=\frac{1}{\sqrt{3}}(1,1,1)$ 

\begin{center}
$M\in(\mathcal{S})\Leftrightarrow8x'^2+8y'^2-z'^2-\sqrt{3}z'-19=0\Leftrightarrow8\left(x'-\frac{\sqrt{3}}{16}\right)^28y'^2-z'^2=19+\frac{3}{32}$.
\end{center}

La surface $(\mathcal{S})$ est un hyperboloïde à une nappe.
}
}
