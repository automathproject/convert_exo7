\uuid{6201}
\auteur{queffelec}
\datecreate{2011-10-16}
\isIndication{false}
\isCorrection{false}
\chapitre{Compacité}
\sousChapitre{Compacité}

\contenu{
\texte{
Soit $E$ l'ensemble des suites infinies de nombres réels $x=(x_1,x_2,\cdots)$
à valeurs $0$ ou $1$.
Si $x$ et $y$ sont deux éléments de $E$, on pose
$$d(x,y)=\sup_{k\geq1}({1\over k}\vert x_k-y_k\vert)$$
}
\begin{enumerate}
    \item \question{Montrer que $d$ est une distance sur $E$.}
    \item \question{Soit $\varepsilon>0$; montrer qu'il existe une partie finie $E_\varepsilon$
de $E$ qui possède la propriété suivante : les boules fermées de rayon 
$\varepsilon$ centrées en un point de $E_\varepsilon$ recouvrent $E$.}
    \item \question{Montrer que $E$ est compact.}
\end{enumerate}
}
