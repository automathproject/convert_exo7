\uuid{7002}
\auteur{blanc-centi}
\datecreate{2015-07-04}
\isIndication{false}
\isCorrection{true}
\chapitre{Equation différentielle}
\sousChapitre{Résolution d'équation différentielle du premier ordre}

\contenu{
\texte{
\
}
\begin{enumerate}
    \item \question{Montrer que toute solution sur $\R$ de $y'+e^{x^2}y=0$ tend vers 0 en $+\infty$.}
\reponse{Notons $A(x) = \int_0^xe^{t^2}\,\dd t$, une primitive de $e^{x^2}$.
On ne sait pas expliciter cette primitive.
Les solutions de $y'+e^{x^2}y=0$ s'écrivent 
$f(x)=\lambda e^{-A(x)}$. 

Si $x\ge 1$, on a par positivité de l'intégrale $A(x) = \int_0^xe^{t^2}\,\dd t\ge 0$ et 
comme $e^{t^2} \ge 1$ alors 
$$A(x) = \int_0^xe^{t^2}\,\dd t \ge \int_0^x 1 \,\dd t = x$$
En conséquence :
$$0 \le e^{-A(x)} \le e^{-x}$$
Ainsi, 
$$0\le |f(x)|\le |\lambda|e^{-x}$$ et $f(x)\xrightarrow[x\to +\infty]{}0$.}
    \item \question{Montrer que toute solution sur $\R$ de $y''+e^{x^2}y=0$ est bornée.
(\emph{Indication :} étudier la fonction auxiliaire $u(x)=y(x)^2+e^{-x^2}y'(x)^2$.)}
\reponse{Supposons que $y$ vérifie sur $\R$ l'équation, et 
posons $u(x)=y(x)^2+e^{-x^2}y'(x)^2$. La fonction $u$ est à valeurs positives, 
dérivable, et 
$$u'(x) = 2y'(x)y(x)+e^{-x^2}2y''(x)y'(x)-2xe^{-x^2}y'(x)^2$$
en utilisant que $e^{-x^2}y''(x) = -y(x)$ 
(car $y$ est solution de l'équation différentielle) on obtient :
$$u'(x)=-2xe^{-x^2}y'^2.$$ 

Ainsi la fonction $u$ est croissante sur $]-\infty;0[$ 
et décroissante sur $]0;+\infty[$, donc pour tout $x\in\R$, $0 \le u(x)\le u(0)$. 
Or $y^2(x)\le u(x)$ par construction, donc
$$\forall x\in\R,\qquad |y(x)|\le \sqrt{u(0)}=\sqrt{y(0)^2+y'(0)^2}$$}
\end{enumerate}
}
