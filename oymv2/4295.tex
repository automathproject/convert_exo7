\uuid{oALd}
\exo7id{4295}
\auteur{quercia}
\datecreate{2010-03-12}
\isIndication{false}
\isCorrection{true}
\chapitre{Calcul d'intégrales}
\sousChapitre{Intégrale impropre}

\contenu{
\texte{

}
\begin{enumerate}
    \item \question{Soit $f : {]0,{+\infty}[} \to \R$ une fonction continue telle que
$\begin{cases} f(x) \to \ell \cr f(x) \text{ si } x\to{0^+}\to L \text{ si } x\to{+\infty} .\cr\end{cases}$

Pour $a > 0$, établir la convergence et calculer la valeur de
$ \int_{t=0}^{+\infty} \frac {f(at) - f(t)}t \,d t$.}
\reponse{$ \int_{t=x}^y \frac {f(at)}t \,d t =  \int_{t=ax}^{ay} \frac {f(t)}t \,d t  \Rightarrow 
  \int_{t=x}^y \frac {f(at)-f(t)}t \,d t =  \int_{t=ax}^x \frac {f(t)}t \,d t
 + \int_{t=y}^{ay} \frac {f(t)}t \,d t$.

On obtient $ \int_{t=0}^{+\infty} \frac {f(at) - f(t)}t \,d t = (L-\ell)\ln a$.}
    \item \question{Application : Calculer $ \int_{t=0}^1 \frac{t-1}{\ln t}\,d t$.}
\reponse{$I =  \int_{t=0}^{+\infty} \frac{e^{-t}-e^{-2t}}t\,d t = \ln 2$.}
\end{enumerate}
}
