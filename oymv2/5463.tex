\uuid{5463}
\auteur{rouget}
\datecreate{2010-07-10}
\isIndication{false}
\isCorrection{true}
\chapitre{Calcul d'intégrales}
\sousChapitre{Autre}

\contenu{
\texte{
Déterminer les fonctions $f$ continues sur $[0,1]$ vérifiant $\left|\int_{0}^{1}f(t)\;dt\right|=\int_{0}^{1}|f(t)|\;dt$.
}
\reponse{
Si $\int_{0}^{1}f(t)\;dt\geq0$,

\begin{align*}\ensuremath
\left|\int_{0}^{1}f(t)\;dt\right|=\int_{0}^{1}|f(t)|\;dt&\Leftrightarrow\int_{0}^{1}f(t)\;dt=\int_{0}^{1}|f(t)|\;dt\Leftrightarrow
\int_{0}^{1}(|f(t)|-f(t))\;dt=0\\
 &\Leftrightarrow|f|-f=0\;(\mbox{fonction continue positive d'intégrale nulle})\\
 &\Leftrightarrow f=|f|\Leftrightarrow f\geq0.
\end{align*}

Si $\int_{0}^{1}f(t)\;dt\leq0$, alors $\int_{0}^{1}-f(t)\;dt\geq0$ et d'après ce qui précède, $f$ est solution si et seulement si $-f=|-f|$ ou encore $f\leq0$.

En résumé, $f$ est solution si et seulement si $f$ est de signe constant sur $[0,1]$.
}
}
