\uuid{7088}
\auteur{megy}
\datecreate{2017-01-21}
\isIndication{true}
\isCorrection{false}
\chapitre{Géométrie affine euclidienne}
\sousChapitre{Géométrie affine euclidienne du plan}

\contenu{
\texte{
Soit $\mathcal D$ une droite, $A$ et $B$ deux points distincts  n'appartenant pas à cette droite, et $A'$ le symétrique de $A$ par rapport à $\mathcal D$. On suppose que $A'B$ n'est pas parallèle à $\mathcal D$. Construire à la règle seule le symétrique $B'$ de $B$ par rapport à $\mathcal D$.
}
\indication{Considérer le quadrilatère $ABB'A'$ et ses diagonales : elles se croisent sur la droite.}
}
