\uuid{6581}
\auteur{hueb}
\datecreate{2011-10-16}
\isIndication{false}
\isCorrection{false}
\chapitre{Fonction holomorphe}
\sousChapitre{Fonction holomorphe}

\contenu{
\texte{
Montrer que
$$f(x+iy) = x^2 + i y^3$$
n'est holomorphe en aucun point bien que les
equations de Cauchy-Riemann soient vérifiées à l'origine, même
sur une parabole que l'on précisera.
}
}
