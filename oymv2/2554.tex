\uuid{2554}
\auteur{tahani}
\datecreate{2009-04-01}

\contenu{
\texte{
Etudier les extrémas locaux
et globaux des fonctions suivantes:
}
\begin{enumerate}
    \item \question{$f(x,y)=x^2+xy+y^2+\frac{1}{4}x^3$}
    \item \question{$f(x,y)=x^2y-x^2/2-y^2$}
    \item \question{$f(x,y)=x^4+y^4-2(x-y)^2$}
    \item \question{$f(x,y)=\sin^2x-\sh^2y$}
    \item \question{$f(x,y)=x^3+y^3$}
    \item \question{$f(x,y)=y^2-3x^2y+2x^4$}
\reponse{
Soit $f(x,y)=x^2+xy+y^2+\frac{1}{4}x^3$, calculons la jacobienne
de $f$: $$Df(x,y)=(2x+y+\frac{3}{4}x^2, x+2y).$$ Les points
critiques de $f$ v\'erifient $Df(x,y)=0$ et par cons\'equent
v\'erifient les \'equations $2x+y+\frac{3}{4}x^2=0$ et $x+2y=0$.
Par cons\'equent $f$ admet deux points critiques $(0,0)$ et
$(-2,1)$. Pour savoir si ces points critiques sont des extr\'emums
de $f$, il faut \'etudier la hessienne de $f$:
$$Hess_f(x,y)= \left(
\begin{array}{cc}
2+\frac{3}{2}x & 1 \\
1 & 2
\end{array}\right)$$ et ses valeurs propres.
Au point $(0,0)$, $$Hess_f(0,0)= \left(
\begin{array}{cc}
2 & 1 \\
1 & 2
\end{array}\right)$$ a pour polynôme caractéristique
$P(\lambda)=(1-\lambda)(3-\lambda)$ et ses deux valeurs propres
sont strictement positive. La fonction $f$ admet donc un minimum
local au point $(0,0)$. Ce minimum n'est pas globale car
$f(0,0)=0$ et $f$ prend des valeurs n\'egatives pour $y=0$ et $x$
qui tend vers $-\infty$.\par
Au point $(-2,1)$, $$Hess_f(0,0)=
\left(
\begin{array}{cc}
-1 & 1 \\
1 & 2
\end{array}\right)$$ a pour polynôme caractéristique
$P(\lambda)=\lambda^2-\lambda-3$. On peut alors soit calculer les
valeurs propres soit remarquer que le determinant de la hessienne
est \'egal au produit des deux valeurs propres (ici $-3$) et
celles-ci sont donc non nulles et de signe contraire. Par
cons\'equent le point $(-2,1)$ est point selle de $f$. ce n'est
pas un extremum.
}
\end{enumerate}
}
