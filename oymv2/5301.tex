\uuid{5301}
\auteur{rouget}
\datecreate{2010-07-04}
\isIndication{false}
\isCorrection{true}
\chapitre{Arithmétique}
\sousChapitre{Arithmétique de Z}

\contenu{
\texte{
Pour $n\in\Nn$, on pose $F_n=2^{2^n}+1$ (nombres de \textsc{Fermat}). Montrer que les nombres de Fermat sont deux à deux premiers entre eux.
}
\reponse{
Soient $n$ et $m$ deux entiers naturels tels que $n<m$. Posons $m=n+k$ avec $k>0$. On note que 

$$F_m=2^{2^{n+k}}+1=(2^{2^n})^{2^k}+1=(F_n-1)^{2^k}+1.$$

En développant l'expression précédente par la formule du binôme de \textsc{Newton} et en tenant compte du fait que $2^k$ est pair puisque $k$ est strictement positif, on obtient une expression de la forme $q.F_n+1+1=q.F_n+2$.

Le P.G.C.D. de $F_n$ et $F_m$ doit encore diviser $F_m-q.F_n=2$ et vaut donc $1$ ou $2$. Enfin, puisque $2^n$ et $2^m$ sont strictement positifs, $F_n$ et $F_m$ sont impairs et leur P.G.C.D. vaut donc $1$ (ce résultat redémontre l'existence d'une infinité de nombres premiers).
}
}
