\uuid{7686}
\auteur{mourougane}
\datecreate{2021-08-11}

\contenu{
\texte{

}
\begin{enumerate}
    \item \question{Une surface régulière de $\Rr^3$ est dite réglée si elle admet des paramétrages
 locaux de la forme $F(t,s)=c(t)+sv(t)$ pour $t,s)\in I\times J$
où $c$ est une courbe régulière de $\Rr^3$ paramétrée sur l'intervalle $I$ de $\Rr$,
et $v~:I\to\Rr^3_{ev}$ une application de classe $\mathcal{C}^\infty$ avec $v(t)$ et $\dot{c}(t)$ 
partout indépendant. 
Montrer que si $J$ est un petit intervalle autour de $0$, $F$ est alors un paramétrage régulier.}
    \item \question{Montrer que la courbure de Gauss d'une surface réglée est négative en tout point.}
\end{enumerate}
}
