\uuid{u4Tm}
\exo7id{6115}
\auteur{queffelec}
\datecreate{2011-10-16}
\isIndication{false}
\isCorrection{false}
\chapitre{Espace topologique, espace métrique}
\sousChapitre{Espace topologique, espace métrique}

\contenu{
\texte{
On considère une suite $(x_n)$ de $[0,1]$
telle que $x_{n+1}-x_n$ tend vers $0$.
}
\begin{enumerate}
    \item \question{Montrer que l'ensemble $A$ de ses valeurs d'adhérence est un intervalle
fermé de $[0,1]$.}
    \item \question{On suppose de plus que cette suite est une suite récurrente i.e. définie par $
x_{n+1}=f(x_n)$ où $f$ est continue de $[0,1]$ dans lui-même, et un point initial
$x_0\in [0,1]$. Montrer alors que la suite converge (on commencera par remarquer
que si
$x\in A$, alors $x=f(x)$, et que si $x_m\in A$ pour un indice $m$, alors la
suite converge.)}
    \item \question{Soit $x=(x_n)$ une suite de $l^{\infty}$; montrer que l'ensemble des valeurs
d'adhé\-rence de la suite $y$ de terme général $y_n={{x_1+x_2+\cdots+x_n}\over
n}$ est un intervalle. En déduire que l'application $f$ de $l^{\infty}$ dans
lui-même qui associe $y$ à $x$, n'est pas bijective.}
\end{enumerate}
}
