\uuid{RJQj}
\exo7id{3243}
\auteur{quercia}
\datecreate{2010-03-08}
\isIndication{false}
\isCorrection{true}
\chapitre{Polynôme, fraction rationnelle}
\sousChapitre{Racine, décomposition en facteurs irréductibles}

\contenu{
\texte{
Soit $P \in \R_n[X]\setminus\{0\}$.

Pour $x\in\R$ on note $V(x)$ le
nombre de changements de signe dans la suite $(P(x),P'(x),\dots,P^{(n)}(x))$
en convenant de retirer les termes nuls.
Soient $\alpha<\beta$ deux r{\'e}els non racines de~$P$. Montrer que le nombre
de racines de~$P$ dans $[\alpha,\beta]$, compt{\'e}es avec leur ordre de multiplicit{\'e},
a m{\^e}me parit{\'e} que~$V(\alpha)-V(\beta)$ et que $V(\alpha)-V(\beta)\ge 0$.
}
\reponse{
$V(\alpha)$ est pair si et seulement si $P(\alpha)$ et $P^{(n)}(\alpha)$
ont m{\^e}me signe, de m{\^e}me pour $V(\beta)$. Comme $P^{(n)}(\alpha) = P^{(n)}(\beta)$
on en d{\'e}duit que $V(\alpha)-V(\beta)$ est pair si et seulement si $P(\alpha)$
et $P(\beta)$ ont m{\^e}me signe, donc si et seulement si $P$ a un nombre pair
de racines dans $[\alpha,\beta]$.

D{\'e}croissance de~$V$~: $V$ est constant sur tout intervalle ne contenant
aucune racine de $P,P',\dots,P^{(n-1)}$. Consid{\'e}rons $x_0\in{[\alpha,\beta[}$
tel que $P^{(k)}(x_0) \ne 0$, $P^{(k+1)}(x_0) = \dots = P^{(\ell-1)}(x_0) = 0$
et $P^{(\ell)}(x_0) \ne 0$. Alors pour $x$ proche de $x_0$ avec $x>x_0$, $P^{(k)}(x)$
a m{\^e}me signe que $P^{(k)}(x_0)$ et $P^{(k+1)}(x),\dots,P^{(\ell)}(x)$ ont
m{\^e}me signe que $P^{(\ell)}(x_0)$ donc les nombres de changements de signe dans
les sous-suites $(P^{(k)}(x),\dots,P^{(\ell)}(x))$ et $(P^{(k)}(x_0),\dots,P^{(\ell)}(x_0))$
sont {\'e}gaux. De m{\^e}me si $P(x_0) = \dots = P^{(\ell-1)}(x_0) = 0$ et
$P^{(\ell)}(x_0) \ne 0$. Ceci prouve que $V(x_0^+) = V(x_0)$ pour tout
$x_0\in{[\alpha,\beta[}$.

On consid{\`e}re {\`a} pr{\'e}sent $x_0\in{]\alpha,\beta]}$ tel que
$P^{(k)}(x_0) \ne 0$, $P^{(k+1)}(x_0) = \dots = P^{(\ell-1)}(x_0) = 0$
et $P^{(\ell)}(x_0) \ne 0$. Alors pour $x$ proche de $x_0$ avec $x<x_0$
la sous-suite $(P^{(k)}(x),\dots,P^{(\ell)}(x))$ a $\ell-k-1$ changements de
signe si $P^{(k)}(x_0)$ et  $P^{(\ell)}(x_0)$ ont m{\^e}me signe, $\ell-k$
changements de signe sinon tandis que la sous-suite $(P^{(k)}(x_0),\dots,P^{(\ell)}(x_0))$
en a un ou z{\'e}ro. De m{\^e}me, si $P(x_0) = \dots = P^{(\ell-1)}(x_0) = 0$ et
$P^{(\ell)}(x_0) \ne 0$ on trouve $\ell$ changements de signe pour
$(P(x),\dots,P^{(\ell)}(x))$ et z{\'e}ro pour $(P(x_0),\dots,P^{(\ell)}(x_0))$
donc dans tous les cas $V(x_0^-) \ge V(x_0)$. Ceci ach{\`e}ve la d{\'e}monstration.
}
}
