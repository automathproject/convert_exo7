\uuid{539}
\auteur{bodin}
\datecreate{2001-11-01}

\contenu{
\texte{
On consid\`ere la fonction $ f : \R \longrightarrow \R$ d\'efinie
par
$$f (x) = \frac{x^{3}}{9} + \frac{2 x}{3} + \frac{1}{9}$$
et on d\'efinit la suite $(x_{n})_{n \geq 0}$ en posant $x_{0} =
0$ et $x_{n + 1} = f (x_{n})$ pour $n \in \N.$
}
\begin{enumerate}
    \item \question{Montrer que l'\'equation $x^{3} - 3 x + 1 = 0$ poss\`ede une
solution unique $\alpha \in ]0, 1\slash 2[.$}
\reponse{La fonction polynomiale $P (x) := x^{3} - 3 x + 1$ est continue et
d\'erivable sur $\R$ et sa d\'eriv\'ee est $P' (x) = 3 x^{2} - 3,$
qui est strictement n\'egative sur $]- 1 , + 1[.$ Par cons\'equent
$P$ est strictement d\'ecroissante sur $]- 1 , + 1[.$ Comme $P (0)
= 1 > 0$ et $P (1\slash 2) = - 3 \slash 8 < 0$ il en r\'esulte
gr\^ace au th\'eor\`eme des valeurs interm\'ediaires qu'il existe
un r\'eel unique $\alpha \in ]0,1\slash 2[$ tel que $P (\alpha) =
0.$}
    \item \question{Montrer que l'\'equation $f (x) = x$ est \'equivalente \`a
l'\'equation $x^{3} - 3 x + 1 = 0$ et en d\'eduire que $\alpha $
est l'unique solution de l'\'equation $f (x) = x$ dans
l'intervalle $[0 , 1\slash 2].$}
\reponse{Comme $ f (x) - x = (x^{3} - 3 x + 1) \slash 9$ il en
r\'esulte que $\alpha$ est l'unique solution de l'\'equation $f
(x) = x$ dans $]0,1\slash 2[.$}
    \item \question{Montrer que  la fonction $f$ est croissante sur $\R^{+}$ et que $f (\R^{+}) \subset
\R^{+}$. En
d\'eduire que la suite $(x_{n})$ est croissante.}
\reponse{Comme $f' (x) = (x^{2} +2)\slash 3 > 0$ pour tout $x \in \R,$ on
en d\'eduit que $f$ est strictement croissante sur $\R.$ Comme $f
(0) = 1 \slash 9$ et $\lim_{x \to + \infty} f (x) = + \infty,$ on
en d\'eduit que $ f (\R^{+}) = [1\slash 9 , + \infty[.$ Comme
$x_{1} = f (x_{0}) = 1 \slash 9  >  0$ alors $x_1 > x_0=0$ ;   $f$ étant
strictement croissante sur $\R^{+},$ on en d\'eduit par
r\'ecurrence que $x_{n + 1} > x_{n}$ pour tout $n \in \N$ ce qui
prouve que la suite $(x_{n})$ est croissante.}
    \item \question{Montrer que
$f (1\slash 2) < 1 \slash 2$ et en d\'eduire que $ 0 \leq  x_{n} <
1 \slash 2$ pour tout $ n \geq 0.$}
\reponse{Un calcul simple montre que  $f (1 \slash 2) < 1
\slash 2.$ Comme $0 = x_{0} < 1 \slash 2$ et que $f$ est
croissante on en d\'eduit par r\'ecurrence que $ x_{n} < 1 \slash
2$ pour tout $n \in \N$ (en effet si $x_n < 1/2$ alors 
$x_{n+1} = f(x_n) < f(1/2) < 1/2$).}
    \item \question{Montrer que la suite
$(x_{n})_{n \geq 0}$ converge vers $\alpha.$}
\reponse{D'apr\`es les questions
pr\'ec\'edentes, la suite $(x_{n})$  est croissante et major\'ee,
elle converge donc vers un nombre r\'eel $\ell \in ]0, 1 \slash 2].$
De plus comme $x_{n + 1} = f (x_{n})$ pour tout $n \in \N,$ on en
d\'eduit par continuit\'e de $f$ que $\ell  = f (\ell).$ Comme $f
(1 \slash 2) < 1\slash 2,$ On en d\'eduit que $\ell \in ]0, 1
\slash 2[$ et v\'erifie l'\'equation $f (\ell) = \ell.$ D'apr\`es
la question 2, on en d\'eduit que $\ell = \alpha$ et donc $(x_{n})
$ converge vers $\alpha.$}
\indication{Pour la premi\`ere question : attention on ne demande pas de calculer $\alpha$ !
L'existence vient du th\'eor\`eme des valeurs interm\'ediaires. L'unicit\'e vient du fait que la fonction est strictement croissante.
  

Pour la derni\`ere question : il faut d'une part montrer que $(x_n)$ converge et on note $\ell$ sa limite
et d'autre part il faut montrer que $\ell = \alpha$.}
\end{enumerate}
}
