\uuid{gtXg}
\exo7id{7226}
\auteur{megy}
\datecreate{2021-03-01}
\isIndication{false}
\isCorrection{false}
\chapitre{Fonction holomorphe}
\sousChapitre{Fonction holomorphe}

\contenu{
\texte{
Soit \(\displaystyle{f(z)=\sum_{n\geqslant 0}a_nz^n}\) une série entière de rayon de convergence \(R>0\) et telle que \(a_0\neq 0\). L'objectif de cet exercice est de démontrer que la fonction \(\frac{1}{f}\) est développable en série entière en \(0\).
}
\begin{enumerate}
    \item \question{On suppose que ceci est le cas et que \(\displaystyle{\frac{1}{f(z)}=\sum_{n=0}^{+\infty}b_nz^n}\). Quelle relation de récurrence vérifie la suite \((b_n)_{n\in \N}\)?}
    \item \question{Soit \((b_n)_{n\in \N}\) une suite vérifiant le relation de récurrence précédente. Montrer qu'il existe \(C>0\) tel que pour tout \(n\geqslant 0\), on a 
\[|b_n|\leqslant \frac{C^n}{|a_0|}.\]}
    \item \question{En déduire que \(\frac{1}{f}\) est développable en série entière en \(0\).}
\end{enumerate}
}
