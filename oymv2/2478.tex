\uuid{vgrK}
\exo7id{2478}
\auteur{matexo1}
\datecreate{2002-02-01}
\isIndication{false}
\isCorrection{true}
\chapitre{Déterminant, système linéaire}
\sousChapitre{Système linéaire, rang}

\contenu{
\texte{
\catcode`\@=11
\def\system#1{\left\{\null\,\vcenter{\openup1\jot\m@th
\ialign{\strut\hfil$##$&$##$\hfil&&\enspace$##$\enspace&
\hfil$##$&$##$\hfil\crcr#1\crcr}}\right.}
\catcode`\@=12

R{\'e}soudre
$$S_1\system{
        &x&+&\cosh a\ &y&+&\cosh 2a\ &z&=&&\cosh 3a\cr
        \cosh a\ &x&+&\cosh 2a\ &y&+&\cosh 3a\ &z&=&&\cosh 4a\cr
        \cosh 2a\ &x&+&\cosh 3a\ &y&+&\cosh 4a\ &z&=&&\cosh 5a\cr}$$
et $$       
S_2  \system{
        &x_1&+&&x_2&+&\ldots&&+&&x_n&=&&1\cr
        &x_1&+&2&x_2&+&\ldots&&+&n&x_n&=&&0\cr
        &x_1&+&2^2&x_2&+&\ldots&&+&n^2&x_n&=&&0\cr
        &   & &   &   & &\vdots&& &   &   & &&\cr
        &x_1&+&2^{n-1}&x_2&+&\ldots&&+&n^{n-1}&x_n&=&&0\cr}   
$$
}
\reponse{
$S_1$: si $a\neq0$, le syst{\`e}me est {\'e}quivalent {\`a}
$y=-1-2\cosh a\ x$ et $z=x+2\cosh a$

si $a=0$, il est {\'e}quivalent {\`a} $x+y+z=1$

\smallskip

$S_2$: On peut soustraire {\`a} chaque ligne la ligne pr{\'e}c{\'e}dente, puis
2 fois la pr{\'e}c{\'e}dente, etc\dots On obtient ainsi un syst{\`e}me triangulaire
cram{\'e}rien et apr{\`e}s bien des calculs la solution $x_k=(-1)^{k+1}C_n^k$.

Voici une solution plus astucieuse.
Soit $P(X)=x_1X+x_2X^2+\dots+x_nX^n$ et $T$ l'op{\'e}rateur $Q(x)\mapsto XQ'(X)$.
Le syst{\`e}me peut s'{\'e}crire $P(1)=1,\ (TP)(1)=0,\ (T^2P)(1)=0,\ldots,\ (T^{n-1}P)(1)=0$.
On en d{\'e}duit que $P'(1)=P''(1)=\ldots=P^{(n-1)}(1)=0$, donc $P$ est de la forme
$P(X)=1+\lambda(1-X)^n$, et $\lambda=-1$ car le terme constant de $P$ est nul.
Donc $x_k=(-1)^{k+1}C_n^k$.
}
}
