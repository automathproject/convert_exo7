\uuid{895}
\auteur{ridde}
\datecreate{1999-11-01}
\isIndication{false}
\isCorrection{false}
\chapitre{Espace vectoriel}
\sousChapitre{Définition, sous-espace}

\contenu{
\texte{
Montrer que $\left\{ (x, y, z) \in \Rr^3 /x + y + z = 0 \text{ et }
2x-y + 3z = 0\right\}$ est un sous-espace vectoriel de $\Rr^3$.
}
}
