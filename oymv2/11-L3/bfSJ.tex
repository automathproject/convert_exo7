\uuid{bfSJ}
\exo7id{6778}
\auteur{gijs}
\datecreate{2011-10-16}
\isIndication{false}
\isCorrection{false}
\chapitre{Champ de vecteurs}
\sousChapitre{Champ de vecteurs}

\contenu{
\texte{
Soient $M\subset \Rr^n$ et $N\subset \Rr^p$ deux sous-variétés de dimension $k$ et $\ell$
respectivement. Soit $F : M \to N$ une application
différentiable et soit $X$ un champ de vecteurs sur
$M$. Trouver un contre exemple pour l'énoncé : $$F(m)
= F(\widehat m) \Longrightarrow  TF(m)(X(m)) =
TF(\widehat m)(X(\widehat m))\ .$$ 
Rappel : $TF(m) \equiv
F'(m)$ est la ``dérivée'' de $F$ au point $m$.
}
}
