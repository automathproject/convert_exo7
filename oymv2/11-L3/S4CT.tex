\uuid{S4CT}
\exo7id{7699}
\auteur{mourougane}
\datecreate{2021-08-11}
\isIndication{false}
\isCorrection{true}
\chapitre{Sous-variété}
\sousChapitre{Sous-variété}

\contenu{
\texte{
On considère l'application
$$\begin{array}{ccc}
 F: \Rr ^2&\to&\Rr ^3\\ 
\left(\begin{array}{c}t \\ s\end{array}\right)
&\mapsto& 
\left(\begin{array}{c}1-t^2\\t(1-t^2)\\s\end{array}\right).
\end{array}$$
}
\begin{enumerate}
    \item \question{Déterminer des déplacements de l'espace $\Rr ^3$ qui conservent l'image de $F$.}
\reponse{Comme pour tout $t\in \Rr $, et tout $a_in\Rr $, $c(t+a)=t_{(0,0,a)}(c(t))$,
 l'image de $F$ est invariante par toutes les translations de vecteur parallèle à $\vec{k}$,
 le troisième vecteur de la base canonique.
 
 On remarque aussi que $F(-t,-s)$ s'obtient à partir de $F(t,s)$ par le demi-tour d'axe des abscisses. 
 Ce demi-tour conserve donc l'image de $F$.}
    \item \question{Montrer que l'image de $F$ est l'ensemble d'équation $y^2=x^2(1-x)$.}
\reponse{Soit $(u,v)\in \Rr ^2$,
 $$(1-t^2)^2(1-(1-t^2))=(1-t^2)^2t^2=[t(1-t^2)]^2.$$
 Donc, l'image de $F$ est incluse dans l'ensemble d'équation $y^2=x^2(1-x)$.
 
 Soit $(x,y,z)$ inclus dans l'ensemble d'équation $y^2=x^2(1-x)$.
 Si $x=0$, $y=0$ et $(x,y,z)=F(1,z)$. Si $x\not=0$, soit $s=z$ et $t=y/x$.
 On vérifie que $F(t,s)=(x,y,z)$.
 
 Donc, l'image de $F$ est l'ensemble d'équation $y^2=x^2(1-x)$.}
    \item \question{L'image de $F$ est-elle une surface régulière de $\Rr ^3$ ?}
\reponse{Au voisinage du point $(0,0,0)$ de l'image de $S$,
 la projection sur le plan $x=0$ donne a chaque point deux antécédents,
 la projection sur la plan $y=0$ aussi,
 et la projection sur la plan $z=0$ donne à chaque point soit une infinité, soit aucun antécédent.
 Par conséquent, au voisinage de $(0,0,0)$ l'image de $S$ n'est pas un graphe.
 L'image de $F$ n'est donc pas une surface régulière de $\Rr ^3$.}
\end{enumerate}
}
