\uuid{gCV1}
\exo7id{7648}
\auteur{mourougane}
\datecreate{2021-08-11}
\isIndication{false}
\isCorrection{false}
\chapitre{Sous-variété}
\sousChapitre{Sous-variété}

\contenu{
\texte{
On considère un cercle $\mathcal{C}_1$ de rayon $1$ qui glisse sur l'axe des $x$ du plan euclidien orienté $\Rr ^2$,
et pour tout nombre réel $r$ strictement positif le cercle $\mathcal{C}_r$ de rayon $r$, concentrique avec $\mathcal{C}_1$
et solidement attaché à $\mathcal{C}_1$.
}
\begin{enumerate}
    \item \question{Déterminer par un paramétrage, la trajectoire $T_r$, appelée ``cycloïde`` 
du point $M$ de coordonnées $(0,1-r)$ de $\mathcal{C}_r$ lorsque $\mathcal{C}_1$ glisse sur l'axe des $x$.}
    \item \question{Faire une ébauche de $T_r$ suivant la position de $r$ par rapport à $1$.}
    \item \question{$T_r$ est-elle une courbe régulière ?}
    \item \question{Calculer la longueur de $T_1$ quand $\mathcal{C}_1$ fait un tour.}
\end{enumerate}
}
