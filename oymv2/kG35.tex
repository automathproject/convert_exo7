\uuid{kG35}
\exo7id{1384}
\auteur{legall}
\datecreate{2003-10-01}
\isIndication{false}
\isCorrection{false}
\chapitre{Groupe, anneau, corps}
\sousChapitre{Algèbre, corps}

\contenu{
\texte{
Un automorphisme d'un corps $\mathbb{K} $ est une 
application bijective $\varphi $ de $\mathbb{K} $ dans lui-m\^eme
telle que $\varphi (1)=1,$ $\varphi (0)=0$ et, pour tout $a,b \in \mathbb{K} 
,$ on ait $\varphi (a+b)=\varphi (a)+\varphi (b)$ et $\varphi 
(ab)=\varphi (a)\varphi (b).$
}
\begin{enumerate}
    \item \question{Soit $\varphi $ un automorphisme de $\Rr .$ Montrer que 
l'application $x\mapsto \varphi (x)$ est croissante. En d\'eduire que 
l'identit\'e est le seul automorphisme de $\Rr $.}
    \item \question{Soit $\psi$ un automorphisme {\em continu} de $\Cc .$ Montrer 
$\psi (x)= x,$ pour tout $x\in \Rr .$ En d\'eduire tous les 
automorphismes {\em continus} de $\Cc$.}
\end{enumerate}
}
