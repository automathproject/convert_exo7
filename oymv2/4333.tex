\uuid{yYBe}
\exo7id{4333}
\auteur{quercia}
\datecreate{2010-03-12}
\isIndication{false}
\isCorrection{true}
\chapitre{Intégration}
\sousChapitre{Intégrale de Riemann dépendant d'un paramètre}

\contenu{
\texte{
On pose $I_n =  \int_0^{\pi/4} \tan^nt\,d t$.\par
}
\begin{enumerate}
    \item \question{Montrer que $I_n \to 0$  lorsque $n\to\infty$.}
\reponse{$I_n+I_{n+2} = \frac1{n+1}$.}
    \item \question{Calculer $I_n$ en fonction de~$n$.}
\reponse{$I_{2k} = \frac1{2k-1} - \frac1{2k-3} + \dots + \frac{(-1)^{k-1}}1 + (-1)^k\frac\pi4$,\par
         $I_{2k+1} = \frac1{2k} - \frac1{2k-2} + \dots + \frac{(-1)^{k-1}}2 - (-1)^k\ln\sqrt2$.}
    \item \question{Que peut-on en déduire~?}
\reponse{$\frac11-\frac13+\frac15 - \dots = \frac\pi4$ et $\frac11-\frac12+\frac13-\dots=\ln 2$.}
\end{enumerate}
}
