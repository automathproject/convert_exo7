\uuid{fnJh}
\exo7id{6359}
\auteur{queffelec}
\datecreate{2011-10-16}
\isIndication{false}
\isCorrection{false}
\chapitre{Autre}
\sousChapitre{Autre}

\contenu{
\texte{
Soit $E$ un espace de Banach et $t\to A(t)$ une application continue de
$\Rr$ dans ${\cal L}(E,E)$. On suppose
que $A$ est périodique de période $\omega$. Cela n'implique pas nécessairement
que les solutions de $\quad (1)\quad x'=A(t).x$ soient également
$\omega$-périodiques.
}
\begin{enumerate}
    \item \question{Dans le cas où $E$ est un espace de dimension $2$ et $A$ est une matrice
constante, donner une condition nécessaire et suffisante pour que les solutions
de $(1)$ soient $\omega$-périodiques.}
    \item \question{Dans le cas général, soit $R(t,a)$ le noyau résolvant associé à $(1)$.
\begin{enumerate}}
    \item \question{Montrer que $R(t+\omega,a+\omega)=R(t,a)$ pour tout $t$.}
    \item \question{Montrer que la  solution $x(t)$ de $(1)$  telle que $x(0)=x_0$ est
$\omega$-périodique si et seulement si 
$ R(\omega,0)x_0=x_0$.}
    \item \question{A quelle condition l'équation
$x'=A(t).x$ a-t-elle une solution $\omega$-périodique ?.}
\end{enumerate}
}
