\uuid{emy6}
\exo7id{3043}
\auteur{quercia}
\datecreate{2010-03-08}
\isIndication{false}
\isCorrection{true}
\chapitre{Logique, ensemble, raisonnement}
\sousChapitre{Relation d'équivalence, relation d'ordre}

\contenu{
\texte{
On d{\'e}finit sur $\R^2$ : $(x,y) \ll (x',y') \iff |x'-x| \le y'-y$.
}
\begin{enumerate}
    \item \question{V{\'e}rifier que c'est une relation d'ordre.}
    \item \question{Dessiner les ensembles des majorants et des minorants d'un couple $(a,b)$.}
    \item \question{L'ordre est-il total ?}
    \item \question{Soit $A = \{ (x,y) \in \R^2$ tq $x^2 + y^2 \le 1 \}$.
     D{\'e}terminer $\sup(A)$.}
\reponse{
Si $(a,b)$ majore $A$, alors $(a,b) \gg (\pm\frac 1{\sqrt2}, \frac 1{\sqrt2})$
     donc $(a,b) \gg (0,\sqrt2)$.
     \par
     R{\'e}ciproque : si $x^2+y^2 \le 1$, alors $(x+y)^2 + (x-y)^2 \le 2$, donc
     $y \pm x \le \sqrt2$, et $(x,y) \ll (0,\sqrt2)$.
     \par
     Finalement, $\sup(A) = (0,\sqrt2)$.
}
\end{enumerate}
}
