\uuid{2017}
\auteur{liousse}
\datecreate{2003-10-01}
\isIndication{false}
\isCorrection{false}
\chapitre{Géométrie affine dans le plan et dans l'espace}
\sousChapitre{Géométrie affine dans le plan et dans l'espace}

\contenu{
\texte{
Soient $D_1$, $D_2$ et $D_3$ trois droites concourrantes en $\Omega$ et soient  $P$, $P'$ et $P''$ trois plans tels que aucun ne contient aucune des 3 droites ci dessus.
   On peut alors d\'efinir les 9 points d'intersections   :
$P$ coupe $D_1$, $D_2$, $D_3$ en    $A$, $B$, $C$ ;
$P'$ coupe $D_1$, $D_2$, $D_3$ en    $A'$, $B'$, $C'$ ;
$P'$ coupe $D_1$  $D_2$, $D_3$ en    $A''$, $B''$, $C''$ ;

  On consid\`ere aussi les intersections suivantes :
$I=(AB')\cap (A'B)$ , $J=(AC')\cap (A'C)$ ,$K=(BC')\cap (B'C)$.

  Montrer que les droites $(A''K)$, $B''J)$ et $(C''I)$ sont parall\`elles ou concourrantes.
  ({\it Indication : utiliser  un bon rep\`ere affine}).
}
}
