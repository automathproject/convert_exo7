\uuid{2427}
\auteur{matexo1}
\datecreate{2002-02-01}
\isIndication{false}
\isCorrection{false}
\chapitre{Polynôme, fraction rationnelle}
\sousChapitre{Autre}

\contenu{
\texte{
On note $\R^n$ l'ensemble des $n$-uplets $(x_1, \ldots, x_n)$
de nombres r\'eels\,; $\R [X]$ l'ensemble des polyn\^omes \`a 
coefficients r\'eels en la variable X\,; $\R[X]_p$ le sous-ensemble
des polyn\^omes de degr\'e $\le p$\,; $\R(X)$ l'ensemble des
fractions rationnelles \`a coefficients r\'eels en la variable X\,;
$\R(X)_p$ le sous-ensemble des fractions rationnelles de 
degr\'e $\le p$\,; $C^k(\R)$ l'ensemble des fonctions r\'eelles
d\'efinies sur $\R$ et $k$ fois contin\^ument d\'erivables ($k
\ge 0$ entier)\,; $C^\infty(\R)$ l'ensemble des fonctions
ind\'efiniment d\'erivables sur $\R$.
}
\begin{enumerate}
    \item \question{Dot\'es des op\'erations
d'addition et de multiplication usuelles, lesquels de ces
ensembles sont des espaces vectoriels\,?}
    \item \question{Montrer que $\R[X]_p \subset
\R[X] \subset \R(X)$ et que $C^\infty(\R) \subset C^k(\R) \subset
C^0(\R)$,
et que ce sont des sous-espaces vectoriels.}
    \item \question{Si l'on identifie les
polyn\^omes et les fractions rationnelles aux fonctions
correspondantes, a-t-on
$\R[X] \subset C^\infty(\R)$ et $\R(X) \subset C^\infty(\R)$\,?}
\end{enumerate}
}
