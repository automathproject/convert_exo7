\uuid{5006}
\auteur{quercia}
\datecreate{2010-03-17}
\isIndication{false}
\isCorrection{true}
\chapitre{Courbes planes}
\sousChapitre{Courbes définies par une condition}

\contenu{
\texte{
Trouver les courbes $\Gamma$ du plan ayant la propriété suivante :
Soit $M \in \Gamma$, $C$ le centre de courbure de $\Gamma$ en $M$ et $N$
le projeté de $O$ sur la normale à $\Gamma$ en $M$.
Alors $\vec{MC} = k\vec{MN}$ où $k$ est un réel fixé.

\'Etudier les cas particuliers :
$k=1$, $k=\frac23$, $k=2$, $k=\frac13$ et $k=-1$.
}
\reponse{
Soit $\theta$ l'angle polaire de $\vec{OM}$ :
$\vec{MC} = \frac{d s}{ d\varphi}\vec n$ et
$\vec{MN} = \frac{d s}{ d\theta}\vec n$.

$\frac{d s}{ d\theta} = k\frac{d s}{ d\varphi}  \Rightarrow 
\varphi = \frac\theta k+b  \Rightarrow  V = a\theta+b$ avec $a = \frac1k-1$.

$\frac\rho{\rho'} = \tan(a\theta+b)  \Rightarrow  \rho = \lambda\cos(a\theta+b)^{-1/a}$
si $a \ne 0$ ou $\rho = \lambda e^{\mu\theta}$ si $a=0$.

\leavevmode\vbox{\halign{$k=#$\hfil &$ \Rightarrow $ #\hfil\cr
1       & Spirale logarithmique.\cr
\frac23 & Parabole de foyer $O$.\cr
2       & Cardioïde.\cr
\frac13 & Hyperbole de centre $O$.\cr
-1      & Lemniscate de Bernouilli.\cr}}
}
}
