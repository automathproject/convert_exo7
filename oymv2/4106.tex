\uuid{grue}
\exo7id{4106}
\auteur{quercia}
\datecreate{2010-03-11}
\isIndication{false}
\isCorrection{true}
\chapitre{Equation différentielle}
\sousChapitre{Equations différentielles linéaires}

\contenu{
\texte{
On considère l'équation différentielle à coefficients continus sur
$\R$ : $x''+p(t)x'+q(t)x=0$. Trouver une condtion nécessaire
portant sur $p$ et $q$ pour qu'il existe deux solutions sur $\R$ dont
le produit vaut constamment~un.
}
\reponse{
$x$ et $1/x$ sont solution $ \Rightarrow $ $x'^2+qx^2=0$ donc une condition nécessaire
est~: $q(t)\le 0$ et $q=-x'^2/x^2$ est de classe $\mathcal{C}^1$ .
Réciproquement, supposons $q$ négative de classe $\mathcal{C}^1$ et soit $r(t)=\sqrt{-q(t)}$.
Si $x$ est solution de $x'=r(t)x$ alors sur tout intervalle~$I$ où $q$ ne
s'annule pas on a $x'' = r(t)x' + r'(t)x$ donc
$$x''+p(t)x'+q(t)x = (r(t)+p(t))x'+(r'(t)+q(t))x = (r(t)p(t)+r'(t))x$$
donc une deuxième condition nécessaire est~: $p(t)q(t)=-\frac12q'(t)$.
Ces deux conditions sont suffisantes si $q$ est strictement négative.
}
}
