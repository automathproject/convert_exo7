\uuid{FEX8}
\exo7id{2424}
\auteur{lescure}
\datecreate{2007-10-01}
\isIndication{false}
\isCorrection{true}
\chapitre{Espace topologique, espace métrique}
\sousChapitre{Espace topologique, espace métrique}

\contenu{
\texte{

}
\begin{enumerate}
    \item \question{Soit $E$ un espace m\'etrique et $A \subset E$ une de ses parties. 
On désigne par $\overline A$ l'adhérence de $A$ et par $\mathrm{Fr} (A)$ la frontière  de $A$ dans $E.$ 
On a $\mathrm{Fr} (A) = \overline{A} \cap \overline{A^c}$.

\begin{enumerate}}
\reponse{\begin{enumerate}}
    \item \question{Montrez que $x \in\mathrm{Fr} (A),$ si et seulement si il existe une 
suite $(x_n)$ d'éléments de $A$ et une suite $(y_n)$ d'éléments du complémentaire
$E \setminus A$ de $A$ dans $E,$ qui convergent l'une et l'autre vers $x.$}
\reponse{Si  $x \in\mathrm{Fr} (A) = \bar A \cap \overline{E \setminus A}$ , alors $\forall n \in \Nn^*$
la boule $B(x,{\frac 1 n})$ rencontre nécessairement $A$  (respectivement $E \setminus A$). Soit donc (axiome du choix)
$x_n $ (respectivement $y_n$) dans  $B(x,{\frac 1 n}) \cap A$ (respectivement $y_n$ dans $B(x,{\frac 1 n})\cap (E \setminus A).$
Alors les suites $x_n$ et $y_n$ répondent clairement à la question : 
On a une suite $(x_n)$ d'éléments de $A$ et une suite $(y_n)$ d'éléments du complémentaire
$E \setminus A$ de $A$ dans $E,$ qui convergent l'une et l'autre vers $x.$}
    \item \question{Soit $E=]-\infty,-1]\cup [0,1[ \cup [2,+\infty[$ muni de la topologie induite
par $\Rr.$ Avec $A=[0,\frac 12],$ qu'elle est la frontière de $A$ dans $E.$ Considérée
 comme sous-partie de $\Rr,$ qu'elle serait la frontière de $A$ dans $\Rr$?}
\reponse{On voit, qu'en posant pour $n \ge 1,$ d'une part $x_n=\frac 12-{\frac 1  {4n}}$ et d'autre part,
$y_n=\frac 12+{\frac 1 {4n}},$ on obtient, respectivement comme plus haut, une suite de points dans $A$
et une autre dans $E \setminus A$ qui convergent vers le même point $\frac 12 \in A$ qui, adhérent 
à $A$ comme à son complémentaire dans $E$ est donc dans la frontière de $A$ dans $E.$ Par contre,
si $x \in A$ est différent de $+\frac 12$, on voit que la boule (dans $E$) de centre $x$ et de rayon
$\frac 12 - x>0$ ne rencontre pas le complémentaire de $A$ et qu'en conséquence $[0, \frac 12[$ est
l'intérieur de $A$ {\it dans} E. 

A contrario une boule de centre $0$ et de rayon strictement positif rencontre toujours le complémentaire
de $A$ {\it dans} $\Rr$ ce qui permet aisément de voir que la frontière de $A$ {\it dans} $\Rr$ est
$\{0,\frac 12\}$.}
\end{enumerate}
}
