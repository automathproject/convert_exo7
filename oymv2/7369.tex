\uuid{boIt}
\exo7id{7369}
\auteur{mourougane}
\datecreate{2021-08-10}
\isIndication{false}
\isCorrection{true}
\chapitre{Groupe, anneau, corps}
\sousChapitre{Anneau}

\contenu{
\texte{

}
\begin{enumerate}
    \item \question{Soit $G$ un groupe, $e$ son élément neutre et $a$ un élément de $G$ et $m$ un entier naturel tel $a^m=e$. Que peut-on dire de l'ordre de l'élément $a$ ?}
    \item \question{Donner la définition d'un idéal dans un anneau commutatif.}
    \item \question{Quels sont les idéaux de l'anneau $(\Z,+,\times)$ ?}
    \item \question{Le sous-ensemble $28\Z\cap 18\Z$ de $\Z$ est-il un idéal de $\Z$ ? Si oui, le déterminer.}
    \item \question{Donner l'exemple de deux groupes finis de même ordre non isomorphes.
Justifier le fait que les deux groupes choisis ne sont pas isomorphes.}
    \item \question{Le groupe $\mathcal{S}_4$ des permutations de $\{1,2,3,4\}$ est-il cyclique ? (justifier)}
\reponse{
On peut affirmer que l'ordre de $a$ est un diviseur de $m$.
Il s'agit de l'idéal $pgcd(18,28)\Z=252\Z$.
$\Z/4\Z$ et $\Z/2\Z\times\Z/2\Z$ sont deux groupes d'ordre $4$, mais le second dont tous les éléments sont d'ordre 1 ou 2 n'est pas cyclique.
Comme $(1,2)(1,3)(1,2)=(2,3)$, $(1,2)$ et $(1,3)$ ne commutent pas. Par conséquent, le groupe $\mathcal{S}_4$ n'est pas cyclique.
}
\end{enumerate}
}
