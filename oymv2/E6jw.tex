\uuid{E6jw}
\exo7id{2239}
\auteur{matos}
\datecreate{2008-04-23}
\isIndication{false}
\isCorrection{true}
\chapitre{Autre}
\sousChapitre{Autre}

\contenu{
\texte{
On consid\`ere le syst\`eme $Ax=b$ avec
\begin{equation}\label{sys}
A=\left(\begin{array}{ccccc}
3&1&0&0&0\\
1&2&1&0&0\\
0&2&3&1&0\\
0&0&1&4&3\\
0&0&0&1&1
\end{array}\right)
\end{equation}
}
\begin{enumerate}
    \item \question{D\'ecomposer $A$ sous la forme $LU$ et en d\'eduire que (\ref{sys}) admet une solution unique $x^*$.}
    \item \question{Ecrire l'it\'eration de Gauss--Seidel pour ce syst\`eme, c'est--\`a--dire, le syst\`eme lin\'eaire donnant $X_{n+1}=(x_{n+1}, y_{n+1}, z_{n+1}, t_{n+1}, u_{n+1})$ en fonction de  $X_{n}=(x_{n}, y_{n}, z_{n}, t_{n}, u_{n})$.}
    \item \question{Pour tout $n\in\Nn$ on pose $e_n=X_n-x^*$. Montrer qu'il existe $a\in [0,1[$ tel que:

$$\forall n\in\Nn \ \ \  \|e_{n+1}\|_\infty \leq a \|e_n\|_\infty .$$
En d\'eduire la convergence de la suite.}
    \item \question{D\'eterminer la matrice de Gauss--Seidel ${\cal L}_1$ associ\'ee \`a $A$. Calculer $\|{\cal L}_1\|_\infty$. En d\'eduire la convergence de $(X_n)$ vers $x^*$.}
    \item \question{Soit $A\in\Rr^{n\times n} $ v\'erifiant la propri\'et\'e suivante:
$$\begin{array}{ccl}
|a_{ij}| & \geq & \sum_{j\neq i}|a_{ij}| \ \ i=2, \cdots , n\\
|a_{11}|& > & \sum_{j\neq 1}|a_{1j}|
\end{array}$$
et sur chaque ligne de $A$ il existe il existe un terme non nul $a_{ij}$ pour $i\geq 2$ et $j<i$.

Montrer qu'alors la m\'ethode de Gauss--Seidel converge.}
\reponse{
It\'eration de Gauss-Seidel : $(D-E)X_{n+1}=FX_n +b$ avec
$$D-E=\left(\begin{array}{ccccc}
3&&&&\\
1&2&&&\\
0&2&3&&\\
0&0&1&4&\\
0&0&0&1&1
\end{array}\right), -F=\left(\begin{array}{ccccc}
0&1&&&\\
&0&1&&\\
&&0&1&\\
&&&0&3\\
&&&&0\end{array}\right)$$
$e_n=X_n-X^*,\quad, X_{n+1}=(D-E)^{-1}FX_n +(D-E)^{-1}b,\quad
  X^*=(D-E)^{-1}FX^* +(D-E)^{-1}b\Rightarrow e_{n+1}=(D-E)^{-1}Fe_n$

On obtient alors $(D-E)e_{n+1}=(D-E)^{-1}Fe_n$ et si on \'ecrit composante \`a
composante on obtient

$3e_{n+1}^1=-e_n^2\Rightarrow  |e_{n+1}^1|\leq \frac{1}{3}\|e_n\|_{\infty}$

$e_{n+1}^1+2e_{n+1}^2=-e_n^3 \Rightarrow |e_{n+1}^2| \leq \frac{1}{6}\|e_n\|_\infty +\frac{1}{2}\|e_n\|_\infty=\frac{2}{3}\|e_n\|_\infty$

$2e_{n+1}^2 +3e_{n+1}^3=-e_n^4\Rightarrow |e_{n+1}^3|\leq \frac{2}{3}\frac{2}{3}\|e_n\|_\infty + \frac{1}{3}\|e_n\|_\infty=\frac{7}{9}\|e_n\|_\infty$

$e_{n+1}32 +4e_{n+1}^4=-3e_n^5\Rightarrow |e_{n+1}^4|\leq \frac{1}{4} \frac{7}{9}\|e_n\|_\infty + \frac{3}{4}\|e_n\|_\infty=\frac{34}{16} \|e_n\|_\infty$

$e_{n+1}^4+e_{n+1}^5=0\Rightarrow |e_{n+1}^5|\leq \frac{17}{18}\|e_n\|_\infty$

et donc
$$\|e_n\|_\infty\leq \frac{17}{18}\|e_n\|_\infty$$
$$(D-E)^{-1}=\left(\begin{array}{ccccc}
\frac{1}{3}&0&0&0&0\\
-\frac{1}{6}& \frac{1}{2}&0&0&0\\
\frac{1}{9}&-\frac{1}{3}&\frac{1}{3}&0&0\\
-\frac{1}{36}&\frac{1}{12}&-\frac{1}{12}&\frac{1}{4}&0\\
\frac{1}{36}&-\frac{1}{12}&\frac{1}{12}&-\frac{1}{4}&1
\end{array}\right),$$

$$L_1=(D-E)^{-1}F=\left(\begin{array}{ccccc}
0&\frac{1}{3}&0&0&0\\
0&-\frac{1}{6}& \frac{1}{2}&0&0\\
0&\frac{1}{9}&-\frac{1}{2}&\frac{1}{3}&0\\
0&-\frac{1}{36}&\frac{1}{12}&-\frac{1}{12}&\frac{3}{4}\\
0&\frac{1}{36}&-\frac{1}{12}&\frac{1}{12}&-\frac{3}{4}
\end{array}\right)$$
et donc $\|L_1\|_\infty=\max (\frac{1}{3},\frac{4}{6},\frac{17}{18},\frac{32}{36},\frac{32}{36})=\frac{17}{18}$.

On en d\'eduit donc la convergence de $(X_n)$ vers $X^*$.
}
\end{enumerate}
}
