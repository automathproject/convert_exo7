\uuid{w6lF}
\exo7id{2277}
\auteur{barraud}
\datecreate{2008-04-24}
\isIndication{false}
\isCorrection{true}
\chapitre{Polynôme}
\sousChapitre{Polynôme}

\contenu{
\texte{
Soit $P\in \Zz[x]$.
}
\begin{enumerate}
    \item \question{Supposons que $P(0)$, $P(1)$ soient impairs. Montrer que
$P$ n'a pas de racine dans $\Zz$.
(\emph{Indication :} Utiliser la r\'eduction modulo $2$.)}
\reponse{Si $P(0)$ et $P(1)$ sont impairs, $\bar{P}(\bar{0})=\bar{1}$ et
     $\bar{P}(\bar{1})=\bar{1}$, donc $\bar{P}$ n'a pas de racine sur
     $\Zz/2\Zz$. Donc $P$ n'a pas de racine sur $\Zz$.}
    \item \question{Soit $n\in \Nn$ tel qu'aucun des entiers $P(0),\dots,P(n-1)$
ne soit divisible par $n$. Montrer que
$P$ n'a pas de racine dans $\Zz$.}
\reponse{Si $n$ ne divise aucun des $P(0),\dots,P(n-1)$, alors
     $\bar{P}(\bar{0})\neq 0$,\dots, $\bar{P}(\overline{n-1})\neq 0$,
     donc $\bar{P}$ n'a pas de racine sur $\Zz/n\Zz$. Donc $P$ n'a pas de
     racine sur $\Zz$.}
\end{enumerate}
}
