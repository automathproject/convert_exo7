\uuid{6805}
\auteur{gijs}
\datecreate{2011-10-16}
\isIndication{false}
\isCorrection{false}
\chapitre{Théorème du point fixe}
\sousChapitre{Théorème du point fixe}

\contenu{
\texte{
Soit $f:\Rr^2 \to \Rr^2$ définie par $f(x,y) =
(x^2-y, x^2 + y^2)$. On définit la fonction $g:\Rr^2 \to \Rr^2$ par $g = f \circ f$.
}
\begin{enumerate}
    \item \question{Montrer que $f$ et $g$ sont de classe $C^1$.}
    \item \question{Calculer pour tout $(x,y) \in \Rr^2$ la matrice
Jacobienne de $f$ en $(x,y)$ notée $D_{(x,y)}f$; 
calculer la matrice Jacobienne de $g$ en $(0,0)$ notée
$D_{(0,0)}g$.}
    \item \question{Montrer qu'il existe $\rho>0$ tel que pour tout $(x,y) \in
\overline{B_\rho((0,0))}$ (la boule fermée de centre
$(0,0)$ et de rayon $\rho$) on a $\Vert D_{(x,y)}g \Vert
\le \frac12$.}
    \item \question{Montrer que la fonction $g$ admet un unique point fixe
dans $\overline{B_\rho((0,0))}$ avec $\rho$ comme dans 3).}
\end{enumerate}
}
