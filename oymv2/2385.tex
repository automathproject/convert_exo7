\uuid{L2Bi}
\exo7id{2385}
\auteur{mayer}
\datecreate{2003-10-01}
\isIndication{true}
\isCorrection{true}
\chapitre{Connexité}
\sousChapitre{Connexité}

\contenu{
\texte{
Soit $A$ et $B$ des parties de $X$. On suppose $B$ connexe et que
$B\cap A$ et $B \cap \complement A$ sont non vides. Montrer que $B$
coupe la fronti\`ere de $A$.
}
\indication{Utiliser la partition $X = \mathring A \cup \mathrm{Fr}\, A  \cup (X \setminus \bar A)$
o\`u  $\mathrm{Fr}\, A = \bar A \setminus \mathring A$ est la frontière de $A$.}
\reponse{
Notons la frontière $\mathrm{Fr}\, A = \bar A \setminus \mathring A$.
Nous avons la partition $X = \mathring A \cup \mathrm{Fr}\, A  \cup (X \setminus \bar A)$.
Si $B \cap \mathrm{Fr}\, A  = \varnothing$ alors $B\subset \mathring A \cup (X \setminus \bar A)$. 

De plus, par hypothèses, $B \cap A \neq \varnothing$ et $B \cap \mathrm{Fr}\, A = \varnothing$ 
or $\mathring A= A \setminus \mathrm{Fr}\, A$ donc $B \cap \mathring A \neq \varnothing$.
Comme $\mathrm{Fr}\, A = \mathrm{Fr}\, (X\setminus A)$ on a $B\cap \mathrm{Fr}\, (X\setminus A) = \varnothing$.
Par hypothèse $B\cap (X\setminus  A) \neq \varnothing$ donc $B\cap  (X\setminus \bar A) = 
(B\cap  (X\setminus A)) \setminus (B\cap \mathrm{Fr}\, (X\setminus A)) \neq \varnothing$.

 Nous avons montrer que $B$ est inclus dans l'union de deux ouverts disjoints 
$\mathring A$ et $X\setminus \bar A$, d'intersection non vide avec $B$, donc $B$ n'est pas connexe.
Par contraposition, si $B$ est connexe alors $B$ ne rencontre pas la frontière 
de $A$.
}
}
