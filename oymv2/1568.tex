\uuid{1568}
\auteur{ridde}
\datecreate{1999-11-01}

\contenu{
\texte{
Soit $E$ un espace euclidien et $u\in O (E)$ telle que $\ker (u-id) \neq E$.
Soit $x\in E$ tel que $u (x) \neq x$. On pose $y = u (x)$. Alors on sait qu'il
existe une unique r\'eflexion $r$ telle que $r (y) = x$.
}
\begin{enumerate}
    \item \question{Montrer que $\ker (u-id) \subset \ker (r-id)$.}
    \item \question{Montrer que $\dim \ker (r\circ u-id) >\dim \ker (u-id)$.}
    \item \question{Montrer par r\'ecurrence que toute isom\'etrie vectorielle est la compos\'ee
de r\'eflexions.}
\end{enumerate}
}
