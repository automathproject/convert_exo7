\uuid{2vfh}
\exo7id{2519}
\auteur{queffelec}
\datecreate{2009-04-01}
\isIndication{false}
\isCorrection{true}
\chapitre{Différentiabilité, calcul de différentielles}
\sousChapitre{Différentiabilité, calcul de différentielles}

\contenu{
\texte{
Montrer que l'identit\'e des accroissements finis n'est pas
vraie pour les fonctions vectorielles en consid\'erant
$f(x)=e^{ix}$.
}
\reponse{
On a $f'(x)=ie^{ix}$ (on peut le v\'erifier en coordonn\'ees). Si
l'\'egalit\'e des accroissement finis \'etait v\'erifi\'ee il
existerait $$c \in ]0,\pi[ \mbox{ tel que }
f(\pi)-f(0)=(\pi-0)ie^{ic}$$ ce qui est impossible car en prenant
les modules on trouverait $2=\pi$.
}
}
