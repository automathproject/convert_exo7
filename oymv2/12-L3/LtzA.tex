\uuid{LtzA}
\exo7id{2496}
\auteur{sarkis}
\datecreate{2009-04-01}
\isIndication{false}
\isCorrection{true}
\chapitre{Différentiabilité, calcul de différentielles}
\sousChapitre{Différentiabilité, calcul de différentielles}

\contenu{
\texte{
Soit $E$ l'ensemble des
fonctions continues de l'intervalle $[0,1]$ dans $\mathbb{R}$ qui
sont continues. Montrez que l'application $\|f\|_1=\int_a^b
|f(t)|dt$ est une norme sur $E$. Montrez que $E$ n'est pas
complet.
}
\reponse{
Il faut trouver une suite de cauchy de fonctions de $E$ qui ne
converge pas dans $E$. Il suffit, par exemple, de prendre une
suites de fonctions $\{f_n\}$  convergeant pour $\|.\|$ vers une
fonction non continue. Par exemple, prendre
$$f_n(x)= \left\{\begin{array}{c}
1 \mbox{ si } x < 1/2\\
1-n(x-1/2) \mbox{ si } 1/2 \leq x \leq 1/2+1/n \\
0 \mbox{ si } x > 1/2+1/n
\end{array} \right \} $$
et
$$f_0(x)= \left\{\begin{array}{c}
1 \mbox{ si } x < 1/2\\
0 \mbox{ si } x \geq 1/2
\end{array} \right \} $$
 On a alors $\|f_n - f_0\|_1=1/(2n)$, la suite converge
simplement et en norme $\|.\|$ vers la fonction  $f_0$ qui n'est
pas continue. Il suffit de montrer alors qu'il n'existe aucune
fonction continue $g$ telle que $\|f-g\|=0$ ce qui interdit
l'existence d'une limite à $f_n$ dans $E$.
}
}
