\uuid{tIv4}
\exo7id{6266}
\auteur{queffelec}
\datecreate{2011-10-16}
\isIndication{false}
\isCorrection{false}
\chapitre{Difféomorphisme, théorème d'inversion locale et des fonctions implicites}
\sousChapitre{Difféomorphisme, théorème d'inversion locale et des fonctions implicites}

\contenu{
\texte{

}
\begin{enumerate}
    \item \question{On considère l'application $\varphi$ de $\Rr^3$ dans lui-même définie par
$(x,y,z)\to (e^{2y}+e^{2z}, e^{2x}-e^{2z}, x-y)$. Montrer que $\varphi$ est un
$C^1$-difféomorphisme de $\Rr^3$ sur son image que l'on précisera.}
    \item \question{Soit $\lambda\in \Rr$ et $F$ l'application de $\Rr^3$ dans lui-même
définie par
$(x,y,z)\to (e^{x-y+2z}+e^{-x+y+2z}, e^{2x}+e^{2y}-2\lambda e^{x-y},
e^{2x}+e^{2y}-2e^{-x+y})$. Montrer que
$F$ s'écrit $G\circ \varphi$, $G$ à préciser, et que c'est un
$C^1$-difféomorphisme de $\Rr^3$ sur son image si et seulement si
$\lambda\geq0$.}
\end{enumerate}
}
