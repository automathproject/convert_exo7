\uuid{3586}
\auteur{quercia}
\datecreate{2010-03-10}
\isIndication{false}
\isCorrection{true}
\chapitre{Réduction d'endomorphisme, polynôme annulateur}
\sousChapitre{Applications}

\contenu{
\texte{
Soit $A \in \mathcal{M}_{3}(\R)$ ayant pour valeurs propres $1,-2,2$, et $n\in \N$.
}
\begin{enumerate}
    \item \question{Montrer que $A^n$ peut s'écrire sous la forme :
    $A^n = \alpha_n A^2 + \beta_n A + \gamma_n I$
    avec $\alpha_n,\beta_n,\gamma_n \in \R$.}
    \item \question{On considère le polynôme $P = \alpha_n X^2 + \beta_n X + \gamma_n$.
    Montrer que : $P(1) = 1$, $P(2) = 2^n$, $P(-2) = (-2)^n$.}
    \item \question{En déduire les coefficients $\alpha_n,\beta_n,\gamma_n$.}
\reponse{
3. $\alpha_n = -\frac13 + \frac {2^n}4 + \frac {(-2)^n}{12}$,
             $\beta_n  = \frac {2^n - (-2)^n}4$,
             $\gamma_n = \frac43 - \frac {2^n}2 + \frac {(-2)^n}{6}$.
}
\end{enumerate}
}
