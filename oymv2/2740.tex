\uuid{2740}
\auteur{tumpach}
\datecreate{2009-10-25}
\isIndication{false}
\isCorrection{false}
\chapitre{Application linéaire}
\sousChapitre{Définition}

\contenu{
\texte{

}
\begin{enumerate}
    \item \question{On munit $\mathbb{R}^2$ d'un rep\`ere orthonorm\'e $(O, \vec{i}, \vec{j})$. Montrer qu'une application lin\'eaire de $\mathbb{R}^{2}$ dans $\mathbb{R}^2$ est uniquement d\'etermin\'ee par ses valeurs sur les vecteurs $\vec{i}$ et $\vec{j}$.}
    \item \question{Quelle est la matrice de la sym\'etrie axiale par rapport \`a l'axe des abscisses dans la base $\{\vec{i}, \vec{j}\}$~?}
    \item \question{Quelle est la matrice de la projection orthogonale sur l'axe des abscisses dans la base $\{\vec{i}, \vec{j}\}$~?}
    \item \question{Quelle est la matrice de la rotation d'angle $\theta$ et de centre $O$ dans la base $\{\vec{i}, \vec{j}\}$~?}
    \item \question{Quelle est la matrice de l'homoth\'etie de centre $O$ et de rapport $k$ dans la base $\{\vec{i}, \vec{j}\}$~?}
    \item \question{Quelle est la matrice de la sym\'etrie centrale de centre $O$  dans la base $\{\vec{i}, \vec{j}\}$~?}
    \item \question{Est-ce qu'une translation est une application lin\'eaire~?}
\end{enumerate}
}
