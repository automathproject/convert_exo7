\uuid{7OmS}
\exo7id{4756}
\auteur{quercia}
\datecreate{2010-03-16}
\isIndication{false}
\isCorrection{true}
\chapitre{Topologie}
\sousChapitre{Topologie des espaces vectoriels normés}

\contenu{
\texte{
Soit $E$ un espace vectoriel norm{\'e} sur $\R$ ou $\C$ de dimension finie,
et $u\in\mathcal{L}(E)$ tel que pour tout $x\in E$ la suite $(u^n(x))_{n\in\N}$ est
born{\'e}e.
}
\begin{enumerate}
    \item \question{Montrer que la suite $(\|\hskip-1pt|u^n\|\hskip-1pt|)_{n\in\N}$ est born{\'e}e.}
\reponse{Soit $(e_1,\dots,e_p)$ une base de~$E$. On remplace la norme sur $E$
         par la norme infinie associ{\'e}e {\`a} $(e_1,\dots,e_p)$. Alors
         $\|\hskip-1pt|u^n\|\hskip-1pt| \le \sum_{i=1}^p\|u^n(e_i)\|$.}
    \item \question{D{\'e}terminer la limite quand $n\to\infty$ de $\frac1{n+1}\sum_{i=0}^n u^i(x)$.}
\reponse{Trigonaliser fortement $u$ (ou son prolongement au complexifi{\'e} de~$E$).
         Comme $(u^n)$ est born{\'e}, les valeurs propres de~$u$ sont de module inf{\'e}rieur
         ou {\'e}gal {\`a}~$1$, et pour celles de module $1$ le bloc triangulaire associ{\'e}
         est en fait diagonal.
         On trouve $\frac1{n+1}\sum_{i=0}^n u^i \xrightarrow[n\to\infty]{}$ projection sur
         $\mathrm{Ker}(u-\mathrm{id})$ parall{\`e}lement {\`a} $\Im(u-\mathrm{id})$.}
\end{enumerate}
}
