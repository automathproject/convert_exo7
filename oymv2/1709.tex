\uuid{1709}
\auteur{barraud}
\datecreate{2003-09-01}

\contenu{
\texte{
Etant donnés quatre nombres réels $(u_{0},v_{0},w_{0},x_{0})$, on définit quatre nouveaux nombres $(u_{1},v_{1},w_{1},x_{1})$ en calculant les
moyennes suivantes :
 $u_{1}= \frac{2u_{0}+ v_{0}+ w_{0}+ x_{0}}{5}$,
 $v_{1}= \frac{ u_{0}+2v_{0}+ w_{0}+ x_{0}}{5}$,
 $w_{1}= \frac{ u_{0}+ v_{0}+2w_{0}+ x_{0}}{5}$, et
 $x_{1}= \frac{ u_{0}+ v_{0}+ w_{0}+2x_{0}}{5}$.
En itérant ce procédé, on définit quatre suites $(u_{n})$, $(v_{n})$, $(w_{n})$, et $(x_{n})$ telles que pour tout $n\in\N$ on ait :
$$
\left\{
\begin{array}{r@{\,}l} 
  u_{n+1}&= \frac{1}{5} (2u_{n}+ v_{n}+ w_{n}+ x_{n}) \\[1.5ex]
  v_{n+1}&= \frac{1}{5} ( u_{n}+2v_{n}+ w_{n}+ x_{n}) \\[1.5ex]
  w_{n+1}&= \frac{1}{5} ( u_{n}+ v_{n}+2w_{n}+ x_{n}) \\[1.5ex]
  x_{n+1}&= \frac{1}{5} ( u_{n}+ v_{n}+ w_{n}+2x_{n})
\end{array}
\right.
$$
}
\begin{enumerate}
    \item \question{Ecrire la matrice $A$ associée à cette relation de récurrence, et la
matrice $B=5A$. Que dire de la diagonalisabilité de $B$ ?}
    \item \question{Sans calculer le polynôme caractéristique de $B$, montrer que $1$ est
valeur propre de $B$. Quelle est la dimension de l'espace propre
associé ? Que dire de la multiplicité de $1$ comme valeur propre de $B$ ?}
    \item \question{En utilisant la trace de $B$, déterminer toutes les valeurs propres de $B$.}
    \item \question{Donner un polynôme annulateur de $B$ de degré 2.}
    \item \question{En déduire l'existence de deux réels $a_{n}$ et $b_{n}$, que l'on
calculera, tels que $B^{n}=a_{n}B+b_{n}I$.}
    \item \question{Calculer $\displaystyle \lim_{n\rightarrow\infty}{\frac{a_{n}}{5^{n}}}$ et
$\displaystyle \lim_{n\rightarrow\infty}{\frac{b_{n}}{5^{n}}}$. En déduire que la
suite de matrices $(A^{n})_{n\in\N}$ est convergente et donner sa
limite.

{\it (On rappelle qu'une suite de matrices $M_{n}$ est dite
convergente si chaque suite de coefficient est convergente. On pourra
utiliser sans démonstration la continuité des opérations élémentaires
sur les matrices pour cette notion de limite, c'est à dire que :
\\
\ - si $(\lambda_{n})$ est une suite convergente alors pour toute matrice
$M$, la suite $(\lambda_{n}M)$ est convergente et
$\lim\limits_{n\rightarrow\infty}(\lambda_{n}M) =
(\lim\limits_{n\rightarrow\infty}\lambda_{n})M$  
\\
\ - si $(M_{n})$ est une suite de matrices convergente alors pour tout
vecteur $X$, la suite de vecteurs $(M_{n}X)$ est convergente et
 $\lim\limits_{n\rightarrow\infty}(M_{n}X) = 
 (\lim\limits_{n\rightarrow\infty} M_{n})X$.)}}
    \item \question{En déduire que les suites $(u_{n})_{n\in\N}$, $(v_{n})_{n\in\N}$,
$(w_{n})_{n\in\N}$, et $(x_{n})_{n\in\N}$ sont convergentes, et
donner leur limite.}
\end{enumerate}
}
