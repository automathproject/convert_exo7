\uuid{DquN}
\exo7id{5245}
\auteur{rouget}
\datecreate{2010-06-30}
\isIndication{false}
\isCorrection{true}
\chapitre{Suite}
\sousChapitre{Convergence}

\contenu{
\texte{
Soit $f$ une application injective de $\Nn$ dans $\Nn$. Montrer que $\lim_{n\rightarrow +\infty}f(n)=+\infty$.
}
\reponse{
Soit $f$ une application de $\Nn$ dans lui-même, injective. Montrons que $\lim_{n\rightarrow +\infty}f(n)=+\infty$.

Soient $A$ un réel puis $m=\mbox{Max}(0,1+E(A))$.

Puisque $f$ est injective, on a $\mbox{card}(f^{-1}(\{0,1,...,m\})\geq m+1$. En particulier, $f^{-1}(\{0,1,...,m\})$ est fini (éventuellement vide).

Posons $n_0=1+\left\{
\begin{array}{l}
0\;\mbox{si}\;f^{-1}(\{0,1,...,m\})=\emptyset\\
\mbox{Max}f^{-1}(\{0,1,...,m\})\;\mbox{sinon}
\end{array}
\right.$.

Par définition de $n_0$, si $n\geq n_0$, $n$ n'est pas élément de $f^{-1}(\{0,1,...,m\})$ et donc $f(n)>m>A$.

On a montré que $\forall A\in\Rr,\;\exists n_0\in\Nn/\;(\forall n\in\Nn),\;(n\geq n_0\Rightarrow f(n)>A)$ ou encore 
$\lim_{n\rightarrow +\infty}f(n)=+\infty$.
}
}
