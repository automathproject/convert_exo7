\uuid{47}
\auteur{bodin}
\datecreate{1998-09-01}

\contenu{
\texte{
Calculer la somme $S_n = 1+z+z^2+\cdots+z^n$.
}
\indication{Calculer $(1-z)S_n$.}
\reponse{
$$S_n = 1+z+z^2+\cdots+z^n = \sum_{k=0}^{n}z^k.$$
Nous devons retrouver le r\'esultat sur la somme $S_n =
\frac{1-z^{n+1}}{1-z}$d'une suite g\'eom\'etrique dans le cas o\`u
$z\not=1$ est un r\'eel. Soit maintenant $z \not= 1$ un nombre
complexe. Calculons $S_n(1-z)$.
\begin{align*}
S_n(1-z) & =(1+z+z^2+\cdots+z^n)(1-z) \text{ d\'eveloppons }\\
         &= 1+z+z^2+\cdots+z^n - z-z^2-\cdots-z^{n+1} \text{ les termes interm\'ediaires s'annulent }\\
         &= 1-z^{n+1}.
\end{align*}
Donc $$S_n = \frac{1-z^{n+1}}{1-z}, \text{ pour } z\not= 1.$$
}
}
