\uuid{856}
\auteur{ruette}
\datecreate{2013-02-01}

\contenu{
\texte{
On consid\`ere l'\'equation diff\'erentielle suivante :
$$
(E)\qquad y'-(1-\tan(x))y=\cos(x).
$$
On s'int\'eresse \`a cette \'equation sur l'intervalle $]-\frac{\pi}{2},\frac{\pi}{2}[$.
}
\begin{enumerate}
    \item \question{\begin{enumerate}}
    \item \question{\'Ecrire l'\'equation homog\`ene associ\'ee \`a (E), puis  r\'esoudre cette \'equation 
sur l'intervalle  $]-\frac{\pi}{2},\frac{\pi}{2}[$.}
    \item \question{Quelle est  la solution de l'\'equation homog\`ene v\'erifiant la condition 
initiale $y(0)=1$ ? On la note $g_{0}$. Donner le tableau de variations de $g_{0}$ sur l'intervalle 
$]-\frac{\pi}{2},\frac{\pi}{2}[$. D\'eterminer les limites aux bornes de l'intervalle. 
Tracer  le graphe de $g_{0}$.}
    \item \question{Tracer sur le m\^eme dessin, rapidement, les graphes des trois  solutions de l'\'equation homog\`ene  v\'erifiant respectivement  $y(0) = 2$, $y(0)=3$ , $y(0) = -1$.}
\end{enumerate}
}
