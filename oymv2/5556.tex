\uuid{mQ2N}
\exo7id{5556}
\auteur{rouget}
\datecreate{2010-07-15}
\isIndication{false}
\isCorrection{true}
\chapitre{Fonction de plusieurs variables}
\sousChapitre{Continuité}

\contenu{
\texte{
Soit $\begin{array}[t]{cccc}f~:&\Rr^2&\longrightarrow&\Rr\\
 &(x,y)&\mapsto&\left\{
\begin{array}{l}
0\;\text{si}\;y=0\\
y^2\sin\left(\frac{x}{y}\right)\;\text{si}\;y\neq0
\end{array}
\right.
\end{array}
$.
}
\begin{enumerate}
    \item \question{Etudier la continuité de $f$.}
\reponse{Posons $\Delta=\{(x,y)/\;y\neq0\}$. $f$ est continue sur $\Rr^2\setminus\Delta$ en vertu de théorèmes généraux. Soit $x_0\in\Rr$.

\begin{center}
$|f(x,y)-f(x_0,0)|=\left\{
\begin{array}{l}
0\;\text{si}\;y=0\\
y^2\left|\sin\left(\frac{x}{y}\right)\right|\;\text{si}\;y\neq0
\end{array}
\right.\leqslant y^2$.
\end{center}
Comme $\displaystyle\lim_{(x,y)\rightarrow(x_0,0)}y^2=0$, $\displaystyle\lim_{(x,y)\rightarrow(x_0,0)}|f(x,y)-f(x_0,0)|=0$ et donc $f$ est continue en $(x_0,0)$. Finalement,

\begin{center}
\shadowbox{
$f$ est continue sur $\Rr^2$.
}
\end{center}}
    \item \question{Etudier l'existence et la valeur éventuelle de dérivées partielles d'ordre 1 et 2.
On montrera en particulier que $\frac{\partial^2f}{\partial x\partial y}$ et $\frac{\partial^2f}{\partial y\partial x}$  sont définies en $(0,0)$ mais n'ont pas la même valeur.}
\reponse{\textbullet~$f$ est de classe $C^2$ sur $\Rr^2\setminus\Delta$. En particulier, d'après le théorème de \textsc{Schwarz}, $\frac{\partial^2f}{\partial x\partial y}=\frac{\partial^2f}{\partial y\partial x}$ sur $\Delta$. pour $(x,y)\in\Rr^2\setminus\Delta$,

\begin{center}
$\frac{\partial f}{\partial x}(x,y)=y\cos\left(\frac{x}{y}\right)$ et $\frac{\partial f}{\partial y}(x,y)=2y\sin\left(\frac{x}{y}\right)-x\cos\left(\frac{x}{y}\right)$,
\end{center}
puis

\begin{center}
$\frac{\partial^2f}{\partial x^2}(x,y)=-\sin\left(\frac{x}{y}\right)$, $\frac{\partial^2f}{\partial x\partial y}(x,y)=\cos\left(\frac{x}{y}\right)-\frac{x}{y}\sin\left(\frac{x}{y}\right)$,
\end{center}
et enfin

\begin{center}
$\frac{\partial^2f}{\partial y^2}(x,y)=2\sin\left(\frac{x}{y}\right)-2\frac{x}{y}\cos\left(\frac{x}{y}\right)-\frac{x^2}{y^2}\sin\left(\frac{x}{y}\right)$.
\end{center}
\textbullet~\textbf{Existence de $\frac{\partial f}{\partial x}(x_0,0)$.} Pour $x\neq x_0$,  $\frac{f(x,0)-f(x_0,0)}{x-x_0}=0$ et donc$\frac{f(x,0)-f(x_0,0)}{x-x_0}\underset{x\rightarrow x_0}{\rightarrow}0$. On en déduit que $\frac{\partial f}{\partial x}(x_0,0)$ existe et $\frac{\partial f}{\partial x}(x_0,0)=0$. En résumé, $f$ admet une dérivée partielle par rapport à sa première variable sur $\Rr^2$ définie par

\begin{center}
\shadowbox{
$\forall(x,y)\in\Rr^2,\;\frac{\partial f}{\partial x}(x,y)=\left\{
\begin{array}{l}
0\;\text{si}\;y=0\\
y\cos\left(\frac{x}{y}\right)\;\text{si}\;y\neq0
\end{array}
\right.$.
}
\end{center}
\textbullet~\textbf{Existence de $\frac{\partial f}{\partial y}(x_0,0)$.} Soit $x_0\in\Rr$. Pour $y\neq0$,  

\begin{center}
$\left|\frac{f(x_0,y)-f(x_0,0)}{y-0}\right|=\left\{
\begin{array}{l}
0\;\text{si}\;y=0\\
y\left|\sin\left(\frac{x_0}{y}\right)\right|\;\text{si}\;y\neq0
\end{array}
\right.\leqslant|y|.
$
\end{center} 

et donc$\frac{f(x_0,y)-f(x_0,0)}{y-0}\underset{y\rightarrow0}{\rightarrow}0$. On en déduit que $\frac{\partial f}{\partial y}(x_0,0)$ existe et $\frac{\partial f}{\partial y}(x_0,0)=0$. En résumé, $f$ admet une dérivée partielle par rapport à sa deuxième variable sur $\Rr^2$ définie par

\begin{center}
\shadowbox{
$\forall(x,y)\in\Rr^2,\;\frac{\partial f}{\partial y}(x,y)=\left\{
\begin{array}{l}
0\;\text{si}\;y=0\\
2y\sin\left(\frac{x}{y}\right)-x\cos\left(\frac{x}{y}\right)\;\text{si}\;y\neq0
\end{array}
\right.$.
}
\end{center}
\textbullet~\textbf{Existence de $\frac{\partial^2f}{\partial x\partial y}(0,0)$.} Pour $x\neq0$,

\begin{center}
$\frac{\frac{\partial f}{\partial y}(x,0)-\frac{\partial f}{\partial y}(0,0)}{x-0}=0$
\end{center}
et donc $\frac{\frac{\partial f}{\partial y}(x,0)-\frac{\partial f}{\partial y}(0,0)}{x-0}$ tend vers $0$ quand $x$ tend vers $0$. On en déduit que $\frac{\partial^2f}{\partial x\partial y}(0,0)$ existe et

\begin{center}
\shadowbox{
$\frac{\partial^2f}{\partial x\partial y}(0,0)=0$.
}
\end{center}
\textbullet~\textbf{Existence de $\frac{\partial^2f}{\partial y\partial x}(0,0)$.} Pour $y\neq0$,

\begin{center}
$\frac{\frac{\partial f}{\partial x}(0,y)-\frac{\partial f}{\partial x}(0,0)}{y-0}=\frac{y\cos\left(\frac{0}{y}\right)}{y}=1$
\end{center}
et donc $\frac{\frac{\partial f}{\partial x}(0,y)-\frac{\partial f}{\partial x}(0,0)}{y-0}$ tend vers $1$ quand $y$ tend vers $0$. On en déduit que $\frac{\partial^2f}{\partial y\partial x}(0,0)$ existe et

\begin{center}
\shadowbox{
$\frac{\partial^2f}{\partial y\partial x}(0,0)=1$.
}
\end{center}}
\end{enumerate}
}
