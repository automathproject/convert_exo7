\uuid{1402}
\auteur{hilion}
\datecreate{2003-10-01}
\isIndication{false}
\isCorrection{false}
\chapitre{Groupe, anneau, corps}
\sousChapitre{Groupe de permutation}

\contenu{
\texte{

}
\begin{enumerate}
    \item \question{Déterminer $\text{card}(S_{3})$ et écrire tous les éléments de $S_{3}$, puis écrire la table de $S_{3}$ et en déduire tous les sous-groupes de $S_{3}$.}
    \item \question{On considère $T$ un triangle équilatéral du plan, de sommets $A,B,C$. 
\begin{enumerate}}
    \item \question{Montrer que les isométries du plan qui préservent $T$ forment un groupe pour la loi $\circ$, que l'on note $G$.}
    \item \question{Montrer qu'un élément de $G$ induit une permutation de l'ensemble $\{A,B,C\}$. On construit ainsi une application $\phi$ de $G$ dans $S_{3}$.}
    \item \question{Montrer que $\phi$ est un isomorphisme.}
\end{enumerate}
}
