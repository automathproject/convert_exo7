\uuid{4243}
\auteur{quercia}
\datecreate{2010-03-12}
\isIndication{false}
\isCorrection{false}
\chapitre{Calcul d'intégrales}
\sousChapitre{Longueur, aire, volume}

\contenu{
\texte{
Soit $f : {[a,b]} \to \R$ de classe $\mathcal{C}^2$.
}
\begin{enumerate}
    \item \question{Montrer que $ \int_{t=a}^b f(t)\,d t = (b-a)\frac {f(a)+f(b)}2
                    +  \int_{t=a}^b \frac {(t-a)(t-b)}2f''(t)\,d t$.}
    \item \question{Application : Soit $f : {[a,b]} \to \R$, $I =  \int_{t=a}^b f(t)\,d t$,
    et $I_n$ la valeur approchée de $I$ obtenue par la méthode des trapèzes
    avec $n$ intervalles.
    Démontrer que $|I-I_n| \le \frac {\sup|f''| (b-a)^3}{12n^2}$.}
\end{enumerate}
}
