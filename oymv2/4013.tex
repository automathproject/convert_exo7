\uuid{4013}
\auteur{quercia}
\datecreate{2010-03-11}

\contenu{
\texte{
Soit $f : \R \to \R$ une fonction de classe $\mathcal{C}^2$.
On suppose : $\forall\ x \in \R,\ |f(x)| \le \alpha$ et $|f''(x)| \le \beta$.
}
\begin{enumerate}
    \item \question{Montrer que : $\forall\ h > 0,\ \forall\ x \in \R,\ |f'(x)| \le \frac {2\alpha}h + \frac {h\beta}2$.}
\reponse{$f(x+h) = f(x) + hf'(x) + \frac {h^2}2f''(x+\theta h)  \Rightarrow 
     f'(x) = \frac {f(x+h) - f(x)}h -\frac h2f''(x+\theta h)$.}
    \item \question{Pour quelle valeur de $h$ obtient-on la meilleure inégalité ?}
\reponse{$h = 2\sqrt{\alpha/\beta}  \Rightarrow  |f'| \le 2\sqrt{\alpha\beta}$.}
\end{enumerate}
}
