\uuid{5106}
\auteur{rouget}
\datecreate{2010-06-30}
\isIndication{false}
\isCorrection{true}
\chapitre{Injection, surjection, bijection}
\sousChapitre{Bijection}

\contenu{
\texte{
Dans chacun des cas suivants, déterminer $f(I)$ puis vérifier que $f$ réalise une bijection
de $I$ sur $J=f(I)$ puis préciser $f^{-1}$~:
}
\begin{enumerate}
    \item \question{$f(x)=x^2-4x+3$, $I=]-\infty,2]$.}
\reponse{$f$ est dérivable sur $I=]-\infty,2]$, et pour $x\in]-\infty,2[$, $f'(x)=2x-4<0$. $f$ est donc continue
et strictement décroissante sur $]-\infty,2]$.
Par suite, $f$ réalise une bijection de $]-\infty,2]$ sur
$f(]-\infty,2])=[f(2),\underset{-\infty}{\mbox{lim}}\;f[=[-1,+\infty[=J$. On note $g$ l'application de $I$ dans $J$ qui, à
$x$ associe $x^2-4x+3(=f(x))$. $g$ est bijective et admet donc une réciproque. Déterminons $g^{-1}$.
Soit 
$y\in[-1,+\infty[$ et $x\in]-\infty,2]$.

$$y=g(x)\Leftrightarrow y=x^2-4x+3\Leftrightarrow x^2-4x+3-y=0.$$
Or, $\Delta'=4-(3-y)=y+1\geq0$. Donc, $x=2+\sqrt{y+1}$ ou $x=2-\sqrt{y+1}$. Enfin, $x\in]-\infty,2]$ et donc,
$x=2-\sqrt{y+1}$. En résumé,

$$\forall x\in]-\infty,2],\;\forall y\in[-1,+\infty[,\;y=g(x)\Leftrightarrow x=2-\sqrt{y+1}.$$
On vient de trouver $g^{-1}$~:

\begin{center}
\shadowbox{$\forall x\in[-1,+\infty[,\;g^{-1}(x)=2-\sqrt{x+1}$.}
\end{center}}
    \item \question{$f(x)=\frac{2x-1}{x+2}$, $I=]-2,+\infty[$.}
\reponse{On vérifie facilement que $f$ réalise une bijection de $]-2,+\infty[$ sur $]-\infty,2[$, notée $g$.
Soient
alors $x\in]-2,+\infty[$ et $y\in]-\infty,2[$.

$$y=g(x)\Leftrightarrow y=\frac{2x-1}{x+2}\Leftrightarrow x(-y+2)=2y+1\Leftrightarrow x=\frac{2y+1}{-y+2}.$$
(on a ainsi trouvé au plus une valeur pour $x$ à savoir $x=\frac{2y+1}{-y+2}$, mais il n'est pas nécessaire 
de vérifier que cette expression est bien définie et élément de $]-2,+\infty[$ car on sait à l'avance que $y$ admet au
moins un antécédent dans $]-2,+\infty[$, et c'est donc nécessairement le bon). En résumé,

$$\forall x\in]-2,+\infty[,\;\forall y\in]-\infty,2[,\;y=g(x)\Leftrightarrow x=\frac{2y+1}{-y+2}.$$
On vient de trouver $g^{-1}$~:~

\begin{center}
\shadowbox{$\forall x\in]-\infty,2[,\;g^{-1}(x)=\frac{2x+1}{-x+2}$}.
\end{center}}
    \item \question{$f(x)=\sqrt{2x+3}-1$, $I=\left[-\frac{3}{2},+\infty\right[$.}
\reponse{$f$ est continue et strictement croissante sur $\left[-\frac{3}{2},+\infty\right[$.
$f$ est donc bijective de
$\left[-\frac{3}{2},+\infty\right[$ sur
$f\left(\left[-\frac{3}{2},+\infty\right[\right)=\left[f\left(-\frac{3}{2}\right),\underset{+\infty}{\mbox{lim}}f\right[=[-1,+\infty[$. Notons encore $f$
l'application de $\left[-\frac{3}{2},+\infty\right[$ dans $[-1,+\infty[$ qui à $x$ associe $\sqrt{2x+3}-1$. Soient alors
$x\in[-\frac{3}{2},+\infty[$ et $y\in[-1,+\infty[$.

$$f(x)=y\Leftrightarrow\sqrt{2x+3}-1=y\Leftrightarrow x=\frac{1}{2}(-3+(y+1)^2)\Leftrightarrow x=\frac{y^2}{2}+y-1.$$
En résumé, $\forall x\in\left[-\frac{3}{2},+\infty\right[,\;\forall y\in[-1,+\infty[,\;y=g(x)\Leftrightarrow x=\frac{y^2}{2}+y-1$. On vient
de trouver $g^{-1}$~:~

\begin{center}
\shadowbox{
$\forall x\in[-1,+\infty[,\;g^{-1}(x)=\frac{x^2}{2}+x-1$.
}
\end{center}}
    \item \question{$f(x)=\frac{x}{1+|x|}$, $I=\Rr$.}
\reponse{$f$ est définie sur $\Rr$, impaire.
Pour $x\in[0,+\infty[$, $0\leq f(x)=\frac{x}{1+x}<\frac{1+x}{1+x}=1$. Donc, $f([0,+\infty[)\subset[0,1[$.
Par parité, $f(]-\infty,0])\subset]-1,0]$ et même $f(]-\infty,0[)\subset]-1,0[$ car l'image par $f$ d'un réel
strictement négatif est un réel strictement négatif.
Finalement, $f(\Rr)\subset]-1,1[$.
Vérifions alors que $f$ réalise une bijection de $\Rr$ sur $]-1,1[$.
Soit $y\in[0,1[$ et $x\in\Rr$. L'égalité $f(x)=y$ impose à $x$ d'être dans $[0,+\infty[$. Mais alors

$$f(x)=y\Leftrightarrow\frac{x}{1+x}=y\Leftrightarrow x=\frac{y}{1-y}.$$
Le réel $x$ obtenu est bien défini, car $y\neq1$, et positif, car $y\in[0,1[$. On a montré que~:

$$\forall y\in[0,1[,\;\exists!x\in\Rr/\;y=f(x)\;(\mbox{à savoir}\;x=\frac{y}{1-y}).$$
Soit $y\in]-1,0[$ et $x\in\Rr$. L'égalité $f(x)=y$ impose à $x$ d'être dans $]-\infty,0[$. Mais alors

$$f(x)=y\Leftrightarrow\frac{x}{1-x}=y\Leftrightarrow x=\frac{y}{1+y}.$$
Le réel $x$ obtenu est bien défini, car $y\neq-1$, et strictement négatif, car $y\in]-1,0[$. On a montré que~:

$$\forall y\in]-1,0[,\;\exists!x\in\Rr/\;y=f(x)\;(\mbox{à savoir}\;x=\frac{y}{1+y}).$$
Finalement,

$$\forall y\in]-1,1[,\;\exists!x\in\Rr/\;y=f(x),$$
ce qui montre que $f$ réalise une bijection de $\Rr$ sur $]-1,1[$. De plus, pour $y\in]-1,1[$ donné,
$f^{-1}(y)=\frac{y}{1-y}$ si $y\geq0$ et $f^{-1}(y)=\frac{y}{1+y}$ si $y<0$. Dans tous les cas, on a
$f^{-1}(y)=\frac{y}{1-|y|}$.

En notant encore $f$ l'application de $\Rr$ dans $]-1,1[$ qui à $x$ associe $\frac{x}{1+|x|}$, on a donc
\begin{center}
\shadowbox{
$\forall x\in]-1,1[,\;f^{-1}(x)=\frac{x}{1-|x|}.$
}
\end{center}}
\end{enumerate}
}
