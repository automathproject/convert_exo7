\uuid{duCO}
\exo7id{90}
\auteur{cousquer}
\datecreate{2003-10-01}
\isIndication{false}
\isCorrection{true}
\chapitre{Nombres complexes}
\sousChapitre{Trigonométrie}

\contenu{
\texte{
A quelle condition sur le réel $m$ l'équation
$\sqrt{3}\cos(x)+\sin(x)=m$ a-t-elle une solution réelle ? Résoudre
cette équation pour $m=\sqrt{2}$.
}
\reponse{
L'équation
$\sqrt{3}\cos(x)-\sin(x)=m$ a des solutions ssi $m\in\left[-2,2\right]$ et pour
  $m=\sqrt{2}$, les solutions sont $x=\pi/12+2k\pi$ ou
  $x=-5\pi/12+2k\pi$, $k\in\mathbb{Z}$.
}
}
