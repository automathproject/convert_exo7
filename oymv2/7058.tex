\uuid{GDlk}
\exo7id{7058}
\auteur{megy}
\datecreate{2017-01-08}
\isIndication{false}
\isCorrection{true}
\chapitre{Géométrie affine euclidienne}
\sousChapitre{Géométrie affine euclidienne du plan}

\contenu{
\texte{
% tags : construction de cercles, cas particulier de CCC
Dans tout l'exercice, on fixe $R>0$. Dénombrer et construire les cercles de rayon $R$ tangents à :
}
\begin{enumerate}
    \item \question{deux cercles distincts $\mathcal C_1$ et $\mathcal C_2$;}
    \item \question{deux droites sécantes $\mathcal D_1$ et $\mathcal D_2$;}
    \item \question{un cercle $\mathcal C$ et une droite $\mathcal D$.}
\reponse{
Tracer le lieu des points à distance $R$ des cercles et droites en présence. Leurs éventuels points d'intersection fournissent des solutions.
}
\end{enumerate}
}
