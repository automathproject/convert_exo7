\uuid{7pTC}
\exo7id{5418}
\auteur{rouget}
\datecreate{2010-07-06}
\isIndication{false}
\isCorrection{true}
\chapitre{Développement limité}
\sousChapitre{Formule de Taylor}

\contenu{
\texte{
Soit $f$ une fonction de classe $C^3$ sur $\Rr$ vérifiant~:~$\forall(x,y)\in\Rr^2,\;f(x+y)f(x-y)\leq(f(x))^2$.
Montrer que $\forall x\in\Rr,\;f(x)f''(x)\leq(f'(x))^2$ (Indication. Appliquer la formule de \textsc{Taylor}-\textsc{Laplace} entre $x$ et $x+y$ puis entre $x$ et $x-y$).
}
\reponse{
Soit $(x,y)\in\Rr\times\Rr$. Puisque $f$ est de classe $C^3$ sur $\Rr$, la formule de \textsc{Taylor}-\textsc{Laplace} à l'ordre $2$ permet d'écrire

$$
\begin{array}{l}
f(x+y)=f(x)+yf'(x)+\frac{y^2}{2}f''(x)+\int_{x}^{x+y}\frac{(x+y-t)^2}{2}f^{(3)}(t)\;dt\;\mbox{et}\\
f(x-y)=f(x)-yf'(x)+\frac{y^2}{2}f''(x)\int_{x}^{x-y}\frac{(x-y-t)^2}{2}f^{(3)}(t)\;dt.
\end{array}$$

Donc,

\begin{align*}\ensuremath
(f(x)^2)&\geq f(x+y)f(x-y)\\
 &=(f(x)+yf'(x)+\frac{y^2}{2}f''(x)+\int_{x}^{x+y}\frac{(x+y-t)^2}{2}f^{(3)}(t)\;dt)\times\\
 &\quad(f(x)-yf'(x)+\frac{y^2}{2}f''(x)+\int_{x}^{x-y}\frac{(x-y-t)^2}{2}f^{(3)}(t)\;dt)\\
 &=(f(x))^2+y^2(f(x)f''(x)-(f'(x))^2)\\
 &\quad+(f(x)-yf'(x)+\frac{y^2}{2}f''(x))\int_{x}^{x+y}\frac{(x+y-t)^2}{2}f^{(3)}(t)\;dt\\
 &\quad+(f(x)+yf'(x)+\frac{y^2}{2}f''(x))\int_{x}^{x-y}\frac{(x-y-t)^2}{2}f^{(3)}(t)\;dt\;(*)
\end{align*}

Maintenant, pour $y\in[-1,1]$, ($f^{(3)}$ étant continue sur $\Rr$ et donc continue sur le segment $[-1,1]$),

$$\left|\int_{x}^{x+y}\frac{(x+y-t)^2}{2}f^{(3)}(t)\;dt\right|\leq|y|.\frac{y^2}{2}\mbox{Max}\{|f^{(3)}(t)|,\;t\in[x-1,x+1]\},$$

et donc,

$$\frac{1}{y^2}\left|\int_{x}^{x+y}\frac{(x+y-t)^2}{2}f^{(3)}(t)\;dt\right|\leq|y|\mbox{Max}\{|f^{(3)}(t)|,\;t\in[x-1,x+1]\}.$$

Cette dernière expression tend vers $0$ quand $y$ tend vers $0$. On en déduit que $\frac{1}{y^2}\left|\int_{x}^{x+y}\frac{(x+y-t)^2}{2}f^{(3)}(t)\;dt\right|$ tend vers $0$ quand $y$ tend vers $0$. De même, $\frac{1}{y^2}\left|\int_{x}^{x-y}\frac{(x-y-t)^2}{2}f^{(3)}(t)\;dt\right|$ tend vers $0$ quand $y$ tend vers $0$.

On simplifie alors $(f(x)^2$ dans les deux membres de $(*)$. On divise les deux nouveaux membres par $y^2$ pour $y\neq 0$ puis on fait tendre $y$ vers $0$ à $x$ fixé. On obtient $0\geq f(x)f''(x)-(f'(x))^2$, qui est l'inégalité demandée.
}
}
