\uuid{TZH0}
\exo7id{191}
\auteur{bodin}
\datecreate{1998-09-01}
\isIndication{true}
\isCorrection{true}
\chapitre{Injection, surjection, bijection}
\sousChapitre{Injection, surjection}

\contenu{
\texte{
Soit $f  : \Rr \rightarrow \Rr$ d\'efinie par $f(x) = 2x/(1+x^2)$.
}
\begin{enumerate}
    \item \question{$f$ est-elle injective ? surjective ?}
\reponse{$f$ n'est pas injective car $f(2) = \frac 45 = f(\frac12)$.
  $f$ n'est pas surjective car $y=2$ n'a pas d'ant\'ec\'edent: en effet 
  l'\'equation $f(x)=2$ devient $2x=2(1+x^2)$ soit $x^2-x+1=0$ qui n'a pas de solutions r\'eelles.}
    \item \question{Montrer que $f(\Rr)=[-1,1]$.}
\reponse{$f(x)=y$ est \'equivalent \`a l'\'equation 
  $yx^2-2x+y=0$. Cette \'equation a des solutions $x$ si et seulement si
  $\Delta = 4-4y^2 \geq 0$ donc il y a des solutions si et seulement si $y\in[-1,1]$. Nous venons de montrer que $f(\Rr)$ est exactement $[-1,1]$.}
    \item \question{Montrer que la restriction $g  : [-1,1] \rightarrow [-1,1]$  $g(x) = f(x)$
est une bijection.}
\reponse{Soit $y\in[-1,1]\setminus\{0\}$ alors les solutions $x$ possibles de l'\'equation $g(x)=y$ sont $x=\frac{1-\sqrt{1-y^2}}{y}$ ou
  $x=\frac{1+\sqrt{1-y^2}}{y}$. La seule solution $x\in[-1,1]$ est
  $x=\frac{1-\sqrt{1-y^2}}{y}$ en effet
  $x=\frac{1-\sqrt{1-y^2}}{y}=\frac{y}{1+\sqrt{1-y^2}} \in[-1,1]$.
  Pour $y=0$, la seule solution de l'\'equation $g(x)=0$ est $x=0$.
  Donc pour $g : [-1,1] \longrightarrow [-1,1]$ nous avons trouv\'e un inverse $h : [-1,1] \longrightarrow [-1,1]$ d\'efini par
  $h(y) = \frac{1-\sqrt{1-y^2}}{y}$ si $y\neq0$ et $h(0)=0$. Donc $g$ est une bijection.}
    \item \question{Retrouver ce r\'esultat en \'etudiant les variations de $f$.}
\reponse{$f'(x) = \frac{2-2x^2}{1+x^2}$, donc $f'$ est strictement positive sur $]-1,1[$ donc $f$ est strictement croissante sur $[-1,1]$ avec $f(-1)=-1$ et $f(1)=1$. Donc la restriction
  de $f$, appelée $g : [-1,1] \longrightarrow [-1,1]$, est une bijection.}
\indication{\begin{enumerate}
    \item $f$ n'est ni injective, ni surjective.
    \item Pour $y\in \Rr$, r\'esoudre l'\'equation $f(x)=y$.
    \item On pourra exhiber l'inverse.
\end{enumerate}}
\end{enumerate}
}
