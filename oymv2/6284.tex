\uuid{Ocqm}
\exo7id{6284}
\auteur{mayer}
\datecreate{2011-10-16}
\isIndication{false}
\isCorrection{false}
\chapitre{Sous-variété}
\sousChapitre{Sous-variété}

\contenu{
\texte{
\label{exn14}
Soit $f: \Rr^n \to \Rr$ un polyn\^ome homogène de degré
$\alpha >0$ à $n$ variables.
}
\begin{enumerate}
    \item \question{En calculant la dérivée de $\lambda \mapsto f(\lambda x)$ de
deux manières différentes, établir l'identité d'Euler:
$$ \sum_{i=1}^n x_i \frac{\partial f}{\partial x_i} = \alpha\, f(x)
\quad  \text{pour tout} \;\; x\in \Rr^n \; .$$}
    \item \question{Soit $a$ un réel non nul. Montrer que $X_a =
f^{-1}(\{a\})$ est une sous-variété de dimension $n-1$ de
$\Rr^n$. \'Etablir ensuite que, pour $a_1>a_2>0$, $X_{a_1}$ et
$X_{a_2}$ sont difféomorphes.}
    \item \question{Supposons que $\varphi$ est un difféomorphisme de $\Rr^n$ avec $\varphi ( X_{a_1} )
=X_{a_2}$ et soit $p\in X_{a_1}$. Exprimer l'espace tangent
$T_{\varphi (p)}X_{a_2}$ en fonction de $T_pX_{a_1}$.}
\end{enumerate}
}
