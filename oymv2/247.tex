\uuid{HoG9}
\exo7id{247}
\auteur{ridde}
\datecreate{1999-11-01}
\isIndication{false}
\isCorrection{false}
\chapitre{Dénombrement}
\sousChapitre{Autre}

\contenu{
\texte{
L'histoire: $n$ personnes apportent chacune un cadeau \`{a} une f\^{e}te, et
chacun tire au sort un cadeau dans le tas form\'{e} par tous les
pr\'{e}sents apport\'{e}s.\ Quelle est la probabilit\'{e} qu'au moins une
personne reparte avec son cadeau? Que devient cette probabilit\'{e} quand le
nombre de personnes devient tr\`{e}s grand, i.e.: $n\rightarrow \infty $?
(On remarquera que l'intuition met en \'{e}vidence deux effets
contradictoires: plus de personnes c'est plus de proba qu'une personne ait
son cadeau car... il y a plus de personnes, mais c'est aussi plus de
cadeaux, donc une proportion plus \'{e}lev\'{e}e de cadeaux ``acceptables'').


Soit $S_n  = \sigma (\left\{ 1, \ldots, n\right\})$. On dit que $\sigma
\in S_n$ est un d\'erangement si $\forall i \in \left\{ 1, \ldots, n\right\}
\, \, \sigma (i) \neq i$. On note $A_i = \left\{ \sigma \in S_n /
\sigma (i) = i\right\}$ et $D_n$ l'ensemble des d\'erangements.
}
\begin{enumerate}
    \item \question{Calculer $\mathrm{Card} (A_i)$.}
    \item \question{Exprimer $S_n - D_n$ en fonction des $A_i$.}
    \item \question{En d\'eduire $\mathrm{Card} (D_n)$ (on pourra utiliser l'exercice \ref{ex105}).}
    \item \question{D\'eterminer la limite de $\dfrac{\mathrm{Card}{D_n}}{\mathrm{Card}{S_n}}$. (on rappelle
que $\lim\limits_{n \rightarrow  + \infty} (1 + x + \ldots + \frac{x^n}{n ! }) = e^x$).}
\end{enumerate}
}
