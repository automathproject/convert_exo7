\uuid{6735}
\auteur{queffelec}
\datecreate{2011-10-16}

\contenu{
\texte{
Soit $f$ une fonction holomorphe sur un ouvert connexe
$\Omega $. Soit $c$ un point de $\Omega $ et $r_0>0$ tel que
$D(c,r_0)\subset \Omega $, où $D(c,r_0)$ est le disque ouvert de centre
$c$ et de rayon
$r_0$. On pose $\mu (r)={1\over 2\pi}\int_0^{2\pi}f(c+re^{i\theta
})d\theta $ pour $0<r<r_0$.
}
\begin{enumerate}
    \item \question{Montrer que 
$$\lim_{r\to 0^+}\mu (r)=f(c)$$}
    \item \question{On suppose $f'(z)$ continue. Montrer que $\mu $ est constante (on
montrera que $\displaystyle{d\mu \over dr}=0$ en dérivant sous le signe
d'intégration).

Soit maintenant $M=\sup_{z\in \Omega }\vert f(z)\vert$ et on suppose
qu'il existe $c\in \Omega $ tel que $\vert f(c)\vert =M$.}
    \item \question{Montrer que $M=\vert f(c+re^{i\theta })$ où $r>0$ est tel que
$D(c,r)\subset \Omega $.}
    \item \question{Soit $V=\{ z\in \Omega \vert \ \vert f(z)\vert =M\}$. Montrer que
$V$ est à la fois un ouvert et un fermé de $\Omega $. En déduire le
principe du maximum : si $f$ atteint son maximum en un point d'un
ouvert connexe $\Omega $, alors $f$ est constante.}
\end{enumerate}
}
