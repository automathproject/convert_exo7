\uuid{IIZS}
\exo7id{2304}
\auteur{barraud}
\datecreate{2008-04-24}
\isIndication{false}
\isCorrection{true}
\chapitre{Anneau, corps}
\sousChapitre{Anneau, corps}

\contenu{
\texte{

}
\begin{enumerate}
    \item \question{Quels sont les restes des division de $10^{100}$ par $13$ et par $19$ ?}
    \item \question{Quel est le reste de la  division de $10^{100}$ par $247=13\cdot 19$ ? 
En d\'eduire que  $10^{99}+1$ est multiple de $247$.}
\reponse{
$13$ est premier et $100=12\cdot8+4$ donc $10^{100}\equiv
10^{4}\equiv(-3)^{4}\equiv3\equiv-10[13]$. De même
$10^{100}\equiv10^{-8}\equiv2^{8}\equiv9\equiv-10[19]$. En utilisant le
lemme chinois, on en déduit que $10^{100}\equiv-10[247]$. Comme
$\pgcd(10,247)=1$, on peut simplifier cette expression par $10$ et on a
$10^{99}\equiv -1[247]$, et donc $247|10^{99}+1$.
}
\end{enumerate}
}
