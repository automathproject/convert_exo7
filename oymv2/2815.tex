\uuid{2815}
\auteur{burnol}
\datecreate{2009-12-15}
\isIndication{false}
\isCorrection{true}
\chapitre{Formule de Cauchy}
\sousChapitre{Formule de Cauchy}

\contenu{
\texte{
Soit $C$ le cercle unité parcouru dans le sens
direct. Calculer 
\[ \int_C \left( z + \frac1z \right)^{n}\frac{dz}z\quad(
n\in\Nn)\] en développant par la formule du binôme et en
utilisant les valeurs connues de $\int_C z^k dz$,
$k\in\Zz$. En déduire $\int_{-\pi}^{+\pi} \cos^{n}
t\,dt$. En déduire la valeur de $\int_{0}^{\pi/2} \cos^{n}
t\,dt$ pour $n$ pair:
\[ I_m = \int_0^{\frac\pi2} \cos^{2m} t\,dt =
\frac{1.3.\cdots.(2m-1)}{2.4.\cdots.(2m)}\frac\pi2\]
}
\reponse{
Comme
$$\left( z+\frac{1}{z} \right)^n = \sum_{k=0}^n \binom{n}{k}z^k z^{k-n}$$
on a
$$\left( z+\frac{1}{z} \right)^n \frac{1}{z}= \sum_{k=0}^n \binom{n}{k} z^{2k-n-1}.$$
Or $\int_C z^j \, dz \neq 0$ si et seulement si $j=-1$. Le seul terme de la somme pr\'ec\'edente qui donne une
contribution non nulle \`a l'int\'egrale est lorsque $k$ v\'erifie $2k-n-1=-1$.
Notons que ceci est possible seulement si $n$ est un nombre pair! D'o\`u :
$$\int_C \left( z+\frac{1}{z} \right)^n \frac{dz}{z}=0 \quad \text{si $n$ est impair}.$$
Sinon, si $n=2k$ est pair, on a :
$$\int_C \left( z+\frac{1}{z} \right)^{2k} \frac{dz}{z}=2i\pi \binom{2k}{k}.$$
Comme $\cos \, t=\frac{1}{2}(e^{it}+e^{-it})$,
$$\int_{-\pi} ^\pi \cos ^n t \, dt =\frac{1}{2^n} \int_{-\pi} ^\pi(e^{it}+e^{-it})^n \, \frac{ie^{it}dt}{ie^{it}}=
\frac{-i}{2^n}\int_C \left( z+\frac{1}{z} \right)^{n} \frac{dz}{z}.$$
D'o\`u $\int_{-\pi}^\pi \cos\, n t dt=0$ si $n$ est impair et
$$\int_{-\pi}^\pi \cos ^{2k}t\, dt =\frac{\pi}{2^{2k-1}}\binom{2k}{k}.$$
Par p\'eriodicit\'e du cosinus ceci donne :
$$I_k =\int_0^{\pi /2} \cos ^{2k}t\, dt =\frac{1}{4} \frac{\pi}{2^{2k-1}}\binom{2k}{k}=\frac{\pi}{2}
\frac{1\cdot 3\cdot ...\cdot (2k-1)}{2\cdot 4\cdot ...\cdot 2k}.$$
}
}
