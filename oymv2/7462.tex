\uuid{7462}
\auteur{mourougane}
\datecreate{2021-08-10}
\isIndication{false}
\isCorrection{false}
\chapitre{Géométrie affine dans le plan et dans l'espace}
\sousChapitre{Géométrie affine dans le plan et dans l'espace}

\contenu{
\texte{
Soit $E$ un espace affine euclidien de dimension 3 muni d'un repère 
cartésien orthonormé. On note $v$ la transformation de $E$ dans $E$
qui  envoie le point de coordonnées $(x,y,z)$ sur le point de 
 coordonnées $(x',y',z')$ définies par:
 $$ x' = \frac{2x-2y+z+1}{3}; y' =\frac{2x+y-2z+2}{3} ; z'
 =\frac{x+2y+2z+5}{3}.$$  
  
 Montrer que $v$ est une isométrie de $E$. Préciser de quel type
 d'isométrie il s'agit. Expliciter son axe et son vecteur de
 glissement.
}
}
