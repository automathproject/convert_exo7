\uuid{opPX}
\exo7id{6333}
\auteur{queffelec}
\datecreate{2011-10-16}
\isIndication{false}
\isCorrection{false}
\chapitre{Théorème de Cauchy-Lipschitz}
\sousChapitre{Théorème de Cauchy-Lipschitz}

\contenu{
\texte{
Soit $I$ un intervalle ouvert de $\Rr$, $E=\Rr^n$ , $f(t,x)$ une fonction
continue de $I\times E$ dans $E$ telle que  $||f(t,x_1)-f(t,x_2)||\leq k(t)\
||x_1-x_2||$, où $k$ est une fonction continue $\geq0$ définie sur $I$.
}
\begin{enumerate}
    \item \question{On considère $J$ intervalle compact $\subset I$ et l'opérateur $T$ défini
sur $C(J,\Rr^n)$ par 
$$Tx(t)=x_0+\int_{t_0}^tf(s,x(s))\ ds;$$ montrer que pour $p$ assez grand, $T^p$
est contractante;  en déduire que l'équation
$x'=f(t,x)$ admet une unique solution définie sur
$J$ tout entier telle que
$x(t_0)=x_0$.}
    \item \question{Montrer que l'équation $x'=f(t,x)$ admet une unique solution telle que
$x(t_0)=x_0$, définie sur
$I$ tout entier (on pourra écrire $I$ comme union d'intervalles compacts).}
    \item \question{Exemples : Montrer que les solutions maximales des équations 

$y''=-\sin y,y(0)=a, y'(0)=b$ (qu'on mettra
sous forme canonique), et $x'=A(t).x,
x(0)=x_0$ où $A(t)\in {\cal L}({\Rr^n})$ est constituée de fonctions continues
sur
$\Rr$, sont définies sur
$\Rr$ tout entier.}
\end{enumerate}
}
