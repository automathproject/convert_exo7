\uuid{635}
\auteur{gourio}
\datecreate{2001-09-01}
\isIndication{true}
\isCorrection{true}
\chapitre{Continuité, limite et étude de fonctions réelles}
\sousChapitre{Limite de fonctions}

\contenu{
\texte{
Calculer :
$$\lim\limits_{x\rightarrow 0}\frac{x}{2+\sin \frac{1}{x}}
,\ \ \lim\limits_{x\rightarrow +\infty }(\ln (1+e^{-x}))^{\frac{1}{x}}
,\ \ \lim\limits_{x\rightarrow 0^{+}}x^{\frac{1}{\ln (e^{x}-1)}}.$$
}
\indication{R\'{e}ponses : $0,\frac{1}{e},e.$
\begin{enumerate}
 \item Borner $\sin\frac 1x$.
 \item Utiliser que $\ln(1+t) = t \cdot \mu(t)$, pour une certaine fonction $\mu$ qui vérifie $\mu(t) \to 1$ lorsque $t\to 0$.
 \item Utiliser que $e^t-1 = t \cdot \mu(t)$, pour une certaine fonction $\mu$ qui vérifie $\mu(t) \to 1$ lorsque $t\to 0$.
\end{enumerate}}
\reponse{
\begin{enumerate}
 \item Comme $-1 \le \sin \frac 1x \le +1$ alors $1 \le 2 + \sin \frac 1x \le +3$.
Donc pour $x>0$, nous obtenons $\frac x3 \le \frac{x}{2+\sin \frac{1}{x}} \le x$.
On obtient une inégalité similaire pour $x<0$.
Cela implique $\lim\limits_{x\rightarrow 0}\frac{x}{2+\sin \frac{1}{x}} = 0$.

 \item Sachant que $\frac{\ln (1+t)}{t} \to 1$ lorsque $t \to 0$, on peut le reformuler
ainsi $\ln(1+t) = t \cdot \mu(t)$, pour une certaine fonction $\mu$ qui vérifie $\mu(t) \to 1$ lorsque $t\to 0$.
Donc $\ln (1+e^{-x}) = e^{-x} \mu(e^{-x})$.
Maintenant 

\begin{align*}
(\ln (1+e^{-x}))^{\frac{1}{x}} 
  &= \exp \left(\frac{1}{x} \ln\left( \ln (1+e^{-x}) \right)  \right) \\
  &= \exp \left(\frac{1}{x} \ln \left( e^{-x} \mu(e^{-x}) \right)  \right) \\
  &= \exp \left(\frac{1}{x} \left( -x + \ln \mu(e^{-x}) \right)  \right) \\
  &= \exp \left( -1 + \frac{\ln \mu(e^{-x})}{x}  \right) \\
\end{align*}
$\mu(e^{-x}) \to 1$ donc $\ln\mu(e^{-x}) \to 0$, donc $\frac{\ln \mu(e^{-x})}{x} \to 0$ lorsque $x \to +\infty$.

Bilan : la limite est $\exp(-1)=\frac 1 e$.
 \item


 \item Sachant $\frac{e^x - 1}{x} \to 1$ lorsque $x\to 0$, on reformule ceci
en $e^x-1 = x \cdot \mu(x)$, pour une certaine fonction $\mu$ qui vérifie $\mu(x) \to 1$ lorsque $x\to 0$.
Cela donne $\ln (e^x-1) = \ln (x \cdot \mu(x)) = \ln x + \ln \mu(x).$
\begin{align*}
 x^{\frac{1}{\ln (e^{x}-1)}} 
  &= \exp\left( \frac{1}{\ln (e^{x}-1)} \ln x  \right) \\
  &= \exp\left( \frac{1}{\ln x + \ln \mu(x)} \ln x  \right) \\
  &= \exp\left( \frac{1}{1 + \frac{\ln \mu(x)}{\ln x}} \right) \\
\end{align*}
Maintenant $\mu(x) \to 1$ donc $\ln \mu(x) \to 0$, et $\ln x \to - \infty$ lorsque $x \to 0$.
Donc $\frac{\ln \mu(x)}{\ln x} \to 0$.
Cela donne 
$$\lim\limits_{x\rightarrow 0^{+}}x^{\frac{1}{\ln (e^{x}-1)}} = \lim\limits_{x\rightarrow 0^{+}} \exp\left( \frac{1}{1 + \frac{\ln \mu(x)}{\ln x}} \right) = \exp\left(1\right) = e.$$
}
}
