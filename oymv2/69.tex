\uuid{RfLM}
\exo7id{69}
\auteur{bodin}
\datecreate{1998-09-01}
\isIndication{true}
\isCorrection{true}
\chapitre{Nombres complexes}
\sousChapitre{Géométrie}

\contenu{
\texte{
Montrer que pour $u,v \in \Cc$, on a $|u+v|^2+|u-v|^2=2(|u|^2+|v|^2).$
Donner une interprétation géométrique.
}
\indication{Pour l'interprétation géométrique cherchez le parallélogramme.}
\reponse{
$$|u+v|^2+|u-v|^2=(u+v)(\bar u +\bar v) + (u-v)(\bar u -\bar v)
= 2u\bar u+2v\bar v = 2|u|^2+2|v|^2.$$
G\'eom\'etriquement il s'agit de l'identit\'e du parall\'elogramme.
Les points d'affixes $0,u,v,u+v$ forment un parall\'elogramme.
$|u|$ et $|v|$ sont les longueurs des cot\'es, et
$|u+v|, |u-v|$ sont les longueurs des diagonales.
Il n'est pas \'evident de montrer ceci sans les nombres complexes !!
}
}
