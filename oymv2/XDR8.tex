\uuid{XDR8}
\exo7id{4176}
\auteur{quercia}
\datecreate{2010-03-11}
\isIndication{false}
\isCorrection{true}
\chapitre{Fonction de plusieurs variables}
\sousChapitre{Dérivée partielle}

\contenu{
\texte{
Soit $f : {\mathcal{M}_n(\C)} \to {\mathcal{M}_n(\C)}, M  \mapsto {\exp(M).}$
}
\begin{enumerate}
    \item \question{Montrer que $f$ est de classe $\mathcal{C}^1$ sur $\mathcal{M}_n(\C)$ et exprimer,
pour $M,H\in \mathcal{M}_n(\C)$, $d f_M(H)$ sous forme d'une série.}
    \item \question{Montrer qu'il existe un voisinage $V$ de $0$ dans $\mathcal{M}_n(\C)$
tel que pour toutes matrices $A,B\in V$ on a~:

$\exp(A) = \exp(B)  \Rightarrow  A=B$.}
    \item \question{Trouver une suite $(M_k)$ de matrices de $\mathcal{M}_2(\C)$ distinctes ayant même
exponentielle et convergeant vers une matrice $A$ (donc il n'existe
pas de voisinage de~$A$ sur lequel la restriction de~$f$ est injective).}
    \item \question{Donner de même un point de non injectivité locale dans $\mathcal{M}_2(\R)$.}
\reponse{
$d f_M(H) = \sum_{k=0}^\infty \frac1{k!}(M^{k-1}H + \dots + HM^{k-1})$.
les matrices $M_k = \begin{pmatrix}0&1/k\cr0&2i\pi\cr\end{pmatrix}$ sont toutes
semblables à~$M_\infty$ et ont même exponentielle $I$.
$M_k=\begin{pmatrix}0&2\pi+1/k\cr-4\pi^2/(2\pi+1/k)&0\cr\end{pmatrix}$.
}
\end{enumerate}
}
