\uuid{2407}
\auteur{mayer}
\datecreate{2003-10-01}
\isIndication{false}
\isCorrection{true}
\chapitre{Théorème du point fixe}
\sousChapitre{Théorème du point fixe}

\contenu{
\texte{
Soient $y\in {\cal C}([a,b])$ et $k\in {\cal C}([a,b]\times
[a,b])$ des fonctions continues. On se propose de r\'esoudre
l'\'equation (int\'egrale de Fredholm) suivante:
\begin{equation}\label{eq 11}
 x(s) -\int _a^b
k(s,t) x(t) \, dt =y(s) \quad \text{pour} \; s\in [a,b] \;
\end{equation}
 d'inconnue $x\in{\cal C}([a,b])$. Pour ce faire on
suppose que le "noyau" $k$ satisfait l'hypoth\`ese suivante:
$$ \lambda := \max _{a\leq s\leq b} \int _a^b |k(s,t)|\, dt <1 \quad \left( \mbox{ou
m\^eme} \quad \max_{a\leq s,t\leq b} |k(s,t)| <\frac{1}{b-a}
\;\right) .$$
}
\begin{enumerate}
    \item \question{Rappeler que $({\cal C}([a,b]),\| .\| _\infty)$ est
un espace complet.}
\reponse{! !}
    \item \question{Soit $x\in {\cal C}([a,b]) \mapsto Ax \in {\cal C}([a,b])$ l'application
donn\'ee par
$$(Ax)(s) :=\int _a^b k(s,t) x(t) \, dt +y(s)  \; .$$
Noter que (\ref{eq 11}) \'equivaut \`a $Ax=x$ et qu'on cherche donc
un point fixe de $x \mapsto Ax$. D\'eduire des hypoth\`eses faites
sur $k$ qu'un tel point fixe $x\in {\cal C}([a,b])$ existe et que
toute suite $A^n x_0$, $x_0\in{\cal C}([a,b])$, converge
uniform\'ement vers ce point fixe $x$.}
\reponse{\begin{align*}
\| Ax_1 - Ax_2\|_\infty 
  & = \| \int_a^b k(s,t)(x_1(t)-x_2(t))dt\|_\infty \\
  & \le  \int_a^b \|k(s,t)\|_\infty \|x_1(t)-x_2(t)\|_\infty dt \\
  & \le \|x_1(t)-x_2(t)\|_\infty \times \lambda \\
  & < \|x_1(t)-x_2(t)\|_\infty.\\
\end{align*}

Donc $A$ est contractante et l'espace ambiant ${\cal C}([a,b])$ est complet, par le
théorème du point fixe, $A$ admet un unique point fixe, $x$.
De plus, pour tout fonction $x_0 \in{\cal C}([a,b])$, la suite
$(A^nx_0)$ converge vers $x$, mais ici la norme est la norme uniforme donc
$\|A^nx_0-x\|_\infty$ tend vers $0$. Donc $(A^nx_0)$ converge uniformément vers $x$.}
    \item \question{{\it D\'ependance continue de la solution $x =
x(y)$}.

Soient $y_1,y_2\in{\cal C}([a,b])$ deux fonctions et $x_1,
x_2\in{\cal C}([a,b])$ les deux solutions associ\'ees de (\ref{eq
11}) ou, de fa\c{c}on \'equivalente, les points fixes des
applications associ\'ees $x\mapsto A_i x$. Montrer que
$$ \| x_1 - x_2\|_\infty = \| A_1 x_1 -A_2
x_2\|_\infty\leq \|y_1 -y_2\|_\infty+ \lambda \|x_1 -x_2 \|_\infty \;
.$$ En d\'eduire que
$$\|x_1 -x_2\|_\infty    \leq \frac{1}{1-\lambda}\| y_1
-y_2\|_\infty$$ et donc que la solution $x$ de (\ref{eq 11})
d\'epend continuement de la fonction $y$.}
\reponse{\begin{align*}
\|x_1-x_2\|_\infty & = \| A_1x_1-A_2x_2\|_\infty \quad \text{ car } A_ix_i=x_1, \\
                   &= \| \int_a^b k_1(s,t) x_1(t)  dt +y_1(s)+ \int _a^b k_2(s,t) x_2(t) \, dt + y_2(s)  \|_\infty \\
                    &\le \| \int_a^b k(s,t)(x_1(t)-x_2(t))dt\|_\infty + \|y_1-y_2\|_\infty \\
  &\le \lambda\|x_1-x_2\|_\infty+\|y_1-y_2\|_\infty \\   
\end{align*}
Donc 
$$\|x_1 -x_2\|_\infty    \leq \frac{1}{1-\lambda}\| y_1
-y_2\|_\infty,$$
ce qui exprime la dépendance continue de la solution par rapport à la fonction $y$.}
\end{enumerate}
}
