\uuid{4xBq}
\exo7id{2868}
\auteur{burnol}
\datecreate{2009-12-15}
\isIndication{false}
\isCorrection{false}
\chapitre{Théorème des résidus}
\sousChapitre{Théorème des résidus}

\contenu{
\texte{
Préciser pourquoi $\int_\Rr \frac{e^{i\xi x}}{1 + x^4}dx$
est une intégrale convergente pour $\xi\in\Rr$, est une
fonction réelle et paire de $\xi$, et utiliser un calcul de
résidus pour établir, pour $\xi\geq0$:
\[ \int_0^{+\infty} \frac{\cos(\xi x)}{1+x^4}dx = \frac\pi2
e^{-\xi/\sqrt2}\sin(\frac\xi{\sqrt2} + \frac\pi4)\]
Cette formule est-elle valable pour $\xi<0$?
}
}
