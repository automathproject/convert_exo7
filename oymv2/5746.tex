\uuid{MX2V}
\exo7id{5746}
\auteur{rouget}
\datecreate{2010-10-16}
\isIndication{false}
\isCorrection{true}
\chapitre{Série entière}
\sousChapitre{Calcul de la somme d'une série entière}

\contenu{
\texte{
Calculer les sommes suivantes dans leur intervalle ouvert de convergence après avoir déterminé le rayon de convergence de la série proposée.

\begin{center}
\begin{tabular}{llll}
\textbf{1) (**)} $\sum_{n=2}^{+\infty}\frac{1}{n(n-1)}x^n$&\textbf{2) (**)} $\sum_{n=0}^{+\infty}\frac{3n}{n+2}x^n$&\textbf{3) (** I)} $\sum_{n=0}^{+\infty}\frac{x^n}{2n+1}$&\textbf{4) (**)} $\sum_{n=0}^{+\infty}\frac{n}{(2n+1)!}x^n$\\
\textbf{5) (*)} $\sum_{n=0}^{+\infty}\frac{x^{4n}}{(4n)!}$&\textbf{6) (**)} $\sum_{n=0}^{+\infty}(\ch n)x^n$&\textbf{7) (** I)} $\sum_{n=1}^{+\infty}\left(\sum_{k=1}^{n}\frac{1}{k}\right)x^n$&\textbf{7)} $\sum_{n=0}^{+\infty}\frac{n^2+4n-1}{n!(n+2)}x^n$\\
\textbf{9) (** I)} $\sum_{n=1}^{+\infty}n^{(-1)^n}x^n$&\textbf{10) (*)} $\sum_{n=1}^{+\infty}(-1)^n\frac{x^{4n-1}}{4n}$&\textbf{11) (**)} $\sum_{n=0}^{+\infty}(n^2+1)2^{n+1}x^n$&\textbf{12) (**)} $\sum_{n=0}^{+\infty}(-1)^{n+1}nx^{2n+1}$
\end{tabular}

\textbf{13) (***)} $\sum_{n=0}^{+\infty}a_nx^n$ où $a_0 = a_1 = 1$ et 
$\forall n\in\Nn$, $a_{n+2}= a_{n+1} + a_n$\rule{7.7cm}{0cm}

\textbf{14) (**)} $\sum_{n=0}^{+\infty}a_nx^n$ où $a_n$ est le nombre de couples $(x,y)$ d'entiers naturels tels que $x+5y = n$.\rule{3.3cm}{0cm}
\end{center}
}
\reponse{
La règle de d'\textsc{Alembert} montre que la série proposée a un rayon de convergence égal à $1$.

\textbf{1ère solution.} Pour $x\in]-1,1[$, on pose $f(x)=\sum_{n=2}^{+\infty}\frac{1}{n(n-1)}x^n$.
$f$ est dérivable sur $]-1,1[$ et pour $x$ dans $]-1,1[$, 

\begin{center}
$f'(x) =\sum_{n=2}^{+\infty}\frac{1}{n-1}x^{n-1}=\sum_{n=1}^{+\infty}\frac{x^n}{n}= -\ln(1-x)$.
\end{center}

Puis, pour $x\in]-1,1[$, $f(x) = f(0) +\int_{0}^{x}f'(t) dt =(1-x)\ln(1-x) + x$.

\textbf{2ème solution.} Pour $x\in]-1,1[$, 

\begin{center}
$f(x) =\sum_{n=2}^{+\infty}\left(\frac{1}{n-1}-\frac{1}{n}\right)x^n =x\sum_{n=2}^{+\infty}\frac{x^{n-1}}{n-1}-\sum_{n=2}^{+\infty}\frac{x^n}{n}=-x\ln(1-x) +\ln(1-x) + x$.
\end{center}

\begin{center}
\shadowbox{
$\forall x\in]-1,1[$, $\sum_{n=2}^{+\infty}\frac{x^n}{n(n-1)}=-x\ln(1-x) +\ln(1-x) + x$.
}
\end{center}
La règle de d'\textsc{Alembert} montre que la série proposée a un rayon égal à $1$.
Pour $x\in]-1,1[\setminus\{0\}$

\begin{center}
$\sum_{n=0}^{+\infty}\frac{3n}{n+2}x^n =3\left(\sum_{n=0}^{+\infty}x^n - 2\sum_{n=0}^{+\infty}\frac{x^n}{n+2}\right) = 3\left(\frac{1}{1-x}- \frac{2}{x^2}\sum_{n=2}^{+\infty}\frac{x^n}{n}\right) = 3\left(\frac{1}{1-x}+\frac{2}{x^2}(x+\ln(1-x))\right)$
\end{center}

\begin{center}
\shadowbox{
$\forall x\in]-1,1[$, $\sum_{n=0}^{+\infty}\frac{3n}{n+2}x^n=\left\{
\begin{array}{l}3\left(\frac{1}{1-x}+\frac{2}{x^2}(x+\ln(1-x))\right)\;\text{si}\;x\in]-1,1[\setminus\{0\}\\
\rule{0mm}{4mm}0\;\text{si}\;x=0
\end{array}
\right.$.
}
\end{center}
La règle de d'\textsc{Alembert} montre que la série proposée a un rayon égal à $1$.

\textbullet~Soit $x\in]0,1[$.

\begin{align*}\ensuremath
\sum_{n=0}^{+\infty}\frac{x^n}{2n+1}&=\sum_{n=0}^{+\infty}\frac{\left(\sqrt{x}\right)^{2n}}{2n+1}=\frac{1}{\sqrt{x}}\sum_{n=0}^{+\infty}\frac{\left(\sqrt{x}\right)^{2n+1}}{2n+1}=\frac{1}{\sqrt{x}}(\ln(1+\sqrt{x}) -\ln(1-\sqrt{x}))\\
 &=\frac{\Argth(\sqrt{x})}{\sqrt{x}}.
\end{align*}	

\textbullet~Soit $x\in]-1,0[$.

\begin{align*}\ensuremath
\sum_{n=0}^{+\infty}\frac{x^n}{2n+1}&=\sum_{n=0}^{+\infty}(-1)^n\frac{(-x)^n}{2n+1}=\frac{1}{\sqrt{-x}}\sum_{n=0}^{+\infty}(-1)^n\frac{\left(\sqrt{-x}\right)^{2n+1}}{2n+1}=\frac{\Arctan(\sqrt{-x})}{\sqrt{-x}}
\end{align*}

\begin{center}
\shadowbox{
$\forall x\in]-1,1[$, $\sum_{n=0}^{+\infty}\frac{x^n}{2n+1}=\left\{
\begin{array}{l}
\frac{\Argth(\sqrt{x})}{\sqrt{x}}\;\text{si}\;x\in]0,1[\\
\rule[-2mm]{0mm}{4mm}1\;\text{si}\;x=0\\
\frac{\Arctan(\sqrt{-x})}{\sqrt{-x}}\;\text{si}\;x\in]-1,0[
\end{array}
\right.$.
}
\end{center}
La règle de d'\textsc{Alembert} montre que la série proposée a un rayon égal à $+\infty$. Pour $x$ réel, 

\begin{center}
$f(x) =\frac{1}{2}\sum_{n=0}^{+\infty}\frac{2n+1-1}{(2n+1)!}x^n=\frac{1}{2}\sum_{n=0}^{+\infty}\frac{1}{(2n)!}x^n-\frac{1}{2}\sum_{n=0}^{+\infty}\frac{1}{(2n+1)!}x^n$.
\end{center}

\textbullet~Si $x > 0$,  

\begin{center}
$f(x) =\frac{1}{2}\sum_{n=0}^{+\infty}\frac{1}{(2n)!}\left(\sqrt{x}\right)^{2n}-\frac{1}{2\sqrt{x}}\sum_{n=0}^{+\infty}\frac{1}{(2n+1)!}\left(\sqrt{x}\right)^{2n+1}=\frac{1}{2}\left(\ch(\sqrt{x})-\frac{1}{\sqrt{x}}\sh(\sqrt{x})\right)$.
\end{center}

\textbullet~Si $x<0$,  

\begin{center}
$f(x) =\frac{1}{2}\sum_{n=0}^{+\infty}(-1)^n\frac{1}{(2n)!}\left(\sqrt{-x}\right)^{2n}-\frac{1}{2\sqrt{-x}}\sum_{n=0}^{+\infty}(-1)^n\frac{1}{(2n+1)!}\left(\sqrt{-x}\right)^{2n+1}=\frac{1}{2}\left(\cos(\sqrt{-x})-\frac{1}{\sqrt{-x}}\sin(\sqrt{-x})\right)$.
\end{center}

\begin{center}
\shadowbox{
$\forall x\in\Rr$, $\sum_{n=0}^{+\infty}\frac{nx^n}{(2n+1)!}=\left\{
\begin{array}{l}
\frac{1}{2}\left(\ch(\sqrt{x})-\frac{1}{\sqrt{x}}\sh(\sqrt{x})\right)\;\text{si}\;x>0\\
\rule[-2mm]{0mm}{4mm}0\;\text{si}\;x=0\\
\frac{1}{2}\left(\cos(\sqrt{-x})-\frac{1}{\sqrt{-x}}\sin(\sqrt{-x})\right)\;\text{si}\;x<0
\end{array}
\right.$.
}
\end{center}
Immédiatement $R = +\infty$ et 

\begin{center}
\shadowbox{
$\forall x\in\Rr$, $\sum_{n=0}^{+\infty}\frac{x^{4n}}{(4n)!}=\frac{1}{2}(\cos x +\ch x)$.
}
\end{center}
$\ch n\underset{n\rightarrow+\infty}{\sim}\frac{e^n}{2}$ et donc $R=\frac{1}{e}$.
Pour $x$ dans $\left]-\frac{1}{e},\frac{1}{e}\right[$,

\begin{align*}\ensuremath 
\sum_{n=0}^{+\infty}\left(\ch n\right)x^n=\frac{1}{2}\left(\sum_{n=0}^{+\infty}(ex)^n+\sum_{n=0}^{+\infty}\left(\frac{x}{e}\right)^n\right)=\frac{1}{2}\left(\frac{1}{1-ex}+\frac{1}{1-\frac{x}{e}}\right)=\frac{1}{2}\frac{2-\left(e+\frac{1}{e}\right)x}{x^2-\left(e+\frac{1}{e}\right)x+1}\\
 &=\frac{1-x\ch1}{x^2-2x\ch1+1}.
\end{align*}

\begin{center}
\shadowbox{
$\forall x\in\left]-\frac{1}{e},\frac{1}{e}\right[$, $\sum_{n=0}^{+\infty}\left(\ch n\right)x^n=\frac{1-x\ch1}{x^2-2x\ch1+1}$.
}
\end{center}
La série proposée est le produit de \textsc{Cauchy} des séries entières  $\sum_{n=0}^{+\infty}x^n$ et  $\sum_{n=1}^{+\infty}\frac{x^n}{n}$ qui sont toutes deux de rayon $1$. Donc $R\geqslant1$. Mais d'autre part, pour tout entier naturel non nul $n$, $a_n=\sum_{k=1}^{n}\frac{1}{k}\geqslant1$ et $R\leqslant1$. Finalement $R = 1$. De plus, pour $x$ dans $]-1,1[$, $f(x)=\frac{1}{1-x}\times-\ln(1-x)=\frac{\ln(1-x)}{x-1}$.

\begin{center}
\shadowbox{
$\forall x\in\left]-1,1\right[$, $\sum_{n=1}^{+\infty}\left(\sum_{k=1}^{n}\frac{1}{k}\right)x^n=\frac{\ln(1-x)}{x-1}$.
}
\end{center}
La règle de d'\textsc{Alembert} montre que le rayon de convergence est égal à $+\infty$.

Pour $n$ entier naturel donné, $\frac{n^2+4n-1}{n!(n+2)}=\frac{n^3+5n^2+3n-1}{(n+2)!}$ puis 

\begin{align*}\ensuremath
n^3+5n^2+3n-1&= (n+2)(n+1)n+2n^2+n-1 = (n+2)(n+1)n+2(n+2)(n+1) - 5n - 5\\
 &= (n+2)(n+1)n + 2(n+2)(n+1) - 5(n+2) + 5
\end{align*}

Donc, pour tout réel $x$,

\begin{center}
$f(x) =\sum_{n=0}^{+\infty}\frac{(n+2)(n+1)n}{(n+2)!}x^n +2\sum_{n=0}^{+\infty}\frac{(n+2)(n+1)}{(n+2)!}x^n- 5\sum_{n=0}^{+\infty}\frac{n+2}{(n+2)!}x^n+ 5\sum_{n=0}^{+\infty}\frac{1}{(n+2)!}x^n.$
\end{center}

Ensuite $f(0) = -\frac{1}{2}$  et pour $x\neq 0$,

\begin{align*}
f(x)&=\sum_{n=1}^{+\infty}\frac{1}{(n-1)!}x^n +2\sum_{n=0}^{+\infty}\frac{1}{n!}x^n- 5\sum_{n=0}^{+\infty}\frac{1}{(n+1)!}x^n+ 5\sum_{n=0}^{+\infty}\frac{1}{(n+2)!}x^n\\
 &= xe^x + 2e^x -5\frac{e^x-1}{x}+ 5\frac{e^x-1-x}{x^2}= \frac{e^x(x^3+2x^2-5x+5) -5x}{x^2}.
\end{align*}

\begin{center}
\shadowbox{
$\forall x\in\Rr$, $\sum_{n=0}^{+\infty}\frac{n^2+4n-1}{n!(n+2)}x^n=\left\{
\begin{array}{l}
\frac{e^x(x^3+2x^2-5x+5) -5x}{x^2}\;\text{si}\;x\in\Rr^*\\
-\frac{1}{2}\;\text{si}\;x=0
\end{array}
\right.$.
}
\end{center}
Pour $n\in\Nn^*$,  $\frac{1}{n}\leqslant a_n=n^{(-1)^n}\leqslant n$ et donc $R = 1$. Pour $x$ dans $]-1,1[$, $f(x)=\sum_{k=0}^{+\infty}\frac{x^{2k+1}}{2k+1}+\sum_{k=1}^{+\infty}(2k)x^k$. Puis 

\begin{center}
$\sum_{k=1}^{+\infty}(2k)x^k= 2x\sum_{k=1}^{+\infty}kx^{k-1}= 2x\left(\sum_{k=0}^{+\infty}x^k\right)'= 2x\left(\frac{1}{1-x}\right)'=\frac{2x}{(1-x)^2}$.
\end{center}

\begin{center}
\shadowbox{
$\forall x\in]-1,1[$, $\sum_{n=1}^{+\infty}n^{(-1)^n}x^n=\Argth x+\frac{2x}{(1-x)^2}$.
}
\end{center}
$R = 1$. Pour $x$ réel non nul dans $]-1,1[$, $f(x)=-\frac{1}{x}\sum_{n=1}^{+\infty}(-1)^{n-1}\frac{(x^4)^n}{n}= -\frac{\ln(1+x^4)}{4x}$ et sinon $f(0) = 0$.

\begin{center}
\shadowbox{
$\forall x\in]-1,1[$, $\sum_{n=1}^{+\infty}(-1)^n\frac{x^{4n-1}}{4n}=\left\{
\begin{array}{l}
-\frac{\ln(1+x^4)}{4x}\;\text{si}\;x\neq0\\
\rule{0mm}{4mm}0\;\text{si}\;x=0
\end{array}
\right.$.
}
\end{center}
La règle de d'\textsc{Alembert} fournit $R=\frac{1}{2}$. Pour $x$ dans $\left]-\frac{1}{2},\frac{1}{2}\right[$,

\begin{align*}\ensuremath
\sum_{n=0}^{+\infty}(n^2+1)2^{n+1}x^n&=2\left(\sum_{n=0}^{+\infty}(n+2)(n+1)(2x)^n -3\sum_{n=0}^{+\infty}(n+1)(2x)^n +2\sum_{n=0}^{+\infty}(2x)^n\right)\\
  &= 2\left(\left(\sum_{n=0}^{+\infty}(2x)^n\right)''- 3\left(\sum_{n=0}^{+\infty}(2x)^n\right)'+2\sum_{n=0}^{+\infty}(2x)^n\right)
 = 2\left(2\frac{1}{1-2x}  - 3\frac{2}{(1-2x)^2}+\frac{4}{(1-2x)^3}\right)\\
 &=2\frac{2(1-2x)^2-6(1-2x)+8}{(1-2x)^3}=2\frac{8x^2+4x}{(1-2x)^3}.
\end{align*}
Pour $x = 1$, la suite $((-1)^{n+1}nx^{2n+1})$ n'est pas bornée et donc $R\geqslant 1$. Mais la série converge si $|x| < 1$ et $R\leqslant 1$. Finalement $R=1$.

Pour $x$ dans $]-1,1[$,

\begin{align*}\ensuremath
\sum_{n=0}^{+\infty}(-1)^{n+1}nx^{2n+1}&=\frac{1}{2}\left(\sum_{n=0}^{+\infty}(-1)^{n+1}(2n+2)x^{2n+1}-2\sum_{n=0}^{+\infty}(-1)^{n+1}x^{2n+1}\right)\\ 
 &=\frac{1}{2}\left(\left(\sum_{n=0}^{+\infty}(-1)^{n+1}x^{2n+2}\right)'+2x\sum_{n=0}^{+\infty}(-x^2)^n\right)=\frac{1}{2}\left(\left(\frac{-x^2}{1+x^2}\right)'+\frac{2x}{1+x^2}\right)\\
 &= -\frac{x(1+x^2)-x^3}{1+x^2)^2}+\frac{x}{1+x^2}=\frac{x^3}{1+x^2)^2}.
\end{align*}

\begin{center}
\shadowbox{
$\forall x\in]-1,1[$, $\sum_{n=0}^{+\infty}(-1)^{n+1}nx^{2n+1}=\frac{x^3}{1+x^2)^2}$.
}
\end{center}
\textbf{1ère solution.} Les racines de l'équation caractéristique $z^2 - z - 1 = 0$ sont $\alpha=\frac{1+\sqrt{5}}{2}$ et $\beta=\frac{1-\sqrt{5}}{2}$. On sait qu'il existe deux nombres réels $\lambda$ et $\mu$ tels que pour tout entier naturel $n$,

\begin{center}
$a_n =\lambda\left(\frac{1+\sqrt{5}}{2}\right)^n+\mu\left(\frac{1-\sqrt{5}}{2}\right)^n$.
\end{center}

Les égalités $n=0$ et $n=1$ fournissent

\begin{center}
$\left\{
\begin{array}{l}
\rule[-2mm]{0mm}{0mm}\lambda+\mu=1\\
\frac{1+\sqrt{5}}{2}\lambda+\frac{1-\sqrt{5}}{2}\mu=1
\end{array}
\right.\Leftrightarrow\left\{
\begin{array}{l}
\rule[-2mm]{0mm}{0mm}\lambda+\mu=1\\
\lambda-\mu=\frac{1}{\sqrt{5}}
\end{array}
\right.\Leftrightarrow
\left\{
\begin{array}{l}
\lambda=\frac{1}{2}\left(1+\frac{1}{\sqrt{5}}\right)\\
\mu=\frac{1}{2}\left(1-\frac{1}{\sqrt{5}}\right)\end{array}
\right.\Leftrightarrow\left\{
\begin{array}{l}
\lambda=\frac{1}{\sqrt{5}}\frac{1+\sqrt{5}}{2}\\
\mu=-\frac{1}{\sqrt{5}}\frac{1-\sqrt{5}}{2}
\end{array}
\right.$.
\end{center}

Finalement, pour tout entier naturel $n$, $a_n =\frac{1}{\sqrt{5}}\left(\frac{1+\sqrt{5}}{2}\right)^{n+1}-\frac{1}{\sqrt{5}}\left(\frac{1-\sqrt{5}}{2}\right)^{n+1}$.

Les séries entières respectivement associées aux suites $\left(\left(\frac{1+\sqrt{5}}{2}\right)^{n+1}\right)$ et $\left(\left(\frac{1-\sqrt{5}}{2}\right)^{n+1}\right)$ ont pour rayons respectifs $\left|\frac{1}{(1+\sqrt{5})/2}\right|=\frac{\sqrt{5}-1}{2}$ et $\left|\frac{1}{(1-\sqrt{5})/2}\right|=\frac{\sqrt{5}+1}{2}$. Ces rayons étant distincts, la série proposée a pour rayon

\begin{center}
$R=\text{Min}\left\{\frac{\sqrt{5}-1}{2},\frac{\sqrt{5}+1}{2}\right\}=\frac{\sqrt{5}-1}{2}$.
\end{center}

Pour $x$ dans $\left]-\frac{\sqrt{5}-1}{2},\frac{\sqrt{5}-1}{2}\right[$, on a

\begin{align*}\ensuremath
\sum_{n=0}^{+\infty}a_nx^n&=\frac{\alpha}{\sqrt{5}}\sum_{n=0}^{+\infty}(\alpha x)^n -\frac{\beta}{\sqrt{5}}\sum_{n=0}^{+\infty}(\beta x)^n
=\frac{1}{\sqrt{5}}\left(\frac{\alpha}{1-\alpha x}-\frac{\beta}{1-\beta x}\right)
=\frac{\alpha-\beta}{\sqrt{5}}\times\frac{1}{\alpha\beta x^2-(\alpha+\beta)x+1}\\
 &=\frac{1}{1-x-x^2}.
\end{align*}

\textbf{2ème solution.} Supposons à priori le rayon $R$ de la série proposée strictement positif. Pour $x$ dans $]-R,R[$, on a

\begin{align*}\ensuremath
f(x)&= 1+ x +\sum_{n=2}^{+\infty}a_nx^n= 1 + x +\sum_{n=0}^{+\infty}a_{n+2}x^{n+2}=1+x+\sum_{n=0}^{+\infty}(a_{n+1}+a_n)x^{n+2}\\
 &= 1 + x + x\sum_{n=0}^{+\infty}a_{n+1}x^{n+1}+ x^2\sum_{n=0}^{+\infty}a_nx^n\;(\text{les deux séries ont même rayon})\\ 
 &=1+x+x(f(x)-1)+ x^2 f(x).
\end{align*}

Donc, nécessairement $\forall x\in]-R,R[$, $f(x) =\frac{1}{1-x-x^2}$.

Réciproquement, la fraction rationnelle ci-dessus n'admet pas $0$ pour pôle et est donc développable en série entière. Le rayon de convergence de la série obtenue est le minimum des modules des pôles de $f$ à savoir $R=\frac{\sqrt{5}-1}{2}$. Notons $\sum_{n=0}^{+\infty}b_nx^n$ ce développement. Pour tout $x$ de $]-R,R[$, on a $\left(\sum_{n=0}^{+\infty}b_nx^n\right)(1-x-x^2)=1$ et donc $\sum_{n=0}^{+\infty}b_nx^n -\sum_{n=0}^{+\infty}b_nx^{n+1}-\sum_{n=0}^{+\infty}b_nx^{n+2}= 1$ ce qui s'écrit encore  $\sum_{n=0}^{+\infty}b_nx^n-\sum_{n=1}^{+\infty}b_{n-1}x_n -\sum_{n=2}^{+\infty}b_{n-2}x^n = 1$. Finalement 

\begin{center}
$\forall x\in]-R,R[$, $b_0 +(b_1-b_0)x+\sum_{n=2}^{+\infty}(bn-b_{n-1}-b_{n-2})x^n = 1$.
\end{center}

Par unicité des coefficients d'un développement en série entière, on a alors $b_0=b_1= 1$ et $\forall n\geqslant2$, $b_n = b_{n-1}+b{n-2}$. On en déduit alors par récurrence que $\forall n\in\Nn$, $b_n = a_n$.

\begin{center}
\shadowbox{
$\forall x\in\left]-\frac{\sqrt{5}-1}{2},\frac{\sqrt{5}-1}{2}\right[$, $\sum_{n=0}^{+\infty}a_nx^n=\frac{1}{1-x-x^2}$.
}
\end{center}

\textbf{Remarque.} En généralisant le travail précédent, on peut montrer que les suites associées aux développements en série entière des fractions rationnelles sont justement les suites vérifiant des relations de récurrence linéaire.
Pour tout entier naturel $n$, $1\leqslant a_n\leqslant n+1$. Donc $R = 1$.

On remarque que pour tout entier naturel $n$, $a_n =\sum_{k+5l=n}^{}1$. La série entière proposée est donc le produit de \textsc{Cauchy} des séries $\sum_{k=0}^{+\infty}x^k$ et $\sum_{l=0}^{+\infty}x^{5l}$. Pour $x$ dans $]-1,1[$, on a donc

\begin{center}
$f(x) =\left(\sum_{k=0}^{+\infty}x^k\right)\left(\sum_{l=0}^{+\infty}x^{5l}\right)=\frac{1}{1-x}\times\frac{1}{(1-x)^5}$.
\end{center}

\textbf{Remarque.} De combien de façons peut -on payer $100$ euros avec des pièces de $1$, $2$, $5$, $10$, $20$ et $50$ centimes d'euros, des pièces de $1$ et $2$ euros et des billets de $10$ et $20$ euros ? Soit $N$ le nombre de solutions. $N$ est le nombre de solutions en nombres entiers a,b,... de l'équation

\begin{center}
$a +2b +5c +10d +20e +50f +100g +200h+500k+1000i+2000j=10000$
\end{center}

et est donc le coefficient de $x^{10000}$ du développement en série entière de 

\begin{center}
$\frac{1}{(1-x)(1-x^2)(1-x^5)(1-x^{10})(1-x^{20})(1-x^{50})(1-x^{100})(1-x^{200})(1-x^{500})(1-x^{1000})(1-x^{2000})}$,
\end{center}

La remarque est néanmoins anecdotique et il semble bien préférable de dénombrer à la main le nombre de solutions. Les 
exercices \ref{ex:rou19x} et \ref{ex:rou20x} de cette planche font bien mieux comprendre à quel point les séries entières sont un outil intéressant pour les dénombrements.
}
}
