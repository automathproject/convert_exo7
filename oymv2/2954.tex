\uuid{pVWg}
\exo7id{2954}
\auteur{quercia}
\datecreate{2010-03-08}
\isIndication{false}
\isCorrection{true}
\chapitre{Nombres complexes}
\sousChapitre{Trigonométrie}

\contenu{
\texte{
Soit $\theta \in \R$.
}
\begin{enumerate}
    \item \question{Simplifier $\cos^4\theta + \cos^4\left(\theta + \frac \pi4\right) +
                \cos^4\left(\theta + \frac {2\pi}4\right) +
                \cos^4\left(\theta + \frac {3\pi}4\right)$.}
\reponse{$= 3/2$.}
    \item \question{Simplifier $\cos^6\theta + \cos^6\left(\theta + \frac \pi6\right) + \dots +
                \cos^6\left(\theta + \frac {5\pi}6\right)$.}
\reponse{$32\cos^6(\theta) = \cos6\theta + 6\cos4\theta + 15\cos2\theta
    + 10 \Rightarrow \Sigma = \frac {15}8$.}
    \item \question{Simplifier $\cos^{2p}\theta + \cos^{2p}\left(\theta + \frac \pi{2p}\right)
                + \dots + \cos^{2p}\left(\theta + \frac {(2p-1)\pi}{2p}\right)$.}
\reponse{$\Sigma_p = \frac {pC_{2p}^p}{2^{2p-1}}$.}
\end{enumerate}
}
