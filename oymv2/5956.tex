\uuid{5956}
\auteur{tumpach}
\datecreate{2010-11-11}

\contenu{
\texte{
Soient $f$ et $g$ deux fonctions de $L^{p}(\mu)$ avec
$1<p<+\infty$. Montrer que la fonction $N~:\mathbb{R} \rightarrow
\mathbb{R}$ d\'efinie par
$$
N(t) = \int_{\Omega} |f(x) + t \cdot g(x)|^{p}\,d\mu
$$
est diff\'erentiable et que sa d\'eriv\'ee en $t=0$ est donn\'ee
par~:
$$
\frac{dN}{dt}_{|t=0} = p \int_{\Omega} |f(x)|^{p-2}f(x)
g(x)\,d\mu,
$$
o\`u par convention $|f(x)|^{p-2}f(x) = 0$ lorsque $f=0$.
}
\reponse{
Soient $f$ et $g$ deux fonctions de $L^{p}(\mu)$ avec
$1<p<+\infty$. La fonction $\varphi(t) = |f(x)+\tan (x)|^{p}$ est de
classe $\mathcal{C}^1$ sur $\mathbb{R}$ et sa d\'eriv\'ee vaut
$$
\varphi'(t) = \lim_{h\rightarrow 0}\frac{|f(x) + \tan (x) + h
g(x)|^{p}-|f(x)+ \tan (x)|^p}{h} = p |f(x)+\tan (x)|^{p-2} (f(x)+ \tan (x))
g(x),
$$
lorsque $f(x)$ et $g(x)$ ont un sens, c'est-\`a-dire pour presque
tout $x$. De plus, d'apr\`es le th\'eor\`eme des accroissements
finis, on a
$$
\frac{|f(x) + \tan (x)|^{p}-|f(x)|^p}{t} = \varphi'(t_{0}) = p
|f(x)+t_0 g(x)|^{p-2} (f(x)+ t_0 g(x)) g(x),
$$
pour un certain $t_0$ compris entre $0$ et $t$. Ainsi pour
$|t|\leq 1$,
\begin{equation*}
\begin{array}{ll}
\left|\frac{|f(x) + \tan (x)|^{p}-|f(x)|^p}{t}\right| &=  p |f(x)+t_0
g(x)|^{p-1} |g(x)| \\& \leq p \left(|f(x)| + |g(x)|\right)^{p}\\ &
\leq  2^{p-1}\,p \left(|f(x)|^p + |g(x)|^p \right),
\end{array}
\end{equation*}
o\`u la premi\`ere in\'egalit\'e d\'ecoule de l'in\'egalit\'e
triangulaire et de la majoration $|g(x)| \leq (|f(x)| + |g(x)|)$,
et o\`u la deuxi\`eme in\'egalit\'e provient de la convexit\'e de
la fonction $x\mapsto x^p$ pour $p>1$ impliquant en particulier~:
$\left(\frac{u+v}{2} \right)^p \leq \frac{u^p}{2} +
\frac{v^p}{2}$. Il en d\'ecoule que $t\mapsto \frac{|f(x) +
\tan (x)|^{p}-|f(x)|^p}{t}$ est uniform\'ement born\'ee par une
fonction int\'egrable. Le th\'eor\`eme de convergence domin\'ee
permet alors de d\'eriver sous le signe somme et
$$
\frac{dN}{dt}_{t=0} = p\int_{\Omega} |f(x)|^{p-2} f(x) g(x)\,d\mu.
$$
}
}
