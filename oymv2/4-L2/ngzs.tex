\uuid{ngzs}
\exo7id{3683}
\auteur{quercia}
\datecreate{2010-03-11}
\isIndication{false}
\isCorrection{true}
\chapitre{Espace euclidien, espace normé}
\sousChapitre{Produit scalaire, norme}

\contenu{
\texte{
Soit $E = \R_n[X]$ et $(P\mid Q) =  \int_{t=0}^1 P(t)Q(t)\,d t$.
}
\begin{enumerate}
    \item \question{Montrer que $E$, muni de $(\ \mid\ )$, est un espace euclidien.}
    \item \question{Soit $K = \R_{n-1}[X]^\bot$ et $P \in K\setminus\{0\}$.
    Quel est le degré de~$P$~?}
    \item \question{Soit $\Phi\ :\ x  \mapsto  \int_{t=0}^1 P(t)t^x\,d t$.
    Montrer que $\Phi$ est une fonction rationnelle.}
    \item \question{Trouver $\Phi$ à une constante multiplicative près.}
    \item \question{En déduire les coefficients de~$P$.}
    \item \question{En déduire une base orthogonale de~$E$.}
\reponse{
$ \int_{t=0}^1 t^kt^x\,d t = \frac 1{k+x+1}$.
$\Phi$ a pour pôles au plus simples $-1,-2,\dots,-n-1$ et
             pour racines $0,1,\dots,n-1$. Comme $\Phi(x) \to 0$ lorsque $x\to\infty$,
             on a donc $\Phi(x) = \lambda\frac{x(x-1)\dots(x-n+1)}{(x+1)\dots(x+n+1)}$.
$a_k =$ résidu de $\Phi$ en $-k-1 =
             (-1)^{n+k}\lambda\frac{(n+k)!}{(k!)^2(n-k)!}$.
}
\end{enumerate}
}
