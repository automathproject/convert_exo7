\uuid{QePt}
\exo7id{7401}
\auteur{mourougane}
\datecreate{2021-08-10}
\isIndication{false}
\isCorrection{true}
\chapitre{Groupe, anneau, corps}
\sousChapitre{Autre}

\contenu{
\texte{

}
\begin{enumerate}
    \item \question{Enoncer le théorème de la division euclidienne dans $k[X]$.}
\reponse{Pour tout $P\in k[X]$ et $D\in k[X]^*$, il existe un unique
couple $(Q,\,R)$ dans $k[X]$ tel que $P=QD+R$ et $\deg(R)<\deg(D)$.}
    \item \question{Enoncer le théorème de la division euclidienne dans $\Z[i]$.}
\reponse{Pour tout $a\in \Z[i]$ et $d\in \Z[i]^*$, il existe un
couple $(q,\,r)$ dans $k[X]$ tel que $a=qd+r$ et $|r|<|d|$.}
    \item \question{Soit $G$ un groupe et $a$ un élément de $G$ d'ordre $n$. Soit $k$ un entier naturel. Quel est l'ordre de $a^k$ ?}
\reponse{L'ordre de $a^k$ est $\frac{n}{n\wedge k}$. En effet, soient
$d=n\wedge k$, $n'=n/d$ et $k'=k/d$, de sorte que $n'\wedge k'=1$.
Pour tout $m$, on a $a^{km}=1\iff n|km \iff n'| k'm\iff n'|m$. Donc
l'ordre de $a^k$ est $n'$.}
    \item \question{Démontrer que tout groupe d'ordre $13$ est commutatif.}
\reponse{Soit $G$ d'ordre 13 et $x$ un élément non trivial de $G$. Par
le théorème de Lagrange, $x$ ne peut qu'être d'ordre 13. Mais
$\langle x\rangle$ contient déjà 13 éléments, donc $G=\langle
x\rangle$. Ainsi $G$ est abélien car monogène.}
    \item \question{Donner l'exemple d'un nombre premier qui ne peut pas s'écrire comme somme de deux carrés.}
\reponse{Par exemple 7, puisque 7 est congru à 3 modulo 4 ou bien pusique ni 7-$1^2$=6, ni $7-2^2=3$, ni avec $n\in\N, n\geq 3$, $7-n^2<0$ ne sont des carrés.}
\end{enumerate}
}
