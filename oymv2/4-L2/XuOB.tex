\uuid{XuOB}
\exo7id{3557}
\auteur{quercia}
\datecreate{2010-03-10}
\isIndication{false}
\isCorrection{true}
\chapitre{Réduction d'endomorphisme, polynôme annulateur}
\sousChapitre{Polynôme annulateur}

\contenu{
\texte{
Soit~$E$ un~$\C$-ev de dimension~$n\in\N^*$ et $u_1,\dots,u_p$ ($p\ge 2$)
des endomorphismes de~$E$ vérifiant~:
$$\forall\ k,\ u_k^2 = -\mathrm{id}_E,
  \quad\forall\ k\ne\ell,\ u_k\circ u_\ell = -u_\ell\circ u_k.$$
}
\begin{enumerate}
    \item \question{Montrer que les~$u_k$ sont des automorphismes et qu'ils sont diagonalisables.}
    \item \question{Montrer que~$n$ est pair.}
    \item \question{Donner le spectre de chaque~$u_k$.}
    \item \question{Donner les ordres de multiplicité des valeurs propres des~$u_k$.}
    \item \question{Calculer~$\det(u_k)$.}
\reponse{
3. $\mathrm{Spec}(u_k)\subset \{i,-i\}$ d'après la relation $u_k^2 = -\mathrm{id}_E$.
    Si le spectre était réduit à un élément alors $u_k$ serait scalaire car
    diagonalisable, mais ceci est incompatible avec la relation d'anticommutation
    entre~$u_k$ et $u_\ell$. Donc $\mathrm{Spec}(u_k)=\{i,-i\}$.
    
  4. $u_\ell$ avec $\ell\ne k$ échange les sous-espaces propres de~$u_k$
    donc ils ont même dimension~$n/2$.
}
\end{enumerate}
}
