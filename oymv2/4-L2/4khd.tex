\uuid{4khd}
\exo7id{3750}
\auteur{quercia}
\datecreate{2010-03-11}
\isIndication{false}
\isCorrection{true}
\chapitre{Espace euclidien, espace normé}
\sousChapitre{Projection, symétrie}

\contenu{
\texte{
Soit $E$ un espace euclidien de dimension $n$.
Soient $F,G$ deux sous-espaces vectoriels de $E$ de même dimension.
}
\begin{enumerate}
    \item \question{Montrer qu'il existe $u \in {\cal O}(E)$ tel que $u(F) = G$.}
    \item \question{$(*)$ Montrer qu'il existe $u \in {\cal O}(E)$ tel que $u(F) = G$ et $u(G) = F$.}
\reponse{
On raisonne par récurrence sur $d = \dim(F) = \dim(G)$.
Pour $d=0$ il n'y a rien à prouver.

Pour $d \ge 1$ on considère $a \in F$ et $b \in G$ unitaires tels que
$(a\mid b)$ soit maximal.
Soient $F_1$ l'orthogonal de $a$ dans $F$ et $G_1$ l'orthogonal de $b$ dans $G$
(sous-espaces vectoriels de dimensions égales à $d-1$).
Le choix de $a,b$ fait que $F_1$ est orthogonal à $b$ et $G_1$ est orthogonal
à $a$, donc $F_1$ et $G_1$ sont tous deux inclus dans l'orthogonal de
$\mathrm{vect}(a,b)$. On peut trouver un endomorphisme de cet orthogonal qui
échange $F_1$ et $G_1$, que l'on complète par la symétrie orthogonale dans
$\mathrm{vect}(a,b)$ qui échange $a$ et $b$.
}
\end{enumerate}
}
