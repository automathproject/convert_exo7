\uuid{jvFN}
\exo7id{3513}
\auteur{quercia}
\datecreate{2010-03-10}
\isIndication{false}
\isCorrection{true}
\chapitre{Réduction d'endomorphisme, polynôme annulateur}
\sousChapitre{Valeur propre, vecteur propre}

\contenu{
\texte{
Soient $\alpha, \beta, \gamma \in  K$ distincts, et
$\varphi : { K_2[X]} \to { K_2[X]}, P\mapsto R$
où $R$ est le reste de la division euclidienne de $X^3P$ par
$(X-\alpha)(X-\beta)(X-\gamma)$.
Chercher les valeurs et les vecteurs propres de $\varphi$.
}
\reponse{
$\alpha^3 : (X-\beta)(X-\gamma)$,  \quad
         $\beta^3  : (X-\alpha)(X-\gamma)$, \quad
         $\gamma^3 : (X-\alpha)(X-\beta)$.
}
}
