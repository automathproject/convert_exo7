\uuid{VQ2w}
\exo7id{5668}
\auteur{rouget}
\datecreate{2010-10-16}
\isIndication{false}
\isCorrection{true}
\chapitre{Réduction d'endomorphisme, polynôme annulateur}
\sousChapitre{Applications}

\contenu{
\texte{
\label{ex:rou18}
}
\begin{enumerate}
    \item \question{Soit $J_n=\left(
\begin{array}{ccccc}
0&1&0&\ldots&0\\
\vdots&\ddots&\ddots&\ddots&\vdots\\
\vdots& & &\ddots&0\\
0& & &\ddots&1\\
1&0&\ldots&\ldots&0
\end{array}
\right)$   (de format $n\geqslant 3$). Diagonaliser $J_n$.}
\reponse{$J^n = I$. $J$ annule le polynôme $X^n-1$ qui est à racines simples dans $\Cc$ et donc $J$ est diagonalisable dans $\Cc$.

Les valeurs propres de $J$ sont à choisir parmi les racines $n$-èmes de $1$ dans $\Cc$. On pose $\omega=e^{2i\pi/n}$. Vérifions que $\forall k\in\llbracket0,n-1\rrbracket$, $\omega^k$ est valeur propre de $J$.

Soient $k\in\llbracket0,n-1\rrbracket$ et $X =(x_j)_{1\leqslant j\leqslant n}$ un élément de $\mathcal{M}_{n,1}(\Cc)$.

\begin{center}
$JX =\omega^kX\Leftrightarrow\left\{
\begin{array}{l}
x_2=\omega^kx_1\\
x_3=\omega^kx_2\\
\vdots\\
x_n=\omega^kx_{n-1}\\
x_1=\omega^kx_n
\end{array}
\right.\Leftrightarrow\left\{
\begin{array}{l}
x_2=\omega^kx_1\\
x_3=(\omega^k)^2x_1\\
\vdots\\
x_n=(\omega^k)^{n-1}x_{1}\\
x_1=(\omega^k)^nx_1
\end{array}
\right.\Leftrightarrow\left\{
\begin{array}{l}
x_2=\omega^kx_1\\
x_3=(\omega^k)^2x_1\\
\vdots\\
x_n=(\omega^k)^{n-1}x_{1}
\end{array}
\right.
$
\end{center}

et donc

\begin{center}
$JX =\omega^kX\Leftrightarrow X\in\text{Vect}(U_k)\;\text{où}\;U_k=\left(
\begin{array}{c}
1\\
\omega^k\\
(\omega^k)^2\\
\vdots\\
(\omega^k)^{n-1}
\end{array}
\right)$.
\end{center}

Donc $\forall k\in\llbracket0,n-1\rrbracket$, $\omega^k$ est valeur propre de $J$. Les valeurs propres de $J$ sont les $n$ racines $n$-èmes de $1$. Ces valeurs propres sont toutes simples. Le sous espace propre associé à $\omega^k$, $0\leqslant k\leqslant n-1$, est la droite vectorielle $D_k=\text{Vect}(U_k)$.

Soit $P$ la matrice de \textsc{Vandermonde} des racines $n$-èmes de l'unité c'est-à-dire $P=(\omega^{(j-1)(k-1)})_{0\leqslant j,k\leqslant n-1}$ puis $D =\text{diag}(1,\omega,...,\omega^{n-1})$, alors on a déjà vu que $P^{-1}=\frac{1}{n}\overline{P}$ (exercice \ref{ex:rou16}) et on a 

\begin{center}
\shadowbox{
$J =PDP^{-1}$ avec $D =\text{diag}(\omega^{j})_{1\leqslant j\leqslant n}$, $P=\left(\omega^{(j-1)(k-1)}\right)_{1\leqslant j,k\leqslant n}$ et $P^{-1}=\frac{1}{n}\overline{P}$ avec $\omega=e^{2i\pi/n}$.
}
\end{center}

\textbf{Remarque.} La seule connaissance de $D$ suffit pour le 2).}
    \item \question{En déduire la valeur de $\left|
\begin{array}{ccccc}
a_0&a_1&\ldots&a_{n-2}&a_{n-1}\\
a_{n-1}&a_0&a_1& &a_{n-2}\\
\vdots& &\ddots&\ddots&\vdots\\
a_2& &\ddots&a_0&a_1\\
a_1&a_2&\ldots&a_{n-1}&a_0
\end{array}
\right|$.}
\reponse{Soit $A$ la matrice de l'énoncé.

\begin{center}
$A= a_0I + a_1J + a_2J^2+...+a_{n-1}J^{n-1}= Q(J)$ où $Q = a_0+a_1X+...+a_{n-1}X^{n-1}$.
\end{center}

D'après 1), $A = P\times Q(D)\times P^{-1}$ et donc $A$ est semblable à la matrice $\text{diag}(Q(1),Q(\omega),...,Q(\omega^{n-1}))$. Par suite, $A$ a même déterminant que la matrice $\text{diag}(Q(1),Q(\omega),...,Q(\omega^{n-1}))$. D'où la valeur du déterminant circulant de l'énoncé :

\begin{center}
\shadowbox{
$\left|
\begin{array}{ccccc}
a_0&a_1&\ldots&a_{n-2}&a_{n-1}\\
a_{n-1}&a_0&a_1& &a_{n-2}\\
\vdots& &\ddots&\ddots&\vdots\\
a_2& &\ddots&a_0&a_1\\
a_1&a_2&\ldots&a_{n-1}&a_0
\end{array}
\right|=\prod_{k=0}^{n-1}\left(\sum_{j=0}^{n-1}e^{2i(j-1)(k-1)\pi/n}a_j\right)$.
}
\end{center}}
\end{enumerate}
}
