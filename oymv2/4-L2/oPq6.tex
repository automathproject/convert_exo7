\uuid{oPq6}
\exo7id{2613}
\auteur{delaunay}
\datecreate{2009-05-19}
\isIndication{false}
\isCorrection{true}
\chapitre{Réduction d'endomorphisme, polynôme annulateur}
\sousChapitre{Diagonalisation}

\contenu{
\texte{
Soit $E$ un espace vectoriel de dimension $n$. On cherche à déterminer une matrice $A\in{\cal M}_n(\R)$ telle que $A^2=-I_n$, où $I_n$ désigne la matrice identité d'ordre $n$. On notera $f$ l'endomorphisme de $E$ de matrice $A$ dans la base canonique.
}
\begin{enumerate}
    \item \question{Démontrer que l'existence d'une telle matrice implique la parité de $n$.}
\reponse{{\it Démontrons que l'existence d'une telle matrice implique la parité de $n$.}

Supposons qu'il existe $A\in{\cal M}_n(\R)$ telle que $A^2=-I_n$, on a alors
$$\det(A^2)=(\det A)^2=(-1)^n,$$
ce qui implique $n$ pair, car un carré est toujours positif.}
    \item \question{On suppose maintenant que $n=4$.
   \begin{enumerate}}
\reponse{On suppose maintenant que $n=4$.
   \begin{enumerate}}
    \item \question{Démontrer que pour tout $\vec x\in E$, $\vec x\neq 0$, les vecteurs $\vec x$ et $f(\vec x)$ sont linéairement indépendants.}
\reponse{{\it Démontrons que pour tout $\vec x\in E$, $\vec x\neq 0$, les vecteurs $\vec x$ et $f(\vec x)$ sont linéairement indépendants.}

Soit $\vec x\in E$, on suppose $\vec x\neq 0$, supposons qu'il existe des réels $a,b$ tels que $a\vec x+bf(\vec x)=\vec0$, on a alors
$$a\vec x+bf(\vec x)=\vec0\Longrightarrow f(a\vec x+bf(\vec x))=\vec0\Longrightarrow af(\vec x)-b\vec x=\vec 0,$$
car $f^2=-\mathrm{id}_E$. Or,
$$\left\{\begin{align*}a\vec x+bf(\vec x)=\vec0 \\  af(\vec x)-b\vec x=\vec 0\end{align*}\right.\Longrightarrow(a^2+b^2)\vec x=\vec0,$$
ce qui implique $a^2+b^2=0$ car $\vec x\neq \vec 0$, et, donc $a=b=0$. Ce qui prouve que les vecteurs $\vec x$ et $f(\vec x)$ sont linéairement indépendants.}
    \item \question{Soit $\vec x_1\neq0$, on note $F$ le sous-espace vectoriel de $E$ engendré par les vecteurs $\vec x_1$ et $f(\vec x_1)$.
             \begin{enumerate}}
\reponse{Soit $\vec x_1\neq0$, on note $F$ le sous-espace vectoriel de $E$ engendré par les vecteurs $\vec x_1$ et $f(\vec x_1)$.
            \begin{enumerate}}
    \item \question{Démontrer que $F$ est stable par $f$.}
\reponse{{\it Démontrons que $F$ est stable par $f$.}

Soit $\vec x\in F$, il existe $(a,b)\in\R^2$ tel que $\vec x=a\vec x_1+bf(\vec x_1)$, d'où
$$f(\vec x)=f(a\vec x_1+bf(\vec x_1))=af(\vec x_1)+bf^2(\vec x)=af(\vec x_1)-b\vec x_1\in F.$$
D'où la stabilité de $F$ par $f$.}
    \item \question{Soit $\vec x_2\in E$, on suppose que $\vec x_2\not\notin F$, démontrer que ${\cal B}=(\vec x_1,f(\vec x_1),\vec x_2,f(\vec x_2))$ est une base de $E$.}
\reponse{Soit $\vec x_2\in E$, on suppose que $\vec x_2\not\notin F$. 
	       
	       {\it Démontrons que ${\cal B}=(\vec x_1,f(\vec x_1),\vec x_2,f(\vec x_2))$ est une base de $E$.}
La dimension de $E$ étant égale à $4$, il suffit de démontrer que les vecteurs sont linéairement indépendants. Supposons qu'il existe $(a_1,b_1,a_2,b_2)\in\R^4$ tel que $$a_1\vec x_1+b_1f(\vec x_1)+a_2\vec x_2+b_2f(\vec x_2)=\vec0,$$
on a alors,
$$a_2\vec x_2+b_2f(\vec x_2)\in F,$$
et, comme $F$ est stable par $f$, 
$$f(a_2\vec x_2+b_2f(\vec x_2))=a_2f(\vec x_2)-b_2\vec x_2\in F.$$
Ce qui implique 
$$(a_2^2+b_2^2)\vec x_2\in F\quad{\hbox{d'où}}\quad a_2^2+b_2^2=0 $$
car on a supposé $\vec x_2\not\notin F$. On a donc $a_2=b_2=0$ et, par conséquent, $a_1\vec x_1+b_1f(\vec x_1)=0$, or les vecteurs $\vec x_1$ et $f(\vec x_1)$ sont linéairement indépendants, ce qui implique $a_1=b_1=0$. D'où l'indépendance des vecteurs $\vec x_1,f(\vec x_1),\vec x_2,f(\vec x_2)$.}
\end{enumerate}
}
