\uuid{clyi}
\exo7id{1147}
\auteur{barraud}
\datecreate{2003-09-01}
\isIndication{false}
\isCorrection{false}
\chapitre{Déterminant, système linéaire}
\sousChapitre{Calcul de déterminants}

\contenu{
\texte{
Soient $a,b,c$ trois réels et $\Delta_{n}$ le déterminant de taille
  $n$ suivant~:
$$
\Delta_{n}=\left\vert
  \begin{array}{c@{\ }c@{}c@{\ }c}
   a     & b      &        & 0
\\[-1ex]
   c     & \ddots & \ddots &                  \\[-2ex]
         & \ddots & \ddots & b \\[-1ex]
   0 & & c      & a \\
\end{array}
\right\vert
$$
}
\begin{enumerate}
    \item \question{On pose $\Delta_{0}=1$, $\Delta_{1}=a$. Montrer que $\forall n\in\N,
\Delta_{n+2}=a\Delta_{n+1}-bc\Delta_{n}$.}
    \item \question{On suppose que $a^{2}=4bc$. Montrer par récurrence que~: 
$$\forall
n\in\N, \Delta_{n}=(n+1)\,\frac{a^{n}}{2^{n}}$$}
\end{enumerate}
}
