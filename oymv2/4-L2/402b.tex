\uuid{402b}
\exo7id{1607}
\auteur{barraud}
\datecreate{2003-09-01}
\isIndication{false}
\isCorrection{false}
\chapitre{Réduction d'endomorphisme, polynôme annulateur}
\sousChapitre{Valeur propre, vecteur propre}

\contenu{
\texte{
Soient $A$ et $B$ deux matrices de $\mathcal{M}_{n}(\R)$ telles que
  $$
  AB-BA=A
  $$
  Le but de cet exercice est de montrer que $A$ est nilpotente, c'est à
  dire $$\exists k\in\N, A^{k}=0.$$


  On note $E$ l'espace vectoriel $\mathcal{M}_{n}(\R)$ et on considère
  l'application~:\quad
  $$\displaystyle \psi
   \begin{array}{ccc}
     E & \rightarrow  & E \\
     M &\mapsto  & MB-BM
   \end{array}
  $$
}
\begin{enumerate}
    \item \question{Montrer que $\psi$ est linéaire de $E$ dans $E$.}
    \item \question{Montrer par récurrence que~: $\forall k\in\N\quad
    \psi(A^{k})=kA^{k}$.}
    \item \question{On suppose que $\forall k\in\N, A^{k}\neq0$.
    Montrer que $\psi$ a une infinité de valeurs propres.}
    \item \question{Conclure.}
\end{enumerate}
}
