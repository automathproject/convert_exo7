\uuid{iRJe}
\exo7id{1311}
\auteur{ortiz}
\datecreate{1999-04-01}
\isIndication{false}
\isCorrection{true}
\chapitre{Groupe, anneau, corps}
\sousChapitre{Groupe, sous-groupe}

\contenu{
\texte{
Les ensembles suivants, pour les lois
consid\'er\'ees, sont-ils des groupes ?
}
\begin{enumerate}
    \item \question{$]-1,1[$ muni de la loi d\'efinie par $x\star y=\frac{x+y}{1+xy}$ ;}
\reponse{Oui.}
    \item \question{$\{z\in\Cc:{|z|}=2\}$ pour la multiplication usuelle ;}
\reponse{Non. Le seul élément qui peut être l'élément neutre
  est $1$ qui n'appartient pas à l'ensemble.}
    \item \question{$\Rr_+$ pour la multiplication usuelle;}
\reponse{Non. $0$ n'a pas d'inverse.}
    \item \question{$\left\{x\in\Rr\mapsto ax+b : a\in\Rr\setminus\left\{0\right\},b\in\Rr\right\}$
pour la loi de composition des applications.}
\reponse{Oui.}
\end{enumerate}
}
