\uuid{ZBJH}
\exo7id{7304}
\auteur{mourougane}
\datecreate{2021-08-10}
\isIndication{false}
\isCorrection{false}
\chapitre{Groupe, anneau, corps}
\sousChapitre{Anneau}

\contenu{
\texte{

}
\begin{enumerate}
    \item \question{Déterminer l'ordre de $\overline{2}$ dans $(\Z/10\Z,+)$.}
    \item \question{Montrer que $\{\overline{0},\overline{3},\overline{6},\overline{9}\}$ est un sous-groupe de $(\Z/12\Z,+)$.}
    \item \question{Expliciter un sous-groupe d'ordre $6$ de $(\Z/12\Z,+)$.}
    \item \question{Quels sont les ordres possibles des sous-groupes de $(\Z/6\Z,+)$ ?}
    \item \question{Soient $n$ un entier naturel non nul. Le but de cette question est de déterminer tous 
les sous-groupes de $\Z/n\Z$. Soit $d$ un diviseur de $n$.
\begin{enumerate}}
    \item \question{Expliciter un sous-groupe $G_{d}$ de $\Z/n\Z$ d'ordre $d$.}
    \item \question{Soit $H$ un sous-groupe de $\Z/n\Z$ d'ordre $d$. Montrer que pour tout 
$\overline x\in H$,
$d\cdot \overline x=0.$ Combien d'éléments de $\Z/n\Z$ vérifient cette équation ? 
En déduire que $H=G_{d}$.}
    \item \question{Conclure.}
\end{enumerate}
}
