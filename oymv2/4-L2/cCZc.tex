\uuid{cCZc}
\exo7id{2566}
\auteur{delaunay}
\datecreate{2009-05-19}
\isIndication{false}
\isCorrection{true}
\chapitre{Réduction d'endomorphisme, polynôme annulateur}
\sousChapitre{Diagonalisation}

\contenu{
\texte{
Soit $$A=\begin{pmatrix}1&0&0 \\ 0&1&0 \\ 1&-1&2\end{pmatrix}$$
D\'emontrer que $A$ est diagonalisable et trouver une matrice $P$ telle que $P^{-1}AP$ soit diagonale.
}
\reponse{
Soit $$A=\begin{pmatrix}1&0&0 \\ 0&1&0 \\ 1&-1&2\end{pmatrix}$$
{\it D\'emontrons que $A$ est diagonalisable et trouvons une matrice $P$ telle que $P^{-1}AP$ soit diagonale.}

Commen\c cons par calculer le polyn\^ome caract\'eristique de $A$ :
$$P_A(X)=\begin{vmatrix}1-X&0&0 \\  0&1-X&0 \\ 1&-1&2-X\end{vmatrix}=(1-X)^2(2-X)$$
Les racines du polyn\^ome caract\'eristique sont les r\'eels $1$ avec la multiplicit\'e $2$, et $2$ avec la multiplicit\'e $1$.

D\'eterminons les sous-espaces propres associ\'es : Soit $E_1$ le sous-espace propre associ\'e \`a la valeur propre double $1$.

$E_1=\{V(x,y,z)\in\R^3/\ A.V=V\}$, 
$$V\in E_1\iff \left\{ \begin{align*}x&=x \\  y&=y \\  x-y+z&=0\end{align*}\right.\iff x-y+z=0$$
$E_1$ est donc un plan vectoriel, dont les vecteurs $e_1=(1,1,0)$ et $e_2=(0,1,1)$ forment une base.

Soit $E_2$ le sous-espace propre associ\'e \`a la valeur propre simple $2$.

$E_2=\{V(x,y,z)\in\R^3/\ A.V=2V\}$, 
$$V\in E_2\iff \left\{ \begin{align*}x&=2x \\  y&=2y \\  x-y+2z&=2z\end{align*}\right.\iff x=0, y=0$$
$E_2$ est donc une droite vectorielle, dont le vecteur $e_3=(0,0,1)$ est une base.

Les dimensions des sous-espaces propres sont \'egales \`a la multiplicit\'e des valeurs propres correspondantes, la matrice $A$ est donc diagonalisable.
Dans la base $(e_1, e_2, e_3)$ l'endomorphisme repr\'esent\'e par $A$ (dans la base canonique) a pour matrice.
$$D=\begin{pmatrix}1&0&0 \\  0&1&0 \\  0&0&2\end{pmatrix}$$
la matrice de passage $$P=\begin{pmatrix}1&0&0 \\  1&1&0 \\  0&1&1\end{pmatrix}$$ v\'erifie $P^{-1}AP=D$.
}
}
