\uuid{ZnjC}
\exo7id{3691}
\auteur{quercia}
\datecreate{2010-03-11}
\isIndication{false}
\isCorrection{true}
\chapitre{Espace euclidien, espace normé}
\sousChapitre{Produit scalaire, norme}

\contenu{
\texte{
Soit~$H$ un espace euclidien, $(y_j)_{j\in I}$ une famille de vecteurs
de~$H$ telle qu'il existe $A$ et~$B$ strictement positifs vérifiant~:
$$\forall\ x\in H,\ A\|x\|^2 \le \sum_{j\in I}(x\mid y_j)^2 \le B\|x\|^2.$$
}
\begin{enumerate}
    \item \question{Montrer que $(y_j)_{j\in I}$ engendre~$H$.}
\reponse{Le sous-espace vectoriel engendré a un orthogonal nul.}
    \item \question{On choisit $H=\R^2$. Montrer que $y_1=\left(\begin{smallmatrix}0\cr1\end{smallmatrix}\right)$,
    $y_2=\left(\begin{smallmatrix}-\sqrt3/2\cr-1/2\cr\end{smallmatrix}\right)$, $y_3=y_2$ conviennent.}
\reponse{N'importe quelle famille génératrice convient (équivalence des normes).}
    \item \question{Si $A=B=1$ et $\|y_j\|=1$ pour tout~$j$, montrer que $(y_j)_{j\in I}$ est une
    base orthonormale.}
\reponse{$1=\|y_i\|^2 = \|y_i\|^4 + \sum_{j\ne i}(y_i\mid y_j)^2 \Rightarrow \forall\ j\ne i,\ (y_i\mid y_j) = 0$.}
    \item \question{Si $A=B$, montrer que pour tout~$x\in H$, $x=\frac1A\sum_{j\in I}(x\mid y_j)y_j$.}
\reponse{Par polarisation on a~:
    $\forall\ x,y,\ \sum_{j\in I}(x\mid y_j)(y\mid y_j) = A(x\mid y)$
    donc $\sum_{j\in I}(x\mid y_j)y_j-Ax\in E^\bot$.}
\end{enumerate}
}
