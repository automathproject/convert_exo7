\uuid{nWhL}
\exo7id{1524}
\auteur{barraud}
\datecreate{2003-09-01}
\isIndication{false}
\isCorrection{false}
\chapitre{Endomorphisme particulier}
\sousChapitre{Autre}

\contenu{
\texte{
On considère un espace euclidien $(E,<>)$. On rappelle que l'adjoint
$u^{*}$ d'un endomorphisme $u$ est l'endomorphisme caractérisé par~:
$$
\forall (x,y)\in E^{2},\ <u(x),y>=<x,u^{*}(y)>
$$

On dit qu'un endomorphisme $u$ de $E$ est une similitude de $E$ si et
seulement si $u$ est la composée d'une homotétie et d'une isométrie,
c'est à dire si et seulement si~:
$$
 \exists\alpha\in\R\setminus\{0\},\ \exists v\in O(E)\ /\ u=\alpha v.
$$
}
\begin{enumerate}
    \item \question{Redémontrer l'équivalence entre les trois caractérisations suivantes des
isométries~:
\begin{align*}
  v\text{ est une isométrie} 
  &\Leftrightarrow \forall x\in E\quad \Vert v(x)\Vert=\Vert x\Vert\\
  &\Leftrightarrow \forall (x,y)\in E^{2}\quad <v(x),v(y)>=<x,y>\\
  &\Leftrightarrow v^{*}v=\mathrm{id}
\end{align*}}
\end{enumerate}
}
