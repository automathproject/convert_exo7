\uuid{pzZH}
\exo7id{5673}
\auteur{rouget}
\datecreate{2010-10-16}
\isIndication{false}
\isCorrection{true}
\chapitre{Réduction d'endomorphisme, polynôme annulateur}
\sousChapitre{Diagonalisation}

\contenu{
\texte{
Soit $A$ une matrice carrée de format $2$ telle que $A^2$ est diagonalisable et $\text{Tr}A\neq 0$. Montrer que $A$ est diagonalisable dans $\Cc$.
}
\reponse{
$A$ est de format $2$ et donc, soit a deux valeurs propres distinctes et est dans ce cas diagonalisable dans $\Cc$, soit a une valeur propre double $\lambda$ non nulle car $\text{Tr}A = 2\lambda \neq 0$.

Dans ce dernier cas, $A^2$ est diagonalisable et est donc est semblable à $\text{diag}(\lambda^2,\lambda^2) =\lambda^2I$. Par suite, $A^2 =\lambda^2I$. Ainsi, $A$ annule le polynôme $X^2-\lambda^2=(X-\lambda)(X+\lambda)$ qui est scindé sur $\Rr$ à racines simples. Dans ce cas aussi, $A$ est diagonalisable.
}
}
