\uuid{WSaZ}
\exo7id{3451}
\auteur{quercia}
\datecreate{2010-03-10}
\isIndication{false}
\isCorrection{true}
\chapitre{Déterminant, système linéaire}
\sousChapitre{Calcul de déterminants}

\contenu{
\texte{
Calculer les déterminants suivants :
}
\begin{enumerate}
    \item \question{$\begin{vmatrix}  x &a &b &x \cr                                      %+---------+
                 b &x &x &a \cr                                      %|  x,a,b  |
                 x &b &a &x \end{vmatrix}$.                                      %+---------+}
\reponse{$(b-a)^2(a+b+2x)(a+b-2x)$.}
    \item \question{$\begin{vmatrix} a-b-c &2a    &2a    \cr                      %+-----------------+
                2b    &b-c-a &2b    \cr                      %|  a-b-c, 2a, 2a  |
                2c    &2c    &c-a-b \cr \end{vmatrix}$.                  %+-----------------+}
\reponse{$(a+b+c)^3$}
    \item \question{$\begin{vmatrix} a+b     &b+c     &c+a     \cr                  %+---------------+
                a^2+b^2 &b^2+c^2 &c^2+a^2 \cr                  %|  a+b b+c c+a  |
                a^3+b^3 &b^3+c^3 &c^3+a^3 \cr \end{vmatrix}$.              %+---------------+}
\reponse{$2abc(a-b)(b-c)(c-a)$}
    \item \question{$\begin{vmatrix} a+b &ab &a^2+b^2 \cr                         %+-----------------+
                b+c &bc &b^2+c^2 \cr                         %|  a+b ab a¿+b¿   |
                c+a &ca &c^2+a^2 \cr \end{vmatrix}
                $.                     %+-----------------+}
\reponse{$(a-b)(b-c)(c-a)(ab+ac+bc)$.}
    \item \question{$\begin{vmatrix} a^2 &b^2 &ab  \cr                                %+-------------+
                b^2 &ab  &a^2 \cr                                %|  a¿ b¿ ab   |
                ab  &a^2 &b^2 \cr \end{vmatrix}$.                            %+-------------+}
\reponse{$-(a^3-b^3)^2$.}
    \item \question{$\begin{vmatrix}   a & b      &        &(0)  \cr     %+-------------------------+
                   c & \ddots & \ddots       \cr     %|  Matrice tridiagonale   |
                     & \ddots & \ddots & b   \cr     %+-------------------------+
                  (0)&        & c      & a   \cr \end{vmatrix}$.
     \label{tridiag}}
\reponse{$\frac {\alpha^{n+1} - \beta^{n+1}}{\alpha - \beta}$
              où $\alpha \ne \beta$ sont les racines de $X^2 - aX + bc = 0$.
              \par
              $(n+1)\left(\frac a2\right)^n$ si $\alpha = \beta$.}
    \item \question{$\begin{vmatrix}a      &b       &\dots   &b       \cr               %+-----------+
               b      &\ddots  &(0)     &\vdots  \cr               %|  anneau   |
               \vdots &(0)     &\ddots  &b       \cr               %+-----------+
               b      &\dots   &b       &a       \cr\end{vmatrix}$.
     
                                                   %+---------------------------+
                                                   %|  Déterminants de Pascal   |
                                                   %+---------------------------+}
\reponse{$a^{n-3}(a-b)\bigl(a^2 + ab -2(n-2)b^2\bigr)$.}
    \item \question{$\begin{vmatrix} C_n^0\hfill  & C_n^1\hfill  & \dots & C_n^p\hfill  \cr
                C_{n+1}^0    & C_{n+1}^1    & \dots & C_{n+1}^p    \cr
                \vdots       & \vdots       &       & \vdots       \cr
                C_{n+p}^0    & C_{n+p}^1    & \dots & C_{n+p}^p    \cr \end{vmatrix}$.
 
 
                                                                %+--------------+
                                                                %|  ai+bi, bi   |
                                                                %+--------------+}
\reponse{1.}
    \item \question{$\begin{vmatrix}
               a_1 + b_1 & b_1       & \dots  & \dots  & b_1       \cr
               b_2       & a_2 + b_2 & b_2    & \dots  & b_2       \cr
               \vdots    &           & \ddots &        & \vdots    \cr
               \vdots    &           &        & \ddots & b_{n-1}   \cr
               b_n       & \dots     & \dots  & b_n    & a_n + b_n \cr \end{vmatrix}$.}
\reponse{$a_1 a_2 \dots a_n\left(1+\frac{b_1}{a_1}+ \dots
              + \frac{b_n}{a_n}\right)$.}
    \item \question{$\begin{vmatrix} a_1-b_1 &\dots &a_1-b_n \cr                        %+-----------+
                \vdots  &\dots &\vdots  \cr                        %|   ai-bj   |
                a_n-b_1 &\dots &a_n-b_n \cr \end{vmatrix}$, $(n \ge 3)$.       %+-----------+}
\reponse{0}
    \item \question{$\begin{vmatrix} 1     & 2     & \dots & n     \cr   %+--------------------------+
                2     & 3     & \dots & 1     \cr   %|  Déterminant circulant   |
                \vdots& \vdots&       & \vdots\cr   %+--------------------------+
                n     & 1     & \dots & n-1   \cr\end{vmatrix}$.}
\reponse{$\varepsilon_n \frac {n^{n-1}(n+1)}2$.}
    \item \question{$\begin{vmatrix} 0      &1     &2      &\dots  &n-1     \cr         %+-----------+
                1      &0     &1      &       &\vdots  \cr         %|   |i-j|   |
                2      &1     &0      &\ddots &2       \cr         %+-----------+
                \vdots &      &\ddots &\ddots &1       \cr
                n-1    &\dots &2      &1      &0       \cr\end{vmatrix}$.
   
 
 
 Pour \ref{tridiag} : Chercher une relation de récurrence linéaire d'ordre 2.
 On notera $\alpha$ et $\beta$ les racines dans $\C$ de l'équation
 caractéristique, et on exprimera le déterminant en fonction de $\alpha$ et
 $\beta$.}
\reponse{$(-1)^{n-1}(n-1)2^{n-2}$.}
\end{enumerate}
}
