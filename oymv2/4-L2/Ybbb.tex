\uuid{Ybbb}
\exo7id{3158}
\auteur{quercia}
\datecreate{2010-03-08}
\isIndication{false}
\isCorrection{false}
\chapitre{Arithmétique}
\sousChapitre{Anneau Z/nZ, théorème chinois}

\contenu{
\texte{
Soit $p$ un nombre premier impair.
Montrer que $\dot k$ est un carr{\'e} dans l'anneau $\Z/p\Z$ si et seulement si
$k^{(p+1)/2} \equiv k (\mathrm{mod}\, p)$.
}
}
