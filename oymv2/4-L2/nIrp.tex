\uuid{nIrp}
\exo7id{7291}
\auteur{mourougane}
\datecreate{2021-08-10}
\isIndication{false}
\isCorrection{false}
\chapitre{Groupe, anneau, corps}
\sousChapitre{Groupe, sous-groupe}

\contenu{
\texte{

}
\begin{enumerate}
    \item \question{Dire si les couples suivants sont des groupes : $(\Z,+)$ ; $(\Z,\times)$ ; $(\C^*,+)$ ; $(\C^*, \times)$.}
    \item \question{Déterminer une loi de composition interne pour laquelle $\{-1,1\}$ est un groupe.
Même question avec $\R\times\R$.}
    \item \question{Soit $(G,\star)$ un groupe et $E$ un ensemble.
Montrer que l'ensemble $\mathcal{A}(E,G)$ des applications
de $E$ dans $G$ peut être muni d'une structure naturelle de groupe.}
    \item \question{L'ensemble $\{a,b,c\}$ muni de la loi de composition interne définie par la table
$$\begin{array}{|c|c|c|c|}\hline
 \star& a&b&c\\ \hline
 a&a&b&c\\ \hline
b&b&a&b\\ \hline
c& c& b&a\\\hline
\end{array}
$$
admet-il un élément neutre ? est-il commutatif ?
tout élément admet-il un symétrique ?
est-il un groupe ?}
    \item \question{La réunion de deux sous-groupes d'un groupe $G$ est-elle un sous-groupe ? Et l'intersection ?}
    \item \question{Donner l'exemple d'un ensemble avec une loi de composition interne sans élément neutre.}
\end{enumerate}
}
