\uuid{s1b1}
\exo7id{2601}
\auteur{delaunay}
\datecreate{2009-05-19}
\isIndication{false}
\isCorrection{true}
\chapitre{Réduction d'endomorphisme, polynôme annulateur}
\sousChapitre{Diagonalisation}

\contenu{
\texte{
Soit $A$ la matrice suivante :
$$A=\begin{pmatrix}0&1 \\ 1&0\end{pmatrix}$$
}
\begin{enumerate}
    \item \question{Diagonaliser la matrice $A$.}
\reponse{{\it Diagonalisons la matrice $A$.} (2 points)

Son polyn\^ome caract\'eristique est \'egal \`a
$$P_A(X)=\begin{vmatrix}-X&1 \\  1&-X\end{vmatrix}=X^2-1=(X-1)(X+1).$$
La matrice $A$ admet deux valeurs propres distinctes, elle est donc diagonalisable.

{\it D\'eterminons une base de vecteurs propres de $A$}. 

Soit $\vec u=(x,y)\in\R^2$,
$$A\vec u=\vec u\iff x=y\ {\hbox{et}}\ A\vec u=-\vec u\iff x=-y.$$
Notons $\vec u_1=(1,1)$ et $\vec u_2=(-1,1)$, le vecteur $\vec u_1$ est un vecteur propre associ\'e \`a la valeur propre $1$ et le vecteur $\vec u_2$ est un vecteur propre associ\'e \`a la valeur propre $-1$, ils sont lin\'eairement ind\'ependants, ils forment donc une base de $\R^2$. Ainsi, on a $A=PDP^{-1}$, o\`u
$$P=\begin{pmatrix}1&-1 \\  1&1\end{pmatrix}\ {\hbox{et}}\ D=\begin{pmatrix}1&0 \\  0&-1\end{pmatrix}$$}
    \item \question{Exprimer les solutions du syst\`eme diff\'erentiel $X'=AX$ dans une base de vecteurs propres et tracer ses trajectoires.}
\reponse{{\it Exprimons les solutions du syst\`eme diff\'erentiel $X'=AX$ dans une base de vecteurs propres et tra\c cons ses trajectoires.} (3 points)

Soit $Y$ tel que $PY=X$, on a alors 
$$X'=AX\iff PY'=APY\iff Y'=P^{-1}APY\iff Y'=DY.$$
Les solutions du syst\`eme diff\'erentiel $X'=AX$ dans la base de vecteurs propres $(\vec u_1,\vec u_2)$ sont les solutions du syst\`eme $Y'=DY$. Si $Y=(x,y)$, on a $x'(t)=x(t)$ et $y'(t)=-y(t)$, ainsi, les solutions du syst\`eme sont $x(t)=ae^t$ et $y(t)=be^{-t}$ o\`u $a$ et $b$ sont des constantes r\'eelles arbitraires. Les trajectoires, exprim\'ees dans la base de vecteurs propres $(\vec u_1,\vec u_2)$,  sont donc les courbes d'\'equation $y=c/x$ avec $c\in\R$, ce sont des branches d'hyperboles.}
\end{enumerate}
}
