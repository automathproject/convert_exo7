\uuid{2khS}
\exo7id{3735}
\auteur{quercia}
\datecreate{2010-03-11}
\isIndication{false}
\isCorrection{true}
\chapitre{Espace euclidien, espace normé}
\sousChapitre{Projection, symétrie}

\contenu{
\texte{
Soit $\vec v \in E\setminus\{\vec0\}$ et $\lambda \in \R$.
On pose pour $\vec x \in E$ :
$f(\vec x) = \vec x + \lambda (\vec x\mid \vec v)\vec v$.

Déterminer $\lambda$ pour que $f \in {\cal O}(E)$. Reconnaître alors $f$.
}
\reponse{
$\lambda = 0$, $f = \mathrm{id}_E$ et $\lambda = -\frac2{\|\vec v\,\|^2}$,
         $f =$ la symétrie par rapport à vect$(\vec v)$.
}
}
