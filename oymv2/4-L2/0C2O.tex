\uuid{0C2O}
\exo7id{5298}
\auteur{rouget}
\datecreate{2010-07-04}
\isIndication{false}
\isCorrection{true}
\chapitre{Arithmétique}
\sousChapitre{Arithmétique de Z}

\contenu{
\texte{

}
\begin{enumerate}
    \item \question{Montrer que $\forall(k,n)\in(\Nn^*)^2,\;[k\wedge n=1\Rightarrow n|C_n^k]$.}
\reponse{Pour $1\leq k\leq n$, $kC_n^k=nC_{n-1}^{k-1}$. Donc, si $k$ et $n$ sont premiers entre eux, puisque $n$ divise $kC_n^k$, le théorème de \textsc{Gauss} permet d'affirmer que $n$ divise $C_n^k$.}
    \item \question{Montrer que $\forall n\in\Nn^*,\;(n+1)|C_{2n}^n$.}
\reponse{De même, $(n+1)C_{2n}^{n-1}=nC_{2n}^{n}$ montre que $(n+1)$ divise $nC_{2n}^{n}$ et, puisque $n$ et $(n+1)$ sont premiers entre eux (d'après \textsc{Bezout} puisque $(n+1)-n=1$), $(n+1)$ divise $C_{2n}^{n}$ d'après le théorème de \textsc{Gauss}.}
\end{enumerate}
}
