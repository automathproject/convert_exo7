\uuid{ZvXQ}
\exo7id{3782}
\auteur{quercia}
\datecreate{2010-03-11}
\isIndication{false}
\isCorrection{true}
\chapitre{Espace euclidien, espace normé}
\sousChapitre{Endomorphismes auto-adjoints}

\contenu{
\texte{
Soit $u \in \mathcal{L}(E)$ auto-adjoint tel que $\mathrm{tr}(u) = 0$.
}
\begin{enumerate}
    \item \question{Montrer qu'il existe un vecteur $\vec x$ non nul tel que
    $u(\vec x) \perp \vec x$.}
\reponse{Soit $(\vec u_1,\dots,\vec u_n)$ une base propre pour $u$.
             On prend $\vec x = \vec u_1 + \dots + \vec u_n$.}
    \item \question{En déduire qu'il existe une base orthonormée $(\vec e_i)$ telle que :
    $\forall\ i,\ (u(\vec e_i) \mid \vec e_i) = 0$.}
\reponse{On norme $\vec x$ et on le complète en une base orthonormée.
             La matrice de $u$ dans cette base est symétrique, de trace nulle,
             et la diagonale commence par 0. On termine par récurrence.}
\end{enumerate}
}
