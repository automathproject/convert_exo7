\uuid{mEXs}
\exo7id{2641}
\auteur{debievre}
\datecreate{2009-05-19}
\isIndication{true}
\isCorrection{true}
\chapitre{Fonction de plusieurs variables}
\sousChapitre{Extremums locaux}

\contenu{
\texte{
Pour chacune des fonctions suivantes \'etudier la nature du point
critique donn\'e :
}
\begin{enumerate}
    \item \question{$f(x,y)=x^2-xy+y^2$ au point critique $(0,0)$ ;}
\reponse{$df=(2x-y)\mathrm dx + (2y-x)\mathrm dy$ et
$\mathrm{Hess}_f=\left[\begin{matrix} 2 & -1\\ -1 & 2 \end{matrix}\right]$ 
d'o\`u
\begin{align*}
(u,v)\mathrm{Hess}_f(0,0)\left[\begin{matrix} u \\ v \end{matrix}\right]
&=
(u,v)
\left[\begin{matrix} 2 & -1\\ -1 & 2 \end{matrix} \right]
\left[\begin{matrix} u \\ v \end{matrix}\right]
\\
&=
u(2u-v)+v(2v-u)=2(u^2-uv+v^2)
\\
&=
2\left(\left(u-\tfrac v2\right)^2 + \tfrac 34 v^2\right) .
\end{align*}
Par cons\'equent la forme hessienne au point $(0,0)$
est positive et ce point pr\'esente donc un minimum local.}
    \item \question{$f(x,y)=x^2+2xy+y^2+6$ au point critique $(0,0)$ ;}
\reponse{$f(x,y)=x^2+2xy+y^2+6= (x+y)^2+6$ d'o\`u 
le point $(0,0)$
pr\'esente un minimum local.}
    \item \question{$f(x,y)=x^3+2xy^2-y^4+x^2+3xy+y^2+10$ au point critique
$(0,0)$.}
\reponse{$df=(3x^2+2x+2y^2 +3y)\mathrm dx + (4xy-4y^3 +3x+2y)\mathrm dy$ et
\[
\mathrm{Hess}_f=\left[\begin{matrix} 
6x +2 & 4y+3\\ 4y+3 &  -12y^2 +4x +2
\end{matrix}\right]
\] 
d'o\`u
\begin{align*}
(u,v)\mathrm{Hess}_f(0,0)
\left[\begin{matrix} u \\ v \end{matrix}\right]&=
(u,v)\left[\begin{matrix} 
2 & 3\\ 3 &  2
\end{matrix}\right]
\left[\begin{matrix} u \\ v \end{matrix}\right]
\\
&=
(2u+3v)u+(3u+2v)v=2(u^2+3uv+v^2)
\\
&=
2\left(\left(u+\tfrac {3v}2\right)^2 - \tfrac 54 v^2\right) .
\end{align*}
Par cons\'equent la forme hessienne au point $(0,0)$
est non d\'eg\'en\'er\'ee et 
ind\'efinie et ce point pr\'esente  un point selle.}
\indication{Rappel: Pour qu'un point critique non d\'eg\'en\'er\'e
pr\'esente un maximum relatif (resp. minimum relatif) il faut et il suffit que
la forme hessienne en ce point soit n\'egative (resp. positive) ;
pour qu'un point critique non d\'eg\'en\'er\'e
pr\'esente un point selle il faut et il suffit que
la forme hessienne en ce point soit (non d\'eg\'en\'er\'ee et)
ind\'efinie.}
\end{enumerate}
}
