\uuid{vtM9}
\exo7id{4302}
\auteur{quercia}
\datecreate{2010-03-12}
\isIndication{false}
\isCorrection{true}
\chapitre{Calcul d'intégrales}
\sousChapitre{Intégrale impropre}

\contenu{
\texte{

}
\begin{enumerate}
    \item \question{Montrer que pour $0 \le x \le \sqrt n$ on a : $\Bigl(1-\frac{x^2}n\Bigr)^n \le e^{-x^2}$
    et pour $x$ quelconque : $e^{-x^2} \le \Bigl(1+\frac{x^2}n\Bigr)^{-n}$.}
    \item \question{Calculer les intégrales
    $I_n =  \int_{t=0}^{\sqrt n} \Bigl(1-\frac{t^2}n\Bigr)^n\, d t$ et
    $J_n =  \int_{t=0}^{+\infty} \Bigl(1+\frac{t^2}n\Bigr)^{-n}\, d t$
    en fonction des intégrales :
    $K_p =  \int_{t=0}^{\pi/2} \cos^pt\,d t$.}
    \item \question{On admet que $K_p \sim \sqrt{\frac\pi{2p}}$ quand $p\to\infty$.
    Calculer $ \int_{t=0}^{+\infty} e^{-t^2}\,d t$.}
\reponse{
$I_n = \sqrt nK_{2n+1}$,  $J_n = \sqrt nK_{2n-1}$.
$\frac{\sqrt\pi}2$.
}
\end{enumerate}
}
