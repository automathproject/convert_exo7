\uuid{rGhZ}
\exo7id{7063}
\auteur{megy}
\datecreate{2017-01-11}
\isIndication{true}
\isCorrection{true}
\chapitre{Géométrie affine dans le plan et dans l'espace}
\sousChapitre{Propriétés des triangles}

\contenu{
\texte{
%  triangle rectangle, cercle
On donne deux cercles $\mathcal C$ et $\mathcal C'$ de rayons distincts, de centres $O$ et $O'$, tangents extérieurement en  un point $A$. On admet qu'il existe trois tangentes communes  à $\mathcal C$ et $\mathcal C'$ : la tangente commune en $A$, qui est directement constructible, et deux autres droites. L'objectif de l'exercice est de tracer ces deux dernières tangentes.
}
\begin{enumerate}
    \item \question{Considérons donc une droite tangente à $\mathcal C$ en $B$ et à $\mathcal C'$ en $C$, avec $B\neq C$. La tangente commune en $A$ aux deux cercles coupe $(BC)$ en $I$. Montrer que $I$ est le milieu de $[BC]$ et que $ABC$ est rectangle en $A$.}
    \item \question{Finir l'exercice (c'est-à-dire construire $B$ et $C$) de l'une des deux façons suivantes:
\begin{enumerate}}
    \item \question{Soit $D$ tel que $ABDC$ soit un rectangle. Quels sont les points d'intersection entre $(DB)$, $(DC)$ et $(OO')$ ? En déduire une construction du point $D$.}
    \item \question{Montrer que $OIO'$ est rectangle en $I$ et en déduire une construction du point $I$.}
\reponse{
Par définition, $(IA)$ et $(IB)$ sont tangentes au cercle $\mathcal C$, donc $IA=IB$. On a de même $IA=IC$ et donc $I$ est le milieu de $[BC]$. Le triangle $ABC$ est donc un triangle d'écolier et il est rectangle en $A$.
}
\indication{\begin{enumerate}
\item Penser au triangle de l'écolier.
\end{enumerate}}
\end{enumerate}
}
