\uuid{1614}
\auteur{liousse}
\datecreate{2003-10-01}
\isIndication{false}
\isCorrection{false}
\chapitre{Réduction d'endomorphisme, polynôme annulateur}
\sousChapitre{Diagonalisation}

\contenu{
\texte{
Soit $E$ un espace vectoriel 
de dimension $n$ et $f$ une 
application lin\'eaire de $E$ dans $E$.
}
\begin{enumerate}
    \item \question{Montrer que la condition $f^2=0$ est \'equivalente \`a 
$Im f \subset Ker f$. Quelle
condition v\'erifie alors le rang de $f$ ? 
On suppose dans la suite que $f^2=0$.}
    \item \question{Soit $F$ un suppl\'ementaire de $Ker f$ dans $E$ et 
soit $(e_1, \ldots , e_r)$ une base de $F$.
Montrer que la famille des vecteurs 
$(e_1, \ldots , e_r,f(e_1), \ldots , f(e_r))$ est libre.
Montrer comment la compl\'eter si n\'ecessaire par 
des vecteurs de $Ker f$ pour obtenir une base de $E$.
Quelle est la matrice de $f$ dans cette base ?}
    \item \question{Sous quelle condition n\'ecessaire et suffisante a-t-on $Im f= Ker f$ ?}
    \item \question{Exemple. Soit $f$ une 
application lin\'eaire de $\Rr^3$ dans $\Rr^3$ de matrice 
dans la base canonique 
$M(f) = \left( \begin{array}{ccc} 1&0&1 \\ 2&0&2 \\ -1&0&-1 \\ \end{array}\right)$. 
Montrer que $f^2=0$.
D\'eterminer une nouvelle base dans laquelle la matrice de 
$f$ a la forme indiqu\'ee dans la question 2).}
\end{enumerate}
}
