\uuid{fYbx}
\exo7id{6335}
\auteur{queffelec}
\datecreate{2011-10-16}
\isIndication{false}
\isCorrection{false}
\chapitre{Théorème de Cauchy-Lipschitz}
\sousChapitre{Théorème de Cauchy-Lipschitz}

\contenu{
\texte{

}
\begin{enumerate}
    \item \question{On cherche à résoudre le problème
$$
 x'= t^2 +t x\quad , \quad  x(0) = 0 \; .
 $$
  \'Ecrire l'équation intégrale associée et utiliser
les cylindres de sécurités pour justifier que le procédé
itératif de Picard donne une suite de fonctions $(x_n)$
convergent uniformément sur $[-1/2, 1/2]$ vers une solution du
problème. Partant de $x_0 \equiv 0$, déterminer ensuite cette
suite $(x_n)$ et la solution du problème donné.}
    \item \question{Résoudre avec ce procédé itératif le problème
$$ x' = t x \quad , \quad x(0) =1 \; , $$
puis aussi
$$\begin{array}{ll}
  x'_1= x_2 x_3 \; ,& x_1 (0) = 0 \; , \\
  x'_2 = - x_1 x_3\; , & x_2(0) =1 \; ,\\
  x'_3 = 2\; , & x_3(0) =0\; ,
\end{array}$$
en commen\c{c}ant avec $x_0 (t) = (0,1,0)$.}
\end{enumerate}
}
