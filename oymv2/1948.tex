\uuid{e2v0}
\exo7id{1948}
\auteur{gineste}
\datecreate{2001-11-01}
\isIndication{false}
\isCorrection{false}
\chapitre{Série numérique}
\sousChapitre{Séries alternées}

\contenu{
\texte{
Pour tout
entier $n>0$, soit $u(n)=(-1)^n/n$ .
Soit $\sigma$ une permutation des entiers
$>0$ et soit $\tau$ la permutation réciproque. On suppose de plus que

(1) pour tout entier $p>0$ on a $\tau(2p-1)<\tau(2p+1)$ et
$\tau(2p)<\tau(2p+2).$

(2) Notant par $p(n)$ le nombre d'entiers $k$ tels que $1\leq k\leq n$ et
$\sigma(k)$ est pair, alors $\alpha=\lim _{n\infty}p(n)/n$ existe et est dans
$]0,1[.$
}
\begin{enumerate}
    \item \question{Dans le cas particulier o\`u $\sigma$ est définie par
$$\sigma(3p)=2p,\ \sigma(3p+1)=4p+1,\ \sigma(3p+2)=4p+3$$
pour tout entier $p>0,$ calculer explicitement $\tau$, et vérifier que $\sigma$ satisfait
(1) et (2), en calculant $p(n)$ pour tout $n$ ainsi que $\alpha.$}
    \item \question{On note $f(n)=\sum_{k=1}^n1/k-\log n,$ et on rappelle le fait,
vu en cours, que  $\lim _{n\infty}f(n)=\gamma$ existe (Constante d'Euler).
On revient au cas général pour $\sigma$, on considère la série de terme
général $v_n=u(\sigma(n))$ et on note $s_n=v_1+\cdots+v_n$.}
    \item \question{Montrer par récurrence  que
$s_n=\sum_{k=1}^{p(n)}\frac{1}{2k}-\sum_{k=1}^{n-p(n)}\frac{1}{2k-1}$ et que
$$s_n=\frac{1}{2}f(p(n))+\frac{1}{2}f(n-p(n))-f(2n-2p(n))
+\frac{1}{2}\log\frac{p(n)}{n-p(n)}-\log 2.$$
En déduire que $\sum _{n=1}^{\infty}v_n$ converge
et calculer sa somme en fonction de $\alpha.$}
\end{enumerate}
}
