\uuid{VJRo}
\exo7id{4138}
\auteur{quercia}
\datecreate{2010-03-11}
\isIndication{false}
\isCorrection{true}
\chapitre{Fonction de plusieurs variables}
\sousChapitre{Dérivée partielle}

\contenu{
\texte{
Soit $f(x,y) = \arcsin\left(\frac{1+xy}{\sqrt{(1+x^2)(1+y^2)}}\right)$
et $g(x,y) = \arctan x - \arctan y$.
}
\begin{enumerate}
    \item \question{Vérifier que $f$ est définie sur $\R^2$.}
    \item \question{Calculer les dérivées partielles premières de $f$ et de $g$.}
    \item \question{Simplifier $f$ à l'aide de $g$.}
\reponse{
$\frac{\partial f}{\partial x} = \frac{\pm 1}{1+x^2}$, $+$ si $y > x$, $-$ si $y < x$.\par
             $\frac{\partial f}{\partial y} = \frac{\pm 1}{1+y^2}$, $-$ si $y > x$, $+$ si $y < x$.
Pour $y \ge x$, $f(x,y) = \frac\pi2 + g(x,y)$,\quad
             pour $y \le x$, $f(x,y) = \frac\pi2 - g(x,y)$.
}
\end{enumerate}
}
