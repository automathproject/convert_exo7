\uuid{oTNz}
\exo7id{7209}
\auteur{megy}
\datecreate{2019-12-27}
\isIndication{true}
\isCorrection{true}
\chapitre{Nombres complexes}
\sousChapitre{Forme cartésienne, forme polaire}

\contenu{
\texte{
Résoudre sur $\C$ l'équation $z^2+\overline{z}-1=0$.
}
\indication{Remarquer que la conjugaison est involutive. On peut aussi utiliser la forme algébrique.}
\reponse{
Soit $z\in\C$. Si $\overline{z}=1-z^2$, alors en conjuguant une deuxième fois on obtient : 
\[ z=1-\overline{z}^2=1-(1-z^2)^2=2z^2-z^4.\]
Donc $z$ est une racine du polynôme $X^4-2X^2+X$. Ce polynôme a deux racines évidentes : $0$ et $1$, et deux autres racines $\frac{-1\pm\sqrt 5}{2}$. Les deux dernières sont les seules à vérifier l'équation d'origine.

Autre approche : on écrit l'inconnue $z\in\C$ sous forme cartésienne $z=a+ib$ avec $a$ et $b$ réels. On obtient le système $\begin{cases}a^2-b^2+a&=1\\2ab-b&=0\end{cases}$.

La seconde équation équivaut à $b=0$ ou $a=\frac{1}{2}$. Si $b=0$, la première équation devient $a^2+a-1=0$, dont les solutions (réelles) sont $\frac{-1\pm\sqrt 5}{2}$. Si $a=1/2$, la première équation devient $b^2=-\frac{1}{4}$, qui n'a pas de solutions réelles.
}
}
