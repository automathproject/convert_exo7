\uuid{690}
\auteur{bodin}
\datecreate{1998-09-01}

\contenu{
\texte{
Soit $f : \Rr \to \Rr$ définie par $f(x) = \frac{\cos x}{1+x^2}.$
Montrer que $f$ est major\'ee sur $\Rr$, minor\'ee sur $\Rr$.
D\'eterminer $\sup_{x\in \Rr} f(x)$.
}
\reponse{
Pour tout $x\in\R$ on a :
          $$ 0\leq |f(x)| = \frac{|\cos x|}{1+x^2} \leq \frac{1}{1+x^2}\leq 1. $$
Par cons\'equent, pour tout $x\in\R$, $f(x)\in [-1,1] $ donc $f$
est minor\'ee ($-1$ est un minorant), major\'ee ($1$ est un
majorant) et $\sup_{x\in\R}f(x)\leq 1$. Comme $f(0)=1$ on a
n\'ecessairement $\sup_{x\in\R}f(x) \geq 1$. Conclusion :
        $$   \sup_{x\in\R}f(x)=1.$$
}
}
