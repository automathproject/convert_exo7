\uuid{5Zub}
\exo7id{7161}
\auteur{megy}
\datecreate{2017-05-13}
\isIndication{false}
\isCorrection{true}
\chapitre{Géométrie affine euclidienne}
\sousChapitre{Géométrie affine euclidienne du plan}

\contenu{
\texte{
Soit $\mathcal C=ABCD$ un carré plein du plan (c'est-à-dire l'enveloppe convexe de ses sommets) et $O$ son centre.
}
\begin{enumerate}
    \item \question{Soit $g\in \operatorname{Isom}(\mathcal C)$ une isométrie du carré, c'est-à-dire une isométrie du plan préservant le carré. Montrer qu'elle permute les sommets, puis que c'est une rotation de centre $O$ ou bien une réflexion d'axe contenant $O$.}
\reponse{Montrons d'abord qu'une isométrie du carré permute les sommets.
\begin{enumerate}}
    \item \question{Déterminer entièrement  le groupe $G=\operatorname{Isom}(\mathcal C)$.}
\reponse{Première preuve : 
considérons une diagonale, par exemple $[AC]$. Comme $g$ est une isométrie, la distance entre $A$ et $C$ est la même qu'entre leurs images. Or, deux points du carré à distance $AC$ sont forcément deux sommets d'une diagonale du carré. L'isométrie $g$ permute donc les sommets.}
    \item \question{Décrire le groupe $H=\operatorname{Isom}^+(\mathcal C)$ des isométries directes du carré et écrire un isomorphisme entre ce groupe et $\Z/4\Z$.}
\reponse{Deuxième preuve :  
une isométrie est affine, donc préserve les barycentres, donc envoie les points extrémaux (ceux qui ne sont pas des barycentres d'autres points du carré) sur d'autres points extrémaux, et donc permute les sommets.}
\end{enumerate}
}
