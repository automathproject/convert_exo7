\uuid{Rbx5}
\exo7id{5249}
\auteur{rouget}
\datecreate{2010-07-04}
\isIndication{false}
\isCorrection{true}
\chapitre{Suite}
\sousChapitre{Autre}

\contenu{
\texte{
Calculer $\mbox{inf}_{\alpha\in]0,\pi[}(\mbox{sup}_{n\in\Nn}(|\sin(n\alpha)|))$.
}
\reponse{
Pour $\alpha\in]0,\pi[$, posons $f(\alpha)=\mbox{sup}_{n\in\Nn}(|\sin(n\alpha)|)$. $\{(\sin(n\alpha),\;n\in\Nn\}$ est une partie non vide et majorée (par $1$) de $\Rr$. Donc, pour tout réel $\alpha$ de $]0,\pi[$, $f(\alpha)$ existe dans $\Rr$.

Si $\alpha$ est dans $[\frac{\pi}{3},\frac{2\pi}{3}]$,

$$f(\alpha)=\mbox{sup}_{n\in\Nn}(|\sin(n\alpha)|)\geq\sin\alpha\geq\frac{\sqrt{3}}{2}=f(\frac{\pi}{3}).$$

Si $\alpha$ est dans $]0,\frac{\pi}{3}]$. Soit $n_0$ l'entier naturel tel que $(n_0-1)\alpha<\frac{\pi}{3}\leq n_0\alpha$ ($n_0$ existe car la suite $(n\alpha)_{n\in\Nn}$ est strictement croissante). Alors, 
 
$$\frac{\pi}{3}\leq n_0\alpha=(n_0-1)\alpha+\alpha<\frac{\pi}{3}+\alpha\leq\frac{\pi}{3}+\frac{\pi}{3}=\frac{2\pi}{3}.$$

Mais alors,

$$f(\alpha)=\mbox{sup}_{n\in\Nn}(|\sin(n\alpha)|)\geq|\sin(n_0\alpha)|\geq\frac{\sqrt{3}}{2}=f(\frac{\pi}{3}).$$

Si $\alpha$ est dans $[\frac{2\pi}{3},\pi[$, on note que 
$$f(\alpha)=\mbox{sup}_{n\in\Nn}(|\sin(n\alpha)|)=\mbox{sup}_{n\in\Nn}(|\sin(n(\pi-\alpha)|)=f(\pi-\alpha)\geq f(\frac{\pi}{3}),$$

car $\pi-\alpha$ est dans $]0,\frac{\pi}{3}]$.

On a montré que $\forall\alpha\in]0,\pi[,\;f(\alpha)\geq f(\frac{\pi}{3})=\frac{\sqrt{3}}{2}$. Donc, $\mbox{inf}_{\alpha\in]0,\pi[}(\mbox{sup}_{n\in\Nn}(|\sin(n\alpha)|))$ existe dans $\Rr$ et 

$$\mbox{inf}_{\alpha\in]0,\pi[}(\mbox{sup}_{n\in\Nn}(|\sin(n\alpha)|))= \mbox{Min}_{\alpha\in]0,\pi[}(\mbox{sup}_{n\in\Nn}(|\sin(n\alpha)|))=f(\frac{\pi}{3})=\frac{\sqrt{3}}{2}.$$
}
}
