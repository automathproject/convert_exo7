\uuid{iG1S}
\exo7id{2508}
\auteur{queffelec}
\datecreate{2009-04-01}
\isIndication{false}
\isCorrection{true}
\chapitre{Différentiabilité, calcul de différentielles}
\sousChapitre{Différentiabilité, calcul de différentielles}

\contenu{
\texte{
Soit, dans ${\Rr^n}$, $F$
un sous-espace ferm\'e, et soit $f:{\Rr^n}\to\Rr$ d\'efinie par
$f(x)=d(x,F)$. On rappelle que $f$ est $1$-lipschitzienne, et que
pour chaque $x$ il existe $y\in F$ tel que $f(x)=d(x,y)$.
}
\begin{enumerate}
    \item \question{On suppose que $f$ est diff\'erentiable en $x\notin F$.
Montrer que $||Df(x)||_{{\cal L}(\Rr^n,\Rr)}\leq 1$.}
\reponse{Pour montrer que $||Df(x)||_{{\cal L}(\Rr^n,\Rr)}\leq 1$,
il faut montrer que si $h \in \mathbb{R}^n$, on a $|Df(x).h|\leq
||h||$. On a $$|Df(x).h|=|D_hf(x)|=|\lim_{t \rightarrow
0}\frac{f(x+th)-f(x)}{t}|.$$ Or $f$ est $1$-lipschitzienne et donc
$|f(x+th)-f(x)|\leq ||th||=t||h||.$ Par cons\'equent pour tout $h
\in \mathbb{R}^n$, $|Df(x).h|\leq ||h||$ ce qui donne
l'in\'egalit\'e demand\'ee.}
    \item \question{On consid\`ere la fonction $\varphi : t\in[0,1]\to
f((1-t)x+ty)$; en calculant $\varphi'(0)$ de deux fa\c cons,
montrer que $Df(x).{\frac{x-y}{||x-y||}}=1$ et $||Df(x)||_{{\cal
L}(\Rr^n,\Rr)}= 1$.}
\reponse{$$\varphi'(0)=\lim_{t
\rightarrow 0}\frac{\varphi(t)-\varphi(0)}{t-0}=\lim_{t
\rightarrow 0}\frac{f((1-t)x+ty)-f(x)}{t}=$$ $$\lim_{t \rightarrow
0}\frac{f(x+t(y-x))-f(x)}{t}=Df(x).(y-x).$$ Ou encore, soit $\psi:
\mathbb{R} \rightarrow \mathbb{R}^n$ l'application
$\psi(t)=(1-t)x+ty$, on a alors $\varphi(t)=f \circ \psi$ et
d'apr\`es la formule de diff\'erentielle d'une composition:
$$\varphi'(0)=Df(\psi(0)).D\psi(0)=Df(x).(y-x).$$
Or,
$$d(x,F)=d(x,y)=||x-y||=\frac{1}{1-t}||(1-t)(x-y)||=$$
$$\frac{1}{1-t}||[(1-t)x+ty]-[ty+(1-t)y]||
=\frac{1}{1-t}d((1-t)x+ty,y).$$ Notons $x_t=(1-t)x+ty$, on a alors
$$d(x_t,y)=(1-t)d(x,F).$$
Or, $\varphi(t)=d(x_t,F) \leq d(x_t,y) \leq (1-t)d(x,y) \leq
\varphi(0)$ et donc
$$|\varphi'(0)|=\lim_{t
\rightarrow 0}\frac{|\varphi(t)-\varphi(0)|}{t} =\lim_{t
\rightarrow 0}\frac{\varphi(0)-\varphi(t)}{t}\geq$$
$$ \lim_{t \rightarrow 0}\frac{d(x,y)-(1-t)d(x,y)}{t} \geq
d(x,y)=||x-y||.$$ Donc
$$|Df(x)(x-y)| \geq ||x-y||$$ d'o\`u la deuxi\`eme inégalité.}
    \item \question{En d\'eduire que $y$ est unique.}
\reponse{Raisonnons par l'absurde. Supposons qu'il existe deux point
$y_1$ et $y_2$ tels que $d(x,F)=d(x,y_1)=d(x,y_2).$ Alors, de la
même manière que précédement, on a
$Df(x).(x-y_1)=Df(x).(x-y_2)=d(x,F)$ et donc
$Df(x).(x-y_1+x-y_2)=2d(x,F)$. Or, $||x-y_1+x-y_2||< 2d(x,F)$ car
les vecteurs $x-y_1$ et $x-y_2$ ne sont pas alignés. Mais alors
cela contredit le fait que $||Df(x)||_{\cal
L(\mathbb{R}^n,\mathbb{R})}=1$.}
\end{enumerate}
}
