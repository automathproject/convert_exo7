\uuid{4664}
\auteur{quercia}
\datecreate{2010-03-14}

\contenu{
\texte{
Soit $E$ l'ensemble des fonctions de~$\R$ dans~$\C$ de la forme~:
$f(x)=\sum_{n=-\infty}^{+\infty} c_ne^{2i\pi nx}$ où $\sum |c_n|$ converge.
On pose pour $f\in E$~: $\|f\| = \sum_{n=-\infty}^{+\infty}|c_n|$.
}
\begin{enumerate}
    \item \question{Justifier la définition de $\|f\|$ et montrer que $E$ est un espace vectoriel normé complet.}
    \item \question{Montrer que $E$ est une $\C$-algèbre et que $\|fg\| \le \|f\|\,\|g\|$.}
    \item \question{Soit $\varphi : E \to \C$ un morphisme d'algèbres.
  \begin{enumerate}}
    \item \question{On suppose $\varphi$ continu, montrer qu'il existe $z_0\in\mathbb{U}$ tel
    que $\forall\ f\in E,\ \varphi(f) = \sum_{n=-\infty}^{+\infty}c_nz_0^n$.}
    \item \question{Vérifier que la formule précédente définit effectivement un morphisme
    continu de~$E$ dans~$\C$.}
\reponse{
Si $f=0$ alors $c_n = 0$ pour tout~$n$ par intégration
    terme à terme. Donc une fonction $f\in E$ possède un unique développement
    trigonométrique et $\|f\|$ est bien défini. Alors $E$ est isomorphe
    à $\ell^1(\Z)$ qui est un espace vectoriel normé complet.
Produit de convolution de deux $\Z$-suites sommables.
\begin{enumerate}
$z_0 = \varphi(x \mapsto e^{2i\pi x})$. On a $|z_0| = 1$ car la suite
    $(\varphi(x \mapsto e^{2in\pi x})_{n\in\Z})$ est bornée.
}
\end{enumerate}
}
