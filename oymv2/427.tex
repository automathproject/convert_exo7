\uuid{427}
\auteur{cousquer}
\datecreate{2003-10-01}

\contenu{
\texte{
Trouver le polynôme $P$ de degré inférieur ou égal à $3$ tel que :
$$P(0)=1\quad\text{et}\quad P(1)=0\quad\text{et}\quad P(-1)=-2\quad\text{et}\quad P(2)=4.$$
}
\reponse{
On cherche $P$ sous la forme $P(X)=aX^3+bX^2+cX+d$, ce qui donne le système linéaire suivant à résoudre:
$$\left\{\begin{array}{rcrcrcrcr} &&&&&&d&=&1\\a&+&b&+&c&+&d&=&0\\-a&+&b&-&c&+&d&=&-2\\8a&+&4b&+&2c&+&d&=&4\end{array}\right.$$
Après calculs, on trouve une unique solution :
 $a=\frac{3}{2}$, $b=-2$, $c=-\frac{1}{2}$, $d=1$ c'est-à-dire 
 $$P(X)=\frac{3}{2}X^3-2X^2-\frac{1}{2}X+1.$$
}
}
