\uuid{5656}
\auteur{rouget}
\datecreate{2010-10-16}
\isIndication{false}
\isCorrection{true}
\chapitre{Réduction d'endomorphisme, polynôme annulateur}
\sousChapitre{Polynôme caractéristique, théorème de Cayley-Hamilton}

\contenu{
\texte{
Soit $A$ une matrice rectangulaire de format $(p,q)$ et $B$ une matrice de format $(q,p)$. Comparer les polynômes caractéristiques de $AB$ et $BA$.
}
\reponse{
Si $p = q$, le résultat est connu : $\chi_{AB}=\chi_{BA}$.

Supposons par exemple $p < q$. On se ramène au cas de matrices carrées en complétant. 
Soient $A'=\left(
\begin{array}{c}
A\\
0_{q-p,q}
\end{array}
\right)$ et $B'=\left(
\begin{array}{cc}
B&0_{q,q-p}
\end{array}
\right)$. $A'$ et $B'$ sont des matrices carrées de format $q$ et $A'B'$ et $B'A'$ ont même polynôme caractéristique.

Un calcul par blocs donne $B'A'=BA$ et $A'B'=\left(
\begin{array}{c}
A\\
0_{q-p,q}
\end{array}
\right)\left(
\begin{array}{cc}
B&0_{q,q-p}
\end{array}
\right)=\left(
\begin{array}{cc}
AB&0_{p,q-p}\\
0_{q-p,p}&0_{q-p,q-p}
\end{array}
\right)$. Donc $\chi_{BA}=(-X)^{q-p}\chi_{AB}$ ou encore, avec une écriture plus symétrique, $(-X)^p\chi_{BA}= (-X)^q\chi_{AB}$ ce qui vrai dans tous les cas.

\begin{center}
\shadowbox{
$\forall A\in\mathcal{M}_{p,q}(\Kk)$, $\forall B\in\mathcal{M}_{q,p}(\Kk)$, $(-X)^p\chi_{BA}= (-X)^q\chi_{AB}$.
}
\end{center}
}
}
