\uuid{2037}
\exo7id{2389}
\auteur{mayer}
\datecreate{2003-10-01}
\isIndication{true}
\isCorrection{true}
\chapitre{Connexité}
\sousChapitre{Connexité}

\contenu{
\texte{
Soit $I$ un intervalle ouvert de $\Rr$ et soit $f:I\to \Rr$
une application d\'erivable. Notons $A=\{(x,y)\in I\times I \; ;\;
\; x<y \}$.
}
\begin{enumerate}
    \item \question{Montrer que $A$ est une partie connexe de $\Rr^2$.}
\reponse{$A$ est connexe car connexe par arcs.}
    \item \question{Pour $(x,y)\in A$, posons $g(x,y) =\frac{f(y)-f(x)}{y-x}$.
Montrer que $g(A)\subset f'(I) \subset \overline{g(A)}$.}
\reponse{Si $z \in g(A)$ alors il existe $(x,y) \in A$ tel que $g(x,y)=z$.
Donc $z = \frac{f(y)-f(x)}{y-x}$ par le théorème des accroissements finis il existe $t \in ]x,y[ \subset I$ tel que $z = f'(t)$ donc $z \in f'(I)$.
Donc $g(A) \subset f'(I)$.

Si maintenant $z \in f'(I)$, il existe $y\in I$ tel que $z= f'(y)$, mais par définition de la dérivée $f'(y)$ est la limite de $\frac{f(y)-f(x)}{y-x}$  quand $x$ tend vers $y$ (et on peut m\^eme dire que c'est la limite à gauche, i.e. $x<y$). Donc $f'(y)$ est limite de points de $g(x,y)$ avec $x<y$, donc de points de $A$.
Conclusion $z=f'(y)$ est dans $\overline {g(A)}$, et donc $f'(I) \subset \overline {g(A)}$.}
    \item \question{Montrer que $f'(I)$ est un intervalle.}
\reponse{$A$ est connexe, $g$ est continue sur $A$ donc $g(A)$ est un connexe de $\Rr$. Par l'exercice \ref{exocon} comme on a 
$$ g(A)\subset f'(I) \subset \overline{g(A)}$$
avec  ${g(A)}$ connexe alors $f'(I)$ est connexe. Comme $f'(I)$ est un connexe de $\Rr$ c'est un intervalle.}
\indication{\begin{enumerate}
  \item Faire un dessin !

  \item Utiliser le théorème des accroissements finis d'une part. La définition de la dérivée d'autre part.

  \item Utiliser l'exercice \ref{exocon} ou refaire la demonstration.
\end{enumerate}}
\end{enumerate}
}
