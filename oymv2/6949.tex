\uuid{fQYu}
\exo7id{6949}
\auteur{ruette}
\datecreate{2013-01-24}
\isIndication{false}
\isCorrection{true}
\chapitre{Loi, indépendance, loi conditionnelle}
\sousChapitre{Loi, indépendance, loi conditionnelle}

\contenu{
\texte{

}
\begin{enumerate}
    \item \question{Soit $X$ une variable aléatoire de loi de Poisson $\mathcal{P}(\lambda)$. Calculer sa
fonction caractéristique.}
\reponse{$\displaystyle\varphi_X(t)=\sum_{k=0}^{+\infty}e^{itk}e^{-\lambda}
\frac{\lambda^k}{k!}=e^{-\lambda}\sum_{k=0}^{+\infty}
\frac{(\lambda e^{it})^k}{k!}=e^{-\lambda}\exp(\lambda e^{it})$, donc
$\varphi_X(t)=\exp(\lambda(e^{it}-1))$.}
    \item \question{Soit $Y$ une variable aléatoire indépendante de $X$ telle que $P_Y=\mathcal{P}(\lambda')$.
Quelle est la fonction caractéristique de $X+Y$ ? En déduire
que $\mathcal{P}(\lambda)*\mathcal{P}(\lambda')=\mathcal{P}(\lambda+\lambda')$.}
\reponse{$\varphi_Y(t)=\exp(\lambda'(e^{it}-1))$ par 1. Par indépendance,
$\varphi_{X+Y}(t)=\varphi_X(t)\varphi_Y(t)=
\exp((\lambda+\lambda')(e^{it}-1))$. C'est la fonction caractéristique
de la loi de Poisson $\mathcal{P}(\lambda+\lambda')$. Donc
$\mathcal{P}(\lambda)*\mathcal{P}(\lambda')=\mathcal{P}(\lambda+\lambda')$.}
\end{enumerate}
}
