\uuid{6422}
\auteur{potyag}
\datecreate{2011-10-16}
\isIndication{false}
\isCorrection{false}
\chapitre{Géométrie projective}
\sousChapitre{Géométrie projective}

\contenu{
\texte{

}
\begin{enumerate}
    \item \question{Montrer que  le groupe projectif $PGL(L)$ agit transitivement  sur
  l'ensemble de sous-espaces projectifs de la dimension $k$ fixée.}
    \item \question{Montrer que $PGL(L)$ agit transitivement sur l'ensemble de couples
ordonnés

$\{(P_1, P_2)\ \vert\  {\rm dim} P_1= k_1, \  {\rm dim} P_2= k_2,\
{\rm dim} (P_1\cap P_2) = k_3\},$ où $k_1, k_2, k_3$ sont fixés.}
    \item \question{Montrer que $PGL(L)$ agit transitivement sur l'ensemble des drapeaux
projectifs

${\cal D}=\{ (P_1,...,P_k)\ \vert P_1\subset ... \subset P_k, \}$ où la
longueur  $k$ est fixée et  $P_i$ est un sous-espace projectif de $P(L)$ de
dimension $i$ fixée $(i=1,...,k).$}
\end{enumerate}
}
