\uuid{5854}
\auteur{rouget}
\datecreate{2010-10-16}

\contenu{
\texte{
On munit $E=\Rr[X]$ de la norme $\|\;\|_\infty$ définie par : $\forall P\in E$, $\|P\|_\infty=\text{Sup}\left\{\left| \frac{P^{(n)}(0)}{n!}\right|,\;n\in\Nn\right\}$.
}
\begin{enumerate}
    \item \question{Vérifier brièvement que $\|\;\|_\infty$ est une norme sur $E$.}
\reponse{\textbullet~Soit $P\in E$. Si on pose $P=\sum_{k=0}^{+\infty}a_kX^k$, il existe $n\in\Nn$ tel que $\forall k>n$, $a_k=0$. Donc $\|P\|_\infty=\text{Sup}\left\{\left| \frac{P^{(k)}(0)}{k!}\right|,\;k\in\Nn\right\}=\text{Max}\{|a_k|,\;0\leqslant k\leqslant n\}$ existe dans $\Rr$.

\textbullet~$\forall P\in E$, $\|P\|_\infty\geqslant0$.

\textbullet~Soit $P\in E$. $\|P\|_\infty=0\Rightarrow\forall k\in\Nn,\;|a_k|\leqslant0\Rightarrow\forall k\in\Nn,\;a_k=0\Rightarrow P=0$.

\textbullet~Soient $P\in E$ et $\lambda\in\Rr$. $\|\lambda P\|_\infty=\text{Max}\{|\lambda a_k|,\;0\leqslant k\leqslant n\}=|\lambda|\text{Max}\{|a_k|,\;0\leqslant k\leqslant n\}=|\lambda|\|P\|_\infty$.

\textbullet~Soient $P=\sum_{k\geqslant0}^{}a_kX^k$ et $Q=\sum_{k\geqslant0}^{}b_kX^k$ deux polynômes. Pour $k\in\Nn$, $|a_k+b_k|\leqslant|a_k|+|b_k|\leqslant\|P\|_\infty+\|Q\|_\infty$ et donc $\|P+Q\|_\infty\leqslant\|P\|_\infty+\|Q\|_\infty$.

\begin{center}
\shadowbox{
$\|\;\|_\infty$ est une norme sur $E$.
}
\end{center}}
    \item \question{Soit $f$ l'endomorphisme de $E$ défini par $\forall P\in E$, $f(P)=XP$. Démontrer que l'application $f$ est continue sur $(E,\|\;\|_\infty)$ et déterminer $|||f|||$.}
\reponse{$\forall P\in E$, $\|f(P)\|_\infty=\|P\|_\infty$ et donc $\forall P\in E\setminus\{0\}$, $ \frac{\|f(P)\|_\infty}{\|P\|_\infty}=1$. On en déduit que $\text{Sup}\left\{ \frac{\|f(P)\|_\infty}{\|P\|_\infty},\;P\in E\setminus\{0\}\right\}=1$. Ceci montre tout à la fois que $f$ est continue sur $(E,\|\;\|_\infty)$ et $|||f|||=1$.

\begin{center}
\shadowbox{
$f$ est continue sur $(E,\|\;\|_\infty)$ et $|||f|||=1$.
}
\end{center}}
\end{enumerate}
}
