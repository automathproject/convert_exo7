\uuid{aw3g}
\exo7id{2324}
\auteur{matexo1}
\datecreate{2002-02-01}
\isIndication{false}
\isCorrection{false}
\chapitre{Calcul d'intégrales}
\sousChapitre{Théorie}

\contenu{
\texte{
Soient $f$ et $g$ deux fonctions r\'eelles p\'eriodiques de 
p\'eriode $T$ continues sur $\Rr$. On appelle {\sl produit de 
convolution} de $f$ et $g$ la fonction $h$ not\'ee $f \star g$
et d\'efinie par
$$ h(x) = {1 \over T} \int_0^T f(t) g(x-t) ~dt. $$
}
\begin{enumerate}
    \item \question{Montrer que $h$ est une fonction p\'eriodique de 
p\'eriode $T$.}
    \item \question{Montrer
$$h(x)={1 \over T} \int_a^{a+T} f(t) g(x-t) ~dt, \qquad \forall a \in \Rr.$$}
    \item \question{En d\'eduire que $f \star g = g \star f$.}
\end{enumerate}
}
