\uuid{1448}
\auteur{hilion}
\datecreate{2003-10-01}
\isIndication{false}
\isCorrection{false}
\chapitre{Groupe quotient, théorème de Lagrange}
\sousChapitre{Groupe quotient, théorème de Lagrange}

\contenu{
\texte{

}
\begin{enumerate}
    \item \question{Montrer que les sous-groupes de $\Zz$ sont de la forme $n\Zz$ o\`u $n\in\Nn$.
(indication: utiliser la division euclidienne).}
    \item \question{Rappeler pourquoi ces sous-groupes sont distingués. On peut donc considérer les groupes quotients $\Zz/n\Zz$.}
    \item \question{Montrer que $\Zz/n\Zz$ est isomorphe au groupe des racines $n^{\text{i\`eme}}$ de l'unité.}
    \item \question{Montrer que $\Zz/n\Zz$ est isomorphe au groupe engendré par un cycle de longueur $n$ dans $S_N$ ($N\geq n)$.}
    \item \question{Plus généralement, montrer qu'il existe, à isomorphisme près, un seul groupe monogène (ie engendré par un seul élément) d'ordre $n$, appelé groupe cyclique d'ordre $n$.}
\end{enumerate}
}
