\uuid{Ddhi}
\exo7id{3804}
\auteur{quercia}
\datecreate{2010-03-11}
\isIndication{false}
\isCorrection{true}
\chapitre{Espace euclidien, espace normé}
\sousChapitre{Endomorphismes auto-adjoints}

\contenu{
\texte{
Soit $E$ un espace euclidien de dimension~$n$ et $p$ endomorphismes
autoadjoints $u_1,\dots,u_p$. Soit $q_i$ la forme quadratique associée à~$u_i$
($q_i(x) = (u_i(x)\mid x)$). On suppose~:
$$\forall\ x\in E,\ q_1(x) + \dots + q_p(x) =
\|x\|^2\quad\text{et}\quad\mathrm{rg}(u_1)+\dots+\mathrm{rg}(u_p) = n.$$
}
\begin{enumerate}
    \item \question{Montrer que $u_1+\dots+u_p = \mathrm{id}_E$.}
\reponse{$u_1+\dots+u_p$ est l'endomorphisme autoadjoint associé à
         $q_1+\dots+q_p$.}
    \item \question{Montrer que $\Im(u_1)\oplus\dots\oplus\Im(u_p) = E$.}
\reponse{$\Im(u_1)+\dots+\Im(u_p) \supset\Im(u_1+\dots+u_p) = E$ et la somme
         des dimensions est égale à~$\dim E$ donc la somme des sous-espaces est directe.}
    \item \question{Montrer que les~$u_i$ sont en fait des projecteurs orthogonaux et que la
    somme précédente est orthogonale.}
\reponse{On a $\mathrm{Ker}(u_1) = \{x\in E\text{ tq }x = u_2(x) + \dots + u_p(x)\}
\subset \Im(u_2+\dots+u_p) = \Im(u_2)\oplus\dots\oplus\Im(u_p)$ et les deux
termes extrêmes ont même dimension, d'où $\mathrm{Ker}(u_1) = \Im(u_2)\oplus\dots\oplus\Im(u_p)$.
Comme $u_1$ est autoadjoint, $\Im(u_1) \perp \mathrm{Ker}(u_1)$ ce qui prouve
l'orthogonalité de la somme. De plus $\Im(u_1)\subset \mathrm{Ker}(u_j)$ pour $j\ge 1$
donc $q_1(x) = \|x\|^2$ pour tout~$x\in\Im(u_1)$. En appliquant {\bf 1)} à
$\Im(u_1)$ on obtient $u_1(x)=x$ pour tout~$x\in\Im(u_1)$ ce qui prouve
que~$u_1$ est un projecteur, et c'est un projecteur orthogonal car autoadjoint.}
\end{enumerate}
}
