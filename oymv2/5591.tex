\uuid{wFUP}
\exo7id{5591}
\auteur{rouget}
\datecreate{2010-10-16}
\isIndication{false}
\isCorrection{true}
\chapitre{Application linéaire}
\sousChapitre{Morphismes particuliers}

\contenu{
\texte{
\label{ex:rou29}
Soient $p_1$,..., $p_n$ $n$ projecteurs d'un $\Cc$-espace de dimension finie. Montrer que $(p_1+...+p_n\;\text{projecteur})\Leftrightarrow\forall i\neq j,\;p_i\circ p_j=0$.
}
\reponse{
$\Leftarrow$/ Si $\forall i\neq j$, $p_i\circ p_j= 0$ alors 

\begin{center}
$(p_1+...+p_n)^2=p_1^2+...+p_n^2+\sum_{i\neq j}^{}p_i\circ p_j = p_1+...+p_n$,
\end{center}

et $p_1+...+p_n$ est un projecteur.

$\Rightarrow$/ Supposons que $p=p_1 + ... + p_n$ soit un projecteur. Posons $F_i=\text{Im}p_i$, $1\leqslant i\leqslant n$, puis $F=F_1 + ... + F_n$ et $G =\text{Im}p$.
On sait que la trace d'un projecteur est son rang. Par linéarité de la trace, on obtient

\begin{center}
$\text{rg}p=\text{Tr}p=\text{Tr}(p_1) + ...+\text{Tr}(p_n)=\text{rg}(p_1)+... +\text{rg}(p_n)$,
\end{center}

et donc $\text{dim}G=\text{dim}F_1+...+\text{dim}F_n\geqslant\text{dim}F$. D'autre part, $G=\text{Im}(p_1 + ... + p_n)\subset\text{Im}p_1 + ... +\text{Im}p_n=F_1 + ... +F_n =F$.

On obtient donc $G = F$ et aussi $\text{dim}(F_1+...+F_n)=\text{dim}F_1+...+\text{dim}F_n$. D'après l'exercice \ref{ex:rou15}, $F=F_1\oplus...\oplus F_n$ c'est-à-dire 

\begin{center}
$\text{Im}p =\text{Im}(p_1)\oplus ...\oplus\text{Im}(p_n)$.
\end{center}

Il reste à vérifier que pour $i\neq j$ et $x$ dans $E$, $p_i(p_j(x))=0$ ou ce qui revient au même que pour $i\neq j$ et $y$ dans $\text{Im}(p_j)$, $p_i(y)=0$.

Soit $y$ dans $\text{Im}(p_j)$ (et donc dans $\text{Im}p$). Les égalités $y=p_j(y)=p(y)$ fournissent  $\sum_{i\neq j}^{}p_i(y) = 0$. La somme $\sum_{i}^{}\text{Im}(p_i)$ étant directe, on a donc $p_i(y) = 0$ pour chaque $i\neq j$ ce qu'il fallait démontrer.

\begin{center}
\shadowbox{
$p_1+\ldots+p_n$ projecteur$\Leftrightarrow\forall i\neq j,\;p_i\circ p_j=0$.
}
\end{center}
}
}
