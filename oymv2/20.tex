\uuid{20}
\auteur{bodin}
\datecreate{1998-09-01}
\isIndication{true}
\isCorrection{true}
\chapitre{Nombres complexes}
\sousChapitre{Forme cartésienne, forme polaire}

\contenu{
\texte{
Soit $z$ un nombre complexe de module $\rho$,
d'argument $\theta$, et soit $\overline{z}$ son conjugu\'e.
Calculer
$(z+\overline{z})(z^2+\overline{z}^2)\ldots(z^n+\overline{z}^n)$
en fonction de $\rho$ et $\theta$.
}
\indication{Utiliser la formule d'Euler pour faire appara\^{\i}tre des cosinus.}
\reponse{
\'Ecrivons $z = \rho e^{i\theta}$, alors $\overline{z} = \rho
e^{-i\theta}$. Donc
\begin{align*}
P &= \prod_{k=1}^n \left(z^k+{\overline{z}}^k \right)\\
&= \prod_{k=1}^n \rho^k \left(  (e^{i\theta})^k + (e^{-i\theta})^k \right)\\
&= \prod_{k=1}^n \rho^k \left(  e^{ik\theta} + e^{-ik\theta}) \right)\\
&= \prod_{k=1}^n 2 \rho^k \cos {k\theta}\\
&= 2^n.\rho.\rho^2.\ldots.\rho^n \prod_{k=1}^n \cos {k\theta}\\
&= 2^n\rho^{\frac{n(n+1)}{2}} \prod_{k=1}^n \cos {k\theta}.\\
\end{align*}
}
}
