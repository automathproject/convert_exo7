\uuid{jV3c}
\exo7id{2834}
\auteur{burnol}
\datecreate{2009-12-15}
\isIndication{false}
\isCorrection{true}
\chapitre{Théorème des résidus}
\sousChapitre{Théorème des résidus}

\contenu{
\texte{
Déterminer la partie singulière, le résidu, et
  le terme constant des séries de Laurent à
l'origine pour les fonctions:
}
\begin{enumerate}
    \item \question{$f(z) = \frac1{\sin z}$}
\reponse{Comme $\sin(z) =z -\frac{z^3}{3!}+... =z(1+o(z))$,
$$f(-z) =\frac{1}{\sin(z)} =\frac{1}{z}(1+o(z))^{-1} =\frac{1}{z} (1+o(z)) = \frac{1}{z} +o(1) .$$
Par cons\'equent $f$ poss\`ede un p\^ole simple \`a l'origine (ce qui est \'evident puisque $\sin(z)$ poss\`ede un z\'ero simple \`a l'origine) et $\mathrm{Res} (f,0) =1$. On l'obtient aussi par la formule de l'exercice \ref{ex:burnol6.1} et le fait que l'origine est un p\^ole simple:
$$\mathrm{Res} (f,0) =\lim_{z\to 0} z f(z) = \lim_{z\to 0} \frac{z}{\sin(z)} =1.$$
La partie singuli\`ere de la s\'erie de Laurent est $\frac{1}{z}$ et le terme constant est $0$.}
    \item \question{$f(z) = \frac1{\sin z - \sh z}$}
\reponse{On a $\sin(z) -\sh(z) =(z-\frac{z^3}{3!} +\frac{z^5}{5!}+O(z^7)) -(z+\frac{z^3}{3!} +\frac{z^5}{5!}+O(z^7)) =-\frac{z^3}{3} + O(z^7)$. D'o\`u
$$f(z) =\frac{1}{\sin(z) -\sh(z)} = -\frac{3}{z^3} (1+O(z^4)) =-\frac{3}{z^3} + O(z).$$}
    \item \question{$f(z) = \frac1{z\sin(z)\sh(z)}$}
\reponse{On obtient de mani\`ere analogue que $$f(z) =\frac{1}{z \sin(z) \sh(z)} =\frac{1}{z^3} +O(z).$$}
\end{enumerate}
}
