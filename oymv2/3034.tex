\uuid{xGki}
\exo7id{3034}
\auteur{quercia}
\datecreate{2010-03-08}
\isIndication{false}
\isCorrection{false}
\chapitre{Logique, ensemble, raisonnement}
\sousChapitre{Relation d'équivalence, relation d'ordre}

\contenu{
\texte{
Soient ${\cal R}$ et ${\cal S}$ deux relations d'{\'e}quivalence sur un ensemble $E$,
telles que : $$\forall\ x,y \in E,\ x {\cal R} y  \Rightarrow  x {\cal S} y.$$

On d{\'e}finit $\dot{{\cal S}}$ sur $E/{\cal R}$ par :
$\dot x \dot{{\cal S}} \dot y \iff x {\cal S} y$.

V{\'e}rifier que $\dot{{\cal S}}$ est une relation d'{\'e}quivalence, puis d{\'e}finir une
bijection entre $(E/{\cal R})/\dot{{\cal S}}$ et $E/{\cal S}$.
}
}
