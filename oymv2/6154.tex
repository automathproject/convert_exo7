\uuid{6154}
\auteur{queffelec}
\datecreate{2011-10-16}
\isIndication{false}
\isCorrection{false}
\chapitre{Connexité}
\sousChapitre{Connexité}

\contenu{
\texte{
On va démontrer à l'aide de la connexité, le résultat classique:

``$f:{\R}\to{\R}$ continue injective $\Longrightarrow$ $f$ strictement
monoton''.

\noindent Pour cela, considérons l'application $F$ définie sur $\R^2$ par
$F(x,y)=f(x)-f(y)$ et $C=\{(x,y)\in{\R^2}\ /\ x>y\}$
}
\begin{enumerate}
    \item \question{Montrer que $F(C)$ est un connexe de $\R$.}
    \item \question{En déduire le résultat.}
\end{enumerate}
}
