\uuid{3jte}
\exo7id{929}
\auteur{legall}
\datecreate{1998-09-01}
\isIndication{true}
\isCorrection{true}
\chapitre{Application linéaire}
\sousChapitre{Définition}

\contenu{
\texte{
D\'eterminer si les applications $f_{i}$ suivantes sont lin\'eaires :
$$\begin{array}{rl}
f_1 : \Rr^2 \to \Rr^2 & f_1(x,y)=(2x+y,x-y)  \\
f_2 : \Rr^3 \to \Rr^3 & f_2(x,y,z)=(xy,x,y) \\
f_3 : \Rr^3 \to \Rr^3 & f_3(x,y,z)=(2x+y+z,y-z,x+y) \\
f_4 : \Rr^2 \to \Rr^4 & f_4(x,y)=(y,0,x-7y,x+y) \\
f_5 : \Rr_3[X] \to \Rr^3 & f_5(P) = \big( P(-1), P(0), P(1) \big) \\
\end{array}
$$
}
\indication{Une seule application n'est pas lin\'eaire.}
\reponse{
$f_1$ est linéaire. Pour $(x,y) \in \Rr^2$ et $(x',y')\in \Rr^2$ :
\begin{align*}
f_1\big( (x,y)+(x',y') \big) 
  & = f_1\big( x+x',y+y' \big) \\
  & = \big(2(x+x')+ (y+y'), (x+x')-(y+y') \big) \\
  & = \big(2x+y +2x'+y', x-y+x'-y' \big) \\
  & = \big(2x+y,x-y \big) + \big(2x'+y',x'-y' \big) \\
  & = f_1(x,y)+f_1(x',y')
\end{align*}

Pour $(x,y)\in \Rr^2$ et $\lambda\in \Rr$ :
$$f_1\big( \lambda \cdot (x,y)\big) = f_1\big(\lambda x,\lambda y\big)
 = \big( 2\lambda x+\lambda y,\lambda x - \lambda y \big) = \lambda \cdot\big(2x+y,x-y \big) 
= \lambda \cdot f_1(x,y).$$
$f_2$ n'est pas lin\'eaire, en effet par exemple $f_2(1,1,0)+f_2(1,1,0)$ n'est pas \'egal \`a $f_2(2,2,0)$.
$f_3$ est linéaire : il faut vérifier d'abord que pour tout $(x,y,z)$ et $(x',y',z')$ alors
$f_3\big( (x,y,z) + (x',y',z') \big) = f_3(x,y,z)+f_3(x',y',z')$. Et ensuite que pour tout $(x,y,z)$ et $\lambda$
on a $f_3\big(\lambda\cdot(x,y,z) \big) = \lambda \cdot f_3(x,y,z)$.
$f_4$ est linéaire : il faut vérifier d'abord que pour tout $(x,y)$ et $(x',y')$ alors
$f_4\big( (x,y) + (x',y') \big) = f_4(x,y)+f_4(x',y')$. Et ensuite que pour tout $(x,y)$ et $\lambda$
on a $f_4\big(\lambda\cdot(x,y) \big) = \lambda \cdot f_4(x,y)$.
$f_5$ est lin\'eaire : soient $P,P' \in \Rr_3[X]$ alors 

\begin{align*}
f_5\big(P+P'\big) 
  & = \big( (P+P')(-1), (P+P')(0), (P+P')(1) \big) \\
  & = \big( P(-1)+P'(-1), P(0)+P'(0), P(1)+P'(1) \big) \\
  & = \big( P(-1), P(0), P(1) \big)  + \big( P'(-1), P'(0), P'(1) \big) \\
  & = f_5(P)+f_5(P')  \\
\end{align*}

Et si $P\in \Rr_3[X]$ et $\lambda \in \Rr$ :
\begin{align*}
f_5\big( \lambda \cdot P\big) 
  & =  \big( (\lambda P)(-1), (\lambda P)(0), (\lambda P)(1) \big) \\
  & = \big( \lambda \times P(-1), \lambda \times P(0), \lambda \times P(1) \big) \\
  & = \lambda \cdot \big( P(-1), P(0), P(1) \big) \\
  & = \lambda \cdot f_5(P) \\
\end{align*}
}
}
