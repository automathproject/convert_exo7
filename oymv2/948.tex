\uuid{oKUF}
\exo7id{948}
\auteur{ridde}
\datecreate{1999-11-01}
\isIndication{false}
\isCorrection{false}
\chapitre{Application linéaire}
\sousChapitre{Image et noyau, théorème du rang}

\contenu{
\texte{
Soit $f \in \mathcal{L} (E)$ telle que $f^3 = f^2 + f$. Montrer que
$E = \ker (f) \oplus \text{Im} (f)$ (on remarquera que $f \circ (f^2-f-id) = 0$).
}
}
