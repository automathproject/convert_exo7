\uuid{672}
\auteur{vignal}
\datecreate{2001-09-01}
\isIndication{true}
\isCorrection{true}
\chapitre{Continuité, limite et étude de fonctions réelles}
\sousChapitre{Continuité : pratique}

\contenu{
\texte{
Etudier la continuit\'e de $f$
la fonction r\'eelle \`a valeurs r\'eelles
d\'efinie par
$f(x)= \frac{\sin x}{x}$ si $x\not= 0$ et $f(0)=1$.
}
\indication{Le seul probl\`eme est en $x=0$. Montrer que la fonction est bien continue en ce point.}
\reponse{
Soit $x_0 \not= 0$, alors la fonction $f$ est continue en $x_0$, car elle s'exprime sous la forme d'un quotient de fonctions continues o\`u le d\'enominateur ne s'annule pas en $x_0$.
Reste \`a \'etudier la continuit\'e en $0$.
Mais 
$$ \lim_{x\rightarrow 0} \frac{\sin x}{x} = 1 = f(0)$$
donc $f$ est continue en $0$.
}
}
