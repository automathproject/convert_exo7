\uuid{2279}
\auteur{barraud}
\datecreate{2008-04-24}

\contenu{
\texte{

}
\begin{enumerate}
    \item \question{Soient $P\in \Zz[x]$, $n\in \Nn$, 
$m=P(n)$. Montrer que $\forall\, k\in \Zz\ $ $\ m\,|\, P(n+km)$.}
\reponse{Notons $P=\sum_{i=0}^{d}a_{i}X^{i}$. Dans le calcul de $P(n+km)$, en
    développant tous les termes $(n+km)^{i}$ à l'aide du binôme, on
    obtient que $P(n+km)=\sum_{0\leq j\leq i\leq
      d}a_{i}C_{i}^{j}n^{j}(km)^{i-j} =P(n)+mN$ où $N=\sum_{0\leq j<i\leq
      d}a_{i}C_{i}^{j}n^{j}(km)^{i-j}-1\in\Zz$. Donc $m|P(n+km)$.}
    \item \question{En d\'eduire qu'il n'existe aucun polyn\^ome $P\in \Zz[x]$,
non constant, tel que, pour tout $n\in \Zz$,
$P(n)$ soit un nombre premier.}
\reponse{Supposons qu'un tel polynôme existe~: soit $m=P(0)$. $\forall
    k\in\Zz, m|P(km)$. Comme $P(km)$ est premier, on en déduit que
    $P(km)=\pm m$. Ceci est en contradiction avec 
    $\lim_{k\to+\infty}P(km)=\pm\infty$.}
\end{enumerate}
}
