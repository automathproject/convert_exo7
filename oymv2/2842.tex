\uuid{2842}
\auteur{burnol}
\datecreate{2009-12-15}
\isIndication{false}
\isCorrection{true}
\chapitre{Théorème des résidus}
\sousChapitre{Théorème des résidus}

\contenu{
\texte{
\label{ex:burnol6.4}
  Soit $0<a<b<c$ et soit $C$ le cercle de rayon $r$ centré en
l'origine, parcouru dans le sens direct. Calculer $\int_C
\frac1{(z-a)(z-b)(z-c)}dz$ selon la valeur de $r$. On
donnera deux preuves, soit en utilisant le théorème des
résidus, soit en décomposant 
en éléments simples.
}
\reponse{
La fonction $f(z) = 1/(z-a)(z-b)(z-c)$ est holomorphe dans le disque $D(0,a)$. Par cons\'equent
l'int\'egrale est nulle si $r<a$.
Par le th\'eor\`eme des r\'esidus,
\begin{eqnarray*}
\frac{1}{2i\pi}\int_C \frac{1}{(z-a)(z-b)(z-c)} dz &= &\mathrm{Res} (f,a) \quad , \quad \mathrm{Res} (f,a)+\mathrm{Res} (f,b)
\quad ou \\ && \mathrm{Res} (f,a)+\mathrm{Res} (f,b)+\mathrm{Res} (f,c)
\end{eqnarray*}
si $a<r<b$, $b<r<c$ ou $c<r$. Le calcul de ces r\'esidus se fait par la formule de l'exercice \ref{ex:burnol6.1} puisque tous
les p\^oles sont simples:
\begin{eqnarray*}\mathrm{Res} (f,a) =\frac{1}{(a-b)(a-c)} \quad , \quad \mathrm{Res} (f,b)=\frac{1}{(b-a)(b-c)} \\
et \quad \mathrm{Res} (f,c)=\frac{1}{(c-a)(c-b)}.
\end{eqnarray*}
On en d\'eduit fa\c{c}ilement la valeur de l'int\'egrale dans les trois cas.
En ce qui concerne le calcul de cette int\'egrale via la d\'ecomposition en \'el\'ements simples remarquons juste que
$$\int _C \frac{1}{z-d} \, dz$$
vaut $2i\pi$ si $d$ est \`a l'int\'erieur de $C$ et $0$ si $d$ est \`a l'ext\'erieur.
}
}
