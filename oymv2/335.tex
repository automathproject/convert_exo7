\uuid{BErY}
\exo7id{335}
\auteur{cousquer}
\datecreate{2003-10-01}
\isIndication{false}
\isCorrection{false}
\chapitre{Arithmétique dans Z}
\sousChapitre{Pgcd, ppcm, algorithme d'Euclide}

\contenu{
\texte{
Soient $a$ et $b$ deux nombres entiers relatifs.
On note $d$ leur pgcd.
Construisons les suites $a_n$ et $b_n $ $n \in \mathbb{N},$ à valeurs dans $\mathbb{Z}
$de la manière suivante :
\begin{eqnarray*}  
a_0 &=& a\\
b_0 &=& b\\
\end{eqnarray*}
et pour tout $n \in \mathbb{N},$ on pose $a_{n+1}= b_n$ et $b_{n+1}= r$ 
où $r$ est le reste de la division euclidienne de $a_n$ par $b_n.$
}
\begin{enumerate}
    \item \question{Montrer que si $d_n$ est le pgcd de $a_n$ et $b_n$ alors $d_n$ est également le pgcd de $a_{n+1}$ et $b_{n+1}.$}
    \item \question{Déduire de la questionh précédente que $d$ est le pgcd des nombres $a_n$ et $b_n$ pour tout $n \in \mathbb{N}.$}
    \item \question{Montrer que la suite $b_n$ est strictement décroissante.
Que peut-on en déduire?}
    \item \question{Déduire de ce qui précède que pour tout couple d'entiers relatifs $(a,b)$ 
il existe un couple d'entier relatifs $(u,v)$ tel que:
$$
d = au+bv.
$$}
\end{enumerate}
}
