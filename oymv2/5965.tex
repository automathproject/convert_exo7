\uuid{5965}
\auteur{tumpach}
\datecreate{2010-11-11}
\isIndication{false}
\isCorrection{true}
\chapitre{Espace L^p}
\sousChapitre{Espace Lp}

\contenu{
\texte{
Soit $\Omega = \mathbb{N}$ muni de la mesure de comptage. Pour
tout $1\leq p <+\infty$, on note $\ell^p$ l'espace des suites
complexes $(u_{n})_{n\in\mathbb{N}}$ telles que $\|u\|_{p} :=
\left(\sum_{i=0}^{+\infty}
|u_{n}|^{p}\right)^{\frac{1}{p}}<+\infty$. L'espace des suites
born\'ees sera not\'e $\ell^{\infty}.$
}
\begin{enumerate}
    \item \question{Montrer que si $q\leq p$, alors $\ell^{q}\subset \ell^{p}$.
En particulier, pour $1<q<2<p$, on a~:
$$
\ell^{1} \subset \ell^{q} \subset \ell^{2} \subset \ell^{p}
\subset \ell^{\infty}.
$$}
\reponse{Montrons que si $q\leq p$, alors $\ell^{q}\subset \ell^{p}$.
Soit $(u_{n})_{n\in\mathbb{N}}\in \ell^{q}$. Comme
$$
\sum_{i=0}^{+\infty} |u_{n}|^{q} <+\infty,
$$
il existe un rang $N$ tel que pour $n> N$, $|u_{n}|^{q}< 1$. En
particulier la suite $(u_{n})_{n\in\mathbb{N}}$ appartient \`a
$\ell^{\infty}$
 et
$$\|u\|_{\infty} \leq \max\{u_{0}, \dots, u_{N}, 1\}.$$
De plus, pour $n> N$, on a $|u_{n}|^{p} \leq |u_{n}|^{q}$ et
$$
\sum_{i=N+1}^{+\infty} |u_{n}|^{p} \leq \sum_{i=N+1}^{+\infty}
|u_{n}|^{q}\leq \|u\|_{q}^{q} <+\infty,
$$
ce qui implique que $\|u\|_{p} < +\infty$. En conclusion, pour
$1<q<2<p$, on a~:
$$
\ell^{1} \subset \ell^{q} \subset \ell^{2} \subset \ell^{p}
\subset \ell^{\infty}.
$$}
    \item \question{En consid\'erant les suites $u^{(\alpha)}_{n} =
n^{-\alpha}$, montrer que pour $q<p$, l'inclusion $\ell^{q}
\subset \ell^{p}$ est stricte.}
\reponse{La suite $u^{(\alpha)}_{n} = n^{-\alpha}$ appartient \`a
$\ell^{\infty}$ pour tout $\alpha\geq 0$ et \`a $\ell^{p}$ avec
$1\leq p <+\infty$ si et seulement si $\alpha p>1$, i.e $\alpha>
\frac{1}{p}$. En particulier la suite constante \'egale \`a $1$
appartient \`a $\ell^{\infty}$ mais n'appartient \`a aucun
$\ell^{p}$ pour $p<+\infty$. Soit $1<q<p<+\infty$. Pour tout
$\alpha$ tel que $\frac{1}{p} < \alpha < \frac{1}{q}$, la suite
$u^{(\alpha)}$ appartient \`a $\ell^{p}\setminus \ell^q$. C'est le
cas en particulier pour $\alpha = \frac{1}{2}\left(\frac{1}{p} +
\frac{1}{q} \right)$. Ainsi l'inclusion $\ell^{q} \subset
\ell^{p}$ est stricte lorsque $q< p$.}
\end{enumerate}
}
