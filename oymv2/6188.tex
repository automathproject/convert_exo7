\uuid{6188}
\auteur{queffelec}
\datecreate{2011-10-16}
\isIndication{false}
\isCorrection{false}
\chapitre{Compacité}
\sousChapitre{Compacité}

\contenu{
\texte{
Soit $E$ un espace vectoriel normé sur $\Cc$ de boule unité fermée
$\overline B$ et
$F$ un sous-espace vectoriel fermé de $E$. On a montré dans le liste précédente
que si
$F\not=E$,\quad $\sup_{x\in\overline B} d(x,F)=1.$

On va montrer qu'un evn dont la boule unité fermée est compacte est
nécessaire\-ment de dimension finie.
On suppose donc que $\overline B$ est compacte.
}
\begin{enumerate}
    \item \question{Montrer que pour tout $\varepsilon >0$ on peut trouver un nombre fini de
points $x_1,\cdots,x_k\in\overline B$ tels que $\overline B\subset\cup_{j=1}^k
B(x_j,\varepsilon)$.}
    \item \question{Montrer que $E$ est de dimension finie : pour cela, considérer le
sous-espace vectoriel engendré par $x_1,\cdots,x_k$.}
\end{enumerate}
}
