\uuid{6685}
\auteur{queffelec}
\datecreate{2011-10-16}

\contenu{
\texte{

}
\begin{enumerate}
    \item \question{Soit $C\subset{\Cc}$ une courbe orientée $C^1$ et fermée. Soit
$\gamma $ un chemin $C^1$ d'origine $a$ et d'extrémité $b$, tels que $a$ et $b$ ne
soient pas des points de $C$. On suppose que l'intersection de $C$ et
$\gamma $ est constituée d'un nombre fini de points $m_1,\dots,m_n$
et que les tangentes à $C$ et à $\gamma $ sont distinctes en ces points.
Soit $\varepsilon _i=1$ si l'angle de la tangente à $\gamma $ avec la
tangente à $C$ en $m_i$ est entre 0 et $\pi$, $\varepsilon _i=-1$
sinon. Montrer que
$$\sum_i \varepsilon _i=\mathrm{Ind}_C(a)-\mathrm{Ind}_C(b)$$
où $\mathrm{Ind}_C(z)$ désigne l'indice de $z$ par rapport à $C$.}
    \item \question{Calculer l'indice du point $z={3\over 4}$ par rapport à la courbe dont l'équation en
coordonnées polaires est $r=\cos {\theta \over 3}$ avec $0\le \theta 
\le 3\pi$, parcourue dans le sens des $\theta $ croissants.}
\end{enumerate}
}
