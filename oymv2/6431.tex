\uuid{6431}
\auteur{potyag}
\datecreate{2011-10-16}
\isIndication{false}
\isCorrection{false}
\chapitre{Géométrie et trigonométrie hyperbolique}
\sousChapitre{Géométrie et trigonométrie hyperbolique}

\contenu{
\texte{
{\it On notera $\mathbb{H}^2$  le plan de Poincaré dans
l'un de deux modèles du disque ou du demi-plan, muni de la
distance hyperbolique $\rho$.}
}
\begin{enumerate}
    \item \question{Montrer que $\mathbb{H}^2$ est un espace métrique complet
  mais pas compact.}
    \item \question{Dans le modèle du demi-plan on suppose que  $z, w$
  sont deux points distincts dans $\mathbb{H}^2$, montrer que
  $\rho(z,w)=\vert\ln([z^*, z, w, w^*])\vert$, où
$[z^*, z, w, w^*]$ désigne le birapport de quatre points, où
$z^*, w^*$ sont les extrémités de la géodésique passant
par $z$ et $w$ dans l'ordre indiqué sur le Figure \ref{fig:pot1} :}
\end{enumerate}
}
