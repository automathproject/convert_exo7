\uuid{6723}
\auteur{queffelec}
\datecreate{2011-10-16}
\isIndication{false}
\isCorrection{false}
\chapitre{Singularité}
\sousChapitre{Singularité}

\contenu{
\texte{
Développer les fonctions suivantes en série de Laurent dans
chacun des ouverts donnés
}
\begin{enumerate}
    \item \question{$\displaystyle f(z)={1\over (z-1)(z-2)}$ dans $\vert z\vert <1$ ; 
$1<\vert z\vert <2$ ; $2<\vert z\vert $ ;}
    \item \question{$\displaystyle f(z)={1\over (z-a)^k}$ ($k\in {\Nn}^*$) dans $\vert
z\vert <\vert a\vert$ et dans $\vert z\vert >\vert a\vert$ ;}
    \item \question{$\displaystyle f(z)={1\over z(z-a)}$ dans $0<\vert
z\vert <\vert a\vert$ et dans $\vert a\vert <\vert z\vert$ ;}
    \item \question{$\displaystyle f(z)={1\over (z-a)(z-b)}$ ($0<\vert a\vert <\vert
b\vert$) dans $0<\vert z\vert <\vert a\vert$ ; $\vert a\vert<\vert z\vert
<\vert b\vert$ ; $\vert b\vert <\vert z\vert$ ;}
    \item \question{une détermination holomorphe $f$ de $[(z-a)(z-b)]^{1\over 2}$
($0<\vert a\vert =\vert b\vert$) dans $0<\vert z\vert <\vert a\vert$ ;
$\vert b\vert <\vert z\vert$ ;}
    \item \question{$f(z)=z^2\exp{(z^{-1})}$ dans $0<\vert z\vert$.}
    \item \question{$f(z)=\exp{(z+z^{-1})}$ dans $0<\vert z\vert$.}
    \item \question{$f(z)=\sin z\cdot\sin{\left(z^{-1}\right)}$ dans $0<\vert z\vert$.}
    \item \question{$f(z)=\mathrm{cotan} z$ dans $k\pi<\vert z\vert<(k+1)\pi$ ($k\in {\Nn}$) on
pourra exprimer le résultat en fonction des nombres $B_n$ de Bernoulli,
définis par : $${z\over \exp{(z)}-1}=\sum_{n\ge 0}B_n{z^n\over n!}$$
($B_0=1$,
$B_1=-1/2$, et $B_{2n+1}=0$ pour $n\ge 1$).}
\end{enumerate}
}
