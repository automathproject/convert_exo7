\uuid{2305}
\auteur{barraud}
\datecreate{2008-04-24}
\isIndication{false}
\isCorrection{true}
\chapitre{Anneau, corps}
\sousChapitre{Anneau, corps}

\contenu{
\texte{
Soit $C=A\times B$ le produit direct de deux anneaux.
D\'ecrire les ensembles des \'el\'ements inversibles, des diviseurs
de z\'ero et des \'el\'ements nilpotents de l'anneau $C$.
}
\reponse{
$C=A\times B$.
\begin{align*}
(a,b)\in (A\times B)^{\times}
\Leftrightarrow&
\exists(c,d)\in A\times B, \ (a,b)(c,d)=(1,1)\\
\Leftrightarrow&
\exists(c,d)\in A\times B, \ ac=1 \text{ et } bd=1\\
\Leftrightarrow&
a\in A^{\times} \text{ et } b\in B^{\times}\\
\end{align*}
 donc $(A\times B)^{\times}=A^{\times}\times B^{\times}$.

De même, on obtient que l'ensemble $\mathcal{D}_{A\times B}$ des
diviseurs de $0$ de $A\times B$ est
$$
\mathcal{D}_{A\times
  B}=\mathcal{D}_{A}\times B\cup A\times\mathcal{D}_{B}\cup
(A\setminus\{0\})\times\{0\}\cup\{0\}\times(B\setminus\{0\}).
$$
Enfin, pour les nilpotents $Nil(A\times B)=Nil(A)\times Nil(B)$.
}
}
