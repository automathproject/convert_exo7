\uuid{3377}
\auteur{quercia}
\datecreate{2010-03-09}
\isIndication{false}
\isCorrection{false}
\chapitre{Matrice}
\sousChapitre{Autre}

\contenu{
\texte{
Une matrice carrée $M$ est dite {\it magique} si les sommes des coefficients
de $M$ par ligne et par colonne sont constantes.
On note $s(M)$ leur valeur commune.


Soit $U = \begin{pmatrix} 1     &\dots&1     \cr
                     \vdots&     &\vdots\cr
                     1     &\dots&1     \cr\end{pmatrix}$
et ${\cal M} = \{ \text{matrices } n\times n \text{ magiques}\}$.
}
\begin{enumerate}
    \item \question{Montrer que $\cal M$ est une sous-algèbre de $\mathcal{M}_n(K)$ et
    $s : {\cal M} \to  K$ est un morphisme d'algèbre (calculer $MU$ et $UM$).}
    \item \question{Si $M$ est magique inversible, montrer que $M^{-1}$ est aussi magique.}
    \item \question{Montrer que $\cal M$ est la somme directe du sev des matrices magiques
    symétriques et du sev des matrices magiques antisymétriques.}
    \item \question{Pour $M \in \mathcal{M}_n(K)$, on note $\phi_M$ l'endomorphisme de $ K^n$
    canoniquement associé à $M$.

    Soit ${\cal H} = \{ (x_1, \dots ,x_n)  \in  K^n \text{ tq }
                        x_1 + \dots + x_n = 0 \}$
    et   ${\cal K} = \{ ( x, \dots, x) \in  K^n \}$.
  \begin{enumerate}}
    \item \question{Montrer que : $M \in {\cal M} \iff {\cal H} \text{ et } {\cal K}$
        sont stables par $\phi_M$.}
    \item \question{En déduire dim($\cal M$).}
\end{enumerate}
}
