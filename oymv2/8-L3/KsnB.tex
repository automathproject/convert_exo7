\uuid{KsnB}
\exo7id{2162}
\auteur{debes}
\datecreate{2008-02-12}
\isIndication{false}
\isCorrection{true}
\chapitre{Sous-groupe distingué}
\sousChapitre{Sous-groupe distingué}

\contenu{
\texte{
Soit $p$ un nombre premier. Montrer qu'un groupe ab\'elien fini, dont
tous les \'el\'ements diff\'erents de l'\'el\'ement neutre sont d'ordre $p$, est
isomorphe \`a $(\Z /p
\Z)^n $.
}
\reponse{
Soit $G$ un groupe ab\'elien fini tel que $pG=\{0\}$. Pour tout entier $n\in \Z$ et pour tout
$g\in G$, l'\'el\'ement $ng$ ne d\'epend que de la classe de $n$ modulo $p$; on peut le
noter $\overline n \cdot g$. La correspondance $(\overline n, g)\rightarrow \overline n \cdot
g$ d\'efinit une loi externe sur le groupe additif $(\Z/p\Z)^n$ et lui conf\`ere
ainsi une structure de $\mathbb{F}_p$-espace vectoriel. Cet espace vectoriel, \'etant fini, est de
dimension finie. Il est donc isomorphe comme espace vectoriel, et en particulier comme groupe
\`a $(\Z/p\Z)^n$ pour un certain entier $n\geq 0$.
}
}
