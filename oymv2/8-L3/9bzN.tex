\uuid{9bzN}
\exo7id{2182}
\auteur{debes}
\datecreate{2008-02-12}
\isIndication{false}
\isCorrection{true}
\chapitre{Action de groupe}
\sousChapitre{Action de groupe}

\contenu{
\texte{
Soit $G$ un groupe fini agissant sur un ensemble fini $X$.
\smallskip

(a) On suppose que toute orbite contient au moins deux \'el\'ements, que
$|G|=15$ et que $\hbox{\rm card}(X)=17$. D\'eterminer le nombre d'orbites et le
cardinal de chacune.
\smallskip

(b) On suppose que $|G|=33$ et $\hbox{\rm card}(X)=19$. Montrer qu'il existe au
moins une orbite r\'eduite \`a un \'el\'ement.
}
\reponse{
Toute orbite ${\cal O}={\cal O}_x$ d'un \'el\'ement $x\in X$ est en bijection avec
l'ensemble $G/\cdot G(x)$ des classes \`a gauche de $G$ modulo le fixateur $G(x)$ de $G$. En
particulier, le cardinal de ${\cal O}$ divise l'ordre de $G$. De plus la somme des longueurs
des orbites est \'egale au cardinal de l'ensemble $X$.
\smallskip 

(a) Si $|G|=15$, $\hbox{\rm card}(X)=17$ et s'il n'y a pas d'orbite \`a un
seul \'el\'ement, il n'y a qu'une seule possibilit\'e: 4 orbites de longueur $3$ et une de
longueur
$5$.
\smallskip

(b) Supposons $|G|=33$ et $\hbox{\rm card}(X)=19$. Aucune somme de diviseurs $\not=1$ de $33$
n'est \'egale \`a $19$ donc n\'ecessairement il existe au moins une orbite r\'eduite \`a un
\'el\'ement.
\smallskip
}
}
