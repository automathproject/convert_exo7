\uuid{ifJY}
\exo7id{7225}
\auteur{megy}
\datecreate{2021-03-01}
\isIndication{false}
\isCorrection{false}
\chapitre{Fonction holomorphe}
\sousChapitre{Fonction holomorphe}

\contenu{
\texte{
Soit \(\displaystyle{f(z)=\sum_{n=0}^{+\infty}}a_nz^n\) une série entière de rayon de convergence \(R>0\). Un point \(z_0\in \partial B(0,R)\) est un \emph{point régulier} de \(f\) si il existe une  extension analytique de \(f\) dans un voisinage de \(B(0,R)\cup\{z_0\}\). Si \(z_0\in \partial B(0,R)\) n'est pas régulier pour \(f\), on dit que c'est un \emph{point singulier} de \(f\). On note \({\rm Sing}(f)\subset \partial B(0,R)\) l'ensemble des points singuliers de \(f\).
}
\begin{enumerate}
    \item \question{Déterminer les points réguliers et singuliers pour \(\displaystyle{f(z)=\sum_{n=0}^{+\infty}z^n}\) et \(\displaystyle{f(z)=\sum_{n=0}^{+\infty}(-1)^n\frac{z^n}{n}}\).}
    \item \question{Quel est le lien entre la régularité d'un point \(z_0\in \partial B(0,R)\) et la convergence de \(\displaystyle{\sum_{n=0}^{+\infty}a_nz_0^n}\) ?}
    \item \question{Montrer que \({\rm Sing}(f)\) est fermé.}
    \item \question{On considère \(\displaystyle{f(z)=\sum_{n=0}^{+\infty}z^{2^n}}\).
\begin{enumerate}}
    \item \question{Montrer que le rayon de convergence de \(f\) est \(1\).}
    \item \question{Montrer que \(1\) est un point singulier de \(f\). (Indication, montrer que \(\lim_{t\to 1^-}\sum_{n=0}^{+\infty}t^{2^n}=+\infty\)).}
    \item \question{Montrer que pour tout \(m\in \N\), toute racine \(2^m\)-ième de \(1\) est un point singulier de \(f\). (Indication, observer que pour tout \(m\in \N\) et pour tout \(z\in B(0,1)\),  \(f\big(z^{2^{m+1}}\big)=f\big(z^{2^m}\big)-z^{2^m}\)).}
    \item \question{En déduire que \({\rm Sing}(f)=\partial B(0,1)\).}
\end{enumerate}
}
