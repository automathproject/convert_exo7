\uuid{KIHs}
\exo7id{2696}
\auteur{matexo1}
\datecreate{2002-02-01}
\isIndication{false}
\isCorrection{false}
\chapitre{Courbes planes}
\sousChapitre{Autre}

\contenu{
\texte{
On dispose d'un oscilloscope {\`a} deux voies. On applique sur la voie X une
tension sinuso\"{\i}dale de pulsation $\omega$, et sur la voie Y une tension de
m{\^e}me amplitude et de pulsation $2\omega$. En pla\c{c}ant l'oscilloscope en mode
X--Y et pour un choix appropri{\'e} du gain de chaque voie, on observe sur
l'{\'e}cran une courbe param{\'e}tr{\'e}e d{\'e}finie en coordonn{\'e}es cart{\'e}siennes par
les {\'e}quations: \[\hspace*{-2cm}\left\{\begin{array}{l} x(t)=a\sin \omega t \\
y(t)=a\sin 2\omega t \end{array} \right.\]
\begin{itemize}
\item D{\'e}terminer la p{\'e}riode du mouvement T. 
\item Donner le tableau des
variations de $x(t)$ et $y(t)$ sur l'intervalle [0,T], et en d{\'e}duire l'allure
de la courbe. 
\item D{\'e}terminer les coordonn{\'e}es des points de la courbe
d'abscisse ou d'ordonn{\'e}e maximum. 
\item D{\'e}terminer les sym{\'e}tries de la
courbe et donner les transformations correspondantes du param{\`e}tre $t$.
\end{itemize}
}
}
