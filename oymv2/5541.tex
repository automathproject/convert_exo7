\uuid{Z257}
\exo7id{5541}
\auteur{rouget}
\datecreate{2010-07-15}
\isIndication{false}
\isCorrection{true}
\chapitre{Conique}
\sousChapitre{Conique}

\contenu{
\texte{
Le plan est rapporté à un repère orthonormé $\mathcal{R}=\left(0,\overrightarrow{i},\overrightarrow{j}\right)$.
Eléménts caractéristiques de la courbe dont une équation dans $\mathcal{R}$ est

\begin{itemize}
\item
}
\begin{enumerate}
    \item \question{$y=x^2+x+1$,}
    \item \question{$y^2+y-2x=0$,}
    \item \question{$y=\sqrt{2x+3}$.}
\reponse{
\begin{enumerate}
$y=x^2+x+1\Leftrightarrow y=\left(x+\frac{1}{2}\right)^2+\frac{3}{4}\Leftrightarrow\left(y-\frac{3}{4}\right)=\left(x+\frac{1}{2}\right)^2$.
$\mathcal{C}$ est la parabole de sommet $S\left(-\frac{1}{2},\frac{3}{4}\right)$, d'axe focal la droite d'équation
$x=-\frac{1}{2}$, de paramètre $p=\frac{1}{2}$ et donc de foyer
$F\left(-\frac{1}{2},\frac{3}{4}+\frac{1}{4}\right)=\left(-\frac{1}{2},1\right)$ et de directrice d'équation
$y=\frac{3}{4}-\frac{1}{4}=\frac{1}{2}$.
$y^2+y-2x=0\Leftrightarrow\left(y+\frac{1}{2}\right)^2-\frac{1}{4}-2x=0\Leftrightarrow\left(y+\frac{1}{2}\right)^2=2\left(x+\frac{1}{8}\right)$.
$\mathcal{C}$ est la parabole de sommet $S\left(-\frac{1}{8},-\frac{1}{2}\right)$, d'axe focal la droite d'équation
$y=-\frac{1}{2}$, de paramètre $p=1$ et donc de foyer
$F\left(-\frac{1}{8}+\frac{1}{2},-\frac{1}{2}\right)=\left(\frac{3}{8},-\frac{1}{2}\right)$ et de directrice d'équation
$x=-\frac{1}{8}-\frac{1}{2}=-\frac{5}{8}$.
$y=\sqrt{2x+3}\Leftrightarrow y^2=2\left(x+\frac{3}{2}\right)\;\mbox{et}\;y\geq0$. $\mathcal{C}$ est une demi-parabole de sommet
$S\left(-\frac{3}{2},0\right)$, d'axe focal $(Ox)$, de paramètre $p=1$ et donc de foyer $F\left(-\frac{3}{2}+\frac{1}{2},0\right)=(-1,0)$ et
de directrice d'équation $x=-\frac{3}{2}-\frac{1}{2}=-2$.
}
\end{enumerate}
}
