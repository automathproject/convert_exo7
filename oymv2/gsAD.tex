\uuid{gsAD}
\exo7id{3537}
\auteur{quercia}
\datecreate{2010-03-10}
\isIndication{false}
\isCorrection{false}
\chapitre{Réduction d'endomorphisme, polynôme annulateur}
\sousChapitre{Polynôme caractéristique, théorème de Cayley-Hamilton}

\contenu{
\texte{
Soit $A \in \mathcal{M}_n(\C)$.
}
\begin{enumerate}
    \item \question{Soit $\lambda \in \C$ une valeur propre non nulle de $A\overline A$, et $X$ un
    vecteur propre associé. Montrer que $A\overline X$ est aussi vecteur propre de
    $A\overline A$.}
    \item \question{Lorsque $\lambda \notin \R^+$, montrer que $X$ et $A\overline X$ sont
    linéairement indépendants.}
    \item \question{En déduire que $\det(I + A\overline A ) \in \R^+$.}
\end{enumerate}
}
