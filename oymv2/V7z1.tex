\uuid{V7z1}
\exo7id{3362}
\auteur{quercia}
\datecreate{2010-03-09}
\isIndication{false}
\isCorrection{true}
\chapitre{Matrice}
\sousChapitre{Inverse, méthode de Gauss}

\contenu{
\texte{
On note
$U = \begin{pmatrix} 1   &\dots &1   \cr
               \vdots &&\vdots \cr
               1   &\dots &1   \cr \end{pmatrix} \in \mathcal{M}_n(\R)$
     et ${\cal A} = \{ aU + bI,\ a,b \in \R \}\quad (n \ge 2)$.
}
\begin{enumerate}
    \item \question{Montrer que ${\cal A}$ est une sous algèbre commutative de $\mathcal{M}_n(\R)$.}
    \item \question{Soit $M = aU+bI \in {\cal A}$. Montrer que $M$ possède un inverse dans
    ${\cal A}$ si et seulement si
    $b(b+na) \ne 0$, et le cas échéant, donner $M^{-1}$.}
    \item \question{Montrer que si $b(b+na) = 0$, alors $M$ n'est pas inversible dans $\mathcal{M}_n(\R)$.}
    \item \question{Trouver les matrices $M \in {\cal A}$ vérifiant : $M^n = I$.}
\reponse{
$M^{-1} = \frac {-a}{b(na+b)}U + \frac 1b I$.
$$M^n = \frac {(na+b)^n-b^n}nU + b^nI  \Rightarrow 
             \left\{
             \begin{array} {ll}
             n \text{ pair }  &: a = 0, \text{ ou }
                                       -\frac{\smash{2b}}n,\ b = \pm1 \cr
                    n \text{ impair} &: a = 0,\ b = 1. \cr
                    \end{array}\right.$$
}
\end{enumerate}
}
