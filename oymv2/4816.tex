\uuid{jb9M}
\exo7id{4816}
\auteur{quercia}
\datecreate{2010-03-16}
\isIndication{false}
\isCorrection{true}
\chapitre{Topologie}
\sousChapitre{Topologie des espaces vectoriels normés}

\contenu{
\texte{
On munit $E_k = \R_k[X]$ de la norme $\|P\|_k = \sum_{i=0}^k |P(i)|$.
Calculer $\|\hskip-1pt|\varphi\|\hskip-1pt|$ avec
$\varphi : {E_2} \to {E_3}, P \mapsto {X^2P'.}$
}
\reponse{
Les formes lin{\'e}aires $P \mapsto P(0)$, $P \mapsto P(1)$ et $P \mapsto P(2)$
constituent une base de $E_2^*$ donc engendrent les formes lin{\'e}aires
$P \mapsto P'(1)$, $P \mapsto P'(2)$ et $P \mapsto P'(3)$. Apr{\`e}s calculs, on trouve~:
$$\forall\ P\in E_2,\quad
\left\{\begin{aligned}2P'(1) &= &P(2)  & &      &-&P(0)\cr
      2P'(2) &= &3P(2) &-&4P(1) &+&P(0)\cr
      2P'(3) &= &5P(2) &-&8P(1) &+&3P(0).\cr
\end{aligned}\right.$$
En notant $P(0) = a$, $P(1) = b$ et $P(2) = c$ on doit
donc chercher~:
$$\|\hskip-1pt|\varphi\|\hskip-1pt|
=\frac12\sup\{|c-a| + 4|3c-4b+a| + 9|5c-8b+3a|,\text{ tq }|a|+|b|+|c|\le1\}.$$
La fonction $f$~: $(a,b,c) \mapsto|c-a| + 4|3c-4b+a| + 9|5c-8b+3a|$ est convexe donc
son maximum sur l'{\it icosa{\`e}dre\/} $I = \{(a,b,c)\text{ tq }|a|+|b|+|c|\le 1\}$
est atteint en l'un des sommets $(\pm1,0,0)$, $(0,\pm1,0)$, $(0,0,\pm1)$.

Finalement,~$\|\hskip-1pt|\varphi\|\hskip-1pt| = \frac12f(0,1,0) = 44$.
}
}
