\uuid{217}
\auteur{ridde}
\datecreate{1999-11-01}

\contenu{
\texte{
Soit $ (E, \leq)$ un ensemble ordonn\'e. On d\'efinit sur
$\mathcal{P} (E)\setminus\left\{ \emptyset \right\}$ la relation
$\prec$ par 
$$X \prec Y \quad \text{ ssi } \quad (X = Y \ \text{ ou } \ 
\forall x \in X \  \forall y \in Y \  x \leq y).$$ 
V\'erifier que c'est une relation d'ordre.
}
\reponse{
\begin{itemize}
 \item Reflexivit\'e : pour tout $X\in\mathcal{P} (E)$ on a $X \prec X$ car $X=X$.
 \item Anti-sym\'etrie : pour $X,Y\in\mathcal{P} (E)$ tels que $X \prec Y$ et $Y \prec X$, alors par d\'efinition de $\prec$ on a 
$$\forall x \in X \quad \forall y \in Y \quad x\leqslant y \text{ et } y \leqslant x.$$
Comme la relation $\le$ est une relation d'ordre alors $x\leqslant y$ et $y \leqslant x$ implique $x=y$.
Donc 
$$\forall x \in X \quad \forall y \in Y \quad x = y,$$
ce qui implique que $X=Y$ (dans ce cas en fait $X$ est vide ou un singleton).

 \item Transitivit\'e : soit $X,Y, Z \in\mathcal{P} (E)$ tels que  $X \prec Y$ et $Y \prec Z$.
Si $X=Y$ ou $Y=Z$ alors il est clair que $X \prec Z$. Supposons que $X\neq Y$ et $Y\neq Z$
alors 
$$\forall x \in X \quad \forall y \in Y \quad x\leqslant y \qquad \text{ et } \qquad \forall y \in Y \quad \forall z \in Z \quad y\leqslant z.$$

Donc on a 
$$\forall x \in X \quad \forall y \in Y \quad \forall z \in Z \quad x \leqslant y \text{ et }  y \leqslant z,$$ 
alors par transitivit\'e de  la relation $\le$ on obtient :
$$\forall x \in X \quad \forall z \in Z \quad x \leqslant z.$$ 
Donc $X \prec Z$.
 \end{itemize}
}
}
