\uuid{2480}
\auteur{matexo1}
\datecreate{2002-02-01}
\isIndication{false}
\isCorrection{false}
\chapitre{Analyse vectorielle}
\sousChapitre{Forme différentielle, champ de vecteurs, circulation}

\contenu{
\texte{
On rappelle que la formule de Stokes g\'en\'erale affirme que si $\omega
$ est une forme diff\'erentielle de degr\'e $p-1$, $\Omega$ une vari\'et\'e de
$\R^N$ de dimension $p$ et de bord  $\partial  \Omega$, alors
$$\int_\Omega d\omega  = \int_{\partial  \Omega} \omega.$$ 
Dans le cas o\`u $p=1$, $\Omega=[a,b]$ un segnment et $\omega  = f$ une
fonction r\'eelle, que donne cette formule\,? Et si $\Omega$ est la r\'eunion
de plusieurs intervalles\,? Plus g\'en\'eralement, si $\Omega$ est une courbe
de $\R^3$, et $g$ une fonction d\'efinie sur $\R^3$, quel est le travail de $g$
le long de $\Omega$\,? Montrer qu'il ne d\'epend pas du chemin parcouru.
}
}
