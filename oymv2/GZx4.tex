\uuid{GZx4}
\exo7id{5967}
\auteur{tumpach}
\datecreate{2010-11-11}
\isIndication{false}
\isCorrection{true}
\chapitre{Espace L^p}
\sousChapitre{Espace Lp}

\contenu{
\texte{
Soit $\{f_{n}\}_{n\in\mathbb{N}}$ la suite de fonctions d\'efinies
par~:
$$
f_{n}(x) = \frac{1}{\sqrt{n}} \mathbf{1}_{[n, 2n]}(x).
$$
}
\begin{enumerate}
    \item \question{Montrer que $f_{n}$ converge faiblement vers $0$ dans
$L^{2}([0, +\infty[)$ mais ne converge pas fortement dans $L^2([0,
+\infty[)$.}
\reponse{Quelque soit $g$ continue \`a support compact,
$$
\int_{[0, +\infty[} f_{n}(x) g(x)\,dx = \frac{1}{\sqrt{n}}
\int_{n}^{2n} g(x)\,dx \rightarrow 0
$$
quand $n\rightarrow +\infty$. Par densit\'e des fonctions
continues \`a support compact, $f_{n}$ converge faiblement vers
$0$. D'autre part, $f_n$ converge presque partout vers $0$.
Supposons que $f_n$ converge fortement vers une fonction $f$ dans
$L^2([0, +\infty[)$. Alors il existe une sous-suite de $f_n$ qui
converge presque-partout vers $f$, ce qui implique que $f =0$ est
la seule limite possible. Or~:
$$
\|f_{n}\|_{2} = 1
$$
pour tout $n$, donc $\|f_{n}\|_2$ ne tend pas vers $\|f\|_2 = 0$
ce qui contredit le fait que $f_n$ converge vers $f$ dans $L^2([0,
+\infty[)$.}
    \item \question{Montrer que $f_{n}$ converge fortement vers $0$ dans
$L^{p}([0, +\infty[)$ pour $p>2$.}
\reponse{Pour $p>2$, on a~:
$$
\int_{[0, +\infty[} |f_{n}(x)|^p\,dx =  \int_{n}^{2n}
n^{-\frac{p}{2}}\,dx = n^{1 - \frac{p}{2}} \rightarrow 0,
$$
quand $n\rightarrow+\infty$ donc $f_{n}$ converge fortement vers
$0$ dans $L^{p}([0, +\infty[)$.}
\end{enumerate}
}
