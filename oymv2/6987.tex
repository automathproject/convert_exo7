\uuid{6987}
\auteur{blanc-centi}
\datecreate{2015-07-04}
\isIndication{false}
\isCorrection{true}
\chapitre{Courbes planes}
\sousChapitre{Courbes paramétrées}

\contenu{
\texte{
On considère la courbe paramétrée définie par 
$$\left\{\begin{array}{l}x(t)=t+\frac{4}{t}\\ \ \\ y(t)=\frac{t}{3}+2+\frac{3}{t+1}\end{array}\right.$$
}
\begin{enumerate}
    \item \question{Dresser le tableau de variations conjointes de $x$ et $y$.}
    \item \question{Calculer les tangentes horizontales, verticales et les asymptotes.}
    \item \question{Trouver le point singulier de la courbe, étudier son type et 
écrire l'équation de la tangente à la courbe en ces points.}
    \item \question{Tracer la courbe.}
\reponse{
Les fonctions $x$ et $y$ sont de classe $\mathcal{C}^1$ sur $\mathcal{D}$. Soit $t\in\mathcal{D}$:
$$\begin{array}{lcl}
x(t)=t+\frac{4}{t}&\ &y(t)=\frac{t}{3}+2+\frac{3}{t+1}\\
x'(t)=1-\frac{4}{t^2}&\ &y'(t)=\frac{1}{3}-\frac{3}{(t+1)^2}\\
x'(t)>0\Longleftrightarrow |t|>2&\ &y'(t)>0\Longleftrightarrow |t+1|>3\Longleftrightarrow \left\{\begin{array}{l}t>2\\ \text{ou}\\ t<-4\end{array}\right.
\\
x'(t)=0\Longleftrightarrow t\in\{-2;2\} &\ &y'(t)=0\Longleftrightarrow t\in\{-4;2\}
\end{array}$$
$$\begin{array}{c|lcccccccccccr}
t&-\infty&\ &-4& &-2& &-1& &0& &2& &+\infty\\\hline
x'(t)& &+& &+&0&-&&-&&-&0&+&\\\hline
\ &  &\ &&&-4&&&&+\infty&&&&+\infty\\
\ &  &\ &&\nearrow&&\searrow&&&&&&&\\
x&\ & &-5&&&&-5&&&\searrow&&\nearrow&\ \\
\ &  &\nearrow &&&&&&\searrow&&&&&\\
\ &-\infty &\ & &&&&&&-\infty&&4&&\\\hline
\ & &\ & -\frac{1}{3}&&&&+\infty&&&&&&+\infty\\
\ & &\ & &\searrow&&&&\searrow&&&&&\\
y&\ &\nearrow & &&-\frac{5}{3}&&&&5&&&\nearrow&\\
\ & &\ & &&&\searrow&&&&\searrow&&&\\
\ &-\infty & & &&&&-\infty&&&&\frac{11}{3}&&\\\hline
y'(t)&& + &0 &-&&-&&-&&-&0&+&\\
\end{array}$$
Le tableau de variations conjointes indique:
\begin{itemize}
$t=-4$: tangente horizontale, au point de coordonnées $(-5,-\frac{1}{3})$;
$t=-2$: tangente verticale, au point de coordonnées $(-4,-\frac{5}{3})$;
$t=-1$: une asymptote verticale, d'équation $x=-5$;
$t=0$: une asymptote horizontale, d'équation $y=5$;
$t=2$: il y a un point singulier en $(4,\frac{11}{3})$ (voir après).
\end{itemize}
Il reste à étudier le comportement quand $t\to\pm\infty$:
$$\frac{y(t)}{x(t)}=\frac{t^3+16t^2+6t}{3(t^3+t^2+4t+4)}\xrightarrow[t\to\pm\infty]{}\frac{1}{3}$$
puis $y(t)-\frac{1}{3}x(t)\xrightarrow[t\to\pm\infty]{}2$. La courbe a donc pour asymptote, quand $t\to-\infty$ et quand $t\to+\infty$, la m\^eme droite d'équation $y=\frac{1}{3}x+2$.
On constate sur le tableau de variation qu'il n'y a qu'un seul point singulier, correspondant au paramètre $t=2$. Pour conna\^itre l'allure de la courbe au voisinage du point $M(2)$, on fait un développement limité de $x$ et $y$ au voisinage de $t=2$. Comme ici $x$ et $y$ sont de classe $\mathcal{C}^\infty$ et d'expressions assez simples, on peut directement appliquer la formule de Taylor-Young:
$$\left\{\begin{array}{l}
x(t)=x(2)+x'(2)\cdot(t-2)+\frac{1}{2}x''(2)\cdot(t-2)^2+\frac{1}{6}x'''(2)\cdot(t-2)^3+o((t-2)^3)\\
y(t)=y(2)+y'(2)\cdot(t-2)+\frac{1}{2}y''(2)\cdot(t-2)^2+\frac{1}{6}y'''(2)\cdot(t-2)^3+o((t-2)^3)
\end{array}\right.$$ 
On sait déjà que $x(2)=4$, $y(2)=\frac{11}{3}$ et $x'(2)=y'(2)=0$. De plus
$x''(t)=\frac{8}{t^3}$, $x'''(t)=\frac{-24}{t^4}$ et $y''(t)=\frac{6}{(t+1)^3}$, $y'''(t)=\frac{-18}{(t+1)^4}$, ce qui donne
$$M(t)=\begin{pmatrix}4 \\ \frac{11}{3}\end{pmatrix}+
\begin{pmatrix}\frac{1}{2}\\\frac{1}{9}\end{pmatrix}\cdot (t-2)^2+
\begin{pmatrix}-\frac{1}{4}\\ \frac{-1}{27}\end{pmatrix}\cdot (t-2)^3+o((t-2)^3)$$
C'est un point de rebroussement de première espèce. L'équation (sous forme paramétrée) de la tangente $T_{M(2)}$ s'obtient en tronquant le développement limité:
$$\begin{pmatrix}x\\ y\end{pmatrix}\in T_{M(2)}
\Longleftrightarrow\exists\lambda\in\R,\ \begin{pmatrix}x \\ y\end{pmatrix}
=\begin{pmatrix}4\\  \frac{11}{3}\end{pmatrix}+
\begin{pmatrix}\frac{1}{2}\\\frac{1}{9}\end{pmatrix}\cdot \lambda$$
En éliminant le paramètre $\lambda$, on récupère une équation cartésienne 
$$T_{M(2)}:\ y=\frac{11}{3}+\frac{1}{9}\cdot 2(x-4)=\frac{2}{9}x+\frac{25}{9}$$
}
\end{enumerate}
}
