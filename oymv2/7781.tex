\uuid{7781}
\auteur{mourougane}
\datecreate{2021-08-11}
\isIndication{false}
\isCorrection{false}
\chapitre{Sous-groupe distingué}
\sousChapitre{Sous-groupe distingué}

\contenu{
\texte{
On travaille dans $GL(2,k)$ pour un corps $k$ qui a au moins $4$
éléments. Soit $g\in GL(2,k)$. On notera $i_g$ l'automorphisme intérieur donné par $g$.
}
\begin{enumerate}
    \item \question{Démontrer qu'il existe un scalaire non nul $a\in k$ tel que $a^2\not=1$. Que se passe-t-il dans $\mathbb{F}_2$ et dans $\mathbb{F}_3$ ?}
    \item \question{Soit $$t=\begin{pmatrix}1&1\cr 0&1\end{pmatrix}
\quad \text{ et } \quad s=\begin{pmatrix}a&0\cr 0& a^{-1}\end{pmatrix}.$$
Montrer que $T=tst^{-1}s^{-1}$ est une transvection.}
    \item \question{Soit $\tau$ une transvection de $SL(2,k)$. Il existe $g\in GL(2,k)$ tel que $\tau = i_g(T):=gTg^{-1}$. Calculer $\tau$ à l'aide de $g$, $s$ et $t$ et montrer que $D(SL(2,k))$ contient toutes les transvections.}
    \item \question{Déterminer $D(SL(2,k))$.}
    \item \question{Déterminer $D(GL(2,k))$.}
\end{enumerate}
}
