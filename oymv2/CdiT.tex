\uuid{CdiT}
\exo7id{2177}
\auteur{debes}
\datecreate{2008-02-12}
\isIndication{true}
\isCorrection{false}
\chapitre{Action de groupe}
\sousChapitre{Action de groupe}

\contenu{
\texte{
Etant donn\'es un groupe $G$ et un sous-groupe $H$, on d\'efinit le normalisateur $\textrm{Nor}_G(H)$ de $H$ dans $G$ comme l'ensemble des \'el\'ements $g\in G$ tels que $gHg^{-1} = H$. 
\smallskip

(a) Montrer que $\textrm{Nor}_G(H)$ est le plus grand sous-groupe de $G$ contenant $H$ comme sous-groupe distingu\'e. 
\smallskip

(b) Montrer que le nombre de sous-groupes distincts conjugu\'es de $H$ dans $G$ est \'egal \`a l'indice $[G:\textrm{Nor}_G(H)]$ et qu'en particulier c'est un diviseur de l'ordre de $G$.
}
\indication{(a) Aucune difficult\'e.

(b) Le nombre cherch\'e est l'orbite de $H$ sous l'action de $G$ par conjugaison sur ses sous-groupes et $\textrm{Nor}_G(H)$ est le fixateur de $H$ pour cette action.}
}
