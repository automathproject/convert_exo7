\uuid{1167}
\auteur{cousquer}
\datecreate{2003-10-01}
\isIndication{false}
\isCorrection{true}
\chapitre{Déterminant, système linéaire}
\sousChapitre{Système linéaire, rang}

\contenu{
\texte{
Écrire les conditions, portant sur les réels $a$, $b$, $c$, pour que les
systèmes suivants admettent des solutions non nulles~; expliciter
ces solutions.
$$(S_1)\; \left\{\begin{array}{rcl}
    x+y+z &=&0 \\
    (b+c)x+(c+a)y+(a+b)z &=&0 \\
    bcx+acy+abz &=&0
\end{array}\right.
\qquad(S_2)\; \left\{\begin{array}{rcl}
    x-a(y+z) &=&0 \\ 
    y-b(x+z) &=& 0\\
    z-c(x+y)&=&0
\end{array}\right.$$
}
\reponse{
$(S_1)$~: $a=b$ ou $b=c$ ou $c=a$.\\
$(S_2)$~: $2abc+bc+ca+ab=1$.
}
}
