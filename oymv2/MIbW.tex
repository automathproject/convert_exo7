\uuid{MIbW}
\exo7id{1099}
\auteur{legall}
\datecreate{1998-09-01}
\isIndication{false}
\isCorrection{true}
\chapitre{Matrice}
\sousChapitre{Matrice et application linéaire}

\contenu{
\texte{
Soient  
$A=\begin{pmatrix} 
1 & 2 & 1 \cr
3 & 4 & 1 \cr
5 & 6 & 1 \cr
7 & 8 & 1 \cr
\end{pmatrix},\ 
B=\begin{pmatrix} 
2 & 2 & -1 & 7  \cr
4 & 3 & -1 & 11 \cr
0 & -1 & 2 & -4 \cr
3 & 3 & -2 & 11 \cr 
\end{pmatrix} $.
Calculer $\textrm{rg}(A)$ et $\textrm{rg}(B)$. Déterminer une base du
noyau et une base de l'image pour chacune des applications linéaires associées $f_A$ et $f_B$.
}
\reponse{
\begin{enumerate}
Commençons par des remarques élémentaires : la matrice est non nulle donc $\textrm{rg}(A) \ge 1$
et comme il y a $p=4$ lignes et $n=3$ colonnes alors $\textrm{rg}(A) \le \min(n,p)=3$.
Ensuite on va montrer $\textrm{rg}(A) \ge 2$ en effet le sous-déterminant $2\times 2$ 
(extrait du coin en haut à gauche) :
$\begin{vmatrix} 
1 & 2 \cr
3 & 4 \cr
\end{vmatrix}= -2$ est non nul.
Montrons que $\textrm{rg}(A)=2$. Avec les déterminants il faudrait vérifier que pour toutes
les sous-matrices $3\times 3$ les déterminants sont nuls. Pour éviter de nombreux calculs on remarque ici
que les colonnes sont liées par la relation $v_2=v_1+v_3$. Donc $\textrm{rg}(A)=2$.
L'application linéaire associée à la matrice $A$ est l'application 
$f_A : \Rr^3 \to \Rr^4$. Et le théorème du rang 
$\dim \Ker f_A+ \dim \Im f_A = \dim \Rr^3$ donne ici
$\dim \Ker f_A = 3 - \textrm{rg}(A)=1$.

Mais la relation $v_2=v_1+v_3$ donne immédiatement un élément du noyau :
en écrivant $v_1-v_2+v_3=0$ alors $A\begin{pmatrix}1\\-1\\1\end{pmatrix}=\begin{pmatrix}0\\0\\0\end{pmatrix}$
Donc $\begin{pmatrix}1\\-1\\1\end{pmatrix} \in \Ker f_A$. Et comme le noyau est de dimension $1$ alors
$$\Ker f_A = \textrm{Vect} \begin{pmatrix}1\\-1\\1\end{pmatrix}$$
Pour un base de l'image, qui est de dimension $2$, 
     il suffit par exemple de prendre les deux premiers vecteurs colonnes de la matrice $A$ (ils sont clairement non colinéaires) :
$$\Im f_A = \textrm{Vect} \left\{ v_1, v_2 \right\} = 
\textrm{Vect} \left\{  \begin{pmatrix}1\\3\\5\\7\end{pmatrix},  \begin{pmatrix}1\\1\\1\\1\end{pmatrix} \right\}
$$
}
}
