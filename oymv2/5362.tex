\uuid{5362}
\auteur{rouget}
\datecreate{2010-07-06}

\contenu{
\texte{
Montrer que $\left|
\begin{array}{ccc}
-2a&a+b&a+c\\
b+a&-2b&b+c\\
c+a&c+b&-2c
\end{array}\right|=4(b+c)(c+a)(a+b)$.
}
\reponse{
Soit $(a,b,c)\in\Rr^3$. Notons $\Delta$ le déterminant de l'énoncé. Pour $x$ réel, on pose $D(x)=\left|
\begin{array}{ccc}
-2x&x+b&x+c\\
b+x&-2b&b+c\\
c+x&c+b&-2c
\end{array}\right|$ (de sorte que $\Delta=D(a)$)). $D$ est un polynôme de degré inférieur ou égal à $2$. Le coefficient de $x^2$ vaut
 
$$-(-2c)+(b+c)+(b+c)-(-2b)=4(b+c).$$
Puis,

$$D(-b)=\left|
\begin{array}{ccc}
2b&0&-b+c\\
0&-2b&b+c\\
c-b&c+b&-2c
\end{array}\right|=2b(4bc-(b+c)^2)+2b(c-b)^2=0,$$
et par symétrie des rôles de $b$ et $c$, $D(-c)=0$. De ce qui précède, on déduit que si $b\neq c$, $D(x)=4(b+c)(x+b)(x+c)$ (même si $b+c=0$ car alors $D$ est un polynôme de degré infèrieur ou égal à $1$ admettant au moins deux racines distinctes et est donc le polynôme nul).
Ainsi, si $b\neq c$ (ou par symétrie des roles, si $a\neq b$ ou $a\neq c$), on a~:~$\Delta=4(b+c)(a+b)(a+c)$. Un seul cas n'est pas encore étudié à savoir le cas où $a=b=c$. Dans ce cas, 

$$D(a)=\left|
\begin{array}{ccc}
-2a&2a&2a\\
2a&-2a&2a\\
2a&2a&-2a
\end{array}\right|=8a^3\left|
\begin{array}{ccc}
-1&1&1\\
1&-1&1\\
1&1&-1
\end{array}\right|=32a^3=4(a+a)(a+a)(a+a),$$
ce qui démontre l'identité proposée dans tous les cas (on pouvait aussi conclure en constatant que, pour $a$ et $b$ fixés, la fonction $\Delta$ est une fonction continue de $c$ et on obtient la valeur de $\Delta$ pour $c=b$ en faisant tendre $c$ vers $b$ dans l'expression de $\Delta$ déjà connue pour $c\neq b$).

\begin{center}
\shadowbox{
$\Delta=4(a+b)(a+c)(b+c)$.
}
\end{center}
}
}
