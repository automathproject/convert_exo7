\uuid{3707}
\auteur{quercia}
\datecreate{2010-03-11}

\contenu{
\texte{
Soit $E$ un espace vectoriel euclidien orienté de dimension 3, ${\cal B}$ une base orthonormée
directe de $E$ et $f \in \mathcal{L}(E)$ de matrice dans $\cal B$ :
$M = \begin{pmatrix} 0      &-\gamma &\beta   \cr
               \gamma &0       &-\alpha \cr
               -\beta &\alpha  &0       \cr\end{pmatrix}$.
}
\begin{enumerate}
    \item \question{Reconaître $f$.}
\reponse{$f(\vec x) = \vec u\wedge\vec x$ avec
    $\vec u = (\alpha,\beta,\gamma)$.}
    \item \question{Montrer que $\mathrm{id}_E + f$ est une bijection et calculer la bijection
    réciproque.}
\reponse{$\vec y = \frac{\vec x
                     + (\vec u\mid\vec x)\vec u-\vec u\wedge\vec x}
                     {1+\|\vec u\,\|^2}$.}
    \item \question{Montrer que $g = (\mathrm{id} - f) \circ (\mathrm{id} + f)^{-1}$ est une rotation et
    préciser son axe et son angle.}
\reponse{axe dirigé par $\vec u$,
             $\cos\theta = \frac{1-\|\vec u\,\|^2}{1+\|\vec u\,\|^2}$,
             $\sin\theta = \frac{-2\|\vec u\,\|}{1+\|\vec u\,\|^2}$.}
\end{enumerate}
}
