\uuid{SkT1}
\exo7id{3887}
\auteur{quercia}
\datecreate{2010-03-11}
\isIndication{false}
\isCorrection{false}
\chapitre{Continuité, limite et étude de fonctions réelles}
\sousChapitre{Etude de fonctions}

\contenu{
\texte{
On pose $f(x) = \begin{cases}\exp\left(-\frac 1x\right)& \text{ si } x > 0 \cr
                       0                          & \text{ si } x = 0.\cr\end{cases}$
}
\begin{enumerate}
    \item \question{Montrer que $f$ est de classe $\mathcal{C}^\infty$ sur $\R^{+*}$, et que $f^{(n)}(x)$ est de la
forme $\frac {P_n(x)}{x^{2n}}\exp\left(-\frac 1x\right)$ où $P_n$ est une fonction
polynomiale de degré inférieur ou égal à $n-1$ ($n \ge 1$).}
    \item \question{Montrer que $f$ est de classe $\mathcal{C}^\infty$ en $0^+$.}
    \item \question{Montrer que le polynôme $P_n$ possède $n-1$ racines dans $\R^{+*}$.}
\end{enumerate}
}
