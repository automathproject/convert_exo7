\uuid{53H7}
\exo7id{5687}
\auteur{rouget}
\datecreate{2010-10-16}
\isIndication{false}
\isCorrection{true}
\chapitre{Réduction d'endomorphisme, polynôme annulateur}
\sousChapitre{Autre}

\contenu{
\texte{
Soit $A$ une matrice carrée réelle de format $n\geqslant 2$ vérifiant $A^3+A^2+A=0$. Montrer que le rang de $A$ est un entier pair.
}
\reponse{
Soit $P=X^3+X^2+X=X(X-j)(X-j^2)$. $P$ est à racines simples dans $\Cc$ et annulateur de $A$. Donc $A$ est diagonalisable dans $\Cc$ et ses valeurs propres sont à choisir dans $\{0,j,j^2\}$. Le polynôme caractéristique de $A$ est de la forme $(-1)^nX^\alpha(X-j)^\beta(X-j^2)^\gamma$ avec $\alpha+\beta+\gamma=n$. De plus, $A$ est réelle et on sait que $j$ et $j^2=\overline{j}$ ont même ordre de multiplicité ou encore $\gamma=\beta$.

Puisque $A$ est diagonalisable, l'ordre de multiplicité de chaque valeur propre est égale à la dimension du sous-espace propre correspondant et donc

\begin{center}
$\text{rg}(A)=n-\text{dim}(\text{Ker}A)=n-\alpha=2\beta$.
\end{center}

On a montré que $\text{rg}A$ est un entier pair.
}
}
