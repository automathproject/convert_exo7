\uuid{MaqG}
\exo7id{132}
\auteur{cousquer}
\datecreate{2003-10-01}
\isIndication{false}
\isCorrection{false}
\chapitre{Logique, ensemble, raisonnement}
\sousChapitre{Ensemble}

\contenu{
\texte{
Soit $A$ une partie de $E$, on appelle fonction caractéristique de~$A$
l'application~$f$ de~$E$ dans l'ensemble à deux éléments $\{0, 1\}$, telle
que~:
$$f(x)=\begin{cases}
0&  \text{ si } x\notin A \cr 1& \text{ si } x \in A \cr
\end{cases}$$
Soit $A$ et $B$ deux parties de $E$, $f$ et $g$ leurs fonctions
caractéristiques. Montrer que les fonctions suivantes sont les fonctions
caractéristiques d'ensembles que l'on déterminera~:
}
\begin{enumerate}
    \item \question{$1-f$.}
    \item \question{$fg$.}
    \item \question{$f+g-fg$.}
\end{enumerate}
}
