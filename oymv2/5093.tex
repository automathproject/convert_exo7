\uuid{5093}
\auteur{rouget}
\datecreate{2010-06-30}
\isIndication{false}
\isCorrection{true}
\chapitre{Fonctions circulaires et hyperboliques inverses}
\sousChapitre{Fonctions hyperboliques et hyperboliques inverses}

\contenu{
\texte{
Résoudre dans $\Rr$ l'équation $\sh(2+x)+\sh(2+2x)+...+\sh(2+100x)=0$.
}
\reponse{
Soit $x$ un réel.

$$S=\sum_{k=1}^{100}\sh(2+kx)=\frac{1}{2}\left(e^2\sum_{k=1}^{100}e^{kx}-e^{-2}\sum_{k=1}^{100}e^{-kx}\right).$$
Si $x=0$
alors directement $S=100\sh2\neq0$. Si $x\neq0$ alors $e^x\neq1$ et $e^{-x}\neq1$. Dans ce
cas,

$$S=\frac{1}{2}\left(e^2e^x\frac{1-e^{100x}}{1-e^x}-e^{-2}e^{-x}\frac{1-e^{-100x}}{1-e^{-x}}\right)=\frac{1}{2}\left(e^2e^x
\frac{1-e^{100x}}{1-e^x}+e^{-2}\frac{1-e^{-100x}}{1-e^{x}}\right).$$
après multiplication du numérateur et du dénominateur
de la deuxième fraction par $e^x$. Pour $x\neq0$, on a donc~:
\begin{align*}
S=0&\Leftrightarrow e^{x+2}(1-e^{100x})+ e^{-2}(1-e^{-100x})=0\Leftrightarrow
e^{x+2}(1-e^{100x})+e^{-2-100x}(e^{100x}-1)=0\\
 &\Leftrightarrow(1-e^{100x})(e^{x+2}-e^{-100x-2})=0\Leftrightarrow e^{x+2}=e^{-100x-2}\;(\mbox{car}\;
x\neq0)\\
 &\Leftrightarrow x+2=-100x-2\Leftrightarrow x=-\frac{4}{101}.
\end{align*}

\begin{center}
\shadowbox{
$\mathcal{S}=\left\{-\frac{4}{101}\right\}.$
}
\end{center}
}
}
