\uuid{2643}
\auteur{debievre}
\datecreate{2009-05-19}
\isIndication{true}
\isCorrection{true}
\chapitre{Fonction de plusieurs variables}
\sousChapitre{Extremums locaux}

\contenu{
\texte{

}
\begin{enumerate}
    \item \question{Soit 
$f$ une fonction r\'eelle d'une variable r\'eelle
de classe ${\rm C}^2$ dans un voisinage de $0\in \mathbb R$ 
telle que $f(0)=0$ et
$f^\prime(0)\not=0$. Montrer que la fonction 
r\'eelle $F$ des deux variables $x$ et $y$
d\'efinie dans un voisinage de $(0,0)$
par 
$F(x,y)= f(x) f(y)$ n'a pas d'extremum relatif en $(0,0)$.
Est-ce que  le point $(0,0)$ est quand m\^eme critique?
Si oui caract\'eriser sa nature.}
\reponse{$dF=f(y) f'(x)\mathrm dx + f(x) f'(y)\mathrm dy$ et
\begin{equation}
\mathrm{Hess}_f(x,y)=\left[\begin{matrix} 
f(y) f''(x) &  f'(x) f'(y)\\  f'(x) f'(y) & f(x)  f''(y)
\end{matrix}\right]
\label{Hess}
\end{equation} 
d'o\`u
$\mathrm{Hess}_f(0,0)= (f'(0))^2\left[\begin{matrix} 
0 &  1\\  1 & 0
\end{matrix}\right]$
et
\[
(u,v)\mathrm{Hess}_f(0,0)
\left[\begin{matrix} u \\ v \end{matrix}\right]
=(f'(0))^2
(u,v)\left[\begin{matrix} 
0 & 1\\ 1 &  0
\end{matrix}\right]
\left[\begin{matrix} u \\ v \end{matrix}\right]
=2(f'(0))^2uv
\]
Par cons\'equent la forme hessienne au point $(0,0)$
est non d\'eg\'en\'er\'ee et 
ind\'efinie et ce point ne peut pas pr\'esenter 
un extremum relatif.
En effet, 
le point $(0,0)$ est critique mais  un point selle.}
    \item \question{D\'eterminer les points critiques, puis les minima et les maxima locaux de
\[
f(x,y)=\sin(2\pi x)\sin(2\pi y).
\]
Remarque: en utilisant la 
p\'eriodicit\'e de la fonction, on peut limiter le nombre de cas \`a \'etudier.}
\reponse{D'apr\`es la partie (1.) et la p\'eriodicit\'e, les points de la forme
\begin{equation}
(x,y)=(k,l) \in \R^2,\ k,l \in \mathbb Z,
\label{selle}
\end{equation} 
pr\'esentent des points selle.
\'Egalement d'apr\`es la partie (1.),
\begin{align*}
\frac{\partial f}{\partial x}&=f(y) f'(x)=2 \pi \sin (2\pi y) \cos (2\pi x)
\\
\frac{\partial f}{\partial y}&=f(x) f'(y)=2 \pi \sin (2 \pi x) \cos (2 \pi y).
\end{align*}
Par cons\'equent,
pour que le point $(x,y)$  soit critique
il faut et il suffit qu'il soit de la forme
\[
(k,l),\ (k+\tfrac 12,l),\ (k,l+\tfrac 12),\ (k+\tfrac 12,l+\tfrac 12),\  
k,l \in \mathbb Z,
\]
ou
\[
 (k+\tfrac 14,l+\tfrac 14),
\
 (k+\tfrac 14,l+\tfrac 34),
\
 (k+\tfrac 34,l+\tfrac 14),
\
 (k+\tfrac 34,l+\tfrac 34), \ k,l \in \mathbb Z.
\]
D'apr\`es la p\'eriodicit\'e, il suffit d'examiner les huit points
\[
(0,0),\ (\tfrac 12,0),\ (0,\tfrac 12),\ (\tfrac 12,\tfrac 12),
\
 (\tfrac 14,\tfrac 14),
\
 (\tfrac 14,\tfrac 34),
\
 (\tfrac 34,\tfrac 14),
\
 (\tfrac 34,\tfrac 34)
\]
et, d'apr\`es (1.), l'origine pr\'esente un point selle.
D'apr\`es \eqref{Hess},
\begin{align*}
\mathrm{Hess}_f(0,\tfrac 12)&=\left[\begin{matrix} 
f(\tfrac 12) f''(0) &  f'(0) f'(\tfrac 12)\\  f'(0) f'(\tfrac 12) & f(0)  f''(\tfrac 12)
\end{matrix}\right]=16 \pi^2\left[\begin{matrix} 
0 &  -1\\  -1 & 0
\end{matrix}\right]
\\
\mathrm{Hess}_f(\tfrac 12,0)&=\left[\begin{matrix} 
f(0) f''(\tfrac 12) &  f'(\tfrac 12) f'(0)\\  f'(\tfrac 12) f'(0) & f(\tfrac 12)  f''(0)
\end{matrix}\right]=16 \pi^2\left[\begin{matrix} 
0 &  -1\\  -1 & 0
\end{matrix}\right]
\\
\mathrm{Hess}_f(\tfrac 12,\tfrac 12)&=\left[\begin{matrix} 
f(\tfrac 12) f''(\tfrac 12) &  f'(\tfrac 12) f'(\tfrac 12)\\  f'(\tfrac 12) f'(\tfrac 12) & f(\tfrac 12)  f''(\tfrac 12)
\end{matrix}\right]=16 \pi^2\left[\begin{matrix} 
0 &  1\\  1 & 0
\end{matrix}\right]
\end{align*}
d'o\`u les points $(0,\frac 12)$, $(\tfrac 12,0)$ et $(\tfrac 12,\tfrac 12)$
pr\'esentent des points selle.
Il est g\'eo\-m\'etri\-quement 
\'evident que le comportement de la fonction sin entra\^\i ne
que les points $(\tfrac 14,\tfrac 14)$ et  $(\tfrac 34,\tfrac 34)$
pr\'esentent des maxima et que les points
$(\tfrac 14,\tfrac 34)$ et  $(\tfrac 34,\tfrac 14)$
pr\'esentent des minima.}
\indication{Voir les exercices pr\'ec\'edents.}
\end{enumerate}
}
