\uuid{WLm3}
\exo7id{2653}
\auteur{debievre}
\datecreate{2009-05-19}
\isIndication{false}
\isCorrection{false}
\chapitre{Fonction de plusieurs variables}
\sousChapitre{Fonctions implicites}

\contenu{
\texte{
On consid\`ere la courbe $\mathcal C$ d'\'equation
$$y^2(x^2+1)+x^2(y^2+1)=1.$$
}
\begin{enumerate}
    \item \question{Montrer qu'il existe un unique $b>0$ tel que le point de coordonn\'ees $(1/2,b)$ se trouve sur $\mathcal C$. D\'eterminer $b$, puis d\'eterminer l'\'equation de la droite tangente  \`a $\mathcal C$, passant par $(1/2,b)$.}
    \item \question{Trouver l'unique fonction $\varphi:x\in ]-1,1[\to\varphi(x)\in\R^+$ telle que $(x,\varphi(x))\in\mathcal C$ pour tout $x\in]-1,1[$. Montrer que $\varphi(-x)=\varphi(x)$ et que $\varphi$ est d\'ecroissante sur $[0,1[$. Tracer $\mathcal C$.}
    \item \question{\'Enoncer le th\'eor\`eme des fonctions implicites et montrer qu'il existe exactement deux points de la courbe $\mathcal C$ o\`u le th\'eor\`eme des fonctions implicites ne  s'applique pas pour \'ecrire, au voisinage de chacun de ces deux points,
$y$ comme fonction de $x$.}
\end{enumerate}
}
