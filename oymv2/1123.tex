\uuid{1123}
\auteur{liousse}
\datecreate{2003-10-01}
\isIndication{false}
\isCorrection{false}
\chapitre{Déterminant, système linéaire}
\sousChapitre{Calcul de déterminants}

\contenu{
\texte{
Soit $n$ un entier sup\'erieur ou 
\'egal \`a 3. On se place dans $\Rr^{n}.$ On note 
$e_i$ le vecteur de $\Rr^{n}$ dont la i-i\`eme  composante est 
\'egale \`a 1 et toutes les autres sont nulles. \'Ecrire la matrice $n\times n$ 
dont les vecteurs colonnes $C_i$ sont donn\'es par 
$C_i= e_i+e_n$ pour $1\leq i\leq n-1$ et $C_n=e_1+e_2+e_n.$ 
Calculer alors son d\'eterminant.
}
}
