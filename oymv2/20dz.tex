\uuid{20dz}
\exo7id{7187}
\auteur{megy}
\datecreate{2019-07-17}
\isIndication{false}
\isCorrection{false}
\chapitre{Logique, ensemble, raisonnement}
\sousChapitre{Ensemble}

\contenu{
\texte{
Dans l'ensemble $\R$, il existe une notion de \emph{partie bornée} : c'est une partie qui est incluse dans un segment du type $[-M,M]$, pour un certain $M$. Cet exercice montre comment généraliser cette notion de \emph{partie bornée} à un ensemble quelconque.

Soit $E$ un ensemble et  $\mathcal B$ une partie de $\mathcal P(E)$. On dit que $\mathcal B$ est une \emph{bornologie sur $E$} si les conditions suivantes sont vérifiées
\begin{itemize}
\item Si $A\in \mathcal B$ et $B\subseteq A$, alors $B\in \mathcal B$.
\item Si $A\in \mathcal B$ et $B \in \mathcal B$, alors $A\cup B\in \mathcal B$.
\item Pour tout $x\in E$, on a  $\{x\} \in \mathcal B$.
\end{itemize}
Les éléments de $\mathcal B$ sont dits \emph{$\mathcal B$-bornés}, ou simplement \emph{bornés} s'il n'y a pas d'ambiguïté sur la bornologie utilisée.

Dans la suite, on fixe un ensemble $E$.
}
\begin{enumerate}
    \item \question{Montrer que $\mathcal B_1=\{\emptyset, E\}$ est une bornologie de $E$. On l'appelle la \emph{bornologie triviale (ou : grossière)}.}
    \item \question{Montrer que l'ensemble $\mathcal B_2$ des parties finies de $E$ est une bornologie de $E$. On l'appelle la \emph{bornologie discrète}.}
    \item \question{Combien de bornologies différentes y a-t-il si $E$ est vide ? S'il contient (exactement) un élément ? Deux ? Trois ?}
    \item \question{On suppose maintenant que $E=\R$. Soit $\mathcal B_3$ l'ensemble des parties $A\subseteq \R$ bornées au sens classique, autrement dit 
\[ A\in \mathcal B_3 \iff \exists M\in \R, \forall a\in A, |a|\leq M\]
Montrer que $\mathcal B_3$ est une bornologie. On l'appelle la \emph{bornologie usuelle sur $\R$}, et lorsqu'on parle de bornés de $\R$, il est implicite qu'on se réfère à cette bornologie (et non aux deux premières par exemple).}
\end{enumerate}
}
