\uuid{TxO6}
\exo7id{3739}
\auteur{quercia}
\datecreate{2010-03-11}
\isIndication{false}
\isCorrection{false}
\chapitre{Espace euclidien, espace normé}
\sousChapitre{Projection, symétrie}

\contenu{
\texte{
Soit $E$ un espace euclidien de dimension $n$.
Soit $f \in \mathcal{L}(E)$ et $\lambda > 0$.
On dit que $f$ est une similitude de rapport $\lambda$ si :
$\forall\ \vec x \in E,\ \|f(\vec x)\| = \lambda\|\vec x\,\|$.
}
\begin{enumerate}
    \item \question{Montrer que $f$ est une similitude de rapport $\lambda$ si et seulement si :
    $\forall\ \vec x,\vec y\in E,\
    (f(\vec x)\mid f(\vec y)) = \lambda^2(\vec x\mid\vec y)$.}
    \item \question{Caractériser les similtudes par leurs matrices dans une base orthonormée.}
    \item \question{Montrer que $f$ est une similitude si et seulement si $f$ est non nulle et
    conserve l'orthogonalité, c'est à dire~:
    $\forall\ \vec x,\vec y\in E,\ \vec x\perp\vec y
       \Rightarrow  f(\vec x)\perp f(\vec y)$.}
\end{enumerate}
}
