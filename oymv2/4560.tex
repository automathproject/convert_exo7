\uuid{8dwv}
\exo7id{4560}
\auteur{quercia}
\datecreate{2010-03-14}
\isIndication{false}
\isCorrection{true}
\chapitre{Suite et série de fonctions}
\sousChapitre{Autre}

\contenu{
\texte{
0,\pi[}$, TPE MP 2005]
}
\begin{enumerate}
    \item \question{Calculer $S_n(t) = \sum_{p=1}^n t^{p-1}\sin(px)$ puis $S(t) = \lim_{n\to\infty} S_n(t)$.}
\reponse{Lorsque $n\to\infty$, $S_n(t) = \Im\Bigl(\frac{e^{ix}-t^{n}e^{i(n+1)x}}{1-te^{ix}}\Bigr) \to
              \Im\Bigl(\frac{e^{ix}}{1-te^{ix}}\Bigr) = 
              \frac{\sin x}{1-2t\cos x + t^2}$ pour $-1<t<1$.}
    \item \question{Calculer $ \int_{t=0}^1 S_n(t)\,d t$ et  $ \int_{t=0}^1 S(t)\,d t$.}
\reponse{$ \int_{t=0}^1 S_n(t)\,d t = \sum_{p=1}^n \frac{\sin(px)}{p}$.
             \par
             $ \int_{t=0}^1 S(t)\,d t
	      = (t-\cos x = u\sin x) =  \int_{u=-\cot x}^{\tan x/2} \frac{d u}{1+ u^2} = \frac{\pi-x}2$.}
    \item \question{En déduire que $\sum_{n=1}^\infty \frac{\sin nx}n$ converge et donner sa valeur.}
\reponse{TCD~: $|S_n(t)|\le \frac2{\sin x}$ intégrable par rapport à~$t$ sur $[0,1]$. On en déduit $\sum_{p=1}^\infty \frac{\sin(px)}{p} = \frac{\pi-x}2$.}
\end{enumerate}
}
