\uuid{6940}
\auteur{ruette}
\datecreate{2013-01-24}
\isIndication{false}
\isCorrection{true}
\chapitre{Loi, indépendance, loi conditionnelle}
\sousChapitre{Loi, indépendance, loi conditionnelle}

\contenu{
\texte{

}
\begin{enumerate}
    \item \question{Soit $X$ une variable aléatoire de loi normale $\mathcal{N}(0,1)$
et $Z=X^2$. Calculer la fonction de répartition et la densité
de $Z$.

\textit{Remarque : la loi de $Z$ est appelée loi $\chi_2$ à 1 degré de liberté}.}
\reponse{$P(Z\leq t)=0$ si $t<0$ et $P(Z\leq t)=P(-\sqrt{t}\leq X\leq \sqrt{t})$
si $t\geq 0$. Donc, pour $t\geq 0$,
$F_Z(t)=F_X(\sqrt{t})-F_X(-\sqrt{t})$.
$F_X(t)$ est dérivable de dérivée $\frac{1}{\sqrt{2\pi}}e^{-t^2/2}$, donc 
$F_Z$ est  
dérivable sur $]0,+\infty[$ et $F_Z'(t)$ est la densité de $Z$.

$F_Z'(t)=\frac{1}{2\sqrt{t}}(F_X'(\sqrt{t})+F_X'(-\sqrt{t}))$.
Donc la densité de $Z$ est 
%$f_Z(t)=\frac{1}{2\sqrt{t}}
%(f_X(\sqrt{t})+f_X(-\sqrt{t}))$ si $t>0$.\\
$f_Z(t)=\I1_{\Rr_+^*}(t)\frac{1}{\sqrt{2\pi t}}e^{-t/2}$.}
    \item \question{Soit $Y$ une variable aléatoire de loi exponentielle $\mathcal{E}(\lambda)$. Déterminer 
la loi de $Y^3$.}
\reponse{$F_{Y^3}(t)=P(Y^3\leq t)=P(Y\leq\sqrt[3]{t})=F_Y(\sqrt[3]{t})$. 
On a $F_Y(t)=0$ si $t\leq 0$ et $F_Y(t)=\displaystyle\int_0^t \lambda e^{-\lambda t}\,
dt=1-e^{-\lambda t}$ si $t>0$. Donc
$F_{Y^3}(t)=0$ si $t\leq 0$ et $F_{Y^3}(t)=1-e^{-\lambda \sqrt[3]{t}}$ si
$t>0$.
La fonction de répartition est continue sur $\Rr$ et $C^1$ par morceaux 
donc on peut dériver pour trouver la densité de $X^3$, qui est
$\I1_{]0,+\infty[} \frac{\lambda}{3}t^{-2/3}e^{-\lambda \sqrt[3]{t}}$.}
\end{enumerate}
}
