\uuid{KS0Q}
\exo7id{210}
\auteur{bodin}
\datecreate{1998-09-01}
\isIndication{true}
\isCorrection{true}
\chapitre{Logique, ensemble, raisonnement}
\sousChapitre{Relation d'équivalence, relation d'ordre}

\contenu{
\texte{
Soit $\mathcal{R}$ une relation binaire sur un
ensemble $E$, sym\'etrique et transitive. Que penser du
raisonnement suivant ?
\begin{center}
``$x\mathcal{R} y \Rightarrow y\mathcal{R} x$ car $\mathcal{R}$ est sym\'etrique, \\
or $(x \mathcal{R} y \text{ et } y\mathcal{R} x) \Rightarrow x \mathcal{R} x$ car $\mathcal{R}$
est transitive,\\
donc $\mathcal{R}$ est r\'eflexive.''
\end{center}
}
\indication{Il faut trouver l'erreur dans ce raisonnement,
car bien s\^ur s'il y a trois axiomes pour la d\'efinition d'une relation d'\'equivalence, c'est que deux ne suffisent pas !}
\reponse{
Le raisonnement est faux.

L'erreur est due au manque de quantification. En effet, rien ne
prouve que pout tout $x$ un tel $y$ existe. Il peut exister un
\'el\'ement $x$ qui n'est en relation avec personne (m\^eme pas
avec lui).
}
}
