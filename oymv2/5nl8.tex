\uuid{5nl8}
\exo7id{5762}
\auteur{rouget}
\datecreate{2010-10-16}
\isIndication{false}
\isCorrection{true}
\chapitre{Série entière}
\sousChapitre{Développement en série entière}

\contenu{
\texte{
Développer en série entière $F(x)=\int_{0}^{+\infty}e^{-t^2}\sin(tx)\;dt$ et en déduire que pour tout réel $x$, $F(x)=\frac{e^{-x^2/4}}{2}\int_{0}^{x}e^{t^2/4}\;dt$.
}
\reponse{
Soit $x\in\Rr$. La fonction $t\mapsto e^{-t^2}\sin(tx)$ est continue sur $[0,+\infty[$, négligeable devant $\frac{1}{t^2}$ quand $t$ tend vers $+\infty$ et est donc intégrable sur $[0,+\infty[$. La fonction $F$ est donc définie sur $\Rr$ et impaire.

Soit $x\in\Rr$. Pour tout réel $t$, posons $f(t)=e^{-t^2}\sin(tx)$. Pour $t\in\Rr$, on a

\begin{center}
$e^{-t^2}\sin(tx) =\sum_{n=0}^{+\infty}(-1)^n\frac{x^{2n+1}}{(2n+1)!}t^{2n+1}e^{-t^2}$.
\end{center}

Pour $n\in\Nn$ et $t\in\Rr$, posons $f_n(t)=(-1)^n\frac{x^{2n+1}}{(2n+1)!}t^{2n+1}e^{-t^2}$.

\textbullet~Chaque fonction $f_n$, $n\in\Nn$, est continue puis intégrable sur $[0,+\infty[$ car négligeable devant $\frac{1}{t^2}$ quand $t$ tend vers $+\infty$.

\textbullet~La série de fonctions de terme général $f_n$, $n\in\Nn$, converge simplement vers la fonction $f$ sur $[0,+\infty[$.

\textbullet~Ensuite, $\sum_{n=0}^{+\infty}\int_{0}^{+\infty}|f_n(t)|\;dt=\sum_{n=0}^{+\infty}\frac{|x|^{2n+1}}{(2n+1)!}\int_{0}^{+\infty}t^{2n+1}e^{-t^2}\;dt$. Pour $n\in\Nn$, posons $I_n=\int_{0}^{+\infty}t^{2n+1}e^{-t^2}\;dt$.

Soit $n\in\Nn^*$. Soit $A$ un réel strictement positif. Les deux fonctions $t\mapsto t^{2n}$ et $t\mapsto-\frac{1}{2}e^{-t^2}$ sont de classe $C^1$ sur le segment $[0,A]$. On peut donc effectuer une intégration par parties et on obtient

\begin{align*}\ensuremath
\int_{0}^{A}t^{2n+1}e^{-t^2}\;dt&=\int_{0}^{A}t^{2n}\times te^{-t^2}\;dt=\left[-\frac{1}{2}t^{2n}e^{-t^2}\right]_0^A+n\int_{0}^{A}t^{2n-1}e^{-t^2}\;dt\\
 &=-\frac{1}{2}A^{2n}e^{-A^2}+n\int_{0}^{A}t^{2n-1}e^{-t^2}\;dt.
\end{align*}

Quand $A$ tend vers $+\infty$, on obtient $I_{n}=nI_{n-1}$. En tenant compte, de $I_0=\int_{0}^{+\infty}te^{-t^2}\;dt=\frac{1}{2}$ on a donc $\forall n\in\Nn$, $I_n=\frac{n!}{2}$ puis

\begin{center}
$\sum_{n=0}^{+\infty}\int_{0}^{+\infty}|f_n(t)|\;dt=\sum_{n=0}^{+\infty}\frac{n!|x|^{2n+1}}{(2n+1)!}$.
\end{center}

Soient $n\in\Nn$ et $x\in\Rr$. $\left|\frac{\frac{(n+1)!|x|^{2n+3}}{(2n+3)!}}{\frac{n!|x|^{2n+1}}{(2n+1)!}}\right|=\frac{(n+1)x^2}{(2n+3)(2n+2)}$ et donc $\lim_{n \rightarrow +\infty}\frac{\frac{(n+1)!|x|^{2n+3}}{(2n+3)!}}{\frac{n!|x|^{2n+1}}{(2n+1)!}}=0$. D'après la règle de d'\textsc{Alembert}, la série numérique de terme général $\frac{n!|x|^{2n+1}}{(2n+1)!}$ converge.

En résumé, pour tout réel $x$,

\textbullet~Chaque fonction $f_n$, $n\in\Nn$, est continue puis intégrable sur $[0,+\infty[$ car négligeable devant $\frac{1}{t^2}$ quand $t$ tend vers $+\infty$.

\textbullet~La série de fonctions de terme général $f_n$, $n\in\Nn$, converge simplement vers la fonction $f$ sur $[0,+\infty[$.

\textbullet~$\sum_{n=0}^{+\infty}\int_{0}^{+\infty}|f_n(t)|\;dt<+\infty$.

D'après un théorème d'intégration terme à terme, pour tout réel $x$,

\begin{center}
$\int_{0}^{+\infty}e^{-t^2}\sin(tx)\;dt=\sum_{n=0}^{+\infty}\int_{0}^{+\infty}f_n(t)\;dt=\sum_{n=0}^{+\infty}(-1)^n\frac{n!x^{2n+1}}{2(2n+1)!}$.
\end{center}

\begin{center}
\shadowbox{
$\forall x\in\Rr$, $\int_{0}^{+\infty}e^{-t^2}\sin(tx)\;dt=\sum_{n=0}^{+\infty}(-1)^n\frac{n!x^{2n+1}}{2(2n+1)!}$.
}
\end{center}

$F$ est dérivable sur $\Rr$ et pour tout réel $x$,

\begin{center}
$F'(x)=\sum_{n=0}^{+\infty}(-1)^n\frac{n!x^{2n}}{2(2n)!}=\frac{1}{2}-\frac{x}{2}\sum_{n=1}^{+\infty}(-1)^{n-1}\frac{(n-1)!x^{2n-1}}{(2(2n-1)!}=\frac{1}{2}-\frac{x}{2}F(x)$.
\end{center}

Par suite, pour tout réel $x$, $e^{x^2/4}F'(x) +\frac{x}{2}e^{x^2/4}F(x)=\frac{e^{x^2/4}}{2}$ et donc

\begin{center}
$F(x)=F(0)+\frac{e^{-x^2/4}}{2}\int_{0}^{x}e^{t^2/4}\;dt =\frac{e^{-x^2/4}}{2}\int_{0}^{x}e^{t^2/4}\;dt$.
\end{center}

\begin{center}
\shadowbox{
$\forall x\in\Rr$, $\int_{0}^{+\infty}e^{-t^2}\sin(tx)\;dt=\frac{e^{-x^2/4}}{2}\int_{0}^{x}e^{t^2/4}\;dt$.
}
\end{center}
}
}
