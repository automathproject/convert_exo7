\uuid{6693}
\auteur{queffelec}
\datecreate{2011-10-16}

\contenu{
\texte{
Soit $f$ une fonction holomorphe dans un ouvert connexe $\Omega$ contenant
$0$. 
Montrer que  :
}
\begin{enumerate}
    \item \question{Si $f({1\over n})={1\over n+1}$ pour $n$ assez grand alors $f(z)={z\over
z+1}$ sur $\Omega\cap D(0,1)$.}
    \item \question{Si $f({1\over n})=f({1\over 2n})$ pour $n$ assez grand alors $f$ est 
constante sur $\Omega$.}
    \item \question{Si $f({1\over n})=f({1\over n+1})$ pour $n$ assez grand alors $f$ est
constante sur $\Omega$.}
    \item \question{$f({1\over n})=2^{-n}$ pour $n$ assez grand est impossible.}
\end{enumerate}
}
