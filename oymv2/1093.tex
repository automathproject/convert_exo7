\uuid{THaN}
\exo7id{1093}
\auteur{legall}
\datecreate{1998-09-01}
\isIndication{false}
\isCorrection{true}
\chapitre{Matrice}
\sousChapitre{Matrice et application linéaire}

\contenu{
\texte{
\label{exo1093}
Soit  $E$  un espace vectoriel et  $f$  une application linéaire de
$E$  dans  lui-m\^eme telle que  $f^2=f$.
}
\begin{enumerate}
    \item \question{Montrer que  $E= \Ker f \oplus \Im f$.}
    \item \question{Supposons que  $E$ soit de dimension finie  $n$. 
Posons  $r= \dim \Im f$. 
Montrer qu'il existe une base 
$\mathcal{B}= ( e_1, \ldots ,e_n)$ de  $E$  telle que : 
 $f(e_i)=e_i$ si $i\le r$ et $f(e_i)=0$ si $i>r$. 
Déterminer la matrice de  $f$ dans cette base $\mathcal{B}$.}
\reponse{
Nous devons montrer $\Ker f \cap \Im f = \{0\}$ et 
$\Ker f + \Im f = E$.
  \begin{enumerate}
Si $x \in \Ker f \cap \Im f$ alors d'une part $f(x)=0$ et d'autre part il existe $x'\in E$ tel que 
$x=f(x')$. Donc $0=f(x)=f\big(f(x')\big)= f(x')=x$ donc $x=0$ (on a utilisé $f\circ f=f$). 
Donc $\Ker f \cap \Im f = \{0\}$.
Pour $x\in E$ on le réécrit $x=x-f(x) + f(x)$. Alors $x-f(x) \in \Ker f$ 
(car $f\big(x-f(x) \big)=f(x)-f\circ f (x)=0$)
et $f(x)\in \Im f$. Donc $x \in \Ker f + \Im f$. Donc $\Ker f + \Im f = E$.
Conclusion  : $E= \Ker f \oplus \Im f$.
}
\end{enumerate}
}
