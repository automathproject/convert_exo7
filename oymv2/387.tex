\uuid{387}
\auteur{cousquer}
\datecreate{2003-10-01}
\isIndication{false}
\isCorrection{true}
\chapitre{Polynôme, fraction rationnelle}
\sousChapitre{Pgcd}

\contenu{
\texte{
Calculer le pgcd $D$ des polyn\^omes $A$
et $B$ d\'efinis ci-dessous. Trouver des polyn\^omes
$U$ et~$V$ tels que $D=AU+BV$.
}
\begin{enumerate}
    \item \question{$A=X^5+3X^4+2X^3-X^2-3X-2$ \quad et\quad $B=X^4+2X^3+2X^2+7X+6$.}
\reponse{$D = X^2+3X+2 =
A ({1\over18} X- {1\over6})+B (-{1\over18} X^2+{1\over9} X+{5\over18})$.}
    \item \question{$A=X^6-2X^5+2X^4-3X^3+3X^2-2X$ \quad et\quad $B=X^4-2X^3+X^2-X+1$.}
\reponse{$D = 1 = A(-X^3)+B(X^5+X^3+X+1)$.}
\end{enumerate}
}
