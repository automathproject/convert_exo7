\uuid{lZgZ}
\exo7id{5220}
\auteur{rouget}
\datecreate{2010-06-30}
\isIndication{false}
\isCorrection{true}
\chapitre{Suite}
\sousChapitre{Autre}

\contenu{
\texte{
Soient $(u_n)_{n\in\Nn}$ une suite réelle et $(v_n)_{n\in\Nn}$ la suite définie par~:~$\forall n\in\Nn,\;v_n=\frac{u_0+u_1+...+u_n}{n+1}$.
}
\begin{enumerate}
    \item \question{Montrer que si la suite $(u_n)_{n\in\Nn}$ vers un réel $\ell$, la suite $(v_n)_{n\in\Nn}$ converge et a pour limite $\ell$. Réciproque~?}
\reponse{Soit $\varepsilon>0$. Il existe un rang $n_0$ tel que, si $n\geq n_0$ alors $|u_n-\ell|<\frac{\varepsilon}{2}$. Soit $n$ un entier naturel strictement supérieur à $n_0$.

\begin{align*}
|v_n-\ell|&=\left|\frac{1}{n+1}\sum_{k=0}^{n}u_k-\ell\right|=\left|\frac{1}{n+1}\sum_{k=0}^{n}(u_k-\ell)\right|\\
 &\leq\frac{1}{n+1}\sum_{k=0}^{n}|u_k-\ell|=\frac{1}{n+1}\sum_{k=0}^{n_0}|u_k-\ell|+\frac{1}{n+1}\sum_{k=n_0+1}^{n}|u_k-\ell|\\
 &\leq\frac{1}{n+1}\sum_{k=0}^{n_0}|u_k-\ell|+\frac{1}{n+1}\sum_{k=n_0+1}^{n}\frac{\varepsilon}{2}\leq\frac{1}{n+1}\sum_{k=0}^{n_0}|u_k-\ell|+\frac{1}{n+1}\sum_{k=0}^{n}\frac{\varepsilon}{2}\\
 &=\frac{1}{n+1}\sum_{k=0}^{n_0}|u_k-\ell|+\frac{\varepsilon}{2}
\end{align*}
Maintenant, $\sum_{k=0}^{n_0}|u_k-\ell|$ est une expression constante quand $n$ varie et donc, $\lim_{n\rightarrow +\infty}\frac{1}{n+1}\sum_{k=0}^{n_0}|u_k-\ell|=0$. Par suite, il existe un entier $n_1\geq n_0$ tel que pour $n\geq n_1$, $\frac{1}{n+1}\sum_{k=0}^{n_0}|u_k-\ell|<\frac{\varepsilon}{2}$.
Pour $n\geq n_1$, on a alors $|v_n-\ell|<\frac{\varepsilon}{2}+\frac{\varepsilon}{2}=\varepsilon$.
On a montré que $\forall\varepsilon>0,\;\exists n_1\in\Nn/\;(\forall n\in\Nn)(n\geq n_1\Rightarrow|v_n-\ell|<\varepsilon)$. La suite $(v_n)$ est donc convergente et $\lim_{n\rightarrow +\infty}v_n=\ell$.

\begin{center}
\shadowbox{
Si la suite $u$ converge vers $\ell$ alors la suite $v$ converge vers $\ell$.
}
\end{center}
La réciproque est fausse. Pour $n$ dans $\Nn$, posons $u_n=(-1)^n$. La suite $(u_n)$ est divergente. D'autre part, pour $n$ dans $\Nn$, $\sum_{k=0}^{n}(-1)^k$ vaut $0$ ou $1$ suivant la parité de $n$ et donc, dans tous les cas, $|v_n|\leq\frac{1}{n+1}$. Par suite, la suite $(v_n)$ converge et $\lim_{n\rightarrow +\infty}v_n=0$.}
    \item \question{Montrer que si la suite $(u_n)_{n\in\Nn}$ est bornée, la suite $(v_n)_{n\in\Nn}$ est bornée. Réciproque~?}
\reponse{Si $u$ est bornée, il existe un réel $M$ tel que, pour tout naturel $n$, $|u_n|\leq M$.
Pour $n$ entier naturel donné, on a alors

$$|v_n|\leq\frac{1}{n+1}\sum_{k=0}^{n}|u_k|\leq\frac{1}{n+1}\sum_{k=0}^{n}M=\frac{1}{n+1}(n+1)M=M.$$
La suite $v$ est donc bornée.

\begin{center}
\shadowbox{
Si la suite $u$ est bornée alors la suite $v$ est bornée.
}
\end{center}
La réciproque est fausse. Soit $u$ la suite définie par~:~$\forall n\in\Nn,\;u_n=(-1)^nE\left(\frac{n}{2}\right)=
\left\{
\begin{array}{l}
p\;\mbox{si}\;n=2p,\;p\in\Nn\\
-p\;\mbox{si}\;n=2p+1,\;p\in\Nn
\end{array}
\right.$.
$u$ n'est pas bornée car la suite extraite $(u_{2p})$ tend vers $+\infty$ quand $p$ tend vers $+\infty$. 
Mais, si $n$ est impair, $v_n=0$, et si $n$ est pair, $v_n=\frac{1}{n+1}\times u_n=\frac{n}{2(n+1)}$, et dans tous les cas $|v_n|\leq\frac{1}{n+1}\frac{n}{2}\leq\frac{1}{n+1}\frac{n+1}{2}=\frac{1}{2}$ et la suite $v$ est bornée.}
    \item \question{Montrer que si la suite $(u_n)_{n\in\Nn}$ est croissante alors la suite $(v_n)_{n\in\Nn}$ l'est aussi.}
\reponse{Si $u$ est croissante, pour $n$ entier naturel donné on a~:

\begin{align*}
v_{n+1}-v_n&=\frac{1}{n+2}\sum_{k=0}^{n+1}u_k-\frac{1}{n+1}\sum_{k=0}^{n}u_k=\frac{1}{(n+1)(n+2)}\left((n+1)\sum_{k=0}^{n+1}u_k-(n+2)\sum_{k=0}^{n}u_k\right)\\
 &=\frac{1}{(n+1)(n+2)}\left((n+1)u_{n+1}-\sum_{k=0}^{n}u_k\right)=\frac{1}{(n+1)(n+2)}\sum_{k=0}^{n}(u_{n+1}-u_k)\geq0.
\end{align*}
La suite $v$ est donc croissante.

\begin{center}
\shadowbox{
Si la suite $u$ est croissante alors la suite $v$ est croissante.
}
\end{center}}
\end{enumerate}
}
