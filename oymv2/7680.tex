\uuid{7680}
\auteur{mourougane}
\datecreate{2021-08-11}
\isIndication{false}
\isCorrection{false}
\chapitre{Sous-variété}
\sousChapitre{Sous-variété}

\contenu{
\texte{
On cherche des applications d'un ouvert du plan $\Rr^2$ dans un ouvert de la sphère $S^2$ de $\Rr^3$
qui conservent les longueurs, ou les mesures d'angles, ou les aires.
}
\begin{enumerate}
    \item \question{Traduire chacune de ces trois propriétés à l'aide des formes fondamentales.}
    \item \question{Les coordonnées sphériques donnent-elles une application 
qui conserve les longueurs, ou les mesures d'angles, ou les aires ?}
    \item \question{La projection stéréographique depuis le pôle nord 
conserve-t-elle les longueurs, ou les mesures d'angles, ou les aires ?}
    \item \question{L'application du cylindre dans la sphère
$$\begin{array}{ccc}
  ]0,2\pi[\times ]-1,1[&\to& S^2\\ (\theta, h)&\mapsto&
\left( \begin{array}{c}\sqrt{1-h^2} \cos\theta\\ \sqrt{1-h^2} \sin\theta\\ h\end{array}\right)
 \end{array}$$
conserve-t-elle les longueurs, ou les mesures d'angles, ou les aires ?}
    \item \question{L'application du cylindre dans la sphère
$$\begin{array}{ccc}
  ]0,2\pi[\times ]-\infty,+\infty[&\to& S^2\\ (\theta, x)&\mapsto&
\left( \begin{array}{c}\sqrt{1-h^2(x)} \cos\theta\\ \sqrt{1-h^2(x)} \sin\theta\\ h(x)\end{array}\right)
 \end{array}$$
avec $h(x)=\tanh (x)$ conserve-t-elle les longueurs, ou les mesures d'angles, ou les aires ?}
\end{enumerate}
}
