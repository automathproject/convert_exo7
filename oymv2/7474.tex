\uuid{7474}
\auteur{exo7}
\datecreate{2021-08-10}
\isIndication{false}
\isCorrection{false}
\chapitre{Géométrie affine euclidienne}
\sousChapitre{Géométrie affine euclidienne du plan}

\contenu{
\texte{
Soit $E$ un plan affine euclidien muni d'un repère cartésien 
orthonormé. Soient
$A$, $B$, $C$ et $D$ les points de $E$ dont les coordonnées sont
$$A : (0,3), \quad B : (2,1), \quad C : (2,3) \quad \text{ et } \quad D 
: (0,1).$$
}
\begin{enumerate}
    \item \question{Montrer que les droites $(A,B)$ et $(C,D)$ sont 
orthogonales et expliciter les coordonnées de leur
point d'intersection.}
    \item \question{Prouver l'existence d'une {\it {rotation}} qui envoie $A$ 
sur $C$, $C$ sur $B$, $B$ sur $D$ et $D$
sur $A$. Expliciter une représentation matricielle de cette rotation.}
\end{enumerate}
}
