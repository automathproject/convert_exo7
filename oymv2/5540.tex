\uuid{5540}
\auteur{rouget}
\datecreate{2010-07-15}

\contenu{
\texte{
Le plan est rapporté à un repère orthonormé $\mathcal{R}=(0,\overrightarrow{i},\overrightarrow{j})$.
Eléménts caractéristiques de la conique dont une équation cartésienne dans $\mathcal{R}$ est

\begin{itemize}
\item
}
\begin{enumerate}
    \item \question{$y^2=x$,}
    \item \question{$y^2=-x$,}
    \item \question{$y=x^2$,}
    \item \question{$y=-x^2$.}
\reponse{
\begin{enumerate}
$\mathcal{C}$ est la parabole de sommet $O$, d'axe focal $(Ox)$, de paramètre $p=\frac{1}{2}$ tournée vers
les $x$ positifs. Son foyer est le point $F\left(\frac{1}{4},0\right)$ et sa directrice est $\mathcal{D}~:~x=-\frac{1}{4}$.
$\mathcal{C}$ est la parabole de sommet $O$, d'axe focal $(Ox)$, de paramètre $p=\frac{1}{2}$ tournée vers
les $x$ négatifs. Son foyer est le point $F\left(-\frac{1}{4},0\right)$ et sa directrice est $\mathcal{D}~:~x=\frac{1}{4}$.
$\mathcal{C}$ est la parabole de sommet $O$, d'axe focal $(Oy)$, de paramètre $p=\frac{1}{2}$ tournée vers
les $y$ positifs. Son foyer est le point $F\left(0,\frac{1}{4}\right)$ et sa directrice est $\mathcal{D}~:~y=-\frac{1}{4}$.
$\mathcal{C}$ est la parabole de sommet $O$, d'axe focal $(Oy)$, de paramètre $p=\frac{1}{2}$ tournée vers
les $y$ négatifs. Son foyer est le point $F\left(0,-\frac{1}{4}\right)$ et sa directrice est $\mathcal{D}~:~y=\frac{1}{4}$.
}
\end{enumerate}
}
