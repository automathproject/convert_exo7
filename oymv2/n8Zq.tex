\uuid{n8Zq}
\exo7id{5248}
\auteur{rouget}
\datecreate{2010-07-04}
\isIndication{false}
\isCorrection{true}
\chapitre{Suite}
\sousChapitre{Autre}

\contenu{
\texte{
Montrer que l'ensemble $E$ des réels de la forme $u_n=\sin(\ln(n))$, $n$ entier naturel non nul, est dense dans 
$[-1,1]$.
}
\reponse{
Soit $x$ dans $[-1,1]$ et $\varepsilon>0$.

Soit $\theta=\Arcsin x$ (donc $\theta$ est élément de $[-\frac{\pi}{2},\frac{\pi}{2}]$ et $x=\sin\theta$). Pour $k$ entier naturel non nul donné, il existe un entier $n_k$ tel que $\ln(n_k)\leq \theta+2k\pi<\ln(n_k+1)$ à savoir $n_k=E(e^{\theta+2k\pi})$.

Mais,

$$0<\ln(n_k+1)-\ln(n_k)=\ln(1+\frac{1}{n_k})<\frac{1}{n_k}$$

(d'après l'inégalité classique $\ln(1+x)<x$ pour $x>0$, obtenue par exemple par l'étude de la fonction $f~:~x\mapsto\ln(1+x)-x$).
Donc,

$$0\leq\theta+2k\pi-\ln(n_k)<\ln(n_k+1)-\ln(n_k)<\frac{1}{n_k},$$

puis 

\begin{align*}\ensuremath
|\sin(\theta)-\sin(\ln(n_k))|&=2|\sin(\frac{\theta+2k\pi-\ln(n_k)}{2})\cos(\frac{\theta+2k\pi+\ln(n_k)}{2})|\\
 &\leq2\left|\frac{\theta+2k\pi-\ln(n_k)}{2}\right|=|\theta+2k\pi-\ln(n_k)|<\frac{1}{n_k}.
\end{align*}

Soit alors $\varepsilon$ un réel strictement positif.

Puisque $n_k=E(e^{\theta+2k\pi})$ tend vers $+\infty$ quand $k$ tend vers $+\infty$, on peut trouver un entier $k$ tel que  $\frac{1}{n_k}<\varepsilon$ et pour cet entier $k$, on a $|\sin\theta-\sin(\ln(n_k))|<\varepsilon$.

On a montré que $\forall x\in[-1,1],\;\forall\varepsilon>0,\;\exists n\in\Nn^*/\;|x-\sin(\ln n)|<\varepsilon$, et donc $\{\sin(\ln n),\;n\in\Nn^*\}$ est dense dans $[-1,1]$.
}
}
