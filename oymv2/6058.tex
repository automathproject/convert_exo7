\uuid{6058}
\auteur{queffelec}
\datecreate{2011-10-16}
\isIndication{false}
\isCorrection{false}
\chapitre{Espace vectoriel normé}
\sousChapitre{Espace vectoriel normé}

\contenu{
\texte{

}
\begin{enumerate}
    \item \question{On considère dans ${\Rr}^2$ les 4 boules euclidiennes fermées de rayon $1$
centrées  aux points $(1,0), (-1,0), (0,1), (0,-1)$; $A$ leur réunion contient
$0$ comme point intérieur. Trouver le rayon de la plus grande boule ouverte
centrée en $0$ et contenue dans $A$.}
    \item \question{On se pose plus généralement le problème dans ${\Rr}^n$ : $A$ désigne
l'union $\cup_j\overline B(e_j,1)\cup_j\overline B(-e_j,1)$ où $(e_j)$ est la
base canonique de ${\Rr}^n$. Montrer que $x\in A$ si et seulement si $\Vert
x\Vert_2^2\leq 2\Vert x\Vert_\infty$. En déduire que le rayon de la plus grande
boule ouverte centrée en $0$ et contenue dans $A$ est ${2\over\sqrt n}$.}
\end{enumerate}
}
