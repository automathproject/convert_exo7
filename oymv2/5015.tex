\uuid{5015}
\auteur{quercia}
\datecreate{2010-03-17}
\isIndication{false}
\isCorrection{true}
\chapitre{Courbes planes}
\sousChapitre{Enveloppes}

\contenu{
\texte{
Soit $\cal P$ une parabole, $M \in \cal P$ et $\cal C$ le cercle osculateur à
$\cal P$ en $M$.
Montrer que, sauf cas particulier, $\cal C$ recoupe $\cal P$ en un deuxième
point $P$.
Déterminer l'enveloppe des droites $(MP)$.
}
\reponse{
$M = (t^2/2p,t)  \Rightarrow  I = (3t^2/2p+p,-t^3/p^2)$.
         Soit $P = (u^2/2p,u)$ :

         $IP=IM \Leftrightarrow (u-t)^3(u+3t) = 0  \Rightarrow  u = -3t$.

         Enveloppe : $\begin{cases}x = -3t^2/2p\cr y = 3t.\cr\end{cases}$ (Parabole)
}
}
