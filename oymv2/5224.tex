\uuid{5224}
\auteur{rouget}
\datecreate{2010-06-30}

\contenu{
\texte{
Calculer $\lim_{n\rightarrow +\infty}\sum_{k=1}^{n}\frac{1}{1^2+2^2+...+k^2}$.
}
\reponse{
Soit $k$ un entier naturel non nul. On sait que $\sum_{i=1}^{k}i^2=\frac{k(k+1)(2k+1)}{6}$. Déterminons alors trois réels $a$, $b$ et $c$ tels que, pour entier naturel non nul $k$, 

$$\frac{6}{k(k+1)(2k+1)}=\frac{a}{k}+\frac{b}{k+1}+\frac{c}{2k+1}\;(*).$$
Pour $k$ entier naturel non nul donné,

$$\frac{a}{k}+\frac{b}{k+1}+\frac{c}{2k+1}=\frac{a(k+1)(2k+1)+bk(2k+1)+ck(k+1)}{k(k+1)(2k+1)}=
\frac{(2a+2b+c)k^2+(3a+b+c)k+a}{k(k+1)(2k+1)}.$$
Par suite,

$$(*)\Leftarrow\left\{
\begin{array}{l}
2a+2b+c=0\\
3a+b+c=0\\
a=6
\end{array}
\right.\Leftrightarrow\left\{
\begin{array}{l}
a=6\\
b=6\\
c=-24
\end{array}
\right.,$$
et donc,

$$\forall n\in\Nn^*,\;\sum_{k=1}^{n}\frac{6}{k(k+1)(2k+1)}=6\left(\sum_{k=1}^{n}\frac{1}{k}+\sum_{k=1}^{n}\frac{1}{k+1}-4\sum_{k=1}^{n}\frac{1}{2k+1}\right).$$
Ensuite, d'après l'exercice \ref{exo:suprou3bis}, quand $n$ tend vers $+\infty$, $\sum_{k=1}^{n}\frac{1}{k}=\ln n+\gamma+o(1)$ puis 

$$\sum_{k=1}^{n}\frac{1}{k+1}=\sum_{k=2}^{n+1}\frac{1}{k}=H_{n+1}-1=-1+\ln(n+1)+\gamma+o(1)=\ln n+\ln\left(1+\frac{1}{n}\right)+\gamma-1+o(1)=\ln n+\gamma-1+o(1).$$
Enfin,
  
\begin{align*}
\sum_{k=1}^{n}\frac{1}{2k+1}&=-1+\sum_{k=1}^{2n+1}\frac{1}{k}-\sum_{k=1}^{n}\frac{1}{2k}=-1+H_{2n+1}-\frac{1}{2}H_n\\
 &=\ln(2n+1)+\gamma-\frac{1}{2}(\ln n+\gamma)-1+o(1)=\ln2+\ln n+\ln\left(1+\frac{1}{2n}\right)+\gamma-\frac{1}{2}\ln n-\frac{1}{2}\gamma-1+o(1)\\
 &=\frac{1}{2}\ln n+\ln2+\frac{1}{2}\gamma-1+o(1)
\end{align*}
Finalement, quand $n$ tend vers $+\infty$, on a

$$\sum_{k=1}^{n}\frac{1}{1^2+2^2+...+k^2}=6\left(\ln n+\gamma+\ln n+\gamma-1-4\left(\frac{1}{2}\ln n+\ln2+\frac{1}{2}\gamma-1\right)\right)=6(3-4\ln2)+o(1).$$
Donc,

\begin{center}
\shadowbox{
$\lim_{n\rightarrow +\infty}\sum_{k=1}^{n}\frac{1}{1^2+2^2+...+k^2}=6(3-4\ln2)$.
}
\end{center}
}
}
