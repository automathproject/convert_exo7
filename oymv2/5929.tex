\uuid{5929}
\auteur{tumpach}
\datecreate{2010-11-11}
\isIndication{false}
\isCorrection{true}
\chapitre{Autre}
\sousChapitre{Autre}

\contenu{
\texte{
On d\'efinit la mesure ext\'erieure de Lebesgue sur $\mathbb{R}$,
$~m_{*}: \mathcal{P}(\mathbb{R}) \rightarrow \mathbb{R},~$ par la
formule
$$
m_{*}(A) = \inf \left\{\sum_{i-1}^{\infty} (b_{i} - a_{i}) ~|~ A
\subset \bigcup_{i=1}^{\infty} ]a_{i}, b_{i}[ \right\}.
$$
Montrer qu'il s'agit bien d'une mesure ext\'erieure.
}
\reponse{
Il est clair que $m_{*}(\emptyset)=0$ et que si $A\subset B\subset
\mathbb{R}$, alors $m_{*}(A)\leq m_{*}(B)$, il faut donc
uniquement d\'emontrer que $m_{*}$ est $\sigma$-sous-additive.

Soit $\{A_n\}_{n\in \mathbb{N}}\subset \mathcal{P}(\mathbb{R})$,
fixons $\varepsilon>0$ et notons $A=\bigcup\limits_{n=1}^\infty
A_n$. Par d\'efinition de l'infimum, pour tout $n\in \mathbb{N}$,
on peut trouver une suite $\{(a_i^n,b_i^n)\}$ telle que
$A_n\subset \bigcup\limits_{i=1}^\infty ]a_i^n,b_i^n[$ et
$$
\sum\limits_{i=1}^\infty (b_i^n-a_i^n)\leq
m_{*}(A_n)+\frac{\varepsilon}{2^n}
$$
Comme  $A\subset
\bigcup\limits_{i,n} ]a_i^n,b_i^n[$, on  a
$$
m_{*}(A)\leq\sum\limits_{n,i} (b_i^n-a_i^n)\leq \sum\limits_{n=1}^\infty\sum\limits_{i=1}^\infty (b_i^n-a_i^n)
\leq
\sum\limits_{n=1}^\infty(m_{*}(A_n)+\frac{\varepsilon}{2^n})=\varepsilon + \sum\limits_{n=1}^\infty
m_{*}(A_n).
$$
On a donc la $\sigma$-sous-additivit\'e\;
$m_{*}(A)\leq\sum\limits_{n=1}^\infty m_{*}(A_n)$ \;puisque
$\varepsilon$ est arbitraire.
}
}
