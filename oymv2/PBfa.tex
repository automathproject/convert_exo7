\uuid{PBfa}
\exo7id{1326}
\auteur{ortiz}
\datecreate{1999-04-01}
\isIndication{false}
\isCorrection{true}
\chapitre{Groupe, anneau, corps}
\sousChapitre{Groupe, sous-groupe}

\contenu{
\texte{
Soit $G$ un groupe engendr\'e par $a$ et $b.$
Montrer que $<a>\cap<b>\subseteq Z(G)$ o\`u $Z(G)$ d\'esigne le
centre de $G.$
}
\reponse{
Soit $G = \langle a,b \rangle$, tout élément $g$ de $G$ S'écrit $g
= a^{\alpha_1}b^{\beta_1}  a^{\alpha_2}b^{\beta_2}\ldots
a^{\alpha_n}b^{\beta_n}$ avec $\alpha_i,\beta_i \in \Zz$. Si $h
\in \langle a \rangle \cap \langle b \rangle$, alors en
particulier $h \in \langle a \rangle$ et $h = a^\mu$ avec $\mu \in
\Zz$, donc $h$ commute avec $a^{\alpha_i}$ pour tout $\alpha_i$
dans $\Zz$ (en effet $a^{\alpha_i}a^{\mu} = a^{\alpha_i+\mu}=
a^\mu a^{\alpha_i}$. De même $h \in \langle b \rangle$ donc $h$
s'écrit également $h=b^\nu$ ($\nu \in \Zz$) et $h$ commute avec
$b^{\beta_i}$. Donc $hg = (h a^{\alpha_1})b^{\beta_1}\ldots =
(a^{\alpha_1}h)b^{\beta_1}\ldots = a^{\alpha_1}(h
b^{\beta_1})\ldots =  a^{\alpha_1}(b^{\beta_1}h)\ldots =\cdots$
Finalement $hg = a^{\alpha_1}b^{\beta_1}\ldots
a^{\alpha_n}b^{\beta_n}h = gh$. Ainsi $h$ commute  avec tout
élément de $G$ et appartient ainsi au centre de $G$.
}
}
