\uuid{EKVC}
\exo7id{4134}
\auteur{quercia}
\datecreate{2010-03-11}
\isIndication{false}
\isCorrection{true}
\chapitre{Equation différentielle}
\sousChapitre{Equations différentielles non linéaires}

\contenu{
\texte{
Soit $J$ un intervalle de~$\R$ et $f : {J\times \R^n} \to {\R^n}$ continue.
On suppose qu'il existe $a,b$ continues de $J$ dans $\R^+$ telles que,
pour tous $t,y$~: $(f(t,y)\mid y) \le a(t)\|y\|^2 + b(t)$.
Montrer que toute solution maximale de $y' = f(t,y)$ est définie sur $J$ entier.
}
\reponse{
Remarque~: la seule continuité de~$f$ implique l'existence d'une solution
maximale à condition initiale donnée (thm. de Cauchy-Arzela, HP), mais pas son unicité.

thm des bouts~: supposons $y$ solution, définie sur $[t_0,\alpha[$ avec $\alpha < \sup J$.

On a $\frac{d}{d t}(\|y\|^2) = 2(y'\mid y) = 2(f(t,y)\mid y) \le 2a\|y\|^2 + 2b$,
ce que l'on écrit $z' = 2az + 2b-c$ avec $z=\|y\|^2$ et $c$ fonction continue positive.
Donc $z(t) = \exp(2A(t)-2A(t_0))z(t_0) +  \int_{s=t_0}^t\exp(2A(t)-2A(s))(2b(s)-c(s))\,d s$
où $A$ est une primitive de $a$ sur $J$. On en déduit que $z$ est majorée
sur $[t_0,\alpha[$ car $A$ et $b$ sont continues sur $[t_0,\alpha]$ et $c\ge 0$, donc
$\|y'\| = \|f(t,y)\|$ est aussi majorée, et $ \int_{s=t_0}^\alpha y'(s)d s$
est absolument convergente. Ainsi $y$ admet une limite finie en $\alpha^{-}$, et
l'on peut prolonger $y$ au dela de $\alpha$ avec le thm de Cauchy-Arzela~; $y$ n'est
pas maximale.
}
}
