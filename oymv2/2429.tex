\uuid{2429}
\auteur{matexo1}
\datecreate{2002-02-01}

\contenu{
\texte{
Soit $E = \R[X]_n$ (polyn\^omes de degr\'e $\le n$), et $P
\in E$.
}
\begin{enumerate}
    \item \question{Montrer que l'ensemble $F_P$ des polyn\^omes de
$E$ multiples de $P$ est un sous-espace vectoriel de $E$. Quelle en
est la dimension en fonction du degr\'e de $P$\,?}
    \item \question{Soit $Q \in E$ un polyn\^ome sans racine commune avec
$P$, et tel que $\deg P + \deg Q = n+1$. Montrer que $E = F_P
\oplus F_Q$.}
    \item \question{En d\'eduire qu'il existe deux polyn\^omes $U$ et $V$
tels que $U P + V Q = 1$.}
\end{enumerate}
}
