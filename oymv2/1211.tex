\uuid{yrbf}
\exo7id{1211}
\auteur{cousquer}
\datecreate{2003-10-01}
\isIndication{false}
\isCorrection{false}
\chapitre{Suite}
\sousChapitre{Suite de Cauchy}

\contenu{
\texte{
Une suite $(x_n)$ est définie par une relation de récurrence
$x_{n+1}=a\sin x_n+b$ où $a$ est un nombre réel de $\mathopen]0,1\mathclose[$ et
$b$ un nombre réel quelconque. Montrer que
pour tout $p\in\mathbf{N}$, $\vert x_{p+1}-x_p\vert\leq a^p\vert x_1-x_0\vert$.
En déduire que la suite $(x_n)$ est une suite de Cauchy.

Combien de termes faut-il calculer pour obtenir une valeur approchée
de $\lim x_n$ à $10^{-10}$ près si on suppose $a=1/2$, $b=5$, $x_0=1$~?
}
}
