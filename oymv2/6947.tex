\uuid{ukzB}
\exo7id{6947}
\auteur{ruette}
\datecreate{2013-01-24}
\isIndication{false}
\isCorrection{true}
\chapitre{Loi, indépendance, loi conditionnelle}
\sousChapitre{Loi, indépendance, loi conditionnelle}

\contenu{
\texte{
Soit $X$ une variable aléatoire de loi de Poisson $\mathcal{P}(\lambda)$.
}
\begin{enumerate}
    \item \question{Calculer la fonction génératrice de $X$.}
\reponse{$G_X(z)=e^{\lambda(z-1)}$.}
    \item \question{Calculer $E(X)$ et $\text{Var}(X)$.}
\reponse{$E(X)=\text{Var}(X)=\lambda$}
    \item \question{Soit $Y$ une variable aléatoire de loi de Poisson $\mathcal{P}(\lambda')$, indépendante
de $X$. Quelle est la loi de $X+Y$ ? En déduire que
$\mathcal{P}(\lambda) *\mathcal{P}(\lambda')=\mathcal{P}(\lambda+\lambda')$.}
\reponse{$G_{X+Y}(z)=G_X(z)G_Y(z)=e^{(\lambda+\lambda')(z-1)}$.
D'où $X+Y\sim \mathcal{P}(\lambda+\lambda')$, et donc
$\mathcal{P}(\lambda) *\mathcal{P}(\lambda')=\mathcal{P}(\lambda+\lambda')$.}
\end{enumerate}
}
