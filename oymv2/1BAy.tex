\uuid{1BAy}
\exo7id{1145}
\auteur{barraud}
\datecreate{2003-09-01}
\isIndication{true}
\isCorrection{true}
\chapitre{Déterminant, système linéaire}
\sousChapitre{Calcul de déterminants}

\contenu{
\texte{
Soit $a$ un réel.
On note $\Delta_n$ le déterminant suivant : 
$$
\Delta_n = 
\left\vert
\begin{matrix}
       a   &    0   & \cdots & 0      & n-1 \\
       0   &    a   & \ddots & \vdots & \vdots \\
    \vdots & \ddots & \ddots & 0      & 2 \\
       0   & \cdots &   0    & a      & 1 \\
      n-1  & \cdots &   2    & 1      & a
\end{matrix}
\right\vert
$$
}
\begin{enumerate}
    \item \question{Calculer $\Delta_n$ en fonction de $\Delta_{n-1}$.}
    \item \question{Démontrer que : $\displaystyle \forall n\geq2\quad
\Delta_n=a^n-a^{n-2}\sum_{i=1}^{n-1}{i^2}$.}
\reponse{
En développant par rapport à la première colonne on trouve la relation suivante :

$$\Delta_n = a \Delta_{n-1} + (-1)^{n-1}(n-1) 
\left\vert
\begin{matrix}
       0   &    0   & \cdots & 0      & n-1 \\
       a   &    0   & \ddots & \vdots & \vdots \\
    \vdots & \ddots & \ddots & 0      & 3 \\
       0   & \cdots &   a    & 0      & 2 \\
       0  & \cdots &   0    & a      & 1
\end{matrix}
\right\vert
$$
Notons $\delta$ ce dernier déterminant (dont la matrice est de taille $n-1\times n-1$). 
On le calcule en développant par rapport à la première ligne 
$$\delta  = (-1)^{n-2}(n-1)
\left\vert
\begin{matrix}
       a   &    0   & \cdots      & 0 \\
       0   &    a   & \ddots      & \vdots \\
    \vdots & \ddots & \ddots      & 0  \\
       0   & \cdots &   0         & a
\end{matrix}\right\vert
= (-1)^{n-2}(n-1) a^{n-2}.
$$
Donc 
$$\Delta_n = a \Delta_{n-1} - a^{n-2}(n-1)^2.$$
Prouvons la formule 
$$\Delta_n=a^n-a^{n-2}\sum_{i=1}^{n-1}{i^2}$$
par récurrence sur $n\ge 2$.

\begin{itemize}
\textbf{Initialisation.} Pour $n=2$,
$\Delta_2=\begin{vmatrix}a&1\\1&a\end{vmatrix}=a^2-1$
donc la formule est vraie.
\textbf{Hérédité.} Supposons la formule vraie vraie au rang $n-1$,
c'est-à-dire $\Delta_{n-1}=a^{n-1}-a^{n-3}\sum_{i=1}^{n-2}{i^2}$.
Calculons $\Delta_n$ :
\begin{align*}
\Delta_n 
  & = a \Delta_{n-1} - a^{n-2}(n-1)^2  \quad \text{ par la première question } \\
  & = a\Big(a^{n-1}-a^{n-3}\sum_{i=1}^{n-2}{i^2} \Big) - a^{n-2}(n-1)^2 \quad \text{ par l'hypothèse de récurrence} \\
  & = a^n -  a^{n-2}\sum_{i=1}^{n-2}{i^2} - a^{n-2}(n-1)^2 \\
  & = a^n-a^{n-2}\sum_{i=1}^{n-1}{i^2}
\end{align*}
La formule est donc vraie au rang $n$.
\textbf{Conclusion.} Par le principe de récurrence la formule est vraie
pour tout entier $n\ge 2$.
\end{itemize}
}
\indication{Développer par rapport à la première colonne
pour obtenir $\Delta_{n-1}$ et un autre déterminant facile à calculer
en développant par rapport à sa première ligne.}
\end{enumerate}
}
