\uuid{3350}
\auteur{quercia}
\datecreate{2010-03-09}
\isIndication{false}
\isCorrection{true}
\chapitre{Application linéaire}
\sousChapitre{Image et noyau, théorème du rang}

\contenu{
\texte{
Soit $f \in \mathcal{L}(E)$.
On pose $K = \mathrm{Ker} f$, $I = \Im f$,
${\cal K} = \{ g \in \mathcal{L}(E) \text{ tq } f\circ g = 0 \}$ et
${\cal I} = \{ g \in \mathcal{L}(E) \text{ tq } g\circ f = 0 \}$.
}
\begin{enumerate}
    \item \question{Montrer que ${\cal K}$ et ${\cal I}$ sont des sev de $\mathcal{L}(E)$.}
    \item \question{Soit $g \in \mathcal{L}(E)$.
    Montrer que : $g \in {\cal K} \iff \Im g \subset K$,
    et  : $g \in {\cal I} \iff \mathrm{Ker} g \supset I$.}
    \item \question{\begin{enumerate}}
    \item \question{Montrer que l' application
        $\Phi : {{\cal K}} \to {\mathcal{L}({E,K})}, g \mapsto {g^{|K}}$ est un isomorphisme
        d'ev. En déduire $\dim {\cal K}$.}
    \item \question{Chercher de même $\dim {\cal I}$ en introduisant un supplémentaire $I'$
        de $I$.}
    \item \question{Chercher aussi $\dim({\cal K} \cap {\cal I})$.}
\reponse{
3. $\dim {\cal K} =(\dim E)(\dim \mathrm{Ker} f) = \dim {\cal I}$, \quad
         $\dim({\cal K} \cap {\cal I})= (\mathrm{rg} f)^2$.
}
\end{enumerate}
}
