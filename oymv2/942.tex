\uuid{942}
\auteur{legall}
\datecreate{1998-09-01}
\isIndication{false}
\isCorrection{false}
\chapitre{Application linéaire}
\sousChapitre{Image et noyau, théorème du rang}

\contenu{
\texte{
Pour des applications lin\'eaires $f: E\rightarrow F$, $g:
F\rightarrow G$, \'etablir l'\'equivalence
$$g\circ f=0\Longleftrightarrow \text{Im} f\subset \text{Ker} g.$$

Soit $f$ un endomorphisme d'un e.v. $E$, v\'erifiant l'identit\'e
$f^2+f-2i_E=0$. Etablir
$\text{Im} (f-i_E)\subset \text{Ker} (f+2i_E)$;  $\;\;\text{Im} (f+2i_E)\subset \text{Ker}
(f-i_E)$;  $\;\;E=\text{Ker}(f-i_E)\oplus \text{Ker}(f+2i_E)$.
}
}
