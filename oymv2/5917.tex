\uuid{wRv5}
\exo7id{5917}
\auteur{tumpach}
\datecreate{2010-11-11}
\isIndication{false}
\isCorrection{true}
\chapitre{Calcul d'intégrales}
\sousChapitre{Théorie}

\contenu{
\texte{
En utilisant la d\'efinition d'une fonction int\'egrable au sens
de Riemann, montrer les propri\'et\'es suivantes~:
}
\begin{enumerate}
    \item \question{Si $f$ et $g$ sont Riemann-int\'egrables sur $[a,b]$, alors
 $f + g$ est Riemann-int\'egrable sur $[a,b]$.}
\reponse{Soit $\varepsilon>0$ donn\'e. Puisque $f$ est
Riemann-int\'egrable sur $[a,b]$, il existe une subdivision
$\sigma_{1}=\{a_{0}=a< a_{1}< \dots< a_{n}=b\}$ de $[a, b]$ telle
que
$\overline{S}_{f}^{\sigma_{1}}\leq\underline{S}_{f}^{\sigma_{1}} +
\frac{\varepsilon}{2}$. Puisque $g$ est Riemann-int\'egrable sur
$[a,b]$, il existe une subdivision $\sigma_{2}= \{b_{0}=a< b_{1}<
\dots< b_{p}=b\}$ de $[a, b]$ telle que
$\overline{S}_{g}^{\sigma_{2}}\leq\underline{S}_{g}^{\sigma_{2}} +
\frac{\varepsilon}{2}$. On note $\sigma_{1}\cup\sigma_2=\{c_{0}=a<
c_{1} < \dots < c_{q-1} < c_{q} = b\}$ la subdivision de $[a,b]$
obtenue en ordonnant l'ensemble $\{a_{0}, \dots, a_{n}, b_{0},
\dots, b_{n}\}$ par ordre croissant, puis en identifiant les
points qui apparaissent plusieurs fois (on obtient une subdivision
de $[a,b]$ en $q$ intervalles avec  $max\{n, p\}\leq q \leq n+p$).
Puisque $\sigma_{1}\cup\sigma_{2}$ est une subdivision \emph{plus
fine} que $\sigma_{1}$, on a~:
\begin{equation}\label{1}
\overline{S}_{f}^{\sigma_{1}\cup\sigma_{2}}\, \leq\,\,
\overline{S}_{f}^{\sigma_{1}}\quad\quad\text{et}\quad\quad
\underline{S}_{f}^{\sigma_{1}}\leq\,\,
\underline{S}_{f}^{\sigma_{1}\cup\sigma_{2}}.
\end{equation}
De m\^eme,
\begin{equation}\label{2}
\overline{S}_{g}^{\sigma_{1}\cup\sigma_{2}}\, \leq\,\,
\overline{S}_{g}^{\sigma_{2}}\quad\quad\text{et}\quad\quad
\underline{S}_{g}^{\sigma_{2}}\leq\,\,
\underline{S}_{g}^{\sigma_{1}\cup\sigma_{2}}.
\end{equation}
De plus, sur un intervalle $]c_{k-1}, c_{k}[$ donn\'e, on a~:
\begin{eqnarray*}
\sup\{f(x) + g(x), \,x \in \,]c_{k-1}, c_{k}[\} &\leq &
\sup\{f(x), \,x \in \,]c_{k-1}, c_{k}[\} \\& &+ \sup\{g(x), \,x
\in \,]c_{k-1}, c_{k}[\}.
\end{eqnarray*}
De m\^eme~:
\begin{eqnarray*}
\inf\{f(x) + g(x), \,x \in \,]c_{k-1}, c_{k}[\} & \geq &
\inf\{f(x),\,x \in \,]c_{k-1}, c_{k}[\}\\ & & + \inf\{g(x), \,x
\in \,]c_{k-1}, c_{k}[\}.
\end{eqnarray*}
On en d\'eduit que~:
\begin{equation}\label{3}
\overline{S}_{f+g}^{\sigma_{1}\cup\sigma_{2}} \,\leq\,
\overline{S}_{f}^{\sigma_{1}\cup\sigma_{2}}\,+\,\overline{S}_{g}^{\sigma_{1}\cup\sigma_{2}},
\end{equation}
et
\begin{equation}\label{4}
\underline{S}_{f}^{\sigma_{1}\cup\sigma_{2}}\,+\,
\underline{S}_{g}^{\sigma_{1}\cup\sigma_{2}} \,\leq\,
\underline{S}_{f+g}^{\sigma_{1}\cup\sigma_{2}}.
\end{equation}
En utilisant les in\'egalit\'es \eqref{1}, \eqref{2}, \eqref{3} et
\eqref{4}, il vient alors~:
\begin{equation*}
\overline{S}_{f+g}^{\sigma_{1}\cup\sigma_{2}}\,\leq\,
\overline{S}_{f}^{\sigma_{1}}+
\overline{S}_{g}^{\sigma_{2}}\,\leq\,
\underline{S}_{f}^{\sigma_{1}} + \underline{S}_{g}^{\sigma_{2}} +
\varepsilon \leq \underline{S}_{f+g}^{\sigma_{1}\cup\sigma_{2}} +
\varepsilon.
\end{equation*}
D'apr\`es le th\'eor\`eme rappel\'e en introduction, on en d\'eduit que $f+g$ est
Riemann-int\'egrable sur $[a, b]$. De plus, de l'in\'egalit\'e
\begin{equation*}
\underline{S}_{f}^{\sigma_{1}} + \underline{S}_{g}^{\sigma_{2}}
\leq \underline{S}_{f+g}^{\sigma_{1}\cup\sigma_{2}},
\end{equation*}
on d\'eduit que
\begin{equation*}
\sup_{\sigma_1, \sigma_{2}}\left(\underline{S}_{f}^{\sigma_{1}} +
\underline{S}_{g}^{\sigma_{2}} \right) \leq \sup_{\sigma_1,
\sigma_{2}}\underline{S}_{f+g}^{\sigma_{1}\cup\sigma_{2}}.
\end{equation*}
Or $$\sup_{\sigma_1,
\sigma_{2}}\left(\underline{S}_{f}^{\sigma_{1}} +
\underline{S}_{g}^{\sigma_{2}} \right) =
\sup_{\sigma_{1}}\underline{S}_{f}^{\sigma_{1}} +
\sup_{\sigma_{2}}\underline{S}_{g}^{\sigma_{2}} = \int_{a}^{b}
f(x)\,dx + \int_{a}^{b} g(x)\,dx $$ et
$$\sup_{\sigma_1,
\sigma_{2}}\underline{S}_{f+g}^{\sigma_{1}\cup\sigma_{2}} =
\sup_{\sigma}\underline{S}_{f+g}^{\sigma} = \int_{a}^{b}\left(f(x)
+ g(x)\right)\,dx.$$ Ainsi
\begin{equation*}
 \int_{a}^{b}
f(x)\,dx + \int_{a}^{b} g(x)\,dx \leq \int_{a}^{b}\left(f(x) +
g(x)\right)\,dx.
\end{equation*}
De m\^eme, l'in\'egalit\'e
\begin{equation*}
\overline{S}_{f+g}^{\sigma_{1}\cup\sigma_{2}}\,\leq\,
\overline{S}_{f}^{\sigma_{1}}+ \overline{S}_{g}^{\sigma_{2}}
\end{equation*}
implique $ \int_{a}^{b}\left(f(x) + g(x)\right)\,dx \leq
\int_{a}^{b} f(x)\,dx + \int_{a}^{b} g(x)\,dx$. En conclusion,
 $ \int_{a}^{b}\left(f(x) +
g(x)\right)\,dx = \int_{a}^{b} f(x)\,dx + \int_{a}^{b} g(x)\,dx$.}
    \item \question{Si $f$ est Riemann-int\'egrable sur $[a,b]$ et $\lambda \in
\mathbb{R}$, alors $\lambda\,f$ est Riemann-int\'egrable sur
$[a,b]$.}
\reponse{$\cdot$ Pour $\lambda = 0$ il n'y a rien a d\'emontrer. \\
 $\cdot$ Si $f$
est Riemann-int\'egrable sur $[a,b]$ et $\lambda > 0$, alors pour
tout subdivision $\sigma=\{a_{0}=a<\dots<a_{n}=b\}$ de $[a,b]$, on
a:
\begin{eqnarray*}
\inf\{\lambda f(x),\,x \in\,]a_{k-1},a_{k}[\,\} = \lambda
\inf\{f(x),\,x \in\,]a_{k-1},a_{k}[\,\}\\
\sup\{\lambda f(x),\,x \in\,]a_{k-1},a_{k}[\,\} = \lambda
\sup\{f(x),\,x \in\,]a_{k-1},a_{k}[\,\}.
\end{eqnarray*}
Par cons\'equent, $\underline{S}_{\lambda f}^{\sigma} = \lambda
\underline{S}_{f}^{\sigma}$ et $\overline{S}_{\lambda f}^{\sigma}
= \lambda \overline{S}_{f}^{\sigma}$. On en d\'eduit que
\begin{equation*}
\sup_{\sigma}\underline{S}_{\lambda f}^{\sigma} = \lambda
\sup_{\sigma}\underline{S}_{f}^{\sigma} = \lambda \int_{a}^{b}
f(x)\,dx = \lambda \inf_{\sigma}\overline{S}_{f}^{\sigma} =
\inf_{\sigma}\overline{S}_{\lambda f}^{\sigma}.
\end{equation*}
En conclusion, $\lambda f$ est Riemann-int\'egrable et
$\int_{a}^{b} \lambda f(x)\,dx = \lambda \int_{a}^{b} f(x)\,dx$.\\
$\cdot$ Si $f$ est Riemann-int\'egrable sur $[a,b]$ et $\lambda
<0$, alors pour tout subdivision
$\sigma=\{a_{0}=a<\dots<a_{n}=b\}$ de $[a,b]$, on a:
\begin{eqnarray*}
\inf\{\lambda f(x),\,x \in\,]a_{k-1},a_{k}[\,\} = \lambda
\sup\{f(x),\,x \in\,]a_{k-1},a_{k}[\,\}\\
\sup\{\lambda f(x),\,x \in\,]a_{k-1},a_{k}[\,\} = \lambda
\inf\{f(x),\,x \in\,]a_{k-1},a_{k}[\,\}.
\end{eqnarray*}
Par cons\'equent, $\underline{S}_{\lambda f}^{\sigma} = \lambda
\overline{S}_{f}^{\sigma}$ et $\overline{S}_{\lambda f}^{\sigma} =
\lambda \underline{S}_{f}^{\sigma}$. On en d\'eduit que
\begin{equation*}
\sup_{\sigma}\underline{S}_{\lambda f}^{\sigma} = \lambda
\inf_{\sigma}\overline{S}_{f}^{\sigma} = \lambda \int_{a}^{b}
f(x)\,dx = \lambda \sup_{\sigma}\underline{S}_{f}^{\sigma} =
\inf_{\sigma}\overline{S}_{\lambda f}^{\sigma}.
\end{equation*}
En conclusion, $\lambda f$ est Riemann-int\'egrable et
$\int_{a}^{b} \lambda f(x)\,dx = \lambda \int_{a}^{b} f(x)\,dx$.}
    \item \question{Si $f$ et $g$ sont deux fonctions
Riemann-int\'egrables sur $[a,b]$ telles que, pour tout $t\in
[a,b]$, $f(t)\leq g(t)$, alors
$\int_{a}^{b}f(t)\,dt\leq\int_{a}^{b}g(t)\,dt$.}
\reponse{Soient $f$ et $g$  deux fonctions Riemann-int\'egrables sur
$[a,b]$ telles que, pour tout $t\in [a,b]$, $f(t)\leq g(t)$. Soit
$\sigma=\{a_{0}=a< \dots < a_{n}=b\}$ une subdivision de $[a,b]$.
Alors
\begin{equation*}
\inf\{f(x), \, x \in\, ]a_{k-1}, a_{k}[\,\} \leq \inf\{g(x), \, x
\in\, ]a_{k-1}, a_{k}[\,\}.
\end{equation*}
Il en d\'ecoule que
\begin{equation*}
\sup_{\sigma} \underline{S}_{f}^{\sigma} \leq \sup_{\sigma}
\underline{S}_{f}^{\sigma},
\end{equation*}
c'est-\`a-dire $\int_{a}^{b} f(x)\,dx \leq \int_{a}^{b} g(x)\,dx$.}
    \item \question{Une limite
uniforme de fonctions Riemann-int\'egrables sur $[a,b]$ est
Riemann-int\'egrable sur $[a,b]$.}
\reponse{Soit $\{f_{i}\}_{i\in\mathbb{N}}$ une suite de fonctions
Riemann-int\'egrables, qui converge uniform\'ement vers $f$ sur
$[a, b]$. Soit $\varepsilon>0$ donn\'e. Il existe $N>0$ tel que
$\forall i> N$, $\sup_{[a, b]}|f_{i}(t) - f(t)| < \varepsilon$. En
particulier, $f_{i}(t) - \varepsilon < f(t) < f_{i}(t) +
\varepsilon$.  Pour un tel $i$, on en d\'eduit que pour toute
subdivision $\sigma = \{a_{0}= a < \dots < a_{n} = b\}$, on a
\begin{eqnarray*}
\sup_{]a_{k-1}, a_{k}[}f \,\leq \,\sup_{]a_{k-1}, a_{k}[}f_{i} +
\varepsilon\quad \text{et}\quad \inf_{]a_{k-1}, a_{k}[}f \geq
\inf_{]a_{k-1}, a_{k}[}f_{i} - \varepsilon
\end{eqnarray*}
En particulier~:
\begin{eqnarray*}
\sup_{]a_{k-1}, a_{k}[}f - \inf_{]a_{k-1}, a_{k}[}f\,\leq\,
\sup_{]a_{k-1}, a_{k}[}f_{i} - \inf_{]a_{k-1}, a_{k}[}f_{i} +
2\varepsilon.
\end{eqnarray*}
Il en d\'ecoule que~:
\begin{equation*}
\overline{S}_{f}^{\sigma} - \underline{S}_{f}^{\sigma}\, \leq\,
\overline{S}_{f_{i}}^{\sigma} - \underline{S}_{f_{i}}^{\sigma} +
2\varepsilon (b- a).
\end{equation*}
Comme $f_{i}$ est Riemann-int\'egrable, d'apr\`es le th\'eor\`eme
de l'introduction, il existe une subdivision $\sigma$ de $[a, b]$ telle que
$\overline{S}_{f_{i}}^{\sigma} - \underline{S}_{f_{i}}^{\sigma}
\leq \varepsilon$. On en d\'eduit que
\begin{equation*}
\overline{S}_{f}^{\sigma} - \underline{S}_{f}^{\sigma} \leq
\varepsilon \left(1 + 2(b-a)\right),
\end{equation*}
ce qui implique que $f$ est Riemann-int\'egrable.}
\end{enumerate}
}
