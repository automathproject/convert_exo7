\uuid{QIUh}
\exo7id{1444}
\auteur{legall}
\datecreate{1998-09-01}
\isIndication{false}
\isCorrection{false}
\chapitre{Groupe quotient, théorème de Lagrange}
\sousChapitre{Groupe quotient, théorème de Lagrange}

\contenu{
\texte{
Soit $  G  $ un groupe, $  H  $ et $  K  $ deux sous-groupes de $
G  .$ On note $  HK=\{ hk  ;   h\in H   ,   k\in K\}  .$ On suppose
que $  K  $ est distingu\'e dans $ G  .$
}
\begin{enumerate}
    \item \question{Montrer que  $  HK=KH  $ et que $  HK  $ est un sous-groupe de $  G  .$}
    \item \question{Montrer que $  H  $ et $  K  $ sont des sous-groupes de $  KH  $ et que
$  K\cap H  $ est un sous-groupe distingu\'e de $  H  $ et que $  K  $ est distingu\'e
dans $  KH  .$}
    \item \question{Soit $  \varphi : H\rightarrow (HK)/K  $ la restriction \`a $  H  $ de l'application quotient.
Calculer le noyau et l'image de $  \varphi  .$ En d\'eduire que les groupes $  H/(K\cap H)  $ et $  (HK)/K  $ sont isomorphes.}
\end{enumerate}
}
