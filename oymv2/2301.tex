\uuid{2301}
\auteur{barraud}
\datecreate{2008-04-24}

\contenu{
\texte{
Trouver toutes les solutions des syst\`emes suivantes :
}
\begin{enumerate}
    \item \question{$ \quad \begin{cases}
x\equiv 1 \mod 3\\
x \equiv 3 \mod 5\\
x \equiv 4 \mod 7\\
x\equiv 2 \mod 11
\end{cases}
$}
\reponse{$3,5,7,11$ sont deux à deux premiers entre eux, donc la solution est
    unique modulo $1155=3\cdot5\cdot 7\cdot11$.
    
    \begin{align*}
      \begin{cases}        
        x\equiv 1 \mod 3\\
        x \equiv 3 \mod 5\\
        x \equiv 4 \mod 7\\
        x\equiv 2 \mod 11
      \end{cases}
      \Leftrightarrow
      \begin{cases}        
        x\equiv 13 \mod 15\\
        x \equiv 4 \mod 7\\
        x\equiv 2 \mod 11
      \end{cases}
      \Leftrightarrow&
      \begin{cases}        
        x\equiv 88 \mod 105\\
        x\equiv 2 \mod 11
      \end{cases}\\
      \Leftrightarrow&
      \begin{cases}        
        x\equiv 508 \mod 1155
      \end{cases}
    \end{align*}}
    \item \question{$ \quad \begin{cases}
x\equiv 997\mod 2001\\
x \equiv 998 \mod 2002\\
x \equiv 999 \mod 2003
\end{cases}.
$}
\reponse{Un diviseur commun de $2001$ et $2002$ divise leur différence, et
      donc $\pgcd(2001,2002)=1$. De même, $\pgcd(2002,2003)=1$, et comme
      $2\!\!\!\not| 2001$, $\pgcd(2001,2003)=1$.

      $2001,2002,2003$ sont donc deux à deux premiers entre eux, et la
      solution est donc unique modulo $2001\cdot2002\cdot2003$.
    \begin{align*}
      \begin{cases}        
        x \equiv 997 \mod 2001\\
        x \equiv 998 \mod 2002\\
        x \equiv 999 \mod 2003
      \end{cases}
      \Leftrightarrow&
      \begin{cases}        
        x \equiv -1004 \mod 2001\\
        x \equiv -1004 \mod 2002\\
        x \equiv -1004 \mod 2003
      \end{cases}\\
      \Leftrightarrow&
        x \equiv -1004 \mod (2001\cdot2002\cdot2003)
    \end{align*}}
\end{enumerate}
}
