\uuid{bZCU}
\exo7id{3904}
\auteur{quercia}
\datecreate{2010-03-11}
\isIndication{false}
\isCorrection{true}
\chapitre{Continuité, limite et étude de fonctions réelles}
\sousChapitre{Etude de fonctions}

\contenu{
\texte{
Soit ${f} : {\R^{+*}} \to {\R^{+*}}$ telle que : 
$\forall\ x,y>0,\ f(xf(y))=yf(x)$ et $f(x)\to +\infty$ lorsque $x\to 0^+$
}
\begin{enumerate}
    \item \question{Montrer que $f$ est involutive.}
\reponse{Pour $x=1$ on a $f\circ f(y) = yf(1)$ donc $f$ est injective
    et pour $y=1$~: $f(xf(1))=f(x)$ d'où $f(1)=1$.}
    \item \question{Montrer que $f$ conserve le produit. Que peut-on dire de la monotonie de $f$, de sa continuité ?}
\reponse{$f(xy) = f(xf(f(y))) = f(y)f(x)$.\par
    Pour $0<x<1$ on a $f(x^n) = f(x)^n \to +\infty$ (lorsque $n\to\infty$) donc
    $f(x)>1$ ce qui entraîne par morphisme la décroissance de~$f$.
    Enfin $f$ est monotone et $f(]0,+\infty[) = {]0,+\infty[}$ donc $f$
    n'a pas de saut et est continue.}
    \item \question{Trouver $f$.}
\reponse{En tant que morphisme continu, $f$ est de la forme $x \mapsto x^\alpha$
    avec $\alpha\in\R$ et l'involutivité et la décroissance donnent $\alpha=-1$.}
\end{enumerate}
}
