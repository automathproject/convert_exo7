\uuid{1096}
\auteur{cousquer}
\datecreate{2003-10-01}

\contenu{
\texte{
Soit $E$ un espace à $n$ dimensions et $f$ un endomorphisme
de $E$.
}
\begin{enumerate}
    \item \question{Montrer que la condition $f^2=0$ est équivalente à
$\mbox{Im} f \subset\ker f$. Quelle condition vérifie alors le
rang de $f$~? On suppose dans le reste de
l'exercice que $f^2=0$.}
    \item \question{\label{qu2}Soit $E_1$ un supplémentaire de $\ker f$ dans $E$ et soit
$(e_1,e_2,\ldots,e_r)$ une base de $E_1$. Montrer que la famille des
vecteurs $(e_1,e_2,\ldots,e_r,f(e_1),f(e_2),\ldots,f(e_r))$ est libre.
Montrer comment on peut la compléter, si nécessaire, par des vecteurs de
$\ker f$ de façon à obtenir une base de~$E$. Quelle est la matrice
de~$f$ dans cette base~?}
    \item \question{Sous quelle condition nécessaire et suffisante a-t-on
$\mbox{Im} f=\ker f$~?}
    \item \question{Exemple~:
Soit $f$ l'endomorphisme de $\mathbb{R}^3$ dont la matrice dans
la base canonique est
$M(f)=\begin{pmatrix}
    1  & 0  & 1\cr
    2  & 0  & 2\cr
    -1  & 0  &  -1\end{pmatrix}$.
Montrer que $f^2=0$.
Déterminer une nouvelle base dans laquelle la matrice de $f$ a la forme
indiquée dans la question~\ref{qu2}).}
\end{enumerate}
}
