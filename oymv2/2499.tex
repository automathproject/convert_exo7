\uuid{2499}
\auteur{sarkis}
\datecreate{2009-04-01}

\contenu{
\texte{
Soient $||.||_1$ et $||.||_2$
deux normes sur $\mathbb{R}^2$ et $M=\left (
\begin{array}{cc}
a & b \\
c & d
\end{array}
\right )$ une matrice de ${\cal M}_{n,n}(\mathbb{R} \mbox{ ou }
\mathbb{C})$. On d\'efinit la norme de $M$ (ou de l'application
lin\'eaire associ\'ee) de la mani\`ere suivante:
$$||M||=\sup_{X \in S_1(0,1)}||M.X||_2$$
o\`u $S_1(0,1)$ est la sph\`ere unit\'e pour la norme $||.||_1$.
Dans chacun des cas suivant, calculez la norme de $M$.
}
\begin{enumerate}
    \item \question{$||(x,y)||_1=||(x,y)||_2=sup(|x|,|y|).$}
    \item \question{$||(x,y)||_1=||(x,y)||_2=\sqrt{x^2+y^2}.$}
    \item \question{$||(x,y)||_1=\sqrt{x^2+y^2}$ et $||(x,y)||_2=sup(|x|,|y|).$}
\reponse{
Soit $X=(x,y)$, on a $M.X=(ax+by, cx+dy)$ or $$|ax+by|\leq
|ax|+|by|\leq (|a|+|b|)\sup(|x|,|y|) \leq (|a|+|b|)||(x,y)||_1.$$
de m\^eme,
$$|cx+dy| \leq (|c|+|d|)||(x,y)||_1.$$
Par cons\'equent $$||M.X||_2 \leq \sup(|a|+|b|,|c|+|d|)
||(x,y)||_1$$ et donc $$||M|| \leq \sup(|a|+|b|,|c|+|d|).$$
Supposons $|a|+|b| \geq |c|+|d|$ (inverser l'ordre sinon) et
prenons $X_0=(a/|a|,b/|b|)$ (on suppose $a \neq 0$ et $b \neq 0$
sinon v\'erification facile). On a alors $||X_0||=1$ et
$$||M.X_0||_2=\sup(|a|+|b|,|ca/|a|+db/|b||) \geq |a|+|b|.1 \geq
(|a|+|b|)||X_0||_1$$ et donc $$||M|| \geq \sup(|a|+|b|,|c|+|d|)$$
et finalement $$||M||= \sup(|a|+|b|,|c|+|d|)$$
}
\end{enumerate}
}
