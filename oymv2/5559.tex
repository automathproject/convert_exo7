\uuid{5559}
\auteur{rouget}
\datecreate{2010-07-15}
\isIndication{false}
\isCorrection{true}
\chapitre{Fonction de plusieurs variables}
\sousChapitre{Extremums locaux}

\contenu{
\texte{
Maximum du produit des distances aux cotés d'un triangle $ABC$ du plan d'un point $M$ intérieur à ce triangle (on admettra que ce maximum existe).
}
\reponse{
Soit $M$ un point intérieur au triangle $ABC$. On pose $a=BC$, $b=CA$ et $c=AB$. On note $x$, $y$, $z$ et $\mathcal{A}$ les aires respectives des triangles $MBC$, $MCA$, $MAB$ et $ABC$.
On a

\begin{center}
$d(M,(BC))d(M,(CA))d(M(AB))=\frac{2\text{aire}(MBC)}{a}\frac{2\text{aire}(MCA)}{b}\frac{2\text{aire}(MAB)}{c}=\frac{8xyz}{abc}=\frac{8}{abc}xy(\mathcal{A}-x-y)$.
\end{center}
On doit donc déterminer le maximum de la fonction $f(x,y)=xy(\mathcal{A}-x-y)$ quand $(x,y)$ décrit le triangle ouvert $T=\{(x,y)\in\Rr^2,\;x>0,\;y>0,\;x+y<\mathcal{A}\}$. On admet que $f$ admet un maximum global sur le triangle fermé $T'=\{(x,y)\in\Rr^2,\;x\geqslant0,\;y\geqslant0,\;x+y\leqslant\mathcal{A}\}$ (cela résulte d'un théorème de math Spé : \og une fonction numérique continue sur un compact admet un minimum et un maximum \fg). Ce maximum est atteint dans l'intérieur $T$ de $T'$ car $f$ est nulle au bord de $T'$ et strictement positive à l'intérieur de $T'$.

Puisque $f$ est de classe $C^1$ sur $T$ qui est un ouvert de $\Rr^2$, $f$ atteint son maximum sur $T$ en un point critique de $f$. Or, pour $(x,y)\in T^2$,

\begin{align*}\ensuremath
\left\{
\begin{array}{l}
\frac{\partial f}{\partial x}(x,y)=0\\
\rule{0mm}{6mm}\frac{\partial f}{\partial y}(x,y)=0
\end{array}
\right.&\Leftrightarrow\left\{
\begin{array}{l}
y(\mathcal{A}-x-y)-xy=0\\
y(\mathcal{A}-x-y)-xy=0
\end{array}
\right.\Leftrightarrow\left\{
\begin{array}{l}
y(\mathcal{A}-2x-y)=0\\
x(\mathcal{A}-x-2y)=0
\end{array}
\right.\\
 &\Leftrightarrow\left\{
\begin{array}{l}
2x+y=\mathcal{A}\\
x+2y=\mathcal{A}
\end{array}
\right.\Leftrightarrow x=y=\frac{\mathcal{A}}{3}.
\end{align*}
Le maximum cherché est donc égal à $\frac{8}{abc}\times\frac{\mathcal{A}}{3}\times\frac{\mathcal{A}}{3}\times\left(\mathcal{A}-\frac{\mathcal{A}}{3}-\frac{\mathcal{A}}{3}\right)=\frac{8\mathcal{A}^3}{27abc}$.
(On peut montrer que ce maximum est obtenu quand $M$ est le centre de gravité du triangle $ABC$).
}
}
