\uuid{Xknk}
\exo7id{6123}
\auteur{queffelec}
\datecreate{2011-10-16}
\isIndication{false}
\isCorrection{false}
\chapitre{Continuité, uniforme continuité}
\sousChapitre{Continuité, uniforme continuité}

\contenu{
\texte{
Soit $S^{n-1}=\{x=(x_1,\cdots,x_n)\in {\Rr}^n/ \ 
\Vert x\Vert^2=\sum_1^nx_i^2=1\}$, la sphère unité de ${\Rr}^n$, $p$ son p\^ole
nord i.e. le point $p=(0,\cdots,0,1)$, et $A=S^{n-1}\backslash \{p\}$.
}
\begin{enumerate}
    \item \question{Montrer que le ``plan" de l'équateur $E$ est homéomorphe à ${\Rr}^{n-1}$.}
    \item \question{A tout point $x$ de $A$ on associe $h(x)$ le point d'intersection de la droite
issue de
$p$ passant par ce point, avec le plan $E$. Expliciter $h$, puis $h^{-1}$ et
montrer ainsi que la sphère est homéomorphe à ${\Rr}^{n-1}$.

(On établira $h(x)=p +{{x-p}\over{1-x_n}}$ et $h^{-1}(y)={{2y}\over{1+\Vert
y\Vert^2}} + p\ {{1-\Vert
y\Vert^2}\over{1+\Vert
y\Vert^2}}$).}
    \item \question{En déduire un homéomorphisme de $\mathbb{S}^1$ sur $\overline{\Rr}$.}
\end{enumerate}
}
