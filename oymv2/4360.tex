\uuid{4360}
\auteur{quercia}
\datecreate{2010-03-12}
\isIndication{false}
\isCorrection{true}
\chapitre{Intégration}
\sousChapitre{Intégrale de Riemann dépendant d'un paramètre}

\contenu{
\texte{
On pose pour $n\ge2$ $u_n= \int_{x=0}^{+\infty} \frac1{1+x^n}\, d x$.
Montrer que la suite $(u_n)$ converge, puis que la série $\sum(u_n-1)$ converge
également.
}
\reponse{
$u_n \to 1$ (lorsque $n\to\infty$) par convergence dominée.

$u_n-1 =  \int_{x=0}^1\Bigl(\frac{1+x^{n-2}}{1+x^n}-1\Bigr)\,d x
       = \frac1n \int_{u=0}^1\frac{u^{1-1/n}}{1+u}(u^{-2/n}-1)\,d u$.

On a $0\le u^{-2/n}-1=\exp\Bigl(-\frac{2\ln(u)}n\Bigr)-1\le -\frac{2\ln(u)}n\exp\Bigl(-\frac{2\ln(u)}n\Bigr)$

d'où $0\le u_n\le\frac2{n^2} \int_{u=0}^1\frac{u^{1-3/n}(-\ln u)}{1+u}\,d u
= O\Bigl(\frac1{n^2}\Bigr)$.
}
}
