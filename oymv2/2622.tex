\uuid{2622}
\auteur{debievre}
\datecreate{2009-05-19}

\contenu{
\texte{
D\'eterminer, pour chacune des fonctions suivantes, le domaine 
de d\'efinition $D_f$. 
Pour chacune des fonctions,
calculer ensuite 
les d\'eriv\'ees partielles en chaque point du
domaine de d\'efinition  lorsqu'elles existent:
}
\begin{enumerate}
    \item \question{$f(x,y)=x^2\exp(xy)$,}
\reponse{$D_f=\R^2$. 
\begin{align*}
\frac{\partial f}{\partial x}&= 2x\exp(xy)+x^2y\exp(xy)
\\
\frac{\partial f}{\partial y}&= x^3\exp(xy)
\end{align*}
\noindent}
    \item \question{$f(x,y)=\ln(x+\sqrt{x^2+y^2})$,}
\reponse{$D_f=\{(x,y); x > 0 \ \text{ou} \ y \ne 0\}\subseteq \R^2$.
\begin{align*}
\frac{\partial f}{\partial x}&= 
\frac{1+\frac{x}{\sqrt{x^2+y^2}}}{x+\sqrt{x^2+y^2}}= 
\frac 1{\sqrt{x^2+y^2}}
\\
\frac{\partial f}{\partial y}&=  
\frac{\frac{y}{\sqrt{x^2+y^2}}}{x+\sqrt{x^2+y^2}}
=\frac{y}{x\sqrt{x^2+y^2}+x^2+y^2}
\end{align*}}
    \item \question{$f(x,y)=\sin^2 x+ \cos^2y$,}
\reponse{$D_f=\R^2$. 
\begin{align*}
\frac{\partial f}{\partial x}&= 2\sin x \cos x
\\
\frac{\partial f}{\partial y}&= -2\sin y \cos y 
\end{align*}}
    \item \question{$f(x,y,z)=x^2y^2\sqrt{z}$.}
\reponse{$D_f=\{(x,y,z); z \ne 0\}\subseteq \R^3$.
\begin{align*}
\frac{\partial f}{\partial x}&= 2xy^2\sqrt{z}
\\
\frac{\partial f}{\partial y}&= 2x^2y\sqrt{z} 
\\
\frac{\partial f}{\partial z}&= \frac {x^2y^2}{2\sqrt{z}}  
\end{align*}}
\indication{Pour calculer les  d\'eriv\'ees partielles par rapport 
\`a une variable, interp\'eter les autres variables comme param\`etres
et utiliser les r\`egles de calcul de la d\'eriv\'ee ordinaires.}
\end{enumerate}
}
