\uuid{jeSc}
\exo7id{3963}
\auteur{quercia}
\datecreate{2010-03-11}
\isIndication{false}
\isCorrection{true}
\chapitre{Dérivabilité des fonctions réelles}
\sousChapitre{Autre}

\contenu{
\texte{

}
\begin{enumerate}
    \item \question{Montrer que : $\forall\ x \ge -1,\ \ln(1+x) \le x$.}
    \item \question{Soit $k \in {]-1,1[}$. On pose $u_n = (1+k)(1+k^2)\dots(1+k^n)$.
    Montrer que la suite $(u_n)$ est convergente
    (traiter séparément les cas $k \ge 0$, $k < 0$).}
\reponse{
Pour $k \ge 0$, la suite $(u_n)$ est croissante et $\ln u_n \le \frac k{1-k}$.

    Pour $k < 0$, $(u_{2n})$ décro\^\i t et converge, et $u_{2n+1} \sim u_{2n}$.
}
\end{enumerate}
}
