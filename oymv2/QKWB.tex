\uuid{QKWB}
\exo7id{3275}
\auteur{quercia}
\datecreate{2010-03-08}
\isIndication{false}
\isCorrection{true}
\chapitre{Polynôme, fraction rationnelle}
\sousChapitre{Fraction rationnelle}

\contenu{
\texte{
Soit $H = \{ F \in { K(X)}$ tel que $F(\scriptstyle X) = F(\frac 1X) \}$.
}
\begin{enumerate}
    \item \question{Montrer que : $F \in H \Leftrightarrow \exists\ G \in { K(X)}$ tel que
    $F(\scriptstyle X) = G\Bigl(\scriptstyle X+\frac 1X\Bigr)$.}
    \item \question{Montrer que $H$ est un sous-corps de ${ K(X)}$.}
    \item \question{Que vaut $\dim_H({ K(X)})$ ? Donner une base de ${ K(X)}$ sur $H$.}
\reponse{
$ \Rightarrow \frac{P(\scriptstyle X)}{Q(\scriptstyle X)} =
  \frac{P(\frac 1X)}{Q(\frac 1X)} = \frac{P(\scriptstyle X)+P(\frac
    1X)}{Q(\scriptstyle X)+Q(\frac 1X)}$.
}
\end{enumerate}
}
