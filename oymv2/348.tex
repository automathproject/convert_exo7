\uuid{348}
\auteur{ridde}
\datecreate{1999-11-01}
\isIndication{false}
\isCorrection{true}
\chapitre{Arithmétique dans Z}
\sousChapitre{Nombres premiers, nombres premiers entre eux}

\contenu{
\texte{
Soit $X$ l'ensemble des nombres premiers de la forme $4k + 3$ avec $k \in \Nn$.
}
\begin{enumerate}
    \item \question{Montrer que $X$ est non vide.}
\reponse{$X$ est non vide car, par exemple pour $k=2$, $4k+3=11$ est premier.}
    \item \question{Montrer que le produit de nombres de la forme $4k + 1$ est encore de cette forme.}
\reponse{$(4k+1)(4\ell+1) = 16k\ell + 4(k+\ell)+1 = 4(4k\ell+k+\ell)+1$.
Si l'on note l'entier $k' = 4k\ell+k+\ell$ alors $(4k+1)(4\ell+1) = 4k'+1$, 
ce qui est bien de la forme voulue.}
    \item \question{On suppose que $X$ est fini et on l'\'ecrit alors $X = \left\{
p_1, \ldots, p_n\right\}$.\\  Soit $a = 4p_1 p_2 \ldots p_n  - 1$. Montrer par l'absurde
que $a$ admet un diviseur premier de la forme $4k + 3$.}
\reponse{Remarquons que $2$ est le seul nombre premier pair, les autres sont de la forme
$4k+1$ ou $4k+3$. Ici $a$ n'est pas divisible par $2$, supposons --par l'absurde-- 
que $a$ n'a pas de diviseur de la forme $4k+3$, alors 
tous les diviseurs de $a$ sont de la forme $4k+1$. C'est-\`a-dire que $a$ s'\'ecrit comme produit
de nombre de la forme $4k+1$, et par la question pr\'ec\'edente $a$ peut s'\'ecrire $a=4k'+1$.
Donc $a \equiv 1 \pmod{4}$. Mais comme $a = 4p_1p_2\ldots p_n -1$, $a \equiv -1 \equiv 3 \pmod{4}$.
Nous obtenons une contradiction. Donc $a$ admet une diviseur premier $p$ de la forme $p=4\ell+3$.}
    \item \question{Montrer que ceci est impossible et donc que $X$ est infini.}
\reponse{Dans l'ensemble $X = \{p_1,\ldots,p_n\}$ il y a tous les nombres premiers de la formes $4k+3$.
Le nombre $p$ est premier et s'\'ecrit $p = 4\ell+3$ donc $p$ est un \'el\'ement de $X$, donc
il existe $i\in \{1,\ldots,n\}$ tel que $p=p_i$.
Raisonnons modulo $p=p_i$: $a \equiv 0 \pmod{p}$ car $p$ divise $a$.
D'autre part $a=4p_1\ldots p_n - 1$ donc $a\equiv -1 \pmod{p}$. (car $p_i$ divise $p_1\ldots p_n$).
Nous obtenons une contradiction, donc $X$ est infini: il existe une infinit\'e de nombre 
premier de la forme $4k+3$.
Petite remarque, tous les nombres de la forme $4k+3$ ne sont pas des nombres premiers, par
exemple pour $k=3$, $4k+3 = 15$ n'est pas premier.}
\end{enumerate}
}
