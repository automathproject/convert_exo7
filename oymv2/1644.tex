\uuid{1644}
\auteur{barraud}
\datecreate{2003-09-01}
\isIndication{false}
\isCorrection{false}
\chapitre{Réduction d'endomorphisme, polynôme annulateur}
\sousChapitre{Diagonalisation}

\contenu{
\texte{
On considère un réel $\alpha$ et l'application $T_{\alpha}$ suivante :
$$
T_{\alpha} :
\begin{array}{ccl}
  \R[X] & \rightarrow  & \R[X] \\
  P     & \mapsto & X(X-1)P''+(1+\alpha X)P'
\end{array}
$$
}
\begin{enumerate}
    \item \question{Montrer que pour tout entier $n>0$, la restriction de $T_{\alpha}$ à
$\R_{n}[X]$ défini un endomorphisme de $\R_{n}[X]$.}
    \item \question{On suppose pour cette question que $n=3$.
\begin{enumerate}}
    \item \question{Ecrire la matrice de $T_{\alpha}$ dans la base
$(1,X,X^{2},x^{3})$.}
    \item \question{Déterminer les valeurs propores de $T_{\alpha}$. On les note
$\lambda_{0},\lambda_{1},\lambda_{2},\lambda_{3}$.}
    \item \question{Déterminer les valeurs de $\alpha$ pour lesquelles $T_{\alpha}$ a des
valeurs propres multiples.}
    \item \question{Donner un vecteur propre de $T_{\alpha}$ pour chaque valeur propre,
lorsque $\alpha=-1$, puis $\alpha=-4$. L'endomorphisme $T_{-4}$ est-il
diagonalisable ?}
\end{enumerate}
}
