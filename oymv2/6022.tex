\uuid{nSg5}
\exo7id{6022}
\auteur{quinio}
\datecreate{2011-05-20}
\isIndication{false}
\isCorrection{true}
\chapitre{Probabilité discrète}
\sousChapitre{Lois de distributions}

\contenu{
\texte{
On jette un dé 180 fois.
On note $X$ la variable aléatoire : <<nombre de sorties du 4>>.
}
\begin{enumerate}
    \item \question{Quelle est la loi de $X$ ?}
\reponse{La loi de $X$ est une loi binomiale $B(180;\frac{1}{6})$
Espérance: 30; écart-type: $\sigma =\sqrt{25}=5$.}
    \item \question{Calculez la probabilité pour que $X$ soit compris entre 29 et 32.}
\reponse{En approchant cette loi par une loi normale $N(30$; $\sigma)$ la
probabilité pour que $X$ soit compris entre 29 et 32 :
$P[(28.5-30)/\sigma \leq (X-30)/\sigma \leq (32.5-30)/\sigma ]\simeq 30.94\%$.
Avec la vraie loi, on trouve la probabilité pour que $X$ soit compris entre 29 et 32 est $30.86\%$.}
\end{enumerate}
}
