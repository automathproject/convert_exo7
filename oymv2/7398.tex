\uuid{7398}
\auteur{mourougane}
\datecreate{2021-08-10}

\contenu{
\texte{

}
\begin{enumerate}
    \item \question{On rappelle que le seul polynôme irréductible de degré $2$ sur $\mathbb{F}_2$ est $X^2+X+1$.
 Montrer que le polynôme $X^4+X+1$ est irréductible dans $\mathbb{F}_2[X]$.}
\reponse{Si le polynôme $P=X^4+X+1$ est réductible, il est divisible par un polynôme de degré~ 1 (donc $X$ ou $X+1$) ou par un polynôme irréductible de degré 2 (donc $X^2+X+1$). On vérifie que les polynômes $X, X+1$ et $X^2+X+1$ ne divisent pas $P$, alors $P$ est irréductible.}
    \item \question{On note $A:=\mathbb{F}_2[X]/<X^4+X+1>$ l'anneau quotient de $\mathbb{F}_2[X]$ par l'idéal engendré par $P$. La classe de $3X^5+X^2+X+7$ est-elle nulle dans $A$ ?
 L'anneau $A$ est-il un corps ? Combien a-t-il d'éléments ?}
\reponse{On effectue la division euclidienne de $3X^5+X^2+X+7$ par $X^4+X+1$. Puisque $3X^5+X^2+X+7=X^5+X^2+X+1=X(X^4+X+1)+1=1$ dans $A$, la classe de $3X^5+X^2+X+7$ n'est pas nulle. \\
 L'anneau $\mathbb{F}_2[X]/P$ est un corps si et seulement si le polynôme $P$ est irréductible dans $\mathbb{F}_2[X]$. Comme $X^4+X+1$ est irréductible, $A$ est un corps. $A$ est de cardinal $2^4=16$.}
    \item \question{On note $\alpha$ la classe du polynôme $X$ dans $A$.
Déterminer $\alpha^4$ et $\alpha^{15}$ comme polynômes de degré au plus $3$ en $\alpha$.}
\reponse{On a une relation $\alpha^4+\alpha+1=0$ dans $A$, alors $\alpha^4=\alpha+1$. Le groupe des inversibles $A^{\times}$ de $A$ contient $15$ éléments, alors le théorème de Lagrange implique que l'élément inversible $\alpha$ vérifie $\alpha^{15}=1$.}
    \item \question{Le polynôme $X^{15}-1$ est-il un multiple de $X^4+X+1$ dans $\mathbb{F}_2[X]$ ?}
\reponse{Puisque $\alpha^4+\alpha+1=0$ et $\alpha^{15}-1=0$, les polynômes $X^{15}-1$ et $X^4+X+1$ représentent la même classe (classe nulle) dans $A$. Alors $X^{15}-1$ est bien un multiple de $X^4+X+1$ dans $\mathbb{F}_2[X]$.}
\end{enumerate}
}
