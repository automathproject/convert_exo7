\uuid{ku7v}
\exo7id{3016}
\auteur{quercia}
\datecreate{2010-03-08}
\isIndication{false}
\isCorrection{false}
\chapitre{Groupe, anneau, corps}
\sousChapitre{Anneau}

\contenu{
\texte{

}
\begin{enumerate}
    \item \question{Soit $d \in \N$. On pose
  $$A_d = \{(x,y) \in \Z^2 \text{ tq } x\equiv y (\mathrm{mod}\, d)\}$$
  ($x=y$ pour $d=0$).
    Montrer que $A_d$ est un sous-anneau de $\Z^2$.}
    \item \question{Montrer que l'on obtient ainsi tous les sous-anneaux de $\Z^2$.}
    \item \question{Soit $I$ un id{\'e}al de $\Z^2$. On note :
    $\begin{cases} I_1 = \{ x \in \Z \text{ tq } (x,0) \in I \}\cr
             I_2 = \{ y \in \Z \text{ tq } (0,y) \in I \}.\end{cases}$

    Montrer que $I_1$ et $I_2$ sont des id{\'e}aux de $\Z$, et que
    $I = I_1 \times I_2$.}
    \item \question{En d{\'e}duire que $I$ est un id{\'e}al principal.}
\end{enumerate}
}
