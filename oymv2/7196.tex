\uuid{7196}
\auteur{megy}
\datecreate{2019-07-23}
\isIndication{false}
\isCorrection{false}
\chapitre{Logique, ensemble, raisonnement}
\sousChapitre{Relation d'équivalence, relation d'ordre}

\contenu{
\texte{
Soient $\mathcal R$ et $\mathcal S$ des relations binaires sur $E$. On dit que $\mathcal R$ est plus fine que $\mathcal S$, ou encore que c'est est un raffinement, si $\forall x, y\in E, x\mathcal R y \implies x\mathcal S y$.  De façon équivalente, $\mathcal R$ est plus fine que $\mathcal S$ si on a l'inclusion de graphes $\Gamma_{\mathcal R} \subseteq \Gamma_{\mathcal S}$.
}
\begin{enumerate}
    \item \question{Montrer que \og être plus fine que \fg{} est une relation d'ordre sur l'ensemble des relations binaires sur $E$.}
    \item \question{Soient $\mathcal R$ et $\mathcal S$ des relations binaires sur $E$. Montrer qu'il existe une relation binaire sur $E$ qui raffine à la fois $\mathcal R$ et $\mathcal S$, et  qu'il existe aussi une relation binaire sur $E$ simultanément moins fine que $\mathcal R$ et $\mathcal S$.}
\end{enumerate}
}
