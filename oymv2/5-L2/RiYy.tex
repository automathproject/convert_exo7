\uuid{RiYy}
\exo7id{5477}
\auteur{rouget}
\datecreate{2010-07-10}
\isIndication{false}
\isCorrection{true}
\chapitre{Equation différentielle}
\sousChapitre{Résolution d'équation différentielle du premier ordre}

\contenu{
\texte{
Résoudre l'équation différentielle $(1-x^2)y'-2xy=x^2$ sur chacun des intervalles $I$
suivants~:~$I=]1,+\infty[$, $I=]-1,1[$, $I=]-1,+\infty[$, $I=\Rr$.
}
\reponse{
L'équation différentielle à résoudre dans cet exercice est linéaire du premier ordre. On note $(E)$ l'équation
différentielle proposée et $(E_H)$ l'équation homogène associée.

Soit $I$ l'un des deux intervalles $]-1,1[$ ou $]1,+\infty[$. Les fonctions $x\mapsto\frac{-2x}{1-x^2}$ et
$x\mapsto\frac{x^2}{1-x^2}$ sont continues sur $I$ et on sait que les solutions de $(E)$ sur $I$ sont de la forme
$f_0+\lambda f_1$ où $f_0$ est une solution particulière de $(E)$ et $f_1$ est une solution particulière non nulle de
$(E_H)$.

Résolution de $(E)$ sur $I$. Soit $f$ une fonction dérivable sur $I$.

\begin{align*}\ensuremath
f\;\mbox{solution de}\;(E)\;\mbox{sur}\;I&\Leftrightarrow\forall x\in I,\;(1-x^2)f'(x)-2xf(x)=x^2\\
 &\Leftrightarrow\forall x\in I,\;((1-x^2)f)'(x)=x^2\Leftrightarrow\exists\lambda\in\Rr/\;\forall x\in
I,\;(1-x^2)f(x)=\frac{x^3}{3}+\lambda\\ &\Leftrightarrow\exists \lambda\in\Rr/\;\forall x\in 
I,\;f(x)=\frac{x^3+\lambda}{3(1-x^2)},\end{align*}

(en renommant $\lambda$ la constante $3\lambda$).

Si $I=]-1,+\infty[$.

Soit $f$ une éventuelle solution de $(E)$ sur $I$. Les restrictions de $f$ à $]-1,1[$ et $]1,+\infty[$ sont
encore solution de $(E)$ et donc de la forme précédente. Par suite, nécessairement, il existe deux constantes
$\lambda_1$ et$\lambda_2$ telles que, pour $-1<x<1$, $f(x)=\frac{x^3+\lambda_1}{3(1-x^2)}$ et pour $x>1$,
$f(x)=\frac{x^3+\lambda_2}{3(1-x^2)}$. Enfin, l'équation impose $f(1)=-\frac{1}{2}$.

En résumé, une éventuelle solution de $(E)$ sur $I$ est nécessairement de la forme~:

$$\forall x>-1,\;f(x)=\left\{
\begin{array}{l}
\frac{x^3+\lambda_1}{3(1-x^2)}\;\mbox{si}\;-1<x<1\\
-\frac{1}{2}\;\mbox{si}\;x=1\\
\frac{x^3+\lambda_2}{3(1-x^2)}\;\mbox{si}\;x>1
\end{array}
\right..$$

Réciproquement, $f$ ainsi définie, est dérivable sur $]-1,1[$ et solution de $(E)$ sur $]-1,1[$, dérivable sur
$]1,+\infty[$ et solution de $(E)$ sur $]1,+\infty[$ et, si $f$ est dérivable en $1$, $f$ vérifie encore $(E)$ pour
$x=1$. Donc, $f$ est solution de $(E)$ sur $]-1,+\infty[$ si et seulement si $f$ est dérivable en $1$.

Pour $-1<x<1$,

$$\frac{f(x)-f(1)}{x-1}=\frac{\frac{x^3+\lambda_1}{3(1-x^2)}+\frac{1}{2}}{x-1}=\frac{2x^3+2\lambda_1+3(1-x^2)}
{6(1-x^2)(x-1)}$$

Quand $x$ tend vers $1$ par valeurs inférieures, le dénominateur de la fraction tend vers $0$ et le numérateur tend vers
$2(1+\lambda_1)$. Donc, si $\lambda_1\neq-1$, $f$ n'est pas dérivable à gauche en $1$. De même, si $\lambda_2$ n'est pas
$-1$, $f$ n'est pas dérivable à droite en $-1$. Ainsi, si $f$ est solution de $(E)$ sur $I$, nécessairement
$\lambda_1=\lambda_2=-1$. Dans ce cas, pour $x\in]-1,+\infty[\setminus\{1\}$,

$$f(x)=\frac{x^3-1}{3(1-x^2)}=\frac{(x-1)(x^2+x+1)}{3(1-x)(1+x)}=-\frac{x^2+x+1}{3(x+1)},$$

ce qui reste vrai pour $x=1$. Ainsi, si $f$ est une solution de $(E)$ sur $]-1,+\infty[$, nécessairement
pour $x>-1$, $f(x)=-\frac{x^2+x+1}{3(x+1)}$. Réciproquement, $f$ ainsi définie est dérivable sur $]-1,+\infty[$ et en
particulier en $1$. $f$ est donc solution de $(E)$ sur $]-1,+\infty[$.

Sur $]-1,+\infty[$, $(E)$ admet une et une seule solution à savoir la fonction $x\mapsto-\frac{x^2+x+1}{3(x+1)}$.

Si $I=\Rr$, soit $f$ une éventuelle solution de $(E)$ sur $\Rr$. La restriction de $f$ à $]-1,+\infty[$ est
nécessairement la fonction précédente. Mais cette fonction tend vers $-\infty$ quand $x$ tend vers $-1$ par valeurs
supérieures. Donc $f$ ne peut être continue sur $\Rr$ et $(E)$ n'a pas de solution sur $\Rr$.
}
}
