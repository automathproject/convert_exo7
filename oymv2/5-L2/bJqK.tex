\uuid{bJqK}
\exo7id{4620}
\auteur{quercia}
\datecreate{2010-03-14}
\isIndication{false}
\isCorrection{true}
\chapitre{Série entière}
\sousChapitre{Autre}

\contenu{
\texte{
Soit $D$ le disque ouvert de~$\C$ de centre~$0$ et rayon~$1$.
}
\begin{enumerate}
    \item \question{Soit $\varphi(z) = \sum_{n\in\N}a_nz^n$ une série entière de rayon $R\ge 1$
    et $r\in{]0,1[}$.
    Montrer que $$a_n = \frac1{2\pi r^n} \int_{\theta=0}^{2\pi}\varphi(re^{i\theta})e^{-in\theta}\, d\theta.$$}
    \item \question{Soit $E$ l'ensemble des fonctions de~$\overline D$ dans~$\C$ continues
    et dont la restriction à~$D$ est somme d'une série entière. Montrer que
    $f \mapsto\|f\| = \sup\{|f(z)|,\ z\in\overline D\}$ définit une norme sur~$E$ et
    que pour cette norme $E$ est complet.}
    \item \question{Montrer que l'ensemble des polynômes à coefficients complexes est dense dans~$E$.}
\reponse{
Complétude~: soit $(f_k)$ une suite d'éléments de~$E$ de Cauchy,
    $f_k(z) = \sum_{n\in\N}a_{n,k}z^n$. On a, à $k$ et $n$ fixés, par convergence dominée~:
    $$\frac1{2\pi} \int_{\theta=0}^{2\pi}f_k(e^{i\theta})e^{-in\theta}\, d\theta
      = \lim_{r\to 1^-}\frac1{2\pi r^n} \int_{\theta=0}^{2\pi}f_k(re^{i\theta})e^{-in\theta}\, d\theta
      = a_{n,k}.$$
    La suite $(f_k)$
    converge uniformément sur~$\overline D$ vers une fonction $\varphi : {\overline D} \to \C$
    continue. On note~:
    $$a_n = \frac1{2\pi} \int_{\theta=0}^{2\pi}\varphi(e^{i\theta})e^{-in\theta}\, d\theta
          = \lim_{k\to\infty} a_{n,k}.$$
    La suite $(a_n)$ est bornée, donc le rayon de convergence de $\sum_{n\in\N}a_nz^n$
    est supérieur ou égal à~$1$. Pour $z\in D$ fixé on a alors lorsque $k\to\infty$ :
    \begin{align*} f_k(z)
        &= \sum_{n\in\N}a_{n,k}z^n
         = \frac1{2\pi} \int_{\theta=0}^{2\pi}\Bigl(\sum_{n\in\N}f_k(e^{i\theta})e^{-in\theta}z^n\Bigr)\, d\theta
         = \frac1{2\pi} \int_{\theta=0}^{2\pi}\frac{f_k(e^{i\theta})}{1-ze^{-i\theta}}\, d\theta\cr
        &\to \frac1{2\pi} \int_{\theta=0}^{2\pi}\frac{\varphi(e^{i\theta})}{1-ze^{-i\theta}}\, d\theta
         = \frac1{2\pi} \int_{\theta=0}^{2\pi}\Bigl(\sum_{n\in\N}\varphi(e^{i\theta})e^{-in\theta}z^n\Bigr)\, d\theta
         = \sum_{n\in\N}a_nz^n\cr
    \end{align*}
    ce qui prouve que $\varphi\in E$. Enfin on a $\|f_k-\varphi\| \to 0$ lorsque $k\to\infty$
    par convergence uniforme, d'où $\varphi = \lim_{k\to\infty}f_k$ dans~$E$.
Soit $f\in E$ et $f_n(z) = f\Bigl(\frac{nz}{n+1}\Bigr)$.
    Comme $f$ est uniformément continue, $f_n$ converge uniformément vers~$f$
    sur $\overline D$. Soit $\varepsilon>0$ et $n$ tel que $\|f-f_n\|_\infty \le \varepsilon$.
    Comme $f_n$ est développable en série entière avec un rayon au moins égal à $1+\frac1n$,
    son développement converge uniformément vers~$f_n$ sur $\overline D$ donc
    il existe $P\in\C[X]$ tel que $\|f_n-P\|_\infty \le \varepsilon$.
}
\end{enumerate}
}
