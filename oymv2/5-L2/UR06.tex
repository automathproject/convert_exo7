\uuid{UR06}
\exo7id{4195}
\auteur{quercia}
\datecreate{2010-03-11}
\isIndication{false}
\isCorrection{false}
\chapitre{Fonction de plusieurs variables}
\sousChapitre{Extremums locaux}

\contenu{
\texte{
Soient dans $\R^2$ : $A=(0,a)$, $B=(b,-c)$ et $M=(x,0)$ ($a,b,c > 0$).
Un rayon lumineux parcourt la ligne brisée $AMB$ à la vitesse $v_1$ de $A$ à $M$
et $v_2$ de $M$ à $B$.
On note $\alpha_1 = \overline{(\vec j,\vec{MA})}$
        $\alpha_2 = \overline{(-\vec j,\vec{MB})}$.
}
\begin{enumerate}
    \item \question{Faire une figure.}
    \item \question{Montrer que le temps de parcours est minimal lorsque
    $\frac {\sin\alpha_1}{v_1} = \frac {\sin\alpha_2}{v_2}$.}
\end{enumerate}
}
