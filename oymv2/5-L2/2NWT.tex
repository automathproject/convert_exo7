\uuid{2NWT}
\exo7id{5729}
\auteur{rouget}
\datecreate{2010-10-16}
\isIndication{false}
\isCorrection{true}
\chapitre{Suite et série de fonctions}
\sousChapitre{Convergence simple, uniforme, normale}

\contenu{
\texte{
Soit $(P_n)_{n\in\Nn}$ une suite de polynômes convergeant uniformément sur $\Rr$ vers une fonction $f$. Montrer que $f$ est un polynôme.
}
\reponse{
Posons $f=\lim_{n \rightarrow +\infty}P_n$.

Le critère de \textsc{Cauchy} de convergence uniforme (appliqué à $\varepsilon=1$) permet d'écrire

\begin{center}
$\exists N\in\Nn/\;\forall n\geqslant N,\;\forall m\geqslant N,\;\forall x\in\Rr,\; |P_n(x) - P_m(x)|\leqslant1$.
\end{center}

Pour $n\geqslant N$, les polynômes $P_N-P_n$  sont bornés sur $\Rr$ et donc constants. Par suite, pour chaque $n\geqslant N$, il existe $a_n\in\Rr$ tel que $P_N-P_n = a_n$\quad$(*)$.
Puisque la suite $(P_n)$ converge simplement sur $\Rr$, La suite $(a_n)=(P_N(0) - P_n(0))$ converge  vers un réel que l'on note $a$. On fait alors tendre $n$ tend vers $+\infty$ dans l'égalité $(*)$ et on obtient 

\begin{center}
$f = P_N - a$
\end{center}

On a montré que $f$ est un polynôme.
}
}
