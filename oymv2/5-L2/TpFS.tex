\uuid{TpFS}
\exo7id{4216}
\auteur{quercia}
\datecreate{2010-03-11}
\isIndication{false}
\isCorrection{true}
\chapitre{Equation différentielle}
\sousChapitre{Equations aux dérivées partielles}

\contenu{
\texte{
Pour $(x,y) \in {\R^2}$, on pose $u = x^2-y^2$, $v = 2xy$.

Soit $F : {\R^2} \to \R, {(u,v)} \mapsto {F(u,v)}$ et $f$ définie par :
$f(x,y) = F(u,v)$.

Montrer que $\frac{\partial^2 F}{\partial u^2} + \frac{\partial^2 F}{\partial v^2} = 0$ entraîne $\frac{\partial^2 f}{\partial x^2} + \frac{\partial^2 f}{\partial y^2} = 0$.
}
\reponse{
$\Delta f = 4(x^2+y^2)\Delta F$.
}
}
