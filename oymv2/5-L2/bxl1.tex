\uuid{bxl1}
\exo7id{4584}
\auteur{quercia}
\datecreate{2010-03-14}
\isIndication{false}
\isCorrection{true}
\chapitre{Série entière}
\sousChapitre{Calcul de la somme d'une série entière}

\contenu{
\texte{
\smallskip
Soit~$A=\left(\begin{smallmatrix}1&1&1\cr1&1&0\cr1&0&0\cr\end{smallmatrix}\right)\in\mathcal{M}_{3}(\R)$.
}
\begin{enumerate}
    \item \question{Montrer que~$A$ est diagonalisable et admet trois valeurs propres réelles
    dont on précisera les parties entières.}
\reponse{$\chi_A(\lambda) = -\lambda^3 + 2\lambda^2 + \lambda - 1$.
    $\chi_A(-1)>0$, $\chi_A(0) < 0$, $\chi_A(1)>0$, $\chi_A(2)>0$, $\chi_A(3)<0$
    donc $\chi_A$ admet une racine dans chacun des intervalles
    $]-1,0[$, $]0,1[$ et $]2,3[$.}
    \item \question{On pose $t_n = \mathrm{tr}(A^n)$. Exprimer $t_n$ en fonction de $t_{n-1},t_{n-2},t_{n-3}$.}
\reponse{Cayley-Hamilton~: $t_n = 2t_{n-1}+t_{n-2}-t_{n-3}$.}
    \item \question{Déterminer le rayon de convergence de la série entière $\sum_{n=0}^\infty t_nz^n$ et calculer sa somme.}
\reponse{Soient $-1<\alpha<0<\beta<1<2<\gamma<3$ les valeurs propres de~$A$.
    On a $t_nz^n = (\alpha z)^n + (\beta z)^n + (\gamma z)^n$ donc la série $\sum_{n=0}^\infty t_nz^n$
    converge si et seulement si $|\gamma z|<1$ et vaut~:
    $$\frac1{1-\alpha z} + \frac1{1-\beta z} + \frac 1{1-\gamma z}
      = \frac1z\frac{\chi'}{\chi}\Bigl(\frac1z\Bigr)
      = \frac{-z^2-4z+3}{z^3-z^2-2z+1}.$$}
\end{enumerate}
}
