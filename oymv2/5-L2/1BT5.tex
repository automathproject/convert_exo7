\uuid{1BT5}
\exo7id{4379}
\auteur{quercia}
\datecreate{2010-03-12}
\isIndication{false}
\isCorrection{true}
\chapitre{Intégration}
\sousChapitre{Intégrale de Riemann dépendant d'un paramètre}

\contenu{
\texte{

}
\begin{enumerate}
    \item \question{Soit $f : {[a,b]} \to \R$ de classe $\mathcal{C}^2$ et $F(x) =  \int_{t=a}^b
    f(t)e^{-itx}\,d t$ avec $a<0<b$. Montrer que $F(x)\to0$ lorsque $x\to+\infty$.}
\reponse{Intégrer par parties.}
    \item \question{Montrer que $F(x) = \frac{f(a)e^{-iax} - f(b)e^{-ibx}}{ix} +  o\Bigl(\frac1x\Bigr)$.}
\reponse{Intégrer deux fois par parties.}
    \item \question{Montrer la convergence de l'intégrale
    $I= \int_{t=-\infty}^{+\infty}e^{-it^2/2}\,d t$.}
\reponse{Pour $0<u<v$~: $ \int_{t=u}^{v}e^{-it^2/2}\,d t =
    \Bigl[\frac{e^{-it^2/2}}{-it}\Bigr]_{t=u}^v -
     \int_{t=u}^v\frac{e^{-it^2/2}}{it^2}\,d t
    \to \frac{e^{-iu^2/2}}{iu}- \int_{t=u}^{+\infty}\frac{e^{-it^2/2}}{it^2}\,d t$
     lorsque $v\to+\infty$.

    Ainsi $ \int_{t=u}^{+\infty}e^{-it^2/2}\,d t$ converge et de même pour
    $ \int_{t=-\infty}^{-u}e^{-it^2/2}\,d t$.}
    \item \question{Soit $g(x) =  \int_{t=a}^bf(t)e^{-ixt^2/2}\,d t$.
    Montrer que $g(x) = \frac{I.f(0)}{\sqrt x} +  o\Bigl(\frac1{\sqrt x}\Bigr)$.}
\reponse{On pose $f(t) = f(0) + t\varphi(t)$ avec $\varphi$ de
    classe $\mathcal{C}^1$. Il vient~:
    \begin{align*}g(x)\sqrt x &= f(0) \int_{u=a\sqrt x}^{b\sqrt x}e^{-iu^2/2}\,d u
      -\frac1{i\sqrt x}\Bigl[e^{-iu^2/2}\varphi(u/\sqrt x)\Bigr]_{u=a\sqrt
    x}^{b\sqrt x} + \frac1{ix} \int_{u=a\sqrt x}^{b\sqrt
    x}e^{-iu^2/2}\varphi'(u/\sqrt x)\,d u\cr
    &\to f(0).I\cr\end{align*}
    lorsque $x\to+\infty$.}
\end{enumerate}
}
