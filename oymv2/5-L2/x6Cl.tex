\uuid{x6Cl}
\exo7id{4591}
\auteur{quercia}
\datecreate{2010-03-14}
\isIndication{false}
\isCorrection{true}
\chapitre{Série entière}
\sousChapitre{Développement en série entière}

\contenu{
\texte{
Soit~$\alpha>0$. On considère la fonction $f_\alpha$ : $x \mapsto\sum_{n=1}^\infty e^{-n^\alpha}e^{inx}$.
Montrer que~$f$ est $\mathcal{C}^\infty$. Donner une CNS sur~$\alpha$ pour que~$f$
soit développable en série entière en tout point de~$\R$.
}
\reponse{
Il y a dérivation terme à terme facilement et indéfiniment.

DSE au voisinage de~$0$~: on envisage de permuter les $\Sigma$ dans~:
$f_\alpha(x) = \sum_{n=1}^\infty\sum_{p=0}^\infty e^{-n^\alpha}\frac{(inx)^p}{p!}$,
ce qui est légitime si la série $\sum_{n=1}^\infty e^{-n^\alpha}e^{n|x|}$ converge.
On en déduit qu'une condition suffisante pour que~$f$ soit DSE au voisinage de~$0$
est $\alpha \ge 1$ (avec convergence si $x\in{]-1,1[}$ pour $\alpha = 1$ et pour
tout~$x\in\R$ si $\alpha > 1$).

Cas $\alpha < 1$~: $|f^{(k)}(0)| = \sum_{n=1}^\infty e^{-n^\alpha}n^k
\ge e^{-N^\alpha}N^k$ avec $N=\lfloor k^{1/\alpha}\rfloor$ donc pour $r>0$
fixé et $k$ tendant vers l'infini on~a
$\ln\Bigl(\Bigl|\frac{f^{(k)}(0)r^k}{k!}\Bigr|\Bigr) \sim
 \Bigl(\frac1\alpha-1\Bigr)k\ln(k)$
et la série de terme général $\frac{f^{(k)}(0)r^k}{k!}$ diverge grossièrement.

DSE au voisinage de~$a\ne 0$~: même raisonnement en écrivant
$f(x) = \sum_{n=1}^\infty\sum_{p=0}^\infty e^{-n^\alpha}e^{ina}\frac{(in(x-a))^p}{p!}$.
En conclusion, $f$ est analytique sur~$\R$ si et seulement si~$\alpha\ge 1$.
}
}
