\uuid{mW3Q}
\exo7id{4540}
\auteur{quercia}
\datecreate{2010-03-14}
\isIndication{false}
\isCorrection{true}
\chapitre{Suite et série de fonctions}
\sousChapitre{Autre}

\contenu{
\texte{
Soit $f_n(x) = \frac{n^x}{(1+x)(1+x/2)\dots(1+x/n)}$.
}
\begin{enumerate}
    \item \question{\'Etudier la convergence simple des fonctions $f_n$.}
    \item \question{On note $f = \lim f_n$. Calculer $f(x)$ en fonction de $f(x-1)$ lorsque
    ces deux quantités existent.}
    \item \question{Montrer que $f$ est de classe $\mathcal{C}^1$ sur son domaine de définition
    (on calculera $f_n'(x)/f_n(x)$).}
\reponse{
$\frac{f_n(x)}{f_{n+1}(x)} = 1 - \frac{x(x+1)}{2n^2}
         +  o\left(\frac1{n^2}\right)$
         donc la série $\sum\ln f_n(x)$ est convergente pour tout $x\notin -\N^*$.
$\frac{f_n'(x)}{f_n(x)} \to -\gamma + \sum_{k=1}^\infty \frac x{k(k+x)}$ lorsque $n\to\infty$.
}
\end{enumerate}
}
