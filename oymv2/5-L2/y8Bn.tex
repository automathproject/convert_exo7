\uuid{y8Bn}
\exo7id{4616}
\auteur{quercia}
\datecreate{2010-03-14}
\isIndication{false}
\isCorrection{true}
\chapitre{Série entière}
\sousChapitre{Analycité}

\contenu{
\texte{
Soit $f(z) = \sum_{n=0}^\infty a_nz^n$ une série entière de rayon de
convergence infini. Montrer l'équivalence entre les propriétés :

\indent 1: Pour tout $a > 0$, la fonction $z \mapsto f(z)e^{-a|z|}$ est bornée sur
           $\C$.\par
\indent 2: $\sqrt[n]{n!\,|a_n|} \to 0$ lorsque $n\to\infty$.\par

On utilisera les formules de Cauchy (cf. exercice~\ref{Cauchy}).
}
\reponse{
$2 \Rightarrow 1$ : évident.\par
         {$1 \Rightarrow 2$ : }Soit $a > 0$ et $M = \sup(|f(z)|e^{-a|z|})$.

         $a_n = \frac1{2\pi} \int_{\theta=0}^{2\pi}
                \frac{f(Re^{i\theta})}{R^ne^{in\theta}}\,d\theta
           \Rightarrow  |a_n| \le M\frac{e^{aR}}{R^n} \le M\inf\limits_{R>0}
                \frac{e^{aR}}{R^n}
          = M\left(\frac{ea}n\right)^n$.

          Donc $\sqrt[n]{n!\,\|a_n\|} \le \sqrt[n]{n!}\frac{ea}n
               \to a$ lorsque $n\to\infty$. CQFD
}
}
