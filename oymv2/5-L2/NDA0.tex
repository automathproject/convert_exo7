\uuid{NDA0}
\exo7id{866}
\auteur{bodin}
\datecreate{1998-09-01}
\isIndication{false}
\isCorrection{true}
\chapitre{Equation différentielle}
\sousChapitre{Résolution d'équation différentielle du deuxième ordre}

\contenu{
\texte{
On consid\`ere l'\'equation  :
$$ y^{\prime\prime} + 2y^\prime + 4y = xe^x \qquad (E) $$
}
\begin{enumerate}
    \item \question{R\'esoudre l'\'equation diff\'erentielle homog\`ene associ\'ee \`a $(E)$.}
\reponse{Le polyn\^ome caract\'eristique associ\'e \`a $E$ est : $p(x) = x^2+2x+4$ ; son discriminant
est $\Delta = -12$ et il a pour racines les 2 nombres complexes $
-1+i\sqrt{3}$ et $-1-i\sqrt3$. Les solutions de l'\'equation
homog\`ene sont donc toutes fonctions :
$$ y(x) = e^{-x}(a\cos{\sqrt3 x} + b\sin{\sqrt3 x})$$ obtenues lorsque $a,b$ d\'ecrivent $\R$.}
    \item \question{Trouver une solution particuli\`ere de $(E)$ (expliquer votre
d\'emarche), puis donner l'ensemble de toutes les solutions de
$(E)$.}
\reponse{Le second membre est de la forme $ e^{\lambda x}Q(x) $ avec $\lambda = 1$ et $Q(x)=x$. On
cherchera une solution de l'\'equation sous la forme : $ y_p(x)=
R(x)e^x$ avec $R$ polyn\^ome de degr\'e \'egal \`a celui de $Q$
puisque $p(1) \not =0$. On pose donc $R(x) = ax+b$. On a
 $$ y_p^{\prime\prime}(x) + 2y_p^{\prime}(x) +4y_p(x) = (7ax+7b+4a)e^x.$$
  Donc $y_p$ est solution
si et seulement si $ 7ax + 7a+4b = x$. On trouve apr\`es
identification des coefficients :
 $$ a=\frac{1}{7} \qquad \hbox{et}\qquad b=\frac{-4}{49}.$$
La fonction $y_p(x)=\frac{1}{7}(x-\frac{4}{7})e^x$ est donc
solution de $E$ et la forme g\'en\'erale des solutions de $E$ est
:
$$ y(x)= e^{-x}(a\cos{\sqrt3 x} + b\sin{\sqrt3 x}) +\frac{1}{7}(x-\frac{4}{7})e^x \; ;\; a,b \in \R.$$}
    \item \question{D\'eterminer l'unique solution $h$ de $(E)$ v\'erifiant
$h(0)=1$ et $h(1)=0$.}
\reponse{Soit $h$ une solution de $E$. Les conditions $h(0)=1$, $h(1)=0$ sont r\'ealis\'ees ssi
$$a=\frac{53}{49}\qquad \hbox{et} \qquad b=-\frac{53\cos\sqrt3+3e^2}{49\sin\sqrt3}.$$}
    \item \question{Soit $f : ]0,\infty [ \longrightarrow \R$
une fonction deux fois d\'erivable sur $]0,\infty [ $ et qui v\'erifie :
$$ t^2f^{\prime \prime}(t) +3tf^\prime (t) + 4f(t) = t\log{t}. $$
    \begin{enumerate}}
\reponse{\begin{enumerate}}
    \item \question{On pose $g(x)=f(e^x)$, v\'erifier que $g$ est solution de $(E)$.}
\reponse{On a : $g^\prime(x) =e^xf^\prime(e^x)$ et
$g^{\prime\prime}(x)=e^xf^\prime(e^x)+e^{2x}f^{\prime\prime}(e^x)$
d'o\`u pour tout $x\in\R$ :
$$g^{\prime\prime}(x)+2g^\prime(x)+4g(x)=e^{2x}f^{\prime\prime}(e^x)+2e^xf^\prime(e^x)+4f(e^x)= e^x\log{e^x}=xe^x$$
donc $g$ est solution de $E$.}
    \item \question{En d\'eduire une expression de $f$.}
\reponse{R\'eciproquement pour $f(t)= g(\log t)$ o\`u $g$ est une
solution de $E$ on montre que $f$ est 2 fois d\'erivable et
v\'erifie l'\'equation donn\'ee en 4. Donc les fonctions $f$
recherch\'ees sont de la forme :
$$\frac{1}{t}(a\cos{(\sqrt3\log t)} + b\sin{(\sqrt3\log t)}) +\frac{t}{7}(\log t-\frac{4}{7}) \; ;\; a,b \in \R.$$}
\end{enumerate}
}
