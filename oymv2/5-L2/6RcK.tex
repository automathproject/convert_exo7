\uuid{6RcK}
\exo7id{5478}
\auteur{rouget}
\datecreate{2010-07-10}
\isIndication{false}
\isCorrection{true}
\chapitre{Equation différentielle}
\sousChapitre{Résolution d'équation différentielle du premier ordre}

\contenu{
\texte{
Résoudre sur $\}-\infty,0[$ et $]0,+\infty[$ l'équation différentielle~:~$|x|y'+(x-1)y=x^3$.
}
\reponse{
\textbf{Résolution} de $(E)$ sur $]0,+\infty[$.

Soit $f$ une fonction dérivable sur $]0,+\infty[$.

\begin{align*}\ensuremath
f\;\mbox{solution de}\;(E)\;&\mbox{sur}\;]0,+\infty[\Leftrightarrow\forall x\in]0,+\infty[,\;|x|f'(x)+(x-1)f(x)=x^3\\
 &\Leftrightarrow\forall x\in]0,+\infty[,\;xf'(x)+(x-1)f(x)=x^3\Leftrightarrow\forall x\in]0,+\infty[,\;f'(x)+(1-\frac{1}{x})f(x)=x^2\\
 &\Leftrightarrow\forall x\in]0,+\infty[,\;e^{x-\ln
x}f'(x)+(1-\frac{1}{x})e^{x-\ln x}f(x)=e^{x-\ln x}x^2\\
 &\Leftrightarrow\forall x\in]0,+\infty[,\;(\frac{e^x}{x}f)'(x)=xe^x=((x-1)e^x)'\\
 &\Leftrightarrow\exists\lambda\in\Rr/\;\forall
x\in]0,+\infty[,\;f(x)=xe^{-x}((x-1)e^x+\lambda)=x^2-x+\lambda xe^{-x}
\end{align*}

Les solutions de $(E)$ sur $]0,+\infty[$ sont les fonctions de la forme $x\mapsto x^2-x+\lambda xe^{-x}$,
$\lambda\in\Rr$.

\textbf{Résolution} de $(E)$ sur $]-\infty,0[$.

Soit $f$ une fonction dérivable sur $]-\infty,0[$.

\begin{align*}\ensuremath
f\;\mbox{solution de}\;(E)\;\mbox{sur}\;]-\infty,0[&\Leftrightarrow\forall x\in]0,+\infty[,\;
-xf'(x)+(x-1)f(x)=x^3\\
 &\Leftrightarrow\forall x\in]-\infty,0[,\;f'(x)+(-1+\frac{1}{x})f(x)=-x^2\\
 &\Leftrightarrow\forall x\in]-\infty,0[,\;e^{-x+\ln|x|}f'(x)+
(-1+\frac{1}{x})e^{-x+\ln|x|}f(x)=-e^{-x+\ln|x|}x^2\\
 &\Leftrightarrow\forall x\in]-\infty,0[,\;(-xe^{-x}y)'=x^3e^{-x}\;(*)
\end{align*}

Déterminons une primitive de la fonction $x\mapsto-x^3e^{-x}$ de la forme $(ax^3+bx^2+cx+d)e^{-x}$.

$$((ax^3+bx^2+cx+d)e^{-x})'=(-(ax^3+bx^2+cx+d)+(3ax^2+2bx+c))e^{-x}=(-ax^3+(3a-b)x^2+(2b-c)x+c-d)e^{-x},$$

et

$$((ax^3+bx^2+cx+d)e^{-x})'=x^3e^{-x}\Leftrightarrow
\left\{
\begin{array}{l}
-a=-1\\
3a-b=0\\
2b-c=0\\
c-d=0
\end{array}
\right.
\Leftrightarrow
\left\{
\begin{array}{l}
a=1\\
b=3\\
c=6=d
\end{array}
\right..$$

Par suite,

\begin{align*}\ensuremath
(*)&\Leftrightarrow\exists\lambda\in\Rr/\;\forall x\in]-\infty,0[,\;xe^{-x}f(x)=(x^3+3x^2+6x+6)e^{-x}+\lambda\\
 &\Leftrightarrow\exists\lambda\in\Rr/\;\forall x\in]-\infty,0[,\;
f(x)=x^2+3x+6+\frac{\lambda e^x+6}{x}.
\end{align*}

Les solutions de $(E)$ sur $]-\infty,0[$ sont les fonctions de la forme $x\mapsto x^2+3x+6+\frac{\lambda e^x+6}{x}$,
$\lambda\in\Rr$.

On peut montrer que l'équation admet une et une seule solution sur $\Rr$ en \og~recollant~\fg~les expressions
précédentes, mais en ce début d'année, on manque encore d'outils.
}
}
