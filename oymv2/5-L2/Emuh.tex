\uuid{Emuh}
\exo7id{851}
\auteur{bodin}
\datecreate{1998-09-01}
\isIndication{false}
\isCorrection{true}
\chapitre{Equation différentielle}
\sousChapitre{Résolution d'équation différentielle du premier ordre}

\contenu{
\texte{
Soit l'\'equation diff\'erentielle
$$(E)\qquad y'+2xy = x.$$
}
\begin{enumerate}
    \item \question{R\'esoudre l'\'equation homog\`ene asoci\'ee.}
    \item \question{Calculer la solution de $(E)$ v\'erifiant $y(0)=1$.}
\reponse{
Les primitives de la fonction $a(x)=2x$ sont les fonctions $A(x)=
x^2/2 + k $ o\`u $k \in \R$ est une constante r\'eelle quelconque.
Donc les solutions de l'\'equation homog\`ene associ\'ee \`a $E$
sont toutes les fonctions d\'efinies sur $\R$ du type : $ y(x)=
ce^{-x^2}$ o\`u $c\in\R$ est une constante arbitraire. On cherche
maintenant une solution particuli\`ere de $E$ sous la forme
$y_p(x)=c(x)e^{-x^2}$ (m\'ethode de la variation de la constante).
On a :\\ $y_p^\prime(x)+ 2xy_p(x)= c^\prime(x)e^{-x^2}$. Donc
$y_p$ est solution de $E$ si et seulement si :
$c^\prime(x)=xe^{x^2}$ pour tout $x\in\R$. On choisit la fonction
$c$ parmi les primitives de la fonction $xe^{x^2}$, par exemple :
$c(x)=1/2e^{x^2}$. Donc la fonction $y_p$
telle que $y_p(x)=1/2e^{x^2}e^{-x^2}=1/2$ est solution de $E$.\\
Par cons\'equent les solutions de $E$ sont toutes les fonctions de
la forme :
  $$ y(x)= ce^{-x^2} + \frac{1}{2} \; c \in\R .$$
Pour $y$ solution de $E_1$, la condition $y(0)=1$ \'equivaut \`a :
$c=1/2$.
}
\end{enumerate}
}
