\uuid{1803}
\auteur{gourio}
\datecreate{2001-09-01}

\contenu{
\texte{
Soit $f :  \R^{2} \rightarrow \R$, d\'efinie par  
\[
\begin{array}{lll}
 f(x, y)=x &\mathrm{si}&
\left|x\right|>\left|y\right|\\ 
f(x, y)=y &\mathrm{ si }&\left|x\right|<\left|y\right|\\ 
f(x, y)= 0 &\mathrm{ si }&
\left|x\right| = \left|y\right|. 
\end{array}
\]
\'Etudier la continuit\'e de $f$, l'existence des d\'eriv\'ees partielles et leur continuit\'e.
}
\indication{Distinguer tout de suite la partie triviale et la partie non triviale
de l'exercice.}
\reponse{
Il est \'evident que, en tout point tel que  $\left|x\right|<\left|y\right|$
ou   $\left|x\right|>\left|y\right|$,
la fonction est continue et les d\'eriv\'ees
partielles existent.

Soit $x \ne 0$. Alors $f$ n'est ni continue en $(x,x)$ ni en $(x,-x)$.
Car
\begin{align*}
\mathrm{lim}_{\begin{smallmatrix} (u,v) \to (x,x)\\ |u|>|v| \end{smallmatrix}}f(u,v)
&= \mathrm{lim}_{u \to x}u = x \ne 0,
\\
\mathrm{lim}_{(u,u) \to (x,x)} f(u,u) &=0,
\\
\mathrm{lim}_{\begin{smallmatrix} (u,v) \to (x,-x)\\ |u|>|v| \end{smallmatrix}}f(u,v)
&= \mathrm{lim}_{u \to x}u = x \ne 0,
\\
\mathrm{lim}_{(u,-u) \to (x,-x)} f(u,u) &=0.
\end{align*}
Par contre, $f$ est continue en $(0,0)$. Car
\[
\mathrm{lim}_{(u,v) \to (0,0)}f(u,v)=0
\]
puisque
\[
\begin{array}{lll}
 f(u, v)=u &\mathrm{si}&
\left|u\right|>\left|v\right|,\\ 
f(u, v)=v &\mathrm{ si }&\left|u\right|<\left|v\right|,\\ 
f(u, v)= 0 &\mathrm{ si }&
\left|u\right| = \left|v\right|, 
\end{array}
\]
et puisque alors $\mathrm{lim}_{u \to 0}u=0$ et $\mathrm{lim}_{v \to 0}v=0$.

Soit  $(x,y)$ un point o\`u 
$\left|x\right| = \left|y\right|$.
Il reste \`a \'etudier les d\'eriv\'ees partielles en un tel point $(x,y)$.
Soit $x \ne 0$.
Alors la fonction $h$ de la variable $t$ d\'efinie par
\[
h(t)=f(x+t,y)=\begin{cases} x+t,\quad & |x+t|>|y|\\ y,\quad & |x+t|<|y| \end{cases}
\]
n'est pas d\'erivable en $t=0$ donc la d\'eriv\'ee partielle
$\frac{\partial f}{\partial x}(x,y)$ n'existe pas.
De m\^eme,
la fonction $k$ de la variable $t$ d\'efinie par
\[
k(t)=f(x,y+t)=\begin{cases} x,\quad & |x|>|y+t|,\\ y+t,\quad & |x|<|y+t|, \end{cases}
\]
n'est pas d\'erivable en $t=0$ donc la d\'eriv\'ee partielle
$\frac{\partial f}{\partial y}(x,y)$ n'existe pas.
Enfin soit $x = 0$.
Alors la fonction $h$ de la variable $t$ d\'efinie par
\[
h(t)=f(t,0)=t
\]
est d\'erivable en $t=0$ donc la d\'eriv\'ee partielle
$\frac{\partial f}{\partial x}(0,0)$ existe.
De m\^eme,
la fonction $k$ de la variable $t$ d\'efinie par
\[
k(t)=f(0,t)=t
\]
est d\'erivable en $t=0$ donc la d\'eriv\'ee partielle
$\frac{\partial f}{\partial y}(0,0)$ existe.
}
}
