\uuid{NuY8}
\exo7id{1261}
\auteur{gourio}
\datecreate{2001-09-01}
\isIndication{false}
\isCorrection{false}
\chapitre{Développement limité}
\sousChapitre{Applications}

\contenu{
\texte{
Soient $u,v,f$ d\'{e}finies par :
$$u(x)=(x^{3}-2x^{2}+1)^{\frac{1}{3}},\ v(x)=\sqrt{x^{2}+x+1},\ f(x)=u(x)-v(x).$$
}
\begin{enumerate}
    \item \question{Donner un \'{e}quivalent de $f$ au voisinage de $-\infty $, en d\'{e}duire
$\lim\limits_{-\infty }f.$}
    \item \question{D\'{e}teminer $\lim\limits_{x\rightarrow -\infty
}u(x)-x,\lim\limits_{x\rightarrow -\infty }v(x)+x.$ En d\'{e}duire
l'\'{e}quation d'une droite asymptote au graphe de $f $ en $-\infty $ et
positionner $f$ par-rapport \`{a} cette asymptote.}
    \item \question{M\^{e}me \'{e}tude en $+\infty $.}
\end{enumerate}
}
