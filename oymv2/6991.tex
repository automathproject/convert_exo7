\uuid{6991}
\auteur{blanc-centi}
\datecreate{2015-07-04}
\isIndication{false}
\isCorrection{true}
\chapitre{Equation différentielle}
\sousChapitre{Résolution d'équation différentielle du premier ordre}

\contenu{
\texte{
Résoudre sur $\R$ les équations différentielles suivantes:
}
\begin{enumerate}
    \item \question{$y'+2y=x^2$ $(E_1)$}
\reponse{Il s'agit d'une équation différentielle linéaire d'ordre 1, à coefficients constants, avec second membre.\\ 

On commence par résoudre l'équation homogène associée $y'+2y=0$: les solutions sont les $y(x)=\lambda e^{-2x}$, $\lambda\in\R$.\\

Il suffit ensuite de trouver une solution particulière de $(E_1)$. Le second membre étant polynomial de degré 2, on cherche une solution particulière de la m\^eme forme: \\
$y_0(x)=ax^2+bx+c$ est solution de $(E_1)$\\
\ \ \ $\Longleftrightarrow \forall x\in\R,\ y_0'(x)+2y_0(x)=x^2$ \\
\ \ \  $\Longleftrightarrow \forall x\in\R,\ 2ax^2+(2a+2b)x+b+2c=x^2$ 

\noindent Ainsi, en identifiant les coefficients, on voit que $y_0(x)=\frac{1}{2}x^2-\frac{1}{2}x+\frac{1}{4}$ convient.

Les solutions de $(E_1)$ sont obtenues en faisant la somme de cette solution particulière et des solutions 
de l'équation homogène: 
$$y(x)=\frac{1}{2}x^2-\frac{1}{2}x+\frac{1}{4}+\lambda e^{-2x}\quad (x\in\R)$$
où $\lambda$ est un paramètre réel.}
    \item \question{$y'+y=2\sin x$ $(E_2)$}
\reponse{Il s'agit d'une équation différentielle linéaire d'ordre 1, à coefficients constants, avec second membre.\\ 


Les solutions de l'équation homogène associée $y'+y=0$ sont les $y(x)=\lambda e^{-x}$, $\lambda\in\R$.\\

Il suffit ensuite de trouver une solution particulière de $(E_2)$. Le second membre est cette fois une fonction trigonométrique, on cherche une solution particulière sous la forme d'une combinaison linéaire de $\cos$ et $\sin$:\\
 $y_0(x)=a\cos x+b\sin x$ est solution de $(E_2)$\\
$\Longleftrightarrow \forall x\in\R,\ y_0'(x)+y_0(x)=2\sin x$ \\
$\Longleftrightarrow \forall x\in\R,\ (a+b)\cos x+(-a+b)\sin x=2\sin x$ 

\noindent Ainsi, en identifiant les coefficients, on voit que $y_0(x)=-\cos x+\sin x$ convient.

Les solutions de $(E_2)$ sont obtenues en faisant la somme de cette solution particulière et des solutions 
de l'équation homogène:
$$y(x)=-\cos x+\sin x+\lambda e^{-x}\quad (x\in\R)$$
où $\lambda$ est un paramètre réel.}
    \item \question{$y'-y=(x+1)e^x$ $(E_3)$}
\reponse{Les solutions de l'équation homogène associée $y'-y=0$ sont les $y(x)=\lambda e^{x}$, $\lambda\in\R$. On remarque que le second membre est le produit d'une fonction exponentielle par une fonction polynomiale de degré $d=1$: or la fonction exponentielle du second membre est la m\^eme ($e^x$) que celle qui appara\^it dans les solutions de l'équation homogène. On cherche donc une solution particulière sous la forme d'un produit de $e^x$ par une fonction polynomiale de degré $d+1=2$:\\
$y_0(x)=(ax^2+bx+c)e^x$ est solution de $(E_3)$ \\
$\Longleftrightarrow \forall x\in\R,\ y_0'(x)-y_0(x)=(x+1)e^x$ \\
$\Longleftrightarrow \forall x\in\R,\ (2ax+b)e^x=(x+1)e^x$ 

\noindent Ainsi, en identifiant les coefficients, on voit que $y_0(x)=(\frac{1}{2}x^2+x)e^x$ convient.

Les solutions de $(E_3)$ sont obtenues en faisant la somme de cette solution particulière et des solutions de l'équation homogène: 
$$y(x)=(\frac{1}{2}x^2+x+\lambda)e^{x}\quad (x\in\R)$$
où $\lambda$ est un paramètre réel.}
    \item \question{$y'+y=x-e^x+\cos x$ $(E_4)$}
\reponse{Les solutions de l'équation homogène associée $y'+y=0$ sont les $y(x)=\lambda e^{-x}$, $\lambda\in\R$. On remarque que le second membre est la somme d'une fonction polynomiale de degré 1, d'une fonction exponentielle (différente de $e^{-x}$) et d'une fonction trigonométrique. D'après le principe de superposition, on cherche donc une solution particulière sous la forme d'une telle somme:\\
$y_0(x)=ax+b+\mu e^x+\alpha\cos x+\beta\sin x$ est solution de $(E_4)$ \\
$\Longleftrightarrow \forall x\in\R,\ y_0'(x)+y_0(x)=x-e^x+\cos x$ \\
$\Longleftrightarrow \forall x\in\R,\ ax+a+b+2\mu e^x+(\alpha+\beta)\cos x+(-\alpha +\beta)\sin x=x-e^x+\cos x$ 

\noindent Ainsi, en identifiant les coefficients, on voit que $$y_0(x)=x-1-\frac{1}{2}e^x+\frac{1}{2}\cos x+\frac{1}{2}\sin x$$ convient.

Les solutions de $(E_4)$ sont obtenues en faisant la somme de cette solution particulière et des solutions de l'équation homogène:
$$y(x)=x-1-\frac{1}{2}e^x+\frac{1}{2}\cos x+\frac{1}{2}\sin x+\lambda e^{-x}\quad (x\in\R)$$
où $\lambda$ est un paramètre réel.}
\end{enumerate}
}
