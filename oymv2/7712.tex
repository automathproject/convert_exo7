\uuid{7712}
\auteur{mourougane}
\datecreate{2021-08-11}
\isIndication{false}
\isCorrection{true}
\chapitre{Sous-variété}
\sousChapitre{Sous-variété}

\contenu{
\texte{

}
\begin{enumerate}
    \item \question{Déterminer les plans tangents et un champs de vecteurs normaux unitaires
au paraboloïde hyperbolique $\mathcal{P}$ d'équation $z=y^2-x^2$ au voisinage du point $A(0,0,0)$
de coordonnées $(0,0,0)$.}
\reponse{Par l'exercice précédent, $\mathcal{P}$ a pour plan tangent au point $M(x,y,z)$
le plan d'équation $Z=2yY-2xX$. 
En particulier, le plan tangent en $A$ est le plan d'équation $Z=0$
engendré par les deux vecteurs de base $\begin{pmatrix} 1\\0\\0\end{pmatrix}$ et $\begin{pmatrix} 0\\1\\0\end{pmatrix}$.
Un champs de vecteurs normaux unitaires est donc
$$N(x,y,y)=\frac{1}{\sqrt{4x^2+4y^2+1}}\begin{pmatrix} 2x\\-2y\\1 \end{pmatrix}.$$}
    \item \question{Déterminer l'application de Weingarten au point $A$ du paraboloïde hyperbolique $\mathcal{P}$.}
\reponse{L'endomorphisme de Weingarten est donné au point $A(0,0,0)$
par $W_A=-dN(A) : T_A\mathcal{P}\to T_A\mathcal{P}$ 
$$\begin{array}{c}   \begin{pmatrix} 1\\0\\0\end{pmatrix} 
\mapsto -\begin{pmatrix} 2\\0\\0 \end{pmatrix}\\
\begin{pmatrix} 0\\1\\0\end{pmatrix}\mapsto\begin{pmatrix}0\\2\\0\end{pmatrix}
 \end{array}$$}
    \item \question{En déduire la courbure de Gauss et les directions principales du paraboloïde hyperbolique $\mathcal{P}$ au point $A$.}
\reponse{Ses valeurs propres sont donc $-2$ et $2$. La courbure de Gauss de $\mathcal{P}$ en $A$ est donc $-4$.
Les directions propres sont donc les axes de coordonnées $x$ et $y$.}
\end{enumerate}
}
