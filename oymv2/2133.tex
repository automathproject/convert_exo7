\uuid{b5bA}
\exo7id{2133}
\auteur{debes}
\datecreate{2008-02-12}
\isIndication{false}
\isCorrection{true}
\chapitre{Ordre d'un élément}
\sousChapitre{Ordre d'un élément}

\contenu{
\texte{
Soit $G$ un groupe d'ordre $2p$ avec $p$ un nombre premier. Montrer qu'il existe un \'el\'ement d'ordre $2$ et un \'el\'ement d'ordre $p$.
}
\reponse{
Si $p=2$ alors $|G|$ est d'ordre $4$: $G$ est le groupe de Klein $(\Zz/2\Zz)^2$ dont tous les \'el\'ements diff\'erents de $1$ sont d'ordre $2$. On peut donc supposer pour la suite que $p$ est impair. En proc\'edant comme dans l'exercice  \ref{ex:deb32}, on montre qu'il existe forc\'ement dans $G$ un \'el\'ement d'ordre $2$. Enfin si tous les \'el\'ements diff\'erents de $1$  \'etaient d'ordre $2$, alors d'apr\`es l'exercice \ref{ex:deb14}, l'ordre de $G$ serait une puissance de $2$. Il existe donc aussi un \'el\'ement d'ordre $p$.
}
}
