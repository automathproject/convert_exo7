\uuid{vKq2}
\exo7id{6870}
\auteur{chataur}
\datecreate{2012-05-13}
\isIndication{true}
\isCorrection{true}
\chapitre{Espace vectoriel}
\sousChapitre{Système de vecteurs}

\contenu{
\texte{

}
\begin{enumerate}
    \item \question{Soient $v_1=(2,1,4)$, $v_2=(1,-1,2)$ et $v_3=(3,3,6)$ des vecteurs de $\Rr^3$, 
trouver trois r\'eels non tous nuls $\alpha,\beta,\gamma$ tels que $\alpha v_1+ \beta v_2 + \gamma v_3=0$.}
\reponse{\begin{align*}
     & \alpha v_1 + \beta v_2 + \gamma v_3 = 0 \\ 
\iff & \alpha (2,1,4) + \beta (1,-1,2) + \gamma (3,3,6) = (0,0,0) \\
\iff &  \Big(2\alpha+\beta+3\gamma,\alpha-\beta+3\gamma,4\alpha+2\beta+6\gamma\Big) = (0,0,0) \\
\iff &
\begin{cases}
  2\alpha+\beta+3\gamma &= 0 \\
  \alpha-\beta+3\gamma  &= 0 \\
  4\alpha+2\beta+6\gamma &= 0 \\
 \end{cases} \\
\iff & \cdots  \qquad  \text{(on résout le système)} \\
\iff & \alpha=-2t, \beta = t, \gamma = t \quad t \in \Rr \\
\end{align*}  

Si l'on prend $t=1$ par exemple alors $\alpha=-2$, $\beta = 1$, $\gamma = 1$
donne bien $-2v_1+v_2+v_3=0$.

Cette solution n'est pas unique, les autres coefficients qui conviennent sont les 
$(\alpha=-2t, \beta = t, \gamma = t)$ pour tout $t \in \Rr$.}
    \item \question{On considère deux plans vectoriels
$$P_1=\{(x,y,z) \in \Rr^3 \mid x-y+z=0\}$$
$$P_2=\{(x,y,z) \in \Rr^3 \mid x-y=0\}$$
trouver un vecteur directeur de la droite $D=P_1\cap P_2$ ainsi qu'une \'equation param\'etr\'ee.}
\reponse{Il s'agit donc de trouver un vecteur $v=(x,y,z)$ dans $P_1$ et $P_2$ et donc qui doit vérifier 
$x-y+z=0$ et $x-y=0$ :

\begin{align*}
     & v=(x,y,z) \in P_1 \cap P_2 \\ 
\iff & x-y+z=0 \text{ et } x-y=0 \\
\iff &
\begin{cases}
  x-y-z = 0 \\
  x-y = 0 \\
 \end{cases} \\
\iff & \cdots \qquad  \text{(on résout le système)} \\
\iff & (x=t, y = t, z = 0) \quad t \in \Rr \\
\end{align*}  

Donc, si l'on fixe par exemple $t=1$, alors $v=(1,1,0)$ 
est un vecteur directeur de la droite vectorielle $D$,
une équation paramétrique étant $D=\{(t,t,0) \mid t\in \Rr \}$.}
\indication{\begin{enumerate}
  \item On pensera \`a poser un syst\`eme.
  \item Trouver un vecteur non-nul commun aux deux plans.
\end{enumerate}}
\end{enumerate}
}
