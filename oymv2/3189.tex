\uuid{3189}
\auteur{quercia}
\datecreate{2010-03-08}
\isIndication{false}
\isCorrection{true}
\chapitre{Polynôme, fraction rationnelle}
\sousChapitre{Autre}

\contenu{
\texte{
Soient $x_1,\dots,x_n\in\C$ distincts et $y_1,\dots,y_n\in\C$.
Trouver $E = \{P\in\C[X]\text{ tq }\forall\ i,\ P^{-1}(\{y_i\}) = \{x_i\}\}$.
}
\reponse{
Clairement $E=\varnothing$ si les $y_i$ ne sont pas distincts.
Si $y_1,\dots,y_n$ sont distincts, soit $P\in E$, $n=\deg(P)$ et $\lambda$
le coefficient dominant de~$P$ ($P\ne0$ car les $y_i$ ne sont pas tous nuls).
Alors $P(X) - y_i$ a pour seule racine $x_i$ donc $P(X) - y_i = \lambda(X-x_i)^n$.
Pour $n=1$ on obtient $P(X) = y_1 + \lambda(X-x_1)$ avec $\lambda\in\C^*$.
Pour $n\ge2$ on obtient $y_2-y_1 = \lambda(X-x_1)^n - \lambda(X-x_2)^n = n\lambda X^{n-1}(x_2-x_1) + \dots$
ce qui est impossible donc $E=\varnothing$.
}
}
