\uuid{OnQ4}
\exo7id{7227}
\auteur{megy}
\datecreate{2021-03-06}
\isIndication{false}
\isCorrection{false}
\chapitre{Formule de Cauchy}
\sousChapitre{Formule de Cauchy}

\contenu{
\texte{
Représenter graphiquement les chemins suivants de \(\C\) :
}
\begin{enumerate}
    \item \question{\(\gamma(t)=t+it\), \(t\in [0,1]\),}
    \item \question{\(\gamma(t)=t^2-it\), \(t\in [0,1]\),}
    \item \question{\(\gamma(t)=|t|+it\), \(t\in [-1,1]\),}
    \item \question{\(\gamma=\gamma_1\vee\gamma_2\vee\gamma_3\) où : \(\gamma_1(t)= it^2\), \(t\in [0,1]\); \(\gamma_2(t)=t+(1-t)i\), \(t\in [0,1]\); \(\gamma_3(t)=e^{-it}\), \(t\in [0,\pi]\).}
\end{enumerate}
}
