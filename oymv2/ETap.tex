\uuid{ETap}
\exo7id{2761}
\auteur{tumpach}
\datecreate{2009-10-25}
\isIndication{false}
\isCorrection{false}
\chapitre{Réduction d'endomorphisme, polynôme annulateur}
\sousChapitre{Valeur propre, vecteur propre}

\contenu{
\texte{

}
\begin{enumerate}
    \item \question{\begin{enumerate}}
    \item \question{Soit $f~:\mathbb{R}^2\rightarrow \mathbb{R}^2$ l'application lin\'eaire d\'efinie par 
$$
f\left(\begin{array}{c}x\\y\end{array}\right) =\frac{1}{5} \left(\begin{array}{c}3x + 4y\\ 4x - 3y\end{array}\right).
$$}
    \item \question{\'Ecrire la matrice de $f$ dans la base canonique de $\mathbb{R}^2$. On la notera $A$.}
    \item \question{Montrer que le vecteur $\vec{v}_1 = \left(\begin{array}{c}2 \\1\end{array}\right)$ est vecteur propre de $f$. Quelle est la valeur propre associ\'ee~?}
    \item \question{Montrer que le vecteur $\vec{v}_2 = \left(\begin{array}{c}-1 \\2\end{array}\right)$ est \'egalement vecteur propre de $f$. Quelle est la valeur propre associ\'ee~?}
    \item \question{Calculer graphiquement l'image du vecteur $\vec{v}_3 = \left(\begin{array}{c}1\\ 3\end{array}\right).$ Retrouver ce r\'esultat par le calcul.}
    \item \question{Montrer que la famille $\{\vec{v}_1, \vec{v}_2\}$ forme une base de $\mathbb{R}^2$.}
    \item \question{Quelle est la matrice de $f$ dans la base $\{\vec{v}_1, \vec{v}_2\}$~? On la notera $D$.}
    \item \question{Soit $P$ la matrice  dont la premi\`ere colonne est le vecteur $\vec{v}_1$ et dont la deuxi\`eme colonne est le vecteur $\vec{v}_2$. Calculer $P^{-1}$.}
    \item \question{Quelle relation y-a-t-il entre $A$, $P$, $P^{-1}$ et $D$~?}
    \item \question{Calculer $A^n$, pour $n\in\mathbb{N}$.}
\end{enumerate}
}
