\uuid{9wBY}
\exo7id{2853}
\auteur{burnol}
\datecreate{2009-12-15}
\isIndication{false}
\isCorrection{true}
\chapitre{Théorème des résidus}
\sousChapitre{Théorème des résidus}

\contenu{
\texte{
Prouver pour $a>1$:
\[ \frac1{2\pi}\int_0^{2\pi} \frac{\sin\theta}{a+\sin\theta}d\theta =
\frac{\sqrt{a^2-1} - a}{\sqrt{a^2 - 1}}\]
En utilisant l'un des exercices précédents montrer que la
formule a un sens et est valable pour $a\in\Cc\setminus[-1,+1]$.
}
\reponse{
Si $z=e^{i\theta}$, alors $\sin \theta =\frac{z-\overline{z}}{2i}$ et $dz = ie^{i\theta} d\theta$.
D'o\`u
$$\frac{1}{2\pi} \int_0^{2\pi} \frac{\sin \theta}{a+\sin \theta} d\theta = \frac{1}{2\pi} \int_{ |z|=1}
\frac{z-\overline{z}}{2ia +z-\overline{z}}\frac{dz}{iz}=
\frac{1}{2i\pi} \int_{ |z|=1} \frac{z^2-1}{z^2 +2iaz -1}\frac{dz}{z}.$$
Il suffit alors d'utiliser le th\'eor\`eme des r\'esidus.
}
}
