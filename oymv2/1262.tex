\uuid{1262}
\auteur{legall}
\datecreate{2003-10-01}

\contenu{
\texte{
Soit $g$ la
fonction $x\mapsto \dfrac{\arctan x}{(\sin x)^3}-\dfrac {1}{x^2}$.
}
\begin{enumerate}
    \item \question{Donner le domaine de d\'efinition de $g$.}
    \item \question{Montrer qu'elle se prolonge par continuit\'e en $0$ en
une fonction
d\'erivable.}
    \item \question{D\'eterminer la tangente en $0$ au graphe de cette fonction et la
position de ce graphe par rapport \`a celle-ci.}
\reponse{
La fonction
$g$ est d\'efinie en $x$ sauf si $\sin (x)=0$ ou $x=0$.
Son domaine de d\'efinition est donc $\Rr -\{ k\pi , k\in \Zz \} .$
On peut prolonger $g$ en une fonction continue en $0$ si et seulement si elle y
admet une limite.
Elle est d\'erivable en ce point si et seulement si elle y admet un 
d\'eveloppement limit\'e \`a l'ordre
$1$. Toutefois, comme l'\'enonc\'e demande la position du graphe de 
$g$ par rapport
\`a sa tangente en $0$, nous allons calculer directement le 
d\'eveloppement limit\'e \`a l'ordre
$2$ de $g$ en $0$.



Le d\'eveloppement limit\'e en $0$ \`a l'ordre $5$ de $\arctan x = x 
-\dfrac{x^3}{3} +\dfrac{x^5}{5}+x^5\epsilon_1 (x).$

Or $\sin x =x-\dfrac{x^3}{3!} +\dfrac{x^5}{5!}+x^5\epsilon_2 (x).$ Donc
$\sin ^3 x =x^3-\dfrac{x^5}{2} + \dfrac{13x^7}{120}   +x^7\epsilon_3 (x)$
et $\dfrac{1}{\sin ^3 x}=\dfrac{1}{x^3} 
(1+\dfrac{x^2}{2}+\dfrac{9x^4}{40} +x^4\epsilon_4 (x)).$
On en d\'eduit que :

$ \dfrac{\arctan x}{(\sin x)^3}-\dfrac {1}{x^2}=\dfrac 
{1}{x^3}(x+\dfrac{x^3}{6}+\dfrac{31x^5}{120}+x^5\epsilon_5 (x))
-\dfrac {1}{x^2}=\dfrac{1}{6}+\dfrac{31x^2}{120}+x^2\epsilon_5 (x)
.$

Ainsi on peut prolonger $g$ en une fonction continue en $0$ en posant 
$g(0)=\dfrac{1}{6}$.
La fonction obtenue est d\'erivable en $0$ et sa d\'eriv\'ee est 
nulle. La tangente en $0$ \`a son graphe est la droite d'\'equation
$y=\dfrac{1}{6}$. Enfin le graphe de $g$ est au-dessus de cette 
droite au voisinage de $0$.
}
\end{enumerate}
}
