\uuid{6772}
\auteur{gijs}
\datecreate{2011-10-16}
\isIndication{false}
\isCorrection{false}
\chapitre{Espace topologique, espace métrique}
\sousChapitre{Espace topologique, espace métrique}

\contenu{
\texte{

}
\begin{enumerate}
    \item \question{Donner les définitions
d'une topologie, d'un espace topologique Hausdorff, d'un
espace topologique quasi compact, d'un espace topologique
compact, d'un espace topologique connexe, et d'un espace
topologique connexe par arcs.}
    \item \question{Donner un exemple d'un espace topologique connexe
mais pas connexe par arcs (une démonstration n'est pas
demandée !).}
    \item \question{Soit $A\subset \Rr^n$. Montrer que $A$ est
quasi compact si et seulement si $A$ est fermé et borné
(énoncer clairement les théorèmes utilisés!).}
    \item \question{Soit $X$ un espace topologique. Montrer que si
$X$ ne contient qu'un seul élément, alors $X$ est
compact.}
    \item \question{Soit $(X,d)$ un espace métrique. Donner la
définition d'une suite de Cauchy dans $X$. Sous quelle
condition dit-on qu'une suite $x_n$ converge vers $a\in
X$ (noté $x_n \to a$)~?}
    \item \question{Soit $(X,d)$ un espace métrique et $x_n$ une
suite dans $X$. Montrer que (1) si $x_n$ est une suite de
Cauchy, alors $x_n$ est bornée, et (2) si $x_n \to a$,
alors $x_n$ est une suite de Cauchy.}
\end{enumerate}
}
