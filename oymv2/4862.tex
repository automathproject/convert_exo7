\uuid{4862}
\auteur{quercia}
\datecreate{2010-03-17}
\isIndication{false}
\isCorrection{true}
\chapitre{Géométrie affine dans le plan et dans l'espace}
\sousChapitre{Sous-espaces affines}

\contenu{
\texte{
Dans ${\cal E}_3$ muni d'un repère $(O,\vec i,\vec j,\vec k\,)$, on donne :
$D : \begin{cases} x - 2z = 1 \cr y - z = 2 \cr\end{cases}$
et $D' : \begin{cases} x + y + z = 1 \cr x - 2y + 2z = a.\cr\end{cases}$
}
\begin{enumerate}
    \item \question{Pour quelles valeurs de $a$, $D$ et $D'$ sont-elles coplanaires ?}
\reponse{$a=-4$}
    \item \question{Donner alors l'équation du plan contenant $D$ et $D'$.}
\reponse{$x-5y+3z = -9$}
\end{enumerate}
}
