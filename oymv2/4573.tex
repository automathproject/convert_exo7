\uuid{SzEg}
\exo7id{4573}
\auteur{quercia}
\datecreate{2010-03-14}
\isIndication{false}
\isCorrection{true}
\chapitre{Série entière}
\sousChapitre{Rayon de convergence}

\contenu{
\texte{
Pour $n\in\N$ et $x\in\R$ on pose $u_n(x) = \Bigl(\frac{x(1-x)}2\Bigr)^{4^n}$.
}
\begin{enumerate}
    \item \question{Déterminer le domaine de convergence de la série $\sum_{n=0}^\infty u_n(x)$.}
\reponse{$]-1,2[$.}
    \item \question{On développe $u_n(x)$ par la formule du binôme~:
    $u_n(x) = \sum_{4^n\le k \le 2.4^n}a_kx^k$. Montrer que le
    rayon de convergence de la série entière $\sum_{k\ge 1} a_kx^k$
    est égal à~$1$ (en convenant que les $a_k$ non définis valent zéro).}
\reponse{Pour $0\le k\le 4^n$, on a $|a_k| \le C_{4^n}^{4^n/2}\Bigm/2^{4^n}$
(atteint pour $k=4^n/2$).

Donc $a_n\to0$ lorsque $n\to\infty$ et si $x>1$ alors
$a_{3*4^n/2}x^{3*4^n/2}$
\hbox to 0pt{\hskip5mm $\not{}$\hss}
$\to 0$ lorsque $n\to\infty$.}
\end{enumerate}
}
