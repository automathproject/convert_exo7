\uuid{2191}
\auteur{debes}
\datecreate{2008-02-12}
\isIndication{false}
\isCorrection{true}
\chapitre{Théorème de Sylow}
\sousChapitre{Théorème de Sylow}

\contenu{
\texte{
Soit $G$ un $p$-groupe et $H$ un sous-groupe distingu\'e de $G$. 
Montrer que $H\cap Z(G)$ n'est pas r\'eduit \`a l'\'el\'ement neutre.
}
\reponse{
Le sous-groupe $H\subset G$ \'etant distingu\'e, $G$ agit par conjugaison sur $H$.
Comme $G$ est un $p$-groupe, $H$ l'est aussi et les orbites non triviales de cette
action sont de longueur divisible par $p$. On d\'eduit que la r\'eunion des orbites
triviales, c'est-\`a-dire l'ensemble $H\cap Z(G)$ des points fixes, est aussi de
cardinal divisible par $p$. Comme il contient l'\'el\'ement neutre, il contient au
moins $p$ \'el\'ements et n'est donc pas r\'eduit \`a l'\'el\'ement neutre.
}
}
