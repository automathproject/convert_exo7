\uuid{2507}
\auteur{queffelec}
\datecreate{2009-04-01}
\isIndication{false}
\isCorrection{true}
\chapitre{Différentiabilité, calcul de différentielles}
\sousChapitre{Différentiabilité, calcul de différentielles}

\contenu{
\texte{
Soit $f$ une application
diff\'erentiable de $\Rr^2$ dans lui-m\^eme, propre (i.e.
$||f(x)||$ tend vers $\infty$ quand $||x||\to\infty$), telle que
pour tout $x\in\Rr^2$ $Df(x)$ soit injective. On va montrer que
$f$ est surjective. Soit $a\in\Rr^2$ et $g(x)=||f(x)-a||^2$;
}
\begin{enumerate}
    \item \question{Calculer $Dg(x)$.}
\reponse{On a $g(x,y)=<f(x,y)-a,f(x,y)-a>$ o\`u $<.,.>$ est le
produit scalaire Euclidien sur $\mathbb{R}^2$. L'application $g$
est diff\'erentiable en tant que compos\'ee et produit de
fonctions diff\'erentiables. La diff\'erentielle $Df$ est donn\'e
par la matrice Jacobienne
$$(\frac{\partial f(x,y)}{\partial x}, \frac{\partial
f(x,y)}{\partial y})$$ et $Dg$ par la matrice
$$(\frac{\partial g(x,y)}{\partial x}, \frac{\partial
g(x,y)}{\partial y})$$On a alors
$$\frac{\partial g(x,y)}{\partial
x}=\frac{\partial}{\partial x} <f(x,y)-a,f(x,y)-a>=$$
$$<\frac{\partial}{\partial
x}(f(x,y)-a),f(x,y)-a>+<f(x,y)-a,\frac{\partial}{\partial
x}(f(x,y)-a)>=$$
$$2<\frac{\partial f(x,y)}{\partial x},f(x)-a>.$$
De m\^eme, $$\frac{\partial g(x,y)}{\partial y}=2<\frac{\partial
f(x,y)}{\partial y},f(x)-a>.$$}
    \item \question{Montrer que $g$ atteint sa borne inf\'erieure en un point
$x_0$ de $\Rr^2$, et que $Dg(x_0)=0$; en d\'eduire le r\'esultat.}
\reponse{L'application $f$ est
continue (car diff\'erentiable) et tend vers l'infini quand
$(x,y)$ tend vers l'infini. Ainsi
$$\forall A >0, \exists B >0, ||(x,y)|| \geq B \Rightarrow
||f(x,y)||\geq A.$$ Soit $m=\inf_{(x,y)\in \mathbb{R}^2} g(x,y)$,
pour $A=m+1$, il existe $B >0$ tel que $$||(x,y)|| \geq B
\Rightarrow g(x,y)=||f(x,y)||^2 \geq A^2\geq (m+1)^2 \geq m+1.$$
On a donc $$m=\inf_{(x,y)\in \mathbb{R}^2} g(x,y)=\inf_{||(x,y)||
\leq B} g(x,y).$$ Or la boule $\overline B(0,B)$ \'etant compacte
et $g$ continue, l'inf y est ateint en un point $X_0=(x_0,y_0) \in
B(0,B) \subset \mathbb{R}^2$. Comme $X_0$ est un minimum global de
$g$, c'est aussi un minimum de la restriction de $g$ sur toute
droite passant par $X_0$. Comme la d\'eriv\'e d'une fonction
r\'eelle en un minimum est nulle, toute les d\'eriv\'ees
partielles de $g$ sont nulles et donc $Dg(X_0)=0$ et par
cons\'equent la matrice jacobienne de $g$ est nulle. On a donc
$$\frac{\partial g}{\partial x}(x_0,y_0)=2<\frac{\partial
f}{\partial x}(x_0,y_0),f(x)-a>=0 \mbox{ et } \frac{\partial
g}{\partial y}(x_0,y_0)=2<\frac{\partial f}{\partial
y}(x_0,y_0),f(x)-a>=0.$$ Comme $Df$ est injective, ses colonnes
forment une base de $\mathbb{R}^2$. Par cons\'equent les
projections de $f(x)-a$ sur la base $(\frac{\partial f}{\partial
x}(x_0,y_0);\frac{\partial f}{\partial x}(x_0,y_0))$ sont nulles
et donc $$f(x_0,y_0)-a=0 \Leftrightarrow f(x_0,y_0)=a$$ et donc
$a$ admet bien un ant\'ec\'edent. Ceci \'etant valable pour tout
$a\in \mathbb{R}^2$, on a montr\'e que $f$ est surjective.}
\end{enumerate}
}
