\uuid{hJ1M}
\exo7id{2558}
\auteur{tahani}
\datecreate{2009-04-01}
\isIndication{false}
\isCorrection{true}
\chapitre{Solution maximale}
\sousChapitre{Solution maximale}

\contenu{
\texte{
On consid\`ere l'\'equation
$$x'=3x^{2/3} \mbox{                          :} (1)$$
avec condition initiale $x(0)=0$.
}
\begin{enumerate}
    \item \question{Soit $\varphi$ une solution de $(1)$ d\'efinie sur
$\mathbb{R}$ telle que $\varphi(0)=0$; on pose $\lambda=\inf \{ t
\leq 0; \varphi(t)=0\} \leq +\infty$. Montrez que $\varphi$ est
identiquement nulle sur $(\lambda,\mu)$.}
\reponse{Soit $\lambda_n$ et $\mu_n$ deux suites de points tels que
$\varphi(\lambda_n)=\varphi(\mu_n)=0$ et convergeant
respectivement vers $\lambda$ et $\mu$, il reste \`a montrer que
$\varphi$ est nulle sur chaque interval $[\lambda_n,\mu_n]$. Soit
$c$ un extremum de $\varphi$ sur cet interval, on a alors
necessairement $\varphi'(c)=0$ et donc $3c^{2/3}=0$ et donc $c=0$
et donc $\varphi(c)=\varphi(0)=0$. Par cons\'equent le sup et le
min de $\varphi$ sur $[\lambda_n,\mu_n]$ sont nuls et donc
$\varphi$ est aussi nulle sur cet intervalle. En passant \`a la
limite, on a prouv\'e que $\varphi$ est nulle sur $]\lambda,\mu[$.}
    \item \question{Montrer que
$\varphi$ vaut $(t-\lambda)^3$ si $t\leq \lambda$, $0$ sur
$[\lambda, \mu]$ et $(t-\mu)^3$ si $t \geq \mu$; en d\'eduire
toutes les solutions maximales de $(1)$ d\'efinies sur
$\mathbb{R}$ avec $x(0)=0$.}
\reponse{On v\'erifie que les solutions propos\'ees v\'erifient
l'\'equation diff\'erentielle $(1)$. La fonction $x^{2/3}$ est
lipschitzienne par rapport \`a $x$ d\`es que $x\neq 0$. Si
$\varphi_2$ est une solution maximale sur $\mathbb{R}$ v\'erifiant
$\varphi_2(0)=0$, il existe alors n\'ecessairement $\lambda, \mu$
(d\'efinis pr\'ec\'edement) tels que $\varphi_2$ est nulle sur
$]\lambda,\mu[$. Par continuit\'e de la solution elle v\'erifie
$\varphi(\lambda)=\varphi(\mu)=0$. Mais alors
$\varphi_2'-\varphi=0$ est donc $\varphi_2=\varphi+K$ o\`u $K$ est
une constante donn\'ee. Du fait que
$\varphi_2(\lambda)=\varphi(\lambda)+K=0+K=0$, on a $K=0$ ce qui
termine la d\'emonstration.}
\end{enumerate}
}
