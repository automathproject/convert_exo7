\uuid{46}
\auteur{cousquer}
\datecreate{2003-10-01}
\isIndication{false}
\isCorrection{true}
\chapitre{Nombres complexes}
\sousChapitre{Racine n-ieme}

\contenu{
\texte{

}
\begin{enumerate}
    \item \question{Montrer que, pour tout $n\in\mathbb{N}^*$ et tout nombre $z\in\mathbb{C}$, on a:
$$\left(z-1\right)\left(1+z+z^2+...+z^{n-1}\right)=z^n-1,$$
et en déduire que, si $z\neq 1$, on a:
$$1+z+z^2+...+z^{n-1}=\frac{z^n-1}{z-1}.$$}
    \item \question{\label{Q2} Vérifier que pour tout $x\in\mathbb{R}$ , on a
$  
\exp(ix)-1=2i\exp\left(\frac{ix}{2}\right)\sin\left(\frac{x}{2}\right).$}
    \item \question{\label{Q3} Soit $n\in\mathbb{N}^*$. Calculer pour tout $x\in\mathbb{R}$ la somme:
$$Z_n=1+\exp(ix)+\exp(2ix)+...+\exp((n-1)ix),$$ 
et en déduire les valeurs de
\begin{eqnarray*}
X_n&=&1+\cos(x)+\cos(2x)+...+\cos((n-1)x)\\
Y_n&=&\sin(x)+\sin(2x)+...+\sin((n-1)x).
\end{eqnarray*}}
\reponse{
Pour 2. Utiliser la formule d'Euler pour $\sin\left(x/2\right)$.

Pour 3.
Pour $x\neq
2k\pi$, $k\in\mathbb{Z}$,
$$Z_n=\frac{\sin\left(\frac{nx}{2}\right)}{\sin\left(\frac{x}{2}\right)}\exp\left(i\left(n-1\right)\frac{x}{2}\right),$$
et pour $x=2k\pi$, $k\in\mathbb{Z}$, $Z_n=n$.

Remarquer que $Z_n=X_n+iY_n$ pour en déduire que
$$X_n=\frac{\cos\left(\frac{(n-1)x}{2}\right)\sin\left(\frac{nx}{2}\right)}{\sin\left(\frac{x}{2}\right)}\;\mbox{
et }\;Y_n=\frac{\sin\left(\frac{(n-1)x}{2}\right)\sin\left(\frac{nx}{2}\right)}{\sin\left(\frac{x}{2}\right)}.$$
}
\end{enumerate}
}
