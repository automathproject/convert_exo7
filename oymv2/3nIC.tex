\uuid{3nIC}
\exo7id{698}
\auteur{bodin}
\datecreate{1998-09-01}
\isIndication{true}
\isCorrection{true}
\chapitre{Dérivabilité des fonctions réelles}
\sousChapitre{Calculs}

\contenu{
\texte{
\'Etudier  la d\'erivabilit\'e des fonctions suivantes :
$$f_1(x)=x^2\cos \frac{1}{x}, \text{\ \  si }x\not=0 \qquad ; \qquad f_1(0)=0 ;$$

$$f_2(x)= \sin x \cdot \sin \frac{1}{x}, \text{\ \  si }x\not=0 \qquad ; \qquad f_2(0)=0 ;$$

$$f_3(x) = \frac{|x|\sqrt{x^2-2x+1}}{x-1}, \text{\ \  si } x\not= 1 \qquad ; \qquad f_3(1)=1.$$
}
\indication{Les probl\`emes sont seulement en $0$ ou $1$. 
$f_1$ est d\'erivable en $0$ mais pas $f_2$.  $f_3$ n'est dérivable ni en $0$, ni en $1$.}
\reponse{
La fonction $f_1$ est d\'erivable en dehors de $x=0$.
En effet $x \mapsto \frac 1x$ est dérivable sur $\Rr^*$ et $x \mapsto \cos x$ est dérivable sur $\Rr$,
donc par composition $x \mapsto  \cos \frac 1x$ est dérivable sur $\Rr^*$. Puis par multiplication par la fonction dérivable 
$x \mapsto x^2$, la fonction $f_1$ est dérivable sur $\Rr^*$. 
Par la suite on omet souvent ce genre de discussion ou on l'abrège sous la forme ``$f$ est dérivable sur $I$ comme somme, produit,
composition de fonctions dérivables sur $I$''.

Pour savoir si $f_1$ est d\'erivable en $0$ regardons le taux d'accroissement:
$$ \frac{f_1(x)-f_1(0)}{x-0}= x\cos \frac 1 x.$$
Mais $x \cos (1/x)$ tend vers $0$ (si $x\rightarrow 0$) car
$|\cos (1/x)| \leq 1$.
Donc le taux d'accroissement tend vers $0$. Donc $f_1$ est d\'erivable en $0$ et $f_1'(0)=0$.
Encore une fois $f_2$ est d\'erivable en dehors de $0$.
Le taux d'accroissement en $x=0$ est :
$$ \frac{f_2(x)-f_2(0)}{x-0}= \frac{\sin x}{x} \sin \frac 1 x$$
Nous savons que $\frac{\sin x}{x} \rightarrow 1$ et que
$\sin 1/x$ n'a pas de limite quand $x\rightarrow 0$. Donc le taux d'accroissement n'a pas de limite, donc $f_2$ n'est pas d\'erivable en $0$.
La fonction $f_3$ s'\'ecrit :
$$f_3(x) = \frac{|x||x-1|}{x-1}.$$
\begin{itemize}
Donc pour $x \geq 1$ on a $f_3(x) = x$ ;
pour $0 \leq x < 1$ on a $f_3(x) = -x$ ;
pour $x <0$ on a $f_3(x) = x$.
La fonction $f_3$ est d\'efinie, continue et d\'erivable sur 
$\Rr \setminus \{0,1\}$. Attention ! La fonction $x \mapsto |x|$ n'est pas dérivable en $0$.
La fonction $f_3$ n'est pas continue en $1$, en effet
$\lim_{x \rightarrow 1+} f_3(x) = +1$ et  $\lim_{x \rightarrow 1-} f_3(x) = -1$. Donc la fonction n'est pas d\'erivable en $1$.
La fonction $f_3$ est continue en $0$. 
Le taux d'accroissement pour $x>0$ est 
$$\frac{f_3(x)-f_3(0)}{x-0}= \frac{-x}{x} = -1$$
et pour $x <0$, 
$$\frac{f_3(x)-f_3(0)}{x-0}= \frac{x}{x} = +1.$$
Donc le taux d'accroissement n'a pas de limite en $0$ et donc $f_3$ n'est pas d\'erivable en $0$.
\end{itemize}
}
}
