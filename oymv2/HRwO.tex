\uuid{HRwO}
\exo7id{5996}
\auteur{quinio}
\datecreate{2011-05-20}
\isIndication{false}
\isCorrection{true}
\chapitre{Probabilité discrète}
\sousChapitre{Probabilité conditionnelle}

\contenu{
\texte{
Dans les barres de chocolat N., on trouve des images équitablement
réparties des cinq personnages du dernier Walt Disney, une image par
tablette. Ma fille veut avoir le héros Princecharmant : combien dois-je
acheter de barres pour que la probabilité d'avoir la figurine attendue dépasse $80$\%? 
Même question pour être sûr à $90$\%.
}
\reponse{
La probabilité d'avoir Princecharmant dans la barre B est $\frac{1}{5}; $ si j'achète $n$ barres, la probabilité de n'avoir la
figurine dans aucune des $n$ barres est $(\frac{4}{5})^{n}$, puisqu'il
s'agit de $n$ événements indépendants de probabilité $\frac{4}{5}$.
Je cherche donc $n$ tel que : $1-(\frac{4}{5})^{n}\geq 0.8$. On a facilement : $n\geq 8$.

Puis, je cherche $m$ tel que : $1-(\frac{4}{5})^{m}\geq 0.9$ ; il
faut au moins $11$ barres pour que la probabilité dépasse $90$\%.
Pour la probabilité $99$\%, $n\geq 21$ .
}
}
