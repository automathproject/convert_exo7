\uuid{fol5}
\exo7id{2561}
\auteur{tahani}
\datecreate{2009-04-01}
\isIndication{false}
\isCorrection{true}
\chapitre{Solution maximale}
\sousChapitre{Solution maximale}

\contenu{
\texte{
Soit l'\'equation
diff\'erentielle $$x'''-xx''=0$$ o\`u $x$ est une application
trois fois d\'erivable, d\'efinie sur un intervalle ouvert de
$\mathbb{R}$ et \`a valeurs dans $\mathbb{R}$.
}
\begin{enumerate}
    \item \question{Mettre cette \'equation diff\'erentielle sous la forme
canonique $y'(t)=f(t,y(t))$, o\`u $f$ est une application que l'on
d\'eterminera.}
\reponse{Posons $y_1=x$, $y_2=x'=y_1'$, $y_3=x''=y_2'$. L'\'equation
devient $y_3'-y_1y_3=0$ et donc en posant $f(t,y_1,y_2,y_3)=
\left( \begin{array}{c} y_2 \\ y_3 \\ y_1y_3
\end{array}\right)$ l'\'equation s'\'ecrit
$$\left( \begin{array}{c} y_1' \\ y_2' \\ y_3'
\end{array}\right) = f(t,y_1,y_2,y_3).$$}
    \item \question{Soient $t_0, a, b, c \in \mathbb{R}$. Montrer
qu'il existe une unique solution maximale $\varphi$ de
l'\'equation $(3)$ qui satisfasse aux conditions initiales
$$\varphi(t_0)=a, \varphi'(t_0) \mbox{ et }
\varphi''(t_0)=c.$$}
\reponse{$f$ \'etant de classe $C^\infty$, elle est lipschitzienne
par rapport \`a la deuxi\`eme variable $(y_1,y_2,y_3)$ et donc le
th\'eor\`eme de Cauchy-Lipschitz permet de conclure.}
    \item \question{Soit $\varphi$ une telle solution
maximale. Calculer la d\'eriv\'ee de la fonction $$t \rightarrow
\varphi''(t)exp\left(- \int_{t_0}^t \varphi(u)du \right)$$ En
d\'eduire que la fonction $\varphi$ est soit convexe, soit concave
sur son intervalle de d\'efinition. D\'eterminer $\varphi$ dans le
cas o\`u $\varphi''(t_0)=0$.}
\reponse{La d\'eriv\'ee de la fonction donn\'ee est nulle. Par cons\'equent,
elle est constante et donc, l'exponentielle \'etant strictement
positive, le signe de $\varphi''$ est constant. Si cette constante
est strictement positive, $\varphi$ est convexe, si elle est
strictement n\'egative, $\varphi$ est concave. Si elle est nulle
$\varphi''=0$ et donc $\varphi(t)=at+b$ qui est bien une solution
de l'\'equation diff\'erentielle et v\'erifie $\varphi''(t_0)=0$.
L'unicit\'e montre que toutes les solutions qui v\'erifient
$\varphi''(t_0)=0$ sont bien de la forme $at+b$.}
\end{enumerate}
}
