\uuid{2498}
\auteur{sarkis}
\datecreate{2009-04-01}
\isIndication{false}
\isCorrection{true}
\chapitre{Différentiabilité, calcul de différentielles}
\sousChapitre{Différentiabilité, calcul de différentielles}

\contenu{
\texte{
Soient $E$ et $F$ deux espaces
norm\'es r\'eels et $f:E \rightarrow F$ une application born\'ee
sur la boule unit\'e de $E$ et v\'erifiant
$$f(x+y)=f(x)+f(y) \mbox{ pour tout } x,y \in E.$$
Montrez que $f$ est lin\'eaire continue.
}
\reponse{
On montre par r\'ecurrence que $f(nx)=nx$ si $n\in \mathbb{N}$.
Montrer $f(-x)=-f(x)$ pour arriver à $f(nx)=nf(x)$ si $n \in
\mathbb{Z}$ puis $f(\frac{p}{q}x)=\frac{p}{q}f(x)$ $p,q \in
\mathbb{Z}$. Ainsi $f$ est lin\'eaire sur $\mathbb{Q}$. Il reste
\`a montrer qu'elle l'est sur $\mathbb{R}$. Soit $x \in E$ et
$\lambda \in \mathbb{R}$, il reste \`a montrer que $f(\lambda
x)=\lambda f(x)$. Prenons $\{\lambda_n\}_{n \in \mathbb{N}}$ tel
que $\lim_{n\rightarrow \infty}\lambda_n=\lambda$. On a alors
$$f(\lambda_x)=f(\lambda_n x +
(\lambda-\lambda_n)x)=\lambda_n f(x)+f((\lambda-\lambda_n)x).$$
Soit $c_n \in \mathbb{\mathbb{Q}}$ tel que
$$||(\lambda-\lambda_n)x||_E \leq c_n \leq 2
||(\lambda-\lambda_n)x||_E.$$ Alors
$$f((\lambda-\lambda_n)x)=f(c_n\frac{\lambda-\lambda_n}{c_n}x)=c_nf(\frac{\lambda-\lambda_n}{c_n}x)$$
et
$$||\frac{\lambda-\lambda_n}{c_n}x|| \leq 1.$$
L'application $f$ \'etant born\'e sur la boulle unit\'e par une
constante $M > 0$, on a $$||f((\lambda-\lambda_n)x)||\leq c_nM$$
et donc $$||f((\lambda-\lambda_n)x)||\leq c_n M\leq
2M||(\lambda-\lambda_n)x||_E$$ et donc $$\lim_{n\rightarrow
\infty}f((\lambda-\lambda_n)x)=0$$, en remarquant qu'on a aussi
$$\lim_{n\rightarrow \infty}\lambda_nf(x)=\lambda f(x)$$ on obtient
$$f(\lambda x)=\lim_{n \rightarrow \infty}[\lambda_n
f(x)+f((\lambda-\lambda_n)x)]=\lambda f(x).$$
}
}
