\uuid{C1Tm}
\exo7id{6895}
\auteur{ruette}
\datecreate{2013-01-24}
\isIndication{false}
\isCorrection{true}
\chapitre{Probabilité discrète}
\sousChapitre{Probabilité et dénombrement}

\contenu{
\texte{
On veut transmettre un message électronique composé des digits 0 et 1. Les 
conditions imparfaites de transmission font en sorte qu'il y a une probabilité 
égale à  0,1 qu'un 0 soit changé en un  1  et un  1  en un  0  lors de la 
réception, et ce de façon indépendante pour chaque digit. Pour améliorer 
la qualité de la transmission, on propose d'émettre le bloc  00000 au lieu de  
0  et le bloc  11111  au lieu de  1  et de traduire une majorité de  0  dans 
un bloc lors de la réception par  0  et une majorité de  1  par  1.
}
\begin{enumerate}
    \item \question{Quelle est la probabilité de recevoir une majorité de 1 si 00000 est émis ?}
\reponse{Soit $X$ la variable aléatoire ``nombre d'erreurs commises lors de la transmission 
de 5 bits''. Alors $X$ suit une loi binomiale $B(5;0,1)$. Recevoir une majorité de 1 
alors que 00000 a été émis correspond à l'événement $X\geq 3$. Sa probabilité est

$\displaystyle
P(X=3)+P(X=4)+P(X=5)={5\choose{3}}(0,1)^3 (0,9)^2 +{5\choose{4}}(0,1)^4 (0,9)^1 +
{5\choose{5}}(0,1)^5 (0,9)^0 
=0,0081+0,00045+0,00001=0,00856.
$}
    \item \question{Quelle est la probabilité de recevoir une majorité de 1 si 11111 est émis ?}
\reponse{Recevoir une majorité de 1 alors que 11111 a été émis correspond à l'événement $X\leq 2$, 
i.e. au complémentaire du précédent. Sa probabilité est donc $1-0,00856=0,99144$. 
Par conséquent, au prix de multiplier par 5 le temps de transmission, on améliore considérablement la fiabilité.}
\end{enumerate}
}
