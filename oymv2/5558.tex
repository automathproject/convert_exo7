\uuid{5558}
\auteur{rouget}
\datecreate{2010-07-15}
\isIndication{false}
\isCorrection{true}
\chapitre{Fonction de plusieurs variables}
\sousChapitre{Extremums locaux}

\contenu{
\texte{
Trouver les extrema locaux de
}
\begin{enumerate}
    \item \question{$\begin{array}[t]{cccc}
f~:&\Rr^2&\rightarrow&\Rr\\
 &(x,y)&\mapsto&x^2+xy+y^2+2x+3y
\end{array}$}
\reponse{$f$ est de classe $C^1$ sur $\Rr^2$ qui est un ouvert de $\Rr^2$. Donc si $f$ admet un extremum local en un point $(x_0,y_0)$ de $\Rr^2$, $(x_0,y_0)$ est un point critique de $f$. Or, pour $(x,y)\in\Rr^2$,

\begin{center}
$\left\{
\begin{array}{l}
\frac{\partial f}{\partial x}(x,y)=0\\
\rule{0mm}{7mm}\frac{\partial f}{\partial x}(x,y)=0
\end{array}
\right.\Leftrightarrow\left\{
\begin{array}{l}
2x+y+2=0\\
x+2y+3=0
\end{array}
\right.\Leftrightarrow\left\{
\begin{array}{l}
x=-\frac{1}{3}\\
\rule{0mm}{7mm}y=-\frac{4}{3}
\end{array}
\right.$.
\end{center}
Donc si $f$ admet un extremum local, c'est nécessairement en $\left(-\frac{1}{3},\frac{4}{3}\right)$ avec $f\left(-\frac{1}{3},\frac{4}{3}\right)=-\frac{7}{3}$. D'autre part,

\begin{align*}\ensuremath
f(x,y)&=x^2+xy+y^2+2x+3y=\left(x+\frac{y}{2}+1\right)^2-\left(\frac{y}{2}+1\right)^2+y^2+3y=\left(x+\frac{y}{2}+1\right)^2+\frac{3y^2}{4}+2y-1\\
 &=\left(x+\frac{y}{2}+1\right)^2+\frac{3}{4}\left(y+\frac{4}{3}\right)^2-\frac{7}{3}\geqslant-\frac{7}{3}=f\left(-\frac{1}{3},\frac{4}{3}\right).
\end{align*}
Donc $f$ admet un minimum local en $\left(-\frac{1}{3},\frac{4}{3}\right)$ égal à $-\frac{7}{3}$ et ce minimum local est un minimum global. D'autre part, $f$ n'admet pas de maximum local.}
    \item \question{$\begin{array}[t]{cccc}
f~:&\Rr^2&\rightarrow&\Rr\\
 &(x,y)&\mapsto&x^4+y^4-4xy
\end{array}$}
\reponse{$f$ est de classe $C^1$ sur $\Rr^2$ qui est un ouvert de $\Rr^2$. Donc si $f$ admet un extremum local en un point $(x_0,y_0)$ de $\Rr^2$, $(x_0,y_0)$ est un point critique de $f$. Or, pour $(x,y)\in\Rr^2$,

\begin{center}
$\left\{
\begin{array}{l}
\frac{\partial f}{\partial x}(x,y)=0\\
\rule{0mm}{7mm}\frac{\partial f}{\partial x}(x,y)=0
\end{array}
\right.\Leftrightarrow\left\{
\begin{array}{l}
4x^3-4y=0\\
4y^3-4x=0
\end{array}
\right.\Leftrightarrow\left\{
\begin{array}{l}
y=x^3\\
x^9-x=0
\end{array}
\right.\Leftrightarrow(x,y)\in\{(0,0),(1,1),(-1,-1)$.
\end{center}
Les points critiques de $f$ sont $(0,0)$, $(1,1)$ et $(-1,-1)$.
Maintenant, pour $(x,y)\in\Rr^2$, $f(-x,-y)=f(x,y)$. Ceci permet de restreindre l'étude aux deux points $(0,0)$  et $(1,1)$.
\textbullet~Pour $x\in\Rr$, $f(x,0)=x^4>0$ sur $\Rr^*$ et $f(x,x)=-4x^2+2x^4=2x^2(-2+x^2)<0$ sur $]-\sqrt{2},0[\cup]0,\sqrt{2}[$. Donc $f$ change de signe dans tous voisinage de $(0,0)$ et puisque $f(0,0)=0$, $f$ n'admet pas d'extremum local en $(0,0)$.
\textbullet~Pour $(h,k)\in\Rr^2$,

\begin{align*}\ensuremath
f(1+h,1+k)-f(1,1)&=(1+h)^4+(1+k)^4-4(1+h)(1+k)+2=6h^2+6k^2-4hk+4h^3+4k^3+h^4+k^4\\
 &\geqslant6h^2+6k^2-2(h^2+k^2)+4h^3+4k^3+h^4+k^4=4h^2+4h^3+h^4+4k^2+4k^3+k^4\\
 &=h^2(2h^2+1)^2+k^2(2k^2+1)^2\geqslant0.
\end{align*}
$f$ admet donc un minimum global en $(1,1)$ (et en $(-1,-1)$) égal à $-2$.}
\end{enumerate}
}
