\uuid{7693}
\auteur{mourougane}
\datecreate{2021-08-11}
\isIndication{false}
\isCorrection{true}
\chapitre{Sous-variété}
\sousChapitre{Sous-variété}

\contenu{
\texte{
Soit $C$ une courbe régulière plane fermée simple convexe paramétrée 
par la longueur d'arc par l'application
$[0,\ell]\to\Rr^2, t\mapsto c(t)$. Un coin est un point de la courbe 
où la fonction courbure a une dérivée nulle. Le but de l'exercice est de montrer 
que la courbe $C$ a au moins trois coins. On suppose que $C$ n'est pas un cercle.
 On notera $v(t)=\dot{c}(t)$ et $\gamma(t)=\ddot{c}(t)$ 
et $\kappa(t)$ la courbure au point de paramètre $t$.
}
\begin{enumerate}
    \item \question{Montrer que $C$ a au moins deux coins distincts $P$ et $Q$. Faire une figure.}
\reponse{La fonction courbure est continue et non constante sur le compact $[0,\ell]$.
 Elle atteint donc son maximum et son minimum en deux points distincts $P$ et $Q$.
 Comme elle est continûment dérivable, en ces points sa dérivée est nulle.
 Ce sont donc des coins.}
    \item \question{Montrer à l'aide d'une intégration par partie que $\int_0^\ell \dot{\kappa}(t)c(t)=0$.}
\reponse{On utilise la formule de Frenet $\dot{n}(t)=-\kappa(t)\dot{c}(t)$.
 \begin{eqnarray*}
  \int_0^\ell \dot{\kappa}c(t)&=-&\int_0^\ell \kappa\dot{c}(t)=\int_0^\ell \dot{n}(t)
 =n(\ell)-n(0)=0.
 \end{eqnarray*}}
    \item \question{On suppose que $P$ et $Q$ sont sur l'axe des $x$ et qu'il n'y a pas d'autres coins.
Aboutir à une contradiction.}
\reponse{Par hypothèse, l'ordonnée de $\dot{\kappa}(t)c(t)$ serait de signe constant sur les deux
 parties de la courbe délimitées par $P$ et $Q$. Ceci contredit l'annulation de l'intégrale.}
    \item \question{Montrer que la courbe $C$ a au moins quatre coins.}
\reponse{S'il n'y a que trois coins, la courbe est partagée en trois parties.
 Sur deux des parties adjacentes $\dot{\kappa}$ a le même signe.
 Le raisonnement précédent en déplaçant la courbe,
 de sorte que $\dot{\kappa}$ soit de signe constant sur chaque demi-plan $(y\geq 0)$
 et $(y\leq 0)$ aboutit à une contradiction.}
\end{enumerate}
}
