\uuid{6sta}
\exo7id{6809}
\auteur{gijs}
\datecreate{2011-10-16}
\isIndication{false}
\isCorrection{false}
\chapitre{Solution maximale}
\sousChapitre{Solution maximale}

\contenu{
\texte{
{\it Introduction.\/}
Si $A$ est une matrice $n\times n$ à coefficients
réels, on sait que les solutions de l'équation 
\begin{equation}
\varphi'(t) = A\varphi(t)
\tag{*}
\end{equation}
sont définies (au moins) sur l'intervalle $[0,\infty[$
(et à valeurs dans $\Rr^n$). Dans la question 1.
on vous demande de démontrer que, sous certaines
hypothèses sur $A$, on a $ \lim_{t\to\infty}
\varphi(t) = 0$.

Le but des questions 2. et 3. est de
démontrer les mêmes résultats pour l'équation
perturbée
\begin{equation}
\varphi'(t) = A\varphi(t) + g(\varphi(t))
\tag {**}
\end{equation}
où $g:\Rr^n \to \Rr^n$ est une fonction de classe $C^1$
vérifiant $g(0) = 0$ et $g'(0) = 0$.

\medskip

Les questions 1. et 2. sont
indépendantes. La question 3. utilise des
résultats des questions 1. et 2.
}
\begin{enumerate}
    \item \question{Soit $M(n,\Rr)$ l'ensemble des matrices carrées 
$n\times n$ à coefficients réels et $M(n,\Cc)$
l'ensemble des matrices  $n\times n$ à
coefficients complexes. Soit $\| \quad \| : \Cc^n \to
\Rr$ la norme définie par $\|(x_1, \ldots, x_n)\|
=  \max_{i} |x_i|$. 

  \begin{enumerate}}
    \item \question{Démontrer
que pour tout $x\in\Cc^n$, $B\in M(n,\Cc)$ on a~:
$$  \max_{i=1,\ldots,n} \left|{\sum_{j=1}^n B_{ij}x_j}\right|
\le (\max_i \sum_j |B_{ij}|) \cdot (\max_j
|x_j|)\ ,$$ 
c'est-à-dire $\|Bx\| \le (\max_i
\sum_j |B_{ij}) \cdot \|x\|$.}
    \item \question{Soit $D\in M(n,\Cc)$ une matrice diagonale avec
$(\lambda_1, \ldots,\lambda_n)$ sur la diagonale. Montrer
que $\forall x\in \Cc^n$~: $\|Dx\| \le (\max_i 
|\lambda_i|) \cdot \|x\|$.}
    \item \question{Soit $A\in M(n,\Cc)$ diagonalisable et  $\lambda_1,
\ldots,\lambda_n $ ses valeurs propres. Soit $\alpha =
\max_i \Re \lambda_i$. Montrer qu'il existe une constante
$K>0$ telle que $\forall x\in \Cc^n$, $\forall t\in
[0,\infty[$~: $\|e^{tA}x\| \le K e^{\alpha t}
\|x\|$.}
    \item \question{Soit $A\in M(n,\Rr)$ diagonalisable sur $\Cc$ et
$\lambda_1,
\ldots,\lambda_n $ ses valeurs propres  (attention, ici
$A$ est  à coefficients réels). Soit en plus $\alpha =
\max_i \Re \lambda_i$  strictement inférieur à 0.
Déduire de ce qui précède que si $\varphi(t)$ est
solution de l'équation (*), alors $ \lim_{t\to
\infty} \varphi(t) = 0$.}
\end{enumerate}
}
