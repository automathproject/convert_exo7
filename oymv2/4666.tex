\uuid{4666}
\auteur{quercia}
\datecreate{2010-03-14}

\contenu{
\texte{
On note~: 
$E^{\phantom 1} = \{\text{fonctions continues }2\pi\text{-périodiques }f : \R \to \C\}$ ;\par
$E^1 = \{f\in E \text{ de classe }\mathcal{C}^1\}$ ;\par
$E_n = \{f\in E\text{ tel que }\forall\ k\in[[-n,n]],\ {\textstyle \int}_{t=0}^{2\pi}f(t)e^{-ikt}\,d t = 0\}$ ;\par
$E_n^1 = E_n \cap E^1.$

On considère sur~$E$ la norme~$\|\ \|_2$ $\Bigl(\|f\|_2 = \sqrt{\frac1{2\pi}\int |f|^2}\;\Bigr)$.
}
\begin{enumerate}
    \item \question{Montrer que~$D : {E_0^1} \to {E_0},  f  \mapsto {f'}$ est une bijection.}
\reponse{Si~$f$ est $\mathcal{C}^1$ $2\pi$-périodique alors $f'$ est continue,
    $2\pi$-périodique de moyenne nulle donc $D(E_0^1)\subset E_0$.
    Ré\-ci\-pro\-que\-ment, si $g\in E_0$ alors toutes les primitives de~$g$ sont
    $2\pi$-périodiques de classe~$\mathcal{C}^1$ et il y en a exactement une qui a
    une valeur moyenne nulle (une seule possibilité de régler la constante).}
    \item \question{$D$ est-elle continue~?}
\reponse{Non (et ceci quelle que soit la norme) car le spectre de~$D$ n'est par borné.}
    \item \question{Montrer que~$D^{-1}$ est continue.}
\reponse{$\|f\|_2^2 = \sum_{k\in\Z^*}|c_k(f)|^2 \le \sum_{k\in\Z^*}|ikc_k(f)|^2 = \|f'\|^2$.}
    \item \question{Montrer que~$D^{-1}(E_n) = E_n^1$ et calculer $\|\kern-1.2pt|D^{-1}_{|E_n}\|\kern-1.2pt|$.}
\reponse{Idem, $\|\kern-1.2pt|D^{-1}_{|E_n}\|\kern-1.2pt| = \frac1{n+1}$.}
\end{enumerate}
}
