\uuid{dnEo}
\exo7id{3803}
\auteur{quercia}
\datecreate{2010-03-11}
\isIndication{false}
\isCorrection{true}
\chapitre{Espace euclidien, espace normé}
\sousChapitre{Endomorphismes auto-adjoints}

\contenu{
\texte{
Soit $E = \mathcal{C}([0,1],\R)$ muni du produit scalaire défini par $(f\mid g) =  \int_0^1 fg$.

Soient $u,v$ les endomorphismes de~$E$ définis par
$u(f)(x) =  \int_0^x f$ et $v(f)(x) =  \int_x^1 f$.
}
\begin{enumerate}
    \item \question{Montrer que $(u(f)\mid g) = (f\mid v(g))$.}
    \item \question{Déterminer les valeurs propres de~$u\circ v$.}
\reponse{
On a pour~$f,g\in E$~: $u\circ v(f) = g \Leftrightarrow g$ est $\mathcal{C}^2$,
$g(0)=g'(1)=0$ et $g''=-f$.
En particulier $u\circ v$ est injectif, $0$ n'est pas valeur propre de~$u\circ v$.

Pour $\lambda\in\R^*$ et $f\in E$ on a $u\circ v(f) = \lambda f$
si et seulement si $f$ est de la forme $x \mapsto ae^{\alpha x} + be^{-\alpha x}$
avec $\alpha^2 = -\frac1\lambda$ et $a+b=a\alpha e^\alpha - b\alpha e^{-\alpha} = 0$.
On obtient $f\ne 0$ en prenant $a\ne 0$, $b=-a$ et $\alpha=i\pi(\frac12+k)$, $k\in\Z$.
Donc $\mathrm{Spec}(u\circ v) = \Bigl\{\frac{1}{\pi^2(\frac 12+k)^2},\ k\in\Z\Bigr\}$.
}
\end{enumerate}
}
