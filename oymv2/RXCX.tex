\uuid{RXCX}
\exo7id{5897}
\auteur{rouget}
\datecreate{2010-10-16}
\isIndication{false}
\isCorrection{true}
\chapitre{Analyse vectorielle}
\sousChapitre{Forme différentielle, champ de vecteurs, circulation}

\contenu{
\texte{
Les formes différentielles suivantes sont elles exactes ? Si oui, intégrer et si non chercher un facteur intégrant.
}
\begin{enumerate}
    \item \question{$\omega= (2x+2y+e^{x+y})(dx+dy)$ sur $\Rr^2$.}
\reponse{Pour $(x,y)\in\Rr^2$, on pose $P(x,y)=2x+2y+e^{x+y}=Q(x,y)$. Les fonctions $P$ et $Q$ sont de classe $C^1$ sur $\Rr^2$ qui est un ouvert étoilé de $\Rr^2$. Donc, d'après le théorème de \textsc{Schwarz}, $\omega$ est exacte sur $\Rr^2$ si et seulement si $ \frac{\partial P}{\partial y}= \frac{\partial Q}{\partial x}$ et comme $ \frac{\partial P}{\partial y}=2+e^{x+y}= \frac{\partial Q}{\partial x}$, la forme différentielle $\omega$ est une forme différentielle exacte sur $\Rr^2$.

Soit $f$ une fonction $f$ de classe $C^1$ sur $\Rr^2$.

\begin{align*}\ensuremath
df=\omega&\Leftrightarrow\forall(x,y)\in\Rr^2,\;\left\{
\begin{array}{l}
 \frac{\partial f}{\partial x}(x,y)=2x+2y+e^{x+y}\\
\rule{0mm}{7mm} \frac{\partial f}{\partial y}(x,y)=2x+2y+e^{x+y}
\end{array}
\right.\\
 &\Leftrightarrow\exists g\in C^1(\Rr,\Rr)/\;\forall(x,y)\in\Rr^2,\;
\left\{
\begin{array}{l}
f(x,y)=x^2+2xy+e^{x+y}+g(y)\\
2x+e^{x+y}+g'(y)=2x+2y+e^{x+y}
\end{array}
\right.\\
 &\Leftrightarrow\exists\lambda\in\Rr/\;\forall(x,y)\in\Rr^2,\;
\left\{
\begin{array}{l}
f(x,y)=x^2+2xy+e^{x+y}+g(y)\\
g(y)=y^2+\lambda
\end{array}
\right.\\
 &\Leftrightarrow\exists\lambda\in\Rr/\;\forall(x,y)\in\Rr^2/\;f(x,y)=(x+y)^2+e^{x+y}+\lambda.
\end{align*}

Les primitives de $\omega$ sur $\Rr^2$ sont les fonctions de la forme $(x,y)\mapsto(x+y)^2+e^{x+y}+\lambda$, $\lambda\in\Rr$.

\textbf{Remarque.} On pouvait aussi remarquer immédiatement que si $f(x,y)=(x+y)^2+e^{x+y}$ alors $df=\omega$.}
    \item \question{$\omega= \frac{xdy-ydx}{(x-y)^2}$ sur $\Omega=\{(x,y)\in\Rr^2/\;y > x\}$}
\reponse{La forme différentielle $\omega$ est de classe $C^1$ sur $\Omega=\{(x,y)\in\Rr^2/\;y > x\}$ qui est un ouvert étoilé de $\Rr^2$ car convexe. Donc, d'après le théorème de \textsc{Schwarz}, $\omega$ est exacte sur $\Omega$ si et seulement si $\omega$ est fermée sur $\Omega$.

$ \frac{\partial}{\partial x}\left( \frac{x}{(x-y)^2}\right)= \frac{\partial}{\partial x}\left( \frac{1}{x-y}+y \frac{1}{(x-y)^2}\right)=- \frac{1}{(x-y)^2}- \frac{2y}{(x-y)^3}=- \frac{x+y}{(x-y)^3}= \frac{x+y}{(y-x)^3}$.

$ \frac{\partial}{\partial y}\left(- \frac{y}{(x-y)^2}\right)= \frac{\partial}{\partial y}\left(- \frac{1}{y-x}-x \frac{1}{(y-x)^2}\right)= \frac{1}{(y-x)^2}+ \frac{2x}{(y-x)^3}= \frac{x+y}{(y-x)^3}= \frac{\partial}{\partial x}\left( \frac{x}{(x-y)^2}\right)$.

Donc $\omega$ est exacte sur l'ouvert $\Omega$. Soit $f$ une fonction $f$ de classe $C^1$ sur $\Rr^2$.

\begin{align*}\ensuremath
df=\omega&\Leftrightarrow\forall(x,y)\in\Omega,\;\left\{
\begin{array}{l}
 \frac{\partial f}{\partial x}(x,y)=- \frac{y}{(x-y)^2}\\
\rule{0mm}{6mm} \frac{\partial f}{\partial y}(x,y)= \frac{x}{(x-y)^2}
\end{array}
\right.\\
 &\Leftrightarrow\exists g\in C^1(\Rr,\Rr)/\;\forall(x,y)\in\Omega,\;
\left\{
\begin{array}{l}
f(x,y)= \frac{y}{x-y}+g(y)\\
 \frac{x}{(x-y)^2}+g'(y)= \frac{x}{(x-y)^2}
\end{array}
\right.\\
 &\Leftrightarrow\exists\lambda\in\Rr/\;\forall(x,y)\in\Omega,\;f(x,y)= \frac{y}{x-y}+\lambda.
\end{align*}

Les primitives de $\omega$ sur $\Omega$ sont les fonctions de la forme $(x,y)\mapsto \frac{y}{x-y}+\lambda$, $\lambda\in\Rr$.}
    \item \question{$\omega= \frac{xdx+ydy}{x^2+y^2}- ydy$}
\reponse{$\omega$ est de classe $C^1$ sur $\Rr^2\setminus\{(0,0)\}$ qui est un ouvert de $\Rr^2$ mais n'est pas étoilé. On se place dorénavant sur $\Omega=\Rr^2\setminus\{(x,0),\;x\in]-\infty,0]\}$ qui est un ouvert étoilé de $\Rr^2$. Sur $\Omega$, $\omega$ est exacte si et seulement si $\omega$ est fermée d'après le théorème de \textsc{Schwarz}.

$ \frac{\partial}{\partial x}\left( \frac{y}{x^2+y^2}-y\right)=- \frac{2xy}{(x^2+y^2)^2}= \frac{\partial}{\partial y}\left( \frac{x}{x^2+y^2}\right)$. Donc $\omega$ est exacte sur $\Omega$. Soit $f$ une application de classe $C^1$ sur $\Omega$.

\begin{align*}\ensuremath
df=\omega&\Leftrightarrow\forall(x,y)\in\Omega,\;\left\{
\begin{array}{l}
 \frac{\partial f}{\partial x}(x,y)= \frac{x}{x^2+y^2}\\
\rule{0mm}{6mm} \frac{\partial f}{\partial y}(x,y)= \frac{y}{x^2+y^2}-y
\end{array}
\right.\\
 &\Leftrightarrow\exists g\in C^1(\Rr,\Rr)/\;\forall(x,y)\in\Omega,\;\left\{
\begin{array}{l}
 \frac{\partial f}{\partial x}(x,y)= \frac{1}{2}\ln(x^2+y^2)+g(y)\\
 \frac{y}{x^2+y^2}+g'(y)= \frac{y}{x^2+y^2}-y
\end{array}
\right.\\
 &\Leftrightarrow\exists\lambda\in\Rr/\;\forall(x,y)\in\Omega,\;f(x,y)= \frac{1}{2}(\ln(x^2+y^2)-y^2)+\lambda.
\end{align*}

Les primitives de $\omega$ sur $\Omega$ sont les fonctions de la forme $(x,y)\mapsto \frac{1}{2}(\ln(x^2+y^2)-y^2)+\lambda$, $\lambda\in\Rr$.

Les fonctions précédentes sont encore des primitives de $\omega$ sur $\Rr^2\setminus\{(0,0)\}$ et donc $\omega$ est exacte sur $\Rr^2\setminus\{(0,0)\}$.}
    \item \question{$\omega= \frac{1}{x^2y}dx - \frac{1}{xy^2}dy$ sur $(]0,+\infty[)^2$ (trouver un facteur intégrant non nul ne dépendant que de $x^2+y^2$).}
\reponse{$\omega$ est de classe $C^1$ sur $]0,+\infty[^2$ qui est un ouvert étoilé de $\Rr^2$. Donc $\omega$ est exacte sur $]0,+\infty[^2$ si et seulement si $\omega$ est fermée sur $]0,+\infty[^2$ d'après le théorème de \textsc{Schwarz}.

$ \frac{\partial}{\partial x}\left(
- \frac{1}{xy^2}\right)= \frac{1}{x^2y^2}$ et $ \frac{\partial}{\partial y}\left(
 \frac{1}{x^2y}\right)=- \frac{1}{x^2y^2}$. Donc $ \frac{\partial}{\partial x}\left(
- \frac{1}{xy^2}\right)\neq \frac{\partial}{\partial y}\left(
 \frac{1}{x^2y}\right)$ et $\omega$ n'est pas exacte sur $]0,+\infty[^2$.

On cherche un facteur intégrant de la forme $h~:~(x,y)\mapsto g(x^2+y^2)$ où $g$ est une fonction non nulle de classe $C^1$ sur $]0,+\infty[$.

$ \frac{\partial}{\partial x}\left(
- \frac{1}{xy^2}g(x^2+y^2)\right)= \frac{1}{x^2y^2}g(x^2+y^2)- \frac{2}{y^2}g'(x^2+y^2)$ et $ \frac{\partial}{\partial y}\left(
 \frac{1}{x^2y}g(x^2+y^2)\right)=- \frac{1}{x^2y^2}g(x^2+y^2)+ \frac{2}{x^2}g'(x^2+y^2)$.

\begin{align*}\ensuremath
h\omega\;\text{est exacte sur}\;]0,+\infty[^2&\Leftrightarrow\forall(x,y)\in]0,+\infty[^2,\; \frac{1}{x^2y^2}g(x^2+y^2)- \frac{2}{y^2}g'(x^2+y^2)=- \frac{1}{x^2y^2}g(x^2+y^2)+ \frac{2}{x^2}g'(x^2+y^2)\\
 &\Leftrightarrow\forall(x,y)\in]0,+\infty[^2,\; \frac{1}{x^2y^2}g(x^2+y^2)- \frac{x^2+y^2}{x^2y^2}g'(x^2+y^2)=0\\
 &\Leftrightarrow\forall t>0,\;-tg'(t)+g(t)=0\Leftrightarrow\exists \lambda\in\Rr/\;\forall t>0,\;g(t)=\lambda t.
\end{align*}

La forme différentielle $(x^2+y^2)\omega$ est exacte sur $]0,+\infty[^2$. De plus, 

\begin{center}
$d\left( \frac{x}{y}- \frac{y}{x}\right)=\left( \frac{1}{y}+ \frac{y}{x^2}\right)dx-\left( \frac{x}{y^2}+ \frac{1}{x}\right)dy=(x^2+y^2)\omega$.
\end{center}}
\end{enumerate}
}
