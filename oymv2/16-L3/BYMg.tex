\uuid{BYMg}
\exo7id{7554}
\auteur{mourougane}
\datecreate{2021-08-10}
\isIndication{false}
\isCorrection{false}
\chapitre{Théorème des résidus}
\sousChapitre{Théorème des résidus}

\contenu{
\texte{

}
\begin{enumerate}
    \item \question{Soit $I=[a,b]$ ($a<b$) un intervalle de $\Rr$ et $\phi : I\to \Cc$ une application continue.
Montrer que $$\left|\int_a^b \phi\right|\leq \int_a^b |\phi|.$$
On pourra considérer un nombre complexe $c$ tel que $\left|\int_a^b \phi\right|=c\int_a^b \phi$.}
    \item \question{Vérifier que la relation définie sur les chemins paramétrés par $\gamma_1\equiv \gamma_2$
s'il existe une bijection $\alpha : J\to I$ dérivable à dérivée continue et partout strictement positive telle que $\gamma_2=\gamma_1\circ\alpha$
est une relation d'équivalence.}
    \item \question{Donner l'exemple de $D$ un ouvert de $\Cc$, $f : D \to \Cc$ une application continue, $I$ un intervalle de $\Rr$
et $\gamma_1 : I\to D$ une application continue, $J$ un intervalle de $\Rr$
et $\alpha : J\to I$ une application continue bijective et $\gamma_2=\gamma_1\circ\alpha$
tels que 
$$\int_I f( \gamma_1(t)) dt\not =\int_J f(\gamma_2(\tau)) d\tau.$$}
    \item \question{Donner l'exemple d'une fonction $f :D\to\Cc$ holomorphe et de deux chemins $\Gamma_1$ et $\Gamma_2$ de même origine et fin,
tels que $$\int_{\Gamma_1}f(z)dz\not=\int_{\Gamma_2}f(z)dz.$$}
    \item \question{Soit $D$ un ouvert de $\Cc$, $f : D \to \Cc$ une application continue, $\Gamma$ un chemin dans $D$.
A-t-on $$re\left(\int_\Gamma f(z)dz\right)=\int_\Gamma re(f(z)) dz ?$$}
\end{enumerate}
}
