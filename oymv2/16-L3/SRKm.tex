\uuid{SRKm}
\exo7id{6753}
\auteur{queffelec}
\datecreate{2011-10-16}
\isIndication{false}
\isCorrection{false}
\chapitre{Théorème des résidus}
\sousChapitre{Théorème des résidus}

\contenu{
\texte{
Soit $f(z) = z^5 +5z^3+z-2$. Montrer que $f$ a trois de  ses
zéros dans le disque $D(0,1)$ et tous ses zéros dans le disque $D(0,3)$.
}
}
