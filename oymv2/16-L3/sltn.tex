\uuid{sltn}
\exo7id{7617}
\auteur{mourougane}
\datecreate{2021-08-10}
\isIndication{false}
\isCorrection{false}
\chapitre{Autre}
\sousChapitre{Autre}

\contenu{
\texte{
On rappelle les équations de Cauchy-Riemann en coordonnées polaires $(r,\theta)$ pour une fonction $f=u+iv$ d'un ouvert $D$ de $\Cc$ dans $\Cc$~:
\begin{center}
 $\displaystyle\frac{\partial u}{\partial r}=\frac{1}{r}\frac{\partial v}{\partial \theta}
\quad \text{ et } \quad \frac{1}{r}\frac{\partial u }{\partial \theta}=-\frac{\partial v}{\partial r}$.
\end{center}
}
\begin{enumerate}
    \item \question{L'application $u: \Rr^2-\{(0,0)\}\to\Rr, (x,y)\mapsto \log_\Rr(x^2+y^2)$ est-elle harmonique sur son domaine de définition ?}
    \item \question{En utilisant les équations de Cauchy-Riemann en coordonnées polaires, déterminer s'il existe ou pas une fonction holomorphe $f$ sur $\Cc-\{0\}$ dont la partie réelle est $u$.}
\end{enumerate}
}
