\uuid{iM2F}
\exo7id{6626}
\auteur{queffelec}
\datecreate{2011-10-16}
\isIndication{false}
\isCorrection{false}
\chapitre{Fonction holomorphe}
\sousChapitre{Fonction holomorphe}

\contenu{
\texte{
Soit $\sum a_nz^n$ une série entière de rayon $1$ .
Montrer les équivalences (i)$\Longleftrightarrow$(ii)

$\Longleftrightarrow$(iii)
où

(i) La série converge uniformément sur $D$.

(ii) La série converge uniformément sur $\overline D$.

(iii) La série converge uniformément sur $\partial D$

(\emph{Indication} : pour l'implication (iii)$\Longrightarrow$(ii), on posera
$r_N=\sum_N^\infty a_n e^{in\theta}$ et on fera une tranformation d'Abel dans
la somme $\sum_M^N a_n \rho^n e^{in\theta}$ où $0\leq\rho\leq1$.)
}
}
