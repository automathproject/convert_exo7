\uuid{HH7M}
\exo7id{2856}
\auteur{burnol}
\datecreate{2009-12-15}
\isIndication{false}
\isCorrection{false}
\chapitre{Théorème des résidus}
\sousChapitre{Théorème des résidus}

\contenu{
\texte{
\label{exo:residuinfini}
  Soit $f$ une fonction analytique pour
$\{|z|>R\}$. On pose:
\[ \mathrm{Res}(f,\infty) = - \frac1{2\pi i}\int_{C_r} f(z)dz\] avec
$C_r$ le cercle $\{|z| = r\}$ parcouru dans le sens
direct. Montrer que le terme de droite est bien indépendant
de $r>R$. On notera le signe $-$. On dit que
$\mathrm{Res}(f,\infty)$ est le ``résidu à l'infini'' de
$f$. Soit $f$ une fonction holomorphe sur $\Cc$ à
l'exception d'un nombre fini de singularités
isolées. Montrer le théorème suivant: \emph{la somme de tous
les résidus (y compris celui à l'infini) de $f$ est nulle}.
}
}
