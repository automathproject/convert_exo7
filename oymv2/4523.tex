\uuid{itNp}
\exo7id{4523}
\auteur{quercia}
\datecreate{2010-03-14}
\isIndication{false}
\isCorrection{true}
\chapitre{Suite et série de fonctions}
\sousChapitre{Convergence simple, uniforme, normale}

\contenu{
\texte{
On consid{\'e}re $f : x\mapsto 2x(1-x)$ d{\'e}finie sur $[0,1]$.
}
\begin{enumerate}
    \item \question{\'Etude de la suite de fonction $g_{n}$, avec $g_{n}=f^{n}=f\circ\ldots\circ f$.}
\reponse{Il y a convergence simple vers la fonction nulle en $0$ et $1$ et {\'e}gale {\`a} $1/2$ ailleurs. La convergence
est uniforme sur tout $[a,b]\subset {]0,1[}$.}
    \item \question{Soit $[a,b]\subset {]0,1[}$ et $h$ continue sur $[a,b]$. Montrer que $h$ est limite uniforme 
sur $[a,b]$ d'une suite de polyn{\^o}mes {\`a} coefficients entiers.}
\reponse{La question pr{\'e}c{\'e}dente donne le r{\'e}sultat pour $1/2$, il suffit alors d'utiliser le th{\'e}or{\`e}me de
Weierstrass et les nombres dyadiques.}
\end{enumerate}
}
