\uuid{2784}
\auteur{burnol}
\datecreate{2009-12-15}
\isIndication{false}
\isCorrection{true}
\chapitre{Fonction holomorphe}
\sousChapitre{Fonction holomorphe}

\contenu{
\texte{
\label{ex:burnol1.1.2}
Si $f$ et $g$ sont deux fonctions dérivables au sens
complexe au point $z_0$ ; montrer que $f+g$, $f-g$ et $fg$ le
sont et donner la valeur de leurs dérivées au point
$z_0$.
}
\reponse{
Consid\'erons le produit $fg$. En utilisant la d\'efinition m\^eme de la d\'eriv\'ee,
on a :
$$
\begin{aligned}
\frac{1}{h} \left( f(z+h)g(z+h) -f(z)g(z)\right) &= f(z+h)\frac{g(z+h)-g(z)}{h} + g(z) \frac{f(z+h)-f(z)}{h}\\
& \longrightarrow f(z) g'(z) +g(z)f'(z) \quad \text{lorsque} \;\; h\to 0 \, .
\end{aligned}$$
Autre mani\`ere:
$$\begin{aligned}
f(z+h)g(z+h)&= (f(z)+f'(z)h +h\epsilon (h))(g(z)+g'(z)h +h\epsilon (h))\\
&= f(z)g(z) + (f(z)g'(z)+f'(z)g(z))h +h \epsilon (h)\, .
\end{aligned}$$
D'o\`u $(fg)'(z)=f(z)g'(z)+f'(z)g(z)$.
}
}
