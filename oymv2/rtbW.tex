\uuid{rtbW}
\exo7id{254}
\auteur{bodin}
\datecreate{1998-09-01}
\isIndication{true}
\isCorrection{true}
\chapitre{Arithmétique dans Z}
\sousChapitre{Divisibilité, division euclidienne}

\contenu{
\texte{
D\'emontrer que le nombre $7^n+1$ est divisible par
$8$ si $n$ est impair ; dans le cas $n$ pair, donner le reste de
sa division par $8$.
}
\indication{Utiliser les modulos (ici modulo $8$),
un entier est divisible par $8$ si et seulement si
il est \'equivalent \`a $0$ modulo $8$.
Ici vous pouvez commencer par calculer $7^n \pmod{8}$.}
\reponse{
Raisonnons modulo $8$ :
$$7 \equiv -1 \pmod{8}.$$
Donc
$$7^n +1 \equiv (-1)^n + 1 \pmod{8}.$$

Le reste de la division euclidienne de $7^n+1$ par $8$ est donc
$(-1)^n+1$ donc Si $n$ est impair alors $7^n+1$ est divisible par
$8$. Et si $n$ est pair $7^n+1$ n'est pas divisible par $8$.
}
}
