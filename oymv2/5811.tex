\uuid{5811}
\auteur{rouget}
\datecreate{2010-10-16}

\contenu{
\texte{
Sur $E =\Rr^2$ ou $\Rr^3$ muni de sa structure euclidienne usuelle, réduire en base orthonormée les formes quadratiques suivantes :
}
\begin{enumerate}
    \item \question{$Q((x,y)) = x^2+10xy+y^2$.}
\reponse{(Quand $x^2$ et $y^2$ ont les mêmes coefficients, penser à faire une rotation d'angle $\frac{\pi}{4}$) En posant $x=\frac{1}{\sqrt{2}}(X+Y)$ et $y =\frac{1}{\sqrt{2}}(X-Y)$, on obtient

\begin{center}
$x^2+10xy+y^2=\frac{1}{2}(X+Y)^2+5(X+Y)(X-Y)+\frac{1}{2}(X-Y)^2=6X^2-4Y^2$.
\end{center}

Ainsi, si on note $(i,j)$ la base canonique de $\Rr^2$ puis $e_1=\frac{1}{\sqrt{2}}(i+j)$ et $e_2=\frac{1}{\sqrt{2}}(i-j)$, on a

\begin{center}
$x^2+10xy+y^2=Q(xi+yj)=Q(Xe_1+Ye_2)=6X^2-4Y^2$.
\end{center}}
    \item \question{$Q((x,y)) = 6x2+4xy+9y2$.}
\reponse{La matrice de $Q$ dans la base canonique $(i,j)$ de $\Rr^2$ est $A=\left(
\begin{array}{cc}
6&2\\
2&9
\end{array}
\right)$. Les deux nombres $5$ et $10$ ont une somme égale à $15 =\text{Tr}(A)$ et un produit égal à $50=\text{det}(A)$ et sont donc les valeurs propres de $A$. On sait alors que dans une base orthonormée $(e_1,e_2)$ de vecteurs propres de $A$ associée à la famille de valeurs propres $(5,10)$, on a $Q(Xe_1+Ye_2)=5X^2+10Y^2$. Déterminons une telle base.

$(A-5I_2)\left(
\begin{array}{c}
x\\
y
\end{array}
\right)=\left(
\begin{array}{c}
0\\
0
\end{array}
\right)\Leftrightarrow x+2y = 0$ et donc $\text{Ker}(A-5I_2) =\text{Vect}(e_1)$ où $e_1=\frac{1}{\sqrt{5}}(2,-1)$  puis $\text{Ker}(A-10I_2) =(\text{Ker}(A-5I_2))^\bot=\text{Vect}(e_2)$ où $e_2=\frac{1}{\sqrt{5}}(1,2)$.

Donc, si $e_1=\frac{1}{\sqrt{5}}(2,-1)$, $e_2=\frac{1}{\sqrt{5}}(1,2)$ et $u=xi+yj=Xe_1+Ye_2$, alors $q(u)=6x2+4xy+9y2= 5X^2 + 10Y^2$. De plus, $x=\frac{1}{\sqrt{5}}(2X+Y)$ et $y=\frac{1}{\sqrt{5}}(-X+2Y)$.}
    \item \question{$Q((x,y,z)) = 4x^2+9y^2-z^2+2\sqrt{6}xy +10\sqrt{2}xz+2\sqrt{3}yz$.}
\reponse{La matrice de $Q$ dans la base canonique est $A=\left(
\begin{array}{ccc}
4&\sqrt{6}&5\sqrt{2}\\
\sqrt{6}&9&\sqrt{3}\\
5\sqrt{2}&\sqrt{3}&-1
\end{array}
\right)$.

\begin{align*}\ensuremath
\chi_A&=\left|
\begin{array}{ccc}
4-X&\sqrt{6}&5\sqrt{2}\\
\sqrt{6}&9-X&\sqrt{3}\\
5\sqrt{2}&\sqrt{3}&-1-X
\end{array}
\right|= (4-X)(X^2-8X-12)-\sqrt{6}(-\sqrt{6}X-6\sqrt{6})+5\sqrt{2}(5\sqrt{2}X-42\sqrt{2})\\
 &= -X^3+12X^2+36X-432 =-(X-6)(X+6)(X-12).
\end{align*}

Ensuite,

\begin{align*}\ensuremath
\left(
\begin{array}{c}
x\\
y\\
z
\end{array}
\right)\in\text{Ker}(A-6I_3)&\Leftrightarrow\left\{
\begin{array}{l}
-2x+\sqrt{6}y+5\sqrt{2}z=0\\
\sqrt{6}x+3y+\sqrt{3}z=0\\
5\sqrt{2}x+\sqrt{3}y-7z=0
\end{array}
\right.\Leftrightarrow\left\{
\begin{array}{l}
z=-\sqrt{2}x-\sqrt{3}y\\
-2x+\sqrt{6}y+5\sqrt{2}(-\sqrt{2}x-\sqrt{3}y)=0\\
5\sqrt{2}x+\sqrt{3}y-7(-\sqrt{2}x-\sqrt{3}y)=0
\end{array}
\right.\\
 &\Leftrightarrow\left\{
\begin{array}{l}
z=-\sqrt{2}x-\sqrt{3}y\\
-12x-4\sqrt{6}y=0\\
12\sqrt{2}x+8\sqrt{3}y=0
\end{array}
\right.\Leftrightarrow\left\{
\begin{array}{l}
y=-\sqrt{\frac{3}{2}}x\\
z=-\sqrt{2}x-\sqrt{3}(-\sqrt{\frac{3}{2}}x)
\end{array}
\right.\Leftrightarrow\left\{
\begin{array}{l}
y=-\sqrt{\frac{3}{2}}x\\
z=\frac{\sqrt{2}}{2}x
\end{array}
\right.\Leftrightarrow\left\{
\begin{array}{l}
x=\sqrt{2}z\\
y=-\sqrt{3}{z}
\end{array}
\right.
\end{align*}

et $\text{Ker}(A-6I_3) =\text{Vect}(e_1)$ où $e_1=\frac{1}{\sqrt{6}}(\sqrt{2},-\sqrt{3},1)$. De même,

   
\begin{align*}\ensuremath
\left(
\begin{array}{c}
x\\
y\\
z
\end{array}
\right)\in\text{Ker}(A+6I_3)&\Leftrightarrow\left\{
\begin{array}{l}
10x+\sqrt{6}y+5\sqrt{2}z=0\\
\sqrt{6}x+15y+\sqrt{3}z=0\\
5\sqrt{2}x+\sqrt{3}y+5z=0
\end{array}
\right.\Leftrightarrow\left\{
\begin{array}{l}
z=-\sqrt{2}x-5\sqrt{3}y\\
10x+\sqrt{6}y+5\sqrt{2}(-\sqrt{2}x-5\sqrt{3}y)=0\\
5\sqrt{2}x+\sqrt{3}y+5(-\sqrt{2}x-5\sqrt{3}y)=0
\end{array}
\right.\\
 &\Leftrightarrow\left\{
\begin{array}{l}
z=-\sqrt{2}x-\sqrt{3}y\\
y=0
\end{array}
\right.\Leftrightarrow\left\{
\begin{array}{l}
y=0\\
z=-\sqrt{2}x
\end{array}
\right.
\end{align*}

 
et $\text{Ker}(A+6I_3) =\text{Vect}(e_2)$ où $e_2=\frac{1}{\sqrt{3}}(1,0,-\sqrt{2})$.

Enfin $\text{Ker}(A-12I_3) =\text{Vect}(e_3)$ où 

\begin{center}
$e_3 =e_1\wedge e_2=\frac{1}{3\sqrt{2}}\left(
\begin{array}{c}
\sqrt{2}\\
-\sqrt{3}\\
1
\end{array}
\right)\wedge\left(
\begin{array}{c}
1\\
0\\
-\sqrt{2}
\end{array}
\right)=\frac{1}{3\sqrt{2}}\left(
\begin{array}{c}
\sqrt{6}\\
3\\
\sqrt{3}
\end{array}
\right)$.
\end{center}

Ainsi, si on pose $P=\left(
\begin{array}{ccc}
\frac{1}{\sqrt{3}}&\frac{1}{\sqrt{3}}&\frac{1}{\sqrt{3}}\\
-\frac{1}{\sqrt{2}}&0&\frac{1}{\sqrt{2}}\\
\frac{1}{\sqrt{6}}&-\frac{2}{\sqrt{6}}&\frac{1}{\sqrt{6}}
\end{array}\right)$ alors $PA{^t}P=\text{diag}(6,-6,12)$ ou encore

\begin{center}
$Q(Xe_1+Ye_2+Ze_3) =6X^2-6Y^2+12Z^2$ où $\text{Mat}_{(i,j,k)}(e_1,e_2,e_3)=\left(
\begin{array}{ccc}
\frac{1}{\sqrt{3}}&\frac{1}{\sqrt{3}}&\frac{1}{\sqrt{3}}\\
-\frac{1}{\sqrt{2}}&0&\frac{1}{\sqrt{2}}\\
\frac{1}{\sqrt{6}}&-\frac{2}{\sqrt{6}}&\frac{1}{\sqrt{6}}
\end{array}\right)$.
\end{center}}
\end{enumerate}
}
