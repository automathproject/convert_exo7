\uuid{rgH6}
\exo7id{3578}
\auteur{quercia}
\datecreate{2010-03-10}
\isIndication{false}
\isCorrection{true}
\chapitre{Réduction d'endomorphisme, polynôme annulateur}
\sousChapitre{Trigonalisation}

\contenu{
\texte{
Soit $A = \begin{pmatrix} -1 &\phantom-2 & 0 \cr 2 &2 &-3 \cr -2 &2 &1 \cr \end{pmatrix}$
et $\varphi$ l'endomorphisme de $\R^3$ canoniquement associé à $A$.
}
\begin{enumerate}
    \item \question{Vérifier que $A$ n'est pas diagonalisable.}
\reponse{1 est valeur propre double, $d_1 = 1$.}
    \item \question{Chercher deux vecteurs propres de $A$ linéairement indépendants.}
\reponse{$\begin{pmatrix} 1\cr 1 \cr 1\end{pmatrix}$, $\begin{pmatrix} 2 \cr 1 \cr 2\end{pmatrix}$.}
    \item \question{Compléter ces vecteurs en une base de $\R^3$.}
\reponse{$\begin{pmatrix} 1 \cr 0 \cr 0\end{pmatrix}$.}
    \item \question{\'Ecrire la matrice de $\varphi$ dans cette base.}
\reponse{$\begin{pmatrix} 1 &0 &6 \cr 0 &0 &-4 \cr 0 &0 &1 \cr\end{pmatrix}$.}
    \item \question{Résoudre le système différentiel : $X' = AX$.}
\reponse{$X = \begin{pmatrix} (6\alpha t + \gamma)e^t           + 2\beta \cr
                            (6\alpha t + \gamma + 3\alpha)e^t + \beta  \cr
                            (6\alpha t + \gamma -  \alpha)e^t + 2\beta \cr \end{pmatrix}$.}
\end{enumerate}
}
