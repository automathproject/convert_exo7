\uuid{9aHw}
\exo7id{5068}
\auteur{rouget}
\datecreate{2010-06-30}
\isIndication{false}
\isCorrection{true}
\chapitre{Nombres complexes}
\sousChapitre{Trigonométrie}

\contenu{
\texte{
Calculer $\cos\frac{\pi}{12}$ et $\sin\frac{\pi}{12}$.
}
\reponse{
$$\cos\frac{\pi}{12}=\cos\left(\frac{\pi}{3}-\frac{\pi}{4}\right)=\cos\frac{\pi}{3}\cos\frac{\pi}{4}+\sin\frac{\pi}{3}
\sin\frac{\pi}{4}=\frac{\sqrt{6}+\sqrt{2}}{4}.$$
De même,
$$\sin\frac{\pi}{12}=\sin\left(\frac{\pi}{3}-\frac{\pi}{4}\right)=\sin\frac{\pi}{3}\cos\frac{\pi}{4}-\sin\frac{\pi}{3}
\sin\frac{\pi}{4}=\frac{\sqrt{6}-\sqrt{2}}{4}.$$

\begin{center}
\shadowbox{
$\cos\frac{\pi}{12}=\frac{\sqrt{6}+\sqrt{2}}{4}$ et $\sin\frac{\pi}{12}=\frac{\sqrt{6}-\sqrt{2}}{4}.
$}\end{center}
}
}
