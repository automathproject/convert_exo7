\uuid{noZt}
\exo7id{1670}
\auteur{roussel}
\datecreate{2001-09-01}
\isIndication{false}
\isCorrection{false}
\chapitre{Réduction d'endomorphisme, polynôme annulateur}
\sousChapitre{Diagonalisation}

\contenu{
\texte{
Pour tout \'el\'ement non nul $a=(a_1, a_2, \ldots ,a_n)$ de
$\mathbb{R}^{n}$, on consid\`ere l'endomorphisme ~$u$ de
$\mathbb{R}^{n}$ dont la matrice dans la base canonique
$\{e_{ij},i,j=1,2, \ldots, n \}$ est la matrice $A=\left ( \alpha
_{i,j}\right )$ o\`u $\alpha _{i,j}=a_ia_j$.
}
\begin{enumerate}
    \item \question{D\'eterminer le noyau et l'image de $u$.}
    \item \question{En d\'eduire les sous-espaces propres de $u$. D\'eterminer les valeurs
propres de $u$. L'endomorphisme $u$ est-il diagonalisable ?}
    \item \question{Quel est le polyn\^ome caract\'eristique de $u$?}
\end{enumerate}
}
