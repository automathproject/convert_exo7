\uuid{6018}
\auteur{quinio}
\datecreate{2011-05-20}

\contenu{
\texte{
Sur un grand nombre de personnes on a constaté que la
répartition du taux de cholestérol suit une loi normale 
avec les résultats suivants:

- $56$\% ont un taux inférieur à $165$ cg;

- $34$\% ont un taux compris entre $165$ cg et $180$ cg;

- $10$\% ont un taux supérieur à $180$ cg.

Quelle est le nombre de personnes qu'il faut prévoir de soigner dans une
population de $10\,000$ personnes, si le taux maximum toléré sans
traitement est de $182$ cg?
}
\reponse{
Si $X$ est de moyenne $m$ et d'écart-type $\sigma$ alors 
$Y=\frac{X-m}{\sigma}$ suit une loi centrée réduite.
Donc si $P[X\leq 165]$ alors $P[\frac{X-m}{\sigma }\leq \frac{165-m}{\sigma }]=0,56$.
Or on peut lire dans la table de Gauss $F(0.15)=0.5596$.

De même, si $P[X\geq 180]$ alors $P[\frac{X-m}{\sigma }\geq \frac{180-m}{\sigma }]=0.1$.
Donc $P[\frac{X-m}{\sigma }\leq \frac{180-m}{\sigma }]=0.9$
et l'on peut lire de même $F(1.28)=0.8997.$

Pour trouver $m$ et $\sigma$ il suffit de résoudre le système d'équations:
$\frac{165-m}{\sigma}=0.15$ et $\frac{180-m}{\sigma}=1.28$ d'où
$\sigma \simeq 13.27$, $m\simeq 163$ cg.
Alors,
$P[X\geq 182]=P[\frac{X-m}{\sigma }\geq \frac{182-m}{\sigma }]=1-F(1.\,43)=0.0764$.

Sur $10\,000$ personnes on estime le nombre de personnes à soigner de
l'ordre de $764$ personnes ; en fait la théorie de l'estimation donnera
une fourchette.
}
}
