\uuid{zs1L}
\exo7id{4619}
\auteur{quercia}
\datecreate{2010-03-14}
\isIndication{false}
\isCorrection{true}
\chapitre{Série entière}
\sousChapitre{Autre}

\contenu{
\texte{
Soit $z_1,\dots,z_p\in \C$, $p_1,\dots,p_p \in \R^+$ tels que $\sum_{i=1}^p p_i=1$, et $\omega \in \R$.
Pour $n>p$ on pose $z_n=e^{i\omega}\sum_{j=1}^p z_{n-j}p_j$.
Étudier la suite $(z_n)$.
}
\reponse{
On pose, sous réserve de convergence,
$f(t) = \sum_{n=1}^\infty z_nt^n$. Alors~:
$$f(t)
=\sum_{n=1}^p z_nt^n + \sum_{j=1}^pe^{i\omega}p_j\sum_{n=p+1}^\infty z_{n-j}t^n
=\sum_{n=1}^p z_nt^n + \sum_{j=1}^pe^{i\omega}p_jt^j\Bigl(f(t)-\sum_{n=1}^{p-j} z_nt^n\Bigr)
$$
soit~:
$$\Bigl(1-\sum_{j=1}^pe^{i\omega}p_jt^j\Bigr)f(t) = P(t)f(t)
= \sum_{n=1}^p z_nt^n - \sum_{j=1}^pe^{i\omega}p_jt^j\sum_{n=1}^{p-j} z_nt^n
=Q(t),$$
donc $f(t) = Q(t)/P(t)$. Réciproquement, soit $Q(t)/P(t) = \sum_{n=1}^\infty a_nt^n$~:
en remontant les calculs précédents on voit que $(a_n)$ vérifie la même
relation de récurrence que~$(z_n)$ avec les mêmes premiers termes d'où
$z_n = a_n$ pour tout~$n$.

Si $|t|< 1$ alors $\Bigl|\sum_{j=1}^pe^{i\omega}p_jt^j\Bigr|< 1$
donc $P$ n'a pas de racine dans le disque unité ouvert.
Si $P$ n'a pas non plus de racine sur le cercle unité alors le développement
en série entière de $Q(t)/P(t)$ a un rayon $>1$ et $z_n\to0$ lorsque $n\to\infty$.
Si $P$ admet des racines dans~$\mathbb{U}$ on peut déja dire que la suite $(z_n)$
est bornée par $\max(|z_1|,\dots,|z_p|)$ puis\dots~?
}
}
