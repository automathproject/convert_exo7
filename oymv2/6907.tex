\uuid{6907}
\auteur{ruette}
\datecreate{2013-01-24}
\isIndication{false}
\isCorrection{true}
\chapitre{Probabilité discrète}
\sousChapitre{Variable aléatoire discrète}

\contenu{
\texte{
Un standard téléphonique reçoit en moyenne 2 appels par minute. Les appels sont répartis au hasard dans le temps.
}
\begin{enumerate}
    \item \question{Quelle est la loi de probabilité régissant le nombre d'appels reçus 
en 3 minutes ? Quelle est la probabilité qu'il n'y ait aucun appel en 3 minutes ?}
\reponse{C'est une loi de Poisson de paramètre $6$ : $P(X=n)=e^{-6}\frac{6^n}{n!}$.
La probabilité qu'il n'y ait aucun appel est $p(X=0)=e^{-6}\simeq 0,002$.}
    \item \question{Quelle est la probabilité que le nombre d'appels en 2 minutes
soit supérieur ou égal à 5 ?}
\reponse{Soit $Y$ la variable aléatoire ``Nombre d'appels reçus en 2 minutes''. 
Alors $Y$ suit une loi de Poisson de paramètre 4. La probabilité qu'il 
y ait entre 0 et 4 appels est $P(Y=0)+P(Y=1)+P(Y=2)+P(Y=3)+P(Y=4)\simeq
0,629$. Donc $P(Y\ge 5)=1-P(Y\le 4)\simeq 0,371$.}
\end{enumerate}
}
