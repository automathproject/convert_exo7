\uuid{2181}
\auteur{debes}
\datecreate{2008-02-12}

\contenu{
\texte{
Soient $G$ un groupe et $H$ un sous-groupe d'indice fini dans $G$.  
On d\'efinit sur $G$ la relation $xRy$ si et seulement si $x\in HyH$.
\smallskip

(a) Montrer que $R$ est une relation d'\'equivalence et que toute classe
d'\'equivalence pour la relation $R$ est une union finie disjointe de classes 
\`a gauche modulo $H$. 
\smallskip

Soit $HxH=\bigcup _{1\leq i\leq d(x)} x_i H$  la partition de la classe $HxH$ en
classes \`a gauche distinctes.
\smallskip

(b) Soit $h\in H$ et $i$ un entier compris entre $1$ et $d(x)$; posons  $h\hskip
2pt \ast \hskip 2pt x_iH= hx_iH$. Montrer que cette formule d\'efinit une action
transitive de $H$ sur l'ensemble  des classes $x_1H, \dots, x_{d(x)} H$ et que le
fixateur de $x_i H$ dans cette action est  $H\cap x_i Hx_i ^{-1}$. En d\'eduire que 

$$d(x)=[H:H\cap x Hx ^{-1}]$$ 

et qu'en particulier $d(x)$ divise l'ordre de $G$.
\smallskip

(c) Montrer que $H$ est distingu\'e dans  $G$  si et seulement si 
$d(x)=1$  pour tout $x\in G$.
\smallskip

(d) On suppose que $G$ est fini et que $[G:H]=p$, o\`u $p$ est le plus petit 
nombre premier divisant l'ordre de $G$. Le but de cette question est de montrer 
que $H$ est distingu\'e dans $G$.

\hskip 5mm (i) Montrer que pour tout $x\in G$ , $d(x)\leq p$. En d\'eduire que
$d(x)=1$ ou $d(x)=p$. 

\hskip 5mm (ii) Montrer que si $H$ n'est pas distingu\'e dans $G$, il existe une
unique classe d'\'equivalence pour la relation $R$ et que $G=H$, ce qui contredit
l'hypoth\`ese  $[G:H ]=p$.
}
\reponse{
(a) Que $R$ soit une relation d'\'equivalence est imm\'ediat. La classe d'un
\'el\'ement $x\in G$ est l'ensemble $HxH$, lequel est \'egal \`a la r\'eunion
des ensembles $hxH$ o\`u $h$ d\'ecrit $H$. Ces derniers ensembles sont des classes
\`a gauche modulo $H$ et sont donc \'egaux ou disjoints.
\smallskip

(b) Pour tout $i=1,\ldots,d(x)$, $hx_iH$ est une classe \`a gauche, contenue dans
$h(HxH)H \subset HxH$, donc est de la forme $x_jH$. La formule $h\hskip 2pt \ast
\hskip 2pt x_iH= hx_iH$ d\'efinit ainsi une permutation de l'ensemble  des classes
$x_1H, \dots, x_{d(x)} H$ (la permutation r\'eciproque est celle induite par
$h^{-1}$) et donc une action de $H$ sur cet ensemble. Cette action est transitive:
pour $i,j\in \{1,\ldots,d(x)\}$, $h=x_i^{-1}x_j$ v\'erifie $h\hskip 2pt \ast
\hskip 2pt x_iH= x_jH$.

\hskip 5mm Un \'el\'ement $h\in H$ est dans le fixateur $H(x_iH)$ d'une classe
$x_iH$ si et seulement si $hx_iH=x_iH$ c'est-\`a-dire si $h\in x_iHx_i^{-1}$. D'o\`u
$H(x_iH)= H\cap x_iHx_i^{-1}$. On obtient alors $d(x)= [H:(H\cap x_iHx_i^{-1})]$ ce
qui prouve que $d(x)$ divise $|H|$ et donc aussi $|G|$.
\smallskip

(c) Si $H$ est distingu\'e dans $G$, alors classes \`a droite et classes \`a
gauche modulo $H$ coincident d'o\`u $HxH=xHH=xH$ et donc $d(x)=1$ pour tout $x\in
G$. Inversement, pour tout $x\in G$, si $d(x)=1$, alors $HxH=xH$ ce qui entraine
$Hx\subset xH$ et donc $x^{-1}Hx \subset H$.
\smallskip

(d) (i) De fa\c con g\'en\'erale, on a $d(x) \leq [G:H]$. On a ainsi $d(x) \leq p$
si $[G:H]=p$. Comme $d(x)$ divise $|G|$ et que $p$ est le plus petit premier
divisant $|G|$, n\'ecessairement $d(x)=1$ ou $d(x)=p$.
\smallskip

(ii) Si $H$ n'est pas distingu\'e alors il existe $x\in G$ avec $d(x)\not=1$ et
donc $d(x)=p$. Mais alors $\hbox{\rm card}(HxH) = d(x)\hskip 2pt |H| = p\hskip 2pt
|H| = [G:H]\hskip 2pt |H| = |G|$. C'est-\`a-dire, il n'existe qu'une seule classe
$HxH=G$, laquelle est aussi la classe de l'\'el\'ement neutre $H1H=H$, ce qui
contredit l'hypoth\`ese $[G:H]=p>1$. Conclusion: le sous-groupe $H$ est distingu\'e
dans $G$.
}
}
