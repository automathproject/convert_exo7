\uuid{flbp}
\exo7id{3082}
\auteur{quercia}
\datecreate{2010-03-08}
\isIndication{false}
\isCorrection{true}
\chapitre{Groupe, anneau, corps}
\sousChapitre{Groupe de permutation}

\contenu{
\texte{
Soit $$\sigma = \begin{pmatrix}1 &2 &3 &\dots &n    &n+1 &n+2 &\dots &2n \cr
                        1 &3 &5 &\dots &2n-1 &2   &4   &\dots &2n \end{pmatrix}.$$
Calculer $\varepsilon(\sigma)$.
}
\reponse{
Compter les inversions ou r{\'e}currence :
         $\epsilon(\sigma) = (-1)^{n(n-1)/2}$.
}
}
