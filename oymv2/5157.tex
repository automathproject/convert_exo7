\uuid{5157}
\auteur{rouget}
\datecreate{2010-06-30}
\isIndication{false}
\isCorrection{true}
\chapitre{Arithmétique dans Z}
\sousChapitre{Divisibilité, division euclidienne}

\contenu{
\texte{
Montrer que pour $n\in\Nn$, $E(\frac{1}{3}(n+2-E(\frac{n}{25})))=E(\frac{8n+24}{25})$.
}
\reponse{
Soit $n\in\Nn$. La division euclidienne de $n$ par $25$ fournit un quotient entier $q$ et et un
reste $r$ élément de $\{0,1,...,24\}$ tels que $n=25q+r$.

On a alors

$$E(\frac{1}{3}(n+2-E(\frac{n}{25})))=E(\frac{25q+r+2-q}{3})=E(8q+\frac{r+2}{3})=8q+E(\frac{r+2}{3}),$$

et

$$E(\frac{8n+24}{25})=E(\frac{8(25q+r)+24}{25})=8q+E(\frac{8r+24}{25}).$$

Pour montrer l'égalité de l'énoncé, il reste donc à vérifier les $25$ égalités $E(\frac{r+2}{3})=E(\frac{8r+24}{25})$,
$0\leq r\leq 24$, $(*)$, ce qui peut déjà se vérifier \og~à la main~\fg.

Diminuons encore le nombre de vérifications. La division euclidienne de $r$ par $3$ s'écrit $r=3k+l$ avec $0\leq
l\leq2$. Mais alors,

$$E(\frac{r+2}{3})=k+E(\frac{l+2}{3})\;\mbox{et}\;E(\frac{8r+24}{25})=E(\frac{25k-k+8l+24}{25})=k+E(\frac{-k+8l+
24}{25}).$$

Si $l=0$, $k$ varie de $0$ à $8$ et dans ce cas, $0\leq\frac{-k+24}{25}=\frac{-k+8l+24}{25}\leq\frac{24}{25}<1$. Par
suite,

$$E(\frac{-k+8l+24}{25})=0=E(\frac{2}{3})=E(\frac{l+2}{3}).$$

On a ainsi vérifié $(*)$ quand $r\in\{0,3,6,9,12,15,18,21,24\}$.

Si $l=1$ ou $l=2$, $E(\frac{l+2}{3})=1$ et d'autre part, $k$ varie de $0$ à $7$. Dans ce cas,

$$1=\frac{-7+8+24}{25}\leq\frac{-k+8l+24}{25}\leq\frac{16+24}{25}<2$$

et donc

$$E(\frac{-k+8l+24}{25})=1=E(\frac{l+2}{3}).$$

On a ainsi vérifié $(*)$ pour les autres valeurs de $r$. Finalement, on a montré que
\begin{center}
\shadowbox{
$\forall n\in\Nn,\;E(\frac{1}{3}(n+2-E(\frac{n}{25})))=E(\frac{8n+24}{25}).$
}
\end{center}
}
}
