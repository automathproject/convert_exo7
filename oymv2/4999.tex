\uuid{Oa44}
\exo7id{4999}
\auteur{quercia}
\datecreate{2010-03-17}
\isIndication{false}
\isCorrection{true}
\chapitre{Courbes planes}
\sousChapitre{Courbes définies par une condition}

\contenu{
\texte{
Trouver les arcs biréguliers du plan dont le cercle osculateur est en
    tout point tangent à une droite fixe.
}
\reponse{
On suppose que la droite est $Ox$ et on paramètre la courbe
             cherchée, $\mathcal{C}$, par une abscisse curviligne $s$.
	     Soient $M=(x,y) \in \mathcal{C}$,         $I=(x-R\frac{dy}{ds}, y+R\frac{dx}{ds})$
	     le centre de courbure en $M$ où $R$ est le rayon de courbure.
             On veut $|R| = \Bigl|y+R\frac{dx}{ds}\Bigr| = |y+R\cos\varphi|$ d'où~:
             $$\pm \frac{d R }{ds} = \frac{dy}{ds} -R\sin\varphi \frac{d \varphi}{ds} +  \frac{d R }{ds}\cos\varphi = \frac{d R }{ds}\cos\varphi.$$
             Ceci implique $\frac{d R }{ds} = 0$ donc $R$ est constant (cercle) ou
             $\varphi\equiv 0 \bmod \pi$ (droite horizontale). Le deuxième cas est
             exclu (courbe birégulière) donc il reste le cas d'un cercle qui
             convient s'il est tangent à $Ox$.
}
}
