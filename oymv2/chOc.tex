\uuid{chOc}
\exo7id{2408}
\auteur{bodin}
\datecreate{2003-10-01}
\isIndication{true}
\isCorrection{true}
\chapitre{Théorème de Stone-Weirstrass, théorème d'Ascoli}
\sousChapitre{Théorème de Stone-Weirstrass, théorème d'Ascoli}

\contenu{
\texte{
Soit $f \in \mathcal{C}([a,b],\Rr)$ telle que
$$\forall n \in \Nn \quad \int_a^b f(t) t^n \, dt = 0.$$
Montrer que $f$ est la fonction nulle.
}
\indication{Approcher $f$ par une suite de polynômes, et se rappeler que si l'intégrale d'une fonction positive et continue est nulle alors...}
\reponse{
Soit $P(x) = a_dx^d+\cdots+a_1x+a_0 \in \Rr[x]$ alors par linéarité de l'intégrale
et grâce à la relation de l'énoncé :
$$\int_a^b f(t)\cdot P(t) \, dt = 0.$$


La fonction $f$ est continue sur le compact $[a,b]$ donc par le théorème de Weierstrass
il existe une suite de polynômes qui converge uniformément vers $f$.
Fixons $\epsilon >0$. Soit $P$ tel que $\| f-P \|_\infty \le \epsilon$.
Alors 
\begin{align*}
|\int_a^b f(t)^2 dt|  
  &= \left| \int_a^b f(t)^2 dt - \int_a^b f(t) \cdot P(t) dt\right| \\
  &=  \left|\int_a^b f(t)\cdot (f(t)-P(t)) dt\right| \\
  &\le  \int_a^b |f(t)|\cdot \| f-P\|_\infty  dt \\
  &\le \epsilon \int_a^b |f| \\.
\end{align*}
Mais  $C = \int_a^b |f|$ est une constante (indépendante de $\epsilon$ et $P$).
Donc on vient de montrer que $|\int_a^b f(t)^2 dt| \le \epsilon C$ avec  
pour tout $\epsilon >0$ donc $\int_a^b f^2 = 0$, or $f^2$ est une fonction continue et positive,
son intégrale est nulle donc $f$ est la fonction nulle.
}
}
