\uuid{WtZ1}
\exo7id{2303}
\auteur{barraud}
\datecreate{2008-04-24}
\isIndication{false}
\isCorrection{true}
\chapitre{Anneau, corps}
\sousChapitre{Anneau, corps}

\contenu{
\texte{

}
\begin{enumerate}
    \item \question{Montrer que $20^{15}-1$ est divisible par $11\times 31\times 61$.}
    \item \question{Trouver le reste de la division de $2^{6754}$  par $1155$.}
\reponse{
$11,31,61$ sont premiers donc $2$ à $2$ premiers entre eux. Ainsi
  $20^{15}\equiv 1[11\cdot31\cdot61]\Leftrightarrow 
  \begin{cases}
    20^{15}\equiv1[11]\\
    20^{15}\equiv1[31]\\
    20^{15}\equiv1[61]
  \end{cases}
  $

  \begin{itemize}
En utilisant le petit théorème de Fermat, on obtient que, modulo
    $11$~: $20^{15}\equiv 20^{5}\equiv -2^{5}\equiv 1[11]$.
$(20^{15})^{2}=20^{30}\equiv 1[31]$. On en déduit que
    $20^{15}\equiv\pm1[31]$. Comme $31\not\equiv1[4]$, d'après le
    théorème de Wilson, $x^{2}=-1$ n'a pas de solution modulo $31$, et
    donc $20^{15}\equiv1[31]$. $20^{2}\equiv-3[31]$ est premier
$20^{15}\equiv(9^{2})^{15}\equiv3^{60}\equiv1[61]$
  \end{itemize}
$1155=11\cdot7\cdot5\cdot3$. De plus (petit théorème de Fermat)
  $2^{6754}\equiv 2^{4}\equiv5[11]$. De même,
  $2^{6754}\equiv2^{4}\equiv2[7]$, $2^{6754}\equiv2^{2}\equiv-1[5]$, et
  $2^{6754}\equiv2^{0}\equiv1[3]$. Or 
  $$
  \begin{cases}
    a\equiv5[11]\\
    a\equiv2[ 7]\\
    a\equiv4[ 5]\\
    a\equiv1[ 3]
  \end{cases}
  \Leftrightarrow
  \begin{cases}
    a\equiv5[11]\\
    a\equiv2[ 7]\\
    a\equiv4[15]
  \end{cases}
  \Leftrightarrow
  \begin{cases}
    a\equiv5[11]\\
    a\equiv-26[105]
  \end{cases}
  \Leftrightarrow
    a\equiv709[1155]\\
  $$
  Donc le reste de la division de $2^{6754}$ par $1155$ est $709$.
}
\end{enumerate}
}
