\uuid{Jzgp}
\exo7id{6474}
\auteur{drutu}
\datecreate{2011-10-16}
\isIndication{false}
\isCorrection{false}
\chapitre{Sous-groupe, morphisme}
\sousChapitre{Sous-groupe, morphisme}

\contenu{
\texte{
Soit $D_\infty $ le groupe des isométries de la droite affine $\R $ formé par l'ensemble des éléments de la forme $\tau_1^n$ et $\tau_1^n \circ \sigma$, où $n\in \Z$, $\tau_1(x)=x+1$ et $\sigma (x)=-x$. On appelle $D_\infty$ {\it le groupe diédral infini}.
}
\begin{enumerate}
    \item \question{Montrer que $H=<\tau_1>$ est le seul sous-groupe cyclique infini d'indice 2 de $D_\infty $. Montrer que $H$ est un sous-groupe distingué de $D_\infty $.}
    \item \question{Montrer que, pour tout sous-groupe $S$ d'ordre 2 de $D_\infty$, on a $D_\infty = SH$.}
    \item \question{Soit $K< D_\infty$ tel que $K \not\subset H$. Montrer que $D_\infty =HK$. En déduire que $K\cap H$ est d'indice 2 dans $K$. Montrer que $K\cap H \neq (e)$ implique $K\simeq D_\infty$.}
    \item \question{Montrer que tout sous-groupe propre de $D_\infty $ est isomorphe soit à $\Z$, soit à $(\pm 1)$, soit à $D_\infty$.}
    \item \question{On note ${\mathcal S}_n$ l'ensemble des sous-groupes $K<D_\infty$ tels que $K \not\subset H$ et $K\cap H=H_n$, où $H_n:=<\tau_1^n>$. Prouver que ${\mathcal S}_n$ contient $n$ éléments.}
\end{enumerate}
}
