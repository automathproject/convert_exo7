\uuid{7064}
\auteur{megy}
\datecreate{2017-01-11}
\isIndication{true}
\isCorrection{false}
\chapitre{Géométrie affine dans le plan et dans l'espace}
\sousChapitre{Propriétés des triangles}

\contenu{
\texte{
Soit $ABCD$ un quadrilatère convexe, et $I$, $J$, $K$, $L$ les milieux de ses côtés. Montrer que $IJKL$ est un parallélogramme soit en utilisant des barycentres, soit le théorème de Thalès. Montrer que l'aire de $ABCD$ est le double de celle de $IJKL$ de deux façons différentes.
}
\indication{Pour l'aire, considérer l'aire du complémentaire de $IJKL$ par exemple, ou bien utiliser les diagonales de $ABCD$.}
}
