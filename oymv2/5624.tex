\uuid{5624}
\auteur{rouget}
\datecreate{2010-10-16}

\contenu{
\texte{
Soit $M\in\mathcal{M}_3(\Rr)$. Montrer que les deux propriétés suivantes sont équivalentes :

\begin{center}
(1) $M^2 = 0$ et (2) $\text{rg}M\leqslant 1$ et $\text{tr}M = 0$.
\end{center}
}
\reponse{
(1) $\Rightarrow$ (2).

$M^2 = 0\Rightarrow\text{Im}M\subset\text{Ker}M\Rightarrow\text{rg}M\leqslant\text{dim}(\text{Ker}M)=3 -\text{rg}M$ et donc $\text{rg}M\leqslant1$.

Si $\text{rg}M=0$ alors $\text{Tr}M = 0$. On suppose maintenant que $\text{rg}M=1$ et donc $\text{dim}(\text{Ker}M)= 2$.

Soit $e_1$ un vecteur non nul de $\text{Im}M$ alors il existe un vecteur $e_3$ (non nul) tel que $Me_3 = e_1$.

On complète la famille libre $(e_1)$ de $\text{Im}M\subset\text{Ker}M$ en $(e_1,e_2)$ base de $\text{Ker}M$. La famille $(e_1,e_2,e_3)$ est une base de $\mathcal{M}_{3,1}(\Rr)$ car

\begin{center}
$ae_1+be_2+ce_3 = 0\Rightarrow M(ae_1+be_2+ce_3) = 0\Rightarrow ce_1 = 0\Rightarrow c = 0$,
\end{center}

puis $a = b = 0$ car la famille $(e_1,e_2)$ est libre.

$M$ est donc semblable à la matrice $\left(
\begin{array}{ccc}
0&0&1\\
0&0&0\\
0&0&0
\end{array}
\right)$ et en particulier $\text{Tr}M = 0$.

(2) $\Rightarrow$ (1).

 
Si $\text{rg}M = 0$, $M^2 = 0$.

Si $\text{rg}M = 1$, on peut se rappeler de l'écriture générale d'une matrice de rang 1 : il existe trois réels $u_1$, $u_2$ et $u_3$ non tous nuls et trois réels $v_1$, $v_2$ et $v_3$ non tous nuls tels que $M=\left(
\begin{array}{ccc}
u_1v_1&u_1v_2&u_1v_3\\
u_2v_1&u_2v_2&u_2v_3\\
u_3v_1&u_3v_2&u_3v_3
\end{array}
\right)$ ou encore il existe deux vecteurs colonnes, tous deux non nuls $U$ et $V$ tels que $M=U{^t}V$. L'égalité $\text{Tr}M =0$ fournit $u_1v_1+u_2v_2+u_3v_3= 0$ ou encore ${^t}UV=0$. Mais alors

\begin{center}
$M^2=U{^t}VU{^t}V=U{^t}({^t}UV){^t}V=0$
\end{center}

Cet exercice admet des solutions bien plus brèves avec des connaissances sur la réduction .
}
}
