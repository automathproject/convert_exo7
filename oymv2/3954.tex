\uuid{3954}
\auteur{quercia}
\datecreate{2010-03-11}
\isIndication{false}
\isCorrection{false}
\chapitre{Dérivabilité des fonctions réelles}
\sousChapitre{Autre}

\contenu{
\texte{
Soit $n \in \N,\ n \ge 2$, et $a,b \in \R$.
Montrer que l'équation $x^n + ax + b = 0$ ne peut avoir plus de deux racines réelles
distinctes si $n$ est pair, et plus de trois racines réelles distinctes si
$n$ est impair.
}
}
