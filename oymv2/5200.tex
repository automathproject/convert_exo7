\uuid{5200}
\auteur{rouget}
\datecreate{2010-06-30}
\isIndication{false}
\isCorrection{true}
\chapitre{Géométrie affine dans le plan et dans l'espace}
\sousChapitre{Géométrie affine dans le plan et dans l'espace}

\contenu{
\texte{
Déterminer un cercle tangent aux trois droites d'équations respectives $y=2x+1$, $y=2x+7$ et $y=-\frac{1}{2}x$.
}
\reponse{
Notons $(D_1)$, $(D_2)$ et $(D_3)$ les droites d'équations respectives $y=2x+1$, $y=2x+7$ et $y=-\frac{1}{2}x$. Soit $\mathcal{C}$ un cercle.

Les droites $(D_1)$ et $(D_2)$ sont parallèles. Donc, $\mathcal{C}$ est un cercle tangent à $(D_1)$ et $(D_2)$ si et seulement si son centre est sur l'ensemble des points à égale distance de $(D_1)$ et $(D_2)$ à savoir la droite d'équation $y=2x+4$ et son rayon est la moitié de la distance de $(D_1)$ à $(D_2)$, ou encore la moitié de la distance d'un point de $(D_1)$, par exemple $(0,1)$, à $(D_2)$. Cette distance vaut $\frac{|2.0-1+7|}{\sqrt{2^2+1^2}}=\frac{6}{\sqrt{5}}$. 
Finalement, $\mathcal{C}$ est un cercle tangent à $(D_1)$ et $(D_2)$ si et seulement si son centre $\Omega$ a des coordonnées de la forme $(a,2a+4)$, $a\in\Rr$, et son rayon vaut $\frac{3}{\sqrt{5}}$.

Un cercle de centre $\Omega$ et de rayon $\frac{3}{\sqrt{5}}$ est tangent à $(D_3)$ si et seulement si la distance de $\Omega$ à $(D_3)$ est le rayon $\frac{3}{\sqrt{5}}$. Donc,

\begin{align*}
\mathcal{C}\;\mbox{solution}&\Leftrightarrow d(\Omega,(D_3))=\frac{3}{\sqrt{5}}
\Leftrightarrow\frac{|a+2(2a+4)|}{\sqrt{5}}=\frac{3}{\sqrt{5}}\Leftrightarrow|5a+8|=3\\
 &\Leftrightarrow5a+8=3\;\mbox{ou}\;5a+8=-3\Leftrightarrow a=-1\;\mbox{ou}\;a=-\frac{11}{5}
\end{align*}

On trouve deux cercles solutions, le cercle $\mathcal{C}_1$ de centre $\Omega_1(-1,2)$ et de rayon $\frac{3}{\sqrt{5}}$ et le cercle $\mathcal{C}_2$ de centre $\Omega_2(-\frac{11}{5},-\frac{2}{5})$ et de rayon $\frac{3}{\sqrt{5}}$
}
}
