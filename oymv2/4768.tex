\uuid{4768}
\auteur{quercia}
\datecreate{2010-03-16}

\contenu{
\texte{
Soit $E$ l'ensemble des fonctions $\R \to \R$ lipschitziennes.
Pour $f \in E$, on pose :
$$
\begin{aligned}
  \|f\| &= |f(0)| + \sup\left( \left|\frac{f(x)-f(y)}{x-y}\right|
    \text{ tq } x\ne y \right),
  \\
  N(f) &= |f(0)| + \sup\left( \left|\frac{f(x)-f(0)}{x}\right| \text{
      tq } x\ne 0 \right).
\end{aligned}
  $$
}
\begin{enumerate}
    \item \question{Montrer que $E$ est un $\R$-ev.}
    \item \question{Montrer que $\|\;.\;\|$ et $N$ sont des normes sur $E$.}
    \item \question{Sont-elles {\'e}quivalentes ?}
\end{enumerate}
}
