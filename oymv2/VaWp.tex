\uuid{VaWp}
\exo7id{2679}
\auteur{matexo1}
\datecreate{2002-02-01}
\isIndication{false}
\isCorrection{true}
\chapitre{Théorème des résidus}
\sousChapitre{Théorème des résidus}

\contenu{
\texte{
Soit $R$ une fraction rationnelle, ou plus g{\'e}n{\'e}ralement une fonction
 m{\'e}romorphe sur $\C$, sans p{\^o}le r{\'e}el.
\begin{itemize}
\item On souhaite calculer
$$ I = \int_0^{+\infty } x R(x^4)\,dx $$
Montrer que cela peut se faire par la m{\'e}thode des r{\'e}sidus, en int{\'e}grant sur un
contour form{\'e} du bord du quart de cercle  $\{ 0< \arg z<{\pi \over 2}, |z|<a\}$.

\item Application: calculer
$$  \int_0^{+\infty } {x\over 1+x^8}\,dx .$$

\item Plus g{\'e}n{\'e}ralement, montrer que si $n$ et $p$ sont des entiers, et $p\geq 3$, on
peut calculer
$$ I(n,p) = \int_0^{+\infty } x^n R(x^p)\, dx $$
sous une condition sur $n$, $p$ que l'on pr{\'e}cisera.

\item Application: calculer
$$ I_{p} = \int_0^{+\infty } {x\over 1+x^{2p}} \,dx.$$
\end{itemize}
}
\reponse{
\begin{itemize}
%\newfunction{Res}
\item Le bord du quart de disque est form{\'e} de trois parties : le segment
$[0,a]$, le quart de cercle, et le segment $[ia, 0]$. L'int{\'e}grale sur le quart de
cercle tend vers z{\'e}ro quand $a\to \infty $ (lemme de Jordan), et en posant $z=iy$, on voit
que
$$ \int_{[ia,0]} z R(z^4)\,dz =  \int_0^a y R(y^4)\,dy $$
donc l'int{\'e}grale sur le contour tend vers $2I$. Donc $I$ est {\'e}gal {\`a} $i\pi $ fois la
somme des r{\'e}sidus de $zR(z^4)$ dans le quart de plan $\{\Re z>0, \Im z >0\}$.

\item Les p{\^o}les sont $e^{i\pi /8}$ et $e^{i3\pi /8}$. 
On trouve $I = \displaystyle{\pi \sqrt 2\over8}$.

\item On int{\`e}gre de m{\^e}me sur le contour form{\'e} du bord de $\{0< \arg z < {2\pi \over
p}, 0<|z|<a\}$. On obtient
$$ \left(1 -e^{i 2\pi  {n+1\over p}}\right) I = 2\pi  i
\sum_{\textstyle
{{\rm p\hat oles\ } z_k\atop 0<\arg z_k<{2\pi \over p}}}
\mbox{\rm Res}\left( z^n R(z^p), z_k\right)
$$

La formule n'est int{\'e}ressante que si le membre de gauche est non nul, c'est-{\`a}-dire
si $n+1$ n'est pas un multiple de $p$.

\item Dans ce cas, il faut donc $p\geq 2$. On est forc{\'e} de prendre une puissance
paire, car on a suppos{\'e} $R$ sans racines r{\'e}elles: ici $R(x) = 1/(1+x^2)$, et $n=1$.
Il y a deux p{\^o}les dans l'angle concern{\'e}, soit $e^{i\pi /2p}$ et $e^{i3\pi /2p}$. Le calcul
est semblable au cas {\it (b)}, et donne
$$ I_{p} = {\pi \over 2p \sin\displaystyle{\pi \over p}}.$$
\end{itemize}
}
}
