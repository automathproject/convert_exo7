\uuid{2BNC}
\exo7id{6327}
\auteur{queffelec}
\datecreate{2011-10-16}
\isIndication{false}
\isCorrection{false}
\chapitre{Théorème de Cauchy-Lipschitz}
\sousChapitre{Théorème de Cauchy-Lipschitz}

\contenu{
\texte{
On considère l'équation différentielle (de Bernoulli) sur $\Rr$ :
$$({\cal E})\quad\quad y'+y+xy^2=0.$$
}
\begin{enumerate}
    \item \question{Recherche des solutions qui ne s'annulent jamais. Transformer l'équation
par le difféomorphisme $(x,y)\in{\Rr}\times{\Rr^*}\to (x,{1\over y})$ en une
équation
$({\cal E'})$ qu'on résoudra. En déduire une famille $(\varphi_\lambda)$ de
solutions de
$({\cal E})$ avec leur intervalle maximal de définition.}
    \item \question{Montrer que par tout point $(x_0,y_0)$ du plan avec $y_0\not=0$, il passe
une solution $\varphi_\lambda$. En déduire toutes les solutions de $({\cal
E})$.}
\end{enumerate}
}
