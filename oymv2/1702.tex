\uuid{1702}
\auteur{barraud}
\datecreate{2003-09-01}
\isIndication{false}
\isCorrection{true}
\chapitre{Réduction d'endomorphisme, polynôme annulateur}
\sousChapitre{Autre}

\contenu{
\texte{
Soit $E$ un $\R$-espace vectoriel de dimension finie $n$, et $u$ un
endomorphisme de $E$.

Soit $x_{0}\in E\setminus\{0\}$. On note $x_{k}=u^{k}(x_{0})$ et $F$ le
sous espace vectoriel engendré par la famille $\{x_{k}, k\in\N\}$, c'est à
dire l'ensemble des combinaisons linéaires finies de vecteurs de ${x_{k},
  k\in\N}$~:
$$
 F=\left\{x\in E\ /\ \exists N\in\N,
   \exists(\alpha_{0}\dots\alpha_{N})\in\R^{N+1},\
   x=\sum_{i=0}^{N}\alpha_{i}x_{i}\right\}
$$
}
\begin{enumerate}
    \item \question{Montrer que $F$ est stable par $u$, c'est à dire que $\forall x\in F,
u(x)\in F$.}
\reponse{$u\big(\sum_{i=1}^{N}\alpha_{i}x_{i}\big)=\sum_{i=1}^{N}\alpha_{i}u(u^{i}(x_{0}))=\sum_{i=1}^{N}\alpha_{i} x_{i+1}$. Donc $\forall x\in F,\ u(x)\in F$.}
    \item \question{Montrer qu'il existe un entier $k\leq n$ tel que
$(x_{0},x_{1},\dots,x_{k})$ soit libre et $(x_{0},x_{1},\dots,x_{k+1})$
soit liée. Montrer alors qu'il existe des scalaires
$(a_{0},a_{1},\dots,a_{k})$ tels que
$$
  x_{k+1}=a_{0}x_{0}+a_{1}x_{1}+\dots+a_{k}x_{k}
$$}
\reponse{Si à un rang $k$, $x_{k+1}$ est une combinaison linéaire des $x_{i}$
    pour $i\leq k$~: $x_{k+1}=\sum_{i=0}^{k}a_{i}x_{i}$. On en déduit que
    $x_{k+2}=\sum_{i=0}^{k}a_{i}x_{i+1}$, et donc que $x_{k+2}\in
    Vect(x_{1},\dots,x_{k+1})\subset Vect(x_{0},\dots, x_{k})$, et par
    récurrence, on obtient finalement que $\forall p>k, x_{p}\in
    Vect(x_{0},\dots,x_{k})$. On en déduit que le rang de la famille
    $\{x_{0},\dots,x_{m}\}$, est strictement croissant avec $m$ puis
    éventuellement constant à partir d'un certain rang. Comme $E$ est de
    dimension finie $n$, on en déduit que ce rang est constant à partir
    d'un rang $k\leq n$~: la famille $(x_{0},\dots,x_{k})$ est alors
    libre, et $x_{k+1}$ est une combinaison linéaire de
    $(x_{0},\dots,x_{k})$.}
    \item \question{\label{q:P0(u)x0=0}
En déduire que le polynôme $P_{0}=X^{k+1}-\sum_{i=0}^{k}a_{i}X^{i}$
satisgfait $\big(P_{0}(u)\big)(x_{0})=0$.}
\reponse{$x_{k+1}-\sum_{i=0}^{k}a_{i}x_{i}=u^{k+1}(x_{0})-\sum_{i=0}^{k}a_{i}u^{i}(x_{0})=0$ donc $P_{0}(u)(x_{0})=0$.}
    \item \question{\label{q:x=P(u)x0}
Montrer que pour tout $x$ de $F$, il existe un polynôme $P\in\R[X]$ tel
que $x=\big(P(u)\big)(x_{0})$.}
\reponse{Si $x\in F$ alors $x=\sum_{i=0}^{N}\alpha_{i}u^{i}(x_{0})$. En posant
    $P=\sum_{i=0}^{N}\alpha_{i} X^{i}$, on a $x=P(u)(x_{0})$.}
    \item \question{A l'aide des questions (\ref{q:P0(u)x0=0}) et (\ref{q:x=P(u)x0}), montrer
que $\forall x\in F, \exists R\in\R_{k}[X], x=\big(R(u)\big)(x_{0})$.

{\it(on pourra effectuer la division eulidienne de $P$ par $P_{0}$)}}
\reponse{Soit $P=QP_{0}+R$ la division euclidienne de $P$ par $P_{0}$, alors
    $\deg(R)<\deg(P_{0})=k+1$. Notons $R=\sum_{i=0}^{k}r_{i} X^{i}$. On a
    $x=P(u)(x_{0})=Q(u)P_{0}(u)(x_{0})+R(u)(x_{0})=R(u)(x_{0})$}
    \item \question{En déduire que $(x_{0}\dots x_{k})$ est une base de $F$.}
\reponse{La famille $(x_{0},\dots,x_{k})$ est donc libre et génératrice dans
    $F$~: c'est une base.}
    \item \question{Ecrire la matrice de la restriction $u_{|_{F}}$ de $u$ à $F$ dans cette
base. Quel est le polynôme caractéristique de $\tilde{u}$~?}
\reponse{La matrice de $u_{|_{F}}$ dans cette base est la matrice compagnon
    associée au polynôme $P_{0}$, et $\chi_{u_{|_{F}}}=P_{0}$.}
    \item \question{Montrer qu'il existe une base $\mathcal{B}$ de $E$ dans la quelle
$$
 \mathrm{Mat}_{_{\mathcal{B}}}(u)=
 \begin{pmatrix}
   C_{1} &0    &\cdots&0\\
   0     &C_{2}&      &\vdots\\
   \vdots&     &\ddots&0     \\
   0     &\cdots&0    &C_{r}
 \end{pmatrix}
$$
où les matrices $C_{i}$ sont des matrices Compagnon.}
\reponse{On choisit un vecteur $y\in E\setminus F$, et on recommence le même
    travail avec ce vecteur, et on continue ainsi jusqu'à avoir obtenu
    une base de tout l'espace. La matrice de $u$ dans la base finale est
    alors du type demandé.}
\end{enumerate}
}
