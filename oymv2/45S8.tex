\uuid{45S8}
\exo7id{7375}
\auteur{mourougane}
\datecreate{2021-08-10}
\isIndication{false}
\isCorrection{true}
\chapitre{Groupe, anneau, corps}
\sousChapitre{Autre}

\contenu{
\texte{

}
\begin{enumerate}
    \item \question{Quelle loi fait de l'ensemble de inversibles de l'anneau $\Z/17\Z$ un groupe ? Ce groupe est-il cyclique ?}
\reponse{C'est la multiplication. Puisque $\Z/17\Z$ est un corps fini, 
par un théorème du cours, le groupe de ses inversibles est un groupe cyclique d'ordre $16$.}
    \item \question{Combien un groupe cyclique d'ordre $91$ a-t-il de générateurs ?}
\reponse{Il en a $\phi(91)=\phi(13\times17)=\phi(13)\times\phi(17)=12\times 16= 192$.}
    \item \question{Décomposer en produits de cycles à supports disjoints, la permutation $$s=\left(\begin{array}{ccccccccc} 
1&2&3&4&5&6&7&8&9\\
5&7&2&6&1&9&3&8&4\end{array}
\right).$$}
\reponse{$s=(15)(273)(469)$.}
    \item \question{Décomposer en produits de transpositions, la permutation
$$\sigma=\left(\begin{array}{ccccccccc} 
1&2&3&4&5&6&7&8&9\\
9&8&6&7&2&3&1&4&5\end{array}
\right).$$}
\reponse{\begin{eqnarray*}
(19)\sigma&=&\left(\begin{array}{ccccccccc} 
1&2&3&4&5&6&7&8&9\\
1&8&6&7&2&3&9&4&5\end{array}\right)\\
(28)(19)\sigma&=&\left(\begin{array}{ccccccccc} 
1&2&3&4&5&6&7&8&9\\
1&2&6&7&8&3&9&4&5\end{array}\right)\\
(36)(28)(19)\sigma&=&\left(\begin{array}{ccccccccc} 
1&2&3&4&5&6&7&8&9\\
1&2&3&7&8&6&9&4&5\end{array}\right)\\
(47)(36)(28)(19)\sigma&=&\left(\begin{array}{ccccccccc} 
1&2&3&4&5&6&7&8&9\\
1&2&3&4&8&6&9&7&5\end{array}\right)\\
(58)(47)(36)(28)(19)\sigma&=&\left(\begin{array}{ccccccccc} 
1&2&3&4&5&6&7&8&9\\
1&2&3&4&5&6&9&7&8\end{array}\right)\\
(79)(58)(47)(36)(28)(19)\sigma&=&\left(\begin{array}{ccccccccc} 
1&2&3&4&5&6&7&8&9\\
1&2&3&4&5&6&7&9&8\end{array}\right)=(89)\\
\end{eqnarray*}
Donc,
$$\sigma=(19)(28)(36)(47)(58)(79)(89).$$}
    \item \question{Le polynôme $P(t_1,t_2,t_3,t_4)=t_1t_2+t_2t_3+t_3t_4+t_4t_1$ en quatre variables est-il symétrique ?}
\reponse{Non, il change dans la transposition de $t_1$ par $t_2$.}
\end{enumerate}
}
