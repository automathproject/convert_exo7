\uuid{jMBB}
\exo7id{7154}
\auteur{megy}
\datecreate{2017-05-13}
\isIndication{false}
\isCorrection{true}
\chapitre{Géométrie affine euclidienne}
\sousChapitre{Géométrie affine euclidienne du plan}

\contenu{
\texte{
%[d'après bac Amérique du sud 2004]
Soient $A$ et $B_0$ deux points et soit $s$ la similitude directe de centre $A$, de rapport $\frac12$ et  d'angle $\frac{3\pi}{4}$. On considère la suite de points $(B_n)_{n\in\N}$ définie par la relation de récurrence $\forall n\in \N,\: B_{n+1}=s(B_n)$.
}
\begin{enumerate}
    \item \question{Faire une figure avec $AB_0=8$ et placer les points $B_n$ jusqu'à $n=4$.}
\reponse{Pour faire la figure on peut
\begin{enumerate}}
    \item \question{Montrer que pour tout  $n\in\N$, les triangles $AB_nB_{n+1}$ et $AB_{n+1}B_{n+2}$ sont semblables.}
\reponse{tracer la droite $(AB_0)$, sa perpendiculaire passant par $A$, puis les deux bissectrices de ces deux droites.}
    \item \question{Dans la suite, on considère le sous-ensemble du plan $S = \bigcup_{n\in \N} [B_nB_{n+1}]$. C'est une spirale polyédrale. Sa longueur est-elle finie ou infinie ? Dans le premier cas, calculer sa longueur.}
\reponse{Tracer les cercles de centre $A$ et de rayons $8$, $4$, $2$ et plus généralement $8\cdot 2^{-k}$. Les points de la suite $(B_n)$ sont sur les intersections de ces droites et cercles.}
\end{enumerate}
}
