\uuid{2677}
\auteur{matexo1}
\datecreate{2002-02-01}

\contenu{
\texte{
Soit $a\in[0,1[$ un r{\'e}el. En int{\'e}grant $e^{az}/\cosh z$ le long du rectangle de
sommets
$-R$, $+R$, $R+i\pi $, $-R+i\pi $, montrer que l'on a
$$ I= \int_{-\infty }^{+\infty } {e^{ax}\over \cosh x} \,dx = {\pi \over \cos(\pi a/2)}.$$
}
\reponse{
Sur le c{\^o}t{\'e} $[R, R+i\pi ]$, la fonction $f(z) = e^{az}/\cosh z$ est major{\'e}e en module
par
$${e^{aR}\over |\cosh(R+iy)|} = {e^{aR}\over \cosh^2R -\sin^2 y}
< {e^{aR}\over \cosh^2R -1} \to 0 \hbox{ quand }R\to \infty $$
De m{\^e}me sur l'autre c{\^o}t{\'e}. L'int{\'e}grale sur le c{\^o}t{\'e} $[R+i\pi , -R+i\pi ]$ vaut
$e^{ia\pi }\int_{-R}^R f(x)\,dx$.

La fonction $f$ n'a qu'un p{\^o}le dans le rectangle, qui est $i\pi /2$. Donc {\`a} la limite
$$ (1+e^{ia\pi }) I = 2i\pi  \mbox{\rm Res}(f, i\pi /2) = 2i\pi  {e^{ia\pi }\over i} = 2\pi  e^{ia\pi }$$
et le r{\'e}sultat en divisant.
}
}
