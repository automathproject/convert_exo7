\uuid{5476}
\auteur{rouget}
\datecreate{2010-07-10}
\isIndication{false}
\isCorrection{true}
\chapitre{Equation différentielle}
\sousChapitre{Résolution d'équation différentielle du premier ordre}

\contenu{
\texte{
Résoudre sur l'intervalle $I$ de $\Rr$ proposé les équations différentielles suivantes~:

$$\begin{array}{ll}
1)\;x\ln xy'+y=x,\;I=]1,+\infty[&2)\;x(xy'+y-x)=1,\;I=]-\infty,0[\\
3)\;2xy'+y=x^4,\;I=]-\infty,0[&4)\;y'+2y=x^2-3x,\;I=\Rr\\
5)\;y'+y=\frac{1}{1+2e^x},\;I=\Rr&6)\;y'\sin x-y\cos x+1=0,\;I=]0,\pi[
\end{array}$$
}
\reponse{
Les fonctions $x\mapsto\frac{1}{x\ln x}$ et $x\mapsto\frac{1}{\ln x}$ sont continues sur $I$ et on sait
que les solutions de $(E)$ sur $I$ sont de la forme
$f_0+\lambda f_1$ où $f_0$ est une solution particulière de $(E)$ et $f_1$ est une solution particulière non nulle de
$(E_H)$.

Soit $f$ une fonction dérivable sur $I$.

\begin{align*}\ensuremath
f\;\mbox{solution de}\;(E)\;\mbox{sur}\;I&\Leftrightarrow\forall x\in I,\;x\ln xf'(x)+f(x)=x\Leftrightarrow\forall x\in I,\;\ln
xf'(x)+\frac{1}{x}f(x)=1\\
 &\Leftrightarrow\forall x\in I,\;(\ln x.f)'(x)=1\Leftrightarrow\exists \lambda\in\Rr/\;\forall x\in I,\;f(x)=\frac{x+\lambda}{\ln x}
\end{align*}
Les fonctions $x\mapsto\frac{1}{x}$ et $x\mapsto1+\frac{1}{x^2}$ sont continues sur $I$ et on sait
que les solutions de $(E)$ sur $I$ sont de la forme
$f_0+\lambda f_1$ où $f_0$ est une solution particulière de $(E)$ et $f_1$ est une solution particulière non nulle de
$(E_H)$.

Soit $f$ une fonction dérivable sur $I$.

\begin{align*}\ensuremath
f\;\mbox{solution de}\;(E)\;\mbox{sur}\;I&\Leftrightarrow\forall x\in I,\;x(xf'(x)+f(x)-x)=1
\Leftrightarrow\forall x\in I,\;(xf)'(x)=x+\frac{1}{x}\\
 &\Leftrightarrow\exists\lambda\in\Rr/\;\forall x\in I,\;
xf(x)=\frac{x^2}{2}+\ln(-x)+\lambda\Leftrightarrow\exists\lambda\in\Rr/\;\forall x\in I,\;f(x)=\frac{x}{2}+\frac{\ln
(-x)+\lambda}{x}.
\end{align*}

Les solutions de $(E)$ sur  $I$ sont les fonctions de la forme $x\mapsto\frac{x}{2}+\frac{\ln(-x)+\lambda}{x}$,
$\lambda\in\Rr$.
Les fonctions $x\mapsto\frac{1}{2x}$ et $x\mapsto\frac{x^3}{2}$ sont continues sur $I=]-\infty,0[$ et on
saitque les solutions de $(E)$ sur $I$ sont de la forme
$f_0+\lambda f_1$ où $f_0$ est une solution particulière de $(E)$ et $f_1$ est une solution particulière non nulle de
$(E_H)$.

Soit $f$ une fonction dérivable sur $I$.

\begin{align*}\ensuremath
f\;\mbox{solution de}\;(E)\;\mbox{sur}\;I&\Leftrightarrow\forall x\in I,\;f'(x)+\frac{1}{2x}f(x)=\frac{x^3}{2}\\
 &\forall x\in I,\;e^{\ln|x|/2}f'(x)+\frac{1}{2x}e^{\ln|x|/2}f(x)=\frac{x^3}{2}e^{\ln|x|/2}\Leftrightarrow\forall x\in
I,\;(\sqrt{-x}f)'(x)=-\frac{1}{2}(-x)^{7/2}\\
 &\Leftrightarrow\exists\lambda\in\Rr/\;\forall x\in I,\;\sqrt{-x}f(x)=\frac{1}{9}(-x)^{9/2}+\lambda\\
 &\Leftrightarrow\exists\lambda\in\Rr/\;\forall x\in I,\;f(x)=\frac{1}{9}x^4+\frac{\lambda}{\sqrt{-x}}
\end{align*}

Les solutions de $(E)$ sur  $I$ sont les fonctions de la forme $x\mapsto\frac{x^4}{9}+\frac{\lambda}{\sqrt{-x}}$,
$\lambda\in\Rr$.
Les fonctions $x\mapsto2$ et $x\mapsto x^2-3x$ sont continues sur $\Rr$ et on sait
que les solutions de $(E)$ sur $\Rr$ sont de la forme
$f_0+\lambda f_1$ où $f_0$ est une solution particulière de $(E)$ et $f_1$ est une solution particulière non nulle de
$(E_H)$.

Soit $f$ une fonction dérivable sur $\Rr$.

\begin{align*}\ensuremath
f\;\mbox{solution de}\;(E)\;\mbox{sur}\;\Rr&\Leftrightarrow\forall x\in\Rr,\;f'(x)+2f(x)=x^2-3x\\
 &\Leftrightarrow\forall x\in\Rr,\;e^{2x}f'(x)+2e^{2x}f(x)=(x^2-3x)e^{2x}\Leftrightarrow\forall
x\in\Rr,\;(e^{2x}f)'(x)=(x^2-3x)e^{2x}
\end{align*}

Recherche d'une primitive sur ${\Rr}$ de la fonction $x\mapsto(x^2-3x)e^{2x}$.
\begin{itemize}
[\textbf{1ère méthode}.] Deux intégrations par parties fournissent~:

\begin{align*}\ensuremath
\int_{}^{}(x^2-3x)e^{2x}\;dx&=\frac{1}{2}(x^2-3x)e^{2x}-\frac{1}{2}\int_{}^{}(2x-3)e^{2x}\;dx\\
 &=\frac{1}{2}(x^2-3x)e^{2x}-\frac{1}{4}(2x-3)e^{2x}+\frac{1}{2}\int_{}^{}e^{2x}\;dx\\
 &=\frac{1}{4}(2x^2-8x+3)e^{2x}+\frac{1}{4}e^{2x}+C=\frac{1}{2}(x^2-4x+2)e^{2x}+C
\end{align*}
[\textbf{2ème méthode}.] Cherchons les primitives de $x\mapsto(x^2-3x)e^{2x}$ sous la forme
$x\mapsto(ax^2+bx+c)e^{2x}$.

$$((ax^2+bx+c)e^{2x})'=(2(ax^2+bx+c)+(2ax+b))e^{2x}=(2ax^2+2(a+b)x+b+2c)e^{2x}.$$

Donc,

$$((ax^2+bx+c)e^{2x})'=(x^2-3x)e^{2x}\Leftrightarrow
\left\{
\begin{array}{l}
2a=1\\
2(a+b)=-3\\
b+2c=0
\end{array}
\right.\Leftrightarrow\left\{
\begin{array}{l}
a=\frac{1}{2}\\
b=-2\\
c=1
\end{array}
\right..$$

\end{itemize}

\textbf{Résolution de} $\bf{(E)}$.

\begin{align*}\ensuremath
f\;\mbox{solution de}\;(E)\;\mbox{sur}\;\Rr&\Leftrightarrow\forall
x\in\Rr,\;(e^{2x}f)'(x)=(x^2-3x)e^{2x}\Leftrightarrow\exists\lambda\in\Rr/\;\forall
x\in\Rr,\;e^{2x}f(x)=(\frac{x^2}{2}-2x+1)e^{2x}+\lambda\\
 &\Leftrightarrow\exists\lambda\in\Rr/\;\forall
x\in\Rr,\;f(x)=\frac{x^2}{2}-2x+1+\lambda e^{-2x}.
\end{align*}

Les solutions de $(E)$ sur  $\Rr$ sont les fonctions de la
forme$x\mapsto\frac{x^2}{2}-2x+1+\lambda e^{-2x}$,$\lambda\in\Rr$.
Les fonctions $x\mapsto1$ et $x\mapsto\frac{1}{1+2e^x}$ sont continues sur $\Rr$ et on sait
que les solutions de $(E)$ sur $\Rr$ sont de la forme
$f_0+\lambda f_1$ où $f_0$ est une solution particulière de $(E)$ et $f_1$ est une solution particulière non nulle de
$(E_H)$.

Soit $f$ une fonction dérivable sur $\Rr$.

\begin{align*}\ensuremath
f\;\mbox{solution de}\;(E)\;\mbox{sur}\;\Rr&\Leftrightarrow\forall x\in\Rr,\;f'(x)+f(x)=\frac{1}{1+2e^x}
\Leftrightarrow\forall x\in\Rr,\;e^xf'(x)+e^xf(x)=\frac{e^x}{1+2e^x}\\
 &\Leftrightarrow\exists\lambda\in\Rr/\;\forall x\in\Rr,\;e^xf(x)=\frac{1}{2}\ln(1+2e^x)+\lambda\\
 &\Leftrightarrow\exists\lambda\in\Rr/\;\forall x\in\Rr,\;f(x)=(\frac{1}{2}\ln(1+2e^x)+\lambda)e^{-x}
\end{align*}

Les solutions de $(E)$ sur  $\Rr$ sont les fonctions de la forme
$x\mapsto(\frac{1}{2}\ln(1+2e^x)+\lambda)e^{-x}$,$\lambda\in\Rr$.
Les fonctions $x\mapsto-\frac{\cos x}{\sin x}$ et $x\mapsto-\frac{1}{\sin x}$ sont continues sur $I=]0,\pi[$
et on sait que les solutions de $(E)$ sur $\Rr$ sont de la forme
$f_0+\lambda f_1$ où $f_0$ est une solution particulière de $(E)$ et $f_1$ est une solution particulière non nulle de
$(E_H)$.

Mais $x\mapsto\sin x$ est une solution non nulle de $(E_H)$ sur $I$ et $x\mapsto\cos x$ est une solution de $(E)$ sur
$]0,\pi[$.

Les solutions de $(E)$ sur  $]0,\pi[$ sont les fonctions de la forme $x\mapsto\lambda\sin x+\cos x$, $\lambda\in\Rr$.
}
}
