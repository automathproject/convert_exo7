\uuid{3268}
\auteur{quercia}
\datecreate{2010-03-08}
\isIndication{false}
\isCorrection{true}
\chapitre{Polynôme, fraction rationnelle}
\sousChapitre{Autre}

\contenu{
\texte{
Soit $n \in \N^*$ et ${\cal E}$ l'ensemble des polyn{\^o}mes {\`a} coefficients
entiers, unitaires de degr{\'e} $n$ et dont toutes les racines sont de module 1.
}
\begin{enumerate}
    \item \question{D{\'e}montrer que $\cal E$ est fini.}
\reponse{Les coefficients de $P$ sont born{\'e}s.}
    \item \question{Pour $P \in \cal E$ de racines $x_1,\dots,x_n$, on note $\widetilde P$ le
      polyn{\^o}me unitaire de racines $x_1^2,\dots,x_n^2$.

      D{\'e}montrer que $\widetilde P \in \cal E$.}
\reponse{$\widetilde P(X^2) = (-1)^nP(X)P(-X)  \Rightarrow  \widetilde P \in {\Z[X]}$.}
    \item \question{En d{\'e}duire que : $\forall\ P \in \cal E$, les racines de $P$ sont des racines de
      l'unit{\'e}.}
\reponse{La suite $(\widetilde {\dot{\dot{\widetilde P}}})$ prend un nombre
      fini de valeurs.}
\end{enumerate}
}
