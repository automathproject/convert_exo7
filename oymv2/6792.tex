\uuid{5K10}
\exo7id{6792}
\auteur{gijs}
\datecreate{2011-10-16}
\isIndication{false}
\isCorrection{false}
\chapitre{Difféomorphisme, théorème d'inversion locale et des fonctions implicites}
\sousChapitre{Difféomorphisme, théorème d'inversion locale et des fonctions implicites}

\contenu{
\texte{
\label{gijsexo2}
}
\begin{enumerate}
    \item \question{Donner la définition d'une sous-variété $M$ de
$\Rr^n$ de dimension $k$.}
    \item \question{Enoncer le théorème des fonctions implicites.

\smallskip
 Soit $f : \Rr^n \to \Rr^p$ une fonction
de classe $C^1$ et soit $M = \{\,x\in \Rr^n \mid f(x) =
\mathbf{0}\,\}$. Supposons en plus que $f$ vérifie la
condition~:
$\forall x\in M : \operatorname{rang}(J(x)) = p\ ,$
où $J(x) = \Bigl(\frac{\partial f_i}{\partial
x_j}(x)\Bigr){}_{j=1,\dots, n}^{i=1,\dots, p}$ est la
matrice Jacobienne de $f$ en $x$ de taille $n\times p$.}
    \item \question{Montrer, en utilisant le théorème des
fonctions implicites, que $M$ est une sous-variété de
$\Rr^n$ de dimension $k = n-p$.}
\end{enumerate}
}
