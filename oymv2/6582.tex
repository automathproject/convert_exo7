\uuid{6582}
\auteur{hueb}
\datecreate{2011-10-16}
\isIndication{false}
\isCorrection{false}
\chapitre{Fonction holomorphe}
\sousChapitre{Fonction holomorphe}

\contenu{
\texte{
Etude de l'exponentielle
complexe $f(z) = e^z$ et du logarithme complexe.
}
\begin{enumerate}
    \item \question{Décrire l'image d'une droite $y =c$, $c$ étant une
  constante, par rapport à $f$.}
    \item \question{Décrire l'image d'une droite $x =c$, $c$ étant une
  constante, par rapport à $f$.}
    \item \question{Vérifier que la restriction de $f$ au domaine
  $$W = \{ z = x+iy; |y| < \pi\}$$
  est une bijection de $W$ sur
  $$D=\Bbb C \setminus \{z; z = -x,\ x \in \Bbb R,\ x \geq 0\}.$$}
    \item \question{En déduire l'existence d'une fonction complexe unique $\Phi$,
  avec domaine de définition $D_{\Phi} = D$, de sorte que
  $$
  e^{\Phi (z)} = z, \quad |\textrm{Im} \Phi (z)| < \pi.  $$
  Cette
  fonction est appelée {\it détermination principale\/} du
  logarithme, notée Log; en utilisant un peu plus de théorie on
  montre qu'elle est holomorphe, avec $\textrm{Log}'(z) = \frac 1z$.}
\end{enumerate}
}
