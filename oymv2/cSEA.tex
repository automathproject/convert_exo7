\uuid{cSEA}
\exo7id{3404}
\auteur{quercia}
\datecreate{2010-03-10}
\isIndication{false}
\isCorrection{false}
\chapitre{Matrice}
\sousChapitre{Changement de base, matrice de passage}

\contenu{
\texte{
Soient $A = \begin{pmatrix} 29 & 38 &-18 \cr
                      -11 &-14 &  7 \cr
                       20 & 27 &-12 \cr \end{pmatrix}$
et     $B = \begin{pmatrix}  7 & -8 &  4 \cr
                        3 & -3 &  2 \cr
                       -3 &  4 & -1 \cr \end{pmatrix}$.

Montrer que $A$ et $B$ ont même rang, même déterminant, même trace mais ne sont
pas semblables (calculer $(A-I)^2$ et $(B-I)^2$).
}
}
