\uuid{1839}
\auteur{maillot}
\datecreate{2001-09-01}
\isIndication{false}
\isCorrection{false}
\chapitre{Extremum, extremum lié}
\sousChapitre{Extremum, extremum lié}

\contenu{
\texte{
Soit $f$ une fonction  r\'eelle de classe ${\cal C}^2$ sur un ouvert
$\Omega$ de $\R^2$.
}
\begin{enumerate}
    \item \question{Rappeler une condition n\'ecessaire pour que $f$ pr\'esente un
   extremum local en $(x_0,y_0)$.

Dans la suite de l'exercice, ${\bf a}=(x_0, y_0)$ v\'erifie cette
   condition, c'est-\`a-dire est un {\em point critique} de $f$. On
   pose
$$A = \frac{\partial^2 f}{\partial x^2}({\bf a}),\quad
B = \frac{\partial^2 f}{\partial x \partial y}({\bf a}),\quad
C = \frac{\partial^2 f}{\partial y^2}({\bf a}),$$
$$Q(x,y)=A x^2+2Bxy+Cy^2, \quad \Delta=B^2-AC,$$
$$R(t)=At^2+2Bt+C,\quad S(t)=Ct^2+2Bt+A.$$}
    \item \question{On suppose $\Delta<0$ et $A (\mbox{ou }C) >0$.
  \begin{enumerate}}
    \item \question{Montrer que $\forall t\in \R,$ $R(t)\geq \delta$ et $S(t)\geq
\delta$ pour un certain $\delta>0$.}
    \item \question{On pose $x=r \cos \theta$, $y=r \sin \theta$, avec $\displaystyle
r=\sqrt{x^2+y^2}$, et on suppose que $\sin \theta. \cos \theta \neq
0$.
Montrer successivement~:
$$\begin {array}{l}
\displaystyle Q(x,y) \geq r^2 \delta \sin^2 \theta,\\
Q(x,y) \geq r^2 \delta \cos^2 \theta,\\
Q(x,y) \geq \frac{r^2}{2} \delta.\end{array}$$
En d\'eduire que
$$\forall (x,y) \quad Q(x,y) \geq \frac{r^2}{2} \mbox{Inf}(\delta, 2A,
2C).$$}
    \item \question{Montrer que ${\bf a}$ est un point de minimum local strict de
$f$. On \'ecrira pour cela la formule de Taylor-Young pour $f$ en ce point.}
\end{enumerate}
}
