\uuid{109}
\auteur{bodin}
\datecreate{1998-09-01}

\contenu{
\texte{
Dans $\R^2$, on d\'efinit les ensembles $F_{1}=\{(x,y)\in\R^2,\ y\leq0\}$ et
$F_{2}=\{(x,y)\in\R^2,\ xy\geq1,\ x\geq0\}$. On note $M_1M_2$ la distance usuelle entre deux points $M_1$ et $M_2$ de $\R^2$.
\'Evaluer les propositions suivantes :
}
\begin{enumerate}
    \item \question{$
 \forall\epsilon\in]0,+\infty[\quad
 \exists M_{1}\in F_{1}\quad
 \exists M_{2}\in F_{2}
 \qquad M_{1}M_{2}<\epsilon$}
\reponse{Cette proposition est vraie. En effet soit $\epsilon > 0$,  d\'efinissons $M_1 = (\frac{2}{\epsilon},0)  \in F_1$ et
 $M_2 = (\frac{2}{\epsilon},\frac{\epsilon}{2}) \in F_2$, alors $M_1M_2=\frac{\epsilon}{2} < \epsilon$. Ceci \'etant vrai quelque soit
$\epsilon >0$ la proposition est donc d\'emontr\'ee.}
    \item \question{$
 \exists M_{1}\in F_{1}\quad
 \exists M_{2}\in F_{2}\quad
 \forall\epsilon\in]0,+\infty[
 \qquad M_{1}M_{2}<\epsilon$}
\reponse{Soit deux points fix\'es $M_1$, $M_2$ v\'erifiant cette
proposition, la distance $d= M_1M_2$ est
 aussi petite que l'on veut donc elle est nulle, donc $M_1 = M_2$ ; or les ensembles $F_1$ et $F_2$
sont disjoints. Donc la proposition est fausse. La n\'egation de
cette proposition est :
$$ \forall M_1 \in F_1 \ \  \forall M_2 \in F_2 \quad \exists \epsilon \in ]0,+\infty[ \quad  \quad M_1M_2 \geqslant \epsilon $$ et cela exprime le fait que les ensembles $F_1$ et $F_2$ sont disjoints.}
    \item \question{$
 \exists\epsilon\in]0,+\infty[\quad 
 \forall M_{1}\in F_{1}\quad
 \forall M_{2}\in F_{2}
 \qquad M_{1}M_{2}<\epsilon$}
\reponse{Celle ci est \'egalement fausse, en effet supposons qu'elle soit
vraie, soit alors $\epsilon$ correspondant \`a cette proposition.
Soit $M_1=(\epsilon +2,0)$ et $M_2 = (1,1)$, on a $M_1M_2 >
\epsilon+1$ ce qui est absurde. La n\'egation est :
$$\forall \epsilon \in ]0,+\infty[ \quad  \exists M_1 \in F_1 \ \  \exists M_2 \in F_2 \quad   \quad M_1M_2 \geqslant \epsilon $$
C'est-\`a-dire que l'on peut trouver deux points aussi
\'eloign\'es l'un de l'autre  que l'on veut.}
    \item \question{$
 \forall M_{1}\in F_{1}\quad
 \forall M_{2}\in F_{2}\quad
 \exists\epsilon\in]0,+\infty[
 \qquad M_{1}M_{2}<\epsilon$}
\reponse{Cette proposition est vraie, il suffit de choisir
$\epsilon=M_1M_2+1$. Elle signifie que la distance entre deux
points donn\'es est un nombre fini !}
\indication{Faire un dessin de $F_1$ et de $F_2$.
Essayer de voir si la difficult\'e pour r\'ealiser les assertions vient de
$\epsilon$ ``petit'' (c'est-\`a-dire proche de $0$) ou de $\epsilon$ ``grand''
(quand il tend vers $+\infty$).}
\end{enumerate}
}
