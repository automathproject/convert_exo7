\uuid{1764}
\auteur{maillot}
\datecreate{2001-09-01}
\isIndication{false}
\isCorrection{false}
\chapitre{Topologie}
\sousChapitre{Compacité}

\contenu{
\texte{
Soit $E=\R^d$ muni d'une norme $\|\cdot\|$. On d\'efinit la
\emph{distance} d'un \'el\'ement $x_0$ de $E$ \`a une partie $A$ de $E$,
not\'ee $d(x_0,A)$, par la formule $$d(x_0,A)=\inf_{x\in A}\|x-x_0\|.$$
}
\begin{enumerate}
    \item \question{Supposons $A$ compact. Montrer que pour tout $x_0\in E$
il existe $y\in A$ tel que $d(x_0,A)=\|y-x_0\|$.}
    \item \question{Montrer que le r\'esultat est encore vrai si on suppose seulement
que $A$ est ferm\'e. (On remarquera que pour toute partie $B$ de $A$ on
a $d(x_0,B)\ge d(x_0,A)$.)}
    \item \question{Montrer que l'application qui \`a $x_0$ associe $d(x_0,A)$ est
continue sur $E$ (sans aucune hypoth\`ese sur $A$).}
    \item \question{En d\'eduire que si $A$ est un ferm\'e de $E$ et $B$ un compact de
$E$ tels que $A$ et $B$ sont disjoints, alors il existe une
constante $\delta>0$ telle que
$$\|a-b\| \ge \delta \qquad \forall (a,b)\in A\times B.$$}
    \item \question{Montrer par un contre-exemple que le r\'esultat est faux si on
suppose seulement que $A$ et $B$ sont deux ferm\'es disjoints.}
\end{enumerate}
}
