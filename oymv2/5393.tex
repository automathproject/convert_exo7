\uuid{5393}
\auteur{rouget}
\datecreate{2010-07-06}
\isIndication{false}
\isCorrection{true}
\chapitre{Continuité, limite et étude de fonctions réelles}
\sousChapitre{Autre}

\contenu{
\texte{
\label{exo:rouconti}
Soit $f$ continue sur $[a,b]$ à valeurs dans $[a,b]$. Montrer que $f$ a un point fixe.
}
\reponse{
Pour $x\in[a,b]$, posons $g(x)=f(x)-x$. $g$ est continue sur $[a,b]$ puisque $f$ l'est. De plus, $g(a)=f(a)-a\geq0$ et $g(b)=f(b)-b\leq0$. D'après le théorème des valeurs intermédiaires, $g$ s'annule au moins une fois sur $[a,b]$ ou encore, l'équation $f(x)=x$ admet au moins une solution dans $[a,b]$.
}
}
