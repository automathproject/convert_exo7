\uuid{531}
\auteur{cousquer}
\datecreate{2003-10-01}
\isIndication{false}
\isCorrection{false}
\chapitre{Suite}
\sousChapitre{Convergence}

\contenu{
\texte{

}
\begin{enumerate}
    \item \question{Soit $(u_n)$, $(v_n)$, $(w_n)$ trois suites telles que pour $n$ assez
grand on ait $v_n\leq u_n\leq w_n$. On suppose que $(v_n)$ et $(w_n)$
sont convergentes, et on note $v=\lim v_n$ et $w=\lim w_n$.
Montrer que pour tout $\epsilon$ positif, on a
$v-\epsilon\leq u_n \leq w+\epsilon$ pour $n$~assez grand
(\emph{théorème d'encadrement}).
Que peut-on en déduire si $v=w$~?}
    \item \question{Soit $(u_n)$ une suite convergente de limite~$l$.
Montrer que la suite
$$v_n = \frac{u_1+u_2+ \cdots+u_n}{n}$$
est convergente et a pour limite~$l$.
Pour cela, encadrer $u_n$ à $\epsilon$~près pour $n$~assez grand,
et en déduire un encadrement de~$v_n$.}
\end{enumerate}
}
