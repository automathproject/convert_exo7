\uuid{2oQR}
\exo7id{7096}
\auteur{megy}
\datecreate{2017-01-21}
\isIndication{true}
\isCorrection{true}
\chapitre{Géométrie affine euclidienne}
\sousChapitre{Géométrie affine euclidienne du plan}

\contenu{
\texte{
% tags : homothéties, translations
Soient $f$ et $g$ deux homothéties de même rapport et de centres distincts. Déterminer la nature de $f\circ g^{-1}$.
}
\indication{Quelle est la partie linéaire de $f\circ g^{-1}$ ?}
\reponse{
La partie linéaire de $f\circ g^{-1}$ est l'identité. Donc c'est une translation. En regardant l'image d'un des centres, on trouve que le vecteur est $(1-\lambda)O1O2$.
}
}
