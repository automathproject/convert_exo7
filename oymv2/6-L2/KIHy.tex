\uuid{KIHy}
\exo7id{7116}
\auteur{megy}
\datecreate{2017-01-21}
\isIndication{true}
\isCorrection{true}
\chapitre{Géométrie affine euclidienne}
\sousChapitre{Géométrie affine euclidienne du plan}

\contenu{
\texte{
% similitudes, translations, ou bien complexes
% Ressemble à Van Aubel
Soit un quadrilatère convexe $ABCD$. Les points $E$, $F$, $G$, $H$ sont tels que $AEB$, $BFC$, $CGD$, $DHA$ sont rectangles isocèles en respectivement $E$, $F$, $G$, $H$. Les triangles $AEB$ et $CGD$ sont vers l'extérieur de $ABC$ , les triangles $BFC$ et $DHA$ vers l'intérieur.
Montrer que $EFGH$ est un parallélogramme.
}
\indication{Considérer la similitude de centre $A$, d'angle $\pi/4$ et de rapport $\sqrt 2$, et celle de centre $C$, d'angle $-\pi/4$, et de rapport $1/\sqrt 2$. Ou alors, utiliser les nombres complexes.}
\reponse{
La composée est une translation, on en déduit que $\overrightarrow{HG}=\overrightarrow{EF}$.
}
}
