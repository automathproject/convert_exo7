\uuid{bbAp}
\exo7id{7437}
\auteur{mourougane}
\datecreate{2021-08-10}
\isIndication{false}
\isCorrection{false}
\chapitre{Géométrie affine dans le plan et dans l'espace}
\sousChapitre{Géométrie affine dans le plan et dans l'espace}

\contenu{
\texte{
Dans l'espace affine $E=\Rr^3_{aff}$, on considère un endomorphisme affine $f$ qui admet un hyperplan de points fixes $F$. Soit $A$ un point de $E$ qui n'est pas dans $F$.
}
\begin{enumerate}
    \item \question{Que dire de $f$ si $f(A)=A$ ?}
    \item \question{On suppose maintenant que $f(A)$ est différent de $A$. On suppose de plus que la droite $(Af(A))$ coupe le plan $F$ en un point $\Omega$.
Expliquer comment construire $f(M)$? 
On dit alors que l'application $f$ est une affinité.}
    \item \question{On suppose maintenant que $f(A)$ est différent de $A$. On suppose de plus que la droite $(Af(A))$ ne rencontre pas le plan $F$.
Expliquer comment construire $f(M)$? 
On dit alors que l'application $f$ est une transvection.}
\end{enumerate}
}
