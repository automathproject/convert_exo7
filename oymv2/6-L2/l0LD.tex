\uuid{l0LD}
\exo7id{5058}
\auteur{quercia}
\datecreate{2010-03-17}
\isIndication{false}
\isCorrection{true}
\chapitre{Surfaces}
\sousChapitre{Surfaces paramétrées}

\contenu{
\texte{
Soient ${\cal S}_1$, ${\cal S}_2$ les surfaces d'équations
$x^2 + y^2 + xy = 1$ et $y^2 + z^2 + yz = 1$,
et $\mathcal{C} = {\cal S}_1 \cap {\cal S}_2$.
}
\begin{enumerate}
    \item \question{Donner en tout point de $\mathcal{C}$ le vecteur tangent à $\mathcal{C}$.}
\reponse{$\begin{pmatrix} (x+2y)(y+2z) \cr -(2x+y)(y+2z) \cr (2x+y)(2y+z)\cr \end{pmatrix}$
             sauf pour $M=\pm\frac2{\sqrt3}(1,-2,1)$.}
    \item \question{Montrer que $\mathcal{C}$ est la réunion de deux courbes planes.}
\reponse{$(x-z)(x+y+z)=0$.}
    \item \question{Quelle est la projection de $\mathcal{C}$ sur $Oxz$ ?}
\reponse{segment $x=z\in {[-1,1]}$ et ellipse $x^2+z^2+xz=1$.}
\end{enumerate}
}
