\uuid{cqFq}
\exo7id{7134}
\auteur{megy}
\datecreate{2017-02-08}
\isIndication{true}
\isCorrection{false}
\chapitre{Géométrie affine euclidienne}
\sousChapitre{Géométrie affine euclidienne du plan}

\contenu{
\texte{
% cocyclicité
}
\begin{enumerate}
    \item \question{Soit $\Gamma$ un cercle, $O$ son centre et $M$ un point n'appartenant pas à $\Gamma$. Deux sécantes issues de $M$ coupent $\Gamma$ respectivement en $A$ et $B$, et en $C$ et $D$. Démontrer l'égalité:
\[2 (\overrightarrow{MA},\overrightarrow{MC}) = (\overrightarrow{OB},\overrightarrow{OD}) -  (\overrightarrow{OC},\overrightarrow{OA}).\]}
    \item \question{Soient $A$ et $B$ deux points d'un cercle $\Gamma$ de centre $O$ et de rayon $r$. Sur la droite $(OA)$, soit $C$ le point extérieur au cercle tel que la droite $(CB)$ recoupe le cercle en un point $M$ vérifiant $MC=r$. (On ne demande pas de construire ce point, le placer approximativement sur la figure.) Montrer que $\widehat{ACB} = \frac13 \widehat{AOB}$.}
\indication{Décomposer $(\overrightarrow{MA},\overrightarrow{MC})$ en $(\overrightarrow{MA},\overrightarrow{AD}) + (\overrightarrow{AD},\overrightarrow{MC})$.% puis angles inscrits / angles au centre.}
\end{enumerate}
}
