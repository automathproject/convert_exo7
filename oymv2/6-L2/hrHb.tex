\uuid{hrHb}
\exo7id{2481}
\auteur{matexo1}
\datecreate{2002-02-01}
\isIndication{false}
\isCorrection{false}
\chapitre{Analyse vectorielle}
\sousChapitre{Forme différentielle, champ de vecteurs, circulation}

\contenu{
\texte{
Soit $\Sigma$ une surface de $\R^3$ de bord $\Gamma  = \partial  \Sigma
$, et $\bf U$ un champ de
vecteurs de $\R^3$. En consid\'erant la forme diff\'erentielle $\omega  = {\bf
U}_1 \,dx +
{\bf U}_2 \,dy + {\bf U}_3 \,dz$, montrer la forme vectorielle de la formule de
Stokes\,:
$$\int_\Sigma  \operatorname{rot}{\bf U}\cdot {\bf n} \,d\sigma  = 
\oint_{\Gamma } {\bf U}\cdot d {\bf r},$$
o\`u ${\bf n}$ d\'esigne le vecteur normal \`a $\Sigma$.
Expliciter cette formule dans les cas o\`u $\Sigma $ est donn\'ee\,: a) sous la
forme directe
$z = f(x,y)$, avec $(x,y) \in D \subset \R^2$\,; b) sous la forme intrins\`eque
$f(x,y,z) =
0$\,; c) sous la forme param\'etrique $x= x(u,v), y=y(u,v), z=z(u,v)$, avec
$(u,v)\in
D\subset \R^2$.
}
}
