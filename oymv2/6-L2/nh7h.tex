\uuid{nh7h}
\exo7id{4919}
\auteur{quercia}
\datecreate{2010-03-17}
\isIndication{false}
\isCorrection{true}
\chapitre{Conique}
\sousChapitre{Hyperbole}

\contenu{
\texte{
Soit ${\cal H}$ une hyperbole équilatère de dimension $a$.
On se place dans un ROND $(O,\vec i,\vec j)$ construit sur les
asymptotes de ${\cal H}$.
}
\begin{enumerate}
    \item \question{Déterminer l'équation de ${\cal H}$ dans ce repère.}
    \item \question{Soit $ABC$ un triangle rectangle en $A$ dont les trois sommets sont
    sur ${\cal H}$. Montrer que la tangente en $A$ est orthogonale à $(BC)$.}
    \item \question{Soit $ABC$ un triangle quelconque dont les sommets sont sur ${\cal H}$.
    Montrer que l'orthocentre y est aussi.}
\reponse{
$xy= \frac{a^2}2$.
}
\end{enumerate}
}
