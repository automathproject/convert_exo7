\uuid{mr4E}
\exo7id{4908}
\auteur{quercia}
\datecreate{2010-03-17}
\isIndication{false}
\isCorrection{true}
\chapitre{Conique}
\sousChapitre{Parabole}

\contenu{
\texte{
Soit $\cal P$ une parabole de paramètre $p$ et $A \in \cal P$. Soit $B$ le
point où la normale à $\cal P$ en $A$ recoupe $\cal P$. Déterminer la longueur
minimale de $AB$.
}
\reponse{
$A:(t^2/2p,t)$, $B:(u^2/2p,u)$ avec $t(t+u) = -2p^2$.
         $AB$ est minimal pour $t^2=2p^2$ et vaut alors $3p\sqrt3$.
}
}
