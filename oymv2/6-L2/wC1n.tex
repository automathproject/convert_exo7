\uuid{wC1n}
\exo7id{5519}
\auteur{rouget}
\datecreate{2010-07-15}
\isIndication{false}
\isCorrection{true}
\chapitre{Géométrie affine dans le plan et dans l'espace}
\sousChapitre{Géométrie affine dans le plan et dans l'espace}

\contenu{
\texte{
Déterminer la perpendiculaire commune aux droites $(D)$ et $(D')$~:~$(D)$ $\left\{
\begin{array}{l}
x+y-3z+4=0\\
2x-z+1=0
\end{array}
\right.$ et $(D')$ $\left\{
\begin{array}{l}
x=z-1\\
y=z-1
\end{array}
\right.$.
}
\reponse{
\textbullet~Déterminons un repère de $(D)$.

\begin{center}
$\left\{
\begin{array}{l}
x+y-3z+4=0\\
2x-z+1=0
\end{array}
\right.\Leftrightarrow\left\{
\begin{array}{l}
y-3z=-x-4\\
z=2x+1
\end{array}
\right.\Leftrightarrow\left\{
\begin{array}{l}
y=5x-1\\
z=2x+1
\end{array}
\right.$
\end{center}
Un repère de $(D)$ est $\left(A,\overrightarrow{u}\right)$ où $A(0,-1,1)$ et $\overrightarrow{u}(1,5,2)$.
\textbullet~Puisque un système d'équations de $(D')$ est $\left\{
\begin{array}{l}
x=z-1\\
y=z-1
\end{array}
\right.$, un repère de $(D')$ est $\left(A',\overrightarrow{u'}\right)$ où $A'(-1,-1,0)$ et $\overrightarrow{u'}(1,1,1)$.
\textbullet~$\overrightarrow{u}\wedge\overrightarrow{u'}=\left(
\begin{array}{c}
1\\
5\\
2
\end{array}
\right)\wedge\left(
\begin{array}{c}
1\\
1\\
1
\end{array}
\right)=\left(
\begin{array}{c}
3\\
1\\
-4
\end{array}
\right)\neq\overrightarrow{0}$.
\rule{0mm}{5mm}Puisque $\overrightarrow{u}$ et $\overrightarrow{u'}$ ne sont pas colinéaires, les droites $(D)$ et $(D')$ ne sont parallèles. Ceci assure l'unicité de la perpendiculaire commune  à $(D)$ et $(D')$.

\textbullet~Un système d'équations de la perpendiculaire commune est 
$\left\{
\begin{array}{l}
\left[\overrightarrow{AM},\overrightarrow{u},\overrightarrow{u}\wedge\overrightarrow{u'}\right]=0\\
\rule{0mm}{6mm}\left[\overrightarrow{A'M},\overrightarrow{u'},\overrightarrow{u}\wedge\overrightarrow{u'}\right]=0
\end{array}
\right.$. Or,

\begin{center}
$\left[\overrightarrow{AM},\overrightarrow{u},\overrightarrow{u}\wedge\overrightarrow{u'}\right]=\left|
\begin{array}{ccc}
x&1&3\\
y+1&5&1\\
z-1&2&-4
\end{array}
\right|=-22x+10(y+1)-14(z-1)=-22x+10y-14z+24$,
\end{center}
et

\begin{center}
$\left[\overrightarrow{A'M},\overrightarrow{u'},\overrightarrow{u}\wedge\overrightarrow{u'}\right]=\left|
\begin{array}{ccc}
x+1&1&3\\
y+1&1&1\\
z&1&-4
\end{array}
\right|=-5(x+1)+7(y+1)-2z=-5x+7y-2z+2$.
\end{center}
Donc

\begin{center}
\shadowbox{
\begin{tabular}{c}
un système d'équations cartésienne de la perpendiculaire commune à $(D)$ et $(D')$ est\\
$\left\{
\begin{array}{l}
11x-5y+7z=12\\
5x-7y+2z=2
\end{array}
\right.$.
\end{tabular}
}
\end{center}
}
}
