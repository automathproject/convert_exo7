\uuid{tqTX}
\exo7id{7062}
\auteur{megy}
\datecreate{2017-01-11}
\isIndication{true}
\isCorrection{true}
\chapitre{Géométrie affine dans le plan et dans l'espace}
\sousChapitre{Propriétés des triangles}

\contenu{
\texte{
% orthocentre, triangle rectangle inscrit 
% cercle circonscrit, hauteurs
On donne  un cercle $\mathcal C$, un diamètre $[AB]$ et un troisième point $M$ du cercle. L'objectif est de construire le projeté orthogonal de $M$ sur $(AB)$ à la règle seule.
}
\begin{enumerate}
    \item \question{Montrer qu'il suffit de construire une droite orthogonale à $(AB)$ coupant le cercle en deux points.}
    \item \question{Construire une telle droite.}
\reponse{
Prendre un deuxième point $N$ sur le cercle de telle sorte que $(AM)$ et $(BN)$ se coupent en un point $C$. On peut alors construire l'orthocentre de $ABC$. La troisième hauteur fournit une droite orthogonale à $(AB)$, coupant le cercle en deux points $P$ et $Q$. On peut alors compléter $MPQ$ en un trapèze (isocèle) $MPQR$, en utilisant les diagonales d'un tel trapèze. La droite $(MR)$ est orthogonale à $(AB)$.
}
\indication{\begin{enumerate}
\item Penser à un trapèze.
\item On peut obtenir une telle droite comme hauteur d'un triangle $ABC$ adéquat.
\end{enumerate}}
\end{enumerate}
}
