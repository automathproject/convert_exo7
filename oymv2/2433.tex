\uuid{2433}
\auteur{matexo1}
\datecreate{2002-02-01}

\contenu{
\texte{
Soit $f$ l'endomorphisme de $\R^3$ dont la matrice par
rapport \`a la base canonique $(e_1, e_2, e_3)$ est
$$A= \left( 
       \begin{array}{ccc}
        15 & -11 & 5 \\
        20 & -15 & 8 \\
        8 &  -7 & 6
       \end{array}
       \right).$$
Montrer que les vecteurs
$$ e'_1 = 2e_1+3e_2+e_3,\quad e'_2 = 3e_1+4e_2+e_3,\quad e'_3 =
e_1+2e_2+2e_3$$
forment une base de $\R^3$ et calculer la matrice de $f$ par
rapport \`a cette base.
}
\reponse{
Notons l'ancienne base $\mathcal{B}=(e_1,e_2,e_3)$
et ce qui sera la nouvelle base $\mathcal{B}'=(e'_1,e'_2,e'_3)$.
Soit $P$ la matrice de passage qui contient -en colonnes- les coordonnées des vecteurs
de la nouvelle base $\mathcal{B}'$ exprimés dans l'ancienne base $\mathcal{B}$
$$P=\begin{pmatrix}
2 & 3 & 1 \\
3 & 4 & 2 \\
1 & 1 & 2 \\      
\end{pmatrix}$$

On vérifie que $P$ est inversible (on va même calculer son inverse) donc
$\mathcal{B}'$ est bien une base.
De plus 
$$P^{-1} = \begin{pmatrix}
-6 & 5 & -2 \\
4 & -3 & 1 \\
1 & -1 & 1 \\             
           \end{pmatrix}
\text{ et on calcule  } B=P^{-1} A P = 
\begin{pmatrix}
1 & 0 & 0 \\
0 & 2 & 0 \\
0 & 0 & 3 \\  
\end{pmatrix}$$

$B$ est la matrice de $f$ dans la base $\mathcal{B}'$.
}
}
