\uuid{7029}
\auteur{megy}
\datecreate{2016-10-28}

\contenu{
\texte{
Dans une station de métro, les usagers ont à leur disposition un tapis roulant de 300m de long.
Un piéton marchant à vitesse constante fait l'aller-retour sur le tapis roulant. À l'aller, il met 1 minute et
30 secondes. Au retour, à contre-sens, il met 4 minutes et 30 secondes.
Déterminer la vitesse du piéton et celle du tapis roulant.
}
}
