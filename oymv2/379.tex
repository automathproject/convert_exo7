\uuid{379}
\auteur{bodin}
\datecreate{1998-09-01}
\isIndication{false}
\isCorrection{true}
\chapitre{Polynôme, fraction rationnelle}
\sousChapitre{Pgcd}

\contenu{
\texte{
Calculer pgcd$(P,Q)$ lorsque :
}
\begin{enumerate}
    \item \question{$P=X^3-X^2-X-2$ et $Q=X^5-2X^4+X^2-X-2$,}
\reponse{$\pgcd(X^3-X^2-X-2,X^5-2 X^4+X^2-X-2) = X-2$.}
    \item \question{$P=X^4+X^3-2X+1$ et $Q=X^3+X+1$.}
\reponse{$\pgcd(X^4+X^3-2 X+1,X^3+X+1) = 1$.}
\end{enumerate}
}
