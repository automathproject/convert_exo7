\uuid{7100}
\auteur{megy}
\datecreate{2017-01-21}
\isIndication{true}
\isCorrection{false}
\chapitre{Géométrie affine euclidienne}
\sousChapitre{Géométrie affine euclidienne du plan}

\contenu{
\texte{
% variations avec d'autres types permutations
Soient $A$, $B$, $C$ et $D$ quatre points du plan tels que trois d'entre eux ne soient jamais alignés. Donner une condition nécessaire et suffisante sur $ABCD$ pour qu'il existe une transformation affine $f$ du plan telle que $f(A)=B$, $f(B)=C$, $f(C)=D$ et $f(D)=A$. Montrer qu'une telle transformation, si elle existe, est nécessairement d'ordre quatre dans le groupe affine.
%
}
\indication{Les images de trois points non alignés déterminent de façon unique une transformation affine. En déduire une condition sur le quatrième point.}
}
