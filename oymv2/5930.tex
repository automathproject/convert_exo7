\uuid{5930}
\auteur{tumpach}
\datecreate{2010-11-11}
\isIndication{false}
\isCorrection{true}
\chapitre{Autre}
\sousChapitre{Autre}

\contenu{
\texte{
On d\'efinit $m_{*}~:\mathcal{P}(\Omega) \rightarrow \mathbb{R}$
par
$$
m_{*}(A) = \left\{\begin{array}{ll} 0 & \text{si}~A = \emptyset\\
1 & \text{sinon}. \end{array}\right.
$$
}
\begin{enumerate}
    \item \question{Montrer que $m_*$ est une mesure ext\'erieure.}
\reponse{Il est clair que $m_{*}(\emptyset)=0$ et que $m_{*}$ est
monotone.

Soit maintenant $\{A_i\}_{i\in \mathbb{N}}\subset
\mathcal{P}(\Omega)$. Si parmi les $A_i$ il existe au moins un
ensemble $A_j$ non vide, on a
$$m_{*}(\bigcup\limits_i A_i)=1=m_{*}(A_j)\leq \sum\limits_i m_{*}(A_i).$$
Si tous les $A_i$ sont vides, alors $\bigcup\limits_i A_i =
\emptyset$, et donc
$$m_{*}(\bigcup\limits_i A_i)=0=\sum\limits_i
m_{*}(A_i).$$ Ainsi $m_{*}$ est $\sigma$-sous-aditive et par
consequent $m_{*}$ est une mesure ext\'erieure.}
    \item \question{Quels sont les ensembles $m_*$-mesurables ?}
\reponse{Les seuls ensembles mesurables sont
$\emptyset$ et $\Omega$, puisque si $A\in \mathcal{P}(\Omega)$ est
tel que $A\neq \emptyset$ et $A\neq \Omega$, alors, pour tout
$Q\in \mathcal{P}(\Omega)$ non vide et non inclus dans $A$, on a
$A\cap Q\neq \emptyset$ et $A^c \cap Q\neq \emptyset$, et donc
$$m_{*}(A\cap Q)+m_{*}(A^c\cap Q)=1+1=2\neq m_{*}(Q)=\left\{
\begin{array}{l}
1 \\
0 \;\;.\\
\end{array}
\right.$$}
    \item \question{V\'erifier le th\'eor\`eme de Carath\'eodory sur cet exemple.}
\reponse{Il est clair que l'ensemble des parties
$m_{*}$-mesurables de $\Omega$,
$\mathcal{M}_{m_{*}}=\{\emptyset,\Omega\}$, est une
$\sigma$-alg\`ebre.

Il est facile de voir aussi que
$$\mu=m_{*}|_{\mathcal{M}_{m_{*}}},\,\mu(\emptyset)=0,\;\;\mu(\Omega)=1,$$
est une mesure sur $(\Omega,\mathcal{M}_{m_{*}})$.}
\end{enumerate}
}
