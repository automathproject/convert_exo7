\uuid{rVLs}
\exo7id{5564}
\auteur{rouget}
\datecreate{2010-10-16}
\isIndication{false}
\isCorrection{true}
\chapitre{Espace vectoriel}
\sousChapitre{Définition, sous-espace}

\contenu{
\texte{
Généralisation de l'exercice \ref{ex:rou1}. Soient $n$ un entier supérieur ou égal à $2$ puis $F_1$, ... , $F_n$ $n$ sous-espaces de $E$ où $E$ est un espace vectoriel sur un sous-corps $\Kk$ de $\Cc$.
Montrer que $\left[(F_1\cup ... \cup F_n\;\text{sous-espace de}\; E)\Leftrightarrow(\text{il existe}\;i\in\llbracket1,n\rrbracket/\;\displaystyle\bigcup_{j\neq i}F_j\subset F_i)\right]$.
}
\reponse{
$\Leftarrow)$ Immédiat .

$\Rightarrow)$ On raisonne par récurrence sur $n$.

Pour $n=2$, c'est l'exercice \ref{ex:rou1}.

Soit $n\geqslant 2$. Supposons que toute réunion de $n$ sous-espaces de $E$ est un sous-espace de $E$ si et seulement si l'un de ces sous-espaces contient tous les autres.

Soient $F_1$,..., $F_n$, $F_{n+1}$ $n+1$ sous-espaces vectoriels de $E$ tels que $F_1\cup...\cup F_{n+1}$ soit un sous-espace vectoriel de $E$. Posons $F=F_1\cup...\cup F_n$.

\textbullet~Si $F_{n+1}$ contient $F$, c'est fini.

\textbullet~Si $F_{n+1}\subset F$, alors $=F_1\cup...\cup F_n=F_1\cup...\cup F_n\cup F_{n+1}$ est un sous-espace vectoriel de $E$. Par hypothèse de récurrence, $F$ est l'un des $F_i$ pour un certain $i$ élément de $\llbracket1,n\rrbracket$. $F_i=F$ contient également $F_{n+1}$ et contient donc tous les $F_j$ pour $j$ élément de $\llbracket1,n+1\rrbracket$.

\textbullet~Supposons dorénavant que $F\not\subset F_{n+1}$ et que $F_{n+1}\not\subset F$ et montrons que cette situation est impossible.

Il existe un vecteur $x$ qui est dans $F_{n+1}$ et pas dans $F$ et un vecteur $y$ qui est dans $F$ et pas dans $F_{n+1}$.

Soit $\lambda$ un élément de $\Kk$. $y-\lambda x$ est un élément de $F\cup F_{n+1}$ (puisque $F\cup F_{n+1}$ est un sous-espace) mais $y-\lambda x$ n'est pas dans $F_{n+1}$ car alors $y=(y -\lambda x)+\lambda x$ y serait ce qui n'est pas.

Donc $\forall\lambda\in\Kk$, $y-\lambda x\in F$. On en déduit que pour tout scalaire $\lambda$, il existe un indice $i(\lambda)$ élément de $\llbracket1,n\rrbracket$ tel que $y-\lambda x\in F_{i(\lambda)}$. Remarquons enfin que si $\lambda\neq\mu$ alors $i(\lambda)\neq i(\mu)$. En effet, si pour $\lambda$ et $\mu$ deux scalaires distincts donnés, il existe un indice $i$ élément de $\llbracket1,n\rrbracket$ tel que $y-\lambda x$ et $y-\mu x$ soient dans $F_i$, alors $x=\frac{(y-\mu x)-(y-\lambda x)}{\mu-\lambda}$  est encore dans $F_i$ et donc dans $F$, ce qui n'est pas. 

Comme l'ensemble des scalaires est infini et que l'ensemble des indices ne l'est pas, on vient de montrer que cette dernière situation n'est pas possible, ce qui achève la démonstration.
}
}
