\uuid{6892}
\auteur{ruette}
\datecreate{2013-01-24}

\contenu{
\texte{
On lance trois dés non pipés. On note le nombre de points (1, 2, 3, 4, 5 ou 6) qui 
apparaît sur la face supérieure de chaque dé. Calculer la probabilité d'avoir :
}
\begin{enumerate}
    \item \question{trois 3,}
\reponse{Dés non pipés signifie que la probabilité de tirer 3 (ou tout autre chiffre) sur un 
dé est $1/6$. Les tirages étant indépendants, la probabilité
d'avoir trois 3 est $(1/6)^3 =1/216$.}
    \item \question{deux 2 et un 1,}
\reponse{Il y a trois manières d'obtenir deux 2 et un 1, et 216 tirages possibles, donc la probabilité cherchée est $3/216=1/72$.}
    \item \question{un 1, un 3, un 5,}
\reponse{$6/216$}
    \item \question{la somme des points égale à 9,}
\reponse{Un total de 9 s'obtient par l'une des additions suivantes,
$$9=6+2+1=5+3+1=5+2+2=4+4+1=4+3+2=3+3+3,$$
où on a rangé les résultats d'un tirage par ordre décroissant. On rencontre 
l'addition 6+2+1 dans $3!=6$ tirages différents. De même pour 5+3+1 et 4+3+2. 
En revanche, 5+2+2 et 4+4+1 ne correspondent qu'à 3 tirages et 3+3+3 à un seul. 
Il y a donc 6+6+6+3+3+1=25 tirages qui donnent une somme de 9. La probabilité 
que la somme soit 9 vaut donc $25/216$.}
    \item \question{la somme des points égale à 10.}
\reponse{Un total de 10 s'obtient par l'une des additions suivantes, 
$$10=6+3+1=6+2+2=5+4+1=5+3+2=4+4+2=4+3+3.$$ 
On rencontre les additions 6+3+1, 5+4+1, 5+3+2 dans $3!=6$ tirages différents. 
En revanche, 6+2+2, 4+4+2 et 4+3+3 ne correspondent qu'à 3 tirages. Il y a donc 
6+6+6+3+3+3=27 tirages qui donnent une somme de 9. La probabilité que la somme 
soit 9 vaut donc $27/216$.}
\end{enumerate}
}
