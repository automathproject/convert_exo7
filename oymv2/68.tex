\uuid{68}
\auteur{cousquer}
\datecreate{2003-10-01}

\contenu{
\texte{
Soit $H$ une hyperbole équilatère de centre $O$, et $M$ un point de $H$.
Montrer que le cercle de centre $M$ qui passe par le symétrique de $M$ par
rapport à $O$ recoupe $H$ en trois points qui sont les sommets d'un triangle
équilatéral.

\emph{Indications :} en choisissant un repère adéquat, $H$ a une équation du type
$xy=1$, autrement dit en identifiant le plan de $H$ au plan complexe, $z^2-\bar
z^2=4i$. En notant $a$ l'affixe de $M$, le cercle a pour équation $\vert
z-a\vert^2=4a\bar a$. On pose $Z=z-a$ et on élimine $\bar Z$ entre les
équations du cercle et de l'hyperbole. En divisant par $Z+2a$ pour éliminer la
solution déjà connue du symétrique de $M$, on obtient une équation du type
$Z^3-A=0$.
}
}
