\uuid{YTbJ}
\exo7id{5057}
\auteur{quercia}
\datecreate{2010-03-17}
\isIndication{false}
\isCorrection{false}
\chapitre{Surfaces}
\sousChapitre{Surfaces paramétrées}

\contenu{
\texte{
Soit ${\cal S}$ une surface d'équation cartésienne $z=f(\rho)$ où
$\rho = \sqrt{x^2+y^2}$ et $f$ est une fonction de classe $\mathcal{C}^2$.
Montrer que la position de ${\cal S}$ par rapport à son plan tangent est donnée
par le signe de $f'(\rho)f''(\rho)$. Interpréter géométriquement ce fait.
}
}
