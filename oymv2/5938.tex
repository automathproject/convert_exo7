\uuid{FGOx}
\exo7id{5938}
\auteur{tumpach}
\datecreate{2010-11-11}
\isIndication{false}
\isCorrection{true}
\chapitre{Tribu, fonction mesurable}
\sousChapitre{Tribu, fonction mesurable}

\contenu{
\texte{
Soit $p>0$. Soit $f~:\mathbb{R}^n \rightarrow\mathbb{R}^+$ la fonction
d\'efinie par $$f(x) = |x|^{-p} \mathbf{1}_{\{|x|<1\}}(x).$$ Calculer
l'int\'egrale de $f$ par rapport \`a la mesure de Lebesgue de
$\mathbb{R}^n$ de deux mani\`eres diff\'erentes~:
}
\begin{enumerate}
    \item \question{[(i)] En utilisant les coordonn\'ees polaires et les
m\'ethodes standard de calcul d'int\'egrales~;}
\reponse{[(i)] On a~:
\begin{eqnarray*} \int_{\mathbb{R}^{n}} f(x)\,dx & = &
\int_{\mathbb{R}^n}|x|^{-p} \mathbf{1}_{\{|x|<1\}}(x) \,dx = \int_{|x|<1}
|x|^{-p}\,dx = \int_{r=0}^{1}\int_{\sigma \in \mathcal{S}_{n-1}}
r^{n-p-1} dr d\sigma\\
& = & \frac{2\pi^{\frac{n}{2}}}{\Gamma\left(\frac{n}{2}\right)}
\int_{0}^{1} r^{n-p-1}\,dr.
\end{eqnarray*}
Pour $n \leq p$, il vient
\begin{eqnarray*}
 \int_{\mathbb{R}^{n}} f(x)\,dx &=& +\infty.
\end{eqnarray*}
Pour $p < n$, il vient
\begin{eqnarray*}
\int_{\mathbb{R}^{n}} f(x)\,dx &=&
\frac{2\pi^{\frac{n}{2}}}{\Gamma\left(\frac{n}{2}\right)}
\left[\frac{r^{n-p}}{(n-p)}\right]_{0}^{1} =
\frac{2\pi^{\frac{n}{2}}}{(n - p)\Gamma\left(\frac{n}{2}\right)}
\end{eqnarray*}}
    \item \question{[(ii)] En
calculant la mesure des ensembles $S_{f}(a) = \{x\in\Omega,
f(x)>a\}$ et la d\'efinition de l'int\'egrale de Lebesgue.}
\reponse{[(ii)] Pour $a\in[0,+\infty[$,
\begin{eqnarray*}
S_{f}(a) = \{ x\in\mathbb{R}^{n}, |x|^{-p} \mathbf{1}_{|x|<1} > a\} =
\{x\in\mathbb{R}^{n}, |x|^{-p} > a\}\cap \mathcal{B}(0, 1),
\end{eqnarray*}
o\`u $\mathcal{B}(0, 1)$ est la boule de centre 0 et de rayon 1.
Ainsi
\begin{eqnarray*}
S_{f}(a) = \{ x\in\mathbb{R}^{n}, a^{-\frac{1}{p}} > |x|\}\cap
\mathcal{B}(0, 1).
\end{eqnarray*}
On en d\'eduit que $S_{f}(a) = \mathcal{B}(0, 1)$ si
$a^{-\frac{1}{p}}>1$, i.e. si $a<1$ et que $S_{f}(a)$ est \'egale
\`a la boule $\mathcal{B}(0, a^{-\frac{1}{p}})$ de centre 0 et de
rayon $a^{-\frac{1}{p}}$ lorsque $a\geq 1$. Il vient alors~:
\begin{eqnarray*}
\int_{\mathbb{R}^{n}} f(x)\,dx &=& \int_{0}^{+\infty}
\mu\left(S_{f}(a)\right)\,da = \int_{0}^{1}
\mu\left(\mathcal{B}(0,1) \right)\,da + \int_{1}^{+\infty}
\mu\left(\mathcal{B}(0, a^{-\frac{1}{p}})\right)\,da\\
& = & \frac{\pi^{\frac{n}{2}}}{\Gamma\left(\frac{n}{2}+1\right)} +
\frac{\pi^{\frac{n}{2}}}{\Gamma\left(\frac{n}{2}+1\right)}\int_{1}^{+\infty}
a^{-\frac{n}{p}}\,da.
\end{eqnarray*}
Si $p\geq n$, on obtient $\int_{\mathbb{R}^{n}} f(x)\,dx =+\infty$
et pour $p<n$, on a~:
\begin{eqnarray*}
\int_{\mathbb{R}^{n}} f(x)\,dx & =
&\frac{\pi^{\frac{n}{2}}}{\Gamma\left(\frac{n}{2}+1\right)} +
\frac{\pi^{\frac{n}{2}}}{\Gamma\left(\frac{n}{2}+1\right)}\left[\frac{a^{-\frac{n}{p}+1}}{-\frac{n}{p}+1}\right]_{1}^{+\infty}\\
&=&\frac{\pi^{\frac{n}{2}}}{\Gamma\left(\frac{n}{2}+1\right)}\left(1
+ \frac{p}{n-p} \right) = \frac{2\pi^{\frac{n}{2}}}{(n -
p)\Gamma\left(\frac{n}{2}\right)}.
\end{eqnarray*}}
\end{enumerate}
}
