\uuid{5423}
\auteur{rouget}
\datecreate{2010-07-06}
\isIndication{false}
\isCorrection{true}
\chapitre{Dérivabilité des fonctions réelles}
\sousChapitre{Autre}

\contenu{
\texte{
Soit $f$ de classe $C^1$ sur $\Rr$ vérifiant pour tout $x$ réel, $f\circ f(x)=\frac{x}{2}+3$. En remarquant que $f(\frac{x}{2}+3)=\frac{f(x)}{2}+3$, montrer que $f'$ est constante puis déterminer $f$.
}
\reponse{
$$\forall x\in\Rr f(\frac{x}{2}+3)=f(f\circ f(x))=f\circ f(f(x))=\frac{f(x)}{2}+3.$$

Puisque $f$ est dérivable sur $\Rr$, on obtient en dérivant $\forall x\in\Rr,\;\frac{1}{2}f'(\frac{x}{2}+3)=\frac{1}{2}f'(x)$,

et donc

$$\forall x\in\Rr,\;f'(\frac{x}{2}+3)=f'(x).$$

Soit alors $x$ un réel donné et $u$ la suite définie par $u_0=x$ et $\forall n\in\Nn,\;u_{n+1}=\frac{1}{2}u_n+3$.

D'après ce qui précède, $\forall n\in\Nn,\;f'(x)=f'(u_n)$.
Maintenant, $u$ est une suite arithmético-géométrique et on sait que 

$$\forall n\in\Nn,\;u_n-6=\frac{1}{2^n}(u_0-6)$$ 

ce qui montre que la suite $u$ converge vers $6$. La suite $(f'(u_n))_{n\geq0}$ est constante, de valeur $f'(x)$. $f'$ étant continue sur $\Rr$, on en déduit que

$$\forall x\in\Rr,\;f'(x)=\lim_{n\rightarrow +\infty}f'(u_n)=f'(\lim_{n\rightarrow +\infty}u_n)=f'(6),$$

ce qui montre que la fonction $f'$ est constante sur $\Rr$ et donc que $f$ est affine.

Réciproquement, pour $x$ réel, posons $f(x)=ax+b$.
 
\begin{align*}\ensuremath
f\;\mbox{solution}&\Leftrightarrow\forall x\in\Rr,\;a(ax+b)+b=\frac{x}{2}+3\Leftrightarrow\forall x\in\Rr,\;(a^2-\frac{1}{2})x+ab+b-3=0\\
 &\Leftrightarrow a^2=\frac{1}{2}\;\mbox{et}\;(a+1)b=3\Leftrightarrow(a=\frac{1}{\sqrt{2}}\;\mbox{et}\;b=3(2-\sqrt{2}))\;\mbox{ou}\;(a=-\frac{1}{\sqrt{2}}\;\mbox{et}\;b=3(2+\sqrt{2})).
\end{align*}

On trouve deux fonctions solutions, les fonctions $f_1$ et $f_2$ définies par~:

$$\forall x\in\Rr,\;f_1(x)=\frac{1}{\sqrt{2}}x+3(2-\sqrt{2})\;\mbox{et}\;f_2(x)=-\frac{1}{\sqrt{2}}x+3(2+\sqrt{2}).$$
}
}
