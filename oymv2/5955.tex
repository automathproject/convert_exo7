\uuid{5955}
\auteur{tumpach}
\datecreate{2010-11-11}

\contenu{
\texte{
\ \\

\textbf{Th\'eor\`eme 1.}(Th\'eor\`eme de Riesz)
Pour tout $1\leq p\leq +\infty$, l'espace $L^{p}(\mu)$ est
complet.

\bigskip

\textbf{Th\'eor\`eme 2.}
Soit $p$  tel que $1\leq p \leq +\infty$ et soit
$\{f_{n}\}_{n\in\mathbb{N}}$ une suite de Cauchy dans $L^{p}(\mu)$
convergeant vers une fonction $f\in L^{p}(\mu)$. Alors il existe
une sous-suite de $\{f_{n}\}_{n\in\mathbb{N}}$ qui converge
ponctuellement presque-partout vers $f$.

\bigskip

\emph{Le but de cet exercice est de d\'emontrer les
th\'eor\`emes~1 et 2.}
}
\begin{enumerate}
    \item \question{\emph{Cas de $L^{\infty}(\mu)$.}
  \begin{enumerate}}
\reponse{\emph{Cas de $L^{\infty}(\mu)$.}

\begin{enumerate}}
    \item \question{Soit $\{f_{n}\}_{n\in\mathbb{N}}$ une suite de Cauchy
de $L^{\infty}(\mu)$. Pour $k, m, n\geq 1$,  consid\'erons les
ensembles
\begin{eqnarray*}
A_{k} := \{ x\in\Omega, |f_{k}(x)| > \|f_{k}\|_{\infty}\}~; &
B_{m,n} := \{ x\in\Omega, |f_{m}(x) - f_{n}(x)| > \|f_{m} -
f_n\|_{\infty} \}.
\end{eqnarray*}
Montrer que $E := \bigcup_{k} A_{k}  \bigcup_{n,m} B_{m,n}$ est de
mesure nulle.}
\reponse{Soit $\{f_{n}\}_{n\in\mathbb{N}}$ une suite de Cauchy
de $L^{\infty}(\mu)$. Pour $k, m, n\geq 1$,  soient les ensembles
\begin{eqnarray*}
A_{k} := \{ x\in\Omega, |f_{k}(x)| > \|f_{k}\|_{\infty}\}~; &
B_{m,n} := \{ x\in\Omega, |f_{m}(x) - f_{n}(x)| > \|f_{m} -
f_n\|_{\infty} \},
\end{eqnarray*}
et $E := \bigcup_{k} A_{k}  \bigcup_{n,m} B_{m,n}$. Par
d\'efinition de la norme infinie, les ensembles $A_k$ et $B_{n,m}$
sont de mesure nulle. Par $\sigma$-sous-additivit\'e de $\mu$, on
a
$$
\mu(E) \leq \sum_{k} \mu(A_k) + \sum_{n,m}\mu(B_{n,m}) = 0.
$$}
    \item \question{Montrer que sur le compl\'ementaire de
$E$, la suite $\{f_n\}_{n\in\mathbb{N}}$ converge uniform\'ement
vers une fonction $f$.}
\reponse{Sur $\Omega\setminus E$, on a~:
$$
\sup_{x\in\Omega\setminus E}|f_{n} - f_{m}| \leq \| f_{n} -
f_{m}\|_{\infty},
$$
i.e. $\{f_{n}\}_{n\in\mathbb{N}}$ est une suite de Cauchy uniforme
sur $\Omega\setminus E$. En particulier, pour tout
$x\in\Omega\setminus E$, la suite $\{f_{n}(x)\}_{n\in\mathbb{N}}$
est une suite de Cauchy r\'eelle, donc est convergeante  car
$\mathbb{R}$ est complet. Notons $f$ la limite ponctuelle de
$f_{n}$ sur $\Omega\setminus E$. Montrons que la suite
$\{f_n\}_{n\in\mathbb{N}}$ converge uniform\'ement vers $f$ sur le
compl\'ementaire de $E$. On a
$$
|f_{n}(x) - f(x)| = \lim_{m\rightarrow +\infty}|f_{n}(x) -
f_{m}(x)|\leq\lim_{m\rightarrow +\infty}\|f_{n} -
f_{m}\|_{\infty}.
$$
Comme $\{f_{n}\}_{n\in\mathbb{N}}$ est de Cauchy dans
$L^{\infty}(\mu)$, pour tout $\varepsilon>0$, il existe un rang
$N_{\varepsilon}$ tel que pour $n,m > N_{\varepsilon}$, $\|f_{n} -
f_{m}\|_{\infty} < \varepsilon$.  Alors pour $n> N_{\varepsilon}$,
$$
\sup_{x\in\Omega\setminus E}|f_{n}(x) - f(x)| \leq \varepsilon.
$$
Il est d\'ecoule que  $\{f_n\}_{n\in\mathbb{N}}$ converge
uniform\'ement vers $f$ sur $\Omega\setminus E$.}
    \item \question{En d\'eduire que
$L^{\infty}(\mu)$ est complet.}
\reponse{\'Etendons la fonction $f$ \`a $\Omega$ en posant $f =
0$ sur $E$. Il reste \`a montrer que la fonction $f$ appartient
\`a $L^{\infty}(\mu)$. Pour $n> N_{\varepsilon}$, et
$x\in\Omega\setminus E$, on a
$$
|f(x)| < |f_n(x)| + \varepsilon \leq \|f_n(x)\|_{\infty}  +
\varepsilon
$$
On en d\'eduit que $\|f\|_{\infty} \leq \|f_n(x)\|_{\infty}  +
\varepsilon < +\infty$. Ainsi $L^{\infty}(\mu)$ est complet.}
\end{enumerate}
}
