\uuid{2528}
\auteur{queffelec}
\datecreate{2009-04-01}

\contenu{
\texte{

}
\begin{enumerate}
    \item \question{Montrer que l'application $\varphi: (r,\theta)\to
(x,y)=(r\cos\theta, r\sin\theta)$ est un $C^1$-diff\'eomorphisme
de l'ouvert $]0,\infty[\times]-\pi,\pi[$ sur le plan priv\'e de la
demi-droite $\Rr^-$. Si $f(x,y)=g(r,\theta)$ donner les formules
de passage entre les d\'eriv\'ees partielles de $f$ et celles de
$g$.}
    \item \question{Soit $U$ le plan priv\'e de l'origine, et $f(x,y)=(x^2-y^2,
2xy)$.

Montrer que $f$ est un diff\'eomorphisme local au voisinage de
tout point de $U$ mais n'est pas un diff\'eomorphisme global.}
    \item \question{Soit $g$ l'application de ${\Rr}^2$ dans ${\Rr}^2$ d\'efinie
par $g(x,y)= (x+y,xy)$. Trouver un ouvert connexe maximal
$U\subset\Rr^2$ tel que $g$ soit un diff\'eomorphisme de $U$ sur
$g(U)$.}
    \item \question{Soit $h$ l'application de ${\Rr}^2$ dans ${\Rr}^2$ d\'efinie
par $(x,y)\to (e^x\cos y,e^x\sin y)$.

 Montrer que $h$ est de classe $C^1$ dans ${\Rr}^2$;
que $h'(x,y)$ est un \'el\'ement de Isom(${\Rr}^2,{\Rr}^2$) pour
tout $(x,y)$ de ${\Rr}^2$; mais que $h$ n'est pas un
hom\'eomorphisme de ${\Rr}^2$ sur $h({\Rr}^2)$.}
\reponse{
L'application $\varphi(r,\theta)=(r\cos \theta, r \sin
\theta)$ est de classe $C^\infty$ car ses coordonn\'ees le sont.
Pour montrer que c'est un diff\'eormorphisme global, il suffit de
montrer que c'est un diff\'eo local (th\'eor\`eme de l'inverse
local) et qu'elle est bijective. Calculons la matric jacobienne de
$\varphi$:
$$D\varphi(r,\theta)= \left (
\begin{array}{cc}
\cos \theta & -r\sin \theta \\
\sin \theta & r \cos \theta
\end{array} \right )  $$
}
\end{enumerate}
}
