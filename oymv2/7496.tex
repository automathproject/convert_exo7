\uuid{cboZ}
\exo7id{7496}
\auteur{mourougane}
\datecreate{2021-08-10}
\isIndication{false}
\isCorrection{false}
\chapitre{Géométrie affine euclidienne}
\sousChapitre{Géométrie affine euclidienne du plan}

\contenu{
\texte{
Soit $E$ un plan affine euclidien. Soit $\mathcal{R}=(O,i,j)$ un repère 
cartésien orthonormé de $E$.
}
\begin{enumerate}
    \item \question{Déterminer l'expression analytique de la translation $t$ de vecteur $\vec{u}=3i+j$.}
    \item \question{Déterminer l'expression analytique de la symétrie orthogonale $s$ d'axe la droite $d$ d'équation $(x+y=1)$.}
    \item \question{Déterminer l'expression analytique de la composée $f=t\circ s$.}
    \item \question{Démontrer que $f$ est une isométrie. Préciser le déterminant de sa partie linéaire. Que peut-on en déduire ?}
    \item \question{Déterminer l'ensemble des points fixes de $f$.
    Déterminer son axe (on pourra montrer que pour tout point $M$ du plan le milieu du segment $[M,f(M)]$ est sur une droite dont on précisera l'équation).}
    \item \question{Déterminer la composante de translation de $f$.}
\end{enumerate}
}
