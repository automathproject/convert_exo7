\uuid{T3DW}
\exo7id{2645}
\auteur{debievre}
\datecreate{2009-05-19}
\isIndication{true}
\isCorrection{true}
\chapitre{Fonction de plusieurs variables}
\sousChapitre{Extremums locaux}

\contenu{
\texte{
Soit $\cal C$ la courbe plane 
d'\'equation $f(x,y)=ye^{x}+e^y\sin(2x)=0$.
}
\begin{enumerate}
    \item \question{Appliquer le th\'eor\`eme des fonctions implicites
\`a la courbe ${\cal C}$ au point $(0,0)$.}
\reponse{Puisque 
$\frac{\partial f}{\partial x}= ye^{x}+2e^y\cos(2x)$ et 
$\frac{\partial f}{\partial y}=e^{x}+e^y\sin(2x)$ il s'ensuit que
$\frac{\partial f}{\partial x}(0,0)=2$ et 
$\frac{\partial f}{\partial y}(0,0)=1$.
Par cons\'equent,
il existe une fonction 
$h$ de la variable $x$ d\'efinie au voisinage de $0$ telle que
$h(0)=0$ et telle que,
pour qu'au voisinage de $(0,0)$
les coordonn\'ees $x$ et $y$ du point $(x,y)$
satisfassent \`a l'\'equation $ye^{x}+e^y\sin(2x)=0$
il faut et il suffit que
$y=h(x)$; 
de m\^eme il existe une fonction $k$ 
de la variable $y$ d\'efinie au voisinage de $0$  telle que
$h(0)=0$ et telle que,
pour qu'au voisinage de $(0,0)$
les coordonn\'ees $x$ et $y$ du point $(x,y)$
satisfassent \`a  l'\'equation $ye^{x}+e^y\sin(2x)=0$
il faut et il suffit que $x=k(y)$. 
En plus,
\[
h'(0)=-\frac{\frac{\partial f}{\partial x}(0,0)}
{\frac{\partial f}{\partial y}(0,0)}= -2,
\ 
k'(0)= -\frac{\frac{\partial f}{\partial y}(0,0)}
{\frac{\partial f}{\partial x}(0,0)} =-\tfrac 12.
\]}
    \item \question{D\'eterminer la limite de $y/x$ quand $(x,y)$ tend  le long la courbe ${\cal C}$
vers
$(0,0)$.}
\reponse{Puisque le point $(0,0)$ appartient \`a la courbe $\cal C$,
en $0$, les fonctions $h$ et $k$ prennent les valeurs
$h(0)=0$ et $k(0)=0$. Par cons\'equent,
\[
\mathrm{lim}_{\begin{smallmatrix}
(x,y)\to(0,0), (x,y)\ne 0\\
ye^{x}+e^y\sin(2x)=0
\end{smallmatrix}}y/x =h'(0) = -2.
\]}
\indication{Rappel du th\'eor\`eme des fonctions implicites
pour une fonction $f$ de classe $C^1$
de deux variables d\'efinie dans un ouvert du plan: {\em Soit $(x_0,y_0)$ un point tel que
$\frac{\partial f}{\partial y}(x_0,y_0)~\ne~0$.
Au voisinage de $x_0$,
il existe une fonction $h$ de classe $C^1$ de la variable $x$
d\'efinie dans un intervalle ouvert 
appropri\'e telle que $h(x_0)=y_0$ et telle que,
pour qu'au voisinage de $(x_0,y_0)$
les coordonn\'ees $x$ et $y$ du point $(x,y)$
satisfassent \`a l'\'equation
$f(x,y)=0$
il faut et il suffit que $y=h(x)$ et, s'il en est ainsi,}
\[
h'(x_0)=-\frac{\frac{\partial f}{\partial x}(x_0,y_0)}
{\frac {\partial f}{\partial y}(x_0,y_0)}.
\]
{\em D\`es que l'intervalle de d\'efinition de la fonction $h$ est fix\'e la fonction
$h$ est unique.\/}}
\end{enumerate}
}
