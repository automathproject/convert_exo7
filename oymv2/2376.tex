\uuid{zF1d}
\exo7id{2376}
\auteur{mayer}
\datecreate{2003-10-01}
\isIndication{true}
\isCorrection{true}
\chapitre{Compacité}
\sousChapitre{Compacité}

\contenu{
\texte{
Soit $E$ un espace norm\'e. Si $A$ et $B$ sont deux parties de $E$, on note
$A+B$ l'ensemble $\{a+b \; ; \; a\in A \; \text{et} \; b\in B\}$.
}
\begin{enumerate}
    \item \question{Montrer que si $A$ est compact et $B$ est ferm\'e, alors $A+B$ est ferm\'e.}
\reponse{Pour montrer que $A+B$ est fermé, nous allons montrer que toute suite de $A+B$ qui converge, converge vers un élément de $A+B$. Soit $(x_n)$ un suite
de $A+B$ qui converge vers $x\in E$. Alors il existe $a_n\in A$ et $b_n\in B$ tel que $x_n = a_n+b_n$.
Comme $A$ est compact on peut extraire une sous-suite $(a_{\phi(n)})$ qui converge vers $a\in A$. Alors $b_{\phi(n)} = x_{\phi(n)}-a_{\phi(n)}$ est convergente vers $x-a$. Notons $b= x-a$ comme $B$ est fermé alors $b\in B$.
Maintenant $x = a+b$ donc $x\in A+B$.}
    \item \question{Donner un exemple de deux ferm\'es de $\Rr^2$ dont la somme n'est pas ferm\'e.}
\reponse{Soit $F = \{ (x,y)\in \Rr^2 \mid xy\ge 1 \text{ et } x \ge 0\}$ , soit
$G = \{ (x,y)\in \Rr^2 \mid y \le 0 \text{ et } x\ge0\}$. Alors
$F+G = \{ (x,y)\in \Rr^2 \mid x > 0\}$ qui n'est pas un fermé.}
\indication{On pourra utiliser la caractérisation de la fermeture par des suites.}
\end{enumerate}
}
