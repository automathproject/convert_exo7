\uuid{I1uv}
\exo7id{620}
\auteur{cousquer}
\datecreate{2003-10-01}
\isIndication{false}
\isCorrection{false}
\chapitre{Continuité, limite et étude de fonctions réelles}
\sousChapitre{Limite de fonctions}

\contenu{
\texte{

}
\begin{enumerate}
    \item \question{Rappeler que pour tout nombre réels $\epsilon > 0$ il existe un entier $n$ tel que: 
\begin{eqnarray*}
      \frac 1{2n\pi} &<& \epsilon\\
\frac 1{(2n+1)\pi} &<& \epsilon.\\
    \end{eqnarray*}}
    \item \question{Montrer que pour tout nombre réel $l,$ et pour tout $\epsilon > 0,$ il existe $x \in ]-\epsilon,\epsilon[$ tel que:
$$
|\sin \frac 1x -l | > \frac 12.
$$}
    \item \question{En déduire que la fonction $x \mapsto \sin \frac 1x$ n'a pas de limite lorsque $x$ tend vers $0.$}
    \item \question{Montrer que la fonction définie par $f(x) =x \sin(\frac 1x)$ pour $x \not= 0$ et $f(0) =0$ est continue sur $\mathbb{R}.$}
\end{enumerate}
}
