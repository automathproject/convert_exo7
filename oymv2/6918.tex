\uuid{6918}
\auteur{ruette}
\datecreate{2013-01-24}

\contenu{
\texte{
Pour une élection, une population de $N$ individus a eu à choisir
entre voter pour le candidat $A$ ou le candidat $B$. On note $m$ le
nombre de personnes ayant voté pour $A$.  On interroge au
hasard $k$ individus différents dans cette population ($1\leq k\leq N$).
}
\begin{enumerate}
    \item \question{On désigne par $a_1,\ldots, a_m$ les $m$ personnes qui ont
voté pour $A$. Pour $i\in\{1,\ldots,m\}$, on note $X_i$ l'indicatrice de l'événement ``la
personne $a_i$ est interrogée''. Quelle est la loi de $X_i$~?}
    \item \question{Déterminer l'espérance ainsi que la matrice de covariance 
de $(X_{1},\ldots,X_{m})$.}
    \item \question{On pose $S=\sum_{i=1}^{m}X_i$. Quelle est la loi de $S$~?}
    \item \question{Déterminer le nombre moyen de personnes votant pour $A$ sur 
$k$ personnes tirées au hasard.}
    \item \question{Déterminer la variance de $S$.}
\end{enumerate}
}
