\uuid{5Tfv}
\exo7id{3744}
\auteur{quercia}
\datecreate{2010-03-11}
\isIndication{false}
\isCorrection{false}
\chapitre{Espace euclidien, espace normé}
\sousChapitre{Projection, symétrie}

\contenu{
\texte{
Soit $E$ un espace euclidien de dimension $n$.
Soit $f \in {\cal O}(E)$.
}
\begin{enumerate}
    \item \question{On suppose $n$ impair et $f \in {\cal O}^+(E)$.
    Montrer que 1 est valeur propre de $f$.
    (comparer $\det(f-\mathrm{id})$ et $\det(f^{-1}-\mathrm{id})$)}
    \item \question{Que peut-on dire quand $n$ est pair ?}
    \item \question{Soit $n$ quelconque, $f \in {\cal O}^{-}(E)$.
    Montrer que $-1$ est valeur propre de $f$.}
\end{enumerate}
}
