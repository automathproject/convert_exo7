\uuid{3066}
\auteur{quercia}
\datecreate{2010-03-08}
\isIndication{false}
\isCorrection{true}
\chapitre{Propriétés de R}
\sousChapitre{Les rationnels}

\contenu{
\texte{
Soit $a \in \Q^+$ tel que $\sqrt a \notin \Q$.

Montrer qu'il existe $C>0$ tel que pour tout rationnel $r=\frac pq$, on a :
$\bigl|r-\sqrt a\bigr| \ge \frac C{q^2}$.
}
\reponse{
Supposons d'abord $\bigl|r-\sqrt a\bigr| \le \frac 1{q^2}$.
Cela implique $|r| \le \sqrt a + 1$.

Majorons $|r^2-a|$ :
$$|r^2-a|=|r-\sqrt a| \times |r+\sqrt a| \le |r-\sqrt a| \times \big( |r|+\sqrt a \big) \le |r-\sqrt a| \times \big( 2\sqrt a+1 \big)$$

Minorons $|r^2-a|$, en posant $r=\frac pq$, $a=\frac mn$.
$$|r^2-a|=\left|\left(\frac pq\right)^2-\frac mn \right| =\left| \frac{np^2-mq^2}{nq^2}\right| \ge \frac{1}{nq^2}$$
La dernière inégalité provient que le numérateur $np^2-mq^2$ n'est pas nul (sinon $\sqrt a$ serait rationnel).

On déduit de ces deux majorations : 
$$\frac{1}{nq^2} \le |r^2-a| \le |r-\sqrt a| \times \big( 2\sqrt a+1 \big)$$
Et donc :
$$ |r-\sqrt a| \ge  \frac{1}{n( 2\sqrt a+1)}\frac{1}{q^2}.$$


Cette inégalité est aussi clairement vérifier si $\bigl|r-\sqrt a\bigr| > \frac 1{q^2}$.

La constante $C=\frac{1}{n( 2\sqrt a+1)}$ convient.
}
}
