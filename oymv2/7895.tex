\uuid{7895}
\auteur{mourougane}
\datecreate{2021-08-11}
\isIndication{false}
\isCorrection{false}
\chapitre{Groupe orthogonal et quaternions}
\sousChapitre{Groupe orthogonal et quaternions}

\contenu{
\texte{

}
\begin{enumerate}
    \item \question{Soit $P=\begin{bmatrix}a&b\\c&d\end{bmatrix}\in SU(2)$. Montrer que \( c=-\overline{b}, d=\overline{a} \quad \text{ et } \quad \overline{a}a+\overline{b}b=1\)
et écrire la forme générale d'une matrice $P$ de $SU(2)$ à l'aide de deux nombres complexes, puis de quatre nombres réels.}
    \item \question{En déduire un homéomorphisme de $SU(2)$ sur la sphère unité $S^3$ de $\Cc^2$. 
(On munit ici $S^3$ de la topologie induite par celle de $\Cc^2$ et $SU(2)$ de la topologie induite par la topologie d'une norme sur l'espace vectoriel $M(2,\Cc)$.)}
    \item \question{Soit $-1<c<1$. Décrire topologiquement le sous-espace de $SU(2)$ des matrices de trace $c$, appelé ``latitude $c$''.}
    \item \question{Montrer que les latitudes sont des classes de conjugaison dans $SU(2)$. 
(On pourra remarquer que les éléments de $SU(2)$ sont associés à des endomorphismes normaux (i.e. qui commutent avec leur adjoint)).}
    \item \question{Quelles sont les autres classes de conjugaison ?}
    \item \question{Décrire topologiquement le sous-groupe $D$ des matrices diagonales de $SU(2)$.}
\end{enumerate}
}
