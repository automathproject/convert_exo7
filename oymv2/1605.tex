\uuid{1605}
\auteur{barraud}
\datecreate{2003-09-01}
\isIndication{false}
\isCorrection{false}
\chapitre{Réduction d'endomorphisme, polynôme annulateur}
\sousChapitre{Valeur propre, vecteur propre}

\contenu{
\texte{
On considère les matrices suivantes :
$$
  A=\begin{pmatrix}
   1 &  0 &  0 & -1 \\
  -1 & -1 &  0 &  1 \\
  -2 &  0 &  0 &  2 \\
   0 & -1 & 0 & 0
    \end{pmatrix}
\qquad
  B=\begin{pmatrix}
       1 &  0 &  0 & -1 \\
      -1 & -1 &  0 &  1 \\
      -1 & -1 &  1 &  3 \\
      -1 &  0 & -1 & -1
    \end{pmatrix}
\qquad
  C=\begin{pmatrix}
       0 & -1 &  0 &  0 \\
       0 &  0 &  0 &  0 \\
       1 &  1 &  1 &  1 \\
      -1 &  0 & -1 & -1
    \end{pmatrix}
$$
En effectuant le moins de calculs possible,
}
\begin{enumerate}
    \item \question{montrer que \hspace{1cm}
 $\{0\}\subset\mathrm{Ker} A\subset\mathrm{Ker} A^{2}\subset\mathrm{Ker} A^{3}=\R^{4}$\\
et déterminer les dimensions respectives de $\mathrm{Ker} A$ et $\mathrm{Ker}
A^{2}$,}
    \item \question{déterminer un vecteur $e_{1}$ tel que
 $\R^{4}=\mathrm{Ker} A^{2}\oplus\mathrm{Vect}(e_{1})$,}
    \item \question{montrer que $(e_{1},Ae_{1},A^{2}e_{1})$ est une famille libre,}
    \item \question{montrer que $Ae_{1}\in\mathrm{Ker} A^{2}$, et que
 $\mathrm{Ker} A^{2}=\mathrm{Ker} A\oplus\mathrm{Vect}(Ae_{1})$,}
    \item \question{montrer que $A^{2}e_{1}\in\mathrm{Ker} A$ et déterminer un vecteur $e_{2}$
tel que $\mathrm{Ker} A=\mathrm{Vect}(A^{2}e_{1})\oplus\mathrm{Vect}(e_{2})$,}
    \item \question{montrer que $(e_{1},Ae_{1},A^{2}e_{1},e_{2})$ est une base de
$\R^{4}$.}
    \item \question{Soit $P$ la matrice de passage de la base canonique à la base
$(A^{2}e_{1},Ae_{1},e_{1},e_{2})$. Caluler $P^{-1}AP$.}
\end{enumerate}
}
