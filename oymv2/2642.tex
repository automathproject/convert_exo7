\uuid{2642}
\auteur{debievre}
\datecreate{2009-05-19}

\contenu{
\texte{
Trouver les points critiques de la  fonction $f$  suivante et
d\'eterminer si ce sont des minima locaux, des maxima locaux ou
des points selle.
\[
f(x,y)=\sin x+y^2-2y+1
\]
}
\indication{Voir l'exercice pr\'ec\'edent.}
\reponse{
Puisque
$df=\cos x \,\mathrm dx + (2y-2)\mathrm dy$, les points critiques sont les points
$((k+1/2)\pi,1)$ ($k\in \mathbb Z)$.
En plus,
$\mathrm{Hess}_f=\left[\begin{matrix} - \sin x & 0\\ 0 & 2 \end{matrix}\right]$ 
et 
\[
-\sin ((k+1/2)\pi)=(-1)^{k+1}
\]
d'o\`u 
$\mathrm{Hess}_f((k+1/2)\pi,1)
=\left[\begin{matrix} (-1)^{k+1} & 0\\ 0 & 2 \end{matrix}\right]$.
Par cons\'equent,  si
$k$ est impaire,
le point $((k+1/2)\pi,1)$ pr\'esente un minimum local et,
 si $k$ est paire, le point $((k+1/2)\pi,1)$ pr\'esente un point selle.
}
}
