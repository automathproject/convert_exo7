\uuid{gPDK}
\exo7id{2585}
\auteur{delaunay}
\datecreate{2009-05-19}
\isIndication{false}
\isCorrection{true}
\chapitre{Matrice}
\sousChapitre{Matrice et application linéaire}

\contenu{
\texte{
Soient $A$ et $B$ des matrices non nulles de $M_n(\R)$. On suppose que $A.B=0$.
}
\begin{enumerate}
    \item \question{D\'emontrer que $\Im B\subset \ker A$.}
\reponse{D\'emontrons que $\Im B\subset \ker A$.

Soit $y\in \Im B$, il existe $x\in\R^n$ tel que $y=Bx$, d'o\`u $Ay=ABx=0$, ainsi $y\in\ker A$ ce qui prouve l'inclusion.}
    \item \question{On suppose que le rang de $A$ est \'egal \`a $n-1$, d\'eterminer le rang de $B$.}
\reponse{On suppose que le rang de $A$ est \'egal \`a $n-1$, d\'eterminons le rang de $B$.

On a $\mathrm{rg} B=\dim \Im B$ et on sait que $\dim\Im A+\dim\ker A=n$ par cons\'equent, si $\mathrm{rg} A=n-1$ on a $\dim\ker A=1$ et l'inclusion $\Im B\subset \ker A$ implique $\dim\Im B\leq 1$ or, $B$ est suppos\'ee non nulle d'o\`u $\dim\Im B=1=\mathrm{rg} B$.}
\end{enumerate}
}
