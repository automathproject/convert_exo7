\uuid{PHZP}
\exo7id{7689}
\titre{Exercice 7689}
\theme{Exercices de Christophe Mourougane, 352.00 - Géométrie différentielle (Examen)}
\auteur{mourougane}
\date{2021/08/11}
\organisation{exo7}
\contenu{
  \texte{Soit $C$ la courbe tracée sur la surface d'équation $3z=xy+x^3$
et dont la projection orthogonale sur le plan d'équation $z=0$ 
est la courbe paramétrée $C'$ définie par $x=t$, $y=t^2$ pour $t$ parcourant $[0,1]$.}
\begin{enumerate}
  \item \question{Donner une expression intégrale pour la longueur de $C'$ 
puis celle de $C$, puis les comparer.}
  \item \question{Calculer la longueur de $C'$ puis celle de $C$.}
\end{enumerate}
\begin{enumerate}
  \item \reponse{Le vecteur vitesse de la courbe $C'$ est $(1,2t,0)$ de norme $\sqrt{1+4t^2}$.
 La longueur de la courbe $C'$ est donc $l[C']=\int_0^1\sqrt{1+4t^2}dt$.
 La courbe $C$ est paramétrée par $x=t$, $y=t^2$ $z=2/3t^3$ pour $t$ parcourant $[0,1]$.
 Le vecteur vitesse de la courbe $C$ est $(1,2t,2t^2)$ de norme $\sqrt{1+4t^2+4t^4}=1+2t^2$.
 La longueur de la courbe $C$ est donc $l[C]=\int_0^1 1+2t^2dt$.
 Comme pour tout $t\in [0,1], 0\leq 1+4t^2\leq (1+2t^2)^2$, pour tout $t\in [0,1], \sqrt{1+4t^2}\leq (1+2t^2)$
 et par intégration $l[C']\leq l[C]$.}
  \item \reponse{En utilisant le changement de variables $2t=\sinh (u)$, on obtient le calcul de primitive
 \begin{eqnarray*}
  4\int \sqrt{1 + 4t^2} dt &=& 2\int \cosh^2 udu =\int(\cosh (2u) + 1) du \\
  &=& \sinh (2u)/2+ u  \\
  &=& \sinh u \cosh u + u\\
  &=&2t\sqrt{1 + 4t^2}+\ln(2t+\sqrt{1 + 4t^2}).
 \end{eqnarray*}
 Par conséquent, la longueur de $C'$ est $1/4[2t\sqrt{1 + 4t^2}+\ln(2t+\sqrt{1 + 4t^2})]_0^1
 =1/4(2\sqrt{5}+\ln(2+\sqrt{5}))$.
 La longueur de $C$ est $1+2/3$.}
\end{enumerate}
}