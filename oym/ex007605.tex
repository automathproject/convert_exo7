\uuid{7605}
\titre{Racine de polynômes}
\theme{}
\auteur{mourougane}
\date{2021/08/10}
\organisation{exo7}
\contenu{
  \texte{}
\begin{enumerate}
  \item \question{\'Enoncer le théorème de Rouché.}
  \item \question{Montrer que les zéros du polynôme $ p(z) = z^{4}-7z-1 $ 
sont tous inclus dans le disque $\Delta_2(0)$ centré en l'origine de rayon 2.
On vérifiera soigneusement toutes les hypothèses du théorème utilisé.}
\end{enumerate}
\begin{enumerate}
  \item \reponse{Soit $f$ et $g$ deux applications holomorphes sur un ouvert étoilé $D$ de $\Cc$.
Soit $\Gamma$ un chemin fermé simple dans $D$. 
On suppose que $$\forall z\in \Gamma,\ \ |f(z)-g(z)|<|g(z)|.$$
Alors $f$ et $g$ ont le même nombre de zéros compté avec multiplicité dans l'intérieur de $\Gamma$.}
  \item \reponse{L'ouvert $\Cc$ est étoilé et contient le chemin $\partial\Delta_2$ dont l'intérieur est $\Delta_2$.
Les deux applications $z\mapsto z^4$ et $p$ sont polynômiales donc holomorphes sur $\Cc$.
Soit $z\in\partial\Delta_2$. Alors 
$$|p(z)-z^4|=|7z+1|\leq 7|z|+1\leq 7\times 2+1=15<2^4=|z^4|.$$
Par le théorème de Rouché, le polynôme $p$ et $z\mapsto z^4$ ont le même nombre de zéros compté avec multiplicités
dans $\Delta_2$, c'est à dire $4$.
Comme $p$ est non nul de degré $4$, tous ses zéros sont dans $\Delta_2$.}
\end{enumerate}
}