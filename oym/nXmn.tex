\uuid{nXmn}
\exo7id{3815}
\titre{Matrice orthogonale ?}
\theme{Exercices de Michel Quercia, Problèmes matriciels}
\auteur{quercia}
\date{2010/03/11}
\organisation{exo7}
\contenu{
  \texte{}
\begin{enumerate}
  \item \question{Peut-on définir sur $\R^2$ une structure euclidienne telle que l'endomorphisme
$f$ dont la matrice dans la base canonique est
$M = \begin{pmatrix}1&1\cr-3&-2\cr\end{pmatrix}$ soit une rotation~?}
  \item \question{Généraliser à une matrice $M\in\mathcal{M}_2(\R)$ quelconque.}
  \item \question{Généraliser à une matrice $M\in\mathcal{M}_n(\R)$ quelconque.}
\end{enumerate}
\begin{enumerate}
  \item \reponse{$\mathrm{spec}(M) = \{j,j^2\}  \Rightarrow $ on prend comme base orthonormale
$a = \begin{pmatrix}1\cr0\cr\end{pmatrix}$ et $b = \frac2{\sqrt3}(f(a)+\frac12a) =
\begin{pmatrix}\sqrt3\cr-2\sqrt3\cr\end{pmatrix}$.}
  \item \reponse{$M$ est une matrice de rotation ssi $\mathrm{spec}(M)\subset \mathbb{U}\setminus\{\pm1\}$
ou $M=\pm I$.}
  \item \reponse{$M$ est la matrice d'une application orthogonale ssi $\mathrm{spec}(M)\subset \mathbb{U}$
et $M$ est $\C$-diagonalisable (alors $M$ est $\R$-semblable à une
matrice diagonale par blocs dont les blocs sont des matrices de rotation).}
\end{enumerate}
}