\uuid{4652}
\titre{Convergence uniforme}
\theme{}
\auteur{quercia}
\date{2010/03/14}
\organisation{exo7}
\contenu{
  \texte{}
  \question{Soit $(a_n)$ une suite décroissante de limite nulle.
Montrer que la série $\sum_{n=1}^\infty a_n\sin nx$ converge uniformément sur
$\R$ si et seulement si $na_n \to 0$ lorsque $n\to\infty$.

Pour le sens direct : utiliser le critère de convergence uniforme de Cauchy
et l'inégalité : $\sin x \ge \frac{2x}\pi$ sur $[0,\pi/2]$.}
  \reponse{S'il y a convergence uniforme :
$\|a_n\sin nx + \dots + a_p\sin px\|_\infty \to 0$ lorsque $n,p\to\infty$.

On prend $x = \frac\pi{2p}$ : $0 \le \frac{a_p}p(n+\dots+p)
                                  \le \frac1p(na_n + \dots + pa_p)
                               \to 0$ lorsque $n,p\to\infty$.
$n = [p/2]  \Rightarrow $ cqfd.

Si $na_n \to 0$ : Soit $x \in {]0,\pi]}$ et $n$ tel que
$\frac 1n \le x\le \frac1{n-1}$.

Transformation d'Abel :
$|a_n\sin nx + \dots + a_p\sin px| \le \frac{2a_n}{\sin x/2} \le \frac{2na_n}\pi$,

et $|a_k\sin kx + \dots + a_{n-1}\sin (n-1)x| \le (ka_k + \dots + (n-1)a_{n-1})x
\le \frac{n-k}{n-1}a_k\le 2a_k$.}
}