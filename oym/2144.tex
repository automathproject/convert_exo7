\uuid{2144}
\titre{Exercice 2144}
\theme{Morphisme, sous-groupe distingué, quotient}
\auteur{debes}
\date{2008/02/12}
\organisation{exo7}
\contenu{
  \texte{}
  \question{(a) Montrer que si $m$ et $n$ sont des entiers premiers entre eux et qu'un
\'el\'ement $z$ d'un groupe $G$ v\'erifie $z^m=z^n=e$ o\`u $e$ d\'esigne l'\'el\'ement
neutre de $G$, alors
$z=e$.
\smallskip

(b) Montrer que si $m$ et $n$ sont deux entiers premiers entre eux, l'application
$$\phi: \mu_m \times \mu_n \rightarrow \mu_{mn} $$ qui au couple $(s,t)$ fait correspondre le
produit
$st$ est un isomorphisme de groupes}
  \reponse{}
}