\uuid{00hh}
\exo7id{3706}
\titre{Applications antisymétriques}
\theme{Exercices de Michel Quercia, Espace vectoriel euclidien orienté de dimension 3}
\auteur{quercia}
\date{2010/03/11}
\organisation{exo7}
\contenu{
  \texte{Soit $E$ un espace vectoriel euclidien et $f \in \mathcal{L}(E)$ antisymétrique.}
\begin{enumerate}
  \item \question{Montrer que $\mathrm{id}_E + f \in GL(E)$.}
  \item \question{Montrer que $g = (\mathrm{id} - f) \circ (\mathrm{id} + f)^{-1} \in {\cal O}^+(E)$ et
    $\mathrm{id}+g$ est inversible.}
  \item \question{Réciproquement, soit $h \in {\cal O}^+(E)$ tq $\mathrm{id} + h$ soit inversible.
    Montrer qu'il existe $f$ antisymétrique tel que
    $h = (\mathrm{id} - f) \circ (\mathrm{id} + f)^{-1}$.}
\end{enumerate}
\begin{enumerate}
  \item \reponse{matrice dans une {\it bond}.}
  \item \reponse{$f = 2(h+\mathrm{id})^{-1} - \mathrm{id}$.}
\end{enumerate}
}