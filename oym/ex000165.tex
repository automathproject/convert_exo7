\uuid{165}
\titre{Exercice 165}
\theme{}
\auteur{cousquer}
\date{2003/10/01}
\organisation{exo7}
\contenu{
  \texte{}
  \question{Soit $b\geq 2$ un entier fixé. Démontrer que pour tout 
$N\in \mathbb{N}^{\ast}$, il existe un entier $n\in \mathbb{N}$ 
et des entiers $a_0,a_1,\ldots, a_n$ appartenant à 
$\{\,0,1,\ldots,b-1\,\}$ tels que~;
$$N=a_0+a_1b+\cdots+a_nb^n\quad\mbox{et}\quad a_n\neq 0$$

 \noindent Démontrer que pour chaque $N$, le système $(n,a_0,a_1,\ldots, a_n)$ est déterminé par la propriété ci-dessus.

 \noindent On dit que $a_0,a_1,\ldots,a_n$ sont les chiffres de l'écriture du nombre $N$ suivant la base~$b$.}
  \reponse{}
}