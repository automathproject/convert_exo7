\uuid{4UoM}
\exo7id{7337}
\titre{Exercice 7337}
\theme{Exercices de Christophe Mourougane, Arithmétique 2}
\auteur{mourougane}
\date{2021/08/10}
\organisation{exo7}
\contenu{
  \texte{Les polynômes symétriques élémentaires en $n$ indéterminées sont par définition
\begin{eqnarray*}
 s_1(t_1,t_2,\cdots,t_n)&:=&t_1+t_2+\cdots+t_n\\
s_2(t_1,t_2,\cdots,t_n)&:=&t_1t_2+t_1t_3+\cdots+t_2t_3+t_2t_4
+\cdots +\cdots t_{n-1}t_n\\
\vdots\\
s_n(t_1,t_2,\cdots,t_n)&:=&t_1t_2\cdots t_n.
\end{eqnarray*}}
\begin{enumerate}
  \item \question{Soit $F(X)=a(X-t_1)(X-t_2)\cdots(X-t_n)$. Développer $F(X)$ en puissances de $X$.}
  \item \question{Ecrire $t_1^3+t_2^3$ comme polynôme à coefficients entiers de $s_1$, $s_2$.}
  \item \question{Ecrire $t_1^4+t_2^4$ comme polynôme à coefficients entiers de $s_1$, $s_2$.}
  \item \question{Ecrire $t_1^3+t_2^3+t_3^3$ comme polynôme à coefficients entiers de $s_1$, $s_2$, $s_3$.}
  \item \question{Ecrire $f$
$$f(t_1,t_2,t_3)=t_1^2t_2+t_1^2t_3+t_2^2t_1+t_2^2t_3+t_3^2t_1+t_3^2t_2.$$
 comme polynôme à coefficients entiers en les polynômes symétriques élémentaires.}
\end{enumerate}
\begin{enumerate}

\end{enumerate}
}