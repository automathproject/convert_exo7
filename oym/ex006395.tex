\uuid{6395}
\titre{Exercice 6395}
\theme{}
\auteur{potyag}
\date{2011/10/16}
\organisation{exo7}
\contenu{
  \texte{Soit $f$ une  application  qui préserve les rapports de longueur
: $ \forall x,y,z,t \in \Rr^2$,on a : $\frac{d(f(x),f(y))}{d(f(z), f(t))} = \frac{d(x,y)}{d(z,t)}$. (Par définition une
telle application est une {\it similitude})}
\begin{enumerate}
  \item \question{Monter $\exists k \in \Rr^{+*}$ tel que $d(f(x),f(y)) =
kd(x,y)$.}
  \item \question{Montrer qu'un similitude s'écrit comme composée d'une
homothétie et d'une isométrie.}
\end{enumerate}
\begin{enumerate}

\end{enumerate}
}