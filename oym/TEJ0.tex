\uuid{TEJ0}
\exo7id{3704}
\titre{Décomposition des rotations}
\theme{Exercices de Michel Quercia, Espace vectoriel euclidien orienté de dimension 3}
\auteur{quercia}
\date{2010/03/11}
\organisation{exo7}
\contenu{
  \texte{Soit $(\vec i,\vec j,\vec k)$ une {\it bond} d'un espace vectoriel euclidien orienté de
dimension 3\ $E$, et $f \in {\cal O}^+(E)$.}
\begin{enumerate}
  \item \question{On suppose $f(\vec j) \perp \vec i$.
    Montrer qu'il existe $r,r'$ rotations autour de
    $\vec j$ et $\vec i$ telles que $r' \circ r = f$.}
  \item \question{En déduire que tout $f \in {\cal O}^+(E)$ se décompose de deux manières
    sous la forme : $f = r'' \circ r' \circ r$ où $r,r''$ sont des rotations
    autour de $\vec j$ et $r'$ est une rotation autour de $\vec i$.}
  \item \question{Décomposer $(x,y,z) \mapsto (y,x,z)$ et
    $(x,y,z) \mapsto (x\cos\alpha-y\sin\alpha, x\sin\alpha+y\cos\alpha, z)$.}
\end{enumerate}
\begin{enumerate}

\end{enumerate}
}