\uuid{3598}
\titre{Ensi P 90}
\theme{Exercices de Michel Quercia, Réductions des endomorphismes}
\auteur{quercia}
\date{2010/03/10}
\organisation{exo7}
\contenu{
  \texte{}
  \question{Soit $M \in \mathcal{M}_n(\C)$ diagonalisable.
    Soit $A = \begin{pmatrix}M&M\cr M&M\cr\end{pmatrix} \in \mathcal{M}_{2n}(\C)$.
    La matrice $A$ est-elle diagonalisable ?}
  \reponse{S'inspirer du cas $n=1$. Soit $P = \begin{pmatrix}I&I\cr I&-I\cr\end{pmatrix}$ :
	     $P^{-1}AP = \begin{pmatrix}2M&0\cr 0&0\cr\end{pmatrix}$ est diagonalisable,
	     donc $A$ aussi.}
}