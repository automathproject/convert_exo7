\uuid{5216}
\titre{**I Inégalités de \textsc{Cauchy}-\textsc{Schwarz} et de \textsc{Minkowski}}
\theme{}
\auteur{rouget}
\date{2010/06/30}
\organisation{exo7}
\contenu{
  \texte{Soient $a_1$,..., $a_n$, $b_1$,..., $b_n$ des nombres réels.}
\begin{enumerate}
  \item \question{En considérant la fonction $f~:~x\mapsto\sum_{k=1}^{n}(a_k+xb_k)^2$, montrer que
$|\sum_{k=1}^{n}a_kb_k|\leq\sqrt{\sum_{k=1}^{n}a_k^2}\sqrt{\sum_{k=1}^{n}b_k^2}$ (inégalité de \textsc{Cauchy}-\textsc{Schwarz}).}
  \item \question{En déduire l'inégalité de \textsc{Minkowski}~:~$\sqrt{\sum_{k=1}^{n}(a_k+b_k)^2}\leq\sqrt{\sum_{k=1}^{n}a_k^2}+\sqrt{\sum_{k=1}^{n}b_k^2}$.

(l'inégalité de \textsc{Cauchy}-\textsc{Schwarz} affirme que le produit scalaire de deux vecteurs est inférieur ou égal au produit de leurs normes et l'inégalité de \textsc{Minkowski} est l'inégalité triangulaire).}
\end{enumerate}
\begin{enumerate}
  \item \reponse{Si les $b_k$ sont tous nuls, l'inégalité est claire.
Sinon, pour $x$ réel, posons

\begin{center}
$f(x)=\sum_{k=1}^{n}(a_k+xb_k)^2=\left(\sum_{k=1}^{n}b_k^2\right)x^2+2\left(\sum_{k=1}^{n}a_kb_k\right)x+\sum_{k=1}^{n}a_k^2$.
\end{center}
$f$ est un trinôme du second degré de signe constant sur $\Rr$. Son discriminant réduit est donc négatif ou nul ce qui fournit~:

$$0\geq\Delta'=\left(\sum_{k=1}^{n}a_kb_k\right)^2-\left(\sum_{k=1}^{n}a_k^2\right)\left(\sum_{k=1}^{n}b_k^2\right),$$ 
ou encore $\left|\sum_{k=1}^{n}a_kb_k\right|\leq\sqrt{\sum_{k=1}^{n}a_k^2}\sqrt{\sum_{k=1}^{n}b_k^2}$, qui est l'inégalité de \textsc{Cauchy}-\textsc{Schwarz}.}
  \item \reponse{\begin{align*}
\sum_{k=1}^{n}(a_k+b_k)^2&=\sum_{k=1}^{n}a_k^2+2\sum_{k=1}^{n}a_kb_k+\sum_{k=1}^{n}b_k^2
\leq\sum_{k=1}^{n}a_k^2+2\left|\sum_{k=1}^{n}a_kb_k\right|+\sum_{k=1}^{n}b_k^2\\
 &\leq\sum_{k=1}^{n}a_k^2+2\sqrt{\sum_{k=1}^{n}a_k^2}\sqrt{\sum_{k=1}^{n}b_k^2}+\sum_{k=1}^{n}b_k^2
 \quad(\mbox{\textsc{Cauchy}-\textsc{Schwarz}})\\
 &=\left(\sqrt{\sum_{k=1}^{n}a_k^2}+\sqrt{\sum_{k=1}^{n}b_k^2}\right)^2
\end{align*}
et donc, $\sqrt{\sum_{k=1}^{n}(a_k+b_k)^2}\leq\sqrt{\sum_{k=1}^{n}a_k^2}+\sqrt{\sum_{k=1}^{n}b_k^2}$, qui est l'inégalité de \textsc{Minkowski}.}
\end{enumerate}
}