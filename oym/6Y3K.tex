\uuid{6Y3K}
\exo7id{3845}
\titre{Fonction périodique}
\theme{Exercices de Michel Quercia, Fonctions continues}
\auteur{quercia}
\date{2010/03/11}
\organisation{exo7}
\contenu{
  \texte{Soit $f : \R \to \R$ et $T > 0$. On suppose que $f$ est $T$-périodique cad :
$\forall\ x \in \R,\ f(x+T) = f(x)$.}
\begin{enumerate}
  \item \question{Si $f$ possède une limite en $+\infty$, montrer que $f$ est constante.}
  \item \question{Si $f$ est continue non constante, montrer que $f$ a une plus petite période.}
  \item \question{Si $f$ est continue, montrer que $f$ est bornée et atteint ses bornes.}
\end{enumerate}
\begin{enumerate}

\end{enumerate}
}