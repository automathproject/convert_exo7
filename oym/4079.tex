\uuid{4079}
\titre{Matexo}
\theme{Exercices de Michel Quercia, \'Equations différentielles linéaires (I)}
\auteur{quercia}
\date{2010/03/11}
\organisation{exo7}
\contenu{
  \texte{}
  \question{Soit $k\in\R^*$ fixé. On considère :
$E=\{f\in\mathcal{C}^2([0,1],\R)\text{ tq } f(0) = 0 \text{ et } f(1)=3\}$.

Déterminer $\displaystyle \inf_{f\in E}  \int_{t=0}^1 (f'(t)+kf(t))^2\,d t$.
{\it Indication~: poser $f'+kf = g$ et calculer $f(1)$ en fonction de~$g$.}}
  \reponse{$f(1) = e^{-k} \int_{t=0}^1 g(t)e^{kt}\,d t$.

Avec Cauchy-Schwarz on obtient
 $ \int_{t=0}^1 (f'(t)+kf(t))^2\,d t \ge
            \frac{2k}{1-e^{-2k}}f(1)^2 = 9\frac{2k}{1-e^{-2k}}$.

Il y a égalité pour $f(t) = 3\frac{e^{kt}-e^{-kt}}{e^{k}-e^{-k}}$.}
}