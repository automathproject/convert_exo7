\uuid{0yuX}
\exo7id{4071}
\titre{$x^2f''(x) + xf'(x) = \lambda f(x)$}
\theme{Exercices de Michel Quercia, \'Equations différentielles linéaires (I)}
\auteur{quercia}
\date{2010/03/11}
\organisation{exo7}
\contenu{
  \texte{Déterminer les éléments propres des endomorphismes suivants :}
\begin{enumerate}
  \item \question{$E = \R[X]$
   $\Phi(P)(X) = X^2P''(X) + XP'(X)$.}
  \item \question{$E = \mathcal{C}^\infty(]0,+\infty[)$
   $\Phi(f)(x) = x^2f''(x) + xf'(x)$.}
  \item \question{$E = \mathcal{C}^\infty(]0,1[)$
  $\Phi(f)(x) = \sqrt{\frac {1-x}x}f'(x)$.}
\end{enumerate}
\begin{enumerate}
  \item \reponse{$\lambda = n^2$ : $P(X) = aX^n$.}
  \item \reponse{$\lambda > 0$ : $f(x) = \alpha x^{\sqrt\lambda} + \beta x^{-\sqrt\lambda}$.\par
            $\lambda = 0$ : $f(x) = \alpha + \beta\ln x$.\par
            $\lambda < 0$ : $f(x) = \alpha\cos\bigl(\sqrt{-\lambda}\ln x\bigr)
                                  + \beta \sin\bigl(\sqrt{-\lambda}\ln x\bigr)$.}
  \item \reponse{$\lambda \in \R$ : $f(x) = \alpha\exp\lambda\bigl( \arcsin\sqrt x  - \sqrt{x(1-x)} \bigr)$.}
\end{enumerate}
}