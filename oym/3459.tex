\uuid{3459}
\titre{$(x_i+y_j)^k$}
\theme{Exercices de Michel Quercia, Calculs de déterminants}
\auteur{quercia}
\date{2010/03/10}
\organisation{exo7}
\contenu{
  \texte{}
  \question{Soit $k \le n-1$ et $M = \Bigl( (x_i+y_j)^k \Bigr)$.
\'Ecrire $M$ comme produit de deux matrices et calculer $\det M$.}
  \reponse{$M = ( x_i^{j-1} ) \times ( C_k^{i-1} y_j^{k-i+1} )
  \Rightarrow  \\
  \det M = 
  \begin{cases}0 & \text{ si } k < n-1 \cr
                   
                     \varepsilon_n C_{n-1}^0 C_{n-1}^1 \dots C_{n-1}^{n-1}
		    V(x_1, \dots, x_n ) V(y_1, \dots, y_n )& \text{ si } k = n-1. \end{cases}$
		    
Avec la notation : $\varepsilon_n = \begin{cases}1 &\text{si}n \equiv 0 \text{ ou  }1 (\mathrm{mod}\, 4) \cr
                                   -1 &\text{sinon.}\end{cases}$}
}