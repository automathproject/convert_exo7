\uuid{P2EK}
\exo7id{4120}
\titre{\'Etude qualitative : $y'=x-e^y$}
\theme{Exercices de Michel Quercia, \'Equations différentielles non linéaires (II)}
\auteur{quercia}
\date{2010/03/11}
\organisation{exo7}
\contenu{
  \texte{Soit $y$ une solution maximale de l'équation $y'=x-e^y$.}
\begin{enumerate}
  \item \question{Montrer que $y$ est décroissante puis croissante.}
  \item \question{Montrer que $y$ est définie jusqu'en $+\infty$ et que sa courbe représentative
    admet une branche parabolique horizontale.}
  \item \question{Montrer que $\alpha > -\infty$ et que $y \to +\infty$ lorsque $x\to\alpha^-$.}
\end{enumerate}
\begin{enumerate}
  \item \reponse{Régionnement.}
  \item \reponse{idem.}
  \item \reponse{Pour $x<0$, $y'<-e^y  \Rightarrow  -y'e^{-y}>1  \Rightarrow  x > e^{-y} + C > C$.}
\end{enumerate}
}