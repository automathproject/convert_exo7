\exo7id{2091}
\titre{Exercice 2091}
\theme{}
\auteur{bodin}
\date{2008/02/04}
\organisation{exo7}
\contenu{
  \texte{Soit $f\,:\;\R\to\R$ une fonction continue sur $\R$ et $F(x)=\int_0^x
f(t)d t$. Répondre par vrai ou faux aux affirmations suivantes:}
\begin{enumerate}
  \item \question{$F$ est continue sur $\R$.}
  \item \question{$F$ est dérivable sur $\R$ de dérivée $f$.}
  \item \question{Si $f$ est croissante sur $\R$ alors $F$ est croissante sur $\R$.}
  \item \question{Si $f$ est positive sur $\R$ alors $F$ est positive sur $\R$.}
  \item \question{Si $f$ est positive sur $\R$ alors $F$ est croissante sur $\R$.}
  \item \question{Si $f$ est $T$-périodique sur $\R$ alors $F$ est $T$-périodique sur $\R$.}
  \item \question{Si $f$ est paire alors $F$ est impaire.}
\end{enumerate}
\begin{enumerate}
  \item \reponse{Vrai.}
  \item \reponse{Vrai.}
  \item \reponse{Faux ! Attention aux valeurs négatives par exemple pour $f(x)=x$ alors $F$ est décroissante sur $]-\infty,0]$ et croissante sur $[0,+\infty[$.}
  \item \reponse{Faux. Attention aux valeurs négatives par exemple pour $f(x)=x^2$ alors $F$ est négative sur $]-\infty,0]$ et positive sur $[0,+\infty[$.}
  \item \reponse{Vrai.}
  \item \reponse{Faux. Faire le calcul avec la fonction $f(x) = 1+\sin(x)$ par exemple.}
  \item \reponse{Vrai.}
\end{enumerate}
}