\uuid{289}
\titre{Exercice 289}
\theme{}
\auteur{cousquer}
\date{2003/10/01}
\organisation{exo7}
\contenu{
  \texte{}
\begin{enumerate}
  \item \question{Soit $A$ une partie non vide de $\mathbb{Z}$~; montrer que la famille 
des sous-groupes contenant~$A$ n'est pas vide. Soit $H$ une partie 
contenant~$A$. Montrer l'équivalence des conditions suivantes~:
\begin{enumerate}}
  \item \question{[i)]  $H$ est l'intersection des sous-groupes de $\mathbb{Z}$ 
    qui contiennent~$A$,}
  \item \question{[ii)]  $H$ est le plus petit sous-groupe de $\mathbb{Z}$ qui 
    contient~$A$,}
  \item \question{[iii)]  $H$ est l'ensemble des sommes finies d'éléments de $A$ 
    ou d'éléments dont l'opposé est dans~$A$.}
\end{enumerate}
\begin{enumerate}

\end{enumerate}
}