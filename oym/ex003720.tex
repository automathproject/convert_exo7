\uuid{3720}
\titre{$q(a)q(x)-f^2(a,x)$}
\theme{}
\auteur{quercia}
\date{2010/03/11}
\organisation{exo7}
\contenu{
  \texte{Soit $f$ une forme bilinéaire symétrique sur~$E$ et $q$ la forme quadratique
associée.\par On pose pour~$x\in\ E$~: $\varphi(x) = q(a)q(x)-f^2(a,x)$.}
\begin{enumerate}
  \item \question{Montrer que $\varphi$ est une forme quadratique sur~$E$.}
  \item \question{Si $E$ est de dimension finie comparer les rangs de $\varphi$ et~$q$.}
  \item \question{Dans le cas général, déterminer le noyau de la forme polaire de~$\varphi$ en
    fonction de celui de~$f$ et de~$a$.}
\end{enumerate}
\begin{enumerate}
  \item \reponse{Si $a\in\mathrm{Ker} f$, $\mathrm{Ker}\tilde\varphi = E$.\par
Si $a\notin\mathrm{Ker} f$ et $q(a) = 0$, $\mathrm{Ker}\tilde\varphi = a^\bot$.\par
Si $q(a)\ne 0$, $\mathrm{Ker}\tilde\varphi = \mathrm{Ker}(f) \oplus {<a>}$.}
\end{enumerate}
}