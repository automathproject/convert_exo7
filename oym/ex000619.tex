\exo7id{619}
\titre{Exercice 619}
\theme{}
\auteur{cousquer}
\date{2003/10/01}
\organisation{exo7}
\contenu{
  \texte{}
\begin{enumerate}
  \item \question{Pour tout $n$ entier naturel et tout couple de réels $(x,y)$, établir la formule :
$$
x^n-y^n = (x-y).\sum_{k=0}^{n-1}x^ky^{n-1-k}.
$$}
  \item \question{Déduire de la question précédente que pour tout entier $n$ tout réel strictement positif $a$ et tout couple de réels $(x,y)$ tel que $|x| \leq a$ et $|y| \leq a,$
$$
|x^n-y^n| \leq na^{n-1}|x-y|.
$$}
  \item \question{Déduire de ce qui précède que pour tout $x_0 \in \mathbb{R},$ et pour tout $\epsilon > 0,$ il existe $\alpha > 0$ tel que:
$$
|x-x_0| < \alpha \;\Rightarrow\; |x^n-x_0^n| < \epsilon.
$$
Conclure.}
  \item \question{Sur quel sous ensemble $D$ de $\mathbb{R},$ la fonction de la variable réelle $f$ donnée par 
$$
f(x) := \frac{1-x^n}{1-x}
$$
est-elle définie? Calculer les limites de $f$ aux bornes de $D.$}
\end{enumerate}
\begin{enumerate}

\end{enumerate}
}