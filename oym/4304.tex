\uuid{4304}
\titre{Mines-Ponts MP 2005}
\theme{Exercices de Michel Quercia, Intégrale généralisée}
\auteur{quercia}
\date{2010/03/12}
\organisation{exo7}
\contenu{
  \texte{}
  \question{Nature et calcul de $ \int_{x=0}^{+\infty} \exp\Bigl(-\bigl(x-\frac 1x\bigr)^2\Bigr)\,d x$~?}
  \reponse{Intégrale trivialement convergente. Couper en $\int_0^1$ et $\int_1^{+\infty}$,
changer $x$ en $\frac 1x$ dans l'une des intégrales, regrouper et poser $u=x-\frac1x$.
On obtient $I= \int_{u=0}^{+\infty}e^{-u^2}\,d u = \frac{\sqrt\pi}2$.}
}