\uuid{1612}
\titre{Exercice 1612}
\theme{}
\auteur{liousse}
\date{2003/10/01}
\organisation{exo7}
\contenu{
  \texte{Soient trois vecteurs $e_1,$ $e_2,$ 
$e_3$ formant une base de $\Rr^{3}.$ On note $T$ l'application 
lin\'eaire d\'efinie par $T(e_1)=T(e_3)=e_3$ et $T(e_2)=-e_1+e_2+e_3.$}
\begin{enumerate}
  \item \question{D\'eterminer le noyau de cette application lin\'eaire. Donner la matrice 
$A$ de $T$ dans la base donn\'ee.\\}
  \item \question{On pose $f_1=e_1-e_3,$ $f_2=e_1-e_2,$ $f_3=-e_1+e_2+e_3.$ Calculer 
$e_1,$ $e_2,$ $e_3$ en fonction de $f_1,$ $f_2,$ $f_3.$ Les vecteurs 
$f_1,$ $f_2,$ $f_3$ forment-ils une base de $\Rr^{3}$ ?\\}
  \item \question{Calculer $T(f_1),$ $T(f_2),$ $T(f_3)$ en fonction de $f_1,$ $f_2,$ 
$f_3.$ \'Ecrire la matrice $B$ de $T$ dans cette nouvelle base.\\}
  \item \question{On pose $P=\left (\begin{array}{rcl}1&1&-1\\0&-1&1\\-1&0&1
\end{array}\right ).$ V\'erifier que $P$ est inversible et calculer 
$P^{-1}.$ Quelle relation relie $A,$ $B,$ $P$ et $P^{-1}$ ?}
\end{enumerate}
\begin{enumerate}

\end{enumerate}
}