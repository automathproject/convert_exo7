\exo7id{1430}
\titre{Exercice 1430}
\theme{}
\auteur{legall}
\date{1998/09/01}
\organisation{exo7}
\contenu{
  \texte{Soit $  G  $ un groupe, $  H  $ et $  K  $ deux sous-groupes de $
G  .$ On note $  HK=\{ hk  ;   h\in H   ,   k\in K\}  .$}
\begin{enumerate}
  \item \question{Montrer que $  HK  $ est un sous-groupe de $  G  $ si et seulement si $  HK=KH  .$ En d\'eduire que si $  H  $ est distingu\'e dans $ G  $ alors $  HK  $ est un sous-groupe de $
G  .$}
  \item \question{On suppose d\'esormais que $  \forall  h\in H   ,   k\in K   :  hk=kh  .$
Montrer que l'application $  f : H\times K\rightarrow G   $ d\'efinie par $
\forall  h\in H   ,   k\in K   :  f(h,k)=hk  $ est un homomorphisme de groupes.}
  \item \question{Calculer le noyau et l'image de $  f  .$ Donner une condition n\'ec\'essaire et suffisante pour que $  f  $ soit un isomorphisme de groupes.}
\end{enumerate}
\begin{enumerate}

\end{enumerate}
}