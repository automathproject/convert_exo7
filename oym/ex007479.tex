\uuid{7479}
\titre{Groupe laissant stable une partie}
\theme{}
\auteur{mourougane}
\date{2021/08/10}
\organisation{exo7}
\contenu{
  \texte{Soient $E$ un espace affine euclidien de dimension 3 et $\mathcal{R}$ 
un repère cartésien orthonormé
de $E$. Soit
$n$ un entier $\geq 3$. On considère l'ensemble $X$ des points de $E$ 
dont les coordonnées $(x,y,z)$ dans
$\mathcal{R}$ satisfont aux deux conditions suivantes :}
\begin{enumerate}
  \item \question{Les nombres $x$, $y$ et $z$ sont dans $\Zz$ ;}
  \item \question{Le nombre $x+y+z$ est divisible par $n$.}
\end{enumerate}
\begin{enumerate}

\end{enumerate}
}