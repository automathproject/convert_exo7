\uuid{3280}
\titre{Ensi PC 1999}
\theme{Exercices de Michel Quercia, Décomposition en éléments simples}
\auteur{quercia}
\date{2010/03/08}
\organisation{exo7}
\contenu{
  \texte{}
  \question{D{\'e}composer en {\'e}l{\'e}ments simples sur $\R$ puis sur $\C$~:
$\frac1{(X^2+2X+1)(X^3-1)}$.}
  \reponse{$$\begin{aligned}\frac1{(X^2+2X+1)(X^3-1)}
&= \frac{-1/2}{(X+1)^2} + \frac{-3/4}{X+1} + \frac{1/12}{X-1} + \frac{1/3}{X-j} + \frac{1/3}{X-j^2}\cr
&= \frac{-1/2}{(X+1)^2} + \frac{-3/4}{X+1} + \frac{1/12}{X-1} + \frac13\frac{2X+1}{X^2+X+1}.\cr\end{aligned}$$}
}