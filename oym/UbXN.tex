\uuid{UbXN}
\theme{Exercices de Christophe Mourougane, Géométrie en petites dimensions}
\exercice{7261, mourougane, 2021/08/10}
\enonce
\begin{enumerate}

\item Soit $\mathcal{C}$ un cercle de centre $O$, $P$ et $Q$ deux points de $\mathcal{C}$ non diamétralement opposés. Calculer $\text{Mes}\widehat{OPQ}$ en fonction de $\text{Mes}\widehat{POQ}$. 
\item Soit $d$ la droite perpendiculaire à $(OP)$ passant par $P$. 
En calculant la distance entre $O$ et tout point $M$ de la droite $d$,
montrer que $P$ est l'unique point d'intersection entre $d$ et $\mathcal{C}$. 
La droite $d$ est appelée \emph{tangente au cercle $\mathcal{C}$ en $P$}. \\

