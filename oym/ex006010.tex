\exo7id{6010}
\titre{Exercice 6010}
\theme{}
\auteur{quinio}
\date{2011/05/20}
\organisation{exo7}
\contenu{
  \texte{Dans une poste d'un petit village, on remarque qu'entre 10 heures
et 11 heures, la probabilité pour que deux personnes entrent durant 
la même minute est considérée comme nulle et que l'arrivée des
personnes est indépendante de la minute considérée. 
On a observé que la probabilité pour qu'une personne se présente entre la
minute $n$ et la minute $n+1$ est: $p = 0.1$. On veut calculer la probabilité 
pour que : 3,4,5,6,7,8... personnes se présentent au guichet entre 10h et 11h.}
\begin{enumerate}
  \item \question{Définir une variable aléatoire adaptée, puis répondre au problème considéré.}
  \item \question{Quelle est la probabilité pour que au moins 10 personnes
se présentent au guichet entre 10h et 11h?}
\end{enumerate}
\begin{enumerate}

\end{enumerate}
}