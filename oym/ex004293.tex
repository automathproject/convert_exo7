\uuid{4293}
\titre{$x(f(x)-f(x+1))$}
\theme{}
\auteur{quercia}
\date{2010/03/12}
\organisation{exo7}
\contenu{
  \texte{}
  \question{Soit $f : {[1,+\infty[} \to {\R^+}$ une fonction décroissante telle que
$ \int_{t=1}^{+\infty} f(t)\,d t$ converge.

Montrer que $xf(x) \to 0$ lorsque $x\to+\infty$, puis que
$ \int_{t=1}^{+\infty} t(f(t)-f(t+1))\,d t$ converge, et calculer la valeur
de cette intégrale.}
  \reponse{$0 \le xf(x) \le 2 \int_{t=x/2}^x f(t)\,d t \to 0$ (lorsque $x\to+\infty$).\par
         $ \int_{t=1}^x t(f(t)-f(t+1))\,d t =
           \int_{t=1}^2 tf(t)\,d t +  \int_{t=2}^x f(t)\,d t -  \int_{t=x}^{x+1} (t-1)f(t)\,d t
          \to
           \int_{t=1}^2 tf(t)\,d t +  \int_{t=2}^{+\infty} f(t)\,d t$ (lorsque $x\to+\infty$).}
}