\uuid{7650}
\titre{Calculs de courbure}
\theme{}
\auteur{mourougane}
\date{2021/08/11}
\organisation{exo7}
\contenu{
  \texte{On considère l'espace affine euclidien orienté $\Rr^2$ muni d'un repère $(O,\vec{\imath},\vec{\jmath})$.}
\begin{enumerate}
  \item \question{Calculer la fonction courbure d'un cercle de rayon $r>0$.}
  \item \question{Soit $c~:I\to \Rr^2$ une courbe plane paramétrée par la longueur d'arc. 
Soit $R$ la rotation de centre $0$ et d'angle $\alpha$,
et $S$ la symétrie d'axe $x=y$.
Déterminer la fonction courbure de $R\circ c$ et celle de $S\circ c$.}
  \item \question{Calculer la fonction courbure de la courbe de Lissajous
$\left\{\begin{array}{cccc}
        c~:~&\Rr&\to&\Rr^2\\
&t&\mapsto&\left(\begin{array}{c}\cos 3t\\ \sin 2t\end{array}\right)
        \end{array}
\right.$}
  \item \question{Déterminer une courbe fermée à courbure partout strictement négative.}
  \item \question{Montrer qu'une courbe plane régulière de courbure nulle est un segment de droite.}
\end{enumerate}
\begin{enumerate}

\end{enumerate}
}