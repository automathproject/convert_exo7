\uuid{1632}
\titre{Exercice 1632}
\theme{}
\auteur{barraud}
\date{2003/09/01}
\organisation{exo7}
\contenu{
  \texte{}
  \question{Soit $\alpha$ et $\beta$ deux réels, et $A$ la matrice suivante :
$$
  A=\begin{pmatrix}
            1 & -\alpha &  -\alpha &       1 \\
      1-\beta &  \alpha & \alpha-1 &  -\beta \\
        \beta & -\alpha & 1-\alpha & 1+\beta \\
            0 &  \alpha &   \alpha &       0
    \end{pmatrix}
$$
A quelle condition sur $\alpha$ et $\beta$, $A$ est-elle diagonalisable ?

On suppose $\alpha=0$ et $\beta=0$. Vérifier que $A(A-I)=0$. En déduire
$A^{n}$ et $(A+I)^{-1}$ .}
  \reponse{}
}