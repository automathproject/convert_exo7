\uuid{Reo8}
\exo7id{3847}
\titre{Permutation de décimales}
\theme{Exercices de Michel Quercia, Fonctions continues}
\auteur{quercia}
\date{2010/03/11}
\organisation{exo7}
\contenu{
  \texte{Pour $x \in {[0,1[}$, on note $x = \sum_{k=1}^\infty \frac{x_k}{10^k}$ le
développement décimal propre de $x$.}
\begin{enumerate}
  \item \question{Soit $f : {[0,1[} \to {[0,1[}$ définie par :
      $f(x) = \sum_{k=1}^\infty \frac{x_{k+1}}{10^k}$. Montrer que $f$ est continue
      par morceaux.}
  \item \question{Soit $g : {[0,1[} \to {[0,1[}$ définie par :
      $g(x) = \sum_{k=1}^\infty \left( \frac{x_{2k}}{10^{2k-1}} +
                                        \frac{x_{2k-1}}{10^{2k}} \right)$.
      Déterminer les points où $g$ est continue.}
\end{enumerate}
\begin{enumerate}

\end{enumerate}
}