\uuid{OgY7}
\exo7id{3037}
\titre{Ordre sur les fonctions}
\theme{Exercices de Michel Quercia, Relations d'ordre}
\auteur{quercia}
\date{2010/03/08}
\organisation{exo7}
\contenu{
  \texte{Soit $X$ un ensemble et $E = \R^X$. On ordonne $E$ par :
$f \le g \iff \forall\ x \in X,\ f(x) \le g(x)$.}
\begin{enumerate}
  \item \question{V{\'e}rifier que c'est une relation d'ordre.}
  \item \question{L'ordre est-il total ?}
  \item \question{Comparer les {\'e}nonc{\'e}s : {\it``$f$ est major{\'e}e''}, et {\it``$\{f\}$ est major{\'e}''}.}
  \item \question{Soit $(f_i)_{i\in I}$ une famille major{\'e}e de fonctions de $E$. Montrer
     qu'elle admet une borne sup{\'e}rieure.}
\end{enumerate}
\begin{enumerate}

\end{enumerate}
}