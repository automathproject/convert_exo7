\uuid{2180}
\titre{Exercice 2180}
\theme{}
\auteur{debes}
\date{2008/02/12}
\organisation{exo7}
\contenu{
  \texte{}
  \question{a) Montrer qu'un groupe $G$ v\'erifiant
$$\forall a,b \in G \quad a^2b^2=(ab)^2$$
est commutatif.
\smallskip

(b) Le but de cette question est de donner un exemple de groupe $G$ 
v\'erifiant la propri\'et\'e 
$$\forall a,b\in G \quad a^3b^3=(ab)^3$$
et qui n'est pas commutatif.

\hskip 5mm (i) montrer qu'il existe un automorphisme $\sigma$ de
$\mathbb{F}_3^2$ d'ordre $3$.

\hskip 5mm (ii) montrer que le groupe $G$ d\'efini comme le produit semi-direct
de $\mathbb{F}_3^2$ par $\Z_3$, $\Z_3$ agissant sur $\mathbb{F}_3^2$ via
$\sigma$ r\'epond \`a la question.}
  \reponse{(a) L'identit\'e $a^2b^2=(ab)^2$, par simplification \`a gauche par $a$ et \`a
droite par $b$, se r\'e\'ecrit $ab=ba$.
\smallskip

(b) La correspondance $(x,y) \rightarrow (x+y,y)$ d\'efinit un automorphisme
$\sigma$ de $\mathbb{F}_3^2$ d'ordre $3$. Identifions le groupe $<\sigma>$ au
groupe $\Z/3\Z$ et consid\'erons le produit semi-direct  $\mathbb{F}_3^2 \times \hskip -6pt
{\raise 1.4pt\hbox{${\scriptscriptstyle |}$}} \Z/3\Z$. Pour tout \'el\'ement
$((x,y),i)$, on a $((x,y),i)^2 = ((x,y)+\sigma^i(x,y), 2i)$ et $((x,y),i)^3 =
((x,y)+\sigma^i(x,y)+\sigma^{2i}(x,y), 3i) = ((0,0),0)$ puisque $({\rm Id}+\sigma^i
+\sigma^{2i})(x,y) = (3x+iy+2iy,3y)=(0,0)$. La formule $a^3b^3=(ab)^3$ est donc
satisfaite pour tous $a,b$ dans $\mathbb{F}_3^2 \times \hskip -5pt
{\raise 1.4pt\hbox{${\scriptscriptstyle |}$}} \Z/3\Z$. Mais ce produit semi-direct
n'est pas commutatif car l'action de $\Z/3\Z$ n'est pas l'action triviale.}
}