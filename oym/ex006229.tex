\exo7id{6229}
\titre{Exercice 6229}
\theme{}
\auteur{queffelec}
\date{2011/10/16}
\organisation{exo7}
\contenu{
  \texte{Pour tout $k>0$ on note $H_k$ le sous-espace de $C([0,1])$ constitué des
fonctions lipschitziennes de constante $k$ ie des fonctions $f$ vérifiant
$\vert f(x)-f(y)\vert\leq k\vert x-y\vert$ pour tous $x$ et $y$ dans $[0,1]$.
On pose aussi $H=\bigcup_{k>0}H_k$.}
\begin{enumerate}
  \item \question{Montrer que $H$ contient les fonctions de classe $C^1$ sur $[0,1]$, mais
que la fonction $\sqrt x $ n'est pas dans $H$.}
  \item \question{Montrer que pour tout $k$, $H_k$ est un espace de Banach pour la norme
uniforme.}
  \item \question{Montrer qu'il existe une suite de fonction de $H$ qui converge uniformément
sur $[0,1]$ vers $\sqrt x$. En déduire que $H$ n'est pas complet pour la
norme uniforme.}
  \item \question{Montrer que si on pose $$\Vert f\Vert=\sup_{x\neq y}{{\vert
f(x)-f(y)\vert}\over{\vert x-y\vert}} +\vert f(0)\vert,$$ on définit ainsi une
norme sur l'espace $E$, pour laquelle l'espace est complet.}
\end{enumerate}
\begin{enumerate}

\end{enumerate}
}