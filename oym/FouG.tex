\uuid{FouG}
\exo7id{3672}
\titre{Caractérisation des bases orthonormales}
\theme{Exercices de Michel Quercia, Produit scalaire}
\auteur{quercia}
\date{2010/03/11}
\organisation{exo7}
\contenu{
  \texte{Soit $E$ un espace vectoriel euclidien, et $\vec e_1,\dots,\vec e_n$ des vecteurs unitaires tels que :
$\forall\ \vec x \in E, \quad \|\vec x\,\|^2 = \sum_{i=1}^n (\vec x\mid \vec e_i)^2$.}
\begin{enumerate}
  \item \question{Démontrer que $(\vec e_1,\dots,\vec e_n)$ est une base orthonormale de $E$.}
  \item \question{On remplace l'hypothèse : $\vec e_i$ unitaire par : $\dim E = n$.
  \begin{enumerate}}
  \item \question{Démontrer que $(\vec e_1,\dots,\vec e_n)$ est une base de $E$.}
  \item \question{Démontrer que : $\forall\ \vec x,\vec y \in E,\ (\vec x \mid\vec y) = \sum_{i=1}^n (\vec x\mid \vec e_i)(\vec y\mid \vec e_i)$.}
  \item \question{On note $G$ la matrice de Gram de $\vec e_1,\dots,\vec e_n$. Démontrer que $G^2 = G$ et conclure.}
\end{enumerate}
\begin{enumerate}
  \item \reponse{$\sum_{i=1}^n (\vec e_j\mid \vec e_i)^2 = 1  \Rightarrow {}$
        famille orthonormée et vect$(\vec e_i)^\perp$ = $\{\vec 0\}$.}
\end{enumerate}
}