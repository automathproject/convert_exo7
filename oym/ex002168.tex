\uuid{2168}
\titre{Exercice 2168}
\theme{}
\auteur{debes}
\date{2008/02/12}
\organisation{exo7}
\contenu{
  \texte{}
  \question{\label{ex:deb68}
On appelle cycle une permutation $\sigma $ v\'erifiant la propri\'et\'e
suivante:  il existe une partition de $\{ 1, \dots , n \} $ en deux sous-ensembles
$I$ et $J$ tels que la restriction de $\sigma $ \`a $I$ est l'identit\'e
de $I$
et il existe $a\in J$ tel que $J= \{ a, \sigma (a) , \dots , \sigma ^{r-1}
(a)\} $ o\`u $r$ est le cardinal de $J$. Le sous-ensemble $J$ est appel\'e le
support du cycle $\sigma $.

Un tel cycle sera not\'e $(a, \sigma (a), \dots , \sigma ^{r-1} (a) )$
\smallskip

(a) Soit
$\sigma
\in S_n$ une permutation. On consid\`ere le sous-groupe
$C$ engendr\'e par
$\sigma$ dans
$S_n$. Montrer que la restriction de $\sigma$ \`a chacune des orbites de
$\{1, \dots ,n\} $ sous
l'action de $C $ est un cycle, que ces diff\'erents cycles commutent entre
eux, et que $\sigma $
est le produit de ces cycles.
\smallskip

(b) D\'ecomposer en cycles les permutations suivantes de $\{ 1, \dots , 7\} $ :

$\hskip 20mm 1\hskip 2mm 2\hskip 2mm 3\hskip 2mm 4\hskip 2mm 5\hskip 2mm 6\hskip 2mm 7 \hskip
20mm  1\hskip 2mm 2\hskip 2mm 3\hskip 2mm 4\hskip 2mm 5\hskip 2mm 6\hskip 2mm 7\hskip 20mm   
1\hskip 2mm 2\hskip 2mm 3\hskip 2mm 4\hskip 2mm 5\hskip 2mm 6\hskip 2mm 7$

$\hskip 20mm 3\hskip 2mm 6\hskip 2mm 7\hskip 2mm 2\hskip 2mm 1\hskip 2mm 4\hskip 2mm 5 \hskip
20mm  7\hskip 2mm 4\hskip 2mm 2\hskip 2mm 3\hskip 2mm 5\hskip 2mm 6\hskip 2mm 1\hskip 20mm    
1\hskip 2mm 3\hskip 2mm 7\hskip 2mm 2\hskip 2mm 4\hskip 2mm 5\hskip 2mm 6$

\smallskip
(c) Montrer que si $\sigma $ est un cycle, $\sigma =(a, \sigma (a), \dots ,
\sigma^{r-1}(a)) $, la
conjugu\'ee $\tau \sigma \tau ^{-1} $ est un cycle et que
$\tau \sigma \tau ^{-1} =(\tau (a), \tau (\sigma  (a)), \dots , \tau
(\sigma ^{r-1} (a)))$.
\smallskip

(d) D\'eterminer toutes les classes de conjugaison des permutations dans $S_5$
(on consid\'erera leur d\'ecomposition en cycles). D\'eterminer tous les
sous-groupes distingu\'es de $S_5$.}
  \reponse{}
}