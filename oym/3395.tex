\uuid{3395}
\titre{Calcul de $A^n$ par la formule du binôme}
\theme{Exercices de Michel Quercia, Calcul matriciel}
\auteur{quercia}
\date{2010/03/10}
\organisation{exo7}
\contenu{
  \texte{}
  \question{Soit $A = \begin{pmatrix} 1&0&0\cr 0&1&1\cr 1&0&1 \cr\end{pmatrix}$.
En écrivant $A = I + J$, calculer $A^n$, $n \in \Z$.}
  \reponse{$A^n = \begin{pmatrix} 1 & 0 & 0 \cr C_n^2 & 1 & n \cr n & 0 & 1 \cr\end{pmatrix}$.}
}