\uuid{L7Ra}
\exo7id{5145}
\titre{I}
\theme{Le binôme. Les symboles $\sum_{}
\auteur{rouget}
\date{2010/06/30}
\organisation{exo7}
\contenu{
  \texte{}
\begin{enumerate}
  \item \question{(*) Calculer $\prod_{k=1}^{n}(1+\frac{1}{k})$, $n\in\Nn^*$.}
  \item \question{(***) Calculer $\prod_{k=1}^{n}\cos\frac{a}{2^k}$, $a\in]0,\pi[$, $n\in\Nn^*$.}
\end{enumerate}
\begin{enumerate}
  \item \reponse{Soit $n\in\Nn^*$.

\begin{align*}
\prod_{k=1}^{n}(1+\frac{1}{k})&=\prod_{k=1}^{n}\frac{k+1}{k}=\frac{\prod_{k=1}^{n}(k+1)}{\prod_{k=1}^{n}k}=\frac{(n+1)!}
{n!}=n+1
\end{align*}}
  \item \reponse{Soit $a\in]0,\pi[$ et $n\in\Nn^*$. Alors, pour tout naturel non nul $k$, on a
$0<\frac{a}{2^k}\leq\frac{a}{2}<\frac{\pi}{2}$ et donc $\sin\frac{a}{2^k}\neq0$.

On sait alors que pour tout réel $x$, $\sin(2x)=2\sin x\cos x$. Par suite, pour tout naturel $k$,

$$\sin(2.\frac{a}{2^k})=2\sin\frac{2^k}\cos\frac{a}{2^k}\quad\mbox{et donc}\quad
\cos\frac{a}{2^k}=\frac{\sin(a/2^{k-1})}{2\sin(a/2^k)}.$$

Mais alors,

\begin{align*}
\prod_{k=1}^{n}\cos\frac{a}{2^k}&=\prod_{k=1}^{n}\frac{\sin(a/2^{k-1})}{2\sin(a/2^k)}=\frac{1}{2^n}\frac{\prod_{k=1}^{n
}\sin(a/2^{k-1})}{\prod_{k=1}^{n}\sin(a/2^{k})}
 =\frac{1}{2^n}\frac{\prod_{k=0}^{n-1}
\sin(a/2^{k})}{\prod_{k=1}^{n}\sin(a/2^{k})}
=\frac{\sin a}{2^n\sin(a/2^n)}.
\end{align*}}
\end{enumerate}
}