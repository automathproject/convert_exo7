\uuid{5989}
\titre{Exercice 5989}
\theme{}
\auteur{quinio}
\date{2011/05/18}
\organisation{exo7}
\contenu{
  \texte{}
  \question{La probabilité pour une population d'être atteinte
d'une maladie $A$ est $p$ donné; dans cette même population, un individu
peut être atteint par une maladie $B$ avec une probabilité $q$ donnée aussi; 
on suppose que les maladies sont indépendantes : quelle est la
probabilité d'être atteint par l'une et l'autre de ces maladies?
Quelle est la probabilité d'être atteint par l'une ou l'autre de ces
maladies?}
  \reponse{$P(A\cap B)=pq$ car les maladies sont indépendantes.
$P(A\cup B)=P(A)+P(B)-P(A\cap B)=p+q-pq$}
}