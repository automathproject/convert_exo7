\uuid{MRPY}
\exo7id{3839}
\titre{Orthogonal des polynômes}
\theme{Exercices de Michel Quercia, Espaces vectoriels hermitiens}
\auteur{quercia}
\date{2010/03/11}
\organisation{exo7}
\contenu{
  \texte{Soit $E = \mathcal{C}([0,1],\R)$ muni du produit scalaire usuel, $F$ le sous-espace vectoriel des fonctions
polynomiales et $g$ la fonction exponentielle sur $[0,1]$.}
\begin{enumerate}
  \item \question{Montrer que $g\notin F$.}
  \item \question{Montrer qu'il existe une suite $(f_n)$ de fonctions polynomiales convergeant
    vers $g$ pour la norme eucli\-dienne.}
  \item \question{En déduire que $F$ n'a pas de supplémentaire orthogonal.}
\end{enumerate}
\begin{enumerate}

\end{enumerate}
}