\uuid{6067}
\titre{Exercice 6067}
\theme{}
\auteur{queffelec}
\date{2011/10/16}
\organisation{exo7}
\contenu{
  \texte{Soit $f$ une application de $\R$ dans $\R$, telle que $f(x+y)=f(x)+f(y)$
et $f(xy)=f(x)f(y)$ pour tous $x,y\in \R$. On va montrer que $f$ est soit
nulle, soit la fonction identité.}
\begin{enumerate}
  \item \question{Remarquer que $f(x)\geq0$ si
$x\geq0$ et ainsi, que $f$ est croissante.}
  \item \question{Montrer que pour tout $x$ réel on peut construire une suite $(r_k)$ et une
suite $(s_k)$ de rationnels telles que $r_k\uparrow x$ et $s_k\downarrow x$.
En déduire le résultat.}
\end{enumerate}
\begin{enumerate}

\end{enumerate}
}