\uuid{7723}
\titre{Exercice 7723}
\theme{Exercices de Christophe Mourougane, Théorie des groupes et géométrie}
\auteur{mourougane}
\date{2021/08/11}
\organisation{exo7}
\contenu{
  \texte{}
  \question{Combien le groupe $\mathcal{S}_5$ contient-il de $5$-Sylow ?}
  \reponse{Le théorème de Sylow donne que ce nombre $N$ est congru à 1 modulo $5$ et divise $24$. C'est donc $1$ ou $6$.
Comme $(12)(12345)(12)=(21345)$ n'est pas dans le $5$-Sylow (d'ordre $5$) $<(12345)>$ il y a un $5$-Sylow non distingué.
 Ainsi, $N=6$.}
}