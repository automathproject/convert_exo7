\uuid{4814}
\titre{$uv - vu = \mathrm{id}$}
\theme{}
\auteur{quercia}
\date{2010/03/16}
\organisation{exo7}
\contenu{
  \texte{Soit $E$ un espace vectoriel norm{\'e} et $u,v \in \mathcal{L}(E)$ tels que
$u\circ v - v \circ u = \mathrm{id}_E$.}
\begin{enumerate}
  \item \question{Calculer $u\circ v^n - v^n \circ u$ pour $n \in \N^*$.}
  \item \question{Montrer que $u$ ou $v$ est discontinu.}
\end{enumerate}
\begin{enumerate}
  \item \reponse{$nv^{n-1}$.}
  \item \reponse{Si $u$ et $v$ sont continus,
             $n\|v^{n-1}\| \le 2\|u\|\,\|v^n\| \le 2\|u\|\,\|v^{n-1}\|\,\|v\|$.\par
             S'il existe $k$ tel que $v^k = 0$, on peut remonter jusqu'{\`a} $v = 0$, absurde.
             Sinon, on a aussi une contradiction.}
\end{enumerate}
}