\uuid{5Ivq}
\exo7id{7000}
\titre{Exercice 7000}
\theme{\'Equations différentielles, Second ordre}
\auteur{blanc-centi}
\date{2015/07/04}
\organisation{exo7}
\contenu{
  \texte{Résoudre les équations différentielles suivantes à l'aide du changement de variable suggéré.}
\begin{enumerate}
  \item \question{$x^2y''+xy'+y=0$, sur $]0;+\infty[$, en posant $x=e^t$;}
  \item \question{$(1+x^2)^2y''+2x(1+x^2)y'+my=0$, sur $\R$, en posant $x=\tan t$ (en fonction de $m\in\R$).}
\end{enumerate}
\begin{enumerate}
  \item \reponse{Puisqu'on cherche $y$ fonction de $x\in]0;+\infty[$, et que l'application $t\mapsto e^t$ 
est bijective de $\R$ sur $]0;+\infty[$, on peut poser $x=e^t$ et $z(t)=y(e^t)$. 
On a alors $t= \ln x$ et $y(x) = z(\ln x)$.
Ce qui donne :
\begin{eqnarray*}
y(x)&=&z(\ln x)=z(t)\\
y'(x)&=&\frac{1}{x}z'(\ln x)=e^{-t}z'(t)\\
y''(x)&=&-\frac{1}{x^2}z'(\ln x)+\frac{1}{x^2}z''(\ln x)=-e^{-2t}z'(t)+e^{-2t}z''(t)
\end{eqnarray*}
En remplaçant, on obtient donc que 
$$\forall x\in]0;+\infty[,\ x^2y''+xy'+y=0\Longleftrightarrow\forall t\in\R,\ z''(t)+z(t)=0$$
autrement dit, $z(t)=\lambda\cos t+\mu\sin t$ où $\lambda,\mu\in\R$. 
Finalement, les solutions de l'équation de départ sont de la forme 
$$y(x) = z(\ln x) = \lambda\cos(\ln x)+\mu\sin(\ln x)$$ où $\lambda,\mu\in\R$.}
  \item \reponse{L'application $t\mapsto \tan t$ étant bijective de 
$]-\frac{\pi}{2};\frac{\pi}{2}[$ sur $\R$, on peut poser 
$x=\tan t$ et $z(t)=y(\tan t)$. On a alors $t = \arctan x$ et ainsi :
\begin{eqnarray*}
y(x)&=&z(\arctan x)=z(t)\\
y'(x)&=&\frac{1}{1+x^2}z'(\arctan x)\\
y''(x)&=&\frac{1}{(1+x^2)^2}\big(z''(\arctan x)-2xz'(\arctan x)\big)
\end{eqnarray*}
En remplaçant, on obtient que $z$ est solution de l'équation différentielle $z''+mz=0$. Pour résoudre cette équation, on doit distinguer trois cas:
\begin{itemize}}
  \item \reponse{$m<0$: alors $z(t)=\lambda e^{\sqrt{-m}t}+\mu e^{-\sqrt{-m}t}$ et donc  
$$y(x)=\lambda e^{\sqrt{-m}\arctan x}+\mu e^{-\sqrt{-m}\arctan x},$$}
  \item \reponse{$m=0$: $z''=0$ et donc $z(t)=\lambda t+\mu$ et $y(x)=\lambda\arctan x+\mu$,}
  \item \reponse{$m>0$: alors $z(t)=\lambda\cos(\sqrt{m}t)+\mu\sin(\sqrt{m}t)$ et donc 
$$y(x)=\lambda\cos(\sqrt{m}\arctan x)+\mu\sin(\sqrt{m}\arctan x)$$
où $\lambda,\mu\in\R$.
\end{itemize}}
\end{enumerate}
}