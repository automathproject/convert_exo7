\uuid{5647}
\titre{***}
\theme{}
\auteur{rouget}
\date{2010/10/16}
\organisation{exo7}
\contenu{
  \texte{}
  \question{Soient $A$, $B$, $C$ et $D$ quatre matrices carrées de format $n$. Montrer que si $C$ et $D$ commutent et si $D$ est inversible alors $\text{det}\left(
\begin{array}{cc}
A&B\\
C&D
\end{array}
\right)=\text{det}(AD-BC)$. Montrer que le résultat persiste si $D$ n'est pas inversible.}
  \reponse{Si $D$ est inversible, un calcul par blocs fournit
 
 
\begin{center}
$\left(
\begin{array}{cc}
A&B\\
C&D
\end{array}
\right)\left(
\begin{array}{cc}
D&0\\
-C&D^{-1}
\end{array}
\right)=\left(
\begin{array}{cc}
AD-BC&BD^{-1}\\
CD-DC&I
\end{array}
\right)=\left(
\begin{array}{cc}
AD-BC&BD^{-1}\\
0&I
\end{array}
\right)$ (car $C$ et $D$ commutent)
\end{center}

et donc, puisque

\begin{align*}\ensuremath
\text{det}\left(\left(
\begin{array}{cc}
A&B\\
C&D
\end{array}
\right)\left(
\begin{array}{cc}
D&0\\
-C&D^{-1}
\end{array}
\right)\right)&=\text{det}\left(
\begin{array}{cc}
A&B\\
C&D
\end{array}
\right)\text{det}\left(
\begin{array}{cc}
D&0\\
-C&D^{-1}
\end{array}
\right)=\text{det}\left(
\begin{array}{cc}
A&B\\
C&D
\end{array}
\right)\times\text{det}D\times\text{det}D^{-1}\\
 &=\text{det}\left(
\begin{array}{cc}
A&B\\
C&D
\end{array}
\right)
\end{align*}

et que $\text{det}\left(
\begin{array}{cc}
AD-BC&BD^{-1}\\
0&I
\end{array}
\right)=\text{det}(AD-BC)$, on a bien $\text{det}\left(
\begin{array}{cc}
A&B\\
C&D
\end{array}
\right)=\text{det}(AD-BC)$ (si $C$ et $D$ commutent).

Si $D$ n'est pas inversible, $\text{det}(D-xI)$ est un polynôme en $x$ de degré $n$ et donc ne s'annule qu'un nombre fini de fois. Par suite, la matrice $D-xI$ est inversible sauf peut-être pour un nombre fini de valeurs de $x$. D'autre part, pour toute valeur de $x$, les matrices $C$ et $D-xI$ commutent et d'après ce qui précède, pour toutes valeurs de $x$ sauf peut-être pour un nombre fini, on a

\begin{center}
$\text{det}\text{det}\left(
\begin{array}{cc}
A&B\\
C&D
\end{array}
\right)=\text{det}(A(D-xI)-BC)$.
\end{center}

Ces deux expressions sont encore des polynômes en $x$ qui coïncident donc en une infinité de valeurs de $x$ et sont donc égaux. Ces deux polynômes prennent en particulier la même valeur en $0$ et on a montré que

\begin{center}
si $C$ et $D$ commutent, $\text{det}\left(
\begin{array}{cc}
A&B\\
C&D
\end{array}
\right)=\text{det}(AD-BC)$.
\end{center}}
}