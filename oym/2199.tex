\uuid{2199}
\titre{Exercice 2199}
\theme{Théorème de Sylow}
\auteur{debes}
\date{2008/02/12}
\organisation{exo7}
\contenu{
  \texte{}
  \question{Montrer qu'un groupe d'ordre $200$ n'est pas simple.}
  \reponse{D'apr\`es les th\'eor\`emes de Sylow, le nombre de $5$-Sylow d'un groupe d'ordre
$200=5^2.2^3$ est $\equiv 1\ [\hbox{\rm mod}\ 5]$ et divise $8$. Ce ne peut \^etre
que $1$. L'unique $5$-Sylow est n\'ecessairement distingu\'e puisque ses 
conjugu\'es sont des $5$-Sylow et coincident donc avec lui. Le groupe ne peut pas
\^etre simple.}
}