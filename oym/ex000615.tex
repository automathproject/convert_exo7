\uuid{615}
\titre{Exercice 615}
\theme{}
\auteur{ridde}
\date{1999/11/01}
\organisation{exo7}
\contenu{
  \texte{}
  \question{Soit $f : \Rr^+  \rightarrow \Rr^+ $ croissante telle que $ \lim\limits_{
x \rightarrow  + \infty}f (x + 1)-f (x) = 0 $. Montrer que $\lim\limits_{
x \rightarrow  + \infty}\dfrac{f (x)}x = 0$.
 (on pourra utiliser des $\epsilon$, sommer des in\'egalit\'es et utiliser la
 monotonie de $f$ pour montrer qu'elle est born\'ee sur un segment).\\
 Comment g\'en\'eraliser ce r\'esultat ?}
  \reponse{}
}