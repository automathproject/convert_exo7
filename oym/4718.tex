\uuid{4718}
\titre{$(\sin(n))$ est dense}
\theme{Exercices de Michel Quercia, Topologie de $\mathbb{R}
\auteur{quercia}
\date{2010/03/16}
\organisation{exo7}
\contenu{
  \texte{}
  \question{Soit $a \in \R\setminus \Q$ et $A = \{ ma + n \text{ tq } m\in \Z,\ n\in \N\}$.
Montrer que $A$ est dense dans $\R$.

Application : Montrer que tout r{\'e}el de $[-1,1]$ est valeur d'adh{\'e}rence de la suite
$(\sin n)$.}
  \reponse{}
}