\uuid{XUIR}
\exo7id{7562}
\titre{Calcul d'intégrales et théorème de Liouville}
\theme{Exercices de Christophe Mourougane, Fonctions holomorphes}
\auteur{mourougane}
\date{2021/08/10}
\organisation{exo7}
\contenu{
  \texte{}
\begin{enumerate}
  \item \question{Soit $r>0$ et $D$ un voisinage ouvert de $\overline{\Delta_r}$.
 Soit $f : D\to\Cc$ une fonction holomorphe. Soit $a$ et $b$ deux points distincts dans $\Delta_r$.
 Calculer $$\int_{\partial\Delta_r}\frac{f(\zeta)}{(\zeta-a)(\zeta-b)}d\zeta.$$}
  \item \question{En déduire le théorème de Liouville : Toute fonction holomorphe bornée sur $\Cc$ est constante.}
\end{enumerate}
\begin{enumerate}

\end{enumerate}
}