\exo7id{6235}
\titre{Exercice 6235}
\theme{}
\auteur{queffelec}
\date{2011/10/16}
\organisation{exo7}
\contenu{
  \texte{}
\begin{enumerate}
  \item \question{On considère une matrice $A=(a_{ij})$ à coefficients réels telle que
$\sum_{i,j=1}^n  a_{ij}^2 <1$.
En utilisant le théorème du point fixe, montrer que quels que soient les réels
$b_1,b_2,\cdots b_n$, le système d'équations linéaires
$$x_i-\sum_{j=1}^n  a_{ij} x_j = b_i,\  \  \  1\leq i\leq n$$
admet toujours une solution unique. En déduire $\det (I-A)\neq 0$.}
  \item \question{Montrer sous les mêmes hypothèses que le système non linéaire
$$x_i-\sum_{j=1}^n  \sin(a_{ij} x_j) = b_i,\  \  \  1\leq i\leq n$$
admet une unique solution.}
\end{enumerate}
\begin{enumerate}

\end{enumerate}
}