\uuid{5130}
\titre{*T}
\theme{}
\auteur{rouget}
\date{2010/06/30}
\organisation{exo7}
\contenu{
  \texte{}
  \question{Calculer $(1+i\sqrt{3})^9$.}
  \reponse{$(1+i\sqrt{3})^9=(2e^{i\pi/3})^9=2^9e^{3i\pi}=-512$.

\begin{center}
\shadowbox{
\begin{tabular}{l}
La forme algébrique d'un complexe est particulièrement bien adaptée à l'addition.\\
La forme trigonométrique d'un complexe est particulièrement bien adaptée à la multiplication.
\end{tabular}
}
\end{center}}
}