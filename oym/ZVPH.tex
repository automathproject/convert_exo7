\uuid{ZVPH}
\exo7id{7466}
\titre{Exercice 7466}
\theme{Exercices de Christophe Mourougane, Géométrie euclidienne}
\auteur{mourougane}
\date{2021/08/10}
\organisation{exo7}
\contenu{
  \texte{On considére le plan muni d'un un repère orthonormé ($O, \overrightarrow {\imath},\overrightarrow{\jmath}$) et la courbe $(C)$ d'équation 
\begin{center}$x^{2} + xy + y^{2} + x - y =0 $ \end{center}}
\begin{enumerate}
  \item \question{Montrer que cette courbe possède un centre de symétrie $\Omega$ et donner son équation dans le repère ($\Omega, \overrightarrow {\imath},\overrightarrow{\jmath}$)}
  \item \question{Montrer que dans le repère ($\Omega, \overrightarrow {\imath },\overrightarrow{\jmath}$) la première bissectrice $\Delta$ est axe de symétrie.}
  \item \question{On considère le repère orthonormé direct ($\Omega, \overrightarrow {I},\overrightarrow{J}$) où $\overrightarrow {I}$ est un vecteur unitaire de $\Delta$. Donner l'équation de $(C)$ dans ce repère.}
  \item \question{Montrer que dans le repère ($\Omega, \overrightarrow {\imath },\overrightarrow{\jmath}$) la seconde bissectrice $\Delta^{'}$ est axe de symétrie.

Donner les équations de $\Delta$ et $\Delta^{'}$ dans le repère $(O, \overrightarrow {\imath},\overrightarrow{\jmath})$.

Sachant que $(C)$ est une ellipse, tracer $(C)$ dans le repère $(O, \overrightarrow {\imath},\overrightarrow{\jmath})$.}
\end{enumerate}
\begin{enumerate}

\end{enumerate}
}