\uuid{LbQb}
\exo7id{2802}
\titre{Exercice 2802}
\theme{Dérivabilité au sens complexe, fonctions analytiques, Trigonométrie complexe}
\auteur{burnol}
\date{2009/12/15}
\organisation{exo7}
\contenu{
  \texte{Lorsque $z$ est complexe les fonctions $\sin(z)$, $\cos(z)$, $\sh(z)$ et
   $\ch(z)$ sont définies par les formules:
\begin{align*}
\sin(z) &= \frac{e^{iz} - e^{-iz}}{2i}&\qquad \sh(z) &=
   \frac{e^{z} - e^{-z}}{2}  \\
 \cos(z) &=
   \frac{e^{iz} + e^{-iz}}{2}&\qquad \ch(z) &=
   \frac{e^{z} + e^{-z}}{2}
\end{align*}}
\begin{enumerate}
  \item \question{Montrer que $\cos$ et $\ch$ sont des fonctions paires et
   $\sin$ et $\sh$ des fonctions impaires et donner leurs
   représentations comme
   séries entières. Prouver $e^{iz} = \cos(z) + i \sin(z)$,
   $\sin(iz) = i\sh(z)$, $\cos(iz) = \ch(z)$, $\sh(iz) =
   i\sin(z)$, $\ch(iz) = \cos(z)$.}
  \item \question{Établir les formules:
\[ \cos(z+w) = \cos(z)\cos(w) - \sin(z)\sin(w)\]
\[ \sin(z+w) = \sin(z)\cos(w) + \cos(z)\sin(w)\]
en écrivant de deux manières différentes $e^{\pm
   i(z+w)}$. Donner une autre preuve en utilisant le
   principe du prolongement analytique et la validité
   (admise) des formules pour $z$ et $w$ réels.}
  \item \question{Prouver pour tout $z$ complexe $\cos(\pi + z) = -\cos(z)$,
  $\sin(\pi+ z) =
     -\sin(z)$. Prouver $\cos(\frac\pi2 - z) = \sin(z)$.}
  \item \question{Prouver les formules $\cos^2 z + \sin^2 z = 1$ et $\ch^2 z
   - \sh^2 z = 1$ pour tout $z\in\Cc$.}
\end{enumerate}
\begin{enumerate}

\end{enumerate}
}