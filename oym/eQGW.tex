\uuid{eQGW}
\exo7id{4335}
\titre{Approximation de la mesure de Dirac}
\theme{Exercices de Michel Quercia, Intégrale dépendant d'un paramètre}
\auteur{quercia}
\date{2010/03/12}
\organisation{exo7}
\contenu{
  \texte{Soit $f : {[a,b]} \to {\R^+}$ continue atteignant son maximum en un unique
point $c \in {]a,b[}$.}
\begin{enumerate}
  \item \question{Soit $\mu > 0$ tel que $[c-\mu,c+\mu] \subset [a,b]$.
    Chercher $\lim_{n\to\infty} \biggl( \int_{t=a}^b f^n(t)\,d t\biggm/ \int_{t=c-\mu}^{c+\mu} f^n(t)\,d t\biggr)$.}
  \item \question{Soit $g : {[a,b]} \to {\R^+}$ continue.
    Chercher $\lim_{n\to\infty} \biggl( \int_{t=a}^b f^n(t)g(t)\,d t\biggm/ \int_{t=a}^b f^n(t)\,d t\biggr)$.}
\end{enumerate}
\begin{enumerate}
  \item \reponse{$g(c)$.}
\end{enumerate}
}