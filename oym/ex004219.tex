\exo7id{4219}
\titre{$x^4 \frac{\partial^2 f}{\partial x^2} - \frac{\partial^2 f}{\partial y^2} = 0$}
\theme{}
\auteur{quercia}
\date{2010/03/11}
\organisation{exo7}
\contenu{
  \texte{}
\begin{enumerate}
  \item \question{Trouver les fonctions $g : {\{(u,v)\in\R^2\text{ tq } u > v\}} \to \R$ de classe $\mathcal{C}^2$ vérifiant :
    $\frac{\partial }{\partial u}\Bigl(g+v\frac{\partial g}{\partial v}\Bigr) = \frac{\partial}{\partial  v}\Bigl(g+u\frac{\partial g}{\partial u}\Bigr)$

    (penser au théorème de Poincaré).}
  \item \question{Résoudre sur $\R^{+*}\times\R$ l'équation : $x^4 \frac{\partial^2 f}{\partial x^2} - \frac{\partial^2 f}{\partial y^2} = 0$ en posant
    $u = y+\frac 1x$, $v = y-\frac 1x$.}
\end{enumerate}
\begin{enumerate}
  \item \reponse{Il existe $G$ telle que $\frac{\partial G}{\partial u} = g+u\frac{\partial g}{\partial u} = \frac{\partial (ug)}{\partial u}$ et
                                     $\frac{\partial G}{\partial v} = g+v\frac{\partial g}{\partial v} = \frac{\partial(vg)}{\partial v}$.\par
             Donc $G = ug + \varphi(v) = vg + \psi(u)$, d'où
             $g = \frac{\varphi(v) - \psi(u)}{v-u}$.
             La réciproque est immédiate.}
  \item \reponse{$f(x,y) = x\Bigl(\varphi\Bigl(y-\frac1x\Bigr)
                     + \psi\Bigl(y+\frac1x\Bigr)\Bigr)$.}
\end{enumerate}
}