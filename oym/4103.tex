\uuid{4103}
\titre{Intégrale fonction d'un paramètre}
\theme{Exercices de Michel Quercia, \'Equations différentielles linéaires (II)}
\auteur{quercia}
\date{2010/03/11}
\organisation{exo7}
\contenu{
  \texte{}
  \question{On pose $f(x) =  \int_{t=0}^{+\infty} e^{-t}e^{itx} \frac {d t}{\sqrt t}$.
Former une équation différentielle satisfaite par $f$. En déduire $f$.}
  \reponse{$f'(x) = \frac {x+i}{2(x^2+1)} f(x)
  \Rightarrow  f(x) = \sqrt\pi (x^2+1)^{-1/4}\exp\left(\frac i2\arctan x\right)$.}
}