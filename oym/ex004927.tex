\uuid{4927}
\titre{Centre de symétrie}
\theme{}
\auteur{quercia}
\date{2010/03/17}
\organisation{exo7}
\contenu{
  \texte{Soit ${\cal S}$ une quadrique d'équation $f(x,y,z) = 0$. On note $q$ la forme
quadratique associée à $f$.}
\begin{enumerate}
  \item \question{Montrer que, pour tout point $A$ et tout vecteur $\vec h$, on a :
    $f(A+\vec h) = f(A) + (\vec{\nabla f}(A)\mid \vec h) + q(\vec h)$.}
  \item \question{On suppose que ${\cal S}$ n'est pas incluse dans un plan.
    Montrer qu'un point $\Omega$ est centre de symétrie de ${\cal S}$
    si et seulement si $\vec{\nabla f}(\Omega) = \vec 0$.}
  \item \question{En déduire que si $0$ n'est pas valeur propre de la matrice de $q$,
    alors ${\cal S}$ admet un centre unique.}
\end{enumerate}
\begin{enumerate}

\end{enumerate}
}