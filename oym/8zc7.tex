\uuid{8zc7}
\exo7id{4084}
\titre{Polytechnique PC 2002}
\theme{Exercices de Michel Quercia, \'Equations différentielles linéaires (I)}
\auteur{quercia}
\date{2010/03/11}
\organisation{exo7}
\contenu{
  \texte{Soit l'équation différentielle~:
$(E)\Leftrightarrow u''(x) + (k-2d\cos(x))u(x)= 0$.}
\begin{enumerate}
  \item \question{Existence et domaine de définition des solutions
    maximales $A$ et $B$ telles que
    $A(0)=1$, $A'(0)=0$ et $B(0)=0$, $B'(0)=1$.}
  \item \question{Montrer que $A$ est paire et $B$ est impaire.}
  \item \question{Montrer que $A(k,d,x) = A(k,0,x) +
    2d \int_{t=0}^xB(d,0,x-t)A(k,d,x)\cos(t)\,d t$.}
\end{enumerate}
\begin{enumerate}
  \item \reponse{thm de Cauchy-Lipschitz linéaire.}
  \item \reponse{$x \mapsto A(-x)$ et $x \mapsto -B(-x)$ sont solutions de $(E)$ et
    vérifient les bonnes conditions initiales.}
  \item \reponse{Résoudre $u''(x) + ku(x) = 2d\cos(x)u(x)$ par la formule de Duhamel.}
\end{enumerate}
}