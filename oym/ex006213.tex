\exo7id{6213}
\titre{Exercice 6213}
\theme{}
\auteur{queffelec}
\date{2011/10/16}
\organisation{exo7}
\contenu{
  \texte{}
\begin{enumerate}
  \item \question{Pour $x,y\in\Rr^*$ on pose  $d(x,y)=|x-y|+|{1\over x}-{1\over y}|$.

 Montrer que $d$ définit une distance sur $\Rr^*$
qui induit la topologie usuelle et que $({\Rr^*},d)$ est complet.}
  \item \question{Plus généralement soit $U$ un ouvert d'un espace complet $(X,d)$; comment
peut-on définir une métrique $\delta$ sur $U$, équivalente à la métrique
initiale, qui fasse de $U$ un espace complet ?}
\end{enumerate}
\begin{enumerate}

\end{enumerate}
}