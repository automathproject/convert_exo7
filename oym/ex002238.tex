\uuid{2238}
\titre{Exercice 2238}
\theme{}
\auteur{matos}
\date{2008/04/23}
\organisation{exo7}
\contenu{
  \texte{Soient $A$ et $B$ deux matrices r\'eelles d'ordre $N$ et $a,b$ deux vecteurs de $\Rr^n$. On consid\`ere les deux it\'erations suivantes:
\begin{equation}\label{ite}
\left\{ \begin{array}{ccc}
x_{k+1}&=& By_k +a\\
y_{k+1}&=& Ax_k + b
\end{array}\right.  \ \ k=0, 1, \cdots
\end{equation}
avec $x_0, y_0 \in\Rr^n$ donn\'es.}
\begin{enumerate}
  \item \question{D\'eterminer une condition n\'ecessaire et suffisante de convergence des deux suites de vecteurs.}
  \item \question{Soit $z_k= ( x_k, y_k)^T \in\Rr^{2n}$. Montrer que (\ref{ite}) peut s'\'ecrire
$$z_{k+1}= Cz_k+c$$
o\`u  $C$ est une matrice d'ordre $2n$. Expliciter $C$ et $c$.}
  \item \question{Montrer que $\rho^2(C)=\rho(AB)$.}
  \item \question{On consid\`ere maintenant les deux it\'erations suivantes:
\begin{equation}\label{ite2}
\left\{ \begin{array}{ccc}
x_{k+1}&=& By_k +a\\
y_{k+1}&=& Ax_{k+1} + b
\end{array}\right.  \ \ k=0, 1, \cdots
\end{equation}
Donner une condition n\'ecessaire et suffisante de convergence.

Montrer que (\ref{ite2}) est \'equivalent \`a
$$z_{k+1}= Dz_k+d$$
o\`u  $D$ est une matrice d'ordre $2N$.

Montrer que $\rho (D)=\rho (AB)$.}
  \item \question{{\bf Taux de convergence}

On appelle taux de convergence asymptotique de la matrice it\'erative $M$ le nombre
$$R(M) =-\ln (\rho (M))) .$$
On pose $e^k=x^k-x^*$ l'erreur de l'it\'er\'e d'ordre $k$.
\begin{enumerate}}
  \item \question{Montrer que le nombre d'it\'erations $k$ pour r\'eduire l'erreur d'un facteur $\epsilon$ , i.e., $\frac{\|e^k\|}{\|e^0\|}\leq \epsilon $ v\'erifie
$$k\geq \frac{-\ln \epsilon}{R(M)} .$$}
  \item \question{Comparer le taux de convergence des algorithmes (\ref{ite}) et (\ref{ite2}).}
\end{enumerate}
\begin{enumerate}
  \item \reponse{C'est facile \`a voir que si $(x_k)$ converge vers $x^*$ et $(y_k)$
  converge vers $y^*$, alors $x^*$ et $y^*$ sont solution des syst\`emes
  $(I-BA)x^*=Bb+a$ et $(I-AB)y^*=Aa+b$. On a:
$$\left\{\begin{array}{l}x_{k+1}=B(Ax_{k-1}+b)+a=BAx_{k-1}+Bb+a\\
y_{k+1}=A(By_{k-1}+a)+b = ABy_{k-1}+Aa+b\end{array}\right.$$
et donc $(x_k)$ converge ssi $\rho (BA) <1$ et $(y_k)$ converge ssi $\rho
(AB)<1$.}
  \item \reponse{$z_{k+1}=Cz_k+c$ avec
  $C=\left(\begin{array}{cc}0&B\\A&0\end{array}\right),
  c=\left(\begin{array}{c}a\\b\end{array}\right)$}
  \item \reponse{Soit $\lambda$ valeur propre  non nulle de $C$ et $z=(x,y)^T$ vecteur
  propre associ\'e
$$Cz=\lambda z\Leftrightarrow\left\{\begin{array}{l}By=\lambda x\\Ax=\lambda
    y\end{array}\right.\Rightarrow ABy=\lambda Ax=\lambda^2 y\Rightarrow$$
$\lambda^2$ est valeur propre de $AB$.

Soit maintenant $\alpha$ valeur propre de $AB$ $\Leftrightarrow \exists u\neq 0: \quad
ABu=\alpha u$. On pose $\beta^2=\alpha$ et $x=Bu$, $y=\beta u$
$$C\left(\begin{array}{c}x\\y\end{array}\right)= \left(\begin{array}{c}\beta
    Bu\\ABu\end{array}\right)=\left(\begin{array}{c}\beta Bu\\\beta^2
    u\end{array}\right)=\beta \left(\begin{array}{c}x\\y\end{array}\right)$$
et donc $\rho^2 (C)=\rho (AB)$}
  \item \reponse{$D=\left(\begin{array}{cc}0&B\\0&AB\end{array}\right),\quad
  d=\left(\begin{array}{c}a\\Aa+b\end{array}\right)$.
La d\'emonstration de $\rho (D)=\rho (AB)$ se fait comme dans la question
  pr\'ec\'edente.}
  \item \reponse{\begin{enumerate}}
  \item \reponse{$e^k=M^ke^0\Rightarrow\frac{\|e^k\|}{\|e^0\|}\leq \|M^k\|\leq
  \epsilon$. Il suffit donc d'avoir $\|M^k\|^{1/k} \leq \epsilon^{1/k}\Rightarrow
  \log (\|M^k\|^{1/k})\leq \frac{1}{k}\log\epsilon$ c'est- \`a dire $k\geq
  \frac{\log \epsilon}{\log (\|M^k\|^{1/k})}$ Mais comme $\rho (M) \leq
  \|M^k\|^{1/k}$ on obtient finalement
 $$k\geq -\log\epsilon /R(M)$$}
  \item \reponse{nous avons $\rho^2(C) = \rho (AB) \Rightarrow \rho (C) =\sqrt{\rho (AB)}$ et
  $\rho (D)=\rho (AB)$ . Donc $\rho (D)<\rho (C) \Rightarrow R(D) >R(C)$. Donc on
  atteint la m\^eme r\'eduction d'erreur avec un plus petit nombre
  d'it\'erations de la m\'ethode 2)}
\end{enumerate}
}