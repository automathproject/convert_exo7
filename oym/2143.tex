\uuid{2143}
\titre{Exercice 2143}
\theme{Morphisme, sous-groupe distingué, quotient}
\auteur{debes}
\date{2008/02/12}
\organisation{exo7}
\contenu{
  \texte{}
  \question{(a) Montrer que pour tous entiers premiers entre eux $m,n>0$, les deux groupes $(\Zz/mn\Zz)^\times$ et $(\Zz/m\Zz)^\times \times (\Zz/n\Zz)^\times$ sont isomorphes. En d\'eduire que $\varphi (mn) = \varphi (m) \varphi (n)$, o\`u $\varphi$ est la fonction indicatrice d'Euler.
\smallskip

(b) Le groupe multiplicatif $(\Zz/15\Zz)^\times$ est-il cyclique? Montrer que $(\Zz/8\Zz)^\times \simeq (\Zz/2\Zz) \times (\Zz/2\Zz)$, que $(\Zz/16\Zz)^\times \simeq (\Zz/4\Zz)\times (\Zz/2\Zz)$. Etudier le groupe multiplicatif $(\Zz/24\Zz)^\times$.}
  \reponse{}
}