\exo7id{2623}
\titre{Exercice 2623}
\theme{}
\auteur{debievre}
\date{2009/05/19}
\organisation{exo7}
\contenu{
  \texte{Soit $f$ la fonction sur $\R^2$ d\'efinie par $f(x,y)= x\cos y + y\exp x$.}
\begin{enumerate}
  \item \question{Calculer ses d\'eriv\'ees partielles.}
  \item \question{Soit $v=(\cos \theta, \sin \theta)$, $\theta\in [0,2\pi[$. Calculer $D_vf(0,0)$. Pour quelle(s) valeurs de $\theta$ cette d\'eriv\'ee directionnelle de $f$ est-elle maximale/minimale? Que cela signifie-t-il?}
\end{enumerate}
\begin{enumerate}
  \item \reponse{$\frac{\partial f}{\partial x}=\cos y + y\exp x$,\ 
$\frac{\partial f}{\partial y}=-x\sin y + \exp x $.}
  \item \reponse{$D_vf(0,0)=\cos \theta \,\frac{\partial f}{\partial x}(0,0)
+\sin \theta \,\frac{\partial f}{\partial y}(0,0)
=\cos \theta +\sin \theta$.
Cette d\'eriv\'ee directionnelle de $f$ est maximale
quand $\sin \theta = \cos \theta = \frac{\sqrt{2}}{2}$, c.a.d. quand
  $\theta =\frac \pi 4$,
et minimale quand $\sin \theta = \cos \theta = -\frac{\sqrt{2}}{2}$, 
c.a.d. quand  $\theta =\frac 54 \pi$.

Signification g\'eom\'etrique:
Le plan engendr\'e par le vecteur $(\cos \theta,\sin \theta ,0)$ 
et l'axe des $z$ rencontre le graphe $z=f(x,y)$ en une courbe.
Cette courbe est de pente maximale en valeur absolue pour
$\cos \theta=\sin \theta=\frac{\sqrt{2}}{2}$ et
$\cos \theta=\sin \theta=-\frac{\sqrt{2}}{2}$ (m\^eme plan).
Les deux signes s'expliquent par les deux orientations possibles de cette courbe
(sens du param\'etrage).}
\end{enumerate}
}