\uuid{7FZh}
\exo7id{4484}
\titre{Produits infinis, Polytechnique 2000}
\theme{Exercices de Michel Quercia, Séries numérique}
\auteur{quercia}
\date{2010/03/14}
\organisation{exo7}
\contenu{
  \texte{On considère une suite $(a_n)$ de réels et on définit $P_N=\prod_{n=1}^N(1+a_n)$
et $S_N=\sum_{n=1}^N a_n$.}
\begin{enumerate}
  \item \question{On suppose que pour tout $n$, $a_n\ge 0$. 
  \begin{enumerate}}
  \item \question{Montrer que, pour tout $N$, $1+S_N\le P_N \le e^{S_N}$.}
  \item \question{Comparer les convergences respectives des suites $(S_N)$ et $(P_N)$.}
\end{enumerate}
\begin{enumerate}
  \item \reponse{\begin{enumerate}}
  \item \reponse{$1+S_N\le P_N$ n'est plus triviale mais reste vraie par récurrence
    (la différence est une fonction décroissante de~$a_1$).}
\end{enumerate}
}