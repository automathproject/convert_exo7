\uuid{1899}
\titre{Exercice 1899}
\theme{}
\auteur{legall}
\date{2003/10/01}
\organisation{exo7}
\contenu{
  \texte{}
\begin{enumerate}
  \item \question{Soit $A\in M_n(\Cc )$. Montrer quil existe une suite de 
matrices $(A_n)_{n\in \Nn }$
inversibles convergeant vers $A$ (en un sens que l'on pr\'ecisera).}
  \item \question{Soit $N\in M_n(\Cc )$ une matrice nilpotente. Calculer les 
valeurs propres de $N$. Montrer
que $\hbox{det}(I+N)=1.$}
  \item \question{Soit $A\in M_n(\Cc )$ telle que $AN=NA$. Calculer $\hbox{det}(A+N).$}
\end{enumerate}
\begin{enumerate}

\end{enumerate}
}