\uuid{nr1V}
\exo7id{3770}
\titre{$(X^2-1)P'' + (2X+1)P'$}
\theme{Exercices de Michel Quercia, Endomorphismes auto-adjoints}
\auteur{quercia}
\date{2010/03/11}
\organisation{exo7}
\contenu{
  \texte{Pour $P,Q \in \R_n[X]$ on pose
$(P\mid Q) =  \int_{t=-1}^1 \sqrt{\frac{1-t}{1+t}}P(t)Q(t)\,d t$
et $\Phi(P) = (X^2-1)P'' + (2X+1)P'$.}
\begin{enumerate}
  \item \question{Vérifier que $(P\mid Q)$ existe et qu'on définit ainsi un produit scalaire
    sur $\R_n[X]$.}
  \item \question{Montrer que pour ce produit scalaire, $\Phi$ est auto-adjoint
    $\Bigl($calculer $ \int_{t=-1}^1 (1-t)^{3/2}(1+t)^{1/2}P''(t)Q(t)\,d t$
    par parties$\Bigr)$.}
  \item \question{Déterminer les valeurs propres de $\Phi$ et montrer qu'il existe une
    base propre de degrés étagés.}
\end{enumerate}
\begin{enumerate}
  \item \reponse{$\lambda_k = k(k+1)$.}
\end{enumerate}
}