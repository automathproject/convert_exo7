\uuid{gaor}
\exo7id{4206}
\titre{$y\frac{\partial f}{\partial x} - x\frac{\partial f}{\partial y} = 2f$}
\theme{Exercices de Michel Quercia, \'Equations aux dérivées partielles}
\auteur{quercia}
\date{2010/03/11}
\organisation{exo7}
\contenu{
  \texte{Soit $f : U \to \R$ de classe $\mathcal{C}^1$ vérifiant :
$y\frac{\partial f}{\partial x} - x\frac{\partial f}{\partial y} = 2f$. où $U$ est un ouvert de ${\R^2}$.

On pose $g(\rho,\theta) = f(\rho\cos\theta,\rho\sin\theta)$.
Calculer $\frac{\partial g}{\partial \rho}$, $\frac{\partial g}{\partial \theta}$, puis trouver $f$ \dots}
\begin{enumerate}
  \item \question{Si $U = \{(x,y) \text{ tq } x > 0\}$.}
  \item \question{Si $U = {\R^2}$.}
\end{enumerate}
\begin{enumerate}
  \item \reponse{$g(\rho,\theta) = \lambda(\rho)e^{-2\theta}$.}
  \item \reponse{$g$ est $2\pi$-périodique, donc $\lambda = 0$, et $f = 0$.}
\end{enumerate}
}