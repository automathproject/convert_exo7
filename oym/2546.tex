\uuid{2546}
\titre{Exercice 2546}
\theme{Théorème des fonctions implicites}
\auteur{tahani}
\date{2009/04/01}
\organisation{exo7}
\contenu{
  \texte{}
  \question{Montrer que l'\'equation
$e^x+e^y+x+y-2=0$ d\'efinit, au voisinage de l'origine, une
fonction implicite $\varphi$ de $x$ dont on calculera le
d\'eveloppement limit\'e d'ordre trois en $0$.}
  \reponse{}
}