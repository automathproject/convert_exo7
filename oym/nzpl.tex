\uuid{nzpl}
\exo7id{4086}
\titre{Centrale MP 2004}
\theme{Exercices de Michel Quercia, \'Equations différentielles linéaires (I)}
\auteur{quercia}
\date{2010/03/11}
\organisation{exo7}
\contenu{
  \texte{On considère l'équation différentielle~: $-y'' + \frac y{p^2} = f$
où $p\in\N^*$ et $f$ est une fonction continue donnée.}
\begin{enumerate}
  \item \question{Donner les solutions de cette équation.
    Montrer que $x \mapsto - \int_{t=0}^x pf(t)\sh\Bigl(\frac{x-t}p\Bigr)\,d t$
    est solution.}
  \item \question{Montrer qu'il existe une unique solution telle que $y(0) = y(1) = 0$.
    On la note $u_p$.}
  \item \question{Montrer que $(u_p)$ converge simplement vers une fonction que l'on déterminera.}
\end{enumerate}
\begin{enumerate}
  \item \reponse{$u_p(x) = - \int_{t=0}^x pf(t)\sh\Bigl(\frac{x-t}p\Bigr)\,d t + \frac{\sh(x/p)}{\sh(1/p)} \int_{t=0}^1 pf(t)\sh\Bigl(\frac{1-t}p\Bigr)\,d t$.}
  \item \reponse{TCD : lorsque $p\to\infty$ $u_p(x)\to - \int_{t=0}^x (x-t)f(t)\,d t + x \int_{t=0}^1 (1-t)f(t)\,d t =  \int_{t=0}^x t(1-x)f(t)\,d t +  \int_{t=x}^1 x(1-t)f(t)\,d t$
    (primitive deuxième de~$-f$ s'annulant en~$0$ et~$1$).}
\end{enumerate}
}