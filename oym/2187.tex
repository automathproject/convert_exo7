\uuid{2187}
\titre{Exercice 2187}
\theme{Action de groupe}
\auteur{debes}
\date{2008/02/12}
\organisation{exo7}
\contenu{
  \texte{}
  \question{D\'ecrire le groupe $D_n$ des isom\'etries du plan affine
euclidien qui laissent invariant un
polygone r\'egulier \`a $n$ c\^ot\'es. Montrer que $D_n$ est engendr\'e
par deux \'el\'ements $\sigma $ et $\tau
$ qui v\'erifient les relations: $\sigma ^ n =1$, $\tau ^2=1$ et $\tau
\sigma \tau ^{-1} =\sigma
^{-1}$.  Quel est l'ordre de $D_n$? D\'eterminer le centre de $D_n$.
Montrer que $D_3 \simeq S_3$.}
  \reponse{}
}