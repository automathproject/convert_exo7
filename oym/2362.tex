\uuid{2362}
\titre{Exercice 2362}
\theme{Continuité}
\auteur{mayer}
\date{2003/10/01}
\organisation{exo7}
\contenu{
  \texte{}
  \question{Soient $E$ et $F$ deux espaces norm\'es et $L:E\to F$ une application lin\'eaire
v\'erifiant:
\emph{$(L(x_n))_n$ est born\'ee dans $F$ pour toute suite $(x_n)_n$ de $E$
tendant vers $0\in E$.}
Montrer que $L$ est continue.}
  \reponse{Comme $L$ est linéaire il suffit de montrer que $L$ est continue en $0$.
Supposons que cela ne soit pas vrai, alors il faut nier la continuité de $L$ en $0$ qui s'écrit :
$$\forall \epsilon >0 \qquad \exists \eta >0 \qquad \forall x\in E  \qquad(\|x\|<\eta \Rightarrow \|L(x)\|<\epsilon).$$
La négation s'écrit alors :
$$\exists \epsilon >0 \qquad \forall \eta >0 \qquad \exists x \in E \qquad (\|x\|<\eta \ \text{ et }\  \|L(x)\| \ge \epsilon).$$
Soit donc un tel $\epsilon >0$ de la négation, pour $\eta$ de la forme
$\eta=\frac1n$, on obtient $y_n$ tel que $\|y_n\|<\frac1n$ et $\|L(y_n)\|\ge \epsilon$.
On pose $x_n = \sqrt n y_n$, alors $\|x_n\| = \sqrt n \|y_n\| < \frac 1 {\sqrt n}$ donc $(x_n)$ est une suite de $E$ qui tend vers $0$. 
Par contre $\| L(x_n) \| = \sqrt n \|L(y_n)\| \ge \epsilon \sqrt n$, donc
la suite $(L(x_n))$ n'est pas bornée.
Par contraposition nous avons obtenu le résultat souhaité.}
}