\uuid{3052}
\titre{Parties d{\'e}nombrables}
\theme{Exercices de Michel Quercia, Propriétés de $\Nn$}
\auteur{quercia}
\date{2010/03/08}
\organisation{exo7}
\contenu{
  \texte{}
  \question{Soit $(n_k)$ une suite d'entiers naturels. On dit que la suite est :

\indent\vbox{\halign{&#\hfil\cr
- presque nulle &s'il existe $p \in \N$ tq $\forall\ k \ge p,\ n_k = 0$\cr
- stationnaire  &s'il existe $p \in \N$ tq $\forall\ k \ge p,\ n_k = n_p$.\cr
}}

Montrer que les ensembles des suites presque nulles et des suites stationnaires sont
d{\'e}nombrables.}
  \reponse{}
}