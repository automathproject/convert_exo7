\uuid{Nnvd}
\exo7id{4636}
\titre{$\sum_{n=1}^\infty \frac{\sin n}n = \sum_{n=1}^\infty \frac{\sin^2 n}{n^2}$}
\theme{Exercices de Michel Quercia, Séries de Fourier}
\auteur{quercia}
\date{2010/03/14}
\organisation{exo7}
\contenu{
  \texte{}
\begin{enumerate}
  \item \question{Développer en série de Fourier la fonction $f$, $2\pi$-périodique
    telle que $f(x) = \frac{\pi-x}2$ pour $0 \le x < 2\pi$.}
  \item \question{Donner les développements en série de Fourier de $f(x+1)$ et $f(x-1)$.}
  \item \question{Montrer que $\sum_{n=1}^\infty \frac{\sin n}n
    = \sum_{n=1}^\infty \frac{\sin^2 n}{n^2}$.}
\end{enumerate}
\begin{enumerate}
  \item \reponse{$f(x) = \sum_{n=1}^\infty \frac{\sin nx}n$.}
  \item \reponse{Le premier membre vaut $f(1) = \frac{\pi-1}2$ et le second
             $\frac1{4\pi} \int_{t=0}^{2\pi} (f(t+1)-f(t-1))^2\,d t
             = \frac{\pi-1}2$.}
\end{enumerate}
}