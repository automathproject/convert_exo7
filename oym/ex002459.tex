\uuid{2459}
\titre{$\mathbb C$ isomorphe \`a un sous-espace de $\mathcal M_2 (\mathbb{R})$}
\theme{}
\auteur{matexo1}
\date{2002/02/01}
\organisation{exo7}
\contenu{
  \texte{Soit $E$ l'espace vectoriel des matrices carr\'ees r\'eelles
d'ordre 2.}
\begin{enumerate}
  \item \question{Montrer que les ``vecteurs"
$$ I = \left( \begin{array}{cc} 1&0\\ 0&1 \end{array} \right), \,
J = \left( \begin{array}{cc} 0&1 \\ -1&0 \end{array} \right), \,
K = \left( \begin{array}{cc} 0&1 \\ 1&0 \end{array} \right), \,
L = \left( \begin{array}{cc} 1&0 \\ 0&-1 \end{array} \right)$$
de $E$ sont lin\'eairement ind\'ependants.}
  \item \question{Montrer que tout \'el\'ement $X = \left( \begin{array}{cc} a&b\\ c&d \end{array} \right)$
de $E$ s'\'ecrit de fa\c con unique sous la forme $X = x_1 I +
x_2 J + x_3 K + x_4 L$ et calculer $x_1, x_2, x_3, x_4$ en
fonction de $a,b,c,d$.}
  \item \question{V\'erifier la relation $J^2 = -I$. Calculer $JX$ et $XJ$.
Montrer que l'\'equation $XJ = JX$ est \'equivalente \`a $x_3 =
x_4 = 0$. En d\'eduire que le sous-espace de $E$ engendr\'e par $I,
J$ est isomorphe au corps des complexes $\mathbb C$.}
\end{enumerate}
\begin{enumerate}

\end{enumerate}
}