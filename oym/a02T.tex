\uuid{a02T}
\exo7id{2518}
\titre{Exercice 2518}
\theme{Théorème des accroissements finis}
\auteur{queffelec}
\date{2009/04/01}
\organisation{exo7}
\contenu{
  \texte{}
\begin{enumerate}
  \item \question{Soit $f$ une application r\'eelle continue et d\'erivable
sur $]a,b[$ telle que $f'(x)$ ait une limite quand
$x\buildrel{<}\over{\to}b$; alors $f$ se prolonge en une fonction
continue et d\'erivable \`a gauche au point $b$.}
  \item \question{Soit $f$ une application continue et d\'erivable sur un
intervalle $I\subset\Rr$, et de d\'eriv\'ee croissante; montrer
que $f$ est convexe sur $I$ i.e. $f((1-t)x+ty)\leq
(1-t)f(x)+tf(y)$ pour tous $x<y$ de $I$ et $t\in[0,1]$. (Poser
$z=(1-t)x+ty$ et appliquer les AF \`a $[x,z]$ puis $[z,y]$.)}
\end{enumerate}
\begin{enumerate}

\end{enumerate}
}