\uuid{56aM}
\exo7id{3015}
\titre{Relation d'{\'e}quivalence compatible avec les op{\'e}rations d'anneau}
\theme{Exercices de Michel Quercia, Anneaux}
\auteur{quercia}
\date{2010/03/08}
\organisation{exo7}
\contenu{
  \texte{Soit $A$ un anneau commutatif.}
\begin{enumerate}
  \item \question{Soit $\cal R$ une relation d'{\'e}quivalence compatible
    avec l'addition et la multiplication dans $A$. On note $I$ la classe de $0$.
    Montrer que $I$ est un id{\'e}al de $A$.}
  \item \question{R{\'e}ciproquement, soit $J$ un id{\'e}al de $A$.
    On pose $x \sim y \iff x-y \in J$.
    Montrer que $\sim$ est une relation d'{\'e}quivalence compatible avec $+$ et
    $\times$.}
\end{enumerate}
\begin{enumerate}

\end{enumerate}
}