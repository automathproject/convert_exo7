\uuid{4386}
\titre{Ensi Chimie P 93}
\theme{Exercices de Michel Quercia, Intégrale multiple}
\auteur{quercia}
\date{2010/03/12}
\organisation{exo7}
\contenu{
  \texte{}
  \question{Soit $I =  \int_0^1\frac{\ln(1+x)}{1+x^2}\,d x$.\par
En calculant $J=\iint_D\frac{x\,d x\,d y}{(1+x^2)(1+xy)}$ avec
$D = \{(x,y) \text{ tel que } 0\le x \le 1,\ 0\le y\le 1\}$ de deux façons différentes,
trouver $I$.}
  \reponse{$I=\frac{\pi\ln2}8$.}
}