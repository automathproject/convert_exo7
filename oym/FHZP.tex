\uuid{FHZP}
\exo7id{3340}
\titre{\'Eléments réguliers dans $\mathcal{L}(E)$}
\theme{Exercices de Michel Quercia, Applications linéaires en dimension finie}
\auteur{quercia}
\date{2010/03/09}
\organisation{exo7}
\contenu{
  \texte{Soit $f \in \mathcal{L}(E,F)$.}
\begin{enumerate}
  \item \question{Montrer que :  ($f$ est \rlap{injectif)}\phantom{surjectif)}
    $\iff$ ($\forall\ g \in \mathcal{L}(E),\ f\circ g = 0  \Rightarrow  g = 0$).}
  \item \question{Montrer que :  ($f$ est surjectif)
    $\iff$ ($\forall\ g \in \mathcal{L}(F),\ g\circ f = 0  \Rightarrow  g = 0$).}
\end{enumerate}
\begin{enumerate}

\end{enumerate}
}