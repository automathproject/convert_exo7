\uuid{4942}
\titre{Sphères équidistantes d'une sphère et d'un plan}
\theme{Exercices de Michel Quercia, Quadriques}
\auteur{quercia}
\date{2010/03/17}
\organisation{exo7}
\contenu{
  \texte{}
  \question{Dans l'espace, on considère un plan $P$ et une sphère $S$.
Quel est le lieu des centres des sphères tangentes à $S$ et à $P$ ?}
  \reponse{Soit $r$ le rayon de $S$ et $h$ la distance du centre $I$ de $S$ à $P$.
         On choisit un repère tel que $P = Oxy$ et $I = (0,0,h)$.
         Soit $S'$ une sphère de centre $M(x,y,z)$ :

{Pour $z > 0$ : }
         $S$ et $S'$ extérieures $\Leftrightarrow 2(h+r)z = h^2-r^2+x^2+y^2$.\par
         $S$ à l'intérieur de $S'$ $\Leftrightarrow 2(h-r)z = h^2-r^2+x^2+y^2$ si $h > r$.\par
         $S'$ à l'intérieur de $S$ $\Leftrightarrow 2(h-r)z = h^2-r^2+x^2+y^2$ si $h < r$.\par

{Pour $z < 0$ : }
         $S$ et $S'$ extérieures $\Leftrightarrow 2(h-r)z = h^2-r^2+x^2+y^2$ si $h < r$.\par
	 $S'$ à l'intérieur de $S$ $\Leftrightarrow 2(h+r)z = h^2-r^2+x^2+y^2$ si $h < r$.\par}
}