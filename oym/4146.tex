\uuid{4146}
\titre{Contre-exemple au théorème de Schwarz}
\theme{Exercices de Michel Quercia, Dérivées partielles}
\auteur{quercia}
\date{2010/03/11}
\organisation{exo7}
\contenu{
  \texte{}
  \question{Soit $f:\R \to \R$ une fonction $\pi$-périodique de classe $\mathcal{C}^2$.
On pose pour $(x,y) \in \R^2$ :
$g(x,y) = r^2f(\theta)$ avec $(x,y)=(r\cos\theta,r\sin\theta)$.
Calculer $\frac{\partial g}{\partial x}(0,y)$ et $\frac{\partial g}{\partial y}(x,0)$ en fonction de $f$.
En déduire les valeurs de $\frac{\partial^2 g}{\partial y \partial x}(0,0)$ et $\frac{\partial^2 g}{\partial x \partial y}(0,0)$.
Construire un exemple précis (donner $g(x,y)$ en fonction de $x$ et $y$)
pour lequel ces deux dérivées sont distinctes.}
  \reponse{$\frac{\partial g}{\partial x}(0,y) = - yf'(\pi/2)$, $\frac{\partial  g}{\partial y}(x,0) = xf'(0)$.}
}