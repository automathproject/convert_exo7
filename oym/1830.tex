\uuid{1830}
\titre{Exercice 1830}
\theme{}
\auteur{ridde}
\date{1999/11/01}
\organisation{exo7}
\contenu{
  \texte{}
  \question{Soit $f (x, y) = x^{2} + xy + y^{2}-3x-6y$. Montrer que $f$ admet au plus un extremum.
Ecrire $f (x, y) + 9$ comme la somme de deux carr\'es et en d\'eduire que $f$ admet $-9$
comme valeur minimale.}
  \reponse{}
}