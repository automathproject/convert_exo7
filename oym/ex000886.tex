\uuid{886}
\titre{Exercice 886}
\theme{}
\auteur{legall}
\date{1998/09/01}
\organisation{exo7}
\contenu{
  \texte{}
  \question{D\' eterminer lesquels des
ensembles $E_1$, $E_2$, $E_3$ et $E_4$ sont des sous-espaces
vectoriels de ${\Rr}^3$. 

 $E_1 =\{ (x,y,z)\in {\Rr}^3\ \mid \ 3x-7y = z \} $ 

 $E_2 =\{(x,y,z)\in {\Rr}^3\ \mid \ x^2-z^2=0 \} $  

 $E_3=\{ (x,y,z)\in {\Rr}^3\ \mid \ x+y-z=x+y+z=0 \} $ 

 $E_4 =\{ (x,y,z)\in {\Rr}^3\ \mid \ z(x^2+y^2)=0 \} $}
  \reponse{\begin{enumerate}
  \item 

    \begin{enumerate}
    \item $(0,0,0) \in E_1$.
    \item Soient $(x,y,z)$ et $(x',y',z')$ deux \' el\' ements
de $E_1$. On a donc $3x-7y=z$ et $3x'-7y'=z'$. Donc $3(x+x')-7(y+y')=(z+z')$, d'où $(x+x',y+y',z+z')$  
appartient \`a  $E_1$.

\item Soit $\lambda \in {\R}$ et $(x,y,z)\in E_1$.  Alors la relation $3x-7y=z$ implique
  que $3 (\lambda x) -7(\lambda y)=\lambda z$
  donc que $\lambda(x,y,z)=(\lambda x,\lambda  y,\lambda  z)$ appartient \`a  $E_1$.
    \end{enumerate}


  
  \item $E_2 =\{ (x,y,z)\in {\R}^3 \mid x^2-z^2=0 \} $ c'est-\`a-dire $E_2
    =\{ (x,y,z)\in {\R}^3 \mid x=z \hbox{ ou } x=-z \} $. Donc
    $(1, 0,-1)$ et $(1, 0, 1)$ appartiennent \`a $ E_2$ mais
    $(1, 0,-1)+(1, 0, 1)=(2, 0, 0)$ n'appartient pas \`a $ E_2$ qui n'est en cons\' equence pas un
sous-espace vectoriel de ${\R}^3$.



\item $E_3$ est un sous-espace vectoriel de ${\R}^3$. En effet :
    \begin{enumerate}
    \item $(0, 0, 0) \in E_3$.
    \item Soient $(x, y, z)$ et
      $(x', y', z')$ deux \' el\'ements
      de $E_3$. On a donc $x+y-z=x+y+z=0$ et $x'+y'-z'=x'+y'+z'=0$.
      Donc $(x+x')+(y+y')-(z+z')= (x+x')+(y+y')+(z+z')=0$ et
      $(x, y, z)+(x', y', z')=(x+x',y+y',z+z')$  appartient \`a  $E_3$.
\item Soit $\lambda \in {\R}$ et $(x,y,z)\in E_3$.  Alors la relation $x+y-z=x+y+z=0$ implique
  que $\lambda x+\lambda y-\lambda z=\lambda x+\lambda y+\lambda z=0$
  donc que $\lambda (x, y, z)=(\lambda x,\lambda  y,\lambda  z)$ appartient \`a  $E_3$.
    \end{enumerate}


\item Les vecteurs $(1, 0, 0)$ et
  $(0, 0, 1)$ appartiennent \`a $ E_4$
  mais leur somme $(1, 0, 0)+(0, 0, 1)=(1, 0, 1)$  ne lui
appartient pas donc $E_4$ n'est pas un sous-espace vectoriel de
${\R}^3$.
\end{enumerate}}
}