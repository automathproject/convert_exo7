\uuid{6975}
\titre{Exercice 6975}
\theme{Fonctions circulaires et hyperboliques inverses, Fonctions hyperboliques}
\auteur{blanc-centi}
\date{2014/05/06}
\organisation{exo7}
\contenu{
  \texte{}
  \question{Simplifier l'expression $\displaystyle\frac{2\ch^2(x)-\sh(2x)}{x-\ln(\ch x)-\ln 2}$ 
et donner ses limites en $-\infty$ et $+\infty$.}
  \reponse{Par définition des fonctions $\ch$ et $\sh$, on a 
\begin{eqnarray*}
2\ch^2(x)-\sh(2x)&=&2\left(\frac{e^x+e^{-x}}{2}\right)^2-\frac{e^{2x}-e^{-2x}}{2}\\
 &=&\frac{e^{2x}+2+e^{-2x}}{2}+\frac{e^{-2x}-e^{2x}}{2}\\
 &=&1+e^{-2x}
\end{eqnarray*}

Et en utilisant les deux relations $\ln(ab)=\ln a + \ln b$ et $\ln(e^x) = x$ on calcule :
\begin{eqnarray*}
x-\ln(\ch x)-\ln 2
  &=& x-\ln\left(\frac{e^x+e^{-x}}{2}\right)-\ln 2\\
  &=& x-\ln(e^x+e^{-x}) + \ln 2 - \ln 2\\
  &=& x-\ln\big(e^x(1+e^{-2x})\big)\\
  &=& x-\ln(e^x)-\ln(1+e^{-2x})\\
  &=& x-x-\ln(1+e^{-2x})\\
  &=& -\ln(1+e^{-2x})\\
\end{eqnarray*}
d'où 
$$\frac{2\ch^2(x)-\sh(2x)}{x-\ln(\ch x)-\ln 2}=-\frac{1+e^{-2x}}{\ln(1+e^{-2x})}$$
C'est une expression de la forme $-\frac{u}{\ln u}$ avec $u=1+e^{-2x}$ :
\begin{itemize}
  \item si $x\to +\infty$, alors $u\to 1^+$, $\frac{1}{\ln u} \to +\infty$ 
  donc $-\frac{u}{\ln u}\to -\infty$ ;
  \item si $x\to -\infty$, alors $u\to +\infty$ donc d'après les 
  relations de croissances comparées, $-\frac{u}{\ln u}\to -\infty$.
\end{itemize}}
}