\uuid{7899}
\titre{Sylow des groupes diédraux}
\theme{}
\auteur{mourougane}
\date{2021/08/11}
\organisation{exo7}
\contenu{
  \texte{Soit $\mathcal{P}_n$ un polygone régulier à $n$ côtés dans le plan euclidien orienté.
On appelle groupe diédral $D_{n}$ le groupe des isométries de $\mathcal{P}_n$.}
\begin{enumerate}
  \item \question{Parmi les translations, les rotations, les symétries orthogonales, et les symétries glissées (composées d'une symétrie orthogonale et d'une translation dans l'axe de la symétrie), décrire des isométries du plan qui conservent le polygone régulier $\mathcal{P}_n$.}
  \item \question{Déterminer, à l'aide de l'action naturelle de $D_{n}$ sur l'ensemble des sommets de $\mathcal{P}_n$, le cardinal de $D_{n}$. En déduire la liste complète des éléments de $D_{n}$.}
  \item \question{On suppose $n$ impair. Déterminer les $2$-Sylow de $D_{n}$ et vérifier (sans référence au cours) qu'ils sont conjugués.}
  \item \question{On suppose $n=6$. Déterminer un $2$-Sylow de $D_{6}$.
Déterminer le nombre de $2$-Sylow de $D_{6}$.
Déterminer deux sous-groupes d'ordre $2$ de $D_{6}$ non conjugués dans $D_{6}$.
Donner un $3$-Sylow de $D_{6}$.}
\end{enumerate}
\begin{enumerate}

\end{enumerate}
}