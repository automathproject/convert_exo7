\uuid{4035}
\titre{Dérivation d'un DL d'ordre 2}
\theme{Exercices de Michel Quercia, Développements limités théoriques}
\auteur{quercia}
\date{2010/03/11}
\organisation{exo7}
\contenu{
  \texte{}
  \question{Soit $f : \R \to \R$ convexe dérivable telle que
$f(a+h) = f(a) + hf'(a) + o (h^2)$.

Démontrer que $f$ est deux fois dérivable en $a$ et $f''(a) = 0$
(comparer $f'(a+h)$ aux taux d'accroissement de $f$ entre $a$ et $a+h$,
et entre $a+h$ et $a+2h$).

\'Etudier le cas où $f(a+h) = f(a) + hf'(a) + \frac {Lh^2}2 + o (h^2)$.}
  \reponse{}
}