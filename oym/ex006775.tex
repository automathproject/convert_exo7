\uuid{6775}
\titre{Exercice 6775}
\theme{}
\auteur{gijs}
\date{2011/10/16}
\organisation{exo7}
\contenu{
  \texte{}
\begin{enumerate}
  \item \question{(D'abord un cas particulier) Dans $\Rr^3$ on
considère les objets suivants~: le demi-espace $D =
\{\,(x,y,z)\in \Rr^3 \mid z<1\,\}$, le plan $P =
\{\,(x,y,z)\in \Rr^3 \mid z=0\,\}$, la sphère
unité $S^2 = \{\, (x,y,z) \in \Rr^3 \mid x^2 + y^2 +
z^2 - 1 = 0\,\}$, et le pole nord $N = (0,0,1)$. 

\begin{enumerate}}
  \item \question{On définit une application $p: D \to P$ par la
procédure suivante~: pour un point $A \in D$,
la droite dans $\Rr^3$ qui passe par $A$ et $N$
coupe le plan $P$ en $p(A)$. Trouver
l'expression explicite de l'application $p$ et en
déduire qu'elle est continue.}
  \item \question{On définit une application $i : P \to S^2$ par
la procédure suivante~: pour $B\in P$, la droite
dans $\Rr^3$ qui passe par $B$ et $N$
coupe la sphère unité $S^2$ en $i(B)$. Trouver
l'expression explicite de l'application $i$ et en
déduire qu'elle est continue.}
  \item \question{En utilisant les applications $p$ et $i$,
montrer que $P$ est homéomorphe à $S^2 \setminus
\{N\}$.}
  \item \question{Montrer que $S^2$ est compacte.}
\end{enumerate}
\begin{enumerate}

\end{enumerate}
}