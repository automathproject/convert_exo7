\uuid{LOGC}
\exo7id{4626}
\titre{Usage d'une série entière}
\theme{Exercices de Michel Quercia, Séries de Fourier}
\auteur{quercia}
\date{2010/03/14}
\organisation{exo7}
\contenu{
  \texte{}
\begin{enumerate}
  \item \question{Existe-t-il une fonction $f:\R \to \R$ continue telle que les coefficients
    de Fourier soient : $a_n = \frac1{2^n}$ et $b_n = 0$~?}
  \item \question{Application : calculer $ \int_{t=0}^\pi \frac{d t}{5-4\cos t}$.}
\end{enumerate}
\begin{enumerate}
  \item \reponse{$f(x) = \frac 12 + \sum_{n=1}^\infty \frac{\cos nx}{2^n}
              = \frac3{2(5-4\cos x)}$.}
  \item \reponse{$\frac\pi3$.}
\end{enumerate}
}