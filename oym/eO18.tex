\uuid{eO18}
\exo7id{3665}
\titre{Calcul de distance}
\theme{Exercices de Michel Quercia, Produit scalaire}
\auteur{quercia}
\date{2010/03/11}
\organisation{exo7}
\contenu{
  \texte{On munit $E = \R_n[X]$ du produit scalaire :
Pour $P = \sum_i a_iX^i$ et $Q = \sum_i b_iX^i$,
$(P\mid Q) = \sum_i a_ib_i$.

Soit $H = \{ P \in E \text{ tq } P(1) = 0 \}$.}
\begin{enumerate}
  \item \question{Trouver une base orthonormale de $H$.}
  \item \question{Calculer $d(X,H)$.}
\end{enumerate}
\begin{enumerate}
  \item \reponse{$\frac1{\sqrt{n+1}}$.}
\end{enumerate}
}