\uuid{519}
\titre{Exercice 519}
\theme{}
\auteur{ridde}
\date{1999/11/01}
\organisation{exo7}
\contenu{
  \texte{}
  \question{Montrer qu'une suite d'entiers qui converge est
constante \`a partir d'un certain rang.}
  \reponse{Soit $(u_n)$ une suite d'entiers qui converge vers $\ell \in
\Rr$.
Dans l'intervalle $I = ] \ell - \frac12, \ell +\frac12[$ de
longueur $1$, il existe au plus un \'el\'ement de $\Nn$. Donc $I \cap
\Nn$ est soit vide soit un singleton $\{a \}$.

La convergence de  $(u_n)$ s'\'ecrit :
$$ \forall \epsilon > 0 \ \ \exists N \in \Nn \text{\  \ tel que\ \  }
(n \geqslant N \Rightarrow |u_n-\ell| < \epsilon).$$ Fixons $\epsilon =
\frac 12$, nous obtenons un $N$ correspondant. Et pour $n \geqslant N$,
$u_n \in I$. Mais de plus $u_n$ est un entier, donc
  $$ n \geqslant N \Rightarrow u_n \in I \cap \Nn.$$
En cons\'equent, $I\cap \Nn$ n'est pas vide (par exemple $u_N$ en
est un \'el\'ement) donc $I \cap \Nn = \{ a \}$. L'implication
pr\'ec\'edente s'\'ecrit maintenant :
$$ n \geqslant N \Rightarrow u_n = a.$$
Donc la suite $(u_n)$ est stationnaire (au moins) \`a partir de
$N$. En prime, elle est bien \'evidemment convergente vers $\ell = a
\in \Nn$.}
}