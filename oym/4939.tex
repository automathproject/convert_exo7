\uuid{4939}
\titre{$MF = ed(M,D)$}
\theme{Exercices de Michel Quercia, Quadriques}
\auteur{quercia}
\date{2010/03/17}
\organisation{exo7}
\contenu{
  \texte{}
  \question{Dans l'espace, on considère un point $F$, une droite $D$ ne passant pas par $F$
et un réel $e > 0$. Montrer que l'ensemble, ${\cal S}$, des points $M$ tels que
$MF = ed(M,D)$ est une quadrique. Préciser les différents cas possibles.}
  \reponse{$e<1$ : ellipsoïde de révolution, $e = 1$ : cylindre parabolique,
         $e>1$ : hyperboloïde de révolution.}
}