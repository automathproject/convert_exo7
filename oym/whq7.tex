\uuid{whq7}
\exo7id{3314}
\titre{Permutation de coordonnées dans $ K^n$}
\theme{Exercices de Michel Quercia, Applications linéaires}
\auteur{quercia}
\date{2010/03/09}
\organisation{exo7}
\contenu{
  \texte{Soit $\sigma \in S_n$ (groupe symétrique) et
${f_\sigma} : { K^n} \to { K^n}, {(x_1,\dots x_n)} \mapsto
              {(x_{\sigma(1)},\dots,x_{\sigma(n)})}$

On munit $ K^n$ de la structure d'algèbre pour les opérations
composante par composante.}
\begin{enumerate}
  \item \question{Montrer que $f_{\sigma}$ est un automorphisme d'algèbre.}
  \item \question{Soit $\varphi$ un automorphisme d'algèbre de $ K^n$.
  \begin{enumerate}}
  \item \question{Montrer que la base canonique de $ K^n$ est invariante par $\varphi$
         (étudier $\varphi(e_i^2)$ et $\varphi(e_i\times e_j)$).}
  \item \question{En déduire qu'il existe $\sigma \in S_n$ tel que $\varphi = f_\sigma$.}
\end{enumerate}
\begin{enumerate}

\end{enumerate}
}