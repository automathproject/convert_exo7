\exo7id{2631}
\titre{Exercice 2631}
\theme{}
\auteur{debievre}
\date{2009/05/19}
\organisation{exo7}
\contenu{
  \texte{Soit $C$ le c\^one d'\'equation $z^2= x^2+y^2$
et $C^+$ le demi-c\^one o\`u $z \geq 0$. Pour un 
point quelconque $M_0$ de  $C\setminus \{(0,0,0)\}$, de
coordonn\'ees $(x_0,y_0,\pm \sqrt{x_0^2+y_0^2})$, 
on note ${\cal P}_{M_0}$ le plan tangent au c\^one $C$ en $M_0$.}
\begin{enumerate}
  \item \question{D\'eterminer un vecteur normal et l'\'equation du plan ${\cal P}_{M_0}$.}
  \item \question{Montrer que l'intersection du c\^one $C$ avec 
le plan vertical d'\'equation ${y=ax}$ o\`u $a\in \R$ est constitu\'ee
de deux droites ${\cal D}_1$ et ${\cal D}_2$
et que 
 l'intersection du demi-c\^one $C^+$ avec 
ce plan vertical  est constitu\'ee
de deux demi-droites ${\cal D}^+_1$ et ${\cal D}^+_2$.}
  \item \question{Montrer que le plan tangent au c\^one $C$ 
est le m\^eme en tout point de ${\cal D}_1\setminus \{(0,0,0)\}$ (respectivement  en tout point de
${\cal D}_2\setminus \{(0,0,0)\}$).}
\end{enumerate}
\begin{enumerate}
  \item \reponse{Le vecteur normal du c\^one $C$
au point $(x_0,y_0,z_0)$ de $C$
est le vecteur
$(x_0,y_0,-z_0)$ et le plan tangent au c\^one $C$
en ce point est donn\'e par l'\'equation
\[
x_0x + y_0y -z_0z =0
\]
car l'origine appartient \`a ce plan.}
  \item \reponse{L'intersection du c\^one $C$ avec 
le plan vertical d'\'equation $y=ax$ o\`u $a\in \R$ 
est constitu\'ee des points
$x(1,a, \pm \sqrt {1+a^2})$ o\`u $x \in \R$, c.a.d. des deux droites
\[
{\cal D}_1=\{x(1,a, \sqrt {1+a^2}); x \in \R\},
\quad 
{\cal D}_2=\{x(1,a, - \sqrt {1+a^2}); x \in \R\}.
\]
L'intersection du demi-c\^one $C^+$ avec 
ce plan vertical est donc constitu\'ee des 
deux demi-droites
\begin{align*}
{}&{\cal D}^+_1=\{x(1,a, \sqrt {1+a^2}); x \in \R, x \geq 0\}
\\
{}&{\cal D}^+_2=\{x(-1,-a, \sqrt {1+a^2}); x \in \R, x \geq 0\}.
\end{align*}}
  \item \reponse{Le vecteur normal en un point quelconque 
$x(1,a, \sqrt {1+a^2})$ de ${\cal D}_1$ 
respectivement $x(1,a, -\sqrt {1+a^2})$ de ${\cal D}_2$ 
est le vecteur
$x(1,a, -\sqrt {1+a^2})$ 
respectivement $x(1,a, \sqrt {1+a^2})$ 
d'o\`u 
la direction et donc
le plan tangent au c\^one $C$ 
sont le m\^eme en tout point de ${\cal D}_1\setminus \{(0,0,0)\}$ 
respectivement  ${\cal D}_2\setminus \{(0,0,0)\}$.}
\end{enumerate}
}