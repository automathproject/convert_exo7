\uuid{2228}
\titre{Exercice 2228}
\theme{}
\auteur{matos}
\date{2008/04/23}
\organisation{exo7}
\contenu{
  \texte{}
  \question{Soit $A\in\Rr^{n\times n}$ telle que $A^T$ soit \`a diagonale strictement dominante. Montrer que $A$ admet une d\'ecomposition LU avec $L^T$ \`a diagonale strictement dominante.}
  \reponse{$$A=A_1=\left(\begin{array}{cc}
\alpha & u^T\\v&B_1\end{array}\right),\quad B_1=(b_{ij})_{i,j=1}^{n-1}$$
$A^T$ \'etant \`a diagonale strictement dominante on a:
$$|\alpha|>\sum_{i=1}^{n-1}|v_i|,\quad |u_i| +\sum_{j\neq i}|b_{ji}|<|b_{ii}|$$
Il suffit de montrer que
\begin{itemize}
\item la premi\`ere colonne de $L$ v\'erifie $|l_{11}|>\sum_{i\neq 1}|l_{i1}|$
\item $B_2$ est telle que
$$A_2=\left(\begin{array}{cc}\alpha & u^T\\0& B_2\end{array}\right), \quad C=B_2=B_1-\frac{1}{\alpha} vu^T$$
v\'erifie $|c_{ii}|>\sum_{j\neq i}|c_{ji}|$ avec $C_{ij}=B_{ij}-\frac{1}{\alpha}v_iu_j$ et it\'erer.
\end{itemize}
\begin{itemize}
\item premi\`ere colonne de $L$: $l_{i1}=v_i/\alpha \Rightarrow \sum_{i=2}^n |l_{i1}|=\sum_{i=1}^{n-1}\frac{|v_i|}{\alpha}<1$
\item $\sum_{i\neq j}|c_{ij}| =\sum_{i\neq j}\left|b_{ij}-\frac{1}{\alpha}v_iw_j\right|\leq \sum_{i\neq j}|b_{ij}| +\frac{1}{|\alpha|}|w_j|\sum_{i\neq j}|v_i|$
$$\leq |b_{jj}| - |u_j| + \frac{1}{|\alpha|}|u_j| (|\alpha|-|v_j|)
\leq\left|b_{jj}-\frac{1}{\alpha} u_jv_j\right|=|c_{jj}|$$
donc $B_2^T$ est de diagonale strictement dominante. La d\'emonstration se finit par r\'ecurrence.
\end{itemize}}
}