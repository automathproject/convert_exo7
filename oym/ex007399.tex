\uuid{7399}
\titre{Exercice 7399}
\theme{}
\auteur{mourougane}
\date{2021/08/10}
\organisation{exo7}
\contenu{
  \texte{}
\begin{enumerate}
  \item \question{On considère le code binaire, linéaire engendré par la matrice
$$\left(\begin{array}{ccccccccccccccc}
1&0&0&1&1&0&0&0&0&0&0&0&0&0&0\\
0&1&0&0&1&1&0&0&0&0&0&0&0&0&0\\
0&0&1&0&0&1&1&0&0&0&0&0&0&0&0\\
0&0&0&1&0&0&1&1&0&0&0&0&0&0&0\\
0&0&0&0&1&0&0&1&1&0&0&0&0&0&0\\
0&0&0&0&0&1&0&0&1&1&0&0&0&0&0\\
0&0&0&0&0&0&1&0&0&1&1&0&0&0&0\\
0&0&0&0&0&0&0&1&0&0&1&1&0&0&0\\
0&0&0&0&0&0&0&0&1&0&0&1&1&0&0\\
0&0&0&0&0&0&0&0&0&1&0&0&1&1&0\\
0&0&0&0&0&0&0&0&0&0&1&0&0&1&1\\
\end{array}\right)$$
Quel est son alphabet ? sa longueur ? sa dimension ? un polynôme générateur ? son nombre de mots ?}
  \item \question{Le code est-il cyclique ?}
  \item \question{Ecrire une matrice de contrôle. Montrer que la distance du code est au moins $3$. Combien d'erreurs peut-on alors détecter ? combien d'erreurs peut-on alors corriger ?}
  \item \question{Le mot $(0,0,0,1,0,1,1,1,1,1,1,0,0,1,1)$
est-il un mot de code ? Si non, en supposant qu'il n'a qu'une erreur, écrire le mot de code dont il provient.}
\end{enumerate}
\begin{enumerate}
  \item \reponse{Soit $C$ le code donné. L'alphabet de $C$ est l'ensemble $\{0,1\}$. La longueur des mots est 15, alors la longueur de C est 15. 
La matrice génératrice donnée est exactement la matrice du code engendré par le polynôme $g=1+X^3+X^4.$ Puisque $g$ est un diviseur unitaire de $X^{15}-1$ de degré $4$, 
$$X^{15}-1=(X^4+X^3+1)(X^{11}+X^{10}+X^9+X^8+X^6+X^4+X^3+1)$$
le code engendré par $g$ est cyclique, de dimension $11=15-4$. Alors $C$ est cyclique de dimension 11 engendré par $g$. Il contient $2^{11}$ mots.}
  \item \reponse{Oui, voir au-dessus.}
  \item \reponse{On trouve le polynôme de contrôle: 
$$h=\frac{X^{15}-1}{g}=X^{11}+X^{10}+X^9+X^8+X^6+X^4+X^3+1$$
Alors la matrice de contrôle est
$$H=\left(\begin{array}{ccccccccccccccc}
1&1&1&1&0&1&0&1&1&0&0&1&0&0&0\\
0&1&1&1&1&0&1&0&1&1&0&0&1&0&0\\
0&0&1&1&1&1&0&1&0&1&1&0&0&1&0\\
0&0&0&1&1&1&1&0&1&0&1&1&0&0&1\\
\end{array}\right)$$
Deux colonnes quelconques de $H$ sont toujours linéairement indépendantes, donc la distance du code est au moins~ $3$. Puisque le code contient un mot de poids 3, alors la distance est en fait $3$. Alors $C$ peut corriger une erreur et en détecter deux.}
  \item \reponse{Un mot $m$ appartient à $C$ ssi $H\cdot \mathrm{tr} m=\vec{0}$. \\
Le produit de $H$ et $m=(0,0,0,1,0,1,1,1,1,1,1,0,0,1,1)$ est la colonne obtenue comme somme des colonnes étoilées
$$\left(\begin{array}{ccccccccccccccc}
0&0&0&*&0&*&*&*&*&*&*&0&0&*&*\\ \hline
1&1&1&1&0&1&0&1&1&0&0&1&0&0&0\\
0&1&1&1&1&0&1&0&1&1&0&0&1&0&0\\
0&0&1&1&1&1&0&1&0&1&1&0&0&1&0\\
0&0&0&1&1&1&1&0&1&0&1&1&0&0&1\\
\end{array}\right)$$
soit $(0,0,0,0)$. Alors $m$ est un mot de code.}
\end{enumerate}
}