\exo7id{6833}
\titre{Exercice 6833}
\theme{}
\auteur{gijs}
\date{2011/10/16}
\organisation{exo7}
\contenu{
  \texte{Dans cet exercice, $\mathcal{O}$ désigne un ouvert de
$\Rr^n$.}
\begin{enumerate}
  \item \question{Quand dit-on qu'une fonction $f:\mathcal{O} \to \Rr$
est de classe $C^k$ ($k\ge1$)~?}
  \item \question{Démontrer par récurrence sur $k$ que si $f
: \mathcal{O} \to \Rr$ et $g: \mathcal{O} \to \Rr\setminus\{0\}$ sont de classe
$C^k$, alors  $h:\mathcal{O} \to \Rr$, $h(x) =
f(x)/g(x)$ est aussi de classe $C^k$.}
  \item \question{Démontrer par récurrence sur $k$ que la
composée de deux fonctions de classe $C^k$ est aussi de
classe $C^k$.}
  \item \question{Soit $Gl(2,\Rr)$ l'ensemble des matrices à
coefficients réels $2\times2$ inversibles.
Dé\-mon\-trer que l'applica\-tion $I: Gl(2,\Rr) \to
Gl(2,\Rr)$, $I(\left(\smallmatrix a&b \\ c&d
\endsmallmatrix\right)) = \left(\smallmatrix a&b \\ c&d
\endsmallmatrix\right)^{-1}$ est de classe $C^k$ pour
tout $k\ge1$.}
  \item \question{\'Enoncer le théorème de l'inversion locale
pour une fonction $f:\mathcal{O} \to \Rr^n$.}
  \item \question{On se restreint au cas $n=2$ et on se place
dans les conditions du théorème de l'inversion
locale. Démontrer que si $f$ est de classe $C^k$,
$k>1$, alors la réciproque $f^{-1}$ (donnée par le
théorème de l'inversion locale) est  de classe
$C^k$.}
\end{enumerate}
\begin{enumerate}

\end{enumerate}
}