\uuid{4895}
\titre{Projections en cascade}
\theme{Exercices de Michel Quercia, Propriétés des triangles}
\auteur{quercia}
\date{2010/03/17}
\organisation{exo7}
\contenu{
  \texte{}
  \question{Soient $A,B,C$ trois points non alignés et $M_1 \in (AB)$.
On construit les points $M_2,M_3,M_4$ de la manière suivante~:

-- $M_2$ est le projeté de $M_1$ sur $(BC)$ parallèlement à $(AC)$.\par
-- $M_3$ est le projeté de $M_2$ sur $(AC)$ parallèlement à $(AB)$.\par
-- $M_4$ est le projeté de $M_3$ sur $(AB)$ parallèlement à $(BC)$.\par

On recommence ensuite les mêmes constructions à partir de $M_4$, ce qui
donne les points $M_5, M_6, M_7$.

Montrer que $M_7 = M_1$.}
  \reponse{$M_1 \mapsto M_4$ est affine, et échange $A$ et $B$. $\Rightarrow$ involutive.}
}