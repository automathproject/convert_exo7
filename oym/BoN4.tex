\uuid{BoN4}
\exo7id{7336}
\titre{Exercice 7336}
\theme{Exercices de Christophe Mourougane, Arithmétique 2}
\auteur{mourougane}
\date{2021/08/10}
\organisation{exo7}
\contenu{
  \texte{Considérons l'anneau $\Z[i\sqrt{3}]$, le sous-anneaux de $\C$ engendré par $i\sqrt{3}$.}
\begin{enumerate}
  \item \question{Montrer que $\Z[i\sqrt{3}]:=\{a+bi\sqrt{3}, \ (a,b)\in\Z^2\}.$}
  \item \question{\`A tout élément $x=a+bi\sqrt{3}$ de $\mathbf{Z}[i\sqrt{3}]$ on associe son conjugué $\overline x=a-bi\sqrt{3}$.
 Montrer que pour tous $x,y \in \mathbf{Z}[i\sqrt{3}]$ on a 
$$\overline{x+y}=\overline x + \overline y \textrm{ ~ et ~ } \overline{xy}=\overline x \overline y.$$}
  \item \question{En considérant l'application $N : \mathbf{Z}[i\sqrt{3}]\to \mathbf{Z}, x \mapsto x\overline x$, montrer que les éléments inversibles de $\mathbf{Z}[i\sqrt{3}]$ sont exactement les éléments de norme $1$. Donner la liste des éléments inversibles.}
  \item \question{Quelles sont les normes possibles d'un diviseur de $1+i\sqrt{3}$ dans $\mathbf{Z}[i\sqrt{3}]$ ?}
  \item \question{Montrer que éléments $2$, $1+i\sqrt{3}$ sont premiers entre eux dans $\mathbf{Z}[i\sqrt{3}]$.
Peut-on leur écrire un couple de Bézout ?}
  \item \question{Donner deux factorisations différentes de $4$ dans $\mathbf{Z}[i\sqrt{3}]$. 
Le lemme d'Euclide est-il valide dans l'anneau $\mathbf{Z}[\sqrt{-3}]$ ?}
\end{enumerate}
\begin{enumerate}

\end{enumerate}
}