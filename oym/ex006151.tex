\exo7id{6151}
\titre{Exercice 6151}
\theme{}
\auteur{queffelec}
\date{2011/10/16}
\organisation{exo7}
\contenu{
  \texte{Soit $P\in{\Cc}[X]$ un polyn\^ome de racines $z_1,\ldots ,z_n$  distinctes ou
non, situées dans un convexe $K$ de ${\Cc}$.}
\begin{enumerate}
  \item \question{On suppose que $P'(z)=0$ et $z\notin \{z_1,\ldots ,z_n\}$~; montrer qu'il
existe des réels
$\lambda _1(z),\ldots ,\lambda _n(z)$, inconnus mais $>0$, tels que l'on ait :
$\sum^n_{k=1} \lambda _k(z)(z-z_k)=0$.  (Indication~: considérer
${P'(z)\over P(z)}$ et son conjugué).}
  \item \question{Montrer que $P'$ a aussi toutes ses racines dans $K$ (théorème de
Gauss-Lucas).}
\end{enumerate}
\begin{enumerate}

\end{enumerate}
}