\uuid{EuZB}
\exo7id{5950}
\titre{Exercice 5950}
\theme{Théorème de convergence monotone, dominée et lemme de Fatou}
\auteur{tumpach}
\date{2010/11/11}
\organisation{exo7}
\contenu{
  \texte{}
\begin{enumerate}
  \item \question{Montrer que pour tout $x\in \mathbb{R}_+$,
$\displaystyle\left\{\left(1+\frac{x}{n}\right)^n\right\}$ est une
suite croissante et
$$\lim\limits_{n\rightarrow\infty}\left(1+\frac{x}{n}\right)^n
= e^x = \sum\limits_{k=0}^{\infty} \frac{x^k}{k!}.$$}
  \item \question{Calculer la limite
$$ \lim_{n\to\infty} \int_{\mathbb{R}_+} \left(1+\frac{x}{n}\right)^n\mathrm{e}^{-bx} d\lambda (x)$$
o\`{u} $b>1$.}
\end{enumerate}
\begin{enumerate}
  \item \reponse{Montrons que pour tout $x\in \mathbb{R}_+$,
$(1+\frac{x}{n})^n$ est une suite croissante et que
$\lim\limits_{n\rightarrow\infty}(1+\frac{x}{n})^n=e^x$.
 Pour $n\in \mathbb{N}$ on a
$$\left(1+\frac{x}{n}\right)^n=\sum\limits_{k=0}^n
{\binom{n}{k}}\left(\frac{x}{n}\right)^k=\sum\limits_{k=0}^n
a_{n,k}\,\frac{x^k}{k!},$$ o\`{u} \; $\displaystyle
a_{n,k}=\frac{n!}{(n-k)!n^k}=\frac{n(n-1)\cdots(n-k+1)}{n^k}.$

\bigskip

Les assertions suivantes sont vraies: 
\begin{enumerate}}
  \item \reponse{[{i)}] $a_{n+1,k}\geq a_{n,k}$.  En effet,
$\frac{n+1-l}{n+1}\geq \frac{n-l}{n}$ \, pour \, $l\in \mathbb{N}$
\,  car \, $n^2+n-l\cdot n\geq n^2+n-l\cdot n-l$,}
  \item \reponse{[{ii)}] $a_{n,k}<1$ \; (\'{e}vident);}
  \item \reponse{[{iii)}] Pour tout $k\in \mathbb{N}$, \;
$\lim\limits_{n\rightarrow\infty}a_{n,k}=1$.}
\end{enumerate}
}