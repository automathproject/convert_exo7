\exo7id{7004}
\titre{Théorèmes de Thébault et de Van Aubel}
\theme{}
\auteur{megy}
\date{2016/04/26}
\organisation{exo7}
\contenu{
  \texte{Soit $ABCD$ un quadrilatère convexe direct. On construit quatre carrés qui s'appuient extérieurement sur les côtés $[AB]$, $[BC]$, $[CD]$ et $[DA]$. Les centres respectifs de ces carrés sont notés $P$, $Q$, $R$ et $S$.}
\begin{enumerate}
  \item \question{Montrer que dans le carré construit sur $[AB]$, on a $p=\frac{a-ib}{1-i}$. Démontrer des relations analogues pour les autres carrés.}
  \item \question{Montrer le théorème de Van Aubel : $PQRS$ est un \emph{pseudo-carré}, c'est-à-dire que ses diagonales sont de même longueur et se croisent à angle droit. Pour cela, calculer $\frac{s-q}{r-p}$.}
  \item \question{(Théorème de Thébault) Dans le cas particulier où $ABCD$ est un parallélogramme, montrer que $PQRS$ est un carré.}
\end{enumerate}
\begin{enumerate}
  \item \reponse{L'égalité $p=\frac{a-ib}{1-i}$ est équivalente à deux autres assertions ayant un sens géométrique plus clair. D'une part :
\[p=\frac{a-ib}{1-i} \Leftrightarrow p=\frac{a+b}{2}+i\frac{(a-b)}{2}\]
(Attention, $a$ et $b$ sont complexes : ceci n'est pas une forme algébrique.)  D'autre part, 
\[p=\frac{a-ib}{1-i} \Leftrightarrow (1-i)p=a-ib.\]
Ceci fournit deux façons de prouver le résultat demandé.

\begin{enumerate}}
  \item \reponse{Dans le premier cas, on reconnaît que $(a+b)/2$ est l'affixe du milieu $M$ de $A$ et $B$, que $(a-b)/2$ est l'affixe du vecteur $\frac12 \overrightarrow{BA}$, et que $i(a-b)/2$ correspond à la rotation de $\pi/2$ de ce vecteur. On voit donc que le point d'affixe $\frac{a+b}{2}+i\frac{(a-b)}{2}$ est le point obtenu en translatant $M$ du vecteur  $\frac12\overrightarrow{AA'}$ (avec la notation $A'$ introduite plus haut). Ce point est bien le milieu $P$ du carré.}
  \item \reponse{Dans le second cas, on réécrit l'égalité sous la forme 
\[a-p=i(b-p).\]
Ceci est vrai car cela signifie que $A$ est l'image de $B$ par la rotation de centre $P$ est d'angle $\pi/2$, ce qui est vrai dans un carré.}
\end{enumerate}
}