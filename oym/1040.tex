\uuid{1040}
\titre{Exercice 1040}
\theme{Calculs sur les matrices, Opérations sur les matrices}
\auteur{liousse}
\date{2003/10/01}
\organisation{exo7}
\contenu{
  \texte{}
  \question{Effectuer le produit des  matrices  :
$$\left( \begin{array}{cc} 2 & 1  \\ 3& 2 \end{array} \right)\times
 \left( \begin{array}{cc}1 & -1  \\ 1& 2 \end{array} \right) \ \ \ \  \left( \begin{array}
 {ccc} 1 & 2 & 0  \\ 3 & 1 & 4 \end{array} \right) 
\times
\left( \begin{array}{ccc} -1 &-1& 0 \\ 1 & 4 & -1 \\ 2 & 1 & 2\end{array} \right)
 \ \ \ \ \left( \begin{array}{ccc} a &b& c \\ c & b & a \\ 1 & 1 & 1\end{array} \right)
\times
 \left( \begin{array}{ccc} 1 &a& c \\ 1 & b & b \\ 1 & c & a\end{array} \right)$$}
  \reponse{Si $C = A \times B$ alors on obtient le coefficient $c_{ij}$ (situé à la $i$-ème ligne et la $j$-ème colonne
de $C$) en effectuant le produit scalaire du $i$-ème vecteur-ligne de $A$ avec le $j$-éme vecteur colonne de $B$.

On trouve  
$$\left( \begin{array}{cc} 2 & 1  \\ 3& 2 \end{array} \right)\times
 \left( \begin{array}{cc}1 & -1  \\ 1& 2 \end{array} \right)
= \begin{pmatrix}
  3 & 0 \\
  5 & 1 \\  
  \end{pmatrix}$$

$$\left( \begin{array}
 {ccc} 1 & 2 & 0  \\ 3 & 1 & 4 \end{array} \right) 
\times
\left( \begin{array}{ccc} -1 &-1& 0 \\ 1 & 4 & -1 \\ 2 & 1 & 2\end{array} \right)
= \begin{pmatrix}
  1 & 7 & -2 \\
  6 & 5 & 7 \\ 
  \end{pmatrix}$$

$$\left( \begin{array}{ccc} a &b& c \\ c & b & a \\ 1 & 1 & 1\end{array} \right)
\times
 \left( \begin{array}{ccc} 1 &a& c \\ 1 & b & b \\ 1 & c & a\end{array} \right) 
 = \begin{pmatrix}
    a + b + c & a^2+b^2+c^2 & 2ac + b^2 \\
    a + b + c & 2ac + b^2   & a^2+b^2+c^2 \\
    3         &  a+b+c      & a+b+c \\
\end{pmatrix}$$}
}