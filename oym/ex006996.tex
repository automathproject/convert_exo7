\exo7id{6996}
\titre{Exercice 6996}
\theme{}
\auteur{blanc-centi}
\date{2015/07/04}
\organisation{exo7}
\contenu{
  \texte{Pour les équations différentielles suivantes,
trouver les solutions définies sur $\R$ tout entier :}
\begin{enumerate}
  \item \question{$x^2y'-y = 0$ $(E_1)$}
  \item \question{$xy'+y-1 = 0$ $(E_2)$}
\end{enumerate}
\begin{enumerate}
  \item \reponse{$x^2y'-y = 0$ $(E_1)$


Pour se ramener à l'étude d'une équation différentielle de la forme $y'+ay=b$, 
on résout d'abord sur les intervalles où le coefficient de $y'$ ne s'annule pas: 
on se place donc sur $]-\infty;0[$ ou $]0;+\infty[$.

\begin{enumerate}}
  \item \reponse{\textbf{Résolution sur $]-\infty;0[$ ou $]0;+\infty[$.}
 
Sur chacun de ces intervalles, l'équation différentielle se réécrit 
$$y'-\frac{1}{x^2}y=0$$ 
qui est une équation différentielle linéaire homogène d'ordre 1 à coefficients non constants. Ses solutions sont de la forme $y(x)=\lambda e^{-1/x}$ (en effet, sur $]-\infty;0[$ ou $]0;+\infty[$, une primitive de $\frac{1}{x^2}$ est $\frac{-1}{x}$).}
  \item \reponse{\textbf{Recollement en $0$.}
  
Une solution $y$ de $(E_1)$ sur $\R$ doit \^etre solution sur $]-\infty;0[$ et $]0;+\infty[$, il existe donc $\lambda_+,\ \lambda_-\in\R$ tels que 
$$y(x)=\left\lbrace\begin{array}{l}
\lambda_+e^{-1/x}\ \mathrm{si}\ x>0\\
\lambda_-e^{-1/x}\ \mathrm{si}\ x<0
\end{array}\right.$$
Il reste à voir si l'on peut recoller les deux expressions pour obtenir une solution sur $\R$: autrement dit, pour quels choix de $\lambda_+,\ \lambda_-$ la fonction $y$ se prolonge-t-elle en 0 en une fonction dérivable vérifiant $(E_1)$? 
\begin{itemize}}
  \item \reponse{$e^{-1/x}\xrightarrow[x\to\,0^-]{}+\infty$ et $e^{-1/x}\xrightarrow[x\to\,0^+]{}0$, donc $y$ est prolongeable par continuité en 0 si et seulement si \fbox{$\lambda_-=0$}. On peut alors poser \fbox{$y(0)=0$}, quel que soit le choix de $\lambda_+$.}
  \item \reponse{Pour voir si la fonction ainsi prolongée est dérivable en 0, on étudie son taux d'accroissement:
$$\left\{\begin{array}{l}
\text{pour}\ x>0,\ \frac{y(x)-y(0)}{x-0}=\frac{\lambda_+e^{-1/x}}{x}=-\lambda_+(\frac{-1}{x})e^{-1/x}\xrightarrow[x\to\,0^+]{}0\\
\ \\
\text{pour}\ x<0,\ \frac{y(x)-y(0)}{x-0}=0\xrightarrow[x\to\,0^-]{}0
\end{array}\right.$$
Ainsi la fonction $y$ est dérivable en 0 et \fbox{$y'(0)=0$}.}
  \item \reponse{Par construction, l'équation différentielle $(E_1)$ est satisfaite sur $\R^*$. 
Vérifions qu'elle est également satisfaite au point $x=0$: $0^2 \cdot y'(0)-y(0)=-y(0)=0$.
\end{itemize}}
  \item \reponse{\textbf{Conclusion.}
  
Finalement, les solutions sur $\R$ sont exactement les fonctions suivantes: 
$$y(x)=\left\lbrace\begin{array}{l}
\lambda e^{-1/x}\quad \text{si}\ x>0\\
0\qquad \text{si}\ x\le 0
\end{array}\right. \qquad (\lambda\in\R)$$}
\end{enumerate}
}