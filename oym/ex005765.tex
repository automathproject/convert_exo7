\uuid{5765}
\titre{**}
\theme{}
\auteur{rouget}
\date{2010/10/16}
\organisation{exo7}
\contenu{
  \texte{Pour $n\in\Nn^*$ et $x\in]0,+\infty[$, on pose $I_n(x)=\int_{0}^{+\infty}\frac{1}{(t^2+x^2)^n}$.}
\begin{enumerate}
  \item \question{Calculer la dérivée de la fonction $I_n$ sur $]0,+\infty[$.}
  \item \question{En déduire la valeur de $\int_{0}^{+\infty}\frac{1}{(t^2+1)^3}\;dt$.}
\end{enumerate}
\begin{enumerate}
  \item \reponse{Soit $n\in\Nn^*$. Soient $a$ et $A$ deux réels tels que $0<a<A$. On considère $\begin{array}[t]{cccc}
F_n~:&[a,A]\times\Rr&\rightarrow&\Rr\\
 &(x,t)&\mapsto&\frac{1}{(t^2+x^2)^n}
\end{array}$.

\textbullet~Pour chaque $x$ de $[a,A]$, la fonction $t\mapsto F_n(x,t)$ est continue par morceaux et intégrable sur $[0,+\infty[$ car $F_n(x,t)\underset{t\rightarrow+\infty}{\sim}\frac{1}{t^{2n}}>0$ avec $2n>1$.

\textbullet~La fonction $F_n$ est admet sur $[a,A]\times[0,+\infty[$ une dérivée partielle par rapport à sa première variable $x$ définie par :

\begin{center}
$\forall(x,t)\in[a,A]\times[0,+\infty[$, $\frac{\partial F_n}{\partial x}(x,t)=\frac{-2nx}{(t^2+x^2)^{n+1}}$.
\end{center}

De plus,

- pour chaque $x\in[a,A]$, la fonction $t\mapsto\frac{\partial F_n}{\partial x}(x,t)$ est continue par morceaux sur $[0,+\infty[$,\rule[-4mm]{0mm}{0mm}

-pour chaque $t\in[0,+\infty[$, la fonction $x\mapsto\frac{\partial F_n}{\partial x}(x,t)$ est continue sur $[a,A]$,

-pour chaque $(x,t)\in[a,A]\times[0,+\infty[$,

\begin{center}
$\left|\frac{\partial F_n}{\partial x}(x,t)\right|=\frac{2nx}{(t^2+x^2)^{n+1}}\leqslant\frac{2nA}{(t^2+a^2)^{n+1}}=\varphi(t)$,
\end{center}

où la fonction $\varphi$ est continue par morceaux et intégrable sur $[0,+\infty[$ car négligeable devant $\frac{1}{t^2}$ quand $t$ tend vers $+\infty$.

D'après le théorème de dérivation des intégrales à paramètres (théorème de \textsc{Leibniz}), la fonction $I_n$ est de classe $C^1$ sur $[a,A]$ et sa dérivée s'obtient par dérivation sous le signe somme. Ceci étant vrai pour tous réels $a$ et $A$ tels que $0<a<A$, on a montré que la fonction $I_n$ est de classe $C^1$ sur $]0,+\infty[$ et que

\begin{center}
$\forall x>0$, $I_n'(x)=-2nx\int_{0}^{+\infty}\frac{1}{(t^2+x^2)^{n+1}}\;dt=-2nxI_{n+1}(x)$.
\end{center}

\begin{center}
\shadowbox{
$\forall n\in\Nn^*$, $I_n'(x)=-2nxI_{n+1}(x)$.
}
\end{center}}
  \item \reponse{Pour $x>0$, on a $I_1(x)=\left[\frac{1}{x}\Arctan\left(\frac{t}{x}\right)\right]_0^{+\infty}=\frac{\pi}{2x}$. Ensuite, $I_2(x)=-\frac{1}{2x}I_1'(x)=\frac{\pi}{4x^3}$ puis $I_3(x)=-\frac{1}{4x}I_2'(x)=\frac{3\pi}{16x^5}$ et donc $I_3(1)=\frac{3\pi}{16}$.

\begin{center}
\shadowbox{
$\int_{0}^{+\infty}\frac{1}{(t^2+1)^3}\;dt=\frac{3\pi}{16}$.
}
\end{center}}
\end{enumerate}
}