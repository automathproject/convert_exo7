\uuid{267}
\titre{Exercice 267}
\theme{Arithmétique dans $\Zz$, Divisibilité, division euclidienne}
\auteur{cousquer}
\date{2003/10/01}
\organisation{exo7}
\contenu{
  \texte{}
  \question{Montrer que si $n$ est un entier naturel somme de deux carr\'es d'entiers 
alors le reste de la division euclidienne de $n$ par $4$ n'est jamais \'egal \`a $3$.}
  \reponse{Ecrire $n=p^2+q^2$ et \'etudier le reste de la division euclidienne de
$n$ par $4$ en distinguant les diff\'erents cas de parit\'e de $p$ et $q$.}
}