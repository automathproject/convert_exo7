\uuid{2934}
\titre{Symétrique par rapport à une droite}
\theme{Exercices de Michel Quercia, Nombres complexes}
\auteur{quercia}
\date{2010/03/08}
\organisation{exo7}
\contenu{
  \texte{}
  \question{Les points $A,B,M$ ayant pour affixes $a,b,z$, calculer l'affixe du symétrique $M'$ 
de $M$ par rapport à la droite $(AB)$.}
  \reponse{On peut exprimer que $(AB)$ est la médiatrice du segment $[MM']$, mais on choisit ici d'utiliser le cours sur les similitudes. Cherchons donc à écrire la réflexion d'axe $(AB)$ en coordonnée complexe.
Cette réflexion s'écrit $z\mapsto \alpha \bar z + \beta$, avec $|\alpha|=1$ et $arg(\alpha)=2 arg(b-a)$. On obtient donc $\alpha= \frac{b-a}{\overline b - \overline a}$. 

Pour déterminer $\beta$, on peut exprimer le fait que $a$ est fixe:
\[
a = \frac{b-a}{\overline b - \overline a} \bar a + \beta 
\Leftrightarrow 
\beta = \frac{a\overline b - b\overline a }{\overline b - \overline a}
\]
Finalement, la réflexion s'écrit:
\[
z\mapsto \frac{b-a}{\overline b - \overline a} \cdot z + \frac{a\overline b - b\overline a }{\overline b - \overline a}.
\]}
}