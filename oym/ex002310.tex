\exo7id{2310}
\titre{Exercice 2310}
\theme{}
\auteur{barraud}
\date{2008/04/24}
\organisation{exo7}
\contenu{
  \texte{}
\begin{enumerate}
  \item \question{Montrer que $z \in \mathbb{Z}[\sqrt{d}]$ est inversible ssi $N_d(z)
   = \pm 1$.  D\'eterminer les \'el\'ements inversibles de $\mathbb{Z}[\sqrt{-5}]$.}
  \item \question{Montrer que si $N_d(z) = \pm p$, o\`u $p$ est un premier, alors
   $z$ est irr\'eductible dans $\mathbb Z[\sqrt{d}]$.  Donner quelques
   exemples d'\'el\'ements irreductibles dans $\mathbb{Z}[\sqrt{d}]$ pour
   $d=-1$, $2$, $-6$, $p$, o\`u $p$ un premier.}
  \item \question{On note $A=\mathbb{Z}[\sqrt{-5}]$.  Montrer que $3$ et $2+\sqrt{-5}$
   sont irr\'eductibles dans $A$.}
  \item \question{Trouver tous les irr\'eductibles de $A$ de norme $9$.}
  \item \question{Trouver tous les diviseurs de $9$ et de $3(2+\sqrt{-5})$ dans
  l'anneau $A$ \`a association pr\`es.}
  \item \question{Trouver un $pgcd\,(3, 2+\sqrt{-5})$, et montrer que $3$ et
   $2+\sqrt{-5}$ n'ont pas de $ppcm$ dans l'anneau $A$.}
  \item \question{Montrer que l'id\'eal $I = (3, 2+\sqrt{-5}) \subset A$ n'est pas
  principal.  Donc l'anneau $A$ n'est pas principal. Est-il factoriel
  ?}
  \item \question{Montrer que $9$ et $3(2+\sqrt{-5})$ n'ont pas de $pgcd$ dans $A$.
  Poss\`edent-ils un $ppcm$ ?}
\end{enumerate}
\begin{enumerate}
  \item \reponse{\begin{itemize}}
  \item \reponse{{Si $z\in\Z[\sqrt{d}]$ est inversible~:} 

    Alors $zz^{-1}=1$, donc $N(z)N(z^{-1})=1$. Comme $N(z)\in\Z$ et
    $N(z^{-1})\in\Z$, on a donc $N(z)\in\{1,-1\}$.}
  \item \reponse{Si $N(z=\pm1)$~:

    Alors $z\bar{z}=\pm1$, donc $z(\pm\bar{z})=1$. Comme
    $\pm\bar{z}\in\Z[\sqrt{d}]$, $z$ est inversible.
  \end{itemize}}
  \item \reponse{Soient $z_{1},z_{2}\in\Z[\sqrt{d}]$ tels que $z=z_{1}z_{2}$. Alors
$N(z_{1})N(z_{2})=\pm p$. Comme $\pm p$ est irréductible sur $\Z$, on en
déduit que $N(z_{1})=\pm1$ ou $N(z_{2})=\pm1$. D'après la question
précédente, on a $z_{1}\in\Z[\sqrt{d}]^{\times}$ ou
$z_{2}\in\Z[\sqrt{d}]^{\times}$~: on en déduit que $z$ est irréductible
dans $\Z[\sqrt{d}]$.

(Attention~: $p$ est premier donc irréductible dans $\Z$, mais peut être
réductible dans $\Z[\sqrt{d}]$~! cf. $2$ dans $\Z[i]$.)}
  \item \reponse{On a $N(3)=N(2+\sqrt{-5})=9$. On peut montrer en fait que tout élément
$z$ de norme $9$ est irréductible~: si $z=z_{1}z_{2}$, alors
$N(z_{1})N(z_{2})=9$. Donc $\{N(z_{1}),N(z_{2})\}=\{1,9\}$ ou $\{3,3\}$
(dans $\Z[\sqrt{-5}]$, la norme est toujours positive). Or pour tout
$(n,m)\in\Z^{2}, n^{2}+5m^{2}\neq 3$. En effet, si $|m|\geq 1,
n^{2}+5m^{2}\geq 5$ et pour $m=0$, l'équation revient à $n^{2}=3$, qui
n'a pas de solution entière. Ainsi, $N(z_{1})=1$ ou $N(z_{2})=1$, donc
$z_{1}$ ou $z_{2}$ est inversible. $z$ n'a donc pas de factorisation non
triviale~: $z$ est irréductible dans $\Z[\sqrt{-5}]$. En particulier, $3$
et $2+\sqrt{-5}$ le sont.}
  \item \reponse{Tout élément de $A$ de norme $9$ est irréductible. Il suffit donc de
trouver tous les éléments de norme $9$. Soit $z=n+m\sqrt{-5}\in A$. Si
$|m|\geq 2$ ou $|n|\geq 4$, alors $N(z)>9$. On cherche donc les éléments
de norme $9$ parmi les éléments $z=n+m\sqrt{-5}$ avec $|n|\leq3$ et
$|m|\leq 1$. Pour $m=0$, les seules solutions sont $n=\pm 3$, pour
$|m|=1$, les solutions sont obtenues pour $|n|=2$. Ainsi~:
$$
\forall z\in A:\ \ N(z)=9\Leftrightarrow z\in\{\pm3,\pm(2\pm\sqrt{5})\}
$$}
  \item \reponse{On a $N(9)=81$. Donc si $9=z_{1}z_{2}$ est une factorisation de $9$ dans
$A$, $N(z_{1})N(z_{2})$ est une factorisation de $81$ (dans $\Z$), et
plus précisément on a
$\{N(z_{1}),N(z_{2})\}\in\Big\{\{1,81\},\{3,27\},\{9,9\}\Big\}$.

Si $N(z_{1})=1$ ou $N(z_{2})=1$, la factorisation est triviale.

$A$ n'a pas d'élément de norme $3$ donc la paire $\{3,27\}$ n'est pas
réalisable.

Si enfin $N(z_{1})=N(z_{2})=9$, alors
$z_{1},z_{2}\in\{\pm3,\pm(2\pm\sqrt{5})\}$. Comme
$9=3\cdot3=(2+\sqrt{-5})(2-\sqrt{-5})$, tous ces éléments sont diviseurs
de $9$.

Les diviseurs de $9$ sont donc $\{\pm1,\pm3,\pm(2\pm\sqrt{-5}),\pm9\}$. 

\medskip

Comme $N(3(2+\sqrt{-5}))=81$, le même raisonnement montre que si $d\in A$
divise $3(2+\sqrt{-5})$, alors
$d\in\{\pm1,\pm3,\pm(2\pm\sqrt{-5}),\pm3(2\pm\sqrt{-5})\}$.

Si $(2-\sqrt{-5})a=3(2+\sqrt{-5})$, alors $N(a)=9$, donc $a=\pm3$ ou
$\pm(2\pm\sqrt{-5})$. Comme $A$ est intègre, si $a=\pm3$, on obtient
$2-\sqrt{-5}=\pm(2+\sqrt{-5})$, ce qui est faux. Si $a=\pm(2+\sqrt{-5})$,
on obtient $2-\sqrt{-5}=\pm3$, ce qui est faux. Si enfin
$a=\pm(2-\sqrt{-5})$, on obtient $\pm(-1-4\sqrt{-5})=6+3\sqrt{-5})$, ce
qui est encore faux. Donc $2-\sqrt{-5}$ ne divise pas $3(2+\sqrt{-5})$
dans $A$. Tous les autres éléments de norme $9$ divisent
$3(2+\sqrt{-5})$, donc, finalement~:

Les diviseurs de $3(2+\sqrt{-5})$ sont
$\{\pm1,\pm3,\pm(2+\sqrt{-5}),\pm3(2+\sqrt{-5})\}$.

(Attention~: Le seul fait que $3$ et $2+\sqrt{-5}$ soient irréductibles
ne permet pas de conclure~! Si l'anneau n'est pas factoriel, un produit
d'irréductibles $p_{1}p_{2}$ peut avoir d'autres diviseurs (à association
près) que $p_{1}$ et $p_{2}$... cf $3\cdot3=(2+\sqrt{-5})(2-\sqrt{-5})$~!)}
  \item \reponse{On connaît la liste des diviseurs de $3$ et de $2+\sqrt{-5}$. Les seuls
qui soient communs sont $1$ et $-1$. On en déduit que $1$ est un $\pgcd$
de $3$ et $2+\sqrt{-5}$. 

\medskip

$9$ et $3(2+\sqrt{-5})$ sont des multiples communs de $3$ et
$2+\sqrt{-5}$, donc si ces deux éléments admettent un $\mathrm{ppcm}$ $m$, on a
$m|9$ et $m|3(2+\sqrt{-5})$. On connaît la liste des diviseurs de $9$ et
$3(2+\sqrt{-5})$~: à association près, on en déduit que
$m\in\{1,3,2+\sqrt{-5}\}$. Comme $3|m$, la seule possibilité est $m=3$,
et comme $(2+\sqrt{-5})|m$, la seule possibilité est $m=2+\sqrt{-5}$. Il
y a donc contradiction~:

$3$ et $2+\sqrt{-5}$ n'ont pas de $\mathrm{ppcm}$ dans $A$.}
  \item \reponse{Supposons $I$ principal~: soit $a\in A$ un générateur~: $I=(a)$. Alors
$a$ est un diviseur commun à $3$ et $2+\sqrt{-5}$, donc $a=\pm1$. (En
particulier, $I=A$). Soient $u=u_{1}+u_{2}\sqrt{-5}$ et
$v=v_{1}+v_{2}\sqrt{-5}$ deux éléments de $A$. On a~:
\begin{align*}
3u+(2+\sqrt{-5})v=1
&\Leftrightarrow (3u_{1}+2v_{1}-5v_{2})+(3u_{2}+v_{1}+2v_{2})\sqrt{-5}=1\\
&\Leftrightarrow \left\{
\begin{array}{rl}
3u_{1}+2v_{1}-5v_{2}&=1\\
3u_{2}+v_{1}+2v_{2}&=0
\end{array}\right.\\
&\Rightarrow \left\{
\begin{array}{rl}
-v_{1}+v_{2}&\equiv1 [3]\\
v_{1}-v_{2}&\equiv0 [3]
\end{array}\right.
\end{align*}
Donc $\forall u,v\in A,\ 3u+(2+\sqrt{-5})v\neq 1$. Donc $1\notin I$, ce
qui est une contradiction~: $I$ n'est pas principal.

L'anneau $A$ n'est pas principal puisqu'il a au moins un idéal non
principal. Il n'est pas non plus factoriel, puisque
$9=3\,3=(2+\sqrt{-5})(2-\sqrt{-5})$ admet deux factorisation en
irréductibles non équivalentes à association près.}
  \item \reponse{\begin{itemize}}
  \item \reponse{Les diviseurs communs de $9$ et $3(2+\sqrt{-5})$ sont $\{\pm1,\pm 3,
\pm(2+\sqrt{-5})\}$. Si $9$ et $3(2+\sqrt{-5})$ admettent un $\pgcd$ $d$,
alors $d$ est dans cette liste, et divisible par tous les membre de cette
liste. Mais $3$ n'est pas divisible par $2+\sqrt{-5}$ et $2+\sqrt{-5}$ ne
divise pas $3$~: $9$ et $2+\sqrt{-5}$ n'ont pas de $\pgcd$.}
  \item \reponse{Supposons que $9$ et $3(2+\sqrt{-5})$ admettent un $\mathrm{ppcm}$ $M$. Alors il
existe des éléments $a,b\in A$ tels que $M=9a=3(2+\sqrt{-5})b$. Notons
$m=3a=(2+\sqrt{-5})b$ ($A$ est intègre).

$m$ est un multiple commun de $3$ et $2+\sqrt{-5}$.

Soit $k$ un multiple commun de $3$ et $2+\sqrt{-5}$. Alors $3k$ est un
multiple commun de $9$ et $3(2+\sqrt{-5})$, donc $M|3k$~: $\exists c\in
A, 3k=Mc=3mc$. On en déduit que $k=mc$ ($A$ est intègre), donc $m|k$. On
en déduit que $m$ est un $\mathrm{ppcm}$ de $3$ et $2+\sqrt{-5}$, ce qui est
impossible.
  \end{itemize}}
\end{enumerate}
}