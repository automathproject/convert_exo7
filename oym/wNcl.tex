\uuid{wNcl}
\exo7id{1027}
\titre{Exercice 1027}
\theme{Applications linéaires, Image et noyau}
\auteur{ridde}
\date{1999/11/01}
\organisation{exo7}
\contenu{
  \texte{Soit $E$ et $F$ de dimensions finies et $u, v \in \mathcal{L} (E, F)$.}
\begin{enumerate}
  \item \question{Montrer que $\text{rg} (u + v) \leq \text{rg} (u) + \text{rg} (v)$.}
  \item \question{En d\'eduire que $\left|\text{rg} (u) - \text{rg} (v)\right| \leq
\text{rg} (u + v)$.}
\end{enumerate}
\begin{enumerate}
  \item \reponse{Par la formule $\dim(G+H) = \dim(G)+\dim(H)-
\dim(G\cap H)$, on sait que $\dim(G+H) \leqslant \dim(G)+\dim(H)$.
Pour $G=\Im u$ et $H=\Im v$ on obtient :
$\dim (\Im u+\Im v) \leqslant \dim \Im u +\dim \Im v$.
Or $\Im (u+v) \subset \Im u+\Im v$.
Donc $\text{rg} (u + v) \leq \text{rg} (u) + \text{rg} (v)$.}
  \item \reponse{On applique la formule pr\'ec\'edente \`a $u+v$ et $-v$ :
$\text{rg}  ((u+v)+(-v)) \leqslant \text{rg}  (u+v)+\text{rg} (-v)$, or $\text{rg} (-v)=\text{rg} (v)$
donc $\text{rg} (u) \leqslant \text{rg} (u+v)+\text{rg} (v)$.
Donc $\text{rg} (u)-\text{rg} (v)\leqslant \text{rg} (u+v)$.
On recommence en \'echangeant $u$ et $v$ pour obtenir :
$\left|\text{rg} (u) - \text{rg} (v)\right| \leq \text{rg} (u + v)$.}
\end{enumerate}
}