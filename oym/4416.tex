\uuid{4416}
\titre{Mines MP 2000}
\theme{Exercices de Michel Quercia, Séries numérique}
\auteur{quercia}
\date{2010/03/14}
\organisation{exo7}
\contenu{
  \texte{}
  \question{Soit $\alpha >0$. Étudier la série $\sum u_n$, avec 
$u_n=\frac{(-1)^n}{\sqrt{n^{\alpha}+(-1)^n}}\cdotp$}
  \reponse{$u_n = \frac{(-1)^n}{n^{\alpha/2}}-\frac1{2n^{3\alpha/2}}
+  o\Bigl(\frac1{n^{3\alpha/2}}\Bigr)$,
il y a convergence ssi $\alpha>\frac23$.}
}