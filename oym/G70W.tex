\uuid{G70W}
\exo7id{4740}
\titre{Distance {\`a} un ensemble}
\theme{Exercices de Michel Quercia, Topologie dans les espaces vectoriels normés}
\auteur{quercia}
\date{2010/03/16}
\organisation{exo7}
\contenu{
  \texte{Soit $E$ un evn et $A \subset E$ une partie non vide.
Pour $\vec x \in E$ on pose :
$d(\vec x,A) = \inf\{\|\vec x-\vec a\,\| \text{ tq } \vec a \in A \}$.}
\begin{enumerate}
  \item \question{Montrer que : $\forall\ \vec x,\vec y \in E,\
    \Bigl|d(\vec x,A) - d(\vec y,A)\Bigr| \le \|\vec x-\vec y\,\|$.
    (Enlever la valeur absolue et d{\'e}montrer s{\'e}par{\'e}ment chaque in{\'e}galit{\'e})}
  \item \question{Montrer que l'application $\vec x  \mapsto d(\vec x,A)$ est continue.}
\end{enumerate}
\begin{enumerate}

\end{enumerate}
}