\uuid{U2eV}
\exo7id{3294}
\titre{Division de $1$ par $(1-X)^2$}
\theme{Exercices de Michel Quercia, Division suivant les puissances croissantes}
\auteur{quercia}
\date{2010/03/08}
\organisation{exo7}
\contenu{
  \texte{}
\begin{enumerate}
  \item \question{Effectuer la division suivant les puissances croissantes {\`a} un ordre $n$
    quelconque de $1$ par $(1-X)^2$.}
  \item \question{En d{\'e}duire $1 + 2\cos\theta + 3\cos2\theta + \dots + n\cos (n-1)\theta$,
    $n \in \N^*$, $\theta \in \R$.}
\end{enumerate}
\begin{enumerate}
  \item \reponse{$1 = (1-X)^2 (1 + 2X + 3X^2 + \dots + nX^{n-1}) + (n+1)X^n - nX^{n+1}$.}
  \item \reponse{$=\frac {-n\cos n\theta + (n+1)\cos (n-1)\theta -\cos\theta}
                      {4\sin^2{\textstyle{\frac\theta2}}}.$}
\end{enumerate}
}