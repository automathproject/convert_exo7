\uuid{21v4}
\exo7id{4741}
\titre{Distance {\`a} un ensemble (ENS Cachan MP 2002)}
\theme{Exercices de Michel Quercia, Topologie dans les espaces vectoriels normés}
\auteur{quercia}
\date{2010/03/16}
\organisation{exo7}
\contenu{
  \texte{Soit $A$ une partie de~$\R^n$ non vide. On note pour $x\in\R^n$~:
$d_A(x) = \inf\{\|x-y\|\text{ tq }y\in A\}$.}
\begin{enumerate}
  \item \question{Montrer que $d_A$ est continue.}
  \item \question{Soient deux parties de $\R^n$ non vides $A,B$. Donner une condition {\'e}quivalente
    {\`a} $d_A=d_B$.}
  \item \question{On note $\rho(A,B) = \sup\{|d_A(y)-d_B(y)|,\ y\in\R^n\}$, valant {\'e}ventuellement~$+\infty$.

    Montrer que l'on a $\rho(A,B) = \max\Bigl(\sup\limits_{x\in A} d_B(x),\ \sup\limits_{x\in B} d_A(x)\Bigr)$.}
\end{enumerate}
\begin{enumerate}
  \item \reponse{Pour $x,x'\in\R^n$ et $y\in A$ on a $d_A(x)\le \|x-y\| \le \|x-x'\| + \|x'-y\|$.
    En prenant la borne inf{\'e}rieure sur~$y$ on obtient $d_A(x) \le \|x-x'\| + d_A(x')$.
    Par sym{\'e}trie on a aussi $d_A(x') \le \|x-x'\| + d_A(x)$ d'o{\`u}
    $|d_A(x)-d_A(x')\le\|x-x'\|$.}
  \item \reponse{On sait que $\overline A = \{x\in\R^n\text{ tq }d_A(x)=0\}$.
    Donc $d_A = d_B \Rightarrow \overline A = \overline B$ et la r{\'e}ciproque r{\'e}sulte
    de la propri{\'e}t{\'e} facile $d_A = d_{\overline A}$.}
  \item \reponse{On note~: $M = \sup\{|d_A(y)-d_B(y)|,\ y\in\R^n\}$,
    $\alpha = \sup\{d_B(x), x\in A\}$ et $\beta=\sup\{d_A(x),\ x\in B\}$.
    Par restriction de $y$ {\`a} $A\cup B$ on obtient $M \ge \max(\alpha,\beta)$.
    Par ailleurs, pour $y\in\R^n$, $a\in A$ et $b\in B$ on a
    $\|y-a\|-\|y-b\|\le\|a-b\|$ d'o{\`u} $d_A(y)-\|y-b\|\le d_A(b)$ puis
    $d_A(y)-d_B(y)\le \beta$. Par sym{\'e}trie on a aussi
    $d_B(y)-d_A(y)\le \alpha$ donc $|d_A(y)-d_B(y)|\le\max(\alpha,\beta)$
    et finalement $M\le\max(\alpha,\beta)$.}
\end{enumerate}
}