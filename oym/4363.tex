\uuid{4363}
\titre{Ensae MP$^*$ 2000}
\theme{Exercices de Michel Quercia, Intégrale dépendant d'un paramètre}
\auteur{quercia}
\date{2010/03/12}
\organisation{exo7}
\contenu{
  \texte{}
  \question{Soit $\alpha \in \R$. Trouver la limite de $u_n=\sum_{k=1}^n \frac{\sin (k\alpha)}{n+k}\cdotp$}
  \reponse{Si $\alpha\in2\pi\,\Z$ alors $u_n=0$ pour tout~$n$.

Sinon, $u_n = \Im\Bigl( \int_{t=0}^1 \sum_{k=1}^nt^{n+k-1}e^{ik\alpha}\,d t\Bigr)
=\Im\Bigl( \int_{t=0}^1\frac{t^ne^{i\alpha}-t^{2n}e^{i(n+1)\alpha}}{1-te^{i\alpha}}\,d t\Bigr)
\to 0$ (lorsque $n\to\infty$) par convergence dominée.}
}