\uuid{yGJz}
\exo7id{4097}
\titre{Zéros des solutions de $y''+ay'+by = 0$}
\theme{Exercices de Michel Quercia, \'Equations différentielles linéaires (II)}
\auteur{quercia}
\date{2010/03/11}
\organisation{exo7}
\contenu{
  \texte{On considère l'équation $(*) \iff y'' + a(t)y' + b(t)y = 0$, avec $a,b$ continues.}
\begin{enumerate}
  \item \question{Soit $y$ une solution non nulle de $(*)$. Montrer que les zéros de $y$ sont isolés.}
  \item \question{Soient $y,z$ deux solutions de $(*)$ non proportionelles.
  \begin{enumerate}}
  \item \question{Montrer que $y$ et $z$ n'ont pas de zéros commun.}
  \item \question{Montrer que si $u,v$ sont deux zéros consécutifs de $y$, alors $z$ possède un
    unique zéro dans l'intervalle $]u,v[$
    (étudier $\frac zy$).}
\end{enumerate}
\begin{enumerate}
  \item \reponse{\begin{enumerate}}
  \item \reponse{Wronskien.}
  \item \reponse{$\left(\frac zy\right)' = \frac {z'y-zy'}{y^2}$ est de signe constant
$ \Rightarrow  \frac zy$ est monotone.

$\frac zy$ admet des limites infinies en $u$ et $v$. TVI}
\end{enumerate}
}