\uuid{3128}
\titre{Congruences simultan{\'e}es}
\theme{Exercices de Michel Quercia, Relation de Bézout}
\auteur{quercia}
\date{2010/03/08}
\organisation{exo7}
\contenu{
  \texte{}
  \question{Une bande de 17 pirates dispose d'un butin compos{\'e} de $N$ pi{\`e}ces d'or
d'{\'e}gale valeur. Ils d{\'e}cident de se le partager {\'e}galement et de donner le
reste au cuisinier (non pirate). Celui ci re{\c c}oit 3 pi{\`e}ces.

Mais une rixe {\'e}clate et 6 pirates sont tu{\'e}s. Tout le butin est reconstitu{\'e} et
partag{\'e} entre les survivants comme pr{\'e}c{\'e}demment; le cuisinier re{\c c}oit alors 4
pi{\`e}ces.

Dans un naufrage ult{\'e}rieur, seuls le butin, 6 pirates et le cuisinier sont
sauv{\'e}s. Le butin est {\`a} nouveau partag{\'e} de la m{\^e}me mani{\`e}re et le cuisinier
re{\c c}oit 5 pi{\`e}ces.

Quelle est alors la fortune minimale que peut esp{\'e}rer le cuisinier lorsqu'il
d{\'e}cide d'empoisonner le reste des pirates~?}
  \reponse{785.}
}