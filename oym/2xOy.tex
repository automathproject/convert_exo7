\uuid{2xOy}
\exo7id{4995}
\titre{Sous-tangente, sous-normale}
\theme{Exercices de Michel Quercia, Courbes définies par une condition}
\auteur{quercia}
\date{2010/03/17}
\organisation{exo7}
\contenu{
  \texte{Soit $\mathcal{C}$ une courbe du plan. A un point $M$ un point de $\mathcal{C}$, on associe les
points $T$ et $N$ selon le dessin :
$$
\includegraphics[height=4cm]{../images/img004995-1}
$$

Déterminer les courbes vérifiant la condition suivante :}
\begin{enumerate}
  \item \question{$\overline{OT} = {}$cste.}
  \item \question{$\overline{ON} = {}$cste.}
\end{enumerate}
\begin{enumerate}
  \item \reponse{$\rho = \frac1{a\theta+b}$.}
  \item \reponse{$\rho = a\theta+b$. (Spirale d'Archimède)}
\end{enumerate}
}