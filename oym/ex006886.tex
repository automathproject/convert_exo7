\uuid{6886}
\titre{Exercice 6886}
\theme{}
\auteur{exo7}
\date{2012/09/05}
\organisation{exo7}
\contenu{
  \texte{}
  \question{Calculer les déterminants des matrices suivantes :

$$
\begin{pmatrix}
a&b&c\\c&a&b\\b&c&a
\end{pmatrix}
\begin{pmatrix}
1&0&0&1 \\ 0&1&0&0 \\ 1&0&1&1 \\ 2&3&1&1
\end{pmatrix}
\begin{pmatrix}
-1 & 1 & 1 & 1\\ 1 & -1 & 1 & 1\\ 1 & 1 & -1& 1\\ 1 & 1& 1&-1
\end{pmatrix}
\begin{pmatrix}
10 & 0 & -5 & 15 \\ -2 & 7 & 3 & 0 \\ 8 & 14 & 0 & 2 \\ 0 & -21 & 1 & -1
\end{pmatrix}
$$
 
$$
\begin{pmatrix}
a&a&b&0 \\  a&a&0&b \\  c&0&a&a \\ 0&c&a&a
\end{pmatrix}
\begin{pmatrix}
1&0&3&0&0 \\ 0&1&0&3&0 \\ a&0&a&0&3 \\ b&a&0&a&0 \\ 0&b&0&0&a  
\end{pmatrix}
\begin{pmatrix}
1&0&0&1&0 \\ 0&-4&3&0&0 \\ -3&0&0&-3&-2 \\ 0&1&7&0&0 \\ 4&0&0&7&1  
\end{pmatrix}
$$}
  \reponse{\begin{enumerate}
  \item Par la règle de Sarrus :
$$\Delta_1 = \begin{vmatrix}
a&b&c\\c&a&b\\b&c&a
\end{vmatrix} 
= a^3+b^3+c^3-3abc.$$

  \item On développe par rapport à la seconde ligne qui ne contient qu'un coefficient non nul et on calcule le déterminant
$3\times 3$ par la règle de Sarrus :
$$\Delta_2 = \begin{vmatrix}
1&0&0&1 \\ 0&1&0&0 \\ 1&0&1&1 \\ 2&3&1&1
\end{vmatrix}
= +1 \begin{vmatrix}
1&0&1 \\ 1&1&1 \\ 2&1&1
\end{vmatrix}
= -1.$$

  \item
$$\Delta_3 = 
\begin{array}{l|cccc|} 
_{L_1} & -1 & 1 & 1 & 1 \\ _{L_2} & 1 & -1 & 1 & 1\\ _{L_3} & 1 & 1 & -1& 1\\ _{L_4} & 1 & 1& 1&-1
\end{array}
= \begin{array}{l|cccc|} 
 & -1 & 1 & 1 & 1 \\ _{L_2\leftarrow L_2+L_1} & 0 & 0 & 2 & 2\\
 _{L_3\leftarrow L_3+L_1} & 0 & 2 & 0 & 2 \\  _{L_4\leftarrow L_4+L_1} &0 & 2 & 2 & 0
\end{array}
$$
On développe par rapport à la première colonne :
$$\Delta_3 = (-1) \times \begin{array}{|ccc|} 
  0 & 2 & 2\\  2 & 0 & 2 \\  2 & 2 & 0
\end{array} = -16$$

  \item Le déterminant est linéaire par rapport à chacune de ses lignes
et aussi chacune de ses colonnes. 
Par exemple les coefficients de la première ligne sont tous des multiples
de $5$ donc 
$$\Delta_4 = 
\begin{vmatrix}
10 & 0 & -5 & 15 \\ -2 & 7 & 3 & 0 \\ 8 & 14 & 0 & 2 \\ 0 & -21 & 1 & -1
\end{vmatrix}
= 5 \times \begin{vmatrix}
2 & 0 & -1 & 3 \\ -2 & 7 & 3 & 0 \\ 8 & 14 & 0 & 2 \\ 0 & -21 & 1 & -1
\end{vmatrix}
$$
On fait la même chose avec la troisième ligne :
$$\Delta_4 = 5 \times 2 \times \begin{vmatrix}
2 & 0 & -1 & 3 \\ -2 & 7 & 3 & 0 \\ 4 & 7 & 0 & 1 \\ 0 & -21 & 1 & -1
\end{vmatrix}
$$
Et enfin les coefficients la première colonne sont des multiples de $2$ et 
ceux de la troisième colonne sont des multiples de $7$ donc :
$$\Delta_4 = 5 \times 2 \times 2 \times\begin{vmatrix}
1 & 0 & -1 & 3 \\ -1 & 7 & 3 & 0 \\ 2 & 7 & 0 & 1 \\ 0 & -21 & 1 & -1
\end{vmatrix}
= 5 \times 2 \times 2 \times 7  \times
\begin{vmatrix}
1 & 0 & -1 & 3 \\ -1 & 1 & 3 & 0 \\ 2 & 1 & 0 & 1 \\ 0 & -3 & 1 & -1
\end{vmatrix}
$$

Les coefficients sont plus raisonnables !
On fait $L_2\leftarrow L_2+L_1$ et $L_3\leftarrow L_3-2L_1$
pour obtenir :
$$\Delta_4 = 140 \times 
\begin{vmatrix}
1 & 0 & -1 & 3 \\ 0 & 1 & 2 & 3 \\ 0 & 1 & 2 & -5 \\ 0 & -3 & 1 & -1
\end{vmatrix}
=140 \times \begin{vmatrix}
 1 & 2 & 3 \\  1 & 2 & -5 \\ -3 & 1 & -1
\end{vmatrix}
= 140 \times 56 = 7840$$


  \item 
$$\Delta_5 =\begin{array}{l|cccc|} 
_{L_1} &a&a&b&0 \\ _{L_2} & a&a&0&b \\ _{L_3} & c&0&a&a \\ _{L_4} &0&c&a&a
\end{array}
= \begin{array}{l|cccc|} 
 &a&a&b&0 \\ _{L_2\leftarrow L_2-L_1} & 0&0&-b&b \\  & c&0&a&a \\ _{L_4\leftarrow L_4-L_3} & -c&c&0&0
\end{array}
$$
On fait ensuite les opérations suivantes sur les colonnes :
$C_2 \leftarrow C_2+C_1$ et $C_3 \leftarrow C_3-C_4$ pour obtenir 
une dernière ligne facile à développer :
$$\Delta_5 
= \begin{array}{|cccc|} 
 a&2a&b&0 \\ 0&0&-2b&b \\  c&c&0&a \\ -c&0&0&0
\end{array}
= +c \times \begin{array}{|ccc|} 
 2a&b&0 \\ 0&-2b&b \\  c&0&a \\
 \end{array} = bc(bc-4a^2)
$$
  \item
On fait d'abord les opérations $C_1 \leftarrow C_1-C_3$ et $C_2 \leftarrow C_2-C_4$
et on développe par rapport à la première ligne :
$$\Delta_6 = \begin{vmatrix}
1&0&3&0&0 \\ 0&1&0&3&0 \\ a&0&a&0&3 \\ b&a&0&a&0 \\ 0&b&0&0&a  
\end{vmatrix}
=
\begin{vmatrix}
-2&0&3&0&0 \\ 0&-2&0&3&0 \\ 0&0&a&0&3 \\ b&0&0&a&0 \\ 0&b&0&0&a  
\end{vmatrix}
= (-2) \times \begin{vmatrix}
-2&0&3&0 \\ 0&a&0&3 \\ 0&0&a&0 \\ b&0&0&a  
\end{vmatrix}
+ 3  \times \begin{vmatrix}
0&-2&3&0 \\ 0&0&0&3 \\ b&0&a&0 \\ 0&b&0&a  
\end{vmatrix}
$$
Le premier déterminant à calculer se développe par rapport à la deuxième colonne et le second déterminant
par rapport à la première colonne :
$$\Delta_6 = (-2)\times a \times 
\begin{vmatrix}
-2&3&0 \\  0&a&0 \\ b&0&a  
\end{vmatrix}
+ 3  \times b \times 
\begin{vmatrix}
-2&3&0 \\ 0&0&3 \\ b&0&a  
\end{vmatrix}=4a^3+27b^2$$

  \item Nous allons permuter des lignes et des colonnes pour se ramener à une matrice diagonale par blocs.
Souvenons-nous que lorsque l'on échange deux lignes (ou deux colonnes) alors le déterminant change de signe.
Nous allons rassembler les zéros.
On commence par échanger les colonnes $C_1$ et $C_3$ : $C_1\leftrightarrow C_3$ :
$$\Delta_7=
\begin{vmatrix}
1&0&0&1&0 \\ 0&-4&3&0&0 \\ -3&0&0&-3&-2 \\ 0&1&7&0&0 \\ 4&0&0&7&1  
\end{vmatrix}=
- \begin{vmatrix}
0&0&1&1&0 \\ 3&-4&0&0&0 \\ 0&0&-3&-3&-2 \\ 7&1&0&0&0 \\ 0&0&4&7&1  
\end{vmatrix}$$
Puis on échange les lignes $L_1$ et $L_4$ : $L_1\leftrightarrow L_4$ :
$$\Delta_7=+\begin{array}{|ccccc|} 
7&1&0&0&0 \\ 3&-4&0&0&0 \\  0&0&-3&-3&-2 \\ 0&0&1&1&0 \\ 0&0&4&7&1  
\end{array}$$
Notre matrice est sous la forme d'une matrice diagonale par blocs
 et son déterminant est le produit des déterminants.
$$\Delta_7=\begin{array}{|cc|ccc|} 
7&1&0&0&0 \\ 3&-4&0&0&0 \\ \hline 0&0&-3&-3&-2 \\ 0&0&1&1&0 \\ 0&0&4&7&1  
\end{array}
= \begin{vmatrix}
 7&1 \\  3&-4 \\  
  \end{vmatrix} \times
\begin{vmatrix}
 -3&-3&-2 \\  1&1&0 \\  4&7&1  
  \end{vmatrix}
= (-31)\times (-6) = 186
$$

\end{enumerate}}
}