\uuid{6MXp}
\exo7id{3171}
\titre{$P-X \mid P\circ P - X$}
\theme{Exercices de Michel Quercia, Polynômes}
\auteur{quercia}
\date{2010/03/08}
\organisation{exo7}
\contenu{
  \texte{}
\begin{enumerate}
  \item \question{Soit $P \in { K[X]}$. D{\'e}montrer que $P-X$ divise $P\circ P-X$.}
  \item \question{R{\'e}soudre dans $\C$ : $(z^2+3z+1)^2 + 3z^2+8z+4 = 0$.}
\end{enumerate}
\begin{enumerate}
  \item \reponse{$P\circ P - X = (P\circ P - P) + (P - X)$.}
  \item \reponse{$P(z) = z^2+3z+1  \Rightarrow  z=-1,-1,-2\pm i$.}
\end{enumerate}
}