\exo7id{1446}
\titre{Exercice 1446}
\theme{}
\auteur{legall}
\date{1998/09/01}
\organisation{exo7}
\contenu{
  \texte{Soit $  G  $ le sous-groupe de $  Gl(2, \R )  $
engendr\'e par les matrices $  \displaystyle{  A= \frac{1 }{\sqrt
2}
\begin{pmatrix}-1 & 1\cr 1 & 1 \cr \end{pmatrix} }  $ et $  B= \begin{pmatrix}-1 & 0\cr 0 & 1 \cr \end{pmatrix}  .$}
\begin{enumerate}
  \item \question{Soit $  H  $ le sous-groupe de $  G  $ engendr\'e par $  AB  .$ Calculer $  \vert H\vert  $}
  \item \question{Montrer que $  H  $ est distingu\'e dans $  G  .$ Calculer le quotient $  G/H  ;$ en d\'eduire $  \vert G\vert   .$}
\end{enumerate}
\begin{enumerate}
  \item \reponse{Un calcul donne $C^8 = I$ et pour $1\le k \le 7$, $C^k\not= I$.
Donc le groupe $H$ engendr\'e par $C$ est d'ordre $8$. Attention !
m\^eme si $A^2=I$ et $B^2=I$ on a $(AB)^2 \not= I$ car $AB \not=
BA$.}
  \item \reponse{Pour montrer que $H$ est distingu\'e il suffit de montrer que
$ACA^{-1}$ et $BCB^{-1}$ sont dans $H$. Mais $ACA^{-1} = ACA =
AABA = BA = (AB)^{-1} \in H$. De m\^eme $BCB^{-1} = (AB)^{-1}$.
Donc $H$ est distingu\'e dans $H$.

\bigskip
Un \'el\'ement $M$ de $G$ s'\'ecrit
$$M= A^{a_1}B^{b_1}A^{a_2}\ldots A^{a_n}B^{b_n} \quad a_i,b_i \in \Zz.$$
Mais dans $G/H$ tout terme $\overline{AB}$ ou $\overline{BA}$ vaut
$\overline{I}$ Donc $G/H = \{ \overline{I},
\overline{A},\overline{A}^2, \overline{A}^3,\ldots,
\overline{B},\overline{B}^2,\overline{B}^3,\ldots \}$ mais comme
$A^2=B^2=I$ et $AB \in H$ alors $G/H$ s'\'ecrit simplement :
$$G/H = \left\lbrace \overline{I},\overline{A} \right\rbrace.$$

\bigskip
Enfin, par la formule $|G| = |H| \times |G/H|$ nous obtenons $|G|
= 8 \times 2 = 16$.}
\end{enumerate}
}