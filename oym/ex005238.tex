\uuid{5238}
\titre{Exercice 5238}
\theme{}
\auteur{rouget}
\date{2010/06/30}
\organisation{exo7}
\contenu{
  \texte{}
  \question{Etudier les deux suites $u_n=\left(\sum_{k=1}^{n}\frac{1}{\sqrt{k}}\right)-2\sqrt{n+1}$ et $v_n=\left(\sum_{k=1}^{n}\frac{1}{\sqrt{k}}\right)-2\sqrt{n}$.}
  \reponse{Pour $n$ entier naturel non nul donné, on a
$$u_{n+1}-u_n=\frac{1}{\sqrt{n+1}}-2\sqrt{n+2}+2\sqrt{n+1}=\frac{1}{\sqrt{n+1}}-\frac{2}{\sqrt{n+1}+\sqrt{n+2}}>\frac{1}{\sqrt{n+1}}-\frac{2}{\sqrt{n+1}+\sqrt{n+1}}=0.$$
De même,

$$v_{n+1}-v_n=\frac{1}{\sqrt{n+1}}-2\sqrt{n+1}+2\sqrt{n}=\frac{1}{\sqrt{n+1}}-\frac{2}{\sqrt{n+1}+\sqrt{n}}<\frac{1}{\sqrt{n+1}}-\frac{2}{\sqrt{n+1}+\sqrt{n+1}}=0.$$
La suite $u$ est strictement croissante et la suite $v$ est strictement décroissante. Enfin, 

$$v_n-u_n=2\sqrt{n+1}-2\sqrt{n}=\frac{2}{\sqrt{n}+\sqrt{n+1}},$$
et la suite $v-u$ converge vers $0$. Les suites $u$ et $v$ sont ainsi adjacentes et donc convergentes, de même limite.}
}