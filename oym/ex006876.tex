\uuid{6876}
\titre{Exercice 6876}
\theme{}
\auteur{gammella}
\date{2012/05/29}
\organisation{exo7}
\contenu{
  \texte{}
  \question{On considère le champ vectoriel $\vec{V}(x,y)=(1+2xy,x^3-3)$. 
Ce champ est-il un champ de gradient ?}
  \reponse{Au champ $\vec{V}(x,y)$ est associée la forme
$$\omega=(1+2xy) dx + (x^3-3) dy.$$
Cette forme n'est pas exacte puisque
$ \frac{\partial (1+2xy)}{\partial y}\not=  \frac{\partial(x^3-3)}{\partial x}$. Il s'ensuit
que $\vec{V(x,y)}$ n'est pas un champ de gradient.}
}