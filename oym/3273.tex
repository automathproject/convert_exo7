\uuid{3273}
\titre{$F \circ G$ est un polyn{\^o}me}
\theme{Exercices de Michel Quercia, Fractions rationnelles}
\auteur{quercia}
\date{2010/03/08}
\organisation{exo7}
\contenu{
  \texte{}
  \question{Trouver tous les couples $(F,G) \in \bigl({\C(X)}\bigr)^2$ tels que
$F\circ G \in {\C[X]}$ (utiliser l'exercice \ref{Imf}).}
  \reponse{1) $G = $cste.\par
2) $F$ a un seul p{\^o}le $a$
   $ \Rightarrow  F = \frac P{(X-a)^k}$ et $G = a + \frac 1Q$ avec $\deg P \le k$.\par
3) $F \in {\C[X]}  \Rightarrow  G \in {\C[X]}$.}
}