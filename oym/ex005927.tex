\exo7id{5927}
\titre{Exercice 5927}
\theme{}
\auteur{tumpach}
\date{2010/11/11}
\organisation{exo7}
\contenu{
  \texte{Soit $(\Omega, \Sigma)$ un espace mesurable (i.e. un ensemble
$\Omega$ muni d'une tribu $\Sigma \subset \mathcal{P}(\Omega)$).
Soit $\mu$ une mesure finie sur $(\Omega, \Sigma)$. Montrer les
propri\'et\'es suivantes~: ($A,B,A_i$ sont des élements de de $\Sigma$)}
\begin{enumerate}
  \item \question{Si $A_{1}, A_{2}, \dots, A_{k}$ sont deux \`a deux
disjoints, alors
$$
\mu\left(\cup_{i=1}^{k} A_{i}\right) = \sum_{i=1}^{k} \mu(A_{i}).
$$}
  \item \question{Si $B \subset A$ alors $\mu(A\setminus B) =
\mu(A) -\mu(B)$.}
  \item \question{\emph{Monotonie~:} Si $B \subset A$ alors $\mu(B) \leq \mu(A)$.}
  \item \question{\emph{Principe inclusion-exclusion~:} $\mu(A\cup
B) = \mu(A) + \mu(B) - \mu(A\cap B)$.}
  \item \question{$\mu\left(\cup_{i=1}^{+\infty} A_{i}\right) \leq
\sum_{i=1}^{+\infty}\mu(A_{i}).$ \textit{(Rappelons que l'on a
\'egalit\'e si l'union est disjointe.)}}
\end{enumerate}
\begin{enumerate}

\end{enumerate}
}