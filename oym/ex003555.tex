\uuid{3555}
\titre{$P(u) = \sum P(\lambda_i) u_i$}
\theme{}
\auteur{quercia}
\date{2010/03/10}
\organisation{exo7}
\contenu{
  \texte{Soit $E$ un $ K$-ev de dimension finie et $u \in \mathcal{L}(E)$.}
\begin{enumerate}
  \item \question{On suppose $u$  diagonalisable et on note
    $\lambda_1,\dots,\lambda_p$ ses valeurs propres distinctes.
  \begin{enumerate}}
  \item \question{Montrer qu'il existe des endomorphismes $u_1,\dots,u_p$ tels que pour tout
    polynôme $P \in  K[X]$, on ait :\par
    $P(u) = \sum_{i=1}^p P(\lambda_i) u_i$.}
  \item \question{Montrer qu'il existe un polynôme $P_i$ tel que $u_i = P_i(u)$.}
\end{enumerate}
\begin{enumerate}

\end{enumerate}
}