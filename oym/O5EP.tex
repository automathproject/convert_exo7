\uuid{O5EP}
\exo7id{4536}
\titre{Fonction définie par une série}
\theme{Exercices de Michel Quercia, Suites et séries de fonctions}
\auteur{quercia}
\date{2010/03/14}
\organisation{exo7}
\contenu{
  \texte{Soit $u_n(x) = (-1)^n\ln\left(1+\frac x{n(1+x)}\right)$ et
$f(x) = \sum_{n=1}^\infty u_n(x)$.}
\begin{enumerate}
  \item \question{Montrer que la série $f(x)$ converge simplement sur $\R^+$.}
  \item \question{Majorer convenablement le reste de la série, et montrer qu'il y a
    convergence uniforme sur $\R^+$.}
  \item \question{Y a-t-il convergence normale ?}
\end{enumerate}
\begin{enumerate}
  \item \reponse{CSA $ \Rightarrow  |R_n(x)| \le |u_{n+1}(x)| \le
             \ln\left(1+\frac1{n+1}\right)$.}
  \item \reponse{Non, $\|u_n\|_\infty = \ln\left(1+\frac1n\right)$.}
\end{enumerate}
}