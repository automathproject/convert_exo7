\uuid{2a7n}
\exo7id{3817}
\titre{Centrale MP 2001}
\theme{Exercices de Michel Quercia, Problèmes matriciels}
\auteur{quercia}
\date{2010/03/11}
\organisation{exo7}
\contenu{
  \texte{}
\begin{enumerate}
  \item \question{Montrer que toute matrice symétrique réelle positive a ses coefficients
diagonaux positifs. Montrer que si l'un des coefficients diagonaux $u_{ii}$
est nul, alors pour tout $j$ on a $u_{ij} = 0$.}
  \item \question{$U$ est une matrice symétrique réelle positive de la forme
$U = \begin{pmatrix}A&C\cr{}^tC&B\cr\end{pmatrix}$ avec $A$ et $B$ carrées.
Montrer que la matrice $U' = \begin{pmatrix}A&C\cr0&0\cr\end{pmatrix}$ est diagonalisable.}
\end{enumerate}
\begin{enumerate}
  \item \reponse{Inégalité de Cauchy-Schwarz.}
  \item \reponse{Il existe $P$ orthogonale de même taille que $A$ telle
que $D = {}^tPAP$ est diagonale positive.

Alors $\begin{pmatrix}{}^tP&0\cr0&I\cr\end{pmatrix}U\begin{pmatrix}P&0\cr0&I\cr\end{pmatrix}
= \begin{pmatrix}D&{}^tPC\cr{}^tCP&B\cr\end{pmatrix}$ est symétrique positive
donc si $d_{ii} = 0$ alors la ligne $i$ de ${}^tPC$ est nulle.
Ainsi $\begin{pmatrix}{}^tP&0\cr0&I\cr\end{pmatrix}U'\begin{pmatrix}P&0\cr0&I\cr\end{pmatrix}
= \begin{pmatrix}D&{}^tPC\cr0&0\cr\end{pmatrix}$ est, après renumérotation éventuelle des
lignes et colonnes,
de la forme $U'' = \begin{pmatrix}D'&C'\cr0&0\cr\end{pmatrix}$ où $D'$ est diagonale inversible
et $U'$ est semblable à $U''$.
Enfin $U''$ est diagonalisable~:
$\begin{pmatrix}I&D'^{-1}C'\cr0&I\cr\end{pmatrix}U''\begin{pmatrix}I&-D'^{-1}C'\cr0&I\cr\end{pmatrix} = \begin{pmatrix}D'&0\cr0&0\cr\end{pmatrix}$.}
\end{enumerate}
}