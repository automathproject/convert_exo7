\uuid{6920}
\titre{Exercice 6920}
\theme{}
\auteur{ruette}
\date{2013/01/24}
\organisation{exo7}
\contenu{
  \texte{}
  \question{Soit $U$ et $V$ deux variables aléatoires de même loi, à
valeurs dans $\{1,\ldots, N\}$. On pose $X=U-V$ et $Y=U+V$. Déterminer la
covariance entre $X$ et $Y$.}
  \reponse{$U$ et $V$ étant à valeurs dans $\{1, \dots ,N\}$, elles admettent des moments de tout ordre, donc $X$ et $Y$ aussi, les calculs qui suivent sont donc justifiés.
$\text{Cov}(X,Y)=E(XY)-E(X)E(Y)=E(XY)$ car $E(X)=E(U)-E(V)=0$ ($U, V$ ont
même loi).
$E(XY)=E((U-V)(U+V))=E(U^2)-E(V^2)=0$ ($U, V$ ont
même loi). D'où $\text{Cov}(X,Y)=0$.}
}