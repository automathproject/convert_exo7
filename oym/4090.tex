\uuid{4090}
\titre{$f'' + f' + f \to 0$}
\theme{Exercices de Michel Quercia, \'Equations différentielles linéaires (II)}
\auteur{quercia}
\date{2010/03/11}
\organisation{exo7}
\contenu{
  \texte{}
  \question{Soit $f : \R \to\R$ de classe $\mathcal{C}^2$ telle que
$f''(t) + f'(t) + f(t) \to 0$ lorsque $t\to+\infty$. Démontrer que $f(t) \to 0$ lorsque $t\to\infty$.}
  \reponse{On pose $\varphi(t) = f''(t)+f'(t)+f(t)$.

$f(t) = e^{-t/2} \left[ -\frac 2{\sqrt3}
         \int_{u=0}^t \varphi(u)e^{u/2} \sin\left({\frac {\sqrt3\,(u-t)}2 }\right), d u
        + A\cos\left(\frac {\sqrt3\,u}2 \right) + B\sin\left(\frac {\sqrt3\,u}2\right)
        \right ]$.}
}