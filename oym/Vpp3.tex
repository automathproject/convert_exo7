\uuid{Vpp3}
\exo7id{4175}
\titre{Les racines d'un polynôme sont des fonctions $\mathcal{C}^\infty$ des coefficients}
\theme{Exercices de Michel Quercia, Dérivées partielles}
\auteur{quercia}
\date{2010/03/11}
\organisation{exo7}
\contenu{
  \texte{Soit $U$ l'ensemble des polynômes à coefficients réels de degré~$n$
et à racines réelles simples.}
\begin{enumerate}
  \item \question{Montrer que $U$ est ouvert dans~$\R_n[X]$.}
  \item \question{Pour $P\in U$ on note $x_1 < x_2 < \dots < x_n$ les racines de~$P$.
Montrer que l'application $P  \mapsto (x_1,\dots,x_n)$ est de classe $\mathcal{C}^\infty$.}
\end{enumerate}
\begin{enumerate}

\end{enumerate}
}