\uuid{1238}
\titre{Exercice 1238}
\theme{}
\auteur{legall}
\date{1998/09/01}
\organisation{exo7}
\contenu{
  \texte{}
\begin{enumerate}
  \item \question{Soit $  f : { \Rr} \rightarrow { \Rr}   $ la fonction d\'efinie
par $  f(x)=0  $ si $  x \leq 0  $ et $  \displaystyle{ f(x)=\hbox{exp }(\frac{-1}{ x})  }$
sinon. Calculer, pour tout $  n \in { \Nn}  ,$ le d\'eveloppement limit\'e de $  f   $ en $  0  .$
Quelles conclusions en tirer~?}
  \item \question{Soit $  g : { \Rr} \rightarrow { \Rr}   $ la fonction d\'efinie
par $  g(0)=0  $ et, si $  x \not = 0  :$  $  \displaystyle{ g(x)=x^3\hbox{sin}
(\frac{1}{ x})  }.$ Montrer que $  g  $ a un d\'eveloppement limit\'e d'ordre $  2  $ en $  0  $
mais n'a pas de d\'eriv\'ee seconde (en $  0  $).}
\end{enumerate}
\begin{enumerate}

\end{enumerate}
}