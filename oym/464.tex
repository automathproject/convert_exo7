\uuid{464}
\titre{Exercice 464}
\theme{Propriétés de $\Rr$, Maximum, minimum, borne supérieure...}
\auteur{bodin}
\date{1998/09/01}
\organisation{exo7}
\contenu{
  \texte{}
  \question{Le maximum de deux nombres $x,y$ (c'est-\`a-dire le plus grand des
deux) est not\'e $\max(x,y)$. De m\^eme on notera $\min(x,y)$ le plus petit des deux nombres
$x,y$. D\'emontrer que :
$$ \max(x,y)= \frac{x+y+ \vert x-y\vert}{ 2} \quad \hbox{et}\quad  \min(x,y)= \frac{x+y- \vert
x-y\vert}{ 2}. $$
Trouver une formule pour $\max(x,y,z)$.}
  \reponse{Explicitons la formule pour $\max(x,y)$. Si $x\geqslant y$, alors $|x-y|
= x-y$ donc $\frac12(x+y+|x-y|) = \frac12(x+y+x-y) = x$. De m\^eme
si $x \leqslant y$, alors $|x-y| = -x + y$ donc $\frac12(x+y+|x-y|) =
\frac12(x+y-x+y) = y$.

Pour trois \'el\'ements, nous avons $\max(x,y,z) = \max
\big(\max(x,y),z\big)$, donc d'apr\`es les formules pour deux
\'el\'ements :
\begin{align*}
\max(x,y,z) &= \frac{\max(x,y)+z+| \max(x,y)-z |}{2} \\
&= \frac{\frac12(x+y+|x-y|)+z+\left|\frac12(x+y+|x-y|) -z
\right|}{2}.
\end{align*}}
}