\uuid{3858}
\titre{Mines MP 2002}
\theme{}
\auteur{quercia}
\date{2010/03/11}
\organisation{exo7}
\contenu{
  \texte{}
  \question{Soit $f : \R \to \R$ continue telle qu'existe $a$ vérifiant
$f\circ f(a)=a$. $f$ a-t-elle des points fixes~? Généraliser.}
  \reponse{En posant $b=f(a)$ on a $(f(a)-a) + (f(b)-b) = 0$
donc $x \mapsto f(x)-x$ s'annule entre $a$ et $b$.
De même, s'il existe $k\in\N^*$ et $a\in\R$ tels que $f^k(a)=a$
alors $(f(a)-a) + (f^2(a)-f(a)) + \dots + (f^{k}(a)-f^{k-1}(a)) = 0$
donc $f(x)-x$ s'annule entre $\min(a,f(a),\dots,f^{k-1}(a))$ et
$\max(a,f(a),\dots,f^{k-1}(a))$.}
}