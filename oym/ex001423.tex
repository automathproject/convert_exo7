\exo7id{1423}
\titre{Exercice 1423}
\theme{}
\auteur{ortiz}
\date{1999/04/01}
\organisation{exo7}
\contenu{
  \texte{Soient $a=(1,2)(3,4),b=(1,3)(2,4),c=(1,4)(2,3)\in
\mathcal{A}_4,$ $X=\left\{a,b,c\right\},$
$V=\left\{a,b,c,\text{Id}\right\}$ et
$\Phi:\mathcal{S}_4\to\mathcal{S}(X),g\in G\mapsto
\Phi_g=\left[ x\mapsto gxg^{-1}\right].$}
\begin{enumerate}
  \item \question{\begin{enumerate}}
  \item \question{Montrer que $V$ est un sous-groupe distingu\'e de $\mathcal{A}_4$
(on pourra \'etudier l'ordre des \'elements de
$\mathcal{A}_4$).}
  \item \question{Montrer que $<a>$ est un sous-groupe distingu\'e de $V$ et n'est
pas un sous-groupe distingu\'e de $\mathcal{A}_4.$}
\end{enumerate}
\begin{enumerate}

\end{enumerate}
}