\uuid{6Sax}
\exo7id{4672}
\titre{S{\'e}rie altern{\'e}e}
\theme{Exercices de Michel Quercia, Suites convergentes}
\auteur{quercia}
\date{2010/03/16}
\organisation{exo7}
\contenu{
  \texte{On pose $u_n = \frac {96\times(-1)^n}{(2n-3)(2n-1)(2n+1)(2n+3)(2n+5)}$
et $v_n = \sum_{k=0}^n\,u_k$.}
\begin{enumerate}
  \item \question{{\'E}tudier les suites $(v_{2n})$ et $(v_{2n+1})$ et
    montrer que la suite $(v_n)$ est convergente.}
  \item \question{Calculer $\ell = \lim_{n\to\infty} v_n$ {\`a} $10^{-5}$ pr{\`e}s.}
\end{enumerate}
\begin{enumerate}
  \item \reponse{$\ell = \pi$.}
\end{enumerate}
}