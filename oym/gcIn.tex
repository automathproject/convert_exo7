\uuid{gcIn}
\exo7id{4384}
\titre{Intégrales triples}
\theme{Exercices de Michel Quercia, Intégrale multiple}
\auteur{quercia}
\date{2010/03/12}
\organisation{exo7}
\contenu{
  \texte{Calculer $\iiint_D^{} f(x,y,z)\,d xd yd z$ :}
\begin{enumerate}
  \item \question{$D=\{0\le x\le1, 0\le y\le1, 0\le z\le1\}$,   \par\nobreak $f(x,y,z)=\frac1{(x+y+z+1)^3}$.}
  \item \question{$D=\{x^2+y^2+z^2 \le R^2\}$,                  \par\nobreak $f(x,y,z)=\frac1{\sqrt{a^2-x^2-y^2-z^2}} \quad(a>R>0)$.}
  \item \question{$D=\{x\ge0, y\ge0, z\ge0, x+y+z\le1\}$,       \par\nobreak $f(x,y,z)=xyz$.}
  \item \question{$D=\{x\ge0, y\ge0, z\ge0, x+y+z\le1\}$,       \par\nobreak $f(x,y,z)=\frac1{(x+y+z+1)^2}$.}
  \item \question{$D=\{x^2+y^2\le R^2, 0\le z\le a\}$,          \par\nobreak $f(x,y,z)=x^3+y^3+z^3-3z(x^2+y^2)$.}
  \item \question{$D=\{x^2+y^2\le z^2, 0\le z\le 1\}$,          \par\nobreak $f(x,y,z)=\frac z{(x^2+y^2+1)^2}$.}
  \item \question{$D=\left\{\frac{x^2}{a^2} + \frac{y^2}{b^2} + \frac{z^2}{c^2} \le 1\right\}$, \par\nobreak  $f(x,y,z)=x^2+y^2$.}
\end{enumerate}
\begin{enumerate}
  \item \reponse{$\frac12 \ln\left(\frac{32}{27}\right)$.}
  \item \reponse{$2\pi a^2\Arcsin\frac Ra -2\pi R\sqrt{a^2-R^2}$.}
  \item \reponse{$\frac1{720}$.}
  \item \reponse{$\frac34 -\ln2$.}
  \item \reponse{$\frac{\pi R^2a^2}4 (a^2+3R^2)$.}
  \item \reponse{$\frac\pi2 (1-\ln 2)$.}
  \item \reponse{$\frac{4\pi}{15} abc (a^2+b^2)$.}
\end{enumerate}
}