\uuid{0FMF}
\exo7id{2636}
\titre{Exercice 2636}
\theme{Différentielles et dérivées partielles secondes}
\auteur{debievre}
\date{2009/05/19}
\organisation{exo7}
\contenu{
  \texte{}
\begin{enumerate}
  \item \question{Y a-t-il une fonction $g\colon \R^2\to\R$ telle que
\[
 \mathrm{d} g = x^2y^2 \mathrm{d} x + x^3 y\mathrm{d} y?
\]}
  \item \question{Trouver les fonctions $b\colon\R^2\to\R$ telles qu'il existe 
$g\colon \R^2\to\R$ satisfaisant \`a la condition
\[
\mathrm{d} g = x^2y^2 \mathrm{d} x + b(x,y)\mathrm{d} y.
\]
\'Etant donn\'ee alors la fonction $b$, d\'eterminer toutes les fonctions $g$
correspondantes.}
\end{enumerate}
\begin{enumerate}
  \item \reponse{La forme diff\'erentielle
$x^2y^2 \mathrm{d} x + x^3 y\mathrm{d} y$  de degr\'e 1 n'est pas ferm\'ee
car la forme diff\'erentielle de degr\'e 2
\[
d(x^2y^2 \mathrm{d} x + x^3 y\mathrm{d} y) 
= 2x^2y \mathrm{d} y\mathrm{d} x + 3x^2 y\mathrm{d} x\mathrm{d} y
= x^2 y \mathrm{d} x\mathrm{d} y 
\]
est non nulle. Par cons\'equent, une fonction $g\colon \R^2\to\R$ du 
type cherch\'e
ne peut pas exister.}
  \item \reponse{Une fonction $b$ du 
type cherch\'e doit satisfaire \`a l'\'equation diff\'erentielle
partielle
\[
2x^2y -\frac{\partial b}{\partial x} =0
\]
d'o\`u $b(x,y)= \frac 23 x^3 y +k(y)$ o\`u $k$ est une fonction
de la variable $y$. Une fonction $g$ correspondante doit alors satisfaire
aux \'equations diff\'erentielles partielles
\[
\frac{\partial g}{\partial x}=x^2 y^2,
\
\frac{\partial g}{\partial y}=\frac 23 x^3 y +k(y).
\]
Il s'ensuit que $g$ est de la forme
$g(x,y)=\frac 13 x^3 y^2 +K(y)$ o\`u $K$ est une fonction de la varriable $y$.}
\end{enumerate}
}