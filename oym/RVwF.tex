\uuid{RVwF}
\exo7id{4556}
\titre{Centrale MP 2002}
\theme{Exercices de Michel Quercia, Suites et séries de fonctions}
\auteur{quercia}
\date{2010/03/14}
\organisation{exo7}
\contenu{
  \texte{Soit $f : \R \to \R$ continue et $2\pi$-périodique. Pour $n\in\N^*$, on pose
$F_n(x)=\frac{1}{n} \int_{t=0}^nf(x+t)f(t)\,d t$.}
\begin{enumerate}
  \item \question{Montrer que la suite $(F_n)$ converge vers une fonction $F$ que l'on précisera.}
  \item \question{Nature de la convergence~?}
  \item \question{Prouver $\|F\|_{\infty}= |F(0)|$.}
\end{enumerate}
\begin{enumerate}
  \item \reponse{Soit $k=\lfloor n/2\pi\rfloor$.
    On a $F_n(x) = \frac{2k\pi}n \int_{t=0}^{2\pi}f(x+t)f(t)\,d t + \frac1n \int_{t=2k\pi}^nf(x+t)f(t)\,d t
    \to \int_{t=0}^{2\pi}f(x+t)f(t)\,d t$  lorsque $n\to\infty$.}
  \item \reponse{Uniforme.}
  \item \reponse{Cauchy-Schwarz.}
\end{enumerate}
}