\uuid{WTYX}
\exo7id{4899}
\titre{Orthocentre}
\theme{Exercices de Michel Quercia, Propriétés des triangles}
\auteur{quercia}
\date{2010/03/17}
\organisation{exo7}
\contenu{
  \texte{Soit $ABC$ un triangle.
On note :
$\alpha \equiv (\overline{ \vec{AB}, \vec{AC}})$,
$\beta  \equiv (\overline{ \vec{BC}, \vec{BA}})$,
$\gamma \equiv (\overline{ \vec{CA}, \vec{CB}})$.}
\begin{enumerate}
  \item \question{Soit $A'$ le pied de la hauteur issue de $A$.
    Calculer $\frac{\overline{A'B}}{\overline{A'C}}$.}
  \item \question{En déduire les coordonnées barycentriques de l'orthocentre $H$.}
\end{enumerate}
\begin{enumerate}
  \item \reponse{$-\frac{\tan \gamma}{\tan \beta}$.}
  \item \reponse{$H = \text{Bar}(A:\tan\alpha, B:\tan\beta, C:\tan\gamma)
                = \text{Bar}\left(A:\frac a{\cos\alpha},
                                  B:\frac b{\cos\beta},
                                 C:\frac c{\cos\gamma}\right)$.}
\end{enumerate}
}