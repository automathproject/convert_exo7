\uuid{3767}
\titre{Diagonalisation de $C{}^tC$}
\theme{Exercices de Michel Quercia, Endomorphismes auto-adjoints}
\auteur{quercia}
\date{2010/03/11}
\organisation{exo7}
\contenu{
  \texte{}
  \question{Soient $a_1,\dots,a_n \in \R$ et $M = (a_ia_j) \in \mathcal{M}_n(\R)$.
Montrer que $M$ est diagonalisable et déterminer ses éléments propres.}
  \reponse{Si tous les $a_i$ sont nuls, $M = 0$.\par
         Sinon, $M = C^tC  \Rightarrow  E_0 = C^\perp$ et $E_\nu = \text{vect}(C)$
         avec $\nu = \|C\|^2$.}
}