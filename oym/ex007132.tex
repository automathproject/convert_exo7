\exo7id{7132}
\titre{Ptolémée, preuve géométrique du sens direct}
\theme{}
\auteur{megy}
\date{2017/02/08}
\organisation{exo7}
\contenu{
  \texte{% similitudes

On considère l'énoncé suivant :

\begin{theoreme}[Ptolémée] Soit $ABCD$ un quadrilatère convexe, direct. Alors $A, B, C, D$ sont cocycliques si et seulement si $ AC\cdot BD = AB\cdot CD + BC\cdot AD$.
\end{theoreme}

L'objectif de cet exercice est de prouver le sens direct du théorème.

On suppose $A, B, C, D$ cocycliques.}
\begin{enumerate}
  \item \question{Faire une figure au brouillon et montrer que $\widehat{BAC} = \widehat{BDC}$ et trois relations similaires sur d'autres angles.}
  \item \question{Soit $K$ le point de la diagonale $[AC]$ tel que $\widehat{ABK} = \widehat{DBC}$. Faire une figure et construire $K$ en expliquant (faire la figure de telle sorte que $K$ soit lisible).}
  \item \question{Montrer que les triangles $ABK$ et $DBC$ sont semblables, de même que $ABD$ et $KBC$, par des similitudes dont on précisera les centres et les rapports. Note : il suffit pour cela de montrer qu'ils ont mêmes angles. En déduire des relations sur les côtés de ces triangles.}
  \item \question{Conclure.}
\end{enumerate}
\begin{enumerate}

\end{enumerate}
}