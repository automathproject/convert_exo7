\uuid{2263}
\titre{Exercice 2263}
\theme{Anneaux de polynômes I}
\auteur{barraud}
\date{2008/04/24}
\organisation{exo7}
\contenu{
  \texte{}
  \question{Soit $f(x)\in A[x]$ un polyn\^ome
sur un anneau $A$. Supposons que $(x-1)\,|\,f(x^n)$. Montrer que
$(x^n-1)\,|\,f(x^n)$.}
  \reponse{Notons $f(x^{n})=P(x-1)$. Alors $f(1)=0\cdot P(1)=0$ et donc $(x-1)|f$.
  Notons $f=Q(x-1)$. On a alors $f(x^{n})=Q(x^{n})(x^{n}-1)$. $(x^{n}-1)$
  divise bien $f$.}
}