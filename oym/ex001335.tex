\exo7id{1335}
\titre{Exercice 1335}
\theme{}
\auteur{cousquer}
\date{2003/10/01}
\organisation{exo7}
\contenu{
  \texte{On appelle \emph{ordre d'un élément} d'un groupe fini $(G,\ast)$ l'ordre du
sous-groupe engendré dans $G$ par cet élément.}
\begin{enumerate}
  \item \question{Montrer que si $x$ est d'ordre $p$, $p$ est le plus petit entier tel que
$x^p=e$.}
  \item \question{Déterminer les ordres des éléments des groupes rencontrés au 
\textbf{I}.}
  \item \question{Soit $(G,\ast)$ un groupe fini, $a$ un élément de $G$, $H$ un sous-groupe
d'ordre $p$ de $G$~; on note $aH$ l'ensemble $\{a\ast y\mid y\in 
H\}$.\newline
a) Montrer que pour tout $a\in G$, $aH$ a $p$ éléments.\newline
b) Montrer que si $a\in G$ et $b \in G$, $(aH=bH)$ ou $(aH\cap
bH=\emptyset)$.\newline 
c) En déduire que l'ordre de $H$ divise l'ordre de~$G$.}
  \item \question{Montrer que si $G$ est un groupe fini d'ordre $n$, les ordres de tous ses
éléments divisent~$n$.}
  \item \question{Trouver des sous-groupes de $\mathbb{Z}_2$, $\mathbb{Z}_3$, 
$\mathbb{Z}_4$, $\mathbb{Z}_5$, $\mathbb{Z}_6$, 
$\frak{S}_2$, $\frak{S}_3$.}
  \item \question{Si $G$ est un groupe d'ordre~$5$, que peut-on dire de l'ordre de ses
éléments~? En déduire les tables de composition possibles pour un groupe
d'ordre~$5$. Que peut-on dire de deux groupes quelconques d'ordre~$5$~? Mêmes
questions pour les groupes d'ordre~$23$. Généraliser.}
\end{enumerate}
\begin{enumerate}

\end{enumerate}
}