\uuid{4115}
\titre{Centrale MP 2004}
\theme{}
\auteur{quercia}
\date{2010/03/11}
\organisation{exo7}
\contenu{
  \texte{Soit $n>0$ et $(S)\Leftrightarrow\begin{cases}x'(t) = \frac2nx(t)y(t)\cr y'(t) = -x^2(t)+y^2(t).\cr\end{cases}$}
\begin{enumerate}
  \item \question{Soit $\gamma$ : $t \mapsto(x(t),y(t))$ une solution de~$(S)$. Trouver
une autre solution présentant une symétrie avec $\gamma$. Peut-on
avoir comme solution $\sigma(t) = \lambda\gamma(\mu t)$~? En déduire
une propriété géométriques des solutions maximales de~$(S)$.}
  \item \question{Déterminer les courbes du plan formées des points $(x_0,y_0)$ où
les solutions de~$(S)$ ont des tangentes parallèles aux axes $(Ox)$ et
$(Oy)$. En déduire quelques solutions particulières.}
  \item \question{A supposer qu'il existe $\Phi : {I\subset \R} \to \R$ telle que
$\gamma(t) = (x(t),y(t))$ vérifie $y(t) = \Phi(x(t))$, déterminer
$\Phi$ et en déduire toutes les courbes intégrales.}
\end{enumerate}
\begin{enumerate}
  \item \reponse{$\gamma_1(t) = (-x(t),y(t))$ et $\gamma_2(t) = (x(-t),-y(-t))$
sont aussi solutions de~$(S)$.

Par ailleurs, la théorie de Cauchy-Lipschitz s'applique, en particulier
s'il existe $t_0$ tel que $x(t_0) = 0$ alors $x(t) = 0$ pour tout $t$.
De même s'il existe $t_0$ tel que $x(t_0) = y(t_0) = 0$ alors
$x(t)=y(t)=0$ pour tout~$t$.

Pour $\lambda,\mu$ non nuls et $x$ ne s'annulant pas, $t \mapsto(\lambda x(\mu t), \lambda y(\mu t))$
est solution de~$(S)$ si et seulement si $\mu=\lambda$.

L'ensemble des trajectoires maximales
est donc stable par les symétries par rapport aux deux axes et par les
homothéties de centre~$(0,0)$. De plus toute
trajectoire maximale qui touche l'axe des $x$ est symétrique par
rapport à cet axe.}
  \item \reponse{$x'(t_0) = 0 \Leftrightarrow x(t_0) = 0$ ou $y(t_0) = 0$. Dans le premier cas on
a $x(t) = 0$ pour tout $t$ et $y(t)$ est arbitraire (solution de $y' = y^2$).
Dans le second cas $x(t_0)$ est arbitraire (Cauchy-Lipschitz) donc
l'ensemble des points à tangente verticale est la réunion des deux
axes privé de $(0,0)$ (où il n'y a pas de tangente).

$y'(t_0) = 0 \Leftrightarrow x(t_0) = \pm y(t_0)$, quantité arbitraire, donc
l'ensemble des points à tangente horizontale est la réunion des deux
bissectrices des axes, privée de $(0,0)$.

Solutions particulières~: $x(t)=0, y(t) = \frac{1}{\lambda-t}$.}
  \item \reponse{En supposant $\Phi$ de classe $\mathcal{C}^1$ on obtient l'équation
$\frac2nx\Phi\Phi' = \Phi^2-x^2$ soit $\frac2nx\psi' = \psi-x^2$ avec $\psi=\Phi^2$.
On obtient $\psi(x) = |x|^n\Bigl(\lambda + \frac{n}{(n-1)x}\Bigr)$ si $n\ne 1$
et $\psi(x) = |x|(\lambda  - \ln|x|)$ si $n=1$.

Une courbe intégrale (en fait une trajectoire) qui
ne touche aucun des deux axes vérifie l'hypothèse $y = $
fonction de $x$ car $x'$ ne peut s'annuler donc $x$ est une fonction
injective de~$t$. Une trajectoire qui touche l'axe des $y$ est incluse
dans cet axe (déjà vu) et une trajectoire qui touche l'axe des $x$ en
dehors de $(0,0)$ le traverse ($y' \ne 0$), donc est réunion de
sous-arcs localement d'un seul côté de l'axe des $x$,
de la forme $y = \Phi(x)=\pm\sqrt{\psi(x)}$.}
\end{enumerate}
}