\exo7id{2838}
\titre{Exercice 2838}
\theme{}
\auteur{burnol}
\date{2009/12/15}
\organisation{exo7}
\contenu{
  \texte{}
\begin{enumerate}
  \item \question{Soit $\gamma:[0,1]\to\Cc\setminus\{0\}$ un
lacet et soit $N\in \Zz$ son indice par rapport à $0$.  En
utilisant  la notion de variation de
l'argument, montrer qu'il existe une fonction
continue $g:[0,1]\to\Cc$ telle que $\forall t\quad \gamma(t)
= e^{g(t)}$ et  $g(1) - g(0) = 2\pi i N$.  Montrer
 que toute autre fonction continue $G$ avec
$\forall t\quad \gamma(t) = e^{G(t)}$  est de la
forme $g+2\pi i k$ pour un certain $k\in\Zz$. On pose
$h(t,u) = (1-u)\;2\pi i N\,t + ug(t)$ puis $H(t,u) =
e^{h(t,u)}$. Montrer que pour chaque $u\in[0,1]$
l'application $t\mapsto H(t,u)$ est un lacet. En déduire que
le lacet $c_N(t) = e^{2\pi i\; Nt}$ et $\gamma$ sont
homotopes dans $\Cc\setminus\{0\}$.}
  \item \question{On considère le lacet obtenu en suivant d'abord $c_N$ puis
  $c_M$. Montrer que ce lacet est homotope dans
  $\Cc\setminus\{0\}$ au lacet $c_{N+M}$ (il suffit de
  calculer son indice!).}
\end{enumerate}
\begin{enumerate}
  \item \reponse{Le calcul de l'indice de $c_N$ est \'evident. L'affirmation sur l'existence de $g$ continue telle que $\gamma =e^g$ et $g(1)-g(0)=2\pi i N$ est le contenu du polycopi\'e de J.-F.~Burnol 2005/2006, chapitre 30. Le reste est laiss\'e au lecteur.}
\end{enumerate}
}