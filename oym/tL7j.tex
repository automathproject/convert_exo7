\uuid{tL7j}
\exo7id{4549}
\titre{Fonctions réciproques (Pugin, MP$^*$-2001)}
\theme{Exercices de Michel Quercia, Suites et séries de fonctions}
\auteur{quercia}
\date{2010/03/14}
\organisation{exo7}
\contenu{
  \texte{Soit $(f_n)$ une suite de fonctions $[a,b]\to[c,d]$ continues, bijectives, strictement
croissantes, convergeant simplement vers une fonction $f$ : $[a,b]\to[c,d]$ elle aussi
continue, bijective strictement croissante.}
\begin{enumerate}
  \item \question{Montrer qu'il y a convergence uniforme (deuxième théorème de Dini, considérer une subdivision de~$[a,b]$).}
  \item \question{Montrer que les fonctions réciproques $f_n^{-1}$ convergent simplement
    vers une fonction~$g$ et que $g = f^{-1}$.}
  \item \question{Montrer que $(f_n^{-1})$ converge uniformément vers~$f^{-1}$.}
\end{enumerate}
\begin{enumerate}
  \item \reponse{soit $y\in{[c,d]}$ et $x_n = f_n^{-1}(y)$. La suite $(x_n)$ admet au plus
         une valeur d'adhérence, $x = f^{-1}(y)$.}
\end{enumerate}
}