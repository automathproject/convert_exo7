\uuid{3838}
\titre{Forme quadratique associée à la matrice de Gram}
\theme{Exercices de Michel Quercia, Espaces vectoriels hermitiens}
\auteur{quercia}
\date{2010/03/11}
\organisation{exo7}
\contenu{
  \texte{}
  \question{Soit $E$ un espace euclidien, $(\vec e_1,\dots,\vec e_n)$ une base de $E$,
$G$ sa matrice de Gram et $G^{-1} = (a_{ij})$.

Montrer que :
$\forall\ \vec x\in E,\ \sum_{i,j} a_{ij}
  (\vec e_i\mid \vec x) (\vec e_j\mid \vec x) = \|\vec x\,\|^2$.}
  \reponse{Soit $\cal B$ une base orthonormée de $E$ et $P$ la matrice de
passage de $\cal B$ à $(\vec e_i)$.

Le premier membre vaut ${}^tXPG^{-1}{}^tPX = {}^t XX$.}
}