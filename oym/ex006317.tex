\uuid{6317}
\titre{Exercice 6317}
\theme{}
\auteur{queffelec}
\date{2011/10/16}
\organisation{exo7}
\contenu{
  \texte{On considère les deux équations différentielles du second ordre
$$({\cal E}_1)\quad\quad y''=\sin x\qquad({\cal E}_2)\quad\quad y''+ \omega^2y=
\sin x$$ où $\omega$ est un nombre réel de module strictement inférieur à
$1$.}
\begin{enumerate}
  \item \question{Trouver la solution $y$ de l'équation ${\cal E}_1$ vérifiant
$y(0)=0,\   y(\pi)=4\pi.$}
  \item \question{Décrire la solution générale de l'équation ${\cal E}_2$, et prouver
ainsi que la solution
$y_\omega$ vérifiant $y_\omega(0)=0,\   y_\omega(\pi)=4\pi$, a pour expression
$$y_\omega(x)=4\pi{{\sin\omega x}\over{\sin\pi\omega}}- {{\sin x}\over{1-\omega^2}}.$$}
  \item \question{Trouver, à $x$ fixé, la limite de $y_\omega(x)$ quand $\omega$ tend
vers $0$. Interprétation.}
  \item \question{On  restreint $x$ à parcourir l'inter\-valle
$[0,\pi]$, et on suppose $0\leq\omega\leq{1\over2}$. A l'aide
 de la formule de Taylor, 
%à la fonction $f(\omega)=\pi\sin\omega x-x\sin\pi\omega$, 
montrer que 
$\vert
\pi\sin\omega x-x\sin\pi\omega\vert\leq\pi^3\omega^3$.
 En déduire :  $\vert y_\omega(x)-y(x)\vert \leq A\omega^2$,
où $A$ est une constante.}
\end{enumerate}
\begin{enumerate}

\end{enumerate}
}