\uuid{2814}
\titre{Exercice 2814}
\theme{}
\auteur{burnol}
\date{2009/12/15}
\organisation{exo7}
\contenu{
  \texte{}
  \question{Soit $0<a<b$ sur l'axe réel positif et soit $C=\{|z|=r\}$ le
cercle de rayon $r$ centré en l'origine, parcouru dans le
sens direct. Montrer:
\[ \int_C \frac1{(z-a)(z-b)}\,dz = 2\pi i \begin{cases}
0& r<a\\
\frac1{a-b}& a<r<b\\
0& r>b
\end{cases}\]
On pourra réduire la fraction en élément simples, puis se
 ramener au résultat de l'exercice précédent. Ou encore, on
 pourra envisager des développements en
séries, pour se ramener par étapes aux intégrales $\int_C
z^n dz$, $n\in\Zz$.}
  \reponse{On a
$$\frac{1}{(z-a)(z-b)}=\frac{1}{a-b}\left( \frac{1}{z-a}-\frac{1}{z-b}\right)$$
$$\int _C \frac{1}{z-a}dz = 2i\pi \mathrm{Ind} (C,a).$$
Or, $\mathrm{Ind}(C,a)=0$ si $r<a$ et $\mathrm{Ind}(C,a)=1$ si $r>a$. Le m\^eme raisonnement s'applique \`a $\int_C \frac{1}{z-b}dz$,
d'o\`u le r\'esultat annonc\'e.}
}