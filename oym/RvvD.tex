\uuid{RvvD}
\exo7id{3728}
\titre{Matrice des inverses}
\theme{Exercices de Michel Quercia, Formes quadratiques}
\auteur{quercia}
\date{2010/03/11}
\organisation{exo7}
\contenu{
  \texte{Soit $A = (a_{i,j})\in \mathcal{M}_n(\R)$ à coefficients tous non nuls.
On note $A'$ la matrice de coefficient général $1/a_{i,j}$.}
\begin{enumerate}
  \item \question{Trouver les matrices $A$ telles que $A$ et $A'$ sont symétriques définies positives
    (examiner les cas $n=1$, $n=2$, $n=n$).}
  \item \question{Trouver les matrices $A$ telles que $A$ et $A'$ sont symétriques positives
    (examiner les cas $n=2$, $n=3$, $n=n$).}
\end{enumerate}
\begin{enumerate}
  \item \reponse{Il n'y a pas de solution pour $n=2$ donc pas non plus pour $n>2$.}
  \item \reponse{Pour $n=3$ on trouve $A={}^tCC$ où $C$ est une colonne sans zéros,
pour $n\ge 3$ on obtient le même résultat en considérant les blocs $3\times3$
centrés sur la diagonale.}
\end{enumerate}
}