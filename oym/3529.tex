\uuid{3529}
\titre{Ensi Physique 93}
\theme{Exercices de Michel Quercia, Réductions des endomorphismes}
\auteur{quercia}
\date{2010/03/10}
\organisation{exo7}
\contenu{
  \texte{}
  \question{Soient $x_1,\dots,x_n,y_1,\dots,y_n \in \C$.
Calculer $\Delta_n = \det(I+(x_iy_j))$.}
  \reponse{Soit $M = (x_iy_j)$ : $M$ est de rang inférieur ou égal à $1$, donc
$0$ est valeur propre de $M$ d'ordre au moins $n-1$.
Comme $\mathrm{tr}(M) = x_1y_1 + \dots + x_ny_n$, le polynôme caractéristique de
$M$ est $\chi_M(x) = (-x)^{n-1}(x_1y_1 + \dots + x_ny_n - x)$, et le déterminant demandé
est $\Delta_n = \chi_M(-1) = x_1y_1 + \dots + x_ny_n + 1$.}
}