\uuid{4058}
\titre{$y^{(4)}+y''+y = |\sin x|$}
\theme{Exercices de Michel Quercia, \'Equations différentielles linéaires (I)}
\auteur{quercia}
\date{2010/03/11}
\organisation{exo7}
\contenu{
  \texte{}
  \question{Montrer que l'équation : $y^{(4)}+y''+y = |\sin x|$ admet une et une seule solution
$\pi$-périodique.}
  \reponse{$|\sin x| = \frac2\pi -\frac 4\pi\sum_{n=1}^\infty \frac{\cos 2nx}{4n^2-1}
 \Rightarrow  y = \frac2\pi -\frac4\pi\sum_{n=1}^\infty \frac{\cos 2nx}{(4n^2-1)(16n^4-4n^2+1)}$.
\par
Cette série converge et définit une fonction de classe $\mathcal{C}^4$ solution de l'équation.

Unicité : les solutions de l'équation homogène sont combinaison de
$e^{jx}$, $e^{-jx}$, $e^{j^2x}$ et $e^{-j^2x}$ donc non $\pi$-périodiques.}
}