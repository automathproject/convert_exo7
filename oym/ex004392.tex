\uuid{4392}
\titre{Aires}
\theme{}
\auteur{quercia}
\date{2010/03/12}
\organisation{exo7}
\contenu{
  \texte{Calculer l'aire des domaines suivants :}
\begin{enumerate}
  \item \question{$D$ est la partie du disque unité située dans la concavité de
    l'hyperbole d'équation $xy = \frac{\sqrt3}4$.}
  \item \question{$D$ est l'intersection des domaines limités par les ellipses d'équation
    $\frac{x^2}{a^2} + \frac{y^2}{b^2} = 1$ et $\frac{x^2}{b^2} + \frac{y^2}{a^2} = 1$.}
\end{enumerate}
\begin{enumerate}
  \item \reponse{$A= \frac\pi6 -\frac{\sqrt3}4 \ln3$.}
  \item \reponse{$4ab\arctan \frac ba$.}
\end{enumerate}
}