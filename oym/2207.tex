\uuid{2207}
\titre{Exercice 2207}
\theme{Théorème de Sylow}
\auteur{debes}
\date{2008/02/12}
\organisation{exo7}
\contenu{
  \texte{}
  \question{Soient $p$ et $q$ deux nombres premiers. Montrer qu'il n'existe pas de
groupe simple d'ordre $p^2q$.}
  \reponse{Soit $G$ un groupe d'ordre $p^2q$ qu'on suppose simple. On distingue deux cas:
\smallskip

\underbar{1er cas}: $p>q$. Le nombre de $p$-Sylow de $G$ est $\equiv 1\ [\hbox{\rm
mod}\ p]$ et divise $q$. Comme $G$ est simple, ce ne peut \^etre $1$ (car sinon
l'unique $p$-Sylow serait distingu\'e). Il y a donc $q$ $p$-Sylow d'ordre $p^2$,
lesquels sont conjugu\'es. L'action par conjugaison de $G$ sur ces $q$ $p$-Sylow
d\'efinit un morphisme $G \rightarrow S_q$ non trivial (car l'action est
transitive) et   donc injectif puisque le noyau, distingu\'e et $\not=G$, est
forc\'ement trivial. On en d\'eduit que $p^2q$ divise $q!$ et donc $p$ divise un
entier entre $1$ et
$q-1$, ce qui contredit l'hypoth\`ese $p>q$.
\smallskip

\underbar{2\`eme cas}: $p<q$. Le nombre de $q$-Sylow de $G$ est $\equiv 1\
[\hbox{\rm mod}\ q]$ et divise $p^2$. Comme ci-dessus, $G$ \'etant simple, ce ne
peut \^etre $1$. Ce ne peut-\^etre ni $p$ ni $p^2$. En effet, dans le cas
contraire, $p$ serait $\equiv \pm 1\ [\hbox{\rm mod}\ q]$ et donc $p\geq q-1$. Comme
$p<q$, la seule possibilit\'e est $p=q-1$ et donc $p=2$ et $q=3$. Dans ce dernier
cas, il y a $4$ $3$-Sylow d'ordre $3$ qui contiennent $8$ \'el\'ements d'ordre
$3$. Ne reste de la place que pour un seul $2$-Sylow qui devrait \^etre distingu\'e. Ce dernier cas
n'est donc lui non plus pas possible. 
\smallskip

Conclusion: il n'existe pas de groupe $G$ simple d'ordre $p^2q$.}
}