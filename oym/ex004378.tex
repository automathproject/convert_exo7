\uuid{4378}
\titre{Mines-Ponts MP 2004}
\theme{}
\auteur{quercia}
\date{2010/03/12}
\organisation{exo7}
\contenu{
  \texte{Soit $I(a) =  \int_{x=0}^{+\infty}\frac{\sh x}xe^{-ax}\,d x$.}
\begin{enumerate}
  \item \question{Quel est le domaine de définition de~$I$~?}
  \item \question{Étudier la continuité et la dérivabilité de~$I$.}
  \item \question{Calculer $I(a)$.}
\end{enumerate}
\begin{enumerate}
  \item \reponse{$]1,+\infty[$.}
  \item \reponse{$I'(a) = - \int_{x=0}^{+\infty}\sh xe^{-ax}\,d x = \frac12\Bigl(\frac1{a-1}-\frac1{a+1}\Bigr)$.
D'où $I(a) = \frac12\ln\Bigl(\frac{a-1}{a+1}\Bigr) +$ cste et $I(a)\to 0$ lorsque $a\to+\infty$
donc la constante est nulle.}
\end{enumerate}
}