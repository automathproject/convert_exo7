\exo7id{5928}
\titre{Exercice 5928}
\theme{}
\auteur{tumpach}
\date{2010/11/11}
\organisation{exo7}
\contenu{
  \texte{\emph{Le but de cet exercice est de prouver le Th\'eor\`eme de
Carath\'eodory.} 

\textbf{D\'efinition.}
Une mesure ext\'erieure sur un ensemble $\Omega$ est une
application~$m_*~:\mathcal{P}(\Omega) \rightarrow [0,+\infty]$
telle que}
\begin{enumerate}
  \item \question{[(i)] $m_*(\emptyset) = 0~;$}
  \item \question{[(ii)] (monotonie) $A
\subset B \Rightarrow m_*(A) \leq m_*(B)~;$}
  \item \question{[(iii)]
($\sigma$-sous-additivit\'e) Pour toute suite d'ensembles $\{
A_{i} \}_{i\in\mathbb{N}^*} \subset \mathcal{P}(\Omega)$ on a
$$
m_{*}\left(\bigcup_{i=1}^{\infty} A_{i} \right) \leq
\sum_{i=1}^{\infty} m_{*}(A_{i}).
$$}
\end{enumerate}
\begin{enumerate}
  \item \reponse{\begin{enumerate}}
  \item \reponse{cf cours.}
  \item \reponse{clair.}
  \item \reponse{Soit $\{A_{i}\}_{i\in
\mathbb{N}^*}$ un suite quelconque d'ensembles $m_*$-mesurables. On
pose $B_1 = \emptyset$, $B_{2}=A_1$ et $B_{j} = \cup_{i=1}^{j-1}
A_{i}$, pour $j\geq 2$. Soit $Q$ un sous-ensemble de $\Omega$.
Montrons par r\'ecurrence que l'assertion $(P_k)$ suivante est
v\'erifi\'ee pour tout $k\geq 1$~:
$$
(P_k)\quad \quad \quad m_*(Q) = m_*(Q \cap B_{k+1}^c) + \sum
_{j=1}^{k} m_*(Q \cap B_{j}^c \cap A_j).
$$
\begin{itemize}}
  \item \reponse{Pour $k = 1$, $(P_1)$ dit simplement que $m_*(Q) = m_*(Q\cap
A_{1}^c) + m_*(Q\cap A_1)$. Ceci   est une cons\'equence de la
$m_*$-mesurabilit\'e de $A_1$ et de fait que
$$
m_*(Q) \leq  m_*(Q\cap A_{1}^c) + m_*(Q\cap A_1)
$$
(on applique la $\sigma$-sous-additivit\'e de $m_*$ \`a $C_1 =
Q\cap A_{1}^c$, $C_2 = Q\cap A_1$ et $C_i = \emptyset$ pour $i\geq
3$.)}
  \item \reponse{Montrons que $(P_{k}) \Rightarrow (P_{k+1})$~:\\
Puisque $A_{k+1}$ est $m_*$-mesurable, on a~:
$$
m_*(Q\cap B_{k+1}^c) = m_*(Q \cap B_{k+1}^c \cap A_{k+1}^c) +
m_*(Q\cap B_{k+1}^c \cap A_{k+1}).$$ Or $B_{k+1}^c \cap A_{k+1}^c
= (B_{k+1}\cup A_{k+1})^{c} = B_{k+2}^{c}$. Ainsi~:
\begin{equation}\label{la}
m_*(Q\cap B_{k+1}^c) = m_*(Q \cap B_{k+2}^{c}) + m_*(Q\cap
B_{k+1}^c \cap A_{k+1}).
\end{equation}
Supposons que l'assertion $(P_{k})$ soit v\'erifi\'ee, alors
$$
m_*(Q) = m_*(Q \cap B_{k+1}^c) + \sum _{j=1}^{k} m_*(Q \cap
B_{j}^c \cap A_j),
$$
et d'apr\`es \eqref{la}
\begin{eqnarray*} m_*(Q) &=& m_*(Q \cap B_{k+2}^{c})
+ m_*(Q\cap B_{k+1}^c \cap A_{k+1})+ \sum _{j=1}^{k} m_*(Q \cap
B_{j}^c \cap A_j) \\&=& m_*(Q \cap B_{k+2}^{c})+ \sum _{j=1}^{k+1}
m_*(Q \cap B_{j}^c \cap A_j),
\end{eqnarray*}
qui n'est autre que $(P_{k+1})$.}
  \item \reponse{En conclusion, comme
$(P_1)$ est vrai et $(P_{k})\Rightarrow (P_{k+1})$, il en
d\'ecoule que l'assertion $(P_{k})$ est vraie pour tout $k\geq
1$.
\end{itemize}}
  \item \reponse{Comme $B_{k+1} \subset A$, on a $Q \cap B_{k+1}^c \supset Q
\cap A^c$ et, par  monotonie de $m_*$, $$m_*(Q\cap B_{k+1}^c)
\geq m_*(Q \cap A^c).$$ La condition $(P_{k})$ entra\^ine alors
que pour tout $k$~:
$$
m_*(Q) \geq m_*(Q \cap A^c) + \sum_{j=1}^{k} m_*(Q\cap B_{j}^c
\cap A_j).
$$
Donc, en faisant tendre $k$ vers $+\infty$~:
$$
m_*(Q) \geq m_*(Q \cap A^c) + \sum_{j=1}^{\infty} m_*(Q\cap
B_{j}^c \cap A_j).
$$}
  \item \reponse{On a~:$~Q\cap A = \bigcup_{j=1}^{\infty}(Q\cap B_{j}^c \cap
A_{j})~$ et par  $\sigma$-sous-additivit\'e de $m_*$~:
\begin{eqnarray*}
m_*(Q\cap A^c) + m_*(Q\cap A) &=&  m_*(Q\cap A^c) +
m_*\left(\bigcup_{j=1}^{\infty}(Q\cap B_{j}^c \cap A_j\right)\\
& \leq & m_*(Q\cap A^c) + \sum_{j=1}^{\infty} m_*(Q\cap B_{j}^c
\cap A_j)\\&\leq & m_*(Q).
\end{eqnarray*}
On en conclut que $A = \cup_{j=1}^{\infty} A_{j}$ est
$m_*$-mesurable.}
\end{enumerate}
}