\uuid{4811}
\titre{$\mathrm{id} - u$ est bicontinu}
\theme{}
\auteur{quercia}
\date{2010/03/16}
\organisation{exo7}
\contenu{
  \texte{}
  \question{Soit $E$ un $\C$-evn et $u \in {\cal L}_c(E)$ tel que $\mathrm{id}_E - u$ est
bicontinu.
Montrer que pour tout entier $n$, $\mathrm{id}_E - u^n$ est bicontinu et comparer son inverse
{\`a} $\sum_{k=0}^{n-1} \bigl(\mathrm{id}_E - e^{2ik\pi/n}u\bigr)^{-1}$.}
  \reponse{$\frac1{1-X^n} = \frac 1n \sum_{k=0}^{n-1} \frac{1}{1-\omega_k X}$,
$\omega_k = e^{2ik\pi/n}$.
Donc $1 = \sum_{k=0}^{n-1} \frac{1-X^n}{n(1-\omega_k X)}$.

Il s'agit de polyn{\^o}mes, donc on peut remplacer $X$ par $u$, d'o{\`u} :
$(\mathrm{id}_E - u^n)^{-1} = \frac1n \sum_{k=0}^{n-1}\bigl(\mathrm{id}_E - e^{2ik\pi/n}u\bigr)^{-1}$.}
}