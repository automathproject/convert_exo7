\uuid{4884}
\titre{\'Equation barycentrique d'une droite}
\theme{Exercices de Michel Quercia, Barycentres}
\auteur{quercia}
\date{2010/03/17}
\organisation{exo7}
\contenu{
  \texte{}
  \question{Soit $(A,B,C)$ une base affine de ${\cal E}_2$, et $M,M',M''$ trois points de
coordonnées barycentriques
$(\alpha, \beta, \gamma)$,
$(\alpha', \beta', \gamma')$,
$(\alpha'', \beta'', \gamma'')$.

Montrer que $M,M',M''$ sont alignés si et seulement si
$\begin{vmatrix}\alpha& \beta& \gamma \cr
          \alpha'& \beta'& \gamma' \cr
          \alpha''& \beta''& \gamma''\cr\end{vmatrix} = 0$.}
  \reponse{}
}