\exo7id{6030}
\titre{Exercice 6030}
\theme{}
\auteur{quinio}
\date{2011/05/20}
\organisation{exo7}
\contenu{
  \texte{Une compagnie aérienne a demandé des statistiques afin d'améliorer la sûreté au décollage 
et définir un poids
limite de bagages. Pour l'estimation du poids des voyageurs et du poids des
bagages, un échantillon est constitué de 300 passagers qui ont accepté d'être pesés : on a obtenu une moyenne $m_{e}$ de 68kg, 
avec un écart-type $\sigma_{e}$ de 7 kg.}
\begin{enumerate}
  \item \question{Définir un intervalle de confiance pour la moyenne des
passagers. (On admet que le poids des passagers suit une loi normale de moyenne
$m$, d'écart-type $\sigma $.)}
  \item \question{Montrer que l'on peut considérer que le poids des passagers est
une variable aléatoire $X$ de moyenne 70 kg, d'écart-type 8 kg.}
  \item \question{En procédant de même pour le poids des bagages, on admet les résultats :
\begin{itemize}}
  \item \question{Si le poids maximum autorisé est de 20 kg, le poids des bagages
peut être considéré comme une variable aléatoire $Y$ de
moyenne 15 kg, d'écart-type 5 kg.}
  \item \question{La capacité de l'avion est de 300 passagers; l'avion pèse, 
à vide, 250 tonnes. Le décollage est interdit si le poids total dépasse 276.2 tonnes.
Quelle est la probabilité pour que le décollage soit interdit ? 
\end{itemize}}
\end{enumerate}
\begin{enumerate}
  \item \reponse{On peut estimer $m$ par la moyenne de l'échantillon: $68$ kg, et $\sigma$ par
$\sigma _{e}\sqrt{\frac{300}{299}}=7\sqrt{\frac{300}{299}} \simeq 7.0117$ kg. 
On en déduit un intervalle de confiance pour la moyenne $m$ : $I_\alpha = [67.2;68.8]$.}
  \item \reponse{La borne supérieure de l'intervalle étant de $69$ kg, il est
raisonnable de prendre $70$ kg comme espérance de la variable poids d'un passager.}
  \item \reponse{Le décollage est autorisé si le poids total des voyageurs et de
leurs bagages ne dépasse pas $26.2$ tonnes.
Pour chacun des $300$ passagers, notons: $X_{i}$ son poids et $Y_{i}$ le poids de ses bagages.
Faisons l'hypothèse d'indépendance entre les variables $X_{i}$ et $Y_{i}$.
Le poids total $Z=\sum_{i=1}^{300}(X_{i}+Y_{i})$ est la somme de $600$
variables aléatoires indépendantes; le théorème central
limite s'applique sous cette hypothèse.
Comme l'espérance totale est $E(Z)=300 \cdot (70+15)=25\,500$ et la variance
de $Z$ est :
$\text{Var}\, Z=300 \cdot(\text{Var}\,X_{i}+\text{Var}\,Y_{i})$.
 Alors $Z$ suit approximativement une loi
normale de moyenne $m=25\,500$, d'écart-type $\sigma =\sqrt{300\cdot(8^{2}+5^{2})}=163.4$.
Alors $Z' =\frac{Z-m}{\sigma}$ suit approximativement une loi normale
centrée réduite.
Le décollage est interdit si : $Z>26\,200$, c'est-à-dire si  $Z' >4.284$.
On lit dans la table de Gauss: pour $t=4$, $F(t)=0.999\,968=P[Z' \leq 4]$.
Le décollage est interdit pour cause de surcharge pondérale avec une
probabilité inférieure à $0.000\,04$.}
\end{enumerate}
}