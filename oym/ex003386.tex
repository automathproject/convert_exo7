\exo7id{3386}
\titre{$A, A^2, A^3$ données $ \Rightarrow  A^p$}
\theme{}
\auteur{quercia}
\date{2010/03/09}
\organisation{exo7}
\contenu{
  \texte{Soit $A \in \mathcal{M}_n(K)$. On suppose qu'il existe $\lambda,\mu \in  K$ et
$U,V \in \mathcal{M}_n(K)$ tels que :
$\begin{cases} A   = \lambda  U + \mu  V \cr
         A^2 = \lambda^2U + \mu^2V \cr
         A^3 = \lambda^3U + \mu^3V.\cr\end{cases}$}
\begin{enumerate}
  \item \question{Montrer que : $\forall\ p \in \N^*,\  A^p = \lambda^pU + \mu^pV$
    (chercher une relation linéaire entre $A$, $A^2$, $A^3$).}
  \item \question{On suppose ici $\lambda \ne \mu$, $\lambda\ne 0$ et $\mu\ne 0$.
    Soit $X$ un vecteur propre de $A$. Montrer que $X$ est vecteur propre de
    $U$ et de $V$ avec les valeurs propres $0,0$ ou $1,0$, ou $0,1$.}
\end{enumerate}
\begin{enumerate}
  \item \reponse{$A^3 - (\lambda+\mu)A^2 + \lambda\mu A = 0$.}
  \item \reponse{$U=\frac{\mu A-A^2}{\lambda(\mu-\lambda)}$, $V=\frac{\lambda A-A^2}{\mu(\lambda-\mu)}$
    et la valeur propre est $0$, $\lambda$ ou $\mu$.}
\end{enumerate}
}