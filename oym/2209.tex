\uuid{2209}
\titre{Exercice 2209}
\theme{Théorème de Sylow}
\auteur{debes}
\date{2008/02/12}
\organisation{exo7}
\contenu{
  \texte{}
  \question{Soit $G$ un groupe simple d'ordre $60$. 
\smallskip

(a) Montrer que $G$ n'admet pas de sous-groupe d'ordre $20$. 
\smallskip

(b) Montrer que si $G$ admet un sous-groupe $K$ d'ordre $12$, alors $K$ admet $4$
$3$-Sylow. 
\smallskip

(c) Montrer que si $H$ et $K$ sont deux sous-groupes distinct d'ordre $4$ de $G$
alors $H\cap K=\{1\}$. 
\smallskip

(d) Montrer que si $H$ est un $2$-Sylow, alors $H\not= \hbox{\rm Nor}_G(H)$.
\smallskip

(e) Montrer que $G$ poss\`ede $5$ $2$-Sylow.
\smallskip

(f) Conclure en consid\'erant l'action de $G$ par conjugaison sur les $5$-Sylow.}
  \reponse{}
}