\uuid{4879}
\titre{Transitivité des homothéties-translations}
\theme{Exercices de Michel Quercia, Applications affines}
\auteur{quercia}
\date{2010/03/17}
\organisation{exo7}
\contenu{
  \texte{}
  \question{Dans un espace affine ${\cal E}$ on donne quatre points $P,Q,P',Q'$ avec
$P \ne Q$.
Existe-t-il une homothétie-translation $f$ telle que $f(P) = P'$ et
$f(Q) = Q'$ ?}
  \reponse{oui ssi $\vec{P'Q'}$ est colinéaire à $\vec{PQ}$. Dans ce cas, $f$ est unique.}
}