\uuid{6933}
\titre{Exercice 6933}
\theme{}
\auteur{ruette}
\date{2013/01/24}
\organisation{exo7}
\contenu{
  \texte{}
  \question{Chaque jour, un train subit un retard aléatoire au départ, évalué en minutes,
indépendant des retards des autres jours et dont la loi est approximativement 
$\mathcal{E}(\lambda)$ (\textit{on rappelle qu'une loi exponentielle $\mathcal{E}(\lambda)$
a pour espérance $1/\lambda$ et pour écart-type $1/\lambda$}).
Sur 400 jours, le retard moyen est de 10 minutes. 
Donner un intervalle de confiance de niveau approximativement 95\% pour 
$\lambda$.}
  \reponse{Loi estimée $\mathcal{E}(\lambda)$ d'espérance
$m=1/\lambda$ et d'écart-type $\sigma=1/\lambda$. 
Estimateur de $m$ : $\bar X_n=10$. Donc on approxime $\sigma$ par $10$.
TCL : $P(\frac{|S_n-n m|}{\sigma\sqrt{n}}<c)\simeq P(|\mathcal{N}(0,1)|<c)$.
Pour avoir un intervalle de confiance à 95\%, on prend $c=1,96$.
L'encadrement $-c<\frac{\bar X_n-1/\lambda}{\sigma/\sqrt{n}}<c$ donne
$$
\frac{1}{\bar X_n+c\sigma/\sqrt{n}}<\lambda<\frac{1}{\bar X_n-c\sigma/\sqrt{n}}
$$
d'où l'intervalle de confiance $[0,091,0,111]$.}
}