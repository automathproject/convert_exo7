\uuid{3605}
\titre{$\mathrm{Ker} f \oplus \Im f$}
\theme{Exercices de Michel Quercia, Réductions des endomorphismes}
\auteur{quercia}
\date{2010/03/10}
\organisation{exo7}
\contenu{
  \texte{}
  \question{Soit $E$ un $ K$-ev de dimension finie et $f \in \mathcal{L}(E)$.
On suppose qu'il existe $P\in K[X]$ tel que $P(f) = 0$ et $P'(0)\ne 0$.
Montrer que $\mathrm{Ker} f \oplus \Im f = E$.}
  \reponse{Si $P(0)\ne 0$ alors $f$ est bijective. Si $P(0) = 0$ alors
$f^2\circ\text{qqch} = -P'(0)f \Rightarrow  \mathrm{Ker} f^2 = \mathrm{Ker} f$.}
}