\uuid{6937}
\titre{Exercice 6937}
\theme{}
\auteur{ruette}
\date{2013/01/24}
\organisation{exo7}
\contenu{
  \texte{}
  \question{\label{exBateson} A la suite de la formulation des lois de Mendel, Bateson a
effectué des croisements avec des pois de senteur afin d'étudier
la couleur (pourpre ou rouge) et la forme du pollen (allongée ou
ronde). Ces croisements ont été fait sur des hybrides qui sont de
couleur pourpre et ont un pollen de forme allongé ; en
supposant que la couleur et la forme du pollen sont chacun controlés par un gène qui a deux
formes différentes (appelées allèles) notées $S$ et $s$ pour
la couleur et $T$, $t$  pour  la forme du pollen, le génotype d'un
pois hybride pour ces deux caractères est $(Ss,Tt)$. Les majuscules
désignent les allèles dominants (ici la couleur pourpre et la
forme allongée du pollen).  Les résultats
des croisements sont donnés dans le tableau ci-dessous~:
\begin{center}
\begin{tabular}{|c|cc|}
\hline
& couleur pourpre & couleur rouge \\
\hline
pollen allongé & 1528 & 117 \\
pollen rond & 106 &381\\
\hline
\end{tabular}
\end{center}
En faisant l'hypothèse que les quatre possibilités de
transmission des allèles pour chacun de ces deux gènes sont
équiprobables, peut-on affirmer que les  gènes contrôlant  ces deux caractères
sont transmis indépendamment l'un de l'autre~?\\
Comment s'assurer que l'hypothèse sur la transmission
équiprobable des quatre possibilités de
transmission des allèles est valide pour chacun de ces deux
gènes~?}
  \reponse{}
}