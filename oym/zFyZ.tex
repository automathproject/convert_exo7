\uuid{zFyZ}
\exo7id{4182}
\titre{Point non extrémal}
\theme{Exercices de Michel Quercia, Dérivées partielles}
\auteur{quercia}
\date{2010/03/11}
\organisation{exo7}
\contenu{
  \texte{On pose pour $(x,y)\in\R^2$~:
$$f(x,y) = x^2+y^2-2x^2y - \frac{4x^6y^2}{(x^4+y^2)^2}\quad\text{si }(x,y)\ne 0,
  \qquad f(0,0)=0.$$}
\begin{enumerate}
  \item \question{Montrer que $f$ est continue sur $\R^2$.}
  \item \question{Soit $\theta\in\R$ fixé et $g_\theta(r) = f(r\cos\theta,r\sin\theta)$.
Montrer que $g_\theta$ admet un minimum local strict en~$r=0$.}
  \item \question{Calculer $f(x,x^2)$. Conclusion~?}
\end{enumerate}
\begin{enumerate}
  \item \reponse{$2x^4y^2 \le (x^4+y^2)^2$.}
  \item \reponse{$g_\theta(r)\sim r^2$.}
  \item \reponse{$f(x,x^2)=-x^4$. Donc $(0,0)$ n'est pas minimum local de~$f$.}
\end{enumerate}
}