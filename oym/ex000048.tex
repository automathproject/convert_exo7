\exo7id{48}
\titre{Exercice 48}
\theme{}
\auteur{bodin}
\date{1998/09/01}
\organisation{exo7}
\contenu{
  \texte{}
\begin{enumerate}
  \item \question{R\'esoudre $z^3 = 1$ et montrer que les racines s'\'ecrivent $1$, $j$, $j^2$.
Calculer $1+j+j^2$ et en d\'eduire les racines de $1+z+z^2 =0$.}
  \item \question{R\'esoudre $z^n = 1$ et montrer que les racines s'\'ecrivent
 $1,\epsilon,\ldots,\epsilon^{n-1}$. En d\'eduire les racines de $1+z+z^2+\cdots+z^{n-1} =0$.
Calculer, pour $p \in \Nn$, $1+\epsilon^p+\epsilon^{2p}+\cdots+\epsilon^{(n-1)p}$.}
\end{enumerate}
\begin{enumerate}

\end{enumerate}
}