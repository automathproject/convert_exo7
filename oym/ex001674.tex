\uuid{1674}
\titre{Exercice 1674}
\theme{}
\auteur{gineste}
\date{2001/11/01}
\organisation{exo7}
\contenu{
  \texte{}
  \question{Soit $\theta\in]0,\pi[$. On considère les deux
matrices d'ordre $n$:

$$ A=\left[ \begin{array}{llllll}0&1&0&\cdots&0&0\\
1&0&1&\cdots&0&0\\
0&1&0&\cdots&0&0\\
\cdots&\cdots&\cdots&\cdots&\cdots&\cdots\\
0&0&0&\cdots&0&1\\
0&0&0&\cdots&1&0
 \end{array} \right],
B=\left[ \begin{array}{llllll}
2\cos \theta&1&0&\cdots&0&0\\
1&2\cos \theta&1&\cdots&0&0\\
0&1&2\cos \theta&\cdots&0&0\\
\cdots&\cdots&\cdots&\cdots&\cdots&\cdots\\
0&0&0&\cdots&2\cos \theta&1\\
0&0&0&\cdots&1&2\cos \theta
 \end{array} \right]$$
Montrer par récurrence que
$\det B=\frac{\sin(n+1)\theta}{\sin \theta}$ (Méthode: développer
par rapport à la dernière ligne). Montrer que $\det B$
s'annule pour $n$ valeurs distinctes de $\theta$ de $]0,\pi[$,
et les déterminer. Si $P_A$ est le polynôme caractéristique de $A$, calculer
$P_A(-2\cos \theta)$ et déduire de ce qui précède les valeurs propres de $A.$
Montrer que les valeurs propres des matrices $2I_n+A$ et $2I_n-A$ sont
strictement positives.}
  \reponse{Notons $D_n=\det B.$
Alors $D_1=2\cos \theta=\frac{\sin 2\theta}{\sin \theta}$ et
$D_2=4\cos^2\theta -1=\frac{\sin 3\theta}{\sin \theta}.$ Si $n>2$,
d\'eveloppons $D_n$ par rapport \`a la derni\`ere ligne, en
recommencant encore une fois avec un des d\'eterminants d'ordre
$n-1$ obtenus. On obtient $D_n=2\cos \theta D_{n-1}-D_{n-2}.$
Faisons l'hypoth\`ese de r\'ecurrence que $D_k=\frac{\sin
(k+1)\theta}{\sin \theta}$ pour $k<n.$ On a vu que c'est vrai pour
$k=1$ et 2. Alors par des identit\'es trigonom\'etriques
classiques $D_n=\frac{2\cos\theta \sin n\theta}{\sin \theta}
-\frac{\sin (n-1)\theta}{\sin \theta}= \frac{\sin
(n+1)\theta}{\sin \theta},$ et la r\'ecurrence est \'etendue.
Puisque $\sin x=0 \Leftrightarrow $ il existe un entier relatif
$k$ tel que $x=k\pi$ alors $D_n=0$ si et seulement si il existe
$k=1,2,\ldots,n$ tel que $\theta=\frac{k\pi}{n+1}$ les autres
valeurs de $k$ \'etant exclues car $0<\theta<\pi.$ Par
d\'efinition de $P_A$ on a $P_A(-2\cos \theta)=D_n=\frac{\sin
(n+1)\theta}{\sin \theta}$ qui s'annule pour les $n$ nombres
distincts $-2\cos \frac{k\pi}{n+1},$ $k=1,2,\ldots,n$ qui sont
n\'ecessairement toutes les valeurs propres de $A.$ Les valeurs
propres de $2I_n+ A$ sont donc $2-2\cos
\frac{k\pi}{n+1}=4\sin^2\frac{k\pi}{2n+2}>0.$ Le spectre de $2I_n-
A$ est le m\^eme.}
}