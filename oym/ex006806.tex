\uuid{6806}
\titre{Exercice 6806}
\theme{}
\auteur{gijs}
\date{2011/10/16}
\organisation{exo7}
\contenu{
  \texte{On note $\mathrm{Log} : \Cc\setminus\,]-\infty,0] \to \Cc$
la détermination principale du logarithme, qui est
réelle pour $z$ réel positif.
On pose 

\begin{align*}
f(z) &= \exp\bigl(\tfrac13\mathrm{Log}(z) + \tfrac13\mathrm{Log}(z-1) +
\tfrac13\mathrm{Log}(z+1)\bigr)
\\
g(z) &= \exp\bigl(\tfrac13\mathrm{Log}(-z) +
\tfrac13\mathrm{Log}(1-z^2)\bigr)
\ .
\end{align*}}
\begin{enumerate}
  \item \question{Déterminer les domaines de définition $\Omega_f
\subset \Cc$ de $f$ et $\Omega_g \subset \Cc$ de
$g$ et montrer que $f$ et $g$ sont des branches continues
de $\root3\of {z^3-z}$.}
  \item \question{Est-ce-qu'on peut élargir le domaine de définition de
$f$, c'est à dire~: existe-t-il un ouvert
$\widehat\Omega_f \subset \Cc$ et une fonction
continue $\widehat f : \widehat\Omega_f \to \Cc$
telle que $\Omega_f \subset \widehat\Omega_f$ et pour
tout $z\in \Omega_f$ on a $\widehat f(z) = f(z)$~?
N'oubliez pas de justifier votre réponse~!}
  \item \question{Est-ce-qu'on peut élargir le domaine de définition de
$g$~?}
\end{enumerate}
\begin{enumerate}

\end{enumerate}
}