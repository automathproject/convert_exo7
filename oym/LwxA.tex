\uuid{LwxA}
\exo7id{3595}
\titre{Centrale MP 2000}
\theme{Exercices de Michel Quercia, Réductions des endomorphismes}
\auteur{quercia}
\date{2010/03/10}
\organisation{exo7}
\contenu{
  \texte{Si $A\in M_n(\C)$, on note $C(A)$ le commutant de $A$.}
\begin{enumerate}
  \item \question{Pour $n=2$, montrer que $C(A)$ est de dimension $2$ ou $4$, en donner une base.}
  \item \question{Pour $n\in \N^*$, montrer que $C(A)$ est de dimension $\ge n$ (traiter d'abord le cas où $A$
    est diagonalisable).}
\end{enumerate}
\begin{enumerate}
  \item \reponse{Par similitude on se ramène aux cas~:
    $A=\left(\begin{smallmatrix}\lambda&0\cr0&\lambda\cr\end{smallmatrix}\right)$, $C(A) = \mathcal{M}_2(\C)$ ou
    $A=\left(\begin{smallmatrix}\lambda&0\cr0&\mu\cr\end{smallmatrix}\right)$, $C(A) = \C[A]$ ou
    $A=\left(\begin{smallmatrix}\lambda&1\cr0&\lambda\cr\end{smallmatrix}\right)$, $C(A) = \C[A]$.}
  \item \reponse{Si $A$ est diagonalisable de valeurs propres $\lambda_i$
    avec les multiplicités $n_i$ alors ${\dim(C(A)) = \sum n_i^2 \ge n}$.
    
    Dans le cas général, soit $(A_k)$ une suite de matrices diagonalisables
    convergeant vers $A$ et $(C_k^1,\dots,C_k^n)$ une suite de $n$-uplets
    de matrices commutant avec $A_k$ telles que $(C_k^1,\dots,C_k^n)$
    est une famille orthonormale pour un produit scalaire quelconque choisi
    sur $\mathcal{M}_n(\C)$. Par compacité il existe une sous-suite convergente, donc
    $n$ matrices $C_\infty^i$ formant une famille orthonormale et commutant
    avec~$A$ d'où $\dim(C(A))\ge n$.}
\end{enumerate}
}