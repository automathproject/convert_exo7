\uuid{1903}
\titre{Exercice 1903}
\theme{}
\auteur{maillot}
\date{2001/09/01}
\organisation{exo7}
\contenu{
  \texte{}
  \question{Soit $f:\Rr^d\rightarrow\Rr^d$ continue.
Soit $x_0\in\Rr^d$. Soit $x_n$ la suite d\'efinie par
$$x_{n+1}=f(x_n).$$
Supposons que $||x_n-x_{n+1}||\rightarrow 0$.
Montrer que si $a\in\ A$ alors $f(a)=a$.

Indication: appliquer la d\'efinition de la continuit\'e
de $f$ en $a$ en termes de limites.}
  \reponse{}
}