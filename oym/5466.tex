\uuid{5466}
\titre{Exercice 5466}
\theme{Calculs de primitives et d'intégrales}
\auteur{rouget}
\date{2010/07/10}
\organisation{exo7}
\contenu{
  \texte{}
  \question{Calculer les primitives des fonctions suivantes en précisant le ou les intervalles considérés~:

$$
\begin{array}{lllll}
1)\;\frac{1}{x^3+1}&2)\;\frac{x^2}{x^3+1}&3)\;\frac{x^5}{x^3-x^2-x+1}&4)\;\frac{1-x}{(x^2+x+1)^5}&5)\;\frac{1}{x(x^2+1)^2}\\
6)\;\frac{x^2+x}{x^6+1}&7)\;\frac{1}{x^4+1}&8)\;\frac{1}{(x^4+1)^2}&9)\;\frac{1}{x^8+x^4+1}&10)\;\frac{x}{(x^4+1)^3}\\
11)\;\frac{1}{(x+1)^7-x^7-1}
\end{array}
$$}
  \reponse{\begin{enumerate}
\item  $I$ est l'un des deux intervalles $]-\infty,-1[$ ou $]-1,+\infty[$. $f$ est continue sur $I$ et admet donc des primitives sur $I$.

$$\frac{1}{X^3+1}=\frac{1}{(X+1)(X+j)(X+j^2)}=\frac{a}{X+1}+\frac{b}{X+j}+\frac{\overline{b}}{X+j^2},$$

où $a=\frac{1}{3(-1)^2}=\frac{1}{3}$ et $b=\frac{1}{3(-j)^2}=\frac{j}{3}$. Par suite,

\begin{align*}\ensuremath
\frac{1}{X^3+1}&=\frac{1}{3}(\frac{1}{X+1}+\frac{j}{X+j}+\frac{j^2}{X+j^2})
=\frac{1}{3}(\frac{1}{X+1}+\frac{-X+2}{X^2-X+1})
=\frac{1}{3}(\frac{1}{X+1}-\frac{1}{2}\frac{2X-1}{X^2-X+1}+\frac{3}{2}\frac{1}{X^2-X+1})\\
 &=\frac{1}{3}(\frac{1}{X+1}-\frac{1}{2}\frac{2X-1}{X^2-X+1}
 +\frac{3}{2}\frac{1}{(X-\frac{1}{2})^2+(\frac{\sqrt{3}}{2})^2}).
\end{align*}

Mais alors,

\begin{align*}\ensuremath
\int_{}^{}\frac{1}{x^3+1}\;dx&=\frac{1}{3}(\ln|x+1|-\frac{1}{2}\ln(x^2-x+1)+\frac{3}{2}\frac{2}{\sqrt{3}}\Arctan
\frac{x-\frac{1}{2}}{\frac{\sqrt{3}}{2}})
=\frac{1}{6}\ln\frac{(x-1)^2}{x^2-x+1}+\frac{1}{\sqrt{3}}\Arctan\frac{2x-1}{\sqrt{3}}+C.
\end{align*}

\item  $I$ est l'un des deux intervalles $]-\infty,-1[$ ou $]1,+\infty[$. Sur $I$, $\int_{}^{}\frac{x^2}{x^3+1}\;dx=\frac{1}{3}\ln(x^3+1)+C$.

\item  $X^3-X^2-X+1=X^2(X-1)-(X-1)=(X^2-1)(X-1)=(X-1)^2(X+1)$. Donc, la décomposition en éléments simples de $f=\frac{X^5}{X^3-X^2-X+1}$ est de la forme $aX^2+bX+c+\frac{d_1}{X-1}+\frac{d_2}{(X-1)^2}+\frac{e}{X+1}$.

Détermination de $a$, $b$ et $c$. La division euclidienne de $X^5$ par $X^3-X^2-X+1$ s'écrit $X^5=(X^2+X+2)(X^3-X^2-X+1)+2X^2+X-2$. On a donc $a=1$, $b=1$ et $c=2$.

$e=\lim_{x\rightarrow -1}(x+1)f(x)=\frac{(-1)^5}{(-1-1)^2}=-\frac{1}{4}$. Puis, $d_2=\lim_{x\rightarrow 1}(x-1)^2f(x)=\frac{1^5}{1+1}=\frac{1}{2}$. Enfin, $x=0$ fournit $0=c-d_1+d_2+e$ et donc, $d_1=-2-\frac{1}{2}+\frac{1}{4}=-\frac{9}{4}$. Finalement,

$$\frac{X^5}{X^3-X^2-X+1}=X^2+X+2-\frac{9}{4}\frac{1}{X-1}+\frac{1}{2}\frac{1}{(X-1)^2}-\frac{1}{4}\frac{1}{X+1},$$

et donc, $I$ désignant l'un des trois intervalles $]-\infty,-1[$, $]-1,1[$ ou $]1,+\infty[$, on a sur $I$

$$\int_{}^{}\frac{x^5}{x^3-x^2-x+1}\;dx=\frac{x^3}{3}+\frac{x^2}{2}+2x-\frac{1}{2(x-1)}-\frac{1}{4}\ln|x+1|+C.$$
\item  Sur $\Rr$,

\begin{align*}\ensuremath
\int_{}^{}\frac{1-x}{(x^2+x+1)^5}\;dx&=-\frac{1}{2}\int_{}^{}\frac{2x+1}{(x^2+x+1)^5}\;dx+\frac{3}{2}\int_{}^{}\frac{1}{(x^2+x+1)^5}\;dx
=\frac{1}{8(x^2+x+1)^4}+\frac{3}{2}\int_{}^{}\frac{1}{((x+\frac{1}{2})^2+\frac{3}{4})^5}\;dx\\
 &=\frac{1}{8(x^2+x+1)^4}+\frac{3}{2}\int_{}^{}\frac{1}{((\frac{\sqrt{3}}{2}u)^2+\frac{3}{4})^5}\;\frac{\sqrt{3}}{2}\;du\;(\mbox{en posant}\;x+\frac{1}{2}=\frac{u\sqrt{3}}{2})\\
 &=\frac{1}{8(x^2+x+1)^4}+\frac{2^8\sqrt{3}}{3^4}\int_{}^{}\frac{1}{(u^2+1)^5}\;du.
\end{align*}

Pour $n\in\Nn^*$, posons alors $I_n=\int_{}^{}\frac{du}{(u^2+1)^n}$. Une intégration par parties fournit

$$I_n=\frac{u}{(u^2+1)^n}+2n\int_{}^{}\frac{u^2+1-1}{(u^2+1)^{n+1}}\;du=\frac{u}{(u^2+1)^n}+2n(I_n-I_{n+1}),$$

et donc, $I_{n+1}=\frac{1}{2n}\left(\frac{u}{(u^2+1)^n}+(2n-1)I_n\right)$. Mais alors,

\begin{align*}\ensuremath
I_5&=\frac{1}{8}\frac{u}{(u^2+1)^4}+\frac{7}{8}I_4=\frac{1}{8}\frac{u}{(u^2+1)^4}+\frac{7}{8.6}\frac{u}{(u^2+1)^3}+\frac{7.5}{8.6}I_3\\
 &=\frac{1}{8}\frac{u}{(u^2+1)^4}+\frac{7}{8.6}\frac{u}{(u^2+1)^3}+\frac{7.5}{8.6.4}\frac{u}{(u^2+1)^2}+\frac{7.5.3}{8.6.4}I_2\\
 &=\frac{1}{8}\frac{u}{(u^2+1)^4}+\frac{7}{8.6}\frac{u}{(u^2+1)^3}+\frac{7.5}{8.6.4}\frac{u}{(u^2+1)^2}+\frac{7.5.3}{8.6.4.2}\frac{u}{u^2+1}+\frac{7.5.3.1}{8.6.4.2}I_1\\
 &=\frac{1}{8}\frac{u}{(u^2+1)^4}+\frac{7}{8.6}\frac{u}{(u^2+1)^3}+\frac{7.5}{8.6.4}\frac{u}{(u^2+1)^2}+\frac{7.5.3}{8.6.4.2}\frac{u}{u^2+1}+\frac{7.5.3.1}{8.6.4.2}\Arctan u+C.
\end{align*}

Maintenant,

$$u^2+1=(\frac{2}{\sqrt{3}}(x+\frac{1}{2}))^2+1=\frac{4}{3}x^2+\frac{4}{3}x+\frac{1}{3}+1=\frac{4}{3}(x^2+x+1).$$

Par suite,

\begin{align*}\ensuremath
\frac{2^8\sqrt{3}}{3^4}\int_{}^{}\frac{1}{(u^2+1)^5}\;du&=\frac{2^8\sqrt{3}}{3^4}
\left(
\frac{1}{8}\frac{3^4}{4^4}\frac{\frac{2}{\sqrt{3}}(x+\frac{1}{2})}{(x^2+x+1)^4}
+\frac{7}{8.6}\frac{3^3}{4^3}\frac{\frac{2}{\sqrt{3}}(x+\frac{1}{2})}{(x^2+x+1)^3}
+\frac{7.5}{8.6.4}\frac{3^2}{4^2}\frac{\frac{2}{\sqrt{3}}(x+\frac{1}{2})}{(x^2+x+1)^2}
\right.
\\
 &\;\left.+\frac{7.5.3}{8.6.4.2}\frac{3}{4}\frac{\frac{2}{\sqrt{3}}(x+\frac{1}{2})}{x^2+x+1}
 +\frac{7.5.3.1}{8.6.4.2}\Arctan\frac{2x+1}{\sqrt{3}}+C
 \right).\\
 &=\frac{1}{8}\frac{2x+1}{(x^2+x+1)^4}+\frac{7}{36}\frac{2x+1}{(x^2+x+1)^3}+\frac{35}{108}\frac{2x+1}{(x^2+x+1)^2}+\frac{35}{54}\frac{2x+1}{x^2+x+1}\\
 &\;+\frac{70\sqrt{3}}{81}\Arctan\frac{2x+1}{\sqrt{3}}+C,
\end{align*}

(il reste encore à réduire au même dénominateur).

\item  On pose $u=x^2$ et donc $du=2xdx$

\begin{align*}\ensuremath
\int_{}^{}\frac{1}{x(x^2+1)^2}\;dx&=\int_{}^{}\frac{x}{x^2(x^2+1)^2}\;dx=\frac{1}{2}\int_{}^{}\frac{du}{u(u+1)^2}
=\frac{1}{2}\int_{}^{}(\frac{1}{u}-\frac{1}{u+1}-\frac{1}{(u+1)^2})\;du\\
 &=\frac{1}{2}(\ln|u|-\ln|u+1|+\frac{1}{u+1})+C\\
 &=\frac{1}{2}(\ln\frac{x^2}{x^2+1}+\frac{1}{x^2+1})+C.
\end{align*}
\item  $\int_{}^{}\frac{x^2+x}{x^6+1}\;dx=\int_{}^{}\frac{x^2}{x^6+1}\;dx+\int_{}^{}\frac{x}{x^6+1}\;dx$.

Ensuite, en posant $u=x^3$ et donc $du=3x^2\;dx$,

$$\int_{}^{}\frac{x^2}{x^6+1}\;dx=\frac{1}{3}\int_{}^{}\frac{1}{u^2+1}\;du=\frac{1}{3}\Arctan u+C=\frac{1}{3}\Arctan(x^3)+C,$$

et en posant $u=x^2$ et donc $du=2x\;dx$,

\begin{align*}\ensuremath
\int_{}^{}\frac{x}{x^6+1}\;dx&=\frac{1}{2}\int_{}^{}\frac{1}{u^3+1}\;du
=\frac{1}{6}\ln\frac{(u-1)^2}{u^2-u+1}+\frac{1}{\sqrt{3}}\Arctan\frac{2u-1}{\sqrt{3}}+C\;(\mbox{voir}\;1))\\
 &=\frac{1}{6}\ln\frac{(x^2-1)^2}{x^4-x^2+1}+\frac{1}{\sqrt{3}}\Arctan\frac{2x^2-1}{\sqrt{3}}+C
\end{align*}

Finalement,

$$\int_{}^{}\frac{x^2+x}{x^6+1}\;dx=\frac{1}{3}\Arctan(x^3)+\frac{1}{6}\ln\frac{(x^2-1)^2}{x^4-x^2+1}+\frac{1}{\sqrt{3}}\Arctan\frac{2x^2-1}{\sqrt{3}}+C.$$

\item  $\frac{1}{X^4+1}=\sum_{k=0}^{3}\frac{\lambda_k}{X-z_k}$ où $z_k=e^{i(\frac{\pi}{4}+k\frac{\pi}{2})}$. De plus,
$\lambda_k=\frac{1}{4z_k^3}=\frac{z_k}{4z_k^4}=-\frac{z_k}{4}$. Ainsi,

\begin{align*}\ensuremath
\frac{1}{X^4+1}&=-\frac{1}{4}\left(\frac{e^{i\pi/4}}{X-e^{i\pi/4}}+\frac{e^{-i\pi/4}}{X-e^{-i\pi/4}}
+\frac{-e^{i\pi/4}}{X+e^{i\pi/4}}+\frac{-e^{-i\pi/4}}{X+e^{-i\pi/4}}\right)\\
 &=-\frac{1}{4}\left(\frac{\sqrt{2}X-2}{X^2-\sqrt{2}X+1}-\frac{\sqrt{2}X+2}{X^2+\sqrt{2}X+1}\right).
\end{align*}

Mais,

$$\frac{\sqrt{2}X-2}{X^2-\sqrt{2}X+1}=\frac{1}{\sqrt{2}}\frac{2X-\sqrt{2}}{X^2-\sqrt{2}X+1}-\frac{1}{(X-\frac{1}{\sqrt{2}})^2+(\frac{1}{\sqrt{2}})^2},$$

et donc,

$$\int_{}^{}\frac{\sqrt{2}x-2}{x^2-\sqrt{2}x+1}\;dx=\frac{1}{\sqrt{2}}\ln(x^2-\sqrt{2}x+1)-\sqrt{2}\Arctan(\sqrt{2}x-1)+C,$$

et de même,

$$\int_{}^{}\frac{\sqrt{2}x+2}{x^2+\sqrt{2}x+1}\;dx=\frac{1}{\sqrt{2}}\ln(x^2+\sqrt{2}x+1)+\sqrt{2}\Arctan(\sqrt{2}x+1)+C.$$

Finalement,

$$\int_{}^{}\frac{1}{x^4+1}\;dx=\frac{1}{\sqrt{2}}\ln\frac{x^2-\sqrt{2}x+1}{x^2+\sqrt{2}x+1}-\sqrt{2}(\Arctan(\sqrt{2}x-1)+\Arctan(\sqrt{2}x+1))+C.$$
\item  Une intégration par parties fournit

\begin{align*}\ensuremath
\int_{}^{}\frac{1}{x^4+1}\;dx&=\frac{x}{x^4+1}+\int_{}^{}\frac{4x^4}{(x^4+1)^2}\;dx=\frac{x}{x^4+1}+4\int_{}^{}\frac{x^4+1-1}{(x^4+1)^2}\;dx\\
 &=\frac{x}{x^4+1}+4\int_{}^{}\frac{1}{x^4+1}\;dx-4\int_{}^{}\frac{1}{(x^4+1)^2}\;dx\\
\end{align*}

Et donc,

$$\int_{}^{}\frac{1}{(x^4+1)^2}\;dx=\frac{1}{4}(\frac{x}{x^4+1}+3\int_{}^{}\frac{1}{x^4+1}\;dx)=...$$
\item  Posons $R=\frac{1}{X^8+X^4+1}$.

\begin{align*}\ensuremath
X^8+X^4+1&=\frac{X^{12}-1}{X^4-1}=\frac{\prod_{k=0}^{11}(X-e^{2ik\pi/12})}{(X-1)(X+1)(X-i)(X+i)}\\
 &=(X-e^{i\pi/6})(X-e^{-i\pi/6})(X+e^{i\pi/6})(X+e^{-i\pi/6})(X-j)(X-j^2)(X+j)(X+j^2).
\end{align*}

$R$ est réelle et paire. Donc,

$$R=\frac{a}{X-j}+\frac{\overline{a}}{X-j^2}-\frac{a}{X+j}-\frac{\overline{a}}{X+j^2}+\frac{b}{X-e^{i\pi/6}}
+\frac{\overline{b}}{X-e^{-i\pi/6}}-\frac{b}{X+e^{i\pi/6}}-\frac{\overline{b}}{X+e^{-i\pi/6}}.$$

$a=\frac{1}{8j^7+4j^3}=\frac{1}{4(2j+1)}=\frac{2j^2+1}{4(2j+1)(2j^2+1)}=\frac{-1-2j}{12}$ et donc,

$$\frac{a}{X-j}+\frac{\overline{a}}{X-j^2}=\frac{1}{12}(\frac{-1-2j}{X-j}+\frac{-1-2j^2}{X-j^2})
=\frac{1}{4}\frac{1}{X^2+X+1}=\frac{1}{4}\frac{1}{(X+\frac{1}{2})^2+(\frac{\sqrt{3}}{2})^2},$$

 et par parité,
 
$$\frac{a}{X-j}+\frac{\overline{a}}{X-j^2}-\frac{a}{X+j}-\frac{\overline{a}}{X+j^2}=
\frac{1}{4}(\frac{1}{(X+\frac{1}{2})^2+(\frac{\sqrt{3}}{2})^2}+\frac{1}{(X-\frac{1}{2})^2+(\frac{\sqrt{3}}{2})^2}).$$

Ensuite, $b=\frac{1}{8e^{7i\pi/6}+4e^{3i\pi/6}}=\frac{1}{4e^{i\pi/6}(-2-j^2)}=\frac{e^{-i\pi/6}}{4(-1+j)}
=\frac{e^{-i\pi/6}(-1+j^2)}{12}=\frac{e^{-i\pi/6}(-2-j)}{12}=\frac{-2e^{-i\pi/6}-i}{12}$, et donc,

\begin{align*}\ensuremath
\frac{b}{X-e^{i\pi/6}}
+\frac{\overline{b}}{X-e^{-i\pi/6}}&=\frac{1}{12}(\frac{-2e^{-i\pi/6}-i}{X-e^{i\pi/6}}+\frac{-2e^{i\pi/6}+i}{X-e^{-i\pi/6}})=\frac{1}{12}\frac{-2\sqrt{3}X+3}{X^2-\sqrt{3}X+1}=-\frac{1}{4\sqrt{3}}\frac{2X-\sqrt{3}}{X^2-\sqrt{3}X+1}.
\end{align*}

Par parité,

$$\frac{b}{X-e^{i\pi/6}}
+\frac{\overline{b}}{X-e^{-i\pi/6}}-\frac{b}{X+e^{i\pi/6}}-\frac{\overline{b}}{X+e^{-i\pi/6}}=
-\frac{1}{4\sqrt{3}}\frac{2X-\sqrt{3}}{X^2-\sqrt{3}X+1}+\frac{1}{4\sqrt{3}}\frac{2X+\sqrt{3}}{X^2+\sqrt{3}X+1}.$$

Finalement,

$$\int_{}^{}\frac{1}{x^8+x^4+1}=\frac{1}{2\sqrt{3}}(\Arctan\frac{2x-1}{\sqrt{3}}+\Arctan\frac{2x+1}{\sqrt{3}})+\frac{1}{4\sqrt{3}}\ln\frac{x^2+\sqrt{3}x+1}{x^2-\sqrt{3}x+1}+C.$$

\item  En posant $u=x^2$ et donc $du=2x\;dx$, on obtient $\int_{}^{}\frac{x}{(x^4+1)^3}\;dx=\frac{1}{2}\int_{}^{}\frac{1}{(u^2+1)^3}$.

Pour $n\geq1$, posons $I_n=\int_{}^{}\frac{1}{(u^2+1)^n}\;du$. Une intégration par parties fournit~:

\begin{align*}\ensuremath
I_n=\frac{u}{(u^2+1)^n}+\int_{}^{}\frac{u.(-n)(2u)}{(u^2+1)^{n+1}}\;du=\frac{u}{(u^2+1)^n}+2n\int_{}^{}\frac{u^2+1-1}{(u^2+1)^{n+1}}\;du\\
 &=\frac{u}{(u^2+1)^n}+2n(I_n-I_{n+1}),
\end{align*}

et donc, $\forall n\geq1,\;I_{n+1}=\frac{1}{2n}(\frac{u}{(u^2+1)^n}+(2n-1)I_n)$.

On en déduit que

$$I_3=\frac{1}{4}(\frac{u}{(u^2+1)^2}+3I_2)=\frac{u}{4(u^2+1)^2}+\frac{3}{8(u^2+1)}+\frac{3}{8}\Arctan u+C,$$

et finalement que

$$\int_{}^{}\frac{x}{(x^4+1)^3}\;dx=\frac{1}{16}(\frac{2x^2}{(x^4+1)^2}+\frac{3}{x^4+1}+3\Arctan(x^2))+C.$$

\item  
\begin{align*}\ensuremath
(X+1)^7-X^7-1&=7X^6+21X^5+35X^4+35X^3+21X^2+7X=7X(X^5+3X^4+5X^3+5X^2+3X+1)\\
 &=7X(X+1)(X^4+2X^3+3X^2+2X+1)=7X(X+1)(X^2+X+1)^2.
\end{align*}

Par suite,

$$\frac{7}{(X+1)^7-X^7-1}=\frac{1}{X(X+1)(X-j)^2(X-j^2)^2}=\frac{a}{X}+\frac{b}{X+1}+\frac{c_1}{X-j}
+\frac{c_2}{(X-j)^2}+\frac{\overline{c_1}}{X-j^2}+\frac{\overline{c_2}}{(X-j^2)^2}.$$

$a=\lim_{x\rightarrow 0}xR(x)=1$, $b=\lim_{x\rightarrow -1}(x+1)R(x)=-1$, et

$c_2=\lim_{x\rightarrow j}(x-j)^2R(x)=\frac{1}{j(j+1)(j-j^2)^2}=-\frac{1}{j^2(1-2j+j^2)}=\frac{1}{3}$. Puis, 

$$\frac{c_2}{(X-j)^2}+\frac{\overline{c_2}}{(X-j^2)^2}=\frac{1}{3}(\frac{(X-j^2)^2+(X-j)^2}{(X^2+X+1)^2}
=\frac{2X^2+2X-1}{3(X^2+X+1)^2},$$

et

\begin{align*}\ensuremath
R-(\frac{c_2}{(X-j)^2}+\frac{\overline{c_2}}{(X-j^2)^2})
&=\frac{1}{X(X+1)(X^2+X+1)^2}-\frac{2X^2+2X-1}{3(X^2+X+1)^2}
=\frac{3-X(X+1)(2X^2+2X-1)}{3X(X+1)(X^2+X+1)^2}\\
 &=\frac{-2X(X+1)(X^2+X+1)+3+3X(X+1)}{3X(X+1)(X^2+X+1)^2}
=\frac{-2X^2-2X+3}{3X(X+1)(X^2+X+1)}.
\end{align*}

Puis, $c_2=\frac{-2j^2-2j+3}{3j(j+1)(j-j^2)}=-\frac{5}{j-j^2}=\frac{5(j-j^2)}{(j-j^2)(j^2-j)}=\frac{5(j-j^2)}{3}$.

Ainsi,

\begin{align*}\ensuremath
\frac{1}{(X+1)^7-X^7-1}&=\frac{1}{7}
(\frac{1}{X}-\frac{1}{X+1}+\frac{1}{3}(\frac{5(j-j^2)}{X-j}+\frac{5(j^2-j)}{X-j^2}+\frac{1}{(X-j)^2}+\frac{1}{(X-j^2)^2}))\\
 &=\frac{1}{7}
(\frac{1}{X}-\frac{1}{X+1}-\frac{5}{X^2+X+1}+\frac{1}{3}(\frac{1}{(X-j)^2}+\frac{1}{(X-j^2)^2}))\\
 &=\frac{1}{7}
(\frac{1}{X}-\frac{1}{X+1}-\frac{5}{(X+\frac{1}{2})^2+(\frac{\sqrt{3}}{2})^2}+\frac{1}{3}(\frac{1}{(X-j)^2}+\frac{1}{(X-j^2)^2})).
\end{align*}

Finalement,

\begin{align*}\ensuremath
\int_{}^{}\frac{1}{(x+1)^7-x^7-1}\;dx&=\frac{1}{7}\left(\ln\left|\frac{x}{x+1}\right|-\frac{10}{\sqrt{3}}
\Arctan\frac{2x+1}{\sqrt{3}}-\frac{1}{3}(\frac{1}{x-j}+\frac{1}{x-j^2})\right)+C\\
 &=\frac{1}{7}\left(\ln\left|\frac{x}{x+1}\right|-\frac{10}{\sqrt{3}}
\Arctan\frac{2x+1}{\sqrt{3}}-\frac{2x+1}{3(x^2+x+1)}\right)+C.
\end{align*}

\end{enumerate}}
}