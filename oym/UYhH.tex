\uuid{UYhH}
\exo7id{7829}
\titre{Exercice 7829}
\theme{Exercices de Christophe Mourougane, 328.00 - Formes bilinéaires}
\auteur{mourougane}
\date{2021/08/11}
\organisation{exo7}
\contenu{
  \texte{}
\begin{enumerate}
  \item \question{Montrer que l'application 
$$\begin{array}{cccc}h : &\Rr^3&\to& H_0\\
&(x_1,x_2, x_3)&\mapsto&\left(
\begin{array}{cc}
x_3&x_1+ix_2\\x_1-ix_2&-x_3
\end{array}\right)\end{array}$$
est un isomrphisme entre $\Rr^3$ et le $\Rr$-espace vectoriel
$H_0$ des matrices hermitiennes de trace nulle.}
  \item \question{Montrer que le groupe $SU(2)$ agit sur $H_0$ par conjuguaison.}
  \item \question{Par l'isomorphisme $h$, cette action permet de définir une action 
de $SU(2)$sur $\Rr^3$. Montrer que cette action est par isométrie de
déterminant $1$. (On pourra utiliser la connexité de $SU(2)$
homéomorphe à $S^3$ sphère unité du corps des quaternions.).
En déduire un homomorphisme $\phi$ de $SU(2)$ dans $SO(3)$.}
  \item \question{Montrer que les seules matrices de $SU(2)$ qui commutent à tous
 les éléments de $H_0$ sont $Id$ et $-Id$. En déduire le noyau de
 $\phi$.}
  \item \question{En utilisant les formes réduites des matrices de $SO(3)$,
 montrer que l'application exponentielle de l'espace $so(3)$
des matrices anti-symétriques réelles $3\times 3$ (de trace nulle) 
sur $SO(3)$ est surjective.}
  \item \question{En utilisant la diagonalisation des matrices unitaires dans une
 base orthonormée pour le produit scalaire hermitien standard sur
 $\Cc^2$, montrer que l'application exponentielle de l'espace $su(2)$
des matrices anti-hermitiennes de trace nulle sur $SU(2)$ est surjective.}
  \item \question{Déterminer l'image par $\phi$ des matrices ( avec $a,b,c\in\Rr$)
$$\exp \left(
\begin{array}{cc}
ia&0\\
0&-ia
\end{array}\right)
\ ;\
\exp \left(
\begin{array}{cc}
0&b\\
-b&0
\end{array}\right)
\ ;\
\exp \left(
\begin{array}{cc}
0&ic\\
ic&0
\end{array}\right)$$
En déduire que $\phi$ est surjective.}
\end{enumerate}
\begin{enumerate}

\end{enumerate}
}