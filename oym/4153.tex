\uuid{4153}
\titre{Formes différentielles exactes}
\theme{Exercices de Michel Quercia, Dérivées partielles}
\auteur{quercia}
\date{2010/03/11}
\organisation{exo7}
\contenu{
  \texte{}
  \question{Trouver les fonctions $f:\R \to \R$ de classe $\mathcal{C}^1$ telles que la forme
différentielle
$\omega = f(y)(xe^y\,dx + y\,dy)$ soit exacte. Déterminer alors ses
primitives.}
  \reponse{$f(y) = \lambda e^{-y}$,
         $F(x,y) = \lambda\left({\frac{x^2}2 - (y+1)e^{-y}}\right)$.}
}