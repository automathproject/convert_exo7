\uuid{5005}
\titre{$M - s/2M'$ reste sur $Ox$}
\theme{Exercices de Michel Quercia, Courbes définies par une condition}
\auteur{quercia}
\date{2010/03/17}
\organisation{exo7}
\contenu{
  \texte{}
  \question{Soit $\mathcal{C}$ une courbe plane et $s$ une abscisse curviligne sur $\mathcal{C}$.
A chaque point $M \in \mathcal{C}$ d'abscisse curviligne $s$, on associe le point
$N = M - \frac s2\vec T$.
Trouver $\mathcal{C}$ telle que $N$ reste sur $Ox$.}
  \reponse{$y = \frac s2\sin\varphi  \Rightarrow  s = a\sin\varphi \Rightarrow 
         x = \frac {a\sin2\varphi}4 + \frac {a\varphi}2 + b,\quad
         y = \frac {a\sin^2\varphi}2$. (cycloïde)}
}