\uuid{6Cxr}
\exo7id{3549}
\titre{Ensi PC 1999}
\theme{Exercices de Michel Quercia, Réductions des endomorphismes}
\auteur{quercia}
\date{2010/03/10}
\organisation{exo7}
\contenu{
  \texte{Soit $A\in\mathcal{M}_n(\C)$ inversible diagonalisable et $B\in \mathcal{M}_n(\C)$, $p\in\N^*$
tels que $B^p=A$.}
\begin{enumerate}
  \item \question{Montrer que $B$ est diagonalisable.}
  \item \question{Si $A$ n'est pas inversible la conclusion subsiste-t-elle~?}
\end{enumerate}
\begin{enumerate}
  \item \reponse{Polynôme annulateur simple.}
  \item \reponse{Non, ctrex = $B$ nilpotent.}
\end{enumerate}
}