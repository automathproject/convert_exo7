\uuid{KLjC}
\exo7id{4892}
\titre{Points aux tiers des côtés}
\theme{Exercices de Michel Quercia, Propriétés des triangles}
\auteur{quercia}
\date{2010/03/17}
\organisation{exo7}
\contenu{
  \texte{Soit $ABC$ un triangle, $A_1 = \text{Bar}(B:2,C:1)$, $B_1 = \text{Bar}(C:2,A:1)$ et
$C_1 = \text{Bar}(A:2,B:1)$.

On note $A_2,B_2,C_2$ les points d'intersection des droites $(AA_1)$, $(BB_1)$,
et $(CC_1)$.}
\begin{enumerate}
  \item \question{Montrer que $A_2$ est le milieu de $[B,B_2]$.}
  \item \question{Comparer les surfaces des triangles $ABC$ et $A_2B_2C_2$.}
\end{enumerate}
\begin{enumerate}

\end{enumerate}
}