\uuid{4922}
\titre{$\overline{(\vec{OA},\vec{OM})} \equiv 2\overline{(\vec{AM},\vec{AO})}$}
\theme{Exercices de Michel Quercia, Coniques}
\auteur{quercia}
\date{2010/03/17}
\organisation{exo7}
\contenu{
  \texte{}
  \question{Soient $O,A$ deux points distincts du plan.
Trouver les points $M$ tels que
$\overline{(\vec{OA},\vec{OM})} \equiv 2\overline{(\vec{AM},\vec{AO})}$.}
  \reponse{Soit $\alpha \equiv \overline{(\vec{AM},\vec{AO})}$.\par
         $\frac{OM}{\sin\alpha} = \frac{AM}{\sin(2\alpha)}
           \Rightarrow  2\cos\alpha = \frac{AM}{OM}$.\par
         Al-Khâshi $ \Rightarrow  \frac{AM^2}{OM}(OM-OA) = (OM-OA)(OM+OA)$.\par
         $OM = OA  \Rightarrow  \alpha \equiv \frac\pi4$.\par
         $OM \ne OA  \Rightarrow  OM = 2d(M,\Delta)$ où $\Delta$ est la médiatrice de
         $[O,A]$, et $M$ est du côté de $O$.}
}