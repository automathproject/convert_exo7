\uuid{3179}
\titre{{\'E}quation $P^2+Q^2 = (X^2+1)^2$}
\theme{Exercices de Michel Quercia, Polynômes}
\auteur{quercia}
\date{2010/03/08}
\organisation{exo7}
\contenu{
  \texte{}
  \question{Trouver $P,Q \in {\R[X]}$ premiers entre eux tels que $P^2 + Q^2 = (X^2+1)^2$.}
  \reponse{$\begin{cases}P = a(X^2+1) + bX + c \cr Q = a'(X^2+1) + b'X + c'\end{cases}
  \Rightarrow 
 \begin{cases}P = \cos\theta (X^2-1) + 2X\sin\theta \cr
        Q = \sin\theta (X^2-1) - 2X\cos\theta. \end{cases}$\par
$P \wedge Q = 1$ car $\pm i$ ne sont pas racines de $P$ et $Q$.}
}