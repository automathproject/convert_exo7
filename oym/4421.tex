\uuid{4421}
\titre{Ensi MP 2002}
\theme{Exercices de Michel Quercia, Séries numérique}
\auteur{quercia}
\date{2010/03/14}
\organisation{exo7}
\contenu{
  \texte{}
  \question{On suppose que la série à termes positifs de terme général $u_n$ est divergente et on pose 
$S_n=\sum_{k=0}^n u_k$.

Soit $f : {\R^+} \to {\R^+}$ une application continue décroissante.
Comparer les énoncés~:

1. $f$ est intégrable

2. La série de terme général $u_nf(S_n)$ converge.}
  \reponse{$1 \Rightarrow 2$ par comparaison série-intégrale. Contre-exemple pour $(2)\not\Rightarrow(1)$~:
$u_n = e^{(n+1)^2}-e^{n^2}$, $S_n = e^{(n+1)^2}-1$, $f(t)=\frac1{(t+2)\ln(t+2)}$.}
}