\uuid{yZNY}
\exo7id{5982}
\titre{Exercice 5982}
\theme{Propriétés de $\Rr$, Divers}
\auteur{bodin}
\date{2010/12/06}
\organisation{exo7}
\contenu{
  \texte{Soit $x$ un r\'eel.}
\begin{enumerate}
  \item \question{Donner l'encadrement qui définit la partie entière $E(x)$.}
  \item \question{Soit $(u_n)_{n\in \Nn^*}$ la suite définie par $u_n = \dfrac{E (x) + E (2x) + \ldots + E (nx)}{n^2}$. \\
  Donner un encadrement simple de $n^2 \times u_n$, qui utilise $\sum_{k=1}^n k$.}
  \item \question{En déduire que $(u_n)$ converge et calculer sa limite.}
  \item \question{En d\'eduire que $\Qq$ est dense dans $\Rr$.}
\end{enumerate}
\begin{enumerate}
  \item \reponse{Par définition est l'unique nombre $E(x) \in \Zz$ tel que 
 $$E(x) \le x < E(x)+1.$$}
  \item \reponse{Pour le réel $kx$, ($k=1,\ldots,n$) l'encadrement précédent s'écrit $E(kx) \le kx < E(kx)+1$.
 Ces deux inégalités s'écrivent aussi $E(kx) \le kx$ et $E(kx) > kx - 1$, d'où l'encadrement
 $kx-1 < E(kx) \le kx$. On somme cet encadrement, $k$ variant de $1$ à $n$, pour obtenir :
 $$\sum_{k=1}^n (kx-1) < \sum_{k=1}^n E(kx) \le \sum_{k=1}^n kx.$$
Ce qui donne 
$$ x \cdot \sum_{k=1}^n k \quad - n < n^2 \cdot u_n \le x \cdot  \sum_{k=1}^n k.$$}
  \item \reponse{On se rappelle que $\sum_{k=1}^n k = \frac{n(n+1)}{2}$ donc 
 nous obtenons l'encadrement :
 $$ x\cdot  \frac{1}{n^2} \cdot \frac{n(n+1)}{2} -   \frac{1}{n} < u_n \le x \cdot  \frac{1}{n^2} \cdot  \frac{n(n+1)}{2}.$$
 $\frac{1}{n^2} \cdot \frac{n(n+1)}{2}$ tend vers $\frac 12$, donc par le théorème des gendarmes $(u_n)$ tend vers $\frac x2$.}
  \item \reponse{Chaque $u_n$ est un rationnel (le numérateur et le dénominateur sont des entiers).
Comme la suite $(u_n)$ tend vers $\frac x 2$, alors la suite de rationnels $(2u_n)$ tend vers $x$.
Chaque réel $x\in \Rr$ peut être approché d'aussi près que l'on veut par des rationnels, donc 
$\Qq$ est dense dans $\Rr$.}
\end{enumerate}
}