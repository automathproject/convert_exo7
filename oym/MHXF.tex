\uuid{MHXF}
\exo7id{4525}
\titre{Théorème d'Ascoli}
\theme{Exercices de Michel Quercia, Suites et séries de fonctions}
\auteur{quercia}
\date{2010/03/14}
\organisation{exo7}
\contenu{
  \texte{Soit $(f_n)$ une suite de fonctions ${[a,b]} \to {\R}$
convergeant simplement vers $f$.
On suppose que toutes les fonctions $f_n$ sont $k$-Lipchitizennes
(avec le même $k$).}
\begin{enumerate}
  \item \question{Soit $(a_0,a_1,\dots,a_N)$ une subdivision régulière de $[a,b]$.
    On note $M_n = \max\{|f_n(a_i)-f(a_i)|\text{ tel que } 0\le i \le N\}$.
    Encadrer $\|f_n-f\|_{\infty}$ à l'aide de $M_n$.}
  \item \question{Montrer que $f_n$ converge uniformément vers $f$.}
\end{enumerate}
\begin{enumerate}

\end{enumerate}
}