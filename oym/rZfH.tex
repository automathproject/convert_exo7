\uuid{rZfH}
\exo7id{3343}
\titre{Endomorphisme nilpotent}
\theme{Exercices de Michel Quercia, Applications linéaires en dimension finie}
\auteur{quercia}
\date{2010/03/09}
\organisation{exo7}
\contenu{
  \texte{Un endomorphisme $f \in \mathcal{L}(E)$ est dit {\it nilpotent\/} s'il existe
$p \in \N$ tel que $f^p = 0$. Dans ce cas, {\it l'indice\/} de $f$ est le
plus petit entier $p$ tel que $f^p = 0$.
On considère $f\in\mathcal{L}(E)$ nilpotent d'indice $p$.}
\begin{enumerate}
  \item \question{Soit $\vec u \in E \setminus \mathrm{Ker} f^{p-1}$. Montrer que la famille
    $\bigl(\vec u, f(\vec u), \dots, f^{p-1}(\vec u) \bigr)$ est libre.}
  \item \question{En déduire que si $E$ est de dimension finie $n$, alors $f^n = 0$.}
  \item \question{Soit $g \in GL(E)$ tel que $f\circ g = g\circ f$.
    Montrer que $f + g \in GL(E) \dots$
  \begin{enumerate}}
  \item \question{en dimension finie.}
  \item \question{pour $E$ quelconque.}
\end{enumerate}
\begin{enumerate}

\end{enumerate}
}