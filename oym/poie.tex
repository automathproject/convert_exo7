\uuid{poie}
\exo7id{4385}
\titre{Ensi Chimie P 93}
\theme{Exercices de Michel Quercia, Intégrale multiple}
\auteur{quercia}
\date{2010/03/12}
\organisation{exo7}
\contenu{
  \texte{}
\begin{enumerate}
  \item \question{Calculer $\iiint_D \frac{d x\,d y\,d z}{(1+x^2z^2)(1+y^2z^2)}$ avec
    $D = \{(x,y,z) \text{ tel que } 0\le x \le 1,\ 0\le y\le 1,\ 0\le z\}$.}
  \item \question{En déduire $ \int_{t=0}^{+\infty} \Bigl(\frac{\Arctan t}t\Bigr)^2d t$.}
\end{enumerate}
\begin{enumerate}
  \item \reponse{Intégrer en $z$ d'abord :
             $\frac1{(1+x^2z^2)(1+y^2z^2)} =
              \frac 1{x^2-y^2}\Bigl(\frac{x^2}{1+x^2z^2} - \frac{y^2}{1+y^2z^2}\Bigr)$.
             On obtient $I = \pi\ln 2$.}
  \item \reponse{Intégrer $I$ en $x$ et $y$ d'abord. On obtient
    $I =  \int_{z=0}^{+\infty} \Bigl(\frac{\Arctan z}z\Bigr)^2 d z$.}
\end{enumerate}
}