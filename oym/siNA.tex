\uuid{siNA}
\exo7id{3525}
\titre{Recherche de vecteurs propres pour une valeur propre simple}
\theme{Exercices de Michel Quercia, Réductions des endomorphismes}
\auteur{quercia}
\date{2010/03/10}
\organisation{exo7}
\contenu{
  \texte{Soit $A \in \mathcal{M}_n(K)$ et $\lambda \in  K$ une valeur propre de $A$ telle que
$\mathrm{rg}(A-\lambda I) = n-1$.}
\begin{enumerate}
  \item \question{Quelle est la dimension du sous espace propre $E_\lambda$ ?}
  \item \question{Montrer que les colonnes de ${}^t\text{com}(A - \lambda I)$ engendrent
    $E_\lambda$.}
  \item \question{Exemple : diagonaliser $A = \begin{pmatrix} 0 & \phantom-1 &  2 \cr
                                       1 & 1          &  1 \cr
                                       1 & 0          & -1 \cr \end{pmatrix}$.}
\end{enumerate}
\begin{enumerate}

\end{enumerate}
}