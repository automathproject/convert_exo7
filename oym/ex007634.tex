\exo7id{7634}
\titre{Applications proportionnelles}
\theme{}
\auteur{mourougane}
\date{2021/08/10}
\organisation{exo7}
\contenu{
  \texte{Soit $D$ un ouvert connexe de $\Cc$ contenant le disque unité fermé.
Soient $f$ et $g$ deux applications holomorphes sur $D$ telles que pour tout $z\in\partial\Delta$, $|f(z)|=|g(z)|$.}
\begin{enumerate}
  \item \question{On suppose que $f$ et $g$ n'ont pas de zéro dans $D$. Montrer qu'il existe un nombre complexe $\lambda$ de module $1$ tel que $f=\lambda g$ sur $D$.}
  \item \question{La conclusion est-elle encore vraie si on ne suppose plus que $f$ et $g$ n'ont pas de zéro dans $D$ ?}
\end{enumerate}
\begin{enumerate}
  \item \reponse{Sous cette hypothèse, on peut considérer les applications holomorphes sur $D$, $f/g$ et $g/f$. Elles sont majorées en module par $1$ sur le disque unité, et donc aussi sur le disque fermé par le principe du maximum.
Elles sont donc toutes les deux de module $1$ sur le disque unité, et donc constantes sur le disque unité par le théorème de l'application ouverte. Par connexité de $D$, elles sont constantes sur $D$.
Il existe donc $\lambda\in\Cc$ tel que $f=\lambda g$ sur $D$.
Par égalité des modules, on conclut que $|\lambda|=1$.}
  \item \reponse{Non, les applications $z\mapsto z$ et $z\mapsto z^2$ sont de même module sur le disque unité, mais ne sont pas proportionnelles.}
\end{enumerate}
}