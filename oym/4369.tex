\uuid{4369}
\titre{X MP$^*$ 2000}
\theme{Exercices de Michel Quercia, Intégrale dépendant d'un paramètre}
\auteur{quercia}
\date{2010/03/12}
\organisation{exo7}
\contenu{
  \texte{}
  \question{Étudier la limite en $0+$ de $I(x) =  \int_{t=0}^{+\infty}
\frac{e^{-t}-\cos t}{t}e^{-xt}\, d t$.}
  \reponse{$I'(x) =  \int_{t=0}^{+\infty}(\cos t-e^{-t})e^{-xt}\, d t
=\frac{x}{1+x^2}-\frac1{x+1}$
donc $I(x) = \ln\Bigl(\frac{1+x^2}{(1+x)^2}\Bigr) + \text{cste}$
et $I(x)\to0$ (pour $x\to+\infty$) d'où $\text{cste}=0$.
Alors $I(x)\to0$ pour $x\to0^+$.}
}