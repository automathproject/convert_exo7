\uuid{220}
\titre{Exercice 220}
\theme{}
\auteur{bodin}
\date{1998/09/01}
\organisation{exo7}
\contenu{
  \texte{}
  \question{En utilisant la fonction $x\mapsto(1+x)^n$, calculer :
$$ \sum_{k=0}^{n}C_n^k \quad ; \quad \sum_{k=0}^{n}(-1)^k C_n^k \quad ; \quad
 \sum_{k=1}^{n}kC_n^k \quad ; \quad
 \sum_{k=0}^{n}\frac{1}{k+1}C_n^k.$$}
  \reponse{Soit $f : \Rr \longrightarrow \Rr$ la fonction $f(x) = (1+x)^n$.
Par la formule du bin\^ome de Newton nous savons que
$$f(x) = (1+x)^n = \sum_{k=0}^{n} C_n^k x^k.$$
\begin{enumerate}
  \item En calculant $f(1)$ nous avons $2^n = \sum_{k=0}^{n} C_n^k$.
  \item En calculant $f(-1)$ nous avons $0 = \sum_{k=0}^{n} (-1)^k C_n^k$.
  \item Maintenant calculons  $f'(x) = n(1+x)^{n-1} = \sum_{k=1}^{n} kC_n^k x^{k-1}$. \'Evaluons $f'(1) = n2^{n-1} = \sum_{k=1}^{n} kC_n^k$.
  \item Il s'agit ici de calculer la primitive $F$ de $f$ qui correspond \`a la somme :
$F(x) = \frac{1}{n+1}(1+x)^{n+1}-\frac{1}{n+1} = \sum_{k=0}^{n} \frac{1}{k+1}C_n^k x^{k+1}$. En $F(1) = \frac{1}{n+1}(2^{n+1}-1) = \sum_{k=0}^{n} \frac{1}{k+1}C_n^k $.
\end{enumerate}}
}