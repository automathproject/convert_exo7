\exo7id{5308}
\titre{***}
\theme{}
\auteur{rouget}
\date{2010/07/04}
\organisation{exo7}
\contenu{
  \texte{Soit $u_n=10...01_2$.($n$ chiffres égaux à $0$). Déterminer l'écriture binaire de~:}
\begin{enumerate}
  \item \question{$u_n^2$,}
  \item \question{$u_n^3$,}
  \item \question{$u_n^3-u_n^2+u_n$.}
\end{enumerate}
\begin{enumerate}
  \item \reponse{$u_n^2=(2^{n+1}+1)^2=2^{2n+2}+2^{n+2}+1=10...010...01_2$ ($n-1$ puis $n+1$ chiffres $0$)}
  \item \reponse{\begin{align*}\ensuremath
u_n^3&=(2^{n+1}+1)^3=2^{3n+3}+3.2^{2n+2}+3.2^{n+1}+1=2^{3n+3}+(2+1).2^{2n+2}+(2+1).2^{n+1}+1\\
 &=2^{3n+3}+2^{2n+3}+2^{2n+2}+2^{n+2}+2^{n+1}+1=10...0110...0110...01_2
\end{align*}

($n-1$ puis $n-1$ puis $n$ chiffres $0$)}
  \item \reponse{\begin{align*}\ensuremath
u_n^3-u_n^2+u_n&=2^{3n+3}+3.2^{2n+2}+3.2^{n+1}+1-2^{2n+2}-2^{n+2}-1+2^{n+1}+1=2^{3n+3}+2^{2n+3}+2^{n+2}+1\\
 &=10...010...010...01
\end{align*}

($n-1$ puis $n$ puis $n+1$ chiffres $0$)}
\end{enumerate}
}