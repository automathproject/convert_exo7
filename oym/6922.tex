\uuid{6922}
\titre{Exercice 6922}
\theme{}
\auteur{ruette}
\date{2013/01/24}
\organisation{exo7}
\contenu{
  \texte{}
  \question{Soient $X_1$, $X_2$, $X_3$, trois variables aléatoires de loi normale 
indépendantes telles que $E(X_1 ) = 100$, $\text{Var}(X_1 ) = 100$, $E(X_2 ) = 20$, 
$\text{Var}(X_2 ) = 4$, $E(X_3 ) = 50$, $\text{Var}(X_3 ) = 25$. On forme la combinaison linéaire $Y=X_1 +2X_2 -X_3$.

Déterminer $E(Y)$ et $\text{Var}(Y)$. Quelle est la loi de $Y$ ?}
  \reponse{}
}