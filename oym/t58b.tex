\uuid{t58b}
\exo7id{5152}
\titre{**I}
\theme{Valeurs absolues. Partie entière. Inégalités}
\auteur{rouget}
\date{2010/06/30}
\organisation{exo7}
\contenu{
  \texte{}
\begin{enumerate}
  \item \question{Montrer que~:~$\forall x\in\Rr,\;E(x+1)=E(x)+1$.}
  \item \question{Montrer que~:~$\forall(x,y)\in\Rr^2,\;E(x)+E(y)\leq E(x+y)$.}
  \item \question{Montrer que~:~$\forall(x,y)\in\Rr^2,\;E(x)+E(y)+E(x+y)\leq E(2x)+E(2y)$.}
\end{enumerate}
\begin{enumerate}
  \item \reponse{Soit $x\in\Rr$. Alors, $E(x)\leq x<E(x)+1$ puis $E(x)+1\leq x+1<(E(x)+1)+1$. Comme $E(x)+1\in\Zz$, on a
bien $E(x+1)=E(x)+1$.}
  \item \reponse{Soient $(x,y)\in\Rr^2$. On a $E(x)+E(y)\leq x+y$. Ainsi, $E(x)+E(y)$ est un entier relatif inférieur ou égal
à $x+y$. Comme $E(x+y)$ est le plus grand entier relatif inférieur ou égal à $x+y$, on a donc $E(x)+E(y)\leq E(x+y)$.

Améliorons. $E(x)\leq x<E(x)+1$ et $E(y)\leq y<E(y)+1$ fournit $E(x)+E(y)\leq x+y<E(x)+E(y)+2$ et donc $E(x+y)$ vaut,
suivant le cas, $E(x)+E(y)$ ou $E(x)+E(y)+1$ (et est dans tous les cas supérieur ou égal à $E(x)+E(y)$).}
  \item \reponse{Soit $(x,y)\in\Rr^2$. Posons $k=E(x)$ et $l=E(y)$.
\begin{itemize}}
  \item \reponse{[\textbf{1er cas.}] Si $x\in[k,k+\frac{1}{2}[$ et $y\in[l,l+\frac{1}{2}[$, alors $x+y\in[k+l,k+l+1[$ et donc $E(x+y)=k+l$,
puis $E(x)+E(y)+E(x+y)=k+l+k+l=2k+2l$. D'autre part, $2x\in[2k,2k+1[$ et $2y\in[2l,2l+1[$. Par suite,
$E(2x)+E(2y)=2k+2l$. Dans ce cas, $E(x)+E(y)+E(x+y)=E(2x)+E(2y)$.}
  \item \reponse{[\textbf{2ème cas.}] Si $x\in[k+\frac{1}{2},k+1[$ et $y\in[l,l+\frac{1}{2}[$, alors
$x+y\in[k+l+\frac{1}{2},k+l+\frac{3}{2}[$ et donc $E(x+y)=k+l$ ou $k+l+1$,puis $E(x)+E(y)+E(x+y)=2k+2l$ ou $2k+2l+1$.
D'autre part, $2x\in[2k+1,2k+2[$ et $2y\in[2l,2l+1[$. Par suite, $E(2x)+E(2y)=2k+2l+1$. Dans ce cas,
$E(x)+E(y)+E(x+y)\leq E(2x)+E(2y)$.}
  \item \reponse{[\textbf{3ème cas.}] Si $x\in[k,k+\frac{1}{2}[$ et $y\in[l+\frac{1}{2},l+1[$, on a de
même $E(x)+E(y)+E(x+y)\leq E(2x)+E(2y)$.}
  \item \reponse{[\textbf{4ème cas.}] Si $x\in[k+\frac{1}{2},k+1[$ et $y\in[l+\frac{1}{2},l+1[$, on a
$E(x)+E(y)+E(x+y)=2k+2l+2=E(2x)+E(2y)$.
\end{itemize}
Finalement, on a dans tous les cas $E(x)+E(y)+E(x+y)\leq E(2x)+E(2y)$.}
\end{enumerate}
}