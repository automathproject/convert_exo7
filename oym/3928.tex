\uuid{3928}
\titre{$\arctan((x-\sin a)/\cos a)$}
\theme{Exercices de Michel Quercia, Fonctions circulaires inverses}
\auteur{quercia}
\date{2010/03/11}
\organisation{exo7}
\contenu{
  \texte{}
  \question{Soit $a \in \left[0,\frac \pi2\right[$. On pose
$f(x) = \arcsin \left( \frac {2(x-\sin a)\cos a}{x^2-2x\sin a + 1} \right)$
et $g(x) = \arctan \left( \frac{x-\sin a}{\cos a} \right)$.

Vérifier que $f$ est bien définie,
calculer $\sin\bigl(2g(x)\bigr)$ et comparer $f(x)$ et $g(x)$.}
  \reponse{$\sin\bigl(2g(x)\bigr) = \sin(f(x))$.

$f(x)=-\pi-2g(x)$ pour $x\in ]-\infty,\sin a-\cos a[$ ; \\
$f(x)=2g(x)$ pour $x\in ]\sin a-\cos a,\sin a+\cos a[$ ; \\
$f(x)=\pi-2g(x)$ pour $x\in ]\sin a+\cos a,+\infty[$.}
}