\uuid{QdlH}
\exo7id{6993}
\titre{Exercice 6993}
\theme{\'Equations différentielles, Ordre 1}
\auteur{blanc-centi}
\date{2015/07/04}
\organisation{exo7}
\contenu{
  \texte{}
\begin{enumerate}
  \item \question{Résoudre l'équation différentielle $(x^2+1)y'+2xy=3x^2+1$ sur $\R$.
Tracer des courbes intégrales. Trouver la solution vérifiant $y(0) = 3$.}
  \item \question{Résoudre l'équation différentielle $y'\sin x-y\cos x+1=0$ sur $]0;\pi[$.
Tracer des courbes intégrales. Trouver la solution vérifiant $y(\frac\pi4) = 1$.}
\end{enumerate}
\begin{enumerate}
  \item \reponse{Comme le coefficient de $y'$ ne s'annule pas, on peut réécrire l'équation sous la forme
$$y'+\frac{2x}{x^2+1}y=\frac{3x^2+1}{x^2+1}$$
  \begin{enumerate}}
  \item \reponse{Les solutions de l'équation homogène associée sont les $y(x)=\lambda e^{A(x)}$, 
    où $A$ est une primitive de $a(x)=-\frac{2x}{x^2+1}$ et $\lambda\in\R$. 
    Puisque $a(x)$ est de la forme $-\frac{u'}{u}$ avec $u>0$, on peut choisir $A(x)=-\ln(u(x))$ 
    où $u(x)=x^2+1$. Les solutions sont donc les 
    $\displaystyle y(x)=\lambda e^{-\ln(x^2+1)}=\frac{\lambda}{x^2+1}$.}
  \item \reponse{Il suffit ensuite de trouver une solution particulière de l'équation 
    avec second membre: on remarque  que $y_0(x)=x$ convient.}
  \item \reponse{Les solutions sont obtenues en faisant la somme:  
    $$y(x)=x+\frac{\lambda}{x^2+1}\quad (x\in\R)$$
    où $\lambda$ est un paramètre réel.}
  \item \reponse{$y(0)=3$ si et seulement si $\lambda=3$. La solution cherchée est donc
    $y(x)=x+\frac{3}{x^2+1}$.}
\end{enumerate}
}