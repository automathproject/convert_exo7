\uuid{1204}
\titre{Exercice 1204}
\theme{}
\auteur{ridde}
\date{1999/11/01}
\organisation{exo7}
\contenu{
  \texte{}
  \question{D\'eterminer les suites born\'ees qui v\'erifient
$u_{n + 2} = 3u_{n + 1}-2u_n$.}
  \reponse{L'équation caractéristique est :
$$r^2-3r+2=0$$
dont les solutions sont $\lambda = 2$ et $\mu = 1$. Donc $u_n$ est
de la forme
$$u_n = \alpha 2^n + \beta 1^n = \alpha 2^n+\beta$$
Or la suite $(2^n)_n$ tend vers $+\infty$. Donc si $(u_n)_n$ est
bornée alors $\alpha = 0$. Donc $(u_n)_n$ est la suite constante
égale à $\beta$. Réciproquement toute suite constante qui vérifie
$u_n = \beta$ pour $n\in \Nn$ vérifie bien la relation de
récurrence $u_{n+2} = 3u_{n+1}-2u_n$. Donc les suites cherchées
sont les suites constantes.}
}