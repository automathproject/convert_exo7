\uuid{4864}
\titre{Droites concourantes}
\theme{Exercices de Michel Quercia, Sous-espaces affines}
\auteur{quercia}
\date{2010/03/17}
\organisation{exo7}
\contenu{
  \texte{}
  \question{Dans ${\cal E}_2$ muni d'un repère $(O,\vec i,\vec j)$, on considère les trois
droites : $\begin{cases}D   :&$ax   + by    = c   $\cr
                  D'  :&$a'x  + b'y   = c'  $\cr
                  D'' :&$a''x + b''y  = c''.$\cr\end{cases}$


Montrer que $D,D',D''$ sont parallèles ou concourantes si et seulement si
$\begin{vmatrix}a & b & c \cr a' & b' & c' \cr a'' & b'' & c'' \cr\end{vmatrix} = 0$.}
  \reponse{}
}