\uuid{s9pf}
\exo7id{7390}
\titre{Exercice 7390}
\theme{Exercices de Christophe Mourougane, Examens}
\auteur{mourougane}
\date{2021/08/10}
\organisation{exo7}
\contenu{
  \texte{On admet $281$ est un nombre premier.}
\begin{enumerate}
  \item \question{Peut-on écrire $281$ comme somme de deux carrés ? Si oui, faîtes le.}
  \item \question{Factoriser $280$. On choisit $x=3$. On admet que $3^{35}=60[281]$. Déterminer une racine $c$ de $-1$ modulo $281$.}
  \item \question{Calculer le $\pgcd(281, c+i)$ dans l'anneau $\Z[i]$ des entiers de Gauss.}
\end{enumerate}
\begin{enumerate}
  \item \reponse{$60^2=-53[281]$.
$53^2=-1[281]$.
Donc, $c=53$ est une racine de $-1$ modulo $281$.}
  \item \reponse{$281=5\times(53+i)+(16-5i)$.
$53-i=(3+i)(16-5i)$.
Donc, $\pgcd(281, c+i)=16-5i$.
En calculant $N(16-5i)$ on trouve $$16^2+5^2=281.$$}
\end{enumerate}
}