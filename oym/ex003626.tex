\uuid{3626}
\titre{Base de $( K^3)^*$}
\theme{}
\auteur{quercia}
\date{2010/03/10}
\organisation{exo7}
\contenu{
  \texte{Dans $ K^3$ on considère les formes linéaires :
$f_1(\vec x) = x+2y+3z$,
$f_2(\vec x) = 2x+3y+4z$,
$f_3(\vec x) = 3x+4y+6z$.}
\begin{enumerate}
  \item \question{Montrer que $(f_1,f_2,f_3)$ est une base de $( K^3)^*$.}
  \item \question{Trouver la base duale.}
\end{enumerate}
\begin{enumerate}
  \item \reponse{$( (-2,0,1),\ (0,3,-2),\ (1,-2,1) )$.}
\end{enumerate}
}