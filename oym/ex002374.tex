\exo7id{2374}
\titre{Exercice 2374}
\theme{}
\auteur{mayer}
\date{2003/10/01}
\organisation{exo7}
\contenu{
  \texte{Soit $X$ un espace m\'etrique.}
\begin{enumerate}
  \item \question{Soit $(F_n)_n$ une suite d\'ecroissante de ferm\'es de $X$
et soit $(x_n)_n$ une suite convergente telle que $x_n\in F_n$
pour tout $n\geq 0$. Montrer que
$$\lim _{n\rightarrow \infty} x_n
\in \bigcap _{n\geq 0} F_n\;\; .$$
 Donner un exemple pour lequel
$\bigcap _{n\geq 0} F_n =\emptyset$.}
  \item \question{Soit maintenant $(K_n)_n$ une suite d\'ecroissante de
{\it compacts}  non vides de $X$. V\'erifier que $K=\bigcap_{n\geq 0} K_n $
est non vide et que tout ouvert $\Omega$ qui contient $K$ contient
tous les $K_n$ \`a partir d'un certain rang.}
\end{enumerate}
\begin{enumerate}
  \item \reponse{Soit $x = \lim x_n$. Soit $N\in \Nn$ ; montrons que $x$ est dans $F_N$. On a $x_N \in F_N$, $x_{N+1} \in F_{N+1} \subset F_N$, $x_{N+2} \in F_{N+2} \subset F_{N+1} \subset F_N$, etc. Donc pour tout $n\ge N$ alors $x_n \in F_N$. Comme $F_N$ est fermé, alors la limite $x$ est aussi dans $F_N$.
Ceci étant vrai quelque soit $N$, alors $x\in \bigcap_N F_N$.

Pour construire un exemple comme demandé il est nécessaire que de toute suite on ne puisse pas extraire de sous-suite convergente. Prenons par exemple dans $\Rr$, $F_n = [n,+\infty[$, alors $\bigcap_n F_n = \varnothing$.}
  \item \reponse{\begin{enumerate}}
  \item \reponse{Pour chaque $n$ on prend $x_n \in K_n$, alors pour tout
$n$, $x_n \in K_0$ qui est compact donc on peut extraire une sous-suite convergente. Si $x$ est la limite de cette sous-suite alors $x \in K$.
Donc $K$ est non vide.}
  \item \reponse{Par l'absurde supposons que c'est faux, alors
$$\forall N\in\Nn \quad \exists n \ge N \quad \exists x_n \in K_n \text{ tel que } x_n\notin \Omega.$$
De la suite $(x_n)$, on peut extraire une sous-suite $x_{\phi(n)}$
qui converge vers $x\in K$. Or $x_n \in X\setminus \Omega$ qui est fermé
donc $x\in X\setminus \Omega$. Comme $K\subset \Omega$ alors $x\notin K$
ce qui est contradictoire.}
\end{enumerate}
}