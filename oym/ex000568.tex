\uuid{568}
\titre{Exercice 568}
\theme{}
\auteur{cousquer}
\date{2003/10/01}
\organisation{exo7}
\contenu{
  \texte{D\'eterminer les limites lorsque $n$ tend vers l'infini des suites
ci-dessous~; pour chacune, essayer de pr\'eciser en quelques mots la
m\'ethode employ\'ee.}
\begin{enumerate}
  \item \question{$\displaystyle 1\;;~-\frac{1}{2}\;;~\frac{1}{3}\;;~ 
\ldots\;;~\frac{(-1)^{n-1}}{n}\;;~\ldots$}
  \item \question{$2/1$~; $4/3$~; $6/5$~; $\ldots$~; $2n/(2n-1)$~; $\ldots$}
  \item \question{$0{,}23\;;~0{,}233\;;~\ldots\;;~0{,}233\cdots3\;;~\ldots$}
  \item \question{$\displaystyle\frac{1}{n^2}+\frac{2}{n^2}+\cdots+\frac{n-1}{n^2}$}
  \item \question{$\displaystyle\frac{(n+1)(n+2)(n+3)}{n^3}$}
  \item \question{$\displaystyle\biggl\lbrack \frac{1+3+5+\cdots+(2n-1)}{n+1} -
\frac{2n+1}{2}\biggr\rbrack$}
  \item \question{$\displaystyle\frac{n+(-1)^n}{n-(-1)^n}$}
  \item \question{$\displaystyle\frac{2^{n+1}+3^{n+1}}{2^n + 3^n}$}
  \item \question{$\displaystyle\bigl(1/2+1/4+1/8+\cdots+1/2^n\bigr)$\quad 
puis\quad $\displaystyle \sqrt{2}\;;~\sqrt{2\sqrt{2}}\;;~
\sqrt{2\sqrt{2\sqrt{2}}}\;;~\ldots$}
  \item \question{$\displaystyle\biggl(1-\frac{1}{3}+\frac{1}{9}-\frac{1}{27}+\cdots
+\frac{(-1)^n}{3^n} \biggr)$}
  \item \question{$\bigl( \sqrt{n+1}-\sqrt{n}\bigr)$}
  \item \question{$\displaystyle\frac{n\sin(n!)}{n^2+1}$}
  \item \question{D\'emontrer la formule $1+2^2+3^2+\cdots+n^2 = 
\frac{1}{6}
n(n+1)(2n+1)$ ; en d\'eduire $\lim_{n \to\infty}
\frac{1+2^2+3^2+\cdots+n^2}{n^3}$.}
\end{enumerate}
\begin{enumerate}
  \item \reponse{$0$.}
  \item \reponse{$1$.}
  \item \reponse{$7/30$.}
  \item \reponse{$1/2$.}
  \item \reponse{$1$.}
  \item \reponse{$-3/2$.}
  \item \reponse{$1$.}
  \item \reponse{$3$.}
  \item \reponse{$1$~; $2$.}
  \item \reponse{$3/4$.}
  \item \reponse{$0$.}
  \item \reponse{$0$.}
  \item \reponse{$1/3$.}
\end{enumerate}
}