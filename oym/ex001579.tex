\uuid{1579}
\titre{Exercice 1579}
\theme{}
\auteur{barraud}
\date{2003/09/01}
\organisation{exo7}
\contenu{
  \texte{}
  \question{\'Enoncer le théorème de Cayley-Hamilton.  Le démontrer dans le cas
 particulier où le polynôme caractéristique est scindé à racines simples.}
  \reponse{Soit $u$ un endomorphisme d'un espace vectoriel $E$ de dimension finie,
alors le polynôme caractéristiaque de $u$ est aussi un polynôme
annulateur de $u$.

Preuve si $\chi_{u}$ est scindé à racines simples~: $u$ est alors
diagonalisable et il existe donc une base $B$ dans laquelle $Mat_{B}(u)=
\begin{pmatrix}
  \lambda_{1}\\
  &\ddots\\
  &      &\lambda_{n}
\end{pmatrix}
$. Alors $Mat_{B}(\chi_{u}(u))=\begin{pmatrix}
  \chi_{u}(\lambda_{1})\\
  &\ddots\\
  &      &\chi_{u}(\lambda_{n})
\end{pmatrix}
$. Et comme $\forall i\in\{1,\dots,n\}$ on a $\chi_{u}(\lambda_{i})=0$,
on en déduit que $\chi_{u}(u)=0$.}
}