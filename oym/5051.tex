\uuid{5051}
\titre{Les normales coupent $Oz \Leftrightarrow$ révolution}
\theme{Exercices de Michel Quercia, Surfaces paramétrées}
\auteur{quercia}
\date{2010/03/17}
\organisation{exo7}
\contenu{
  \texte{}
  \question{\label{revolution}
Soit ${\cal S}$ une surface d'équation $z = f(x,y)$.
Montrer que ${\cal S}$ est de révolution si et seulement si en tout point $M$,
la normale à ${\cal S}$ en $M$ est parallèle ou sécante à $Oz$.}
  \reponse{La normale en $M$ est parallèle ou sécante à $Oz
         \Leftrightarrow y\frac{\partial f}{\partial x} - x\frac{\partial f}{\partial y} = 0
	 \Leftrightarrow \frac{\partial f}{\partial \theta} = 0 \Leftrightarrow f = f(\rho)$.}
}