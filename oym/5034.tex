\uuid{5034}
\titre{Calcul de courbure par TFI}
\theme{Exercices de Michel Quercia, Rectification, courbure}
\auteur{quercia}
\date{2010/03/17}
\organisation{exo7}
\contenu{
  \texte{}
  \question{Déterminer le rayon de courbure de la courbe $\mathcal{C}$ d'équation : $2x^2+y^2=1$
aux points intersection de $\mathcal{C}$ et des axes $Ox$ et $Oy$.}
  \reponse{$R=\frac12$ aux sommets principaux $(0,\pm1)$ et $R=\sqrt2$ aux sommets secondaires $(\pm1/\sqrt2,0)$.}
}