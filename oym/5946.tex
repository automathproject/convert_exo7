\uuid{5946}
\titre{Exercice 5946}
\theme{Théorème de convergence monotone, dominée et lemme de Fatou}
\auteur{tumpach}
\date{2010/11/11}
\organisation{exo7}
\contenu{
  \texte{}
  \question{Soit $f\in
\mathcal{L}^1(\mathbb{R})$. Que vaut la limite
$$ \lim_{n\to\infty} \int_{\mathbb{R}} f(x)\cos^n(\pi x) d\lambda (x) \
?$$}
  \reponse{Soit $f\in
\mathcal{L}^1(\mathbb{R}).$ Comme $\cos(\pi x)<1$ si $x\notin
\mathbb{Z},$ $\cos^n(\pi x)\rightarrow 0$ lorsque
$n\rightarrow\infty$ presque partout (pour tout $x\in
\mathbb{R}\setminus\mathbb{Z}$). Notons $f_n(x)=f(x) \cos^n(\pi
x).$ Alors, pour tout $n\in \mathbb{N}$ on a $|f_n(x)|\leq |f(x)|$
et comme $|f|\in \mathcal{L}^1(\mathbb{R}),$ par le
th\'{e}or\`{e}me de convergence domin\'{e}e de Lebesgue,
$$\lim\limits_{n\rightarrow\infty}\int_\mathbb{R} f(x)\cos^n(\pi x)d\lambda(x)=
\lim\limits_{n\rightarrow\infty}\int_\mathbb{R}
f_n(x)d\lambda(x)=\int_\mathbb{R}
\lim\limits_{n\rightarrow\infty}f_n(x)d\lambda(x)=0.$$}
}