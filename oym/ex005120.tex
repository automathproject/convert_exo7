\uuid{5120}
\titre{**T}
\theme{}
\auteur{rouget}
\date{2010/06/30}
\organisation{exo7}
\contenu{
  \texte{Résoudre dans $\Cc$ les équations suivantes~:}
\begin{enumerate}
  \item \question{$z^2+z+1=0$}
  \item \question{$2z^2+2z+1=0$}
  \item \question{$z^2-2z\cos\theta+1=0$, $\theta$ réel donné.}
  \item \question{$z^2-(6+i)z+(11+13i)=0$}
  \item \question{$2z^2-(7+3i)z+(2+4i)=0$.}
\end{enumerate}
\begin{enumerate}
  \item \reponse{$z^2+z+1=0\Leftrightarrow z=-\frac{1}{2}+i\frac{\sqrt{3}}{2}=j\;\mbox{ou}\;z=-\frac{1}{2}-i\frac{\sqrt{3}}{2}=j^2$.}
  \item \reponse{$\Delta'=1^2-2=-1=i^2$. L'équation a donc deux solutions non réelles et conjuguées, à savoir
$z_1=\frac{1}{2}(-1+i)$ et $z_2=\frac{1}{2}(-1-i)$.\rule[-5mm]{0mm}{0mm}}
  \item \reponse{Soit $\theta\in\Rr$. Pour tout complexe $z$, on a
\begin{align*}
z^2-2z\cos\theta+1&=(z-\cos\theta)^2+1-\cos^2\theta=(z-\cos\theta)^2+\sin^2\theta=(z-\cos\theta)^2-(i\sin\theta)^2\\
 &=(z-\cos\theta-i\sin\theta)(z-\cos\theta+i\sin\theta)=(z-e^{i\theta})(z-e^{-i\theta})
\end{align*}
L'équation proposée a donc deux solutions (pas nécessairement distinctes) $z_1=e^{i\theta}$ et $z_2=e^{-i\theta}$. De
plus,
$\Delta'=\cos^2\theta-1=-\sin^2\theta$ et ces solutions sont distinctes si et seulement si
$\theta\notin\pi\Zz$.}
  \item \reponse{Soit $(E)$ l'équation $z^2-(6+i)z+(11+3i)=0$.
Son discriminant est
$\Delta=(6+i)^2-4(11+13i)=-9-40i$. Comme $40=2\times20=2\times(4\times5)$ et que $4^2-5^2=16-25=-9$, on est en droit de deviner que
$\Delta=(4-5i)^2$. L'équation $(E)$ a deux solutions distinctes dans $\Cc$ à savoir $z_1=\frac{6+i+4-5i}{2}=5-2i$ et
$z_2=\frac{6+i-4+5i}{2}=1+3i$.\rule[-5mm]{0mm}{0mm}}
  \item \reponse{Soit (E) l'équation $2z^2-(7+3i)z+(2+4i)=0$.
Son discriminant est $\Delta=(7+3i)^2-8(2+4i)=24+10i$. Comme $10
=2\times5=2\times(5\times1)$ et que $5^2-1^2=24$, on est en droit de deviner que $\Delta=(5+i)^2$. L'équation proposée a deux solutions
distinctes dans $\Cc$ à savoir $z_1=\frac{7+3i+5+i}{4}=3+i$ et $z_2=\frac{7+3i-5-i}{4}=\frac{1}{2}(1+i)$.}
\end{enumerate}
}