\uuid{393}
\titre{Exercice 393}
\theme{}
\auteur{cousquer}
\date{2003/10/01}
\organisation{exo7}
\contenu{
  \texte{}
  \question{Déterminer les polynômes $P\in \mathbb{R}[X]$ et $Q\in \mathbb{R}[X]$, 
premiers entre eux,
tels que $P^2+Q^2=(X^2+1)^2$. En déduire que l'équation $x^2+y^2=z^2$ a une
infinité de solutions (non proportionnelles) dans $\mathbb{Z}$.}
  \reponse{}
}