\uuid{xvuM}
\exo7id{4688}
\titre{Approximation d'un irrationnel}
\theme{Exercices de Michel Quercia, Suites convergentes}
\auteur{quercia}
\date{2010/03/16}
\organisation{exo7}
\contenu{
  \texte{Soit $x \in \R^*$ et $(r_n)$ une suite de rationnels convergeant vers $x$.
On {\'e}crit $r_n = \frac {p_n}{q_n}$ avec $p_n\in\Z$, $q_n\in\N^*$.}
\begin{enumerate}
  \item \question{Montrer que si l'une des suites $(p_n)$, $(q_n)$ est born{\'e}e, alors l'autre l'est
    aussi, et $x \in \Q$.}
  \item \question{En d{\'e}duire que si $x \in \R\setminus \Q$, alors $|p_n| \xrightarrow[n\to\infty]{} +\infty$
     et $q_n \xrightarrow[n\to\infty]{} +\infty$.}
\end{enumerate}
\begin{enumerate}

\end{enumerate}
}