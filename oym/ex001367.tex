\uuid{1367}
\titre{Extrait de l'examen de janvier 1994}
\theme{}
\auteur{ortiz}
\date{1999/04/01}
\organisation{exo7}
\contenu{
  \texte{On d\'efinit
$A=\left\{a+jb:a,b\in \Zz\right\}$  o\`u $j=\exp(\frac
{2i\pi}3)$.}
\begin{enumerate}
  \item \question{Montrer que $A$ est un sous-anneau de $\Cc.$
On d\'esigne par $\mathcal{U}(A)$ le groupe des
\'el\'ements inversibles de $A$ et enfin, on pose,
pour tout $z\in\Cc,$ $N(z)={|z|}^2.$}
  \item \question{\begin{enumerate}}
  \item \question{Montrer que si $z\in A$ alors $N(z)\in\Zz.$}
  \item \question{Soit $z\in A.$ Montrer que  $z\in \mathcal{U}(A)$ si et seulement si $N(z)=1.$}
  \item \question{Soient $a$ et $b$ des entiers. Montrer que si $N(a+jb)=1$ alors  $a,b\in\left\{-1,0,1\right\}.$}
\end{enumerate}
\begin{enumerate}

\end{enumerate}
}