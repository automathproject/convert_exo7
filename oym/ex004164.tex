\uuid{4164}
\titre{Application du théorème des fonctions implicites}
\theme{}
\auteur{quercia}
\date{2010/03/11}
\organisation{exo7}
\contenu{
  \texte{}
  \question{On considère la courbe d'équation $e^{x-y} = 1+2x+y$. Donner la tangente à cette
courbe et la position par rapport à la tangente au point $(0,0)$.}
  \reponse{$y = -\frac x2 + \frac {9x^2}{16} + o (x^2)  \Rightarrow {}$
         tangente de pente $-\frac 12$, au dessus.}
}