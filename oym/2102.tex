\uuid{2102}
\titre{Exercice 2102}
\theme{Groupes, sous-groupes, ordre}
\auteur{debes}
\date{2008/02/12}
\organisation{exo7}
\contenu{
  \texte{}
  \question{(I) Soit $X$ un ensemble et ${\cal P} (X)$ l'ensemble des parties de $X$ ordonn\'e par l'inclusion. Soit $\varphi$ une application croissante de
${\cal P}(X)$ dans lui-m\^eme.
\smallskip 

(a) Montrer que l'ensemble $E$ des parties $A$ de $X$ qui v\'erifient $\varphi (A)\subset A$
est non vide et admet un plus petit \'el\'ement $A_0$.

\smallskip
(b) Montrer que $\varphi (A_0)=A_0$.

\medskip
(II) Soit deux ensembles $X$ et $Y$ munis de deux injections $g$ de $X$ dans $Y$ et $h$ de $Y$
dans $X$. 
\smallskip

(a) Montrer que l'application de ${\cal P}(X)$ dans lui-m\^eme d\'efini par 
$$\varphi (A)=X-h(Y-g(A))$$
est croissante.

\smallskip
(b) D\'eduire de ce qui pr\'ec\`ede qu'il existe une bijection de $X$ sur $Y$.}
  \reponse{(I) (a) $E\not= \emptyset$ car $X\in E$. L'ensemble $A_0=\bigcap_{A\in E} A$ est
de mani\`ere \'evidente le plus petit \'el\'ement de $E$.\smallskip

(b) On a $\varphi(A_0) \subset A_0$ puisque $A_0\in E$.  On d\'eduit, par la croissance
de $\varphi$, que $\varphi(\varphi(A_0)) \subset \varphi(A_0)$, ce qui donne
$\varphi(A_0)\in E$ et donc $A_0\subset \varphi(A_0)$.\medskip

(II) (a) La croissance de $\varphi$ est imm\'ediate. \smallskip

(b) Consid\'erons la partie $A_0$ associ\'ee \`a $\varphi$. D'apr\`es le (b) du (I), on a
$X\setminus h(X\setminus g(A_0))=A_0$. Autrement dit, les parties $A_0$ et $h(X\setminus
g(A_0))$ constituent une partition de $X$. Consid\'erons l'application $f:X\rightarrow X$
d\'efinie comme \'etant $g$ sur $A_0$ et $h^{-1}$ sur $h(Y\setminus g(A_0))$. On voit sans
difficult\'e que $f$ est une bijection (noter que les images respectives des deux restrictions
pr\'ec\'edentes sont $g(A_0)$ et $Y\setminus g(A_0)$ et qu'elles constituent une partition de
$Y$).\smallskip}
}