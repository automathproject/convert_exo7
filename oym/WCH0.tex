\uuid{WCH0}
\exo7id{5442}
\titre{***}
\theme{Calculs de limites, développements limités, développements asymptotiques}
\auteur{rouget}
\date{2010/07/10}
\organisation{exo7}
\contenu{
  \texte{}
\begin{enumerate}
  \item \question{Développement limité à l'ordre $n$ en $0$ de $f(x)=\frac{1}{(1-x)^2(1+x)}$.}
  \item \question{Soit $a_k$ le $k$-ème coefficient. Montrer que $a_k$ est le nombre de solutions dans $\Nn^2$ de l'équation $p+2q=k$.}
\end{enumerate}
\begin{enumerate}
  \item \reponse{Quand $x$ tend vers $0$,
  
\begin{align*}\ensuremath
\frac{1}{(1-x)^2(1+x)}&=\frac{1}{4}\frac{1}{1-x}+\frac{1}{2}\frac{1}{(1-x)^2}+\frac{1}{4}\frac{1}{1+x}
\underset{x\rightarrow0}{=}\frac{1}{4}\left(\sum_{k=0}^{n}x^k+2\sum_{k=0}^{n}(k+1)x^k+\sum_{k=0}^{n}(-1)^kx^k\right)+o(x^n)\\
 &=\sum_{k=0}^{n}\frac{2k+3+(-1)^k}{4}x^k+o(x^n).
\end{align*}}
  \item \reponse{On a aussi,

\begin{align*}\ensuremath
\frac{1}{(1-x)^2(1+x)}&=\frac{1}{(1-x)(1-x^2)}\underset{x\rightarrow0}{=}\left(\sum_{k=0}^{n}x^p\right)\left(\sum_{k=0}^{n}x^{2q}\right)+o(x^n)\\
 &=\sum_{k=0}^{n}\left(\sum_{p+2q=k}^{}1\right)x^k+o(x^n)=\sum_{k=0}^{n}a_kx^k+o(x^n).
\end{align*}
Par unicité des coefficients d'un développement limité, on a donc

\begin{center}
\shadowbox{
$\forall k\in\Nn$, $a_k=\frac{2k+3+(-1)^k}{4}$.
}
\end{center}
($a_k$ est le nombre de façons de payer $k$ euros en pièces de $1$ et $2$ euros).}
\end{enumerate}
}