\uuid{4325}
\titre{$f(t^n)$}
\theme{}
\auteur{quercia}
\date{2010/03/12}
\organisation{exo7}
\contenu{
  \texte{}
\begin{enumerate}
  \item \question{Soit $f : {[0,1]} \to \R$ continue.
    Montrer que $ \int_{t=0}^1 f(t^n)\,d t \to f(0)$  lorsque $n\to\infty$.}
  \item \question{Chercher un équivalent pour $n\to\infty$ de $ \int_{t=0}^1 \frac{t^n\,d t}{1+t^n}$.}
  \item \question{Chercher un équivalent pour $n\to\infty$ de $-1 +  \int_{t=0}^1 \sqrt{1+t^n}\,d t$.}
\end{enumerate}
\begin{enumerate}
  \item \reponse{Couper en $ \int_{t=0}^{1-\varepsilon} +  \int_{t=1-\varepsilon}^1$}
  \item \reponse{$=\left[\frac{t\ln(1+t^n)}n\right]_{t=0}^1 - \frac1n \int_{t=0}^1 \ln(1+t^n)\,d t \sim \frac{\ln 2}n$.}
  \item \reponse{$\frac1n \int_{t=0}^1  \frac{d t}{\sqrt{1+t}+1} = \frac{2\sqrt2-2+2\ln(2\sqrt2-2)}n$.}
\end{enumerate}
}