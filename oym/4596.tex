\uuid{4596}
\titre{Produit de Cauchy}
\theme{Exercices de Michel Quercia, Séries entières}
\auteur{quercia}
\date{2010/03/14}
\organisation{exo7}
\contenu{
  \texte{}
  \question{Soit $(c_n)$ le produit de Cauchy de la suite $(a_n)$ par la suite
$(b_n)$. On suppose que la série $A(z) = \sum_{n=0}^\infty a_nz^n$
a un rayon $R > 0$ et que $b_n/b_{n+1} \to \lambda$ lorsque $n\to\infty$
avec $|\lambda| < R$.
Montrer que $c_n/b_n\to A(\lambda)$ lorsque $n\to\infty$.}
  \reponse{$\frac{c_n}{b_n} = a_0 + a_1\frac{b_{n-1}}{b_n} + \dots + a_n\frac{b_0}{b_n}
                  = \sum_{k=0}^\infty a_ku_{n,k}$
et le théorème de convergence dominée s'applique.}
}