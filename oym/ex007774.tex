\uuid{7774}
\titre{Théorème de Sylow pour les groupes abéliens finis}
\theme{}
\auteur{mourougane}
\date{2021/08/11}
\organisation{exo7}
\contenu{
  \texte{Soit $G=\{a_1,a_2,\cdots a_n\}$ un groupe fini. On note $\alpha_i$ l'ordre de l'élément $a_i$. Soit $p$ un diviseur premier de $\mid G\mid$ l'ordre de $G$.}
\begin{enumerate}
  \item \question{Montrer que l'application
$$\begin{array}{cccc}
 f~:&\Zz/\alpha_1\Zz\times\Zz/\alpha_2\Zz\times\cdots\times\Zz/\alpha_n\Zz
&\to &G\\
& (\overline{h_1},\overline{h_2},\cdots,\overline{h_n})&\mapsto & a_1^{h_1}a_2^{h_2}\cdots a_n^{h_n}
\end{array}$$
est une application bien définie. Démontrer que c'est un homomorphisme de groupes puis qu'il est surjectif.}
  \item \question{En déduire que $p$ divise $\alpha_1\alpha_2\cdots\alpha_n$.}
  \item \question{Montrer qu'il y a dans $G$ un élément d'ordre $p$.}
  \item \question{En raisonnant par récurrence sur l'ordre du groupe et en considérant l'ensemble $G/<x>$ où $x$ est un élément d'ordre $p$ dans $G$, montrer que $G$ admet un $p$-Sylow.}
  \item \question{Montrer qu'un groupe abélien est simple si et seulement s'il est cyclique, d'ordre un nombre premier.}
\end{enumerate}
\begin{enumerate}

\end{enumerate}
}