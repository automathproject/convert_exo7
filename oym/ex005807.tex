\uuid{5807}
\titre{**}
\theme{}
\auteur{rouget}
\date{2010/10/16}
\organisation{exo7}
\contenu{
  \texte{Soit $E =\mathcal{L}(\Rr^2)$. Soit $(\lambda,\mu)\in\Rr^2$. On pose

\begin{center}
$\forall f\in\mathcal{L}(\Rr^2)$, $Q(f) =\lambda\text{Tr}(f^2) +\mu\text{det}(f)$.
\end{center}}
\begin{enumerate}
  \item \question{Vérifier que $Q$ est une forme quadratique sur $E$.}
  \item \question{Déterminer en fonction de $\lambda$ et $\mu$ le rang et la signature de $Q$.
Analyser en particulier les cas $(\lambda,\mu) = (1,0)$ et $(\lambda,\mu) = (0,1)$.}
\end{enumerate}
\begin{enumerate}
  \item \reponse{Si la matrice de $f$ dans la base canonique de $\Rr^2$ est $A=\left(
\begin{array}{cc}
a&c\\
b&d
\end{array}
\right)$, 

\begin{center}
$Q(f) =\lambda(a^2+2bc+d^2)+\mu(ad-bc)$.
\end{center}

$Q$ est un polynôme homogène de degré $2$ en les coordonnées de $f$ dans la base canonique de $\mathcal{L}(\Rr^2)$ et donc $Q$ est une forme quadratique sur $\mathcal{L}(\Rr^2)$.}
  \item \reponse{\textbullet~Si $\lambda=\mu= 0$, $r = 0$ et $s = (0,0)$. Si $\lambda=0$ et $\mu\neq0$, 

\begin{center}
$Q(f)=\frac{\mu}{4}(a+d)^2-\frac{\mu}{4}(a-d)^2-\frac{mu}{4}(b+c)^2+\frac{\mu}{4}(b-c)^2$,
\end{center}

et donc $r=4$ et $s=(2,2)$.

\textbullet~Si $\lambda\neq0$,

\begin{align*}\ensuremath
Q(f)&=\lambda a^2+\mu ad + (2\lambda-\mu)bc +\lambda d^2=\lambda\left(a+\frac{\mu}{2\lambda}d\right)^2 +(2\lambda-\mu)bc +\left(\lambda-\frac{\mu^2}{4\lambda}\right)d^2\\
 &=\lambda\left(a+\frac{\mu}{2\lambda}d\right)^2+\left(\lambda-\frac{\mu^2}{4\lambda}\right)d^2+  \frac{2\lambda-\mu}{4}(b+c)^2-\frac{2\lambda-\mu}{4}(b-c)^2.
\end{align*}

Maintenant, les quatre formes linéaires $(a,b,c,d)\mapsto a+\frac{\mu}{2\lambda}d$, $(a,b,c,d)\mapsto d$, $(a,b,c,d)\mapsto b+c$ et $(a,b,c,d)\mapsto b-c$ sont linéairement indépendantes. Donc 

- si $\mu=2\lambda\;(\neq0)$, $r=1$,

- si $\mu=-2\lambda\;(\neq2\lambda)$, $r=3$,

- si $|\mu|\neq2|\lambda|\;(\neq0)$, $r=4$.

En particulier, si $\lambda=1$ et $\mu=0$, alors $r=4$ et $s=(3,1)$ et si $\lambda= 0$ et $\mu= 1$,  $r=4$ et $s = (2,2)$.}
\end{enumerate}
}