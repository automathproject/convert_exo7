\uuid{5057}
\titre{Position d'une surface de révolution  par rapport au plan tangent}
\theme{}
\auteur{quercia}
\date{2010/03/17}
\organisation{exo7}
\contenu{
  \texte{}
  \question{Soit ${\cal S}$ une surface d'équation cartésienne $z=f(\rho)$ où
$\rho = \sqrt{x^2+y^2}$ et $f$ est une fonction de classe $\mathcal{C}^2$.
Montrer que la position de ${\cal S}$ par rapport à son plan tangent est donnée
par le signe de $f'(\rho)f''(\rho)$. Interpréter géométriquement ce fait.}
  \reponse{}
}