\uuid{7081}
\titre{Trapèze rectangle}
\theme{}
\auteur{megy}
\date{2017/01/21}
\organisation{exo7}
\contenu{
  \texte{}
  \question{%tags : isocèle, trapèze, symétrie centrale, projection
Soit $\mathcal D$ une droite, $A$ et $B$ deux points hors de cette droite, et $A'$, $B'$ leurs projetés orthogonaux sur $\mathcal D$, supposés distincts. Soit enfin $I$ le milieu de $[AB]$. Montrer que $A'IB'$ est isocèle en $I$.
\begin{center}
\includegraphics{../images/img007081-1}
\end{center}}
  \reponse{On complète le trapèze rectangle $ABB'A'$ en un rectangle comme conseillé, en utiisant la symétrie de centre $I$. 
\begin{center}
\includegraphics{../images/img007081-2}
\end{center}
Le résultat demandé est alors une conséquence du fait que les diagonales d'un rectangle sont égales et se coupent en leur milieu. 

%On peut aussi utiliser le théorème des milieux.}
}