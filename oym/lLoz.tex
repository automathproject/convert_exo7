\uuid{lLoz}
\exo7id{4875}
\titre{$f^3 = \mathrm{id}$}
\theme{Exercices de Michel Quercia, Applications affines}
\auteur{quercia}
\date{2010/03/17}
\organisation{exo7}
\contenu{
  \texte{Soit ${\cal P}$ un plan, et $f : {\cal P} \to {\cal P}$ une application
affine telle que $f^3 = \mathrm{id}$, avec $f \ne \mathrm{id}$.}
\begin{enumerate}
  \item \question{Montrer que si $A \ne f(A)$, alors $A,f(A), f^2(A)$ sont non alignés.}
  \item \question{En déduire que $f$ est le produit de deux symétries.}
\end{enumerate}
\begin{enumerate}

\end{enumerate}
}