\uuid{3289}
\titre{Inversion de la matrice $(1/(a_i-b_j))$}
\theme{Exercices de Michel Quercia, Décomposition en éléments simples}
\auteur{quercia}
\date{2010/03/08}
\organisation{exo7}
\contenu{
  \texte{}
  \question{Soient $a_1, \dots, a_n, b_1, \dots, b_n$, et $c$ des scalaires distincts.
On note $A$ la matrice carr{\'e}e $\left( \frac 1{a_i-b_j} \right)$ et
$B$ la matrice colonne $\left( \frac 1{a_i-c} \right)$.
Montrer que l'{\'e}quation $AX = B$ poss{\`e}de une solution unique en consid{\'e}rant une
fraction rationnelle bien choisie.}
  \reponse{$F = \sum_{j=1}^n \frac {x_j}{X-b_j} - \frac 1{X-c}
 = \lambda \frac {\prod(X-a_i)}{(X-c)\prod(X-bj)}$ o{\`u}
$\lambda = -\prod \frac {c-b_i}{c-a_i}$.}
}