\uuid{3546}
\titre{TPE 93}
\theme{Exercices de Michel Quercia, Réductions des endomorphismes}
\auteur{quercia}
\date{2010/03/10}
\organisation{exo7}
\contenu{
  \texte{}
  \question{Soit $A \in \mathcal{M}_n(\C)$ telle que $A = A^{-1}$. $A$ est-elle diagonalisable ?
Calculer $e^A$. ($e^A = \sum_{k=0}^\infty \frac{A^k}{k!}$)}
  \reponse{$A$ est diagonalisable car $A^2=I$. $e^A = (\ch1)I + (\sh1)A$.}
}