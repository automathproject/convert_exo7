\uuid{3556}
\titre{Mines PSI 1998}
\theme{Exercices de Michel Quercia, Réductions des endomorphismes}
\auteur{quercia}
\date{2010/03/10}
\organisation{exo7}
\contenu{
  \texte{}
  \question{Soit $f$ un endomorphisme diagonalisable d'un ev~$E$ de dimension finie,
$\lambda$ une valeur propre de~$f$ et $p_\lambda$ le projecteur sur le
sous-espace propre associé parallèlement à la somme des autres sous-espaces
propres. Montrer que $p_\lambda$ est un polynôme en~$f$.}
  \reponse{Soit $P$ un polynôme tel que $P(\lambda) = 1$ et $P(\mu) = 0$ pour
toutes les autres valeurs propres, $\mu$, de~$f$. Alors $p_\lambda = P(f)$.}
}