\uuid{fcmp}
\exo7id{3518}
\titre{\'Etude d'une matrice}
\theme{Exercices de Michel Quercia, Réductions des endomorphismes}
\auteur{quercia}
\date{2010/03/10}
\organisation{exo7}
\contenu{
  \texte{Soit $A=\begin{pmatrix}a_1 &1  &       &(0)\cr
                 a_2 &   &\ddots &   \cr
              \vdots &   &       &1  \cr
                 a_n &(0)&       &0  \cr\end{pmatrix}$
où les $a_i$ sont des réels positifs ou nuls, avec $a_1a_n > 0$.}
\begin{enumerate}
  \item \question{Quel est le polynôme caractérique de~$A$~?}
  \item \question{Montrer que $A$ admet une unique valeur propre $r>0$ et que l'on
    a~$r < 1 + \max(a_1,\dots,a_n)$.}
  \item \question{Soit $\lambda$ une valeur propre complexe de~$A$.
    Montrer que $|\lambda| \le r$ et $|\lambda|=r  \Rightarrow  \lambda=r$.}
  \item \question{Montrer qu'il existe un entier $k$ tel que $A^k$ a tous ses
    coefficients strictement positifs.}
\end{enumerate}
\begin{enumerate}
  \item \reponse{$(-1)^n(X^n-a_nX^{n-1}-\dots-a_1)$.}
  \item \reponse{\'Etude de $x \mapsto (x^n-a_nx^{n-1}-\dots-a_1)/x^n$.}
  \item \reponse{Inégalité triangulaire.}
  \item \reponse{Expression générale de $A^k$.}
\end{enumerate}
}