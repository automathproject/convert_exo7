\uuid{3614}
\titre{Plan affine stable}
\theme{Exercices de Michel Quercia, Réductions des endomorphismes}
\auteur{quercia}
\date{2010/03/10}
\organisation{exo7}
\contenu{
  \texte{}
  \question{Soit $E = \R^3$ et $H:x+2y+3z=1$ un plan {\it affine\/} de $E$.
Montrer que si $H$ est stable par $f\in\mathcal{L}(E)$ alors $1$ est
valeur propre de~$f$.}
  \reponse{Soit $\varphi(x,y,z) = x+2y+3z$. $f$ conserve la surface de
niveau $\varphi=1$ donc par linéarité $f\circ\varphi=\varphi$ et
$\varphi$ est vecteur propre de ${}^tf$.}
}