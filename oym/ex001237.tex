\uuid{1237}
\titre{Exercice 1237}
\theme{}
\auteur{legall}
\date{1998/09/01}
\organisation{exo7}
\contenu{
  \texte{Donner le d\'eveloppement limit\'e en $  0  $ des
fonctions~:}
\begin{enumerate}
  \item \question{$  x\mapsto \ln (\hbox{cos}
(x))  $ (\`a l'ordre $  6  $).}
  \item \question{$  x \mapsto \tan(x)  $ (\`a l'ordre $ 7  $).}
  \item \question{$  x \mapsto \sin(\tan(x))  $ (\`a l'ordre $ 7  $).}
  \item \question{$  x\mapsto (\ln(1+x))^2  $ (\`a l'ordre $ 4  $).}
  \item \question{$  x\mapsto \exp(\sin(x))  $ (\`a l'ordre $ 3  $).}
  \item \question{$  x \mapsto \sin^6(x)  $ (\`a l'ordre $ 9  .$)}
\end{enumerate}
\begin{enumerate}
  \item \reponse{$\displaystyle{\ln(\cos x) = -{\frac {1}{2}}{x}^{2}-{\frac {1}{12}}{x}^{4}-{\frac {1}{45}}{x}^{6}+
o\left ({x}^{6}\right ) }$.}
  \item \reponse{$\displaystyle{\tan x = x+{\frac {1}{3}}{x}^{3}+{\frac {2}{15}}{x}^{5}+{\frac {17}{315}}{x}^{
7}+o\left ({x}^{7}\right )}$.}
  \item \reponse{$\displaystyle{\sin(\tan x) = x+{\frac {1}{6}}{x}^{3}-{\frac {1}{40}}{x}^{5}-{\frac {55}{1008}}{x}^
{7}+o\left ({x}^{7}\right ) }$.}
  \item \reponse{$\displaystyle{\left( \ln (1+x) \right)^2={x}^{2}-{x}^{3}+{\frac {11}{12}}{x}^{4}+o\left ({x}^{4}\right ) }$.}
  \item \reponse{$\displaystyle{\exp(\sin x) = 1+x+{\frac {1}{2}}{x}^{2}+o\left ({x}^{3}\right ) }$.}
  \item \reponse{$\displaystyle{\sin^6 x = {x}^{6}+o\left ({x}^{6}\right ) }$.}
\end{enumerate}
}