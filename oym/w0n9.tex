\uuid{w0n9}
\exo7id{3268}
\titre{Polyn{\^o}mes dont les racines sont de module 1}
\theme{Exercices de Michel Quercia, Fonctions symétriques}
\auteur{quercia}
\date{2010/03/08}
\organisation{exo7}
\contenu{
  \texte{Soit $n \in \N^*$ et ${\cal E}$ l'ensemble des polyn{\^o}mes {\`a} coefficients
entiers, unitaires de degr{\'e} $n$ et dont toutes les racines sont de module 1.}
\begin{enumerate}
  \item \question{D{\'e}montrer que $\cal E$ est fini.}
  \item \question{Pour $P \in \cal E$ de racines $x_1,\dots,x_n$, on note $\widetilde P$ le
      polyn{\^o}me unitaire de racines $x_1^2,\dots,x_n^2$.

      D{\'e}montrer que $\widetilde P \in \cal E$.}
  \item \question{En d{\'e}duire que : $\forall\ P \in \cal E$, les racines de $P$ sont des racines de
      l'unit{\'e}.}
\end{enumerate}
\begin{enumerate}
  \item \reponse{Les coefficients de $P$ sont born{\'e}s.}
  \item \reponse{$\widetilde P(X^2) = (-1)^nP(X)P(-X)  \Rightarrow  \widetilde P \in {\Z[X]}$.}
  \item \reponse{La suite $(\widetilde {\dot{\dot{\widetilde P}}})$ prend un nombre
      fini de valeurs.}
\end{enumerate}
}