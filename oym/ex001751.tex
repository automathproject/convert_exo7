\uuid{1751}
\titre{Exercice 1751}
\theme{}
\auteur{maillot}
\date{2001/09/01}
\organisation{exo7}
\contenu{
  \texte{Soient $A$ et $B$ deux parties non vides et majorées de $\R$. On définit
:
$$
A+B=\{c\in\R\ | \ \exists a\in A,\exists b\in B, c=a+B\}.
$$}
\begin{enumerate}
  \item \question{Montrer que $A+B$ admet une borne supérieure, puis que
$\sup(A+B)=\sup A+\sup B$.}
  \item \question{Montrer l'implication :
$$
\exists M\in\R\;\forall x\in A,\forall y\in B,\;x+y<M\Rightarrow\sup A+\sup B\leq M.
$$}
\end{enumerate}
\begin{enumerate}

\end{enumerate}
}