\uuid{3476}
\titre{$MA=0$, Chimie P' 90}
\theme{Exercices de Michel Quercia, Rang de matrices}
\auteur{quercia}
\date{2010/03/10}
\organisation{exo7}
\contenu{
  \texte{}
  \question{Soit $A \in \mathcal{M}_n(\R)$ et $E = \{ M \in \mathcal{M}_n(\R)$ tq $MA = 0 \}$.
    Quelle est la structure de $E$, sa dimension~?}
  \reponse{$E$ est un sev et un idéal à gauche de $\mathcal{M}_n(\R)$.
	     Il est isomorphe à $\mathcal{L}({H,\R^n})$ où $H$ est un supplémentaire de
	     $\Im A$ dans $\R^n$. $\dim E = n(n-\mathrm{rg}(A))$.}
}