\uuid{nL1D}
\exo7id{5186}
\titre{***I}
\theme{Espaces vectoriels de dimension finie (ou non)}
\auteur{rouget}
\date{2010/06/30}
\organisation{exo7}
\contenu{
  \texte{Soit $E=\Rr_n[X]$, le $\Rr$-espace vectoriel des polynômes à coefficients réels de degré inférieur ou égal à $n$ ($n$
entier naturel donné). Soit $\varphi$ l'application définie par~:~$\forall P\in E,\;\varphi(P)=P(X+1)-P(X)$.}
\begin{enumerate}
  \item \question{Vérifier que $\varphi$ est un endomorphisme de $E$.}
  \item \question{Déterminer $\mbox{Ker}\varphi$ et $\mbox{Im}\varphi$.}
\end{enumerate}
\begin{enumerate}
  \item \reponse{Si $P$ est un polynôme de degré inférieur ou égal à $n$, alors $P(X+1)-P(X)$ est encore un polynôme de degré
inférieur ou égal à $n$. Par suite, $\varphi$ est bien une application de $E$ dans lui-même.
Soient alors $(P,Q)\in E^2$ et $(\lambda,\mu)\in\Rr^2$.

\begin{align*}
\varphi(\lambda P+\mu Q)&=(\lambda P+\mu Q)(X+1)-(\lambda P+\mu
Q)(X)=\lambda(P(X+1)-P(X))+\mu(Q(X+1)-Q(X))\\
 &=\lambda\varphi(P)+\mu\varphi(Q).
\end{align*}
$\varphi$ est linéaire de $E$ vers lui-même et donc un endomorphisme de $E$.}
  \item \reponse{Soit $P\in E$.
$P\in\mbox{Ker }\varphi\Leftrightarrow\forall x\in\Rr,\;P(x+1)=P(x)$. Montrons alors que $P$ est constant.
Soit $Q=P-P(0)$. $Q$ est un polynôme de degré inférieur ou égal à $n$ s'annulant en les entiers naturels 0, 1, 2,...
(car $P(0)=P(1)=P(2)=...$) et a ainsi une infinité de racines deux à deux distinctes. $Q$ est donc le polynôme nul ou
encore $\forall x\in\Rr,\;P(x)=P(0)$. Par suite, $P$ est un polynôme constant.
Réciproquement, les polynômes constants sont clairement dans $\mbox{Ker }\varphi$ et donc

$$\mbox{Ker }\varphi=\{\mbox{polynômes constants}\}=\Rr_0[X].$$
Pour déterminer $\mbox{Im }\varphi$, on note tout d'abord que si $P$ est un polynôme de degré inférieur ou égal à $n$,
alors $\varphi(P)=P(X+1)-P(X)$ est un polynôme de degré inférieur ou égal à $n-1$. En effet, si
$P=a_nX^n+\sum_{k=0}^{n-1}a_kX^k$ (avec $a_n$ quelconque, éventuellement nul) alors

\begin{align*}
\varphi(P)&=a_n((X+1)^n-X^n)+\mbox{termes de degré inférieur on égal à}\;n-1\\
 &=a_n(X^n-X^n)+\mbox{termes de degré
inférieur on égal à}\;n-1\\
 &=\mbox{termes de degré inférieur on égal à}\;n-1.
\end{align*}
Donc, $\mbox{Im }(\varphi)\subset\Rr_{n-1}[X]$. Mais d'après le théorème du rang,

$$\mbox{dim }\mbox{Im }(\varphi)=\mbox{dim }\Rr_n[X]-\mbox{dim }\mbox{Ker }(\varphi)=(n+1)-1=n=\mbox{dim }\Rr_{n-1}[X]
<+\infty,$$
et donc $\mbox{Im }\varphi=\Rr_{n-1}[X]$. (On peut noter que le problème difficile~\og~soit $Q\in\Rr_{n-1}[X]$.
Existe-t-il $P\in\Rr_n[X]$ tel que $P(X+1)-P(X)=Q$~?~\fg~a été résolu simplement par le théorème du rang.)}
\end{enumerate}
}