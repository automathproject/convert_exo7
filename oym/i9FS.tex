\uuid{i9FS}
\exo7id{5053}
\titre{Contour apparent}
\theme{Exercices de Michel Quercia, Surfaces paramétrées}
\auteur{quercia}
\date{2010/03/17}
\organisation{exo7}
\contenu{
  \texte{Soit ${\cal S}$ la surface d'équation cartésienne $z^2-x^2-y^2 = 1$.}
\begin{enumerate}
  \item \question{Reconnaître ${\cal S}$.}
  \item \question{Soit $D$ la droite d'équations : $2x+y = 0$, $z=0$. Déterminer les points
    $M$ de ${\cal S}$ tels que le plan tangent à ${\cal S}$ en $M$ est
    parallèle à $D$.
    (Contour apparent de ${\cal S}$ dans la direction de $D$)}
\end{enumerate}
\begin{enumerate}
  \item \reponse{Hyperboloïde de révolution à deux nappes.}
  \item \reponse{$x = 2y$, $z^2 = 1 + 5y^2$.}
\end{enumerate}
}