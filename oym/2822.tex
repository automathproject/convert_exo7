\uuid{2822}
\titre{Exercice 2822}
\theme{Prolongement analytique et résidus, Un peu de topologie}
\auteur{burnol}
\date{2009/12/15}
\organisation{exo7}
\contenu{
  \texte{}
  \question{Soient $f$ et $g$ deux fonctions entières avec $\forall z\
f(z)g(z) = 0$. Montrer que l'une des deux est identiquement
nulle.}
  \reponse{Si $f(z)g(z) = 0$ pour tout $z\in \C$, alors au moins une des fonctions $f,g$ a un z\'ero non isol\'e.}
}