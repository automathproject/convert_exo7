\uuid{TGoZ}
\exo7id{7756}
\titre{Exercice 7756}
\theme{Exercices de Christophe Mourougane, Théorie des groupes et géométrie}
\auteur{mourougane}
\date{2021/08/11}
\organisation{exo7}
\contenu{
  \texte{}
\begin{enumerate}
  \item \question{Soit $6$ points $A,B,\ldots, F$ du plan $\R^2$ tels que $ABCDEF$ soit un hexagone régulier,
    Etant données les images $A'=h(A)$, $B'=h(B)$, $D'=h(D)$ et $E'=h(E)$ par une homographie $h$ de $P^2(\R)$ dans lui-même, construire à la règle les images des autres points.
    \includegraphics[scale=0.5]{images/img-mour-508-1}}
  \item \question{Même question en supposant données cette fois, les images $A'=h(A)$, $B'=h(B)$, $C'=h(C)$ et $D'=h(D)$.}
\end{enumerate}
\begin{enumerate}

\end{enumerate}
}