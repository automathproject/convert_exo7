\uuid{3022}
\titre{Morphismes $\Z^n \to \Z$}
\theme{Exercices de Michel Quercia, Anneaux}
\auteur{quercia}
\date{2010/03/08}
\organisation{exo7}
\contenu{
  \texte{}
  \question{\label{morphismes}
Chercher les morphismes d'anneaux : $\Z^n \to \Z$.}
  \reponse{$f(x_1,\dots,x_n) = a_1x_1 + \dots + a_nx_n$.\par
$f$ est multiplicative sur la base canonique $ \Rightarrow $ $a_ia_j = 0$ pour
$i\ne j$.\par
$f(1,\dots,1) = 1  \Rightarrow  $ un des $a_i$ vaut $1$, et les autres $0$.\par
conclusion : $f$ = fct coordonn{\'e}e.}
}