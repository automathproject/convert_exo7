\uuid{PYFi}
\exo7id{4650}
\titre{Phénomène de Gibbs pour $\sin kx/k$}
\theme{Exercices de Michel Quercia, Séries de Fourier}
\auteur{quercia}
\date{2010/03/14}
\organisation{exo7}
\contenu{
  \texte{Soit $f_n(x) = \sum_{k=1}^n \frac{\sin kx}k$.}
\begin{enumerate}
  \item \question{Calculer l'abscisse, $x_n$, du premier maximum positif de $f_n$.}
  \item \question{Déterminer $\lim_{n\to\infty} f_n(x_n)$.}
\end{enumerate}
\begin{enumerate}
  \item \reponse{$\frac\pi{n+1}$.}
  \item \reponse{Somme de Riemman : $\ell =  \int_{t=0}^\pi \frac{\sin t}t\,d t$.}
\end{enumerate}
}