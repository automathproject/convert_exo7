\exo7id{3396}
\titre{Calcul de $A^n$ par polynôme annulateur}
\theme{}
\auteur{quercia}
\date{2010/03/10}
\organisation{exo7}
\contenu{
  \texte{Soit $A = \begin{pmatrix} 1 &2 &3\cr 2 &3 &1\cr 3 &1 &2 \cr\end{pmatrix}$.}
\begin{enumerate}
  \item \question{Vérifier que $(A-6I)(A^2-3I) = 0$.}
  \item \question{Soit $n\in\N$ et $P_n$ le polynôme de degré inférieur ou égal à 2
    tel que
    $${P(6) = 6^n},\quad {P\bigl(\sqrt3\bigr) = \bigl(\sqrt3\bigr)^n},\quad
    \text{et } {P\bigl(-\sqrt3\bigr) = \bigl(-\sqrt3\bigr)^n}.$$
    Montrer que $A^n = P_n(A)$.}
  \item \question{Même question pour $n \in \Zz$.}
\end{enumerate}
\begin{enumerate}

\end{enumerate}
}