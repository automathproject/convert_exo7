\uuid{XSTi}
\exo7id{2582}
\titre{Exercice 2582}
\theme{Sujets de l'année 2005-2006, Partiel}
\auteur{delaunay}
\date{2009/05/19}
\organisation{exo7}
\contenu{
  \texte{Soit $a\in\R$ et $A$ la matrice suivante
$$A=\begin{pmatrix}1&0&a \\  0&a&1 \\  a&1&0\end{pmatrix}$$}
\begin{enumerate}
  \item \question{Calculer le d\'eterminant de $A$ et d\'eterminer 
pour quelles valeurs de $a$ la matrice est inversible.}
  \item \question{Calculer $A^{-1}$ lorsque $A$ est inversible.}
\end{enumerate}
\begin{enumerate}
  \item \reponse{Calculons le d\'eterminant de $A$ et d\'eterminons 
pour quelles valeurs de $a$ la matrice est inversible.

$$\det A=\begin{vmatrix}1&0&a \\  0&a&1 \\  a&1&0\end{vmatrix}=\begin{vmatrix}a&1 \\ 1&0\end{vmatrix}+a\begin{vmatrix}0&a \\  a&1\end{vmatrix}=-1-a^3.$$
La matrice $A$ est inversible si et seulement si son d\'eterminant est non nul, c'est-\`a-dire si et seulement si $1+a^3\neq 0$, ce qui \'equivaut \`a $a\neq-1$ car $a\in \R$.}
  \item \reponse{Calculons $A^{-1}$ lorsque $A$ est inversible, c'est-\`a-dire $a\neq-1$. Pour cela nous allons d\'eterminer la comatrice $\tilde A$ de $A$. On a 
$$\tilde A=\begin{pmatrix}-1&a&-a^2 \\  a&-a^2&-1 \\  -a^2&-1&a\end{pmatrix},$$
on remarque que $\tilde A=^t\!\!\!\tilde A$ et on a bien $A ^t\!\!\tilde A=^t\!\!\!\tilde AA=(-1-a^3)I_3$ d'o\`u
$$A^{-1}={\frac{1}{(-1-a^3)}}\tilde A={\frac{1}{-1-a^3}}\begin{pmatrix}-1&a&-a^2 \\  a&-a^2&-1 \\  -a^2&-1&a\end{pmatrix}.$$}
\end{enumerate}
}