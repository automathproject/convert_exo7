\uuid{643}
\titre{Exercice 643}
\theme{}
\auteur{ridde}
\date{1999/11/01}
\organisation{exo7}
\contenu{
  \texte{}
  \question{Soit $f : \Rr \rightarrow \Rr$ continue telle que $\lim\limits_{ -\infty}f =  -\infty$
et $\lim\limits_{  + \infty}f =   + \infty$. Montrer que $f$ s'annule. Appliquer
ceci aux polynômes de degr\'e impair.}
  \reponse{Il existe $x < 0$ tel que $f(x) <0$ et $y>0$ tel que $f(y) > 0$,
d'après le théorème des valeurs intermédiaires,
il existe $z \in ]x,y[$ tel que $f(z) = 0$.
Donc $f$ s'annule. 
Les polynômes de degré impair vérifient les propriétés des limites, donc
s'annulent. Ceci est faux, en général,  pour les polynômes de degré pair, par exemple regardez $f(x) = x^2+1$.}
}