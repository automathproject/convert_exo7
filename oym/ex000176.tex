\exo7id{176}
\titre{Exercice 176}
\theme{}
\auteur{cousquer}
\date{2003/10/01}
\organisation{exo7}
\contenu{
  \texte{On définit une suite $(F_n)$ de la fa\c con suivante :
$$F_{n+1} = F_n+F_{n-1} ; \quad F_0=1 , F_1=1 \ .$$}
\begin{enumerate}
  \item \question{Calculer $F_n$ pour $1<n<10$.}
  \item \question{Montrer que l'équation $x^2=x+1$ admet une unique solution positive 
$a$ que l'on calculera.}
  \item \question{Montrer que, pour tout $n\geq 2$, on a
$$a^{n-2}< F_n < a^{n-1} \ .$$}
\end{enumerate}
\begin{enumerate}

\end{enumerate}
}