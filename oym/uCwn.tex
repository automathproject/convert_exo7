\uuid{uCwn}
\exo7id{7597}
\titre{Deux équations}
\theme{Exercices de Christophe Mourougane, 446.00 - Fonctions holomorphes (Examens)}
\auteur{mourougane}
\date{2021/08/10}
\organisation{exo7}
\contenu{
  \texte{}
\begin{enumerate}
  \item \question{Trouver toutes les solutions complexes de l'équation $\cosh(z)=4i$.}
  \item \question{Trouver toutes les solutions complexes de l'équation $z^i=-1$.}
\end{enumerate}
\begin{enumerate}
  \item \reponse{Par définition de $\cosh$, on a pour tout $z\in\Cc$, $\cosh (z)=\frac{e^z+e^{-z}}{2}$.
Par conséquent, $z$ est solution de $\cosh(z)=8i$ si et seulement si $e^z$ est solution de
$$X^2-8iX+1=0.$$
si et seulement si $e^z=(4+\sqrt{17})i=(4+\sqrt{17})e^{-i\frac{\pi}{2}}$ \\quad\text{ou}\quad $e^z=(4-\sqrt{17})i=(\sqrt{17}-4)e^{-i\frac{\pi}{2}}$,
si et seulement si $z\in\log (4+\sqrt{17})+i(\frac{\pi}{2}+2\pi\Zz)$ ou $z\in\log (\sqrt{17}-4)+i(-\frac{\pi}{2}+2\pi\Zz)$.}
  \item \reponse{Soit $z\in\Cc$.
\begin{eqnarray*}
 z^i=-1&\iff& e^{i\log z}=e^{i\pi}\\
 &\iff& i\log z=i\pi \mod 2i\pi\\
 &\iff& \log z=\pi \mod 2\pi\\
 &\iff& \exists k\in\Zz, z = e^{(2k+1)\pi}.
\end{eqnarray*}
Les solutions sont les nombres complexes de la forme $e^{(2k+1)\pi}$ avec $k$ entier.}
\end{enumerate}
}