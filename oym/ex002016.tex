\uuid{2016}
\titre{Exercice 2016}
\theme{}
\auteur{liousse}
\date{2003/10/01}
\organisation{exo7}
\contenu{
  \texte{On consid\`ere les cinq points suivants:
 $A(1,2,-1)$, $B(3,2,0)$, $C(2,1,-1)$, $D(1,0,4)$ et $E(-1,1,1)$.}
\begin{enumerate}
  \item \question{Ces quatre points sont-ils coplanaires ?}
  \item \question{D\'eterminer la nature du triangle $ABC$. $A$, $B$ et $C$ sont-ils align\'es, si non
 donner une  \'equation cat\'esienne du plan $P$ qui les contient.}
  \item \question{D\'eterminer les coordonn\'ees
du barycentre $G$ des points $A$, $B$, $C$ et $D$.}
  \item \question{Montrer que $O$, $D$ et $G$ sont align\'es et que la droite $OD$ est perpendiculaire \`a $P$.}
\end{enumerate}
\begin{enumerate}

\end{enumerate}
}