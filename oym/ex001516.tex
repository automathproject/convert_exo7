\exo7id{1516}
\titre{Exercice 1516}
\theme{}
\auteur{ortiz}
\date{1999/04/01}
\organisation{exo7}
\contenu{
  \texte{Soit la forme quadratique $q$ d\'efinie par
$${
q:\left( x_1,x_2,x_3,x_4\right) \in
\Cc^4\mapsto
x_1x_2+x_2x_4-x_3x_4-2x_1x_4-2x_2x_3-x_1x_3}.$$}
\begin{enumerate}
  \item \question{Montrer, sans r\'eduire $q$, qu'il existe
une base $q$-orthonormale de $\Cc^4.$}
  \item \question{En expliciter une.}
\end{enumerate}
\begin{enumerate}

\end{enumerate}
}