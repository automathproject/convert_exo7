\uuid{3335}
\titre{$f\circ g = 0$ et $f+g \in GL(E)$}
\theme{Exercices de Michel Quercia, Applications linéaires en dimension finie}
\auteur{quercia}
\date{2010/03/09}
\organisation{exo7}
\contenu{
  \texte{}
  \question{Soit $E$ de dimension finie et $f,g \in \mathcal{L}(E)$ tels que :
$\begin{cases} f\circ g = 0 \cr f+g \in GL(E).\end{cases}$

Montrer que $\mathrm{rg} f + \mathrm{rg} g = \dim E$.}
  \reponse{$\Im f \subset \mathrm{Ker} g  \Rightarrow  \mathrm{rg} f + \mathrm{rg} g \le \dim E$.\par
         $f+g$ est surjective $ \Rightarrow  \Im f + \Im g = E
                                \Rightarrow  \mathrm{rg} f + \mathrm{rg} g \ge \dim E$.}
}