\uuid{3649}
\titre{$e_n^*$ imposé}
\theme{Exercices de Michel Quercia, Dualité}
\auteur{quercia}
\date{2010/03/10}
\organisation{exo7}
\contenu{
  \texte{}
  \question{Soit $E$ un $ K$-ev de dimension finie $n$,
${\cal F} = ({\vec e}_1,\dots, {\vec e}_{n-1})$ une famille libre de $E$
et $f \in E^*\setminus\{0\}$.
Montrer qu'on peut compléter $\cal F$ en une base
${\cal B} = ({\vec e}_1,\dots,{\vec e}_n)$
telle que $f = e_n^*$ si et seulement si
$f({\vec e}_1) = \dots = f({\vec e}_{n-1}) = 0$.

Y~a-t-il unicité de ${\vec e}_n$ ?}
  \reponse{}
}