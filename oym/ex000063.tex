\exo7id{63}
\titre{Exercice 63}
\theme{}
\auteur{cousquer}
\date{2003/10/01}
\organisation{exo7}
\contenu{
  \texte{Le plan $P$ est rapporté à un repère orthonormé et on identifie $P$ à l'ensemble des nombres complexes $\C$ par 
$$
M(x,y) \mapsto x+iy = z,
$$
où $z$ est appelé l'affixe de $M.$
Soit $g : P \mathrm{rg} P$ qui à tout point $M$ d'fixe $z \not= -1$ associe $g(M)$ d'affixe $z' = \frac{1-z}{1+z}$.}
\begin{enumerate}
  \item \question{Calculer $z' + \bar{z'}$ pour $|z| = 1$.}
  \item \question{En déduire l'image du cercle de rayon $1$ de centre $0$  privé du point de coordonnées $(-1,0)$ par l'application $g.$}
\end{enumerate}
\begin{enumerate}

\end{enumerate}
}