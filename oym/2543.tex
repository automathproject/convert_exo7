\uuid{2543}
\titre{Exercice 2543}
\theme{Théorème des fonctions implicites}
\auteur{tahani}
\date{2009/04/01}
\organisation{exo7}
\contenu{
  \texte{}
  \question{On consid\`ere le syst\`eme
d'\'equations:
$$\left( \begin{array}{c}
x^2+y^2-2z^2=0 \\ x^2+2y^2+z^2=4
\end{array} \right) $$
Montrer que, pour $x$ proche de l'origine, il existe des fonctions
positives $y(x)$ et $z(x)$ telles que $(x,y(x),z(x))$ soit
solution du syst\`eme. On d\'eterminera $y'$ en fonctionde $x,y$
et $z'$ en fonction de $x,z$.}
  \reponse{}
}