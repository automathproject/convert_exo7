\uuid{927}
\titre{Exercice 927}
\theme{}
\auteur{cousquer}
\date{2003/10/01}
\organisation{exo7}
\contenu{
  \texte{\emph{Notations :\\
$\mathcal{C}$~: ensemble des fonctions numériques continues sur $[0,1]$.\\ 
$\mathcal{C}_d$~: ensemble des fonctions numériques ayant
une dérivée continue sur $[0,1]$.\\ 
$\mathcal{C}(\mathbb{R})$ et $\mathcal{C}^1(\mathbb{R})$~: définis de façon analogue
pour les fonctions définies sur $\mathbb{R}$.\\ 
$\mathcal{P}$~: ensemble des polynômes sur $\mathbb{R}$.\\ 
$\mathcal{P}_n$~: ensemble des polynômes sur $\mathbb{R}$, de degré $\leq n$.}


\medskip

Dire si les applications suivantes sont des applications linéaires~:}
\begin{enumerate}
  \item \question{$\mathbb{R} \rightarrow \mathbb{R} : x\mapsto 2x^2$.}
  \item \question{$\mathbb{R} \rightarrow \mathbb{R} : x\mapsto 4x-3$.}
  \item \question{$\mathbb{R} \rightarrow \mathbb{R} : x\mapsto \sqrt{x^2}$.}
  \item \question{$\mathbb{R}^2 \rightarrow \mathbb{R} ^2  :(x,y)\mapsto (y,x)$.}
  \item \question{$\mathcal{C}\rightarrow \mathcal{C} : f \mapsto \{t\mapsto {f(t)\over1+t^2}\}$.}
  \item \question{$\mathcal{C} \rightarrow \mathbb{R} : f\mapsto  f(3/4)$.}
  \item \question{$\mathcal{C} \rightarrow \mathbb{R} : f\mapsto  f(1/4)-\int_{1/2}^1f(t)\,dt$.}
  \item \question{$\mathbb{R}^2 \rightarrow \mathbb{R} :(x,y)\mapsto 3x+5y$.}
  \item \question{$\mathbb{R}^2 \rightarrow \mathbb{R} : (x,y)\mapsto \sqrt{3x^2+5y^2}$.}
  \item \question{$\mathbb{R}^2 \rightarrow \mathbb{R} : (x,y)\mapsto \sin(3x+5y)$.}
  \item \question{$\mathbb{R}^2 \rightarrow \mathbb{R}^2 : (x,y)\mapsto (-x,y)$.}
  \item \question{$\mathbb{R}^2 \rightarrow \mathbb{R} :(x,y)\mapsto  xy$.}
  \item \question{$\mathbb{R}^2 \rightarrow \mathbb{R} : (x,y)\mapsto 
{x^2 y\over x^2+y^2} \mbox{ si } x^2+y^2 \ne0 \mbox{ et} 0 \mbox{ sinon}.$}
  \item \question{$\mathcal{C} \rightarrow \mathcal{C}_d : f\mapsto \{x\mapsto  e^{-x}\int_0^1f(t)\,dt\}$.}
  \item \question{$\mathcal{P} \rightarrow \mathcal{P}_n : A \mapsto $ quotient de $A$ par $B$
à l'ordre $n$ selon les puissances croissantes ($B$ et $n$ fixés,
avec $B(0)\ne0$).}
  \item \question{$\mathbb{R}^2 \rightarrow \mathbb{R}^2 : M \mapsto  M'$ défini par: ${\overrightarrow{OM'}=
{\overrightarrow{OM}\over\bigl\Vert\overrightarrow{OM}\bigr\Vert} \mbox{ si } 
\overrightarrow{OM}\ne\overrightarrow 0 \mbox{ et  } 0 \mbox{ sinon}.}$}
  \item \question{$\mathbb{R}^3 \rightarrow \mathbb{R} : M\mapsto \overrightarrow{OM}\cdot \overrightarrow V$
où $\overrightarrow V=(4,-1,1/2)$.}
  \item \question{$\mathbb{R} \rightarrow \mathbb{R}^3 : x\mapsto (2x,x/\pi,x\sqrt2)$.}
  \item \question{$\mathcal{C} \rightarrow \mathbb{R} : f\mapsto \max_{t\in[0,1]}f(t)$.}
  \item \question{$\mathcal{C} \rightarrow \mathbb{R} : f\mapsto \max_{t\in[0,1]}f(t)-\min_{t\in[0,1]}f(t)$.}
  \item \question{$\mathbb{R}^2 \rightarrow \mathbb{R}^2 : (x,y)\mapsto $ la solution du système d'équations
en $(u,v)$~:
$$\left\{
 \begin{array}{rcl}
   3u-v  & = & x \\
   6u+2v & = & y.
 \end{array}
\right.$$}
  \item \question{$\mathbb{R}^2 \rightarrow \mathbb{R}^2 : (x,y)\mapsto $ le symétrique de $(x,y)$ par rapport à
la droite d'équation $x+y-a=0$ (discuter selon les valeurs de $a$).}
  \item \question{$\mathbb{R}^3 \rightarrow \mathbb{R}^3 : (x,y,z)\mapsto $ la projection de $(x,y,z)$ sur le plan
$x+y+z-a=0$ parallèlement à $Oz$ (discuter selon les valeurs de $a$).}
  \item \question{$\mathcal{C}_d \rightarrow \mathcal{C} : f\mapsto f'$.}
  \item \question{$\mathbb{R}^3 \rightarrow \mathbb{R}^2 : (x,y,z)\mapsto (2x-3y+z,x-y+z/3)$.}
  \item \question{$\mathbb{R} \rightarrow \mathcal{C}d : \lambda\mapsto $ la solution de l'équation
différentielle $y'-{y\over x^2+1}=0$ valant $\lambda$ en $x_0=1$.}
  \item \question{$\mathcal{C}\rightarrow \mathbb{R} : f\mapsto \int_0^1\ln(1+\vert f(t)\vert)\,dt$.}
  \item \question{$\mathbb{R} \rightarrow \mathbb{R} : x\mapsto $ la $17$-ième décimale de $x$
(en écriture décimale).}
  \item \question{$\mathcal{C}_d \rightarrow \mathbb{R} : f \mapsto  f'(1/2)+\int_0^1{f(t)\,dt}$.}
  \item \question{$\mathbb{R} \rightarrow \mathbb{R} : x\mapsto \ln(3^{x\sqrt2})$.}
  \item \question{$\mathbb{R} \times \mathcal{C}(\mathbb{R}) \rightarrow \mathcal{C}(\mathbb{R}) : (\lambda,f)\mapsto $
la primitive de $f$ qui vaut $\lambda$ en $x_0=\pi$.}
  \item \question{$\mathcal{C}^1(\mathbb{R}) \rightarrow \mathcal{C}(\mathbb{R}): f\mapsto \{x\mapsto 
f'(x)+f(x) \cdot
\sin x\}$.}
\end{enumerate}
\begin{enumerate}

\end{enumerate}
}