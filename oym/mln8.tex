\uuid{mln8}
\exo7id{3664}
\titre{Inégalité de Ptolémée}
\theme{Exercices de Michel Quercia, Produit scalaire}
\auteur{quercia}
\date{2010/03/11}
\organisation{exo7}
\contenu{
  \texte{Soit $E$ un espace euclidien.}
\begin{enumerate}
  \item \question{Pour $\vec x \in E\setminus\{\vec 0\}$, on pose
    $f(\vec x) = \frac{\vec x}{\|\vec x\,\|^2}$.
    Montrer que : $\forall\ \vec x,\vec y \in E \setminus\{\vec 0\}$,
    $\|f(\vec x) - f(\vec y)\| = \frac{\|\vec x - \vec y\,\|}{\|\vec x\,\|\,\|\vec y\,\|}$.}
  \item \question{Soient $\vec a,\vec b,\vec c,\vec d \in E$.
    Montrer que $\|\vec a-\vec c\,\|\,\|\vec b-\vec d\,\| \le
         \|\vec a-\vec b\,\|\,\|\vec c-\vec d\,\| +
         \|\vec b-\vec c\,\|\,\|\vec a-\vec d\,\|$.


Indication : se ramener au cas $\vec a = \vec 0$ et utiliser l'application $f$.}
\end{enumerate}
\begin{enumerate}
  \item \reponse{\'Elever au carré.}
\end{enumerate}
}