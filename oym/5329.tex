\uuid{5329}
\titre{***T}
\theme{Polynômes}
\auteur{rouget}
\date{2010/07/04}
\organisation{exo7}
\contenu{
  \texte{}
  \question{Soit $(a_k)_{1\leq k\leq 5}$ la famille des racines de $P=X^5+2X^4-X-1$. Calculer $\sum_{k=1}^{5}\frac{a_k+2}{a_k-1}$.}
  \reponse{On note que $P(1)=1\neq0$ et donc que l'expression proposée a bien un sens.

$$\sum_{k=1}^{5}\frac{a_k+2}{a_k-1}=\sum_{k=1}^{5}(1+\frac{3}{a_k-1})=5-3\sum_{k=1}^{5}\frac{1}{1-a_k}=5-3\frac{P'(1)}{P(1)}=5-3\frac{12}{1}=-31.$$}
}