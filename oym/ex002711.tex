\exo7id{2711}
\titre{Exercice 2711}
\theme{}
\auteur{matexo1}
\date{2002/02/01}
\organisation{exo7}
\contenu{
  \texte{}
\begin{enumerate}
  \item \question{Montrer qu'un cercle $C$ de
diam\`etre $a$ et passant par le p\^ole $O$ peut \^etre
repr\'esent\'e en coordonn\'ees polaires par l'\'equation $\rho = a
\cos(\theta-\theta_0)$.
On consid\`ere un r\'eel $b > 0$ et la concho\"\i
de de $C$ de valeur $b$, c'est-\`a-dire la courbe
$\Gamma$ d\'efinie comme suit: \`a tout point $P$ de $C$, on
associe le point $M$ situ\'e sur la demi-droite $OP$, du
c\^ot\'e oppos\'e \`a $O$ par rapport \`a $P$ et tel que
$PM = b$; $\Gamma$ est le lieu des points $M$. Donner une
\'equation polaire de $\Gamma$. Construire $\Gamma$ en distinguant
quatre cas: $a>b$, $a=b$, $a<b<2a$ et $b \ge 2a$. D\'eterminer en
particulier les points d'inflexion dans chaque cas.}
  \item \question{En s'inspirant de la question pr\'ec\'edente, tracer les
concho\"\i des d'une droite $\rho = {a / \cos(\theta-\theta_0)}$.}
\end{enumerate}
\begin{enumerate}

\end{enumerate}
}