\uuid{1801}
\titre{Exercice 1801}
\theme{Dérivées partielles et directionnelles}
\auteur{ridde}
\date{1999/11/01}
\organisation{exo7}
\contenu{
  \texte{}
  \question{Soit $f : \R \rightarrow
\R$ d\'erivable. Calculer les d\'eriv\'ees partielles de :
\[
g (x, y) = f (x + y),\qquad h (x, y) = f (x^{2} + y^{2}),
\qquad k (x, y) = f (xy)
\]}
  \reponse{\begin{align*}
\frac{\partial g}{\partial x}(x,y)&= f'(x+y)
\\
\frac{\partial g}{\partial y}(x,y)&= f'(x+y)
\\
\frac{\partial h}{\partial x}(x,y)&= 2x f'(x^{2} + y^{2})
\\
\frac{\partial h}{\partial y}(x,y)&= 2y f'(x^{2} + y^{2})
\\
\frac{\partial k}{\partial x}(x,y)&= yf'(xy)
\\
\frac{\partial k}{\partial y}(x,y)&= xf'(xy)
\end{align*}}
}