\uuid{61jk}
\exo7id{4029}
\titre{Recherche d'asymptotes}
\theme{Exercices de Michel Quercia, Calculs de limites par développements limités}
\auteur{quercia}
\date{2010/03/11}
\organisation{exo7}
\contenu{
  \texte{Rechercher si les courbes suivantes admettent une asymptote en $+\infty$ et
déterminer la position s'il y a lieu :}
\begin{enumerate}
  \item \question{$y = \sqrt{x(x+1)}$.}
  \item \question{$y = \sqrt{\frac{x^3}{x-1}}$.}
  \item \question{$y = (x^2-1)\ln\left(\frac{x+1}{x-1}\right)$.}
  \item \question{$y = (x+1)\arctan(1+2/x)$.}
  \item \question{$y = x.\arctan x.e^{1/x}$.}
  \item \question{$y = e^{2/x}\sqrt{1+x^2}\arctan x$.}
  \item \question{$y = \sqrt{x^2-x}\exp\left(\frac 1{x+1}\right)$.}
\end{enumerate}
\begin{enumerate}
  \item \reponse{$y = x + \frac 12 - \frac 1{8x}$.}
  \item \reponse{$y = x + \frac12 + \frac3{8x}$.}
  \item \reponse{$y = 2x - \frac4{3x}$.}
  \item \reponse{$y = \frac{\pi x}4 + \frac\pi4+1 - \frac1{3x^2}$.}
  \item \reponse{$y = \frac {\pi x}2 + \frac \pi2-1 + \frac {\pi/4-1}x$.}
  \item \reponse{$y = \frac{\pi x}2 + \pi-1 + \frac{5\pi/4-2}x$.}
  \item \reponse{$y = x + \frac 12 - \frac 9{8x}$.}
\end{enumerate}
}