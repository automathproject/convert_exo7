\exo7id{120}
\titre{Exercice 120}
\theme{}
\auteur{bodin}
\date{1998/09/01}
\organisation{exo7}
\contenu{
  \texte{Soient $f,g$ deux fonctions de $\Rr$ dans $\Rr$. Traduire
en termes de quantificateurs les expressions suivantes :}
\begin{enumerate}
  \item \question{$f$ est major\'ee;}
  \item \question{$f$ est born\'ee;}
  \item \question{$f$ est paire;}
  \item \question{$f$ est impaire;}
  \item \question{$f$ ne s'annule jamais;}
  \item \question{$f$ est p\'eriodique;}
  \item \question{$f$ est croissante;}
  \item \question{$f$ est strictement d\'ecroissante;}
  \item \question{$f$ n'est pas la fonction nulle;}
  \item \question{$f$ n'a jamais les m\^{e}mes valeurs en deux points distincts;}
  \item \question{$f$ atteint toutes les valeurs de $\Nn$;}
  \item \question{$f$ est inf\'erieure \`a $g$;}
  \item \question{$f$ n'est pas inf\'erieure \`a $g$.}
\end{enumerate}
\begin{enumerate}
  \item \reponse{$\exists M \in \Rr \quad \forall x \in \Rr \qquad f(x) \leq M$;}
  \item \reponse{$\exists M \in \Rr\quad \exists m \in \Rr \quad \forall x \in \Rr \qquad m \leq f(x) \leq M$;}
  \item \reponse{$\forall x \in \Rr \qquad f(x) = f(-x)$;}
  \item \reponse{$\forall x \in \Rr \qquad f(-x) = -f(x)$;}
  \item \reponse{$\forall x \in \Rr \qquad f(x) \not= 0$;}
  \item \reponse{$\exists a \in \Rr^* \quad \forall x \in \Rr \qquad f(x+a) = f(x)$;}
  \item \reponse{$\forall (x,y) \in \Rr^2 \qquad (x\leq y \Rightarrow f(x) \leq f(y))$;}
  \item \reponse{$\forall (x,y) \in \Rr^2 \qquad (x < y \Rightarrow f(x) > f(y))$;}
  \item \reponse{$\exists x \in \Rr \qquad  f(x) \not= 0$;}
  \item \reponse{$\forall (x,y) \in \Rr^2 \qquad (x\not= y \Rightarrow f(x) \not= f(y))$;}
  \item \reponse{$\forall n\in \Nn \quad \exists x \in \Rr \qquad f(x)=n$;}
  \item \reponse{$\forall x \in \Rr \qquad f(x) \leq g(x)$;}
  \item \reponse{$\exists x \in \Rr \qquad f(x) > g(x)$.}
\end{enumerate}
}