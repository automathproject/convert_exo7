\uuid{R646}
\exo7id{4878}
\titre{Symétrie-translation}
\theme{Exercices de Michel Quercia, Applications affines}
\auteur{quercia}
\date{2010/03/17}
\organisation{exo7}
\contenu{
  \texte{Soit $f : {\cal E} \to {\cal E}$ affine.
On dit que $f$ est une {\it symétrie-translation\/}
s'il existe une symétrie $s$ et une translation $t$ telles que
$f = s\circ t = t\circ s$.}
\begin{enumerate}
  \item \question{Soient $s$ une symétrie de base $\cal B$ de direction $\vec {\cal F}$,
    et $t$ une translation de vecteur $\vec u$.

    Montrer que $s\circ t = t\circ s \iff \vec u \in \vec {\cal B}$.}
  \item \question{Soit $f$ une symétrie-translation. Montrer que le couple $(s,t)$ tel que
    $f = s\circ t = t\circ s$ est unique.}
  \item \question{Soit $f$ affine quelconque. Montrer que $f$ est une symétrie-translation si
    et seulement si $f\circ f$ est une translation.}
  \item \question{En déduire que le produit d'une symétrie par une translation quelconques
    est une symétrie-translation.}
  \item \question{AN : décomposer l'application $f$ d'expression analytique dans un
    repère ${\cal R} = (O,\vec e_1, \vec e_2, \vec e_3)$ :
    $$\begin{cases}x' = (x-2y-2z+1)/3 \cr y' = (-2x+y-2z+2)/3 \cr z' = (-2x-2y+z-1)/3.\cr\end{cases}$$}
\end{enumerate}
\begin{enumerate}

\end{enumerate}
}