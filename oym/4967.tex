\uuid{4967}
\titre{Distance d'un point à une droite}
\theme{Exercices de Michel Quercia, Géométrie euclidienne en dimension 3}
\auteur{quercia}
\date{2010/03/17}
\organisation{exo7}
\contenu{
  \texte{}
  \question{Dans un \emph{rond} on donne la droite
$D  : \begin{cases} x +2y - z = -3 \cr  x - y +2z = -4\cr\end{cases}$ et $M(x,y,z)$.
Calculer $d(M,D)$.}
  \reponse{$d^2 = \frac{(x+2y-z+3)^2}6 + \frac{(3x+3z+11)^2}9$.}
}