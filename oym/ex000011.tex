\uuid{11}
\titre{Exercice 11}
\theme{}
\auteur{bodin}
\date{1998/09/01}
\organisation{exo7}
\contenu{
  \texte{}
  \question{Calculer le module et l'argument de $u =
\frac{\sqrt{6}-i\sqrt{2}}{2}$ et $v = 1 - i$. En d\'eduire le
module et l'argument de $w = \frac{u}{v}$.}
  \reponse{Nous avons
$$ u = \frac{\sqrt{6}-\sqrt{2}i}{2}
= \sqrt{2}\left( \frac{\sqrt{3}}{2}-\frac{i}{2} \right) =
\sqrt{2}\left( \cos\frac{\pi}{6} -i\sin\frac{\pi}{6} \right)
=\sqrt{2} e^{-i\frac{\pi}{6}}.$$ puis
$$v = 1-i = \sqrt{2}e^{-i\frac{\pi}{4}}.$$
Il ne reste plus qu'\`a calculer le quotient :
$$ \frac{u}{v} = \frac{\sqrt{2}e^{-i\frac{\pi}{6}}}{\sqrt{2}e^{-i\frac{\pi}{4}}}
= e^{-i\frac{\pi}{6}+i\frac{\pi}{4}} = e^{i\frac{\pi}{12}}.$$}
}