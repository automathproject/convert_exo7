\uuid{Sdrl}
\exo7id{3457}
\titre{Racines de l'unité}
\theme{Exercices de Michel Quercia, Calculs de déterminants}
\auteur{quercia}
\date{2010/03/10}
\organisation{exo7}
\contenu{
  \texte{On note $\omega = e^{2i\pi/n}$, $\alpha = e^{i\pi/n}$ et $D$ le déterminant
$n \times n$ : $D = \det\Bigl( \omega^{(k-1)(l-1)} \Bigr)$.}
\begin{enumerate}
  \item \question{Calculer $D^2$.}
  \item \question{Montrer que $D = \prod_{k < \ell}(\omega^\ell - \omega^k)
            = \prod_{k < \ell}\left(\alpha^{k+\ell}\cdot
              2i\sin\frac {\ell-k}n \pi\right)$.}
  \item \question{Exprimer $D$ sous forme trigonométrique.}
\end{enumerate}
\begin{enumerate}
  \item \reponse{$M^2 = \begin{pmatrix}n      &0      &\dots     &0     \cr
                              0      &0      &          &n     \cr
                              \vdots &       &\dots           \cr
                              0      &n      &          &0     \cr \end{pmatrix}
                \Rightarrow  D^2 = \varepsilon_{n-1}n^n$.}
  \item \reponse{$n^{n/2}\exp\left({i\frac \pi4 (n-1)(3n+2)}\right)$.}
\end{enumerate}
}