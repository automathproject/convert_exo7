\uuid{uk4S}
\exo7id{2901}
\titre{Calcul de sommes}
\theme{Exercices de Michel Quercia, Coefficients du binôme}
\auteur{quercia}
\date{2010/03/08}
\organisation{exo7}
\contenu{
  \texte{Soient $n,p \in \N^*$ avec $n \ge p$.}
\begin{enumerate}
  \item \question{V{\'e}rifier que $C_n^kC_k^p = C_n^pC_{n-p}^{k-p}$ pour $p \le k \le n$.}
  \item \question{Calculer $\sum_{k=0}^n \,(-1)^kC_n^kC_k^p$.}
  \item \question{En d{\'e}duire $\sum_{k=0}^n (-1)^kC_n^kk^p = 0$ si $p < n$.}
\end{enumerate}
\begin{enumerate}
  \item \reponse{$0$ si $p < n$, $(-1)^n$ si $p = n$.}
\end{enumerate}
}