\uuid{4653}
\titre{Fonction continue dont la série de Fourier diverge en 0}
\theme{}
\auteur{quercia}
\date{2010/03/14}
\organisation{exo7}
\contenu{
  \texte{}
\begin{enumerate}
  \item \question{Soit $f : \R \to \C$ paire, $2\pi$-périodique, telle que, pour tout 
$x\in [0,\pi]$, $f(x)=\sum_{p=1}^{\infty}\frac{1}{p^2}\sin \Bigl((2^{p^3}+1)\frac{x}{2}\Bigr)$. 
Vérifier que $f$ est définie et continue sur $\R$.}
  \item \question{Soit $A_0 = \frac{1}{\pi} \int_0^{\pi} f(t)\,d t$, pour $n\in \N^*$, $A_n=\frac{2}{\pi} \int_0^{\pi} f(t) \cos (nt)\,d t$.

Pour $\nu \in \N$, on pose
$a_{0,\nu}=\frac12 \int_0^{\pi} \sin \Bigl((2\nu +1)\frac{t}{2}\Bigr)\,d t $ et
$a_{n,\nu}= \int_0^{\pi} \cos (nt)\sin \Bigl((2\nu +1)\frac{t}{2}\Bigr)\,d t $.

Pour $q\in \N$, on note 
$s_{q,\nu}=\sum_{i=0}^q a_{i,\nu}$. Montrer que si $\nu$ est fixé, $s_{n,\nu}\to0$ lorsque $n\to\infty$. Calculer 
explicitement les $a_{n,\nu}$. En déduire que, pour tout $q$, pour tout $\nu$, $s_{q,\nu}>0$, et prouver que
$\mathop{\max}\limits_{q\in \N} (s_{q,\nu})=s_{\nu,\nu}$.}
  \item \question{Montrer qu'il existe $B>0$ tel que, pour tout $\nu \ge 1$, $s_{\nu,\nu}\ge B \ln \nu$.}
  \item \question{Montrer que, pour tout $n\in \N$, $A_n=\frac{2}{\pi}\sum_{p=1}^{\infty} \frac{1}{p^2}a_{n,2^{p^3-1}}$.}
  \item \question{Pour $n\in \N ^*$, on pose $T_n=\frac{\pi}{2} \sum_{k=0}^{n} A_k$. Vérifier que 
$T_n=\sum_{p=1}^{\infty} \frac{1}{p^2} s_{n,2^{p^3-1}}$. Montrer qu'il existe $D>0$ tel que, pour 
tout $p\ge 1$, $T_{2^{p^3-1}}\ge Dp$, et constater que la série de Fourier de $f$ diverge au point $0$.}
\end{enumerate}
\begin{enumerate}

\end{enumerate}
}