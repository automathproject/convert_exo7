\exo7id{5975}
\titre{Exercice 5975}
\theme{}
\auteur{tumpach}
\date{2010/11/11}
\organisation{exo7}
\contenu{
  \texte{Soit $x\in\mathbb{R}^{d}$, $d = 1, 2$ et $r = |x|$. On consid\`ere
$f~: \mathbb{R}^d\rightarrow \mathbb{R}$ donn\'ee par $$f(x) =
h(r) = r^2 (1+r^2)^{-2}.$$}
\begin{enumerate}
  \item \question{Calculer
pour $d = 1$  le r\'earrangement \`a sym\'etrie sph\'erique
d\'ecroissant $f^*$ de $f$.}
  \item \question{M\^eme question pour $d =
2$.}
  \item \question{Calculer $\|f\|_{2}^{2}$ pour $d = 1$ puis $d = 2$.}
\end{enumerate}
\begin{enumerate}
  \item \reponse{La fonction $h$ atteint son maximum en $r = 1$ et $h(1) =
\frac{1}{4}.$ Pour un r\'eel positif $t\leq\frac{1}{4}$ donn\'e,
on cherche \`a r\'esoudre $t = h(r) = r^2 (1+r^2)^{-2}.$ On
obtient deux solutions
$$
\begin{array}{lcl}
r_+ &=& \left(\frac{1-2t}{2t} + \frac{\sqrt{1-4t}}{2t}
\right)^{\frac{1}{2}}\\
r_{-} & = & \left(\frac{1-2t}{2t} - \frac{\sqrt{1-4t}}{2t}
\right)^{\frac{1}{2}}
\end{array}
$$
Ainsi $\mu\left(f > t\right) = \mathcal{V}_{d}\left(r_{+}^d -
r_-^d\right)$. De plus, par d\'efinition, $f^*$ v\'erifie
$\mu\left(f^*>t\right) = \mu\left(f > t\right)$ et
$\mu\left(f^*>t\right) = \mathcal{V}_{d}\, r^{d}$ o\`u $r$ et $t$
sont li\'es par $t = f^*(r)$. Pour $d = 1$, on a donc $r = r_+ -
r_-$ et $t$ est donn\'e par~:
$$
\begin{array}{lcl} r^2 &=& r_+^2 + r_-^2 - 2r_+ r_- = \frac{1-2t}{t} -
2\sqrt{\frac{(1-2t)^2}{4t^2} - \frac{1-4t}{4t^2}}\\
& = & \frac{1 - 4t}{t}.
\end{array}
$$
Il en d\'ecoule que $t = f^*(r) = (4 + r^2)^{-1}.$}
  \item \reponse{Pour $d = 2$, on a
$$
r^2 = r_+^2 - r_-^2 = \frac{\sqrt{1-4t}}{t},
$$
ce qui implique que
$$
t = f^*(r) = r^{-4}\left(\sqrt{4 + r^4} - 2 \right).
$$}
  \item \reponse{Calculons $\|f\|_{2}^{2}$ pour $d = 1$. On a
 $$
 \begin{array}{lcl}
\|f\|_{2}^{2} & = & \|f^{*}\|_{2}^{2}\\
& = & 2\int_{0}^{+\infty} (4 + r^2)^{-2}\,dr \\
& = & \frac{1}{4} \int_{0}^{+\infty}(1 + s^2)^{-2}\,ds\\ & =
&\frac{1}{8} \int_{\mathbb{R}} (1 + |x|^2)^{-2}\,dx =
\frac{\pi}{16},
\end{array}
 $$
 o\`u la derni\`ere \'egalit\'e d\'ecoule de l'exercice \ref{exo:beta} (question 3.) sur la fonction Bêta, car~:
$$
\int_{\mathbb{R}} (1 + |x|^2)^{-2}\,dx = 2\mathcal{V}_1
B\left(\frac{3}{2}, \frac{3}{2}\right) = 4
\frac{\Gamma(3/2)^2}{\Gamma(3)} = 4\frac{(1/2\Gamma(1/2))^2}{2!} =
\frac{\pi}{2}.
$$

Pour $d = 2$, on a
$$
\|f\|_{2}^2 = \int_{\mathbb{R}^2} f(x)^2\,dx =
\int_{r=0}^{+\infty}\int_{\theta=0}^{2\pi} h(r)^2 \,r dr d\theta =
2\pi \int_{0}^{+\infty} r^5 (1+r^2)^{-4}\,dr = \frac{\pi}{3},
$$
o\`u la derni\`ere \'egalit\'e d\'ecoule de l'exercice \ref{exo:beta} sur la fonction Bêta, car~:
$$
\int_{\mathbb{R}^2} (1 + |x|^2)^{-4}\,dx = 4\mathcal{V}_{6}
B\left(4 - \frac{6}{2}, \frac{6}{2} + 1\right) = 4 \mathcal{V}_{6}
B\left(1, 4\right),
$$
et
$$
\int_{\mathbb{R}^2} (1 + |x|^2)^{-4}\,dx = \int_{r =
0}^{+\infty}\int_{\mathcal{S}^5} (1 + r^{2})^{-4}\, r^5 \,dr
d\sigma = \mathcal{A}_{5} \int_{0}^{+\infty} (1 + r^{2})^{-4}\,
r^5 \,dr
$$
d'o\`u~:
$$
\begin{array}{lcl}
\int_{0}^{+\infty} (1 + r^{2})^{-4}\, r^5 \,dr  & =&  4
\frac{\mathcal{V}_{6}}{\mathcal{A}_5} B\left(1, 4\right) =
\frac{4}{6} \frac{\Gamma(1)\Gamma(4)}{\Gamma(5)},
 =
\frac{2}{3}\frac{3!}{4!} = \frac{1}{6}
\end{array}
$$}
\end{enumerate}
}