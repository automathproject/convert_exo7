\uuid{3527}
\titre{Ensi Chimie P 94}
\theme{Exercices de Michel Quercia, Réductions des endomorphismes}
\auteur{quercia}
\date{2010/03/10}
\organisation{exo7}
\contenu{
  \texte{}
  \question{Soit $A=(a_{ij}) \in \mathcal{M}_n(\R)$ telle que $a_{ij} = \frac ij$.
    $A$ est-elle diagonalisable~?}
  \reponse{$\mathrm{rg} A = 1$ donc $\dim\mathrm{Ker} A = n-1$ et $0$ est valeur propre
             d'ordre au moins $n-1$. La somme des valeurs propres est
             $\mathrm{tr} A = n$ donc la drenière valeur propre est $n$ et le
             sous-espace propre associé est de dimension~$1$. Donc $A$
             est diagonalisable.}
}