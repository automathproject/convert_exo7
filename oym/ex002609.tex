\uuid{2609}
\titre{Exercice 2609}
\theme{}
\auteur{delaunay}
\date{2009/05/19}
\organisation{exo7}
\contenu{
  \texte{Soit $a\in\R$ et $A$ la matrice
$$A=\begin{pmatrix}0&1&0 \\  0&a&0 \\  0&a-2&2\end{pmatrix}.$$}
\begin{enumerate}
  \item \question{Pour quelles valeurs de $a$ la matrice $A$ est-elle diagonalisable ? 

Lorsque $A$ est diagonalisable, déterminer une base de vecteurs propres de $A$.}
  \item \question{Soit $E$ l'espace vectoriel des solutions du système $x'=Ax$, où $x$ est une fonction de la variable réelle $t$ à valeur dans $\R^3$.
   \begin{enumerate}}
  \item \question{Lorsque $A$ est diagonalisable, donner une base de $E$ en fonction des vecteurs propres et des valeurs propres de $A$. Ecrire la solution générale du système.}
  \item \question{Lorsque $A$ n'est pas diagonalisable, intégrer directement le système $x'=Ax$.}
\end{enumerate}
\begin{enumerate}
  \item \reponse{{\it Pour quelles valeurs de $a$ la matrice $A$ est-elle diagonalisable ?}

Déterminons le polynôme caractéristique de la matrice $A$.
$$P_A(X)=\det (A-XI)=\begin{vmatrix}-X&1&0 \\  0&a-X&0 \\ 0&a-2&2-X\end{vmatrix}=-X(a-X)(2-X).$$
Ce polynôme admet trois racines $0,a$ et $2$. Ainsi, si $a\notin\{0,2\}$ la matrice est diagonalisable. Examinons les cas $a=0$ et $a=2$.

Si $a=0$, la valeur propre $0$ est valeur propre double, on a
$$A=\begin{pmatrix}0&1&0 \\  0&0&0 \\  0&-2&2\end{pmatrix}.$$
Le sous-espace propre associé à $0$ est égal à $\ker A=\{\vec u=(x,y,z),\ A\vec u=\vec 0\}$,
$$\vec u\in\ker A\iff\left\{\begin{align*}y&=0 \\ -2y+2z&=0\end{align*}\right.\iff y=z=0.$$
Le sous-espace propre associé à la valeur propre double $0$ est une droite vectorielle, la droite engendrée par $(1,0,0)$, la matrice n'est donc pas diagonalisable.

Si $a=2$, la valeur propre $2$ est double, on a
$$A=\begin{pmatrix}0&1&0 \\  0&2&0 \\  0&0&2\end{pmatrix}.$$
Le sous-espace propre associé à $2$ est égal à $E_2=\{\vec u=(x,y,z),\ A\vec u=2\vec u\}$,
$$\vec u\in E_2\iff\left\{\begin{align*}y&=2x \\ 2y&=2y \\ 2z&=2z\end{align*}\right.\iff y=2x.$$
Le sous-espace propre associé à la valeur propre double $2$ est un plan vectoriel, le plan d'équation $y=2x$, la matrice est donc diagonalisable.

Ainsi la matrice $A$ est diagonalisable si et seulement si $a\neq 0$.

{\it Lorsque $A$ est diagonalisable, déterminons une base de vecteurs propres de $A$.}

On a $a\neq 0$ et on distingue les cas $a\neq 2$ et $a=2$.

Si $a\neq 2$, les sous-espaces propres associés aux valeurs propres $0$ et $2$ sont lisibles sur la matrice, on a
$$E_0=\ker A=\R.(1,0,0)\quad{\hbox{et}}\quad E_2=\R.(0,0,1),$$
On détermine $E_a=\{\vec u=(x,y,z),\ A\vec u=a\vec u\}$.
$$\vec u\in E_a\iff\left\{\begin{align*}y&=ax \\  ay&=ay \\  (a-2)y+2z&=az\end{align*}\right.\iff\left\{\begin{align*}y&=ax \\  (a-2)y&=(a-2)z\end{align*}\right.\iff\left\{\begin{align*}y&=ax \\  y&=z\end{align*}\right.$$
C'est la droite vectorielle engendrée par le vecteur $\vec e=(1,a,a)$. Ainsi, une base de vecteurs propres est donnée par les vecteurs $(1,0,0), (0,0,1)$ et $(1,a,a)$.

Si $a=2$, nous avons vu que le sous-espace propre associé à la valeur propre double $2$ est le plan d'équation $y=2x$. 
Ainsi, une base de vecteurs propres est donnée par les vecteurs $(1,0,0), (0,0,1)$ et $(1,2,0)$.}
  \item \reponse{{\it Soit $E$ l'espace vectoriel des solutions du système $x'=Ax$, où $x$ est une fonction de la variable réelle $t$ à valeur dans $\R^3$.}
   \begin{enumerate}}
  \item \reponse{{\it Lorsque $A$ est diagonalisable, donnons une base de $E$ en fonction des vecteurs propres et des valeurs propres de $A$ et écrivons la solution générale du système.}

Si $\lambda_1, \lambda_2, \lambda_3$ sont les valeurs propres de $A$ et $\vec e_1,\vec e_2$ et $\vec e_3$ les vecteurs propres associés, on sait qu'une base de l'espace des solutions du système différentiel $x'=Ax$ est donnée par 
$$e^{\lambda_1 t}\vec e_1,\ e^{\lambda_2 t}\vec e_2,\ e^{\lambda_3 t}\vec e_3.$$
Ainsi, si $a\neq 2$ cette base est donnée par
$$(1,0,0),\ e^{2t}(0,0,1),\ e^{at}(1,a,a)$$
et si $a=2$, elle est donnée par
$$(1,0,0),\ e^{2t}(0,0,1),\ e^{2t}(1,2,0).$$}
  \item \reponse{{\it Lorsque $A$ n'est pas diagonalisable, intégrons directement le système $X'=AX$}.

Lorsque $A$ n'est pas diagonalisable, $a=0$ et 
$$A=\begin{pmatrix}0&1&0 \\  0&0&0 \\  0&-2&2\end{pmatrix}.$$
Le système $X'=AX$ est équivalent à
$$\begin{pmatrix}x'(t) \\  y'(t) \\  z'(t)\end{pmatrix}=\begin{pmatrix}0&1&0 \\  0&0&0 \\  0&-2&2\end{pmatrix}
\begin{pmatrix}x(t) \\  y(t) \\  z(t)\end{pmatrix}\iff
\left\{\begin{align*}x'&=y \\  y'&=0 \\  z'&=-2y+2z\end{align*}\right.$$
Si $y'=0$, alors $y(t)=\alpha$, $\alpha\in\R$. Ainsi, si $x'=\alpha$, $x(t)=\alpha t+\beta$, $\beta\in\R$ et la troisième équation devient
$$z'=2z-2\alpha$$
et sa solution s'écrit $z(t)=\gamma e^{2t}+\alpha$, $\gamma\in\R$. D'où la solution générale du système
$$X(t)=\begin{pmatrix}x(t) \\  y(t) \\  z(t)\end{pmatrix}=\begin{pmatrix}\alpha t+\beta \\  \alpha \\  \gamma e^{2t}+\alpha\end{pmatrix}=\alpha\begin{pmatrix}t \\ 1 \\ 1\end{pmatrix}+\beta\begin{pmatrix}1 \\ 0 \\ 0\end{pmatrix}+\gamma\begin{pmatrix}0 \\ 0 \\  e^{2t}\end{pmatrix}$$
$(\alpha,\beta,\gamma)\in\R^3$.}
\end{enumerate}
}