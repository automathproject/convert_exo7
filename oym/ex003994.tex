\uuid{3994}
\titre{Limite de $f(x) - xf'(x)$}
\theme{}
\auteur{quercia}
\date{2010/03/11}
\organisation{exo7}
\contenu{
  \texte{Soit $f : \R \to \R$ convexe dérivable.}
\begin{enumerate}
  \item \question{Montrer que $p = \lim_{x\to+\infty} (f(x) - xf'(x))$ existe.}
  \item \question{On suppose $p$ fini. En utilisant le fait que $f(x) - xf'(x)$ est
    bornée au voisinage de $+\infty$, montrer que $\frac{f(x)}x$ et $f'(x)$
    admettent une même limite $m$ finie en $+\infty$.}
  \item \question{Montrer alors que $f(x) - mx - p \to 0$ lorsque $x\to {+\infty}$.}
\end{enumerate}
\begin{enumerate}
  \item \reponse{Fonction décroissante sur $\R^+$.}
  \item \reponse{$f(x) - xf'(x) = -x^2\frac {d}{d x}\left(\frac {f(x)}x\right)$.
    Donc, $x \mapsto \frac {f(x) - p}x \searrow$ et
          $x \mapsto \frac {f(x) - f(0)}x \nearrow$.}
  \item \reponse{$p\le f(x) - mx \le f(x) - xf'(x)$.}
\end{enumerate}
}