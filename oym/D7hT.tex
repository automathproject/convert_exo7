\uuid{D7hT}
\exo7id{7276}
\titre{Exercice 7276}
\theme{Exercices de Christophe Mourougane, Géométrie en petites dimensions}
\auteur{mourougane}
\date{2021/08/10}
\organisation{exo7}
\contenu{
  \texte{Étant données deux demi-droites \([OA)\) et \([OB)\), on veut construire 
la bissectrice de l'angle géométrique \(\widehat{AOB}\).}
\begin{enumerate}
  \item \question{Comment construire au compas deux points \(P \in ]OA)\) et \(Q \in ]OB)\) 
de sorte que le triangle \(OPQ\) soit isocèle?}
  \item \question{Montrer que la médiatrice du segment \([PQ]\) est aussi la bissectrice 
de l'angle \(\widehat{POQ}\). (Indication: utiliser une symétrie axiale bien 
choisie.)}
  \item \question{Expliquer comment construire à la règle et au compas la bissectrice de 
l'angle \(\widehat{AOB}\).}
\end{enumerate}
\begin{enumerate}

\end{enumerate}
}