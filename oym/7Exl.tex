\uuid{7Exl}
\exo7id{4637}
\titre{$f(x+\pi)$}
\theme{Exercices de Michel Quercia, Séries de Fourier}
\auteur{quercia}
\date{2010/03/14}
\organisation{exo7}
\contenu{
  \texte{Soit $f:\R \to \R$ $2\pi$-périodique continue par morceaux.
Que peut-on dire des coefficients de Fourier de $f$ si l'on~a~:}
\begin{enumerate}
  \item \question{$\forall\ x \in \R,\ f(x+\pi) = f(x)$ ?}
  \item \question{$\forall\ x \in \R,\ f(x+\pi) = -f(x)$ ?}
\end{enumerate}
\begin{enumerate}
  \item \reponse{$a_{2p+1} = b_{2p+1} = 0$.}
  \item \reponse{$a_{2p} = b_{2p} = 0$.}
\end{enumerate}
}