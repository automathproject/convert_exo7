\uuid{2jt4}
\exo7id{3629}
\titre{Base duale des polynômes}
\theme{Exercices de Michel Quercia, Dualité}
\auteur{quercia}
\date{2010/03/10}
\organisation{exo7}
\contenu{
  \texte{Soit $E =  \R_{2n-1}[X]$, et $x_1,\dots,x_n \in \R$ distincts.
On note :
$${\phi_i} : E\to \R, P \mapsto {P(x_i)} ;
 \qquad
 {\psi_i} : E \to \R, P \mapsto {P'(x_i)}$$}
\begin{enumerate}
  \item \question{Montrer que $(\phi_1,\dots,\phi_n,\psi_1,\dots,\psi_n)$ est une base de
    $E^*$.}
  \item \question{Chercher la base duale.
    On notera $P_i = \prod_{j\ne i}\frac{X-x_j}{x_i-x_j}$ et
    $d_i = P_i'(x_i)$.}
\end{enumerate}
\begin{enumerate}

\end{enumerate}
}