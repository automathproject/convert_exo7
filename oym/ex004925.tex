\uuid{4925}
\titre{\'Etude d'équations}
\theme{}
\auteur{quercia}
\date{2010/03/17}
\organisation{exo7}
\contenu{
  \texte{Déterminer les natures des surfaces d'équation :}
\begin{enumerate}
  \item \question{$x^2 + y^2 + z^2 -2xy + 2xz + 3x - y + z + 1 = 0$.}
  \item \question{$(x-y)(y-z) + (y-z)(z-x) + (z-x)(x-y) + (x-y) = 0$.}
  \item \question{$x^2 + 9y^2 + 4z^2 - 6xy - 12yz + 4zx + 4 = 0$.}
  \item \question{$x^2 - 2y^2 - z^2 + 2xz - 4yz + 3 = 0$.}
  \item \question{$2x^2 + 2y^2 + z^2 + 2xz - 2yz + 4x - 2y - z + 3 = 0$.}
  \item \question{$xy + xz + yz + 1 = 0$.}
  \item \question{$2x^2 + 2y^2 - z^2 + 5xy - yz + xz = 0$.}
  \item \question{$xy + yz = 1$.}
  \item \question{$x^2 + 4y^2 + 5z^2 - 4xy - 2x + 4y = 0$.

On fera le minimum de calculs nécéssaires pour pouvoir conclure.}
\end{enumerate}
\begin{enumerate}
  \item \reponse{vp $=1,1\pm\sqrt2$.
             er : $X^2 + (1+\sqrt2\,)Y^2 + (1-\sqrt2\,)Z^2 + \frac34 = 0  \Rightarrow $
             hyperboloïde à 2 nappes.}
  \item \reponse{vp $=-\frac32,-\frac32,0$.
             er : $-\frac32(X^2+Y^2)+\frac13 = 0  \Rightarrow $
             cylindre de révolution.}
  \item \reponse{vp $= 0,0,14$. er : $14Z^2 + 4 = 0  \Rightarrow  \varnothing$.}
  \item \reponse{vp $=0,-1\pm\sqrt7$.
             er : $-(1+\sqrt7\,)Y^2 + (-1+\sqrt7\,)Z^2 + 3 = 0  \Rightarrow $
             cylindre hyperbolique.}
  \item \reponse{vp $=0,2,3$.
             er : $2Y^2 + 3Z^2 - \frac{8X}{\sqrt6} = 0  \Rightarrow $
             paraboloïde elliptique.}
  \item \reponse{vp $=-\frac12,-\frac12,1$.
             er : $-\frac{X^2}2 - \frac{Y^2}2 + Z^2 + 1 = 0 \Rightarrow $
             hyperboloïde de révolution à une nappe.}
  \item \reponse{vp $=0,\frac92,-\frac32$.
             er : $\frac{9Y^2}2 - \frac{3Z^2}2 = 0  \Rightarrow $
             deux plans sécants.}
  \item \reponse{vp $=0,\pm\frac1{\sqrt2}$.
             er : $Y^2 - Z^2 =  \sqrt2  \Rightarrow $
             cylindre hyperbolique.}
  \item \reponse{vp $=1,1,0$.
             er : $X^2 + Y^2 =  \frac15  \Rightarrow $
             cylindre de révolution.}
\end{enumerate}
}