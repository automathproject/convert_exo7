\uuid{Uwyg}
\exo7id{3856}
\titre{$f(x)=g(x)$}
\theme{Exercices de Michel Quercia, Fonctions continues}
\auteur{quercia}
\date{2010/03/11}
\organisation{exo7}
\contenu{
  \texte{}
\begin{enumerate}
  \item \question{Soit $f : {[0,1]} \to {[0,1]}$ continue. Montrer qu'il existe $x \in {[0,1]}$
tel que $f(x) = x$.}
  \item \question{Soient ${f,g} : {[0,1]} \to {[0,1]}$ continues telles que $f\circ g = g\circ f$.
Montrer qu'il existe $x \in {[0,1]}$
tel que $f(x) = g(x)$ (on pourra s'intéresser aux points fixes de~$f$).}
\end{enumerate}
\begin{enumerate}
  \item \reponse{$f(x)-x$ change de signe entre $0$ et $1$.}
  \item \reponse{Sinon $f-g$ est de signe constant, par exemple positif.
Si $a$ est le plus grand point fixe de~$f$ alors $g(a)>a$ et $g(a)$ est
aussi point fixe de~$f$, absurde.}
\end{enumerate}
}