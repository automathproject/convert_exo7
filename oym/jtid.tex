\uuid{jtid}
\exo7id{3526}
\titre{\'Eléments propres de $C{}^tC$}
\theme{Exercices de Michel Quercia, Réductions des endomorphismes}
\auteur{quercia}
\date{2010/03/10}
\organisation{exo7}
\contenu{
  \texte{Soit $C = \begin{pmatrix} a_1\cr \vdots\cr a_n \end{pmatrix} \in \mathcal{M}_{n,1}(\R)$ et $M = C{}^tC$.}
\begin{enumerate}
  \item \question{Chercher le rang de $M$.}
  \item \question{En déduire le polynôme caractéristique de $M$.}
  \item \question{$M$ est-elle diagonalisable ?}
\end{enumerate}
\begin{enumerate}
  \item \reponse{1 si $C \ne 0$, 0 si $C = 0$.}
  \item \reponse{$\dim(E_0) \ge n-1  \Rightarrow  X^{n-1}$ divise $\chi_M  \Rightarrow 
             \chi_M = (-1)^n(X^n - (a_1^2+\dots+a_n^2)X^{n-1})$.}
  \item \reponse{Oui.}
\end{enumerate}
}