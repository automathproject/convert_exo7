\uuid{3679}
\titre{Coordonnées des vecteurs de Schmidt}
\theme{Exercices de Michel Quercia, Produit scalaire}
\auteur{quercia}
\date{2010/03/11}
\organisation{exo7}
\contenu{
  \texte{}
  \question{Soit $E$ un espace euclidien, ${\cal B} = (\vec u_1,\dots,\vec u_n)$ une base de $E$
et ${\cal B}' = (\vec e_1,\dots,\vec e_n)$ la base orthonormée déduite de
$\cal B$ par la méthode de Schmidt.

On note $G_n$ le déterminant de Gram de $\vec u_1,\dots,\vec u_n$,
et $\Delta_{i,n}$ le cofacteur de $(\vec u_i\mid \vec u_n)$ dans $G_n$.

Montrer que $\vec e_n = \frac1{\sqrt{G_{n-1}G_n}}\sum_{i=1}^n \Delta_{i,n}\vec u_i$.}
  \reponse{Soit $X$ la matrice de $\vec e_n$ dans ${\cal B}$.
         On a $GX = \begin{pmatrix}0\cr\vdots\cr0\cr\lambda\cr\end{pmatrix}$ et ${}^tXGX = \lambda x_p = 1$.
	 On applique alors les formules de Cramer.}
}