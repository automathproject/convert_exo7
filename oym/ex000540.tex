\uuid{540}
\titre{Exercice 540}
\theme{}
\auteur{cousquer}
\date{2003/10/01}
\organisation{exo7}
\contenu{
  \texte{Soit $a\in \mathbb{R}$. On considère la suite $(u_n)$ définie par 
$u_0=a$ et $u_{n+1}=e^{u_n}-2$ pour $n\geq 0$.}
\begin{enumerate}
  \item \question{Étudier cette suite si $a=0$.}
  \item \question{Étudier cette suite si $a=-10$.}
  \item \question{Étudier cette suite si $a=3$.}
  \item \question{Généraliser en discutant selon la valeur de~$a$.}
\end{enumerate}
\begin{enumerate}

\end{enumerate}
}