\exo7id{6189}
\titre{Exercice 6189}
\theme{}
\auteur{queffelec}
\date{2011/10/16}
\organisation{exo7}
\contenu{
  \texte{Voici quelques applications du fait important suivant : dans un espace
mé\-trique compact, toute suite ayant une seule valeur d'adhérence converge.}
\begin{enumerate}
  \item \question{Soit $(a_n)$ une suite bornée de réels, telle que $(e^{ita_n})$ converge pour
un ensemble non dénombrable de $t\in \Rr$; montrer que la suite $(a_n)$
converge.}
  \item \question{Soit $f$ une application de ${\Rr}$ dans ${\Rr}$ et $G$ son graphe.
Montrer que  si $G$ est connexe par arcs, $f$ est continue.}
  \item \question{Soit $f$ une application de $X$ dans $Y$, espaces métriques et $G$ le graphe
de
$f$. Montrer que $G$ est fermé dans $X\times Y$ si $f$ est continue.
Montrer que la réciproque est vraie lorsque
$Y$ est compact.}
  \item \question{Soit $X$ un espace métrique, $Y$ un espace métrique compact et
$f:X\times Y\to \Rr$ une application continue telle que, pour tout $x\in X$,
l'équation $f(x,y)=0$ ait une unique solution $y\in Y$. Montrer que
l'application $u:x\in X\to y\in Y$ ainsi définie est continue.}
\end{enumerate}
\begin{enumerate}

\end{enumerate}
}