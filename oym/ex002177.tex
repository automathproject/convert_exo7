\uuid{2177}
\titre{Exercice 2177}
\theme{}
\auteur{debes}
\date{2008/02/12}
\organisation{exo7}
\contenu{
  \texte{}
  \question{Etant donn\'es un groupe $G$ et un sous-groupe $H$, on d\'efinit le normalisateur $\textrm{Nor}_G(H)$ de $H$ dans $G$ comme l'ensemble des \'el\'ements $g\in G$ tels que $gHg^{-1} = H$. 
\smallskip

(a) Montrer que $\textrm{Nor}_G(H)$ est le plus grand sous-groupe de $G$ contenant $H$ comme sous-groupe distingu\'e. 
\smallskip

(b) Montrer que le nombre de sous-groupes distincts conjugu\'es de $H$ dans $G$ est \'egal \`a l'indice $[G:\textrm{Nor}_G(H)]$ et qu'en particulier c'est un diviseur de l'ordre de $G$.}
  \reponse{}
}