\uuid{4588}
\titre{Ensae MP$^*$ 2000}
\theme{Exercices de Michel Quercia, Séries entières}
\auteur{quercia}
\date{2010/03/14}
\organisation{exo7}
\contenu{
  \texte{}
  \question{Soit $(u_n)$ définie par, pour tout $n\in \N$, $\sum_{k=0}^n\frac{u_{n-k}}{k!\strut}=1$. Trouver la
limite de $(u_n)$.}
  \reponse{$f(t) = \sum_{n=0}^\infty u_nt^n = \frac{e^{-t}}{1-t\strut} = \sum_{n=0}^\infty\sum_{k=0}^n\frac{(-1)^k}{\strut k!}t^n$
donc $u_n\to\frac1{\strut e}$ lorsque $n\to\infty$.}
}