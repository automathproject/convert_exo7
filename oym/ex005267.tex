\uuid{5267}
\titre{***}
\theme{}
\auteur{rouget}
\date{2010/07/04}
\organisation{exo7}
\contenu{
  \texte{}
  \question{Soit $A=(a_{i,j})_{1\leq i,j\leq n}$ $(n\geq2)$ définie par 

$$\forall i\in\{1,...,n\},\;a_{i,j}=
\left\{
\begin{array}{l}
i\;\mbox{si}\;i=j\\
1\;\mbox{si}\;i>j\\
0\;\mbox{si}\;i<j
\end{array}
\right..$$

Montrer que $A$ est inversible et calculer son inverse.}
  \reponse{Soit $\mathcal{B}=(e_i)_{1\leq i\leq n}$ la base canonique de $\Cc^n$ et $(e_i')_{1\leq i\leq n}$ la famille d'éléments de $\Cc^n$ de matrice $A$ dans la base $\mathcal{B}$.

Par définition, on a

$$\forall i\in\{1,...,n-1\},\;e_i'=ie_i+\sum_{j=i+1}^{n}e_j\;\mbox{et}\;e_n'=ne_n.$$

En retranchant membre à membre ces égalités, on obtient

$$\forall i\in\{1,...,n-1\},\;e_i'-e_{i+1}'=i(e_i-e_{i+1})\;\mbox{et}\;e_n'=ne_n,$$

ou encore

$$\forall i\in\{1,...,n-1\},\;e_i-e_{i+1}=\frac{1}{i}(e_i'-e_{i+1}')\;\mbox{et}\;e_n=\frac{1}{n}e_n'.$$

Mais alors, pour $i\in\{1,...,n-1\}$, on a

\begin{align*}\ensuremath
e_i&=\sum_{j=i}^{n-1}(e_j-e_{j+1})+e_n=\sum_{j=i}^{n-1}\frac{1}{j}(e_j'-e_{j+1}')+\frac{1}{n}e_n'
=\sum_{j=i}^{n-1}\frac{1}{j}e_j'-\sum_{j=i+1}^{n}\frac{1}{j-1}e_j'+\frac{1}{n}e_n'\\
 &=\frac{1}{i}e_i'+\sum_{j=i+1}^{n}\frac{1}{j}e_j'-\sum_{j=i+1}^{n}\frac{1}{j-1}e_j'\\
 &=\frac{1}{i}e_i'-\sum_{j=i+1}^{n}\frac{1}{j(j-1)}e_j'
\end{align*}

Mais alors, $\Cc^n=\mbox{Vect}(e_1,...,e_n)\subset\mbox{Vect}(e_1',...,e_n')$, ce qui montre que la famille $\mathcal{B}'=(e_1',...,e_n')$ est génératrice de $\Cc^n$ et donc une base de $\Cc^n$. Par suite, $A$ est inversible et 

$$A^{-1}=\mbox{Mat}_{\mathcal{B}'}\mathcal{B}=(a_{i,j}')_{1\leq i,j\leq n}\;\mbox{où}\;
a_{i,j}'=
\left\{
\begin{array}{l}
\frac{1}{i}\;\mbox{si}\;i=j\\
-\frac{1}{i(i-1)}\;\mbox{si}\;i>j\\
0\;\mbox{si}\;i<j
\end{array}
\right..
$$}
}