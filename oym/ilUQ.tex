\uuid{ilUQ}
\exo7id{3558}
\titre{Ensi PC 1999}
\theme{Exercices de Michel Quercia, Réductions des endomorphismes}
\auteur{quercia}
\date{2010/03/10}
\organisation{exo7}
\contenu{
  \texte{Soit $E$ un ev de dimension finie et $u\in \mathcal{L}(E)$ tel que $u\circ u = 0$.}
\begin{enumerate}
  \item \question{Quelle relation y a-t-il entre $\mathrm{Ker} u$ et $\Im u$~?
    Montrer que $2\mathrm{rg} u \le \dim E$.}
  \item \question{On suppose ici $\dim E = 4$ et $\mathrm{rg} u = 2$.
    Montrer qu'il existe une base $(\vec{e_1},\vec{e_2},\vec{e_3},\vec{e_4})$
    de~$E$ telle que~:
    $u(\vec{e_1}) = \vec{e_2}$, $u(\vec{e_2}) = \vec{0}$,
    $u(\vec{e_3}) = \vec{e_4}$, $u(\vec{e_4}) = \vec{0}$.}
  \item \question{On suppose $\dim E = n$ et $\Im u = \mathrm{Ker} u$.
    Est-ce que $u$ est diagonalisable~?}
\end{enumerate}
\begin{enumerate}

\end{enumerate}
}