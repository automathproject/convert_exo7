\exo7id{3632}
\titre{Base duale de $(1, X, X(X-1), \dots)$}
\theme{}
\auteur{quercia}
\date{2010/03/10}
\organisation{exo7}
\contenu{
  \texte{Soit $E =  K_n[X]$. On note $P_0 = 1$, $P_i = X(X-1)\cdots(X-i+1)$ pour $i\ge1$,
et $f_i : P  \mapsto P(i)$.}
\begin{enumerate}
  \item \question{Montrer que $(P_0,\dots,P_n)$ est une base de $E$ et ${\cal B} = (f_0,\dots,f_n)$
    est une base de $E^*$.}
  \item \question{Décomposer la forme linéaire $P_n^*$ dans la base $\cal B$.
    (On pourra utiliser les polynômes :
    $Q_i = \prod_{1\le j\le n; j\ne i}(X-j)$)}
  \item \question{Décomposer de même les autres formes linéaires $P_k^*$.}
\end{enumerate}
\begin{enumerate}

\end{enumerate}
}