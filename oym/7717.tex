\uuid{7717}
\titre{Exercice 7717}
\theme{Exercices de Christophe Mourougane, Théorie des groupes et géométrie}
\auteur{mourougane}
\date{2021/08/11}
\organisation{exo7}
\contenu{
  \texte{}
  \question{Soit $P$ un plan muni d'une forme quadratique.
Montrer que les conditions suivantes sont équivalentes.
\begin{itemize}
 \item $P$ est un plan hyperbolique.
\item $q$ est non-dégénérée d'indice $1$.
\item le discriminant de $q$ est égal à $-1$, modulo un carré
(et donc dans une base orthonormée, l'expression de $q$ est $q(x)=ax_1^2+bx_2^2$, avec des scalaires $a$ et $b$ tels que $-ab$ est un carré). 
\end{itemize}}
  \reponse{}
}