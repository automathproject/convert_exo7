\uuid{2op6}
\exo7id{2646}
\titre{Exercice 2646}
\theme{Extremums locaux, gradient, fonctions implicites}
\auteur{debievre}
\date{2009/05/19}
\organisation{exo7}
\contenu{
  \texte{}
\begin{enumerate}
  \item \question{D\'eterminer les points stationnaires de 
la fonction $f$ de deux variables d\'efinie par
$f(x,y)= x\hskip 2pt (x+1)^2 - y^2$ et pr\'eciser
la nature de chacun d'eux.}
  \item \question{Tracer la courbe  
constitu\'ee des points tels que $f(x,y)=0$ et $x \geq 0$.
(\emph{Indication}: \'Etudier  la fonction 
$x\mapsto \sqrt{x} \hskip 2pt (x+1)$ pour $x\geq 0$).}
  \item \question{Montrer que 
le point $(-1,0)$ est un point isol\'e de la partie
\[ {\cal C}=\{(x,y); f(x,y)=0\}
\]
du plan, c'est-\`a-dire, le point $(-1,0)$ appartient \`a cette partie
et il existe un nombre r\'eel 
$\varepsilon >0$ tel que 
$D_{\varepsilon} \cap {\cal C} =\{(-1,0)\}$ o\`u
$D_{\varepsilon}$ est le disque ouvert centr\'e en $(-1,0)$ et de rayon $\varepsilon$.}
  \item \question{\'Enoncer le th\'eor\`eme des fonctions implicites.}
  \item \question{Montrer que, quel que soit le point 
$(x_0,y_0)$ de ${\cal C}$ distinct de $(-1,0)$,
au moins une des deux alternatives (i) ou (ii) ci-dessous est v\'erifi\'ee:
\begin{enumerate}}
  \item \question{[(i)]
Il existe une fonction  $h$ de classe $C^1$ de la variable $x$ 
 d\'efinie dans un intervalle ouvert 
appropri\'e telle que $h(x_0)=y_0$ et telle que,
pour qu'au voisinage de $(x_0,y_0)$
les coordonn\'ees $x$ et $y$ du point $(x,y)$
satisfassent \`a l'\'equation
$f(x,y)=0$ il faut et il suffit que
$y=h(x)$.}
  \item \question{[(ii)]
Il existe une fonction  $k$ de classe $C^1$ de la variable $y$ 
 d\'efinie dans un intervalle ouvert 
appropri\'e telle que $h(y_0)=x_0$ et telle que,
pour qu'au voisinage de $(x_0,y_0)$
les coordonn\'ees $x$ et $y$ du point $(x,y)$
satisfassent \`a l'\'equation
$f(x,y)=0$ il faut et il suffit que
$x=k(y)$.}
\end{enumerate}
\begin{enumerate}
  \item \reponse{Puisque $\frac{\partial f}{\partial y} = -2y$ et
\[
\frac{\partial f}{\partial x} =  (x+1)^2 + 2x(x+1)=(x+1)(3x+1)=3x^2+4x+1, 
\] 
les points stationnaires de $f$ sont les points
$(-1,0)$ et $(-1/3,0)$.
En plus,
\[
\mathrm{Hess}_f(x,y)=\left[\begin{matrix} 
6x+4 & 0\\  0 & -2
\end{matrix}\right]
\] 
d'o\`u
$\mathrm{Hess}_f(-1,0)= \left[\begin{matrix} 
-2 &  0\\  0 & -2
\end{matrix}\right]$
et
$\mathrm{Hess}_f(-1/3,0)= \left[\begin{matrix} 
2 &  0\\  0 & -2
\end{matrix}\right]$.
Par cons\'equent la forme hessienne au point $(-1,0)$
est d\'efinie n\'egative et ce point pr\'esente  un maximum local;
de m\^eme, la forme hessienne au point $(-1/3,0)$
est non d\'eg\'en\'er\'ee  et ind\'efinie 
et ce point pr\'esente  un point selle.}
  \item \reponse{La courbe $y= \sqrt{x} \hskip 2pt (x+1)$ pour $x\geq 0$
passe par les points $(0,0)$, $(\tfrac 13,\tfrac 43 \sqrt 3)$, 
$(1,2)$, et $(2,3 \sqrt 2)$;
elle a une tangente verticale \`a l'origine,
le point $(\tfrac 13,\tfrac 43 \sqrt 3)$
est un point d'inflexion, la pente en ce point vaut $\sqrt 3$,
et c'est la pente minimale de la courbe.
Ces faits se d\'eduisent
des expressions
$y'= \tfrac 32 \sqrt x + \tfrac 12 (\sqrt x)^{-1}$ et
$y''= \tfrac 34 x^{-\tfrac 12} - \tfrac 14 x^{-\tfrac 32}$.
La courbe constitu\'ee des points tels que $f(x,y)=0$ et $x \geq 0$
s'obtient par r\'eflexion de la courbe
$y= \sqrt{x} \hskip 2pt (x+1)$ pour $x\geq 0$
par rapport \`a l'axe des $x$.}
  \item \reponse{Dans la boule ouverte 
\[
\{(x,y,z);(x+1)^2+y^2+x^2 <1\} \subseteq \R^3, 
\]
le graphe
$z=f(x,y)$ de la fonction $f$ ne rencontre le plan des $x$ et $y$ qu'au point
$(-1,0)$. Par cons\'equent, l'intersection 
$D \cap \cal C$ du disque
\[
D=\{(x,y); (x+1)^2+y^2<1\}
\]
avec $\cal C$ ne consiste qu'au point $(-1,0)$.}
  \item \reponse{Voir l'indication de l'exercice pr\'ec\'edent.}
  \item \reponse{Quel que soit le point 
$(x_0,y_0)$ de ${\cal C}$
distinct de  $(-1,0)$,
d'apr\`es (1.),
\[
\left(\frac{\partial f}{\partial x}(x_0,y_0),\frac{\partial f}{\partial y}(x_0,y_0) \right) 
\ne (0,0).
\] 
L'assertion est donc une cons\'equence imm\'ediate du th\'eor\`eme des 
fonctions implicites.}
\end{enumerate}
}