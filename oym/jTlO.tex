\uuid{jTlO}
\exo7id{3953}
\titre{Polynômes de Legendre}
\theme{Exercices de Michel Quercia, Dérivation}
\auteur{quercia}
\date{2010/03/11}
\organisation{exo7}
\contenu{
  \texte{On pose $f(t) = (t^2-1)^n$.}
\begin{enumerate}
  \item \question{Montrer que : $\forall\ k \in \{0,\dots,n-1\},\ f^{(k)}(1) = f^{(k)}(-1) = 0$.}
  \item \question{Calculer $f^{(n)}(1)$ et $f^{(n)}(-1)$.}
  \item \question{Montrer que $f^{(n)}$ s'annule au moins $n$ fois dans l'intervalle $]-1,1[$.}
\end{enumerate}
\begin{enumerate}
  \item \reponse{$f^{(n)}(1) = 2^nn!$, $f^{(n)}(-1) = (-2)^nn!$.}
\end{enumerate}
}