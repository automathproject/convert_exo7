\uuid{4825}
\titre{$A\cup B$ ferm{\'e} $ \Rightarrow $ $A\cup B =E$.}
\theme{}
\auteur{quercia}
\date{2010/03/16}
\organisation{exo7}
\contenu{
  \texte{}
  \question{Soit $E$ un evn de dimension sup{\'e}rieure ou {\'e}gale {\`a} $2$
et $A,B$ deux parties de $E$ telles que
$A$ est ouvert non vide, $B$ est fini et $A\cup B$ est ferm{\'e}.
Montrer que $A\cup B = E$.}
  \reponse{$E\setminus B$ est connexe par arcs et contient au moins un point $a\in A$.
Soit $x\in E\setminus B$ et $\varphi : {[0,1]} \to {E\setminus B}$ un arc
continu joignant $a$ {\`a} $x$ dans $E\setminus B$.
Alors $\varphi^{-1}(A) = \varphi^{-1}(A\cup B)$ est non vide, relativement ouvert
et relativement ferm{\'e} dans $[0,1]$, donc c'est $[0,1]$ ce qui prouve
que $x\in A$.}
}