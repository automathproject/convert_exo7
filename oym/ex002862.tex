\uuid{2862}
\titre{Exercice 2862}
\theme{}
\auteur{burnol}
\date{2009/12/15}
\organisation{exo7}
\contenu{
  \texte{}
  \question{Que vaut $\Gamma(\frac12) = \int_0^\infty
\frac{e^{-t}}{\sqrt t}dt$? (faire un changement de variable
$t=\pi u^2$ pour se ramener à la Gaussienne). En considérant
un contour passant par l'axe réel, puis un quart
de cercle, puis l'axe imaginaire, puis un petit quart de
cercle évitant l'origine prouver:
\[ \int_0^\infty
\frac{e^{-t}}{\sqrt t}dt =
\exp(i\frac\pi4)\int_0^\infty\frac{e^{-ix}}{\sqrt x}dx\] et
en déduire les valeurs des intégrales  $\int_0^\infty
\frac{\cos x}{\sqrt x}dx$ et $\int_0^\infty \frac{\sin
x}{\sqrt x}dx$ (qui ne sont que semi-convergentes).
Comparer aux intégrales de Fresnel.}
  \reponse{}
}