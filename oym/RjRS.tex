\uuid{RjRS}
\exo7id{3673}
\titre{Matrice de Gram}
\theme{Exercices de Michel Quercia, Produit scalaire}
\auteur{quercia}
\date{2010/03/11}
\organisation{exo7}
\contenu{
  \texte{Soient $\vec x_1,\dots,\vec x_n$ des vecteurs d'un espace vectoriel euclidien $E$, et $G$ leur matrice
de Gram.}
\begin{enumerate}
  \item \question{Montrer que $\mathrm{rg} G = \mathrm{rg}(\vec x_1,\dots,\vec x_n)$.}
  \item \question{Montrer que $\det G$ est inchangé si on remplace $\vec x_k$ par
    $\vec x_k - \sum_{i\ne k} \lambda_i \vec x_i$.}
  \item \question{Soit $F = \text{vect}(\vec x_1,\dots,\vec x_n)$ et $\vec x \in E$.
    On note $d(\vec x,F) = \min(\|\vec x-\vec y\,\|, \vec y \in F)$.
    \par Montrer que $d(\vec x,F)^2 =
    \frac {\text{Gram}(\vec x_1,\dots,\vec x_n, \vec x)}{\text{Gram}(\vec x_1,\dots,\vec x_n)}$.}
\end{enumerate}
\begin{enumerate}

\end{enumerate}
}