\uuid{5949}
\titre{Exercice 5949}
\theme{Théorème de convergence monotone, dominée et lemme de Fatou}
\auteur{tumpach}
\date{2010/11/11}
\organisation{exo7}
\contenu{
  \texte{}
  \question{Donner un exemple de fonction
$f:\mathbb{R}\rightarrow \mathbb{R}$ qui est int\'{e}grable au
sens de Lebesgue mais pas au sens de Riemann.}
  \reponse{La fonction de Dirichlet restreint
\`a l'intervalle $[a,b]$,
$f(x)=\mathbf{1}_\mathbb{Q}\left|_{[a,b]}(x)\right.,$ est int\'{e}grable
au sens de Lebesgue et son int\'egrale par rapport \`a la mesure
de Lebesgue vaut $0$. Mais elle n'est pas int\'{e}grable au sens
de Riemann: $\underline{S}(f,\tau)=0$ et
$\overline{S}(f,\tau)=b-a\;$ pour toute subdivision $\tau$ de
l'intervalle $[a,b].$}
}