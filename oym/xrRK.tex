\uuid{xrRK}
\exo7id{4222}
\titre{Formule de la moyenne généralisée}
\theme{Exercices de Michel Quercia, Intégrale de Riemann}
\auteur{quercia}
\date{2010/03/12}
\organisation{exo7}
\contenu{
  \texte{Soient ${f,g} : {[a,b]} \to \R$ continues, $f$ positive.}
\begin{enumerate}
  \item \question{Démontrer qu'il existe $c \in {[a,b]}$ tel que
    $ \int_{t=a}^b f(t)g(t)\,d t = g(c) \int_{t=a}^b f(t)\,d t$.}
  \item \question{Si $f$ ne s'annule pas, montrer que $c \in {]a,b[}$.}
  \item \question{Application : Soit $f$ continue au voisinage de $0$. Déterminer
    $\lim_{x\to 0}\frac1{x^2} \int_{t=0}^x tf(t)\,d t$.}
\end{enumerate}
\begin{enumerate}
  \item \reponse{$\frac12f(0)$.}
\end{enumerate}
}