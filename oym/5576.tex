\uuid{5576}
\titre{**}
\theme{Algèbre linéaire I}
\auteur{rouget}
\date{2010/10/16}
\organisation{exo7}
\contenu{
  \texte{}
  \question{\label{ex:rou14}
Soient $F$, $G$ et $H$ trois sous-espaces d'un espace vectoriel $E$ de dimension finie sur $\Kk$.

Montrer que : $\text{dim}(F+G+H)\leqslant\text{dim}F+\text{dim}G+\text{dim}H-\text{dim}(F\cap G)-\text{dim}(G\cap H)-\text{dim}(H\cap F)+\text{dim}(F\cap G\cap H)$.

Trouver un exemple où l'inégalité est stricte.}
  \reponse{\begin{align*}\ensuremath
\text{dim}(F+G+H)&=\text{dim}((F+G)+H)=\text{dim}(F+G)+\text{dim}H -\text{dim}((F+G)\cap H)\\
 &=\text{dim}F+\text{dim}G+\text{dim}H -\text{dim}(F\cap G)-\text{dim}((F+G)\cap H).
\end{align*}

Maintenant , $F\cap H+G\cap H\subset (F+G)\cap H$ (car si $x$ est dans $F\cap H+G\cap H$ il existe $y$ dans $F$ et dans $H$ et $z$ dans $G$ et dans $H$ tel que $x = y+z$ et $x$ est bien dans $F+G$ et aussi dans $H$). Donc

\begin{align*}\ensuremath
\text{dim}((F+G)\cap H)&\geqslant\text{dim}(F\cap H + G\cap H)=\text{dim}(F\cap H)+\text{dim}(G\cap H)-\text{dim}((F\cap H)\cap (G\cap H))\\
 &=\text{dim}(F\cap H)+\text{dim}(G\cap H)-\text{dim}(F\cap G\cap H)
\end{align*}

et finalement

\begin{center}
\shadowbox{
$\text{dim}(F+G+H)\leqslant\text{dim}F+\text{dim}G+\text{dim}H-\text{dim}(F\cap G)-\text{dim}(F\cap H)-\text{dim}(G\cap H)+\text{dim}(F\cap G\cap H)$.
}
\end{center}

Le cas de trois droites vectorielles de $\Rr^2$ deux à deux distinctes fournit un cas d'inégalité stricte.}
}