\exo7id{5117}
\titre{***I Théorème de \textsc{Cantor}}
\theme{}
\auteur{rouget}
\date{2010/06/30}
\organisation{exo7}
\contenu{
  \texte{}
\begin{enumerate}
  \item \question{Montrer qu'il existe une injection de $E$ dans $\mathcal{P}(E)$.}
  \item \question{En considérant la partie $A=\{x\in E/\;x\notin f(x)\}$, montrer qu'il n'existe pas de bijection $f$ de $E$
sur $\mathcal{P}(E)$.}
\end{enumerate}
\begin{enumerate}
  \item \reponse{Il y a l'injection triviale $\begin{array}[t]{cccc}
f~:&E&\rightarrow&\mathcal{P}(E)\\
 &x&\mapsto&\{x\}
\end{array}$.}
  \item \reponse{Soit $f$ une application quelconque de $E$ dans $\mathcal{P}(E)$. Montrons que $f$ ne peut être
surjective.
Soit $A=\{x\in E/\;x\notin f(x)\}$. Montrons que $A$ n'a pas d'antécédent par $f$. Supposons par
l'absurde que $A$ a un antécédent $a$. Dans ce cas, où est $a$~?~

$$a\in A\Rightarrow a\notin f(a)=A,$$
ce qui est absurde et

$$a\notin A\Rightarrow a\in f(a)=A,$$
ce qui est absurde. Finalement, $A$ n'a pas d'antécédent et $f$ n'est pas surjective. On a montré le théorème de
\textsc{Cantor}~:~pour tout ensemble $E$ (vide, fini ou infini), il n'existe pas de bijection de $E$ sur
$\mathcal{P}(E)$.}
\end{enumerate}
}