\uuid{MTph}
\exo7id{7252}
\titre{Diagramme de Voronoï}
\theme{Exercices de Christophe Mourougane, Géométrie en petites dimensions}
\auteur{mourougane}
\date{2021/08/10}
\organisation{exo7}
\contenu{
  \texte{}
\begin{enumerate}
  \item \question{Que dire de la frontière entre deux cellules de Voronoï ?}
  \item \question{Que dire du point commun à trois cellules de Voronoï, appelé 
sommet de Voronoï?}
  \item \question{Que dire du cercle centré en un sommet de Voronoï et passant 
par un germe d'une des trois cellules?}
  \item \question{Ajouter un point au triangle équilatéral de l'exercice précédent, 
et tracer le nouveau diagramme de Voronoï.}
  \item \question{Ajouter un cinquième point très proche d'un sommet de Voronoï 
et tracer le nouveau diagramme de Voronoï. Toutes les cellules 
ont-elles changé ?}
\end{enumerate}
\begin{enumerate}

\end{enumerate}
}