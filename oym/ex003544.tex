\exo7id{3544}
\titre{INT gestion 94}
\theme{}
\auteur{quercia}
\date{2010/03/10}
\organisation{exo7}
\contenu{
  \texte{Soit $A=\left(\begin{smallmatrix}1&1&1&1\cr -1&1&-1&1\cr -1&1&1&-1\cr -1&-1&1&1\cr\end{smallmatrix}\right)$.\par}
\begin{enumerate}
  \item \question{Calculer $\det A$.}
  \item \question{Calculer $(A-xI)(\,{}^t\!A-xI)$ et en déduire $\chi_A(x)$.}
  \item \question{Montrer que $A$ est $\C$-diagonalisable.}
\end{enumerate}
\begin{enumerate}
  \item \reponse{$(A-xI)(\,{}^t\!A-xI) = (x^2-2x+4)I$, $\chi_A(x) = x^2-2x+4$.}
  \item \reponse{$\,{}^t\!A = 2I-A$ donc $(A-xI)((2-x)I-A) = (x^2-2x+4)I$.
             En prenant pour $x$ une des racines du polynôme $x^2-2x+4$, on obtient
             un polynôme scindé à racines simples annulant $A$.}
\end{enumerate}
}