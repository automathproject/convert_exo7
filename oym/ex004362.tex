\uuid{4362}
\titre{Centrale MP 2000}
\theme{}
\auteur{quercia}
\date{2010/03/12}
\organisation{exo7}
\contenu{
  \texte{On considère $f(x)= \int_{t=0}^{+\infty}\frac{d t}{t^x(1+t)}$.}
\begin{enumerate}
  \item \question{Domaine de définition, monotonie, convexité de $f$ (sans dériver $f$).}
  \item \question{Continuité, dérivabilité, calcul de $f^{(k)}(x)$.}
  \item \question{Donner un équivalent de $f(x)$ en $0$ et en $1$.}
  \item \question{Calculer $f(1/n)$ pour $n\in \N$, $n\ge 2$.}
\end{enumerate}
\begin{enumerate}
  \item \reponse{$D_f = {]0,1[}$. $f$ est convexe sur~$]0,1[$ par intégration de
    l'inégalité de convexité pour $x \mapsto t^{-x}$ et $f(x)\to +\infty$ (lorsque $x\to0$ ou $x\to 1$)
    par convergence monotone donc $f$ décroît puis recroît.}
  \item \reponse{En~$0$~: $\frac1{\strut t^x(1+t)} = \frac1{\strut t^{x+1}} - \frac1{\strut t^{x+1}(1+t)}$
    donc $f(x) =   \int_{t=0}^1\frac{d t}{t^x(1+t)} + \frac1{\strut x}
    -  \int_{t=1}^{+\infty} \frac{d t}{\strut t^{x+1}(1+t)} \sim \frac1{\strut x}.$
    
    En~$1$~: $\frac1{\strut t^x(1+t)} = \frac1{\strut t^x} - \frac{t^{1-x}}{\strut 1+t}$
    donc $f(x) = \frac1{\strut 1-x} -  \int_{t=0}^1\frac{t^{1-x}}{\strut 1+t}\,d t
    +  \int_{t=1}^{+\infty}\frac{d t}{t^x(1+t)} \sim \frac1{\strut 1-x}$.}
  \item \reponse{$f(1/n) =_{(t=u^n)}  \int_{u=0}^{+\infty} \frac{nu^{n-1}\,d u}{u(1+u^n)}
    =_{(v=1/u)}  \int_{v=0}^{+\infty} \frac{n\,d v}{1+v^n} = \frac\pi{\strut\sin(\pi/n)}$
    (formule bien connue\dots)}
\end{enumerate}
}