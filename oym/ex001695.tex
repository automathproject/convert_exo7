\uuid{1695}
\titre{Exercice 1695}
\theme{}
\auteur{bodin}
\date{1999/11/01}
\organisation{exo7}
\contenu{
  \texte{Soit $ \rho  $ l'application de $ \R_4[X] $ dans
lui-m\^eme qui \`a un polyn\^ome $ P  $ associe le reste de la
division euclidienne de $ P $ par $ (X^2-1) .$}
\begin{enumerate}
  \item \question{Montrer que $ \rho  $ est lin\'eaire.}
  \item \question{Montrer que $ \rho ^2= \rho .$ En d\'eduire que $
\rho  $ est diagonalisable.}
  \item \question{D\'eterminer (de pr\'ef\'erence
sans calcul) une base de vecteurs propres pour $ \rho  .$}
\end{enumerate}
\begin{enumerate}

\end{enumerate}
}