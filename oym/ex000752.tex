\uuid{752}
\titre{Exercice 752}
\theme{}
\auteur{bodin}
\date{1998/09/01}
\organisation{exo7}
\contenu{
  \texte{}
  \question{Vérifier
$$
\Arcsin x + \Arccos x = \frac{\pi}{2}\qquad\text{ et } \quad
\Arctan x + \Arctan\frac{1}{x} = \text{sgn}(x)\frac{\pi}{2}.
$$}
  \reponse{\ 
\begin{enumerate}
    \item Soit $f$ la fonction définie sur $[-1,1]$ par 
$f(x) = \Arcsin x +\Arccos x$: $f$ est continue sur l'intervalle $[-1,1]$, 
et dérivable sur $]-1,1[$. Pour tout $x\in]-1,1[$, 
$f'(x)= \frac{1}{\sqrt{1-x^2}}+\frac{-1}{\sqrt{1-x^2}}  = 0$. 
Ainsi $f$ est constante sur $]-1,1[$, donc sur $[-1,1]$ (car continue aux extrémités). 
Or $f(0) = \Arcsin 0 +\Arccos 0 = \frac \pi 2$
donc pour tout $x\in[-1,1]$, $f(x) = \frac \pi 2$.

   \item Soit $g(x) = \Arctan x + \Arctan \frac 1x$.
Cette fonction est définie sur $]-\infty,0[$ et sur $]0,+\infty[$ (mais pas en $0$).
On a 
$$g'(x)= \frac{1}{1+x^2} + \frac{-1}{x^2} \cdot \frac{1}{1+\frac{1}{x^2}} = 0,$$
donc $g$ est constante sur chacun de ses intervalles de définition:
$g(x) = c_1$ sur $]-\infty,0[$ et $g(x) = c_2$ sur $]0,+\infty[$.
Sachant $\Arctan 1 = \frac\pi4$, on calcule $g(1)$ et $g(-1)$ on obtient $c_1 = -\frac \pi 2$
et $c_2 = +\frac \pi 2$.
\end{enumerate}}
}