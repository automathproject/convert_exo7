\uuid{3437}
\titre{Décomposition d'un vecteur en dimension 3}
\theme{Exercices de Michel Quercia, Déterminants}
\auteur{quercia}
\date{2010/03/10}
\organisation{exo7}
\contenu{
  \texte{}
  \question{\def \dddd #1#2#3{\det(\vec #1,\vec #2, \vec #3)}\relax
Soient $\vec a, \vec b, \vec c, \vec d$ quatre vecteurs d'un ev $E$ de
dimension 3.
On note : det le déterminant dans une base fixée de~$E$.

Démontrer que : $\d abc \vec d = \dddd abd \vec c + \dddd adc \vec b + \dddd dbc \vec a$.}
  \reponse{Si $(\vec a, \vec b, \vec c)$ est une base, décomposer $\vec d$.
         Si $\vec a = \lambda \vec b + \mu \vec c$, on obtient $\vec 0 = \vec 0$.}
}