\uuid{4859}
\titre{Ensi Physique P 94}
\theme{Exercices de Michel Quercia, Sous-espaces affines}
\auteur{quercia}
\date{2010/03/17}
\organisation{exo7}
\contenu{
  \texte{}
  \question{Soient $I,J,K$ trois points du plan. Montrer l'équivalence entre les trois
    propriétés~:
    
       a) $I$, $J$, $K$ sont alignés.\par
       b) Il existe $M$ tel que $\det(\vec{MI},\vec{MJ}) +  \det(\vec{MJ},\vec{MK}) +  \det(\vec{MK},\vec{MI}) = 0$.\par
       c) Pour tout point $M$, on a $\det(\vec{MI},\vec{MJ}) +  \det(\vec{MJ},\vec{MK}) +  \det(\vec{MK},\vec{MI}) = 0$.\par}
  \reponse{$   \det(\vec{MI},\vec{MJ}) +  \det(\vec{MJ},\vec{MK}) +  \det(\vec{MK},\vec{MI}) = \det(\vec{IJ},\vec{IK})$.}
}