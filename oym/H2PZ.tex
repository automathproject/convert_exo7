\uuid{H2PZ}
\exo7id{5083}
\titre{***}
\theme{Trigonométrie}
\auteur{rouget}
\date{2010/06/30}
\organisation{exo7}
\contenu{
  \texte{}
\begin{enumerate}
  \item \question{Résoudre dans $\Rr$ l'équation $\cos(3x)=\sin(2x)$.}
  \item \question{En déduire les valeurs de $\sin x$ et $\cos x$ pour $x$ élément de
$\left\{\frac{\pi}{10},\frac{\pi}{5},\frac{3\pi}{10}\right\}$.}
\end{enumerate}
\begin{enumerate}
  \item \reponse{\begin{align*}
\cos(3x)=\sin(2x)&\Leftrightarrow\cos(3x)=\cos(\frac{\pi}{2}-2x)\Leftrightarrow(\exists k\in\Zz/\;3x=\frac{\pi}{2}-2x+2k\pi)\;\mbox{ou}\;
(\exists k\in\Zz/\;3x=-\frac{\pi}{2}+2x+2k\pi)\\
 &\Leftrightarrow(\exists k\in\Zz/\;x=\frac{\pi}{10}+\frac{2k\pi}{5})\;\mbox{ou}\;(\exists k\in\Zz/\;x=-\frac{\pi}{2}+2k\pi)
\end{align*}

\begin{center}
\shadowbox{
$\mathcal{S}_{[0,2\pi]}=\left\{\frac{\pi}{10},\frac{\pi}{2},\frac{9\pi}{10},\frac{13\pi}{10},\frac{3\pi}{2},
\frac{17\pi}{10}\right\}.$
}
\end{center}}
  \item \reponse{$\cos(3x)=\Re(e^{3ix})=\Re((\cos x+i\sin x)^3)=\cos^3x-3\cos x\sin^2x=\cos^3x-3\cos
x(1-\cos^2x)=4\cos^3x-3\cos x$.

\begin{center}
\shadowbox{
$\forall x\in\Rr,\;\cos(3x)=4\cos^3x-3\cos x.$
}
\end{center}
Par suite,

\begin{align*}
\cos(3x)=\sin(2x)&\Leftrightarrow4\cos^3x-3\cos x=2\sin x\cos x\Leftrightarrow\cos x(4\cos^2x-3-2\sin x)=0\\
 &\Leftrightarrow\cos x(-4\sin^2x-2\sin x+1)=0\Leftrightarrow(\cos x=0)\;\mbox{ou}\;(4\sin^2x+2\sin x-1=0).
\end{align*}
D'après 1), l'équation $4\sin^2x+2\sin x-1=0$ admet entre autre pour solutions $\frac{\pi}{10}$ et $\frac{13\pi}{10}$
(car, dans chacun des deux cas, $\cos x\neq0$), ou encore, l'équation $4X^2+2X-1=0$ admet pour solutions les deux
nombres \textbf{distincts} $X_1=\sin\frac{\pi}{10}$ et $X_2=\sin\frac{13\pi}{10}$, qui sont donc les deux solutions de
cette équation. Puisque $X_1>0$ et que $X_2<0$, on obtient

$$X_1=\frac{-1+\sqrt{5}}{4}\;\mbox{et}\;X_2=\frac{-1-\sqrt{5}}{4}.$$
Donc, (puisque $\sin\frac{13\pi}{10}=-\sin\frac{3\pi}{10}$),

\begin{center}
\shadowbox{
$\sin\frac{\pi}{10}=\frac{-1+\sqrt{5}}{4}\;\mbox{et}\;\sin\frac{3\pi}{10}=\frac{1+\sqrt{5}}{4}.$
}
\end{center}
Ensuite, $\sin\frac{3\pi}{10}=\cos\left(\frac{\pi}{2}-\frac{3\pi}{10}\right)=\cos\frac{\pi}{5}$, et donc

\begin{center}
\shadowbox{
$\cos\frac{\pi}{5}=\frac{1+\sqrt{5}}{4}.$
}
\end{center}
Puis

\begin{center}
\shadowbox{$\cos\frac{\pi}{10}=\sqrt{1-\sin^2\frac{\pi}{10}}=\frac{1}{4}\sqrt{10+2\sqrt{5}}$}
\end{center}
et de même 

\begin{center}
\shadowbox{$\sin\frac{\pi}{5}=\frac{1}{4}\sqrt{10-2\sqrt{5}}=\cos\frac{3\pi}{10}$}.
\end{center}}
\end{enumerate}
}