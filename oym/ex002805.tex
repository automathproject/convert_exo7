\uuid{2805}
\titre{Exercice 2805}
\theme{}
\auteur{burnol}
\date{2009/12/15}
\organisation{exo7}
\contenu{
  \texte{Les fonctions de Bessel sont très importantes en
     Analyse. Elles apparaissent très souvent dans des
     problèmes de physique mathématique. L'analyse complexe
     permet d'étudier de manière approfondie ces
     fonctions. Ici nous nous contentons des tout débuts de
     la théorie. Nous ne considérons que les
     fonctions\footnote{dites ``fonctions de Bessel
     de
     première espèce (et d'indices entiers)''.} $J_0$, $J_1$,
     $J_2$, \dots, qui sont définies par les formules:
\footnote{Autrement dit:
\[ J_\nu(z) = \frac{z^{\nu}}{2.4.\dots.(2\nu)}\left(1 -
     \frac{z^2}{2.(2\nu+2)} 
     + \frac{z^4}{2.4.(2\nu+2).(2\nu+4)}
     - \frac{z^6}{2.4.6.(2\nu+2).(2\nu+4).(2\nu+6)} +\dots\right)\]
Remarquez que seule la constante
     $2.4.\dots.(2\nu) = 2^\nu \nu!$ nous restreint (pour le
     moment) à des valeurs entières de $\nu$. Si on en fait
     abstraction on obtient
     avec $\nu = -\frac12$ la fonction ``multiforme''
     $z^{-1/2}\cos(z)$ ; tandis qu'avec
     $\nu=+\frac12$ on obtient $z^{-1/2}\sin(z)$. Les
     définitions exactes sont $J_{-1/2}(z) =
     \sqrt{\frac2{\pi z}}\cos(z)$ et $J_{1/2}(z) =
     \sqrt{\frac2{\pi z}}\sin(z)$.}
\[ \nu\in\Nn,z\in\Cc\qquad J_\nu(z) = \sum_{n=0}^\infty 
     (-1)^n\frac{(\frac z2)^{2n+\nu}}{n!(n+\nu)!}\]
\medskip}
\begin{enumerate}
  \item \question{Montrer que le rayon de convergence de la série
     définissant $J_\nu$ est $+\infty$.}
  \item \question{En dérivant terme à terme prouver les formules:
\begin{align*}
 (z^\nu J_\nu)' &= z^\nu J_{\nu-1}\qquad\quad\ (\nu\geq1)\\
 (z^{-\nu} J_\nu)' &= - z^{-\nu} J_{\nu+1}\qquad(\nu\geq0)
\end{align*}
En particulier on a $(zJ_1)' = zJ_0$ et $J_0^{\,'} = - J_1$.}
  \item \question{Réécrire les équations précédentes sous la forme 
$(z\frac d{dz} + \nu) J_\nu = z J_{\nu-1}$
($\nu\geq1$) et $(z\frac d{dz} - \nu) J_\nu = - z
     J_{\nu+1}$ ($\nu\geq0$)
et en déduire $-(\frac d{dz} + \frac{\nu + 1}z)(\frac d{dz}
     - \frac\nu z)J_\nu = J_\nu$, puis, après simplification,
     l'équation différentielle de Bessel:
\[ z^2 J_\nu'' + zJ_\nu' + (z^2 - \nu^2)J_\nu = 0\]}
  \item \question{Montrer, pour tout $\nu\in\Nn$, que la série
     entière définissant $J_\nu$ est la seule (à une
     constante multiplicative près) qui donne une solution
     de l'équation différentielle de Bessel.\footnote{les
     autres solutions de l'équation différentielle sont
     singulières en $z=0$, avec une composante logarithmique
     ($\nu\in\Zz$). Pour $\nu\notin\Zz$ il y a une solution
     en $z^\nu(\sum_{k\geq0} c_k z^k)$ et une autre en
     $z^{-\nu}(\sum_{k\geq0} d_k z^k)$.}}
\end{enumerate}
\begin{enumerate}

\end{enumerate}
}