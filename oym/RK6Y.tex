\uuid{RK6Y}
\exo7id{4022}
\titre{Exposants variables}
\theme{Exercices de Michel Quercia, Calculs de limites par développements limités}
\auteur{quercia}
\date{2010/03/11}
\organisation{exo7}
\contenu{
  \texte{}
\begin{enumerate}
  \item \question{$x^{\arcsin x}                                             \to 1$ lorsque  $x\to 0^+$.}
  \item \question{$\frac{(\sin x)^{\sin x}-1}{x^x-1}                         \to 1$ lorsque  $x\to 0^+$.}
  \item \question{$(2-x)^{\tan(\pi x/2)}                                     \to e^{2/\pi}$ lorsque  $x\to 1$.}
  \item \question{$(2-x)^{\tan(\pi x/2)}                                     \to -> 1$ lorsque  $x\to 2^-$.}
  \item \question{$(\sin x + \cos x)^{1/x}                                   \to e$ lorsque  $x\to 0$.}
  \item \question{$(\cos 2x - 2\sin x)^{1/x}                                 \to e^{-2}$. lorsque  $x\to 0$}
  \item \question{$(\sin x)^{\tan x}                                         \to 1$. lorsque  $x\to \pi/2$}
  \item \question{$(\tan x)^{\cos x/\cos 2x}                                 \to e^{-1/\sqrt2}$ lorsque  $x\to \pi/4$.}
  \item \question{$(\tan x)^{\cos x/\cos 2x}                                 \to 1$ lorsque  $x\to (\pi/2)^-$.}
  \item \question{$(\sin x)^{1/\ln x}                                        \to e$ lorsque  $x\to 0^+$.}
  \item \question{$(\ln x)^{x-1}                                             \to 1$ lorsque  $x\to 1^+$.}
  \item \question{$(\ln x)^{\ln(e-x)}                                        \to -> 1$ lorsque  $x\to e^-$.}
\end{enumerate}
\begin{enumerate}

\end{enumerate}
}