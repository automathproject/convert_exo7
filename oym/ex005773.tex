\uuid{5773}
\titre{*** I}
\theme{}
\auteur{rouget}
\date{2010/10/16}
\organisation{exo7}
\contenu{
  \texte{Soit $E=\Rr[X]$. Pour $(P,Q)\in E^2$, on pose $\varphi(P,Q)=\int_{0}^{+\infty}P(t)Q(t)e^{-t}\;dt$. Pour $n\in\Nn$, on pose $h_n=(X^ne^{-X})^{(n)}e^X$.}
\begin{enumerate}
  \item \question{Montrer que $\varphi$ est un produit scalaire sur $E$.}
  \item \question{\begin{enumerate}}
  \item \question{Pour $n\in\Nn$, préciser les coefficients de $h_n$. Montrer que la famille $(h_n)_{n\in\Nn}$ est une base de $E$.}
  \item \question{Montrer que la famille $(h_n)_{n\in\Nn}$ est une base orthogonale de l'espace préhilbertien $(E,\varphi)$.}
  \item \question{Pour $n\in\Nn$, déterminer $\|h_n\|$. En déduire une base orthonormée de l'espace préhilbertien $(E,\varphi)$.}
\end{enumerate}
\begin{enumerate}
  \item \reponse{\textbullet~Soient $P$ et $Q$ deux polynômes. La fonction $t\mapsto P(t)Q(t)e^{-t}$ est continue sur $[0,+\infty[$ et est négligeable en $+\infty$ devant $\frac{1}{t^2}$ d'après un théorème de croissances comparées. Donc la fonction $t\mapsto P(t)Q(t)e^{-t}$ est intégrable sur $[0,+\infty[$ et $\varphi(P,Q)$ existe dans $\Rr$.

\textbullet~La symétrie, la bilinéarité et la positivité de l'application $\varphi$ sont claires. De plus, pour $P\in E$,

\begin{align*}\ensuremath
\varphi(P,P)=0&\Rightarrow\int_{0}^{+\infty}P^2(t)e^{-t}\;dt=0\\
 &\Rightarrow\forall t\in[0,+\infty[,\;P^2(t)e^{-t}=0\;(\text{fonction continue positive d'intégrale nulle})\\
 &\Rightarrow\forall t\in[0,+\infty[,\;P(t)=0\Rightarrow P=0\;(\text{polynôme ayant une infinité de racines}).
\end{align*}

Ainsi, la forme $\varphi$ est définie et finalement

\begin{center}
\shadowbox{
l'application $\varphi$ est un produit scalaire sur $E$.
}
\end{center}}
  \item \reponse{\begin{enumerate}}
  \item \reponse{Soit $n\in\Nn$. La formule de \textsc{Leibniz} permet d'écrire

\begin{center}
$(X^ne^{-X})^{(n)}e^X=\left(\sum_{k=0}^{n}\dbinom{n}{k}(X^n)^{(n-k)}(e^{-X})^{(k)}\right)e^X=\sum_{k=0}^{n}(-1)^{k}\dbinom{n}{k}\frac{n!}{k!}X^k$.
\end{center}

En particulier, $\forall n\in\Nn$, $\text{deg}(h_n)=n$ (et $\text{dom}(h_n)=(-1)^n$) et on sait que

\begin{center}
\shadowbox{
la famille $(h_n)_{n\in\Nn}$ est une base de $\Rr[X]$.
}
\end{center}}
  \item \reponse{Soient $P\in E$ et $n\in\Nn^*$. Soit $A>0$. Les deux fonctions $t\mapsto(t^ne^{-t})^{(n-1)}$ et $P$ sont de classe $C^1$ sur le segment $[0,A]$. On peut donc effectuer une intégration par parties et on obtient

\begin{center}
$\int_{0}^{A}P(t)h_n(t)e^{-t}\;dt=\int_{0}^{A}P(t)(t^ne^{-t})^{(n)}\;dt=\left[P(t)(t^ne^{-t})^{(n-1)}\right]_0^A-\int_{0}^{A}P'(t)(t^ne^{-t})^{(n-1)}\;dt$
\end{center}

Maintenant, $(t^ne^{-t})^{(n-1)}$ peut s'écrire $Q(t)e^{-t}$ où $Q$ est un polynôme et donc $P(t)(t^ne^{-t})^{(n-1)}(t)$ tend vers $0$ quand $t$ tend vers $+\infty$ d'après un théorème de croissances comparées. D'autre part, la formule de \textsc{Leibniz} montre que le polynôme $Q$ a
une valuation au moins égale à $1$. On en déduit que la fonction $t\mapsto P(t)(t^ne^{-t})^{(n-1)}(t)$ s'annule en $0$. En faisant tendre $A$ vers $+\infty$, on obtient

\begin{center}
$\int_{0}^{+\infty}P(t)h_n(t)e^{-t}\;dt=-\int_{0}^{+\infty}P'(t)(t^ne^{-t})^{(n-1)}\;dt$.
\end{center}

De manière générale, pour $0\leqslant k\leqslant n$, les remarques précédentes s'appliquent à la fonction $t\mapsto P^{(k)}(t)(t^ne^{-t})^{(n-k)}$ et par récurrence on obtient

\begin{center}
$\forall k\in\llbracket0,n\rrbracket$, $\int_{0}^{+\infty}P(t)h_n(t)e^{-t}\;dt=(-1)^k\int_{0}^{+\infty}P^{(k)}(t)(t^ne^{-t})^{(n-k)}\;dt$.
\end{center}

En particulier, pour $k=n$ on obtient $\int_{0}^{+\infty}P(t)h_n(t)e^{-t}\;dt=(-1)^n\int_{0}^{+\infty}P^{(n)}(t)t^ne^{-t}\;dt$. Cette égalité reste vraie quand $n=0$ et on a montré que

\begin{center}
$\forall P\in\Rr[X]$, $\forall n\in\Nn$, $\varphi(P,h_n)=\int_{0}^{+\infty}P(t)h_n(t)e^{-t}\;dt=(-1)^n\int_{0}^{+\infty}P^{(n)}(t)t^ne^{-t}\;dt$.
\end{center}

En particulier, si $n\in\Nn^*$ et $\text{deg}(P)<n$, on a $P^{(n)}=0$ et donc $\varphi(P,h_n)=0$. Ainsi, $\forall n\in\Nn^*$, $h_n\in(\Rr_{n-1}[X])^\bot$. Puisque $\forall n\in\Nn$, $\text{deg}(h_n)=n$, on en déduit en particulier que $\forall n\in\Nn^*$, $\forall k\in\llbracket0,n-1\rrbracket$, $\varphi(hn,h_k)=0$ et on a montré que

\begin{center}
\shadowbox{
la famille $(h_n)_{n\in\Nn}$ est une base orthogonale de l'espace préhilbertien $(\Rr[X],\varphi)$.
}
\end{center}}
  \item \reponse{Soit $n\in\Nn$. Puisque $\text{deg}(h_n)=n$ et $\text{dom}(h_n)=(-1)^n$, on a $h_n^{(n)}=(-1)^nn!$. La question précédente fournit alors

\begin{center}
$\|h_n\|^2=(-1)^n\int_{0}^{+\infty}h_n^{(n)}(t)t^ne^{-t}\;dt=n!\int_{0}^{+\infty}t^ne^{-t}\;dt=n!\Gamma(n+1)=n!^2$,
\end{center}

et donc $\|h_n\|=n!$. Par suite,

\begin{center}
\shadowbox{
la famille $\left(\frac{1}{n!}h_n\right)_{n\in\Nn}$ est une base orthonormale de l'espace préhilbertien $(\Rr[X],\varphi)$.
}
\end{center}}
\end{enumerate}
}