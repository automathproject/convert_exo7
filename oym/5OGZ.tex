\uuid{5OGZ}
\exo7id{3892}
\titre{Simplification de $a\ch x + b\sh x$}
\theme{Exercices de Michel Quercia, Fonctions usuelles}
\auteur{quercia}
\date{2010/03/11}
\organisation{exo7}
\contenu{
  \texte{Soient $a,b \in \R$ non tous deux nuls.}
\begin{enumerate}
  \item \question{Peut-on trouver $A,\varphi \in \R$ tels que :
    $\forall\ x \in \R,\ a\ch(x)+b\sh(x) = A\ch(x+\varphi)$ ?}
  \item \question{Peut-on trouver $A,\varphi \in \R$ tels que :
    $\forall\ x \in \R,\ a\ch(x)+b\sh(x) = A\sh(x+\varphi)$ ?}
\end{enumerate}
\begin{enumerate}
  \item \reponse{Oui ssi $|a| > |b|$.}
  \item \reponse{Oui ssi $|a| < |b|$.}
\end{enumerate}
}