\uuid{926}
\titre{Exercice 926}
\theme{}
\auteur{gourio}
\date{2001/09/01}
\organisation{exo7}
\contenu{
  \texte{}
  \question{Soit $$E=\big\{(u_{n})_{n\in
  \Nn}\in \Rr^{\Nn}\ |\ (u_{n})_{n} \text{ converge }\big\}.$$
Montrer que
l'ensemble des suites constantes et l'ensemble des suites convergeant
vers $0$ sont des sous-espaces suppl\'{e}mentaires dans $E.$}
  \reponse{On note $F$ l'espace vectoriel des suites constantes et $G$ l'espace
  vectoriel des suites convergeant vers $0$.
\begin{enumerate}
\item $F\cap G = \{0\}$. En effet une suite constante qui converge
  vers $0$ est la suite nulle.
  \item $F+G=E$. Soit $(u_n)$ un \'el\'ement de $E$. Notons $\ell$ la
    limite de $(u_n)$.  Soit $(v_n)$ la suite d\'efinie par
    $v_n=u_n-\ell$, alors $(v_n)$ converge vers $0$. Donc $(v_n)\in
    G$.  Notons $(w_n)$ la suite constante \'egale \`a $\ell$. Alors nous
    avons $u_n=\ell+u_n-\ell$, ou encore $u_n=w_n+v_n$, ceci pour tout
    $n\in\Nn$. En terme de suite cela donne $(u_n)=(w_n)+(v_n)$. Ce
    qui donne la d\'ecomposition cherch\'ee.
\end{enumerate}
Bilan : $F$ et $G$ sont en somme directe dans $E$ : $E=F\oplus G$.}
}