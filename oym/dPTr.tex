\uuid{dPTr}
\exo7id{3788}
\titre{Quotients de Rayleigh}
\theme{Exercices de Michel Quercia, Endomorphismes auto-adjoints}
\auteur{quercia}
\date{2010/03/11}
\organisation{exo7}
\contenu{
  \texte{Soit $f \in \mathcal{L}(E)$ auto-adjoint et
$\lambda_1 \le \lambda_2 \le \dots \le \lambda_n$ ses valeurs propres.}
\begin{enumerate}
  \item \question{Montrer que : $\forall\ \vec x \in E,\
                       \lambda_1 \|\vec x\,\|^2
                   \le (f(\vec x)\mid\vec x)
                   \le \lambda_n \|\vec x\,\|^2$.}
  \item \question{Montrer que si l'une de ces deux inégalités est une égalité pour un vecteur
    $\vec x\ne\vec 0$, alors $\vec x$ est vecteur propre de~$f$.}
  \item \question{Soit $(\vec e_1,\dots,\vec e_n)$ une base orthonormée de $E$ telle que
    pour tout $i$ : $(f(\vec e_i)\mid\vec e_i) = \lambda_i$.

    Montrer que : $\forall\ i,\ f(\vec e_i) = \lambda_i\vec e_i$.}
\end{enumerate}
\begin{enumerate}

\end{enumerate}
}