\exo7id{959}
\titre{Exercice 959}
\theme{}
\auteur{legall}
\date{1998/09/01}
\organisation{exo7}
\contenu{
  \texte{Soit $E = \Rr_n[X]$ et soient $A$ et $B$ deux polyn\^omes \`a coefficients réels de
degr\'e $n+1$. On consid\`ere l'application $f$ qui \`a tout polyn\^ome $P$ de $E$, associe
le reste de la division euclidienne de $AP$ par $B$.}
\begin{enumerate}
  \item \question{Montrer que $f$ est un endomorphisme de $E$.}
  \item \question{Montrer l'\'equivalence
$$
f \hbox{ est bijective} \Longleftrightarrow \hbox{$A$ et $B$ sont premiers entre eux}.
$$}
\end{enumerate}
\begin{enumerate}
  \item \reponse{Soit $P \in E$ et $\lambda \in \Rr$, alors la division euclidienne de $AP$ par $B$ s'\'ecrit $AP= Q\cdot B + R$, donc en multipliant par $\lambda$ on obtient :
$A\cdot (\lambda P)= (\lambda Q)B+\lambda R$.
ce qui est la division euclidienne de $A\cdot (\lambda P)$ par $B$, 
donc si $f(P)= R$ alors $f(\lambda P)=\lambda R$. Donc $f(\lambda P)=\lambda f(P)$.

Soient $P, P' \in E$. On \'ecrit les divisions euclidiennes :
$$AP= Q\cdot B + R,\quad  AP'=Q'\cdot B+R'.$$
En additionnant :
$$A(P+P')=(Q+Q')B+(R+R')$$
qui est la division euclidienne de $A(P+P')$ par $B$.
Donc si $f(P)=R$, $f(P')=R'$ alors $f(P+P')=R+R'=f(P)+f(P')$.

Donc $f$ est lin\'eaire.}
  \item \reponse{Sens $\Rightarrow$. Supposons $f$ est bijective, donc en particulier $f$ est surjective,
    en particulier il existe $P\in E$ tel que $f(P)=1$ ($1$ est le polyn\^ome constant \'egale \`a $1$). La division euclidienne est donc $AP=BQ+1$, autrement dit 
$AP-BQ=1$. Par le th\'eor\`eme de B\'ezout, $A$ et $B$ sont premiers entre eux.}
  \item \reponse{Sens $\Leftarrow$. Supposons $A, B$ premiers entre eux. Montrons que $f$ est injective.
Soit $P\in E$ tel que $f(P)=0$. Donc la division euclidienne s'\'ecrit : $AP=BQ+0$.
Donc $B$ divise $AP$. Comme $A$ et $B$ sont premiers entre eux, par le lemme de Gauss,
alors $B$ divise $P$. Or $B$ est de degr\'e $n+1$ et $P$ de degr\'e moins que $n$, 
donc la seule solution est $P=0$. Donc $f$ est injective. 
Comme $f : E\longrightarrow E$ est injective et $E$ est de dimension finie, alors $f$ est bijective.}
\end{enumerate}
}