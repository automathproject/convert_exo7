\uuid{yMrA}
\exo7id{3140}
\titre{Nombres premiers congrus {\`a} 1 modulo 4}
\theme{Exercices de Michel Quercia, Factorisation en nombres premiers}
\auteur{quercia}
\date{2010/03/08}
\organisation{exo7}
\contenu{
  \texte{On rappelle que si $p$ est premier et $n\wedge p = 1$, alors
$n^{p-1} \equiv 1(\mathrm{mod}\, p)$.}
\begin{enumerate}
  \item \question{Soit $n\in \N$ et $p\ge 3$ un diviseur premier de $n^2+1$.
    Montrer que $p\equiv 1(\mathrm{mod}\, 4)$.}
  \item \question{En d{\'e}duire qu'il y a une infinit{\'e} de nombres premiers de la forme $4k+1$.}
\end{enumerate}
\begin{enumerate}
  \item \reponse{$(-1)^{(p-1)/2}\equiv 1(\mathrm{mod}\, p)$.}
\end{enumerate}
}