\uuid{li8q}
\exo7id{2654}
\titre{Exercice 2654}
\theme{Sujets d'examen}
\auteur{debievre}
\date{2009/05/19}
\organisation{exo7}
\contenu{
  \texte{On consid\`ere la fonction 
$$f(x,y)=(1+2\cos^2(\pi x))(1-\exp(-y^2))+\sin(\pi x).$$
Son graphe est reproduit dans la figure ci-dessous.

\centerline{\includegraphics[height=8cm, keepaspectratio]{../images/img002654-1}}}
\begin{enumerate}
  \item \question{Trouver tous les points critiques de $f$ et d\'eterminer leur nature. Vos r\'esultats sont-ils compatibles avec le graphe de la fonction, reproduit ci-dessus?}
  \item \question{D\'eterminer l'\'equation du plan tangent au graphe de $f$ au point de coordonn\'ees $(1, 1, f(1,1))$. Tracer la droite d'intersection de ce plan avec le plan $xOy$.}
\end{enumerate}
\begin{enumerate}

\end{enumerate}
}