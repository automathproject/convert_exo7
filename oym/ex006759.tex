\exo7id{6759}
\titre{Exercice 6759}
\theme{}
\auteur{queffelec}
\date{2011/10/16}
\organisation{exo7}
\contenu{
  \texte{\label{gijsexoav}
On considère la série $\displaystyle \sum_{n\in \Zz}
\frac{1}{(z-n)^2}$.}
\begin{enumerate}
  \item \question{Montrer que cette série converge normalement sur tout compact $K$
de $\Cc$.
En déduire que $\displaystyle f(z) = \sum_{n\in \Zz}
\frac{1}{(z-n)^2}$ est une fonction méromorphe sur $\Cc$. Vérifier que
l'on a $f(z+1) = f(z)$.}
  \item \question{Déterminer le résidu de $f$ en chacun de ses p\^oles. Montrer
que, si l'on note $z=x+iy$, $f(z)$ tend vers 0, uniformément par rapport à
$x$, lorsque $|y|$ tend vers $\infty$.}
  \item \question{Montrer que $\displaystyle f(z) = \left(\frac{\pi}{\sin(\pi z)}
\right)^2$. (On pourra utiliser le théorème de Liouville.) En déduire que
l'on a $\displaystyle \sum_{n=1}^\infty \frac1{n^2} = \frac{\pi^2}{6}$.}
\end{enumerate}
\begin{enumerate}

\end{enumerate}
}