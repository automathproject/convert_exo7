\uuid{0ENB}
\exo7id{3641}
\titre{trace sur $\mathcal{M}_n(K)$}
\theme{Exercices de Michel Quercia, Dualité}
\auteur{quercia}
\date{2010/03/10}
\organisation{exo7}
\contenu{
  \texte{Soit $E = \mathcal{M}_n(K)$. Pour $A \in \mathcal{M}_n(K)$, on note
$ {\phi_A} : E \to K, M \mapsto {\mathrm{tr}(AM).}$}
\begin{enumerate}
  \item \question{Montrer que $E^* = \{ \phi_A \text{ tq } A \in E\}$.}
  \item \question{On note $\cal S$ l'ensemble des matrices symétriques
         et $\cal A$ l'ensemble des matrices antisymétriques.
    Montrer que ${\cal S}^\circ = \{ \phi_A \text{ tq } A \in {\cal A}\}$
     et ${\cal A}^\circ = \{ \phi_A \text{ tq } A \in {\cal S}\}$.}
\end{enumerate}
\begin{enumerate}

\end{enumerate}
}