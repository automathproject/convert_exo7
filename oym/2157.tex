\uuid{2157}
\titre{Exercice 2157}
\theme{Morphisme, sous-groupe distingué, quotient}
\auteur{debes}
\date{2008/02/12}
\organisation{exo7}
\contenu{
  \texte{}
  \question{Soit $G$ un sous-groupe d'indice fini du groupe multiplicatif $\C ^\times $. Montrer que $G=
\C ^\times$.}
  \reponse{Soit $z\in \C$ quelconque et $\zeta\in \C$ une racine $n$-i\`eme de $z$. Le sous-groupe $G$
est distingu\'e dans $\C$ (puisque $\C$ est commutatif). Si $n$ est l'indice de $G$ dans
$\C$, on a donc $\zeta^n = z\in G$ (voir l'exercice \ref{ex:le18}). D'o\`u $\C\subset G$. L'inclusion
inverse est triviale.}
}