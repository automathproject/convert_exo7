\uuid{4098}
\titre{$y'' + \left(1+\frac\lambda{t^2}\right)y = 0$}
\theme{Exercices de Michel Quercia, \'Equations différentielles linéaires (II)}
\auteur{quercia}
\date{2010/03/11}
\organisation{exo7}
\contenu{
  \texte{}
  \question{Soit $\lambda > 0$ et $y$ une solution de
$y'' + \left(1+\frac\lambda{t^2}\right)y = 0$.
Montrer que pour tout $a \in \R$, $y$ a un zéro dans l'intervalle $]a,a+\pi[$.
(\'Etudier $z = y'\varphi - y\varphi'$ où $\varphi(t) = \sin(t-a)$)}
  \reponse{$z' = -\frac\lambda{t^2}\sin(t-a)y(t)$ donc si $y$ ne s'annule pas
sur $]a,a+\pi[$, alors $z$ est strictement monotone sur $[a,a+\pi]$.
Mais $z(a+\pi)-z(a) = y(a+\pi)+y(a)  \Rightarrow $ contradiction de signe.}
}