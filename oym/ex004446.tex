\uuid{4446}
\titre{Mines MP 2003}
\theme{}
\auteur{quercia}
\date{2010/03/14}
\organisation{exo7}
\contenu{
  \texte{Soit la suite de terme général $u_n = \frac{\ln 2}2 + \frac{\ln 3}3 + \dots + \frac{\ln n}n$.}
\begin{enumerate}
  \item \question{Donner un équivalent de~$u_n$ en~$+\infty$.}
  \item \question{Montrer que la suite de terme général~: $v_n = u_n - \frac{\ln^2n}2$ est convergente.}
  \item \question{Soit $\ell = \lim_{n\to\infty} v_n$. Donner un équivalent de~$v_n-\ell$.}
\end{enumerate}
\begin{enumerate}
  \item \reponse{Comparaison série-intégrale~: $u_n\sim\frac{\ln^2 n}2$.}
  \item \reponse{Comparaison série-intégrale encore ($v_n$ est la somme des aires entre
    les rectangles aux points entiers et la courbe de~$t\to\ln(t)/t$).}
  \item \reponse{$v_n-\ell = -\sum_{k=n}^\infty\Bigl( \int_{t=k}^{k+1}\frac{\ln t}t\,d t - \frac{\ln(k+1)}{k+1}\Bigr) = -\sum_{k=n}^\infty w_k$
    avec $w_k\sim\frac{\ln k}{2k^2}$ donc $v_n-\ell\sim- \int_{t=n}^{+\infty}\frac{\ln t}{2t^2}\,d t\sim-\frac{\ln n}{2n}$.}
\end{enumerate}
}