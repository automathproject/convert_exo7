\exo7id{3639}
\titre{Polynômes à deux variables}
\theme{}
\auteur{quercia}
\date{2010/03/10}
\organisation{exo7}
\contenu{
  \texte{Soit $f : {\R^2} \to \R$. On dit que $f$ est polynomiale si elle est de la
forme : $f(x,y) = \sum_{i,j} a_{ij}x^iy^j$, la somme portant sur un nombre
fini de termes. Le degré de $f$ est alors $\max(i+j \text{ tq } a_{ij} \ne 0)$.

On note $E_k$ l'ensemble des fonctions $\R^2 \to \R$ polynomiales de degré
inférieur ou égal à $k$.}
\begin{enumerate}
  \item \question{Montrer que $E_k$ est un $\R$-ev de dimension finie et donner sa dimension.}
  \item \question{Soient $A=(0,0)$, $B=(1,0)$, $C=(0,1)$.
    Montrer que les formes linéaires $f  \mapsto f(A)$, $f  \mapsto f(B)$, $f  \mapsto f(C)$
    constituent une base de $E_1^*$.}
  \item \question{Chercher de même une base de $E_2^*$.}
  \item \question{Soit $T$ le triangle plein $ABC$ et $f \in E_1$.
    Montrer que $ \int_{}^{}\!\! \int_T^{} f(x,y)\,d xd y =
    \frac{f(A)+f(B)+f(C)}6$.}
  \item \question{Chercher une formule analogue pour $f \in E_2$.}
\end{enumerate}
\begin{enumerate}

\end{enumerate}
}