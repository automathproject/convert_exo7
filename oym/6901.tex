\uuid{6901}
\titre{Exercice 6901}
\theme{}
\auteur{ruette}
\date{2013/01/24}
\organisation{exo7}
\contenu{
  \texte{}
  \question{Deux urnes sont remplies de boules. La première contient
10 boules noires et 30 boules blanches. La seconde contient 20
boules noires et 20 boules blanches.
On tire une des urnes au hasard, de façon équiprobable, et dans cette urne, 
on tire une boule au hasard. La boule est blanche. Quelle est la probabilité 
qu'on ait tiré cette boule dans la première urne sachant qu'elle est blanche ?}
  \reponse{Soit $U_1$ l'événement ``on tire la boule dans la première urne''
et $U_2$ l'événement ``on tire la boule dans la seconde urne''. Le choix
de l'urne étant équiprobable, on a : $P(U_1)=P(U_2)=0,5$.
Soit $B$ l'événement ``on tire une boule blanche''.
L'énoncé donne les probabilités conditionnelles suivantes : \\
$P(B|U_1)=30/40=0,75$ et $P(B|U_2)=20/40=0,5$. 

On cherche la probabilité $P(U_1|B)$.
La formule de Bayes,
appliquée à la partition $(H_1,H_2)$, nous donne :
$$
P(U_1|B)=\frac{P(B|U_1)P(U_1)}{P(B|U_1)P(U_1)+P(B|U_2)P(U_2)}=
\frac{0,75\times 0,5}{0,75\times 0,5+0,5\times 0,5}=0,6$$
(probabilité
a posteriori)

\medskip
\textit{Interprétation : avant de regarder la couleur de la boule, la probabilité d'avoir choisi la 
première urne est une probabilité a priori $P(U_1)$ soit 50 \%. 
Après avoir regardé la boule, on révise notre jugement et on considère 
$P(U_1|B)$, soit 60 \%.}}
}