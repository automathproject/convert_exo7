\uuid{3155}
\titre{Th{\'e}or{\`e}me de Wilson}
\theme{Exercices de Michel Quercia, Propriétés de $\Zz/n\Zz$}
\auteur{quercia}
\date{2010/03/08}
\organisation{exo7}
\contenu{
  \texte{}
  \question{Soit $n \ge 2$. Montrer que $n$ est premier si et seulement si
$(n-1)! \equiv -1 (\mathrm{mod}\, n)$.}
  \reponse{{\'E}tudier le m{\^e}me produit dans $\Z/n\Z$.}
}