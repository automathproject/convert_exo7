\uuid{4280}
\titre{Mines-Ponts 1999}
\theme{Exercices de Michel Quercia, Intégrale généralisée}
\auteur{quercia}
\date{2010/03/12}
\organisation{exo7}
\contenu{
  \texte{}
  \question{Calculer $I_n =  \int_{t=0}^{+\infty}\frac{d t}{(t+1)(t+2)\dots(t+n)}$.}
  \reponse{Décomposer en éléments simples et intégrer.
On obtient $I_n = \frac1{(n-1)!}\sum_{k=1}^n(-1)^kC_{n-1}^{k-1}\ln k$.}
}