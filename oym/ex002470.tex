\uuid{2470}
\titre{Exercice 2470}
\theme{}
\auteur{matexo1}
\date{2002/02/01}
\organisation{exo7}
\contenu{
  \texte{}
  \question{Soit $M$ une matrice de $\mathcal M_n(\C)$\,; on suppose qu'il existe un
entier $p$ tel que $M^p = I$. Montrer que si $\omega$ est une
racine $p$-i\`eme de l'unit\'e, c'est une valeur propre de $M$ ou
alors $M$ v\'erifie
$$ M^{p-1} + \omega M^{p-2} +\cdots+\omega^{p-2}M +\omega^{p-1}I=0.$$}
  \reponse{}
}