\uuid{987}
\titre{Exercice 987}
\theme{}
\auteur{cousquer}
\date{2003/10/01}
\organisation{exo7}
\contenu{
  \texte{Dans $\mathbb{R}^3$, les vecteurs suivants forment-ils une base~? Sinon
d\'ecrire le sous-espace qu'ils engendrent.}
\begin{enumerate}
  \item \question{$v_1 =(1,1,1), v_2=(3,0,-1),v_3=(-1,1,-1).$}
  \item \question{$v_1 =(1,2,3), v_2=(3,0,-1),v_3=(1,8,13).$}
  \item \question{$v_1 =(1,2,-3), v_2=(1,0,-1),v_3=(1,10,-11).$}
\end{enumerate}
\begin{enumerate}
  \item \reponse{C'est une base.}
  \item \reponse{Ce n'est pas une base : $v_3=4v_1-v_2$. Donc l'espace $\mathrm{Vect} (v_1,v_2,v_3)=\mathrm{Vect}(v_1,v_2)$.}
  \item \reponse{Ce n'est pas une base : $v_3=5v_1-4v_2$. Donc l'espace $\mathrm{Vect} (v_1,v_2,v_3)=\mathrm{Vect}(v_1,v_2)$.}
\end{enumerate}
}