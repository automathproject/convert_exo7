\uuid{4228}
\titre{Sommes de Riemann}
\theme{}
\auteur{quercia}
\date{2010/03/12}
\organisation{exo7}
\contenu{
  \texte{}
\begin{enumerate}
  \item \question{Trouver $\lim_{n\to\infty} \frac 1{n+1} + \frac 1{n+2} + \dots + \frac 1{kn}$
    pour $k$ entier supérieur ou égal à $2$ fixé.}
  \item \question{Trouver $\lim_{n\to\infty} \frac 1{n^2} \Bigl( \sqrt{1(n-1)}
    + \sqrt{2(n-2)} + \dots + \sqrt{(n-1)1} \Bigr)$.}
  \item \question{Trouver $\lim_{n\to\infty} \sqrt[n]{\Bigl(1+\frac 1n\Bigr)\Bigl(1+\frac 2n\Bigr) \dots
             \Bigl(1+\frac nn\Bigr)}$.}
  \item \question{Trouver $\lim_{n\to\infty} \ln\left(1+\frac\pi n\right)\sum_{k=0}^{n-1} \frac1{2+\cos(3k\pi/n)}$.}
  \item \question{Donner un équivalent pour $n\to\infty$ de $\sum_{k=1}^n \sqrt k$.}
  \item \question{Soit $A_1A_2\dots A_n$ un polygone régulier inscrit dans un cercle de
    rayon 1. Chercher $\lim_{n\to\infty} \frac 1n\sum_{k=2}^n A_1A_k$.}
\end{enumerate}
\begin{enumerate}
  \item \reponse{$\ln k$.}
  \item \reponse{$\frac \pi8$.}
  \item \reponse{$\frac 4e$.}
  \item \reponse{$\frac13 \int_{t=0}^{3\pi}\frac{d t}{2+\cos t}
             =  \int_{t=0}^{\pi}\frac{d t}{2+\cos t} = \frac\pi{\sqrt3}$.}
  \item \reponse{$\frac 43n\sqrt n$.}
  \item \reponse{$\frac 4\pi$.}
\end{enumerate}
}