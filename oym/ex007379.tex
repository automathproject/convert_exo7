\uuid{7379}
\titre{Exercice 7379}
\theme{}
\auteur{mourougane}
\date{2021/08/10}
\organisation{exo7}
\contenu{
  \texte{}
\begin{enumerate}
  \item \question{Calculer, avec l'algorithme de Hörner appliqué au polynôme $P(x)=x^2-7$, une valeur approchée à $10^{-1}$ près de $\sqrt{7}$.}
  \item \question{Calculer avec l'algorithme de Hörner une valeur approchée à $10^{-2}$ près de $\sqrt{7}$.}
\end{enumerate}
\begin{enumerate}
  \item \reponse{$2^3<7<3^2$. Donc, $2<\sqrt{7}<3$.
Soit $P(x)=x^2-7$.
$$\begin{array}{c|ccc}
 &1&0&-7\\ \hline \textrm{facteur }2&1&2&-3\\
 &1&4&\\ &1&&
\end{array}$$
Donc
$P(2+y)=y^2+4y-3$. Par conséquent,
$Q(z):=100P(2+z/10)=z^2+40z-300$.
Avec $300/40\sim 7$
On trouve $Q(6)<0$ et $Q(7)>0$.
Donc $$2,6<\sqrt{7}<2.7.$$}
  \item \reponse{$Q(z)=z^2+40z-300$.
$$\begin{array}{c|ccc}
 &1&40&-300\\ \hline \textrm{facteur }6&1&46&-24\\
 &1&52&\\ &1&&
\end{array}$$
Donc, $Q(6+u)=u^2+52u-24$. Par conséquent,
$R(v):=100Q(6+v/10)=v^2+520u-2400$.
Comme $2400/520\sim 5$, on trouve $R(5)>0$ et $R(4)<0$.
Donc $$2,64<\sqrt{7}<2.65.$$}
\end{enumerate}
}