\exo7id{1727}
\titre{Exercice 1727}
\theme{}
\auteur{legall}
\date{1998/09/01}
\organisation{exo7}
\contenu{
  \texte{Soit $  A\in M_n(\mathbb{K} )  .$ On note $  C(A)=\{ B \in M_n(\mathbb{K} )  ;
AB=BA\}   .$}
\begin{enumerate}
  \item \question{On suppose que $  A  $ a des valeurs propres simples. Montrer que les propri\'et\'es suivantes sont
\'equivalentes~:
\vskip1mm
\hskip1mm i) $  B\in C(A)  .$
\vskip1mm
\hskip1mm ii) $  B  $ a une base de vecteurs propres en commun avec $  A  .$
\vskip1mm
\hskip1mm iii) Il existe $  P\in \mathbb{K} _{n-1}[X]  $ tel que $  B=P(A)  .$
\vskip1mm
\hskip1mm iv) Il existe $  P\in \mathbb{K} [X]  $ tel que $  B=P(A)  .$
\vskip1mm}
  \item \question{On suppose que $  n =3  $ (pour simplifier) et que $  A  $ est diagonalisable
avec une valeur propre double. D\'eterminer $  C(A)  .$}
\end{enumerate}
\begin{enumerate}

\end{enumerate}
}