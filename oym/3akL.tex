\uuid{3akL}
\exo7id{2387}
\titre{Exercice 2387}
\theme{Connexité}
\auteur{mayer}
\date{2003/10/01}
\organisation{exo7}
\contenu{
  \texte{}
\begin{enumerate}
  \item \question{Montrer qu'il existe une surjection continue de $\Rr $ sur
$\mathbb{S}^1 =\{z\in \Cc \; ; \;\; |z|=1\}$ et qu'il n'existe pas
d'injection continue de $\mathbb{S}^1$ dans $\Rr$.}
  \item \question{Montrer qu'il n'existe pas d'injection continue de $\Rr^2$
dans $\Rr$.}
\end{enumerate}
\begin{enumerate}
  \item \reponse{\begin{enumerate}}
  \item \reponse{$\phi : \Rr \longrightarrow \mathbb{S}^1$ définie par $\phi(t)=e^{it}$ est une surjection continue.}
  \item \reponse{$\mathbb{S}^1$ est un compact connexe donc, par l'absurde, si $\psi : \mathbb{S}^1 \longrightarrow \Rr$ est une injection continue alors $\psi(\mathbb{S}^1)$ est un compact connexe de $\Rr$ donc un segment compact $I$. Soit $y \in \mathring I$, comme $I$ est l'image de $\mathbb{S}^1$ alors il existe un unique $x \in \mathbb{S}^1$ tel que $f(x)=y$. L'application $f$ induit alors une bijection continue 
$f : \mathbb{S}^1 \setminus \{ x \} \longrightarrow I \setminus \{ y \}$.
Mais $\mathbb{S}^1 \setminus \{ x \}$ est connexe alors que son image par
$f$, qui est $ I \setminus \{ y \}$ ne l'est pas (car $y\in \mathring I$).
L'image d'un connexe par une application continue doit être un connexe, donc nous avons une contradiction.}
\end{enumerate}
}