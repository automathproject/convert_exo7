\uuid{7750}
\titre{Avec un repère projectif}
\theme{Exercices de Christophe Mourougane, Théorie des groupes et géométrie}
\auteur{mourougane}
\date{2021/08/11}
\organisation{exo7}
\contenu{
  \texte{}
  \question{Dans un plan projectif réel, on considère le repère projectif $(A, B, C; I)$.
Soit $A', B', C'$ respectivement sur $(BC), (CA)$ et $(AB)$ tels que
$(AA'), (BB')$ et $(CC')$ soient concourantes en $I$.
Montrer analytiquement que les points $P:=(BC)\cap (B'C')$,
$Q:=(CA)\cap(C'A')$ et $R:=(AB)\cap(A'B')$ sont alignés.}
  \reponse{}
}