\uuid{1499}
\titre{Exercice 1499}
\theme{}
\auteur{ortiz}
\date{1999/04/01}
\organisation{exo7}
\contenu{
  \texte{}
  \question{$\Rr^3$ est muni de sa structure canonique
d'espace vectoriel euclidien. V\'erifier que les
vecteurs $e_1=(1,0,1),$ $e_2=(1,0,2)$ et
$e_3=(1,1,1)$ forment une base de $\Rr^3$ et  en
d\'eterminer l'orthonormalis\'ee de Gram-Schmidt.}
  \reponse{}
}