\uuid{5159}
\titre{**}
\theme{Valeurs absolues. Partie entière. Inégalités}
\auteur{rouget}
\date{2010/06/30}
\organisation{exo7}
\contenu{
  \texte{}
  \question{Montrer que $\forall n\in\Nn^*,\;\forall x\in\Rr,\;E(\frac{E(nx)}{n})=E(x)$.}
  \reponse{Soient $n\in\Nn^*$ et $x\in\Rr$.

\begin{align*}
E(x)\leq x<E(x)+1&\Rightarrow nE(x)\leq nx<nE(x)+n\Rightarrow nE(x)\leq E(nx)<nE(x)+n
\Rightarrow E(x)\leq\frac{E(nx)}{n}<E(x)+1\\
 &\Rightarrow E(\frac{E(nx)}{n})=E(x).
\end{align*}}
}