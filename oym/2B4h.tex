\uuid{2B4h}
\exo7id{3481}
\titre{Matrice à diagonale dominante}
\theme{Exercices de Michel Quercia, Rang de matrices}
\auteur{quercia}
\date{2010/03/10}
\organisation{exo7}
\contenu{
  \texte{Soit $M=(a_{ij}) \in \mathcal{M}_n(\C)$. On dit que $M$ est {\it à diagonale dominante\/}
si : $\forall\ i,\ |a_{ii}| > \sum_{j\ne i}|a_{ij}|$.}
\begin{enumerate}
  \item \question{On transforme $M$ en
    $M' = \begin{pmatrix} a_{11} &\ a_{12}&\dots &a_{1n} \cr
                     0                         \cr
                   \vdots &        &M_1        \cr
                     0                         \cr \end{pmatrix}$
    par la méthode du pivot. Montrer que $M_1$ est à diagonale dominante.}
  \item \question{En déduire que $M$ est inversible.}
\end{enumerate}
\begin{enumerate}

\end{enumerate}
}