\uuid{6762}
\titre{Exercice 6762}
\theme{}
\auteur{queffelec}
\date{2011/10/16}
\organisation{exo7}
\contenu{
  \texte{On se propose de démontrer que pour tout $z$ de $\C$,
$$\prod_{n\ge 1}\left( 1-{z^2\over n^2}\right)={\sin{(\pi z)}\over
\pi z}$$}
\begin{enumerate}
  \item \question{On pose
$$F(z) ={1\over z}+\sum_{n\in {\Zz}, n\ne 0}\left({1\over z-n}+{1\over
n}\right)$$
Montrer que $F$ est une fonction méromorphe sur $\C$.

En utilisant 
$$\sum_{n\in{\Zz}}{1\over (z-n)^2}=\left({\pi\over \sin{(\pi
z)}}\right)^2$$
 montrer que $\displaystyle F(z)-{\pi\over \tan{(\pi z)}}$ est constante
sur $\C\setminus {\Zz}$, puis calculer cette constante par un argument
de parité. En déduire que
$${\pi\over \tan{(\pi z)}}={1\over z}+\sum_{n\ge 1}{2z\over z^2-n^2}$$}
  \item \question{Pour $n\ge 1$, soit $\displaystyle f_n(z)=1-{z^2\over n^2}$. Montrer que
$\prod_{n\ge 1}f_n(z)$ définit une fonction entière $f$.

Montrer que
$$\sum_{n\ge 1}{f_n'(z)\over f_n(z)}={g'(z)\over g(z)}$$
avec $\displaystyle g(z)={\sin{(\pi z)}\over \pi z}$. En déduire le résultat voulu.}
  \item \question{Déduire de la décomposition de $\displaystyle {\sin{(\pi z)}\over
\pi z}$ en produit que
\begin{eqnarray*}
{\sh z\over z}&=&\prod_{n\ge 1}\left( 1+{z^2\over n^2\pi^2}\right)\\
\ch z&=&\prod_{n\ge 1}\left( 1+{4z^2\over (2n-1)^2\pi^2}\right)\\
\cos z&=&\prod_{n\ge 1}\left( 1-{4z^2\over (2n-1)^2\pi^2}\right)\\
\end{eqnarray*}}
\end{enumerate}
\begin{enumerate}

\end{enumerate}
}