\uuid{2837}
\titre{Exercice 2837}
\theme{}
\auteur{burnol}
\date{2009/12/15}
\organisation{exo7}
\contenu{
  \texte{}
  \question{Justifier les affirmations du polycopié
relatives à l'invariance de l'indice d'un lacet par rapport
à un point, lorsque l'on déforme continûment soit le lacet,
soit le point. Montrer que lorsque $\gamma$ est un lacet il
existe $R$ tel que $|z|>R\implies \mathrm{Ind}(\gamma,z) = 0$.}
  \reponse{Soit $\gamma$ un lacet dans $\C$. Alors $\sup_{t\in I} |\gamma(t)| =R/2< \infty$. Si $|z|> R$, la fonction $\xi \mapsto \frac{1}{\xi -z}$ est holomorphe dans le disque $D(0,R)$. Par cons\'equent,
$$\mathrm{Ind}(\gamma , z) =\frac{1}{2i\pi} \int_\gamma \frac{d\xi}{\xi -z} =0.$$
Remarquons que ceci implique que $\mathrm{Ind}(\gamma , z)=0$ pour tout $z$ dans la composante connexe non born\'ee de $\C\setminus \gamma$.}
}