\uuid{v1da}
\exo7id{2954}
\titre{$\sum \cos^{2p}(x+k\pi/2p)$}
\theme{Exercices de Michel Quercia, Nombres complexes}
\auteur{quercia}
\date{2010/03/08}
\organisation{exo7}
\contenu{
  \texte{Soit $\theta \in \R$.}
\begin{enumerate}
  \item \question{Simplifier $\cos^4\theta + \cos^4\left(\theta + \frac \pi4\right) +
                \cos^4\left(\theta + \frac {2\pi}4\right) +
                \cos^4\left(\theta + \frac {3\pi}4\right)$.}
  \item \question{Simplifier $\cos^6\theta + \cos^6\left(\theta + \frac \pi6\right) + \dots +
                \cos^6\left(\theta + \frac {5\pi}6\right)$.}
  \item \question{Simplifier $\cos^{2p}\theta + \cos^{2p}\left(\theta + \frac \pi{2p}\right)
                + \dots + \cos^{2p}\left(\theta + \frac {(2p-1)\pi}{2p}\right)$.}
\end{enumerate}
\begin{enumerate}
  \item \reponse{$= 3/2$.}
  \item \reponse{$32\cos^6(\theta) = \cos6\theta + 6\cos4\theta + 15\cos2\theta
    + 10 \Rightarrow \Sigma = \frac {15}8$.}
  \item \reponse{$\Sigma_p = \frac {pC_{2p}^p}{2^{2p-1}}$.}
\end{enumerate}
}