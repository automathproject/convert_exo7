\uuid{4198}
\titre{Centrale MP 2006}
\theme{}
\auteur{quercia}
\date{2010/03/11}
\organisation{exo7}
\contenu{
  \texte{$E$ désigne l'espace affine euclidien classique. $D_1$, $D_2$, $D_3$
sont trois droites deux à deux non parallèles.
\smallskip
Soit $f : {D_1\times D_2\times D_3} \to \R, {(M_1,M_2,M_3)} \mapsto
{\|\vec{M_1M_2}\|^2 + \|\vec{M_2M_3}\|^2 + \|\vec{M_3M_1}\|^2.}$}
\begin{enumerate}
  \item \question{Montrer que $f$ admet un minimum atteint pour un unique triplet.}
  \item \question{Dans le cas où $D_1$, $D_2$, $D_3$ sont coplanaires et délimitent
    un triangle équilatéral, trouver ce triplet.}
\end{enumerate}
\begin{enumerate}
  \item \reponse{On fixe $A_i\in D_i$ et $\vec u_i$ un vecteur directeur
    de~$D_i$. Soit $M_i = A_i + x_i\vec u_i$. Alors

    \begin{align*}f(M_1,M_2,M_3) &= f(A_1,A_2,A_3)\\
                    &+ 2\Bigl(
                        (\vec{A_1A_2}\mid x_2\vec u_2 - x_1\vec u_1)
                      + (\vec{A_2A_3}\mid x_3\vec u_3 - x_2\vec u_2)
                      + (\vec{A_3A_1}\mid x_1\vec u_1 - x_3\vec u_3)
                     \Bigr)\\
                    &+ \Bigl(
                        \|x_2\vec u_2 - x_1\vec u_1\|^2
                      + \|x_3\vec u_3 - x_2\vec u_2\|^2
                      + \|x_1\vec u_1 - x_3\vec u_3\|^2
                     \Bigr)\\
                    &= a + b(x_1,x_2,x_3) + c(x_1,x_2,x_3).
\end{align*}

    $b$ est une forme linéaire et $c$ est une forme quadratique positive,
    et même définie positive car $\vec u_1,\vec u_2,\vec u_3$ sont deux
    à deux non colinéaires. Il en résulte que
    $f(M_1,M_2,M_3) \to +\infty$ lorsque $|x_1|+|x_2|+|x_3|\to\infty$, donc par
    continuité, $f$ admet un minimum.
    
    Choisissons alors $A_1,A_2,A_3$ de sorte que $f(A_1,A_2,A_3)$ soit égal
    à ce minimum. On a alors $b=0$ car $(A_1,A_2,A_3)$ est point critique
    de~$f$, d'où $f(M_1,M_2,M_3) > f(A_1,A_2,A_3)$ si $(x_1,x_2,x_3)\ne(0,0,0)$
    vu la définie-positivité de~$c$. Ceci prouve l'unicité du triplet
    où $f$ atteint son minimum.}
  \item \reponse{On soup\c conne fortement le triplet constitué des milieux
    des côtés. En notant $A_1,A_2,A_3$ ces milieux, il suffit de
    vérifier que la forme linéaire $b$ de la réponse précédente
    est nulle, et c'est clairement le cas après regroupement autour
    de~$x_1$, $x_2$, $x_3$.}
\end{enumerate}
}