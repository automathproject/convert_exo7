\uuid{Dxxq}
\exo7id{4909}
\titre{Cordes perpendiculaires, Centrale P' 1996}
\theme{Exercices de Michel Quercia, Coniques}
\auteur{quercia}
\date{2010/03/17}
\organisation{exo7}
\contenu{
  \texte{On considère une parabole dans le plan euclidien.}
\begin{enumerate}
  \item \question{Exprimer l'équation d'une droite passant par deux points $A$ et~$B$
    de la parabole à l'aide d'un déterminant d'ordre~3.}
  \item \question{$A,B,C$ étant trois points sur la parabole, exprimer le fait que
    $(AB)$ et $(AC)$ sont perpendiculaires.}
  \item \question{On fixe $A$ sur la parabole, $B$ et $C$ sont deux points de la parabole
    variables tels que $(AB)$ et $(AC)$ sont perpendiculaires.
    Montrer que $(BC)$ passe par un point fixe~$M$.}
  \item \question{Quel est le lieu de $M$ quand $A$ varie~?}
\end{enumerate}
\begin{enumerate}
  \item \reponse{Parabole~: $y^2 = 2px \Rightarrow  x=2pt^2, y=2pt$.
         Corde~: $\begin{vmatrix}2pa^2 &2pb^2 &x\cr
                            2pa   &2pb   &y\cr
                            1     &1     &1\cr\end{vmatrix}=0$.}
  \item \reponse{$a^2  + ab + ac + bc + 1 = 0$.}
  \item \reponse{$c=-\frac{a^2+ab+1}{a+b}$.\par
$(BC)$~: $(2pa+y)b^2 + (2pa^2+2p-x)b -(ax+a^2y+y) = 0$.\par
Point fixe~: $y=-2pa$, $x=2p(a^2+1)$.}
  \item \reponse{Parabole translatée de~${\cal P}$ de $(2p,0)$.}
\end{enumerate}
}