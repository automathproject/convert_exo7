\uuid{3454}
\titre{$\det( A + (\alpha) )$}
\theme{}
\auteur{quercia}
\date{2010/03/10}
\organisation{exo7}
\contenu{
  \texte{Soit $A \in \mathcal{M}_n(K)$ et
$U = \begin{pmatrix} 1       &\dots  &1       \cr
                \vdots  &       &\vdots  \cr
                1       &\dots  &1       \cr \end{pmatrix}$.}
\begin{enumerate}
  \item \question{Démontrer que :
    $\det(A + \alpha U) = \det A + \alpha \sum \text{cofacteurs de } A$.}
  \item \question{En déduire la valeur de
     $D(a,b,c) = \begin{vmatrix}a       & b      & \dots  & b      \cr
                           c       & \ddots & \ddots & \vdots \cr
                           \vdots  & \ddots & \ddots & b      \cr
                           c       & \dots  & c      & a      \end{vmatrix}$,
  \begin{enumerate}}
  \item \question{pour $b \ne c$.}
  \item \question{pour $b = c$.}
\end{enumerate}
\begin{enumerate}
  \item \reponse{Développer.}
  \item \reponse{\begin{enumerate}}
  \item \reponse{$\begin{cases} D(a-b,0,c-b) = (a-b)^n \cr
                       D(a-c,b-c,0) = (a-c)^n \cr\end{cases}  \Rightarrow 
               D(a,b,c) = \frac { c(a-b)^n - b(a-c)^n } {c-b}$.}
  \item \reponse{$\det( (a-b)I + bU ) = (a-b)^n + nb(a-b)^{n-1}$.}
\end{enumerate}
}