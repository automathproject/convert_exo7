\uuid{3954}
\titre{Racines de $x^n + ax + b$}
\theme{Exercices de Michel Quercia, Dérivation}
\auteur{quercia}
\date{2010/03/11}
\organisation{exo7}
\contenu{
  \texte{}
  \question{Soit $n \in \N,\ n \ge 2$, et $a,b \in \R$.
Montrer que l'équation $x^n + ax + b = 0$ ne peut avoir plus de deux racines réelles
distinctes si $n$ est pair, et plus de trois racines réelles distinctes si
$n$ est impair.}
  \reponse{}
}