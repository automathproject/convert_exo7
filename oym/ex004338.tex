\exo7id{4338}
\titre{Convolution (Mines MP 2003)}
\theme{}
\auteur{quercia}
\date{2010/03/12}
\organisation{exo7}
\contenu{
  \texte{Soient $f,g\in\mathcal{C}([0,+\infty[,\R)$. On pose $h(x) =  \int_{t=0}^x f(x-t)g(t)\,d t$.}
\begin{enumerate}
  \item \question{Existence et continuité de~$h$.}
  \item \question{Peut-on inverser $f$ et $g$~?}
  \item \question{On suppose $f$ intégrable sur~$[0,+\infty[$ et $g$ bornée. Montrer que~$h$ est bornée.}
  \item \question{On prend $f(x) = \frac{\sin x}x$ et $g(x) = \cos(\alpha x)$ avec $0\le\alpha\le 1$.
    $h$ est-elle bornée (on pourra étudier les cas $\alpha=0$ et $\alpha=1$)~?}
\end{enumerate}
\begin{enumerate}
  \item \reponse{Pour $\alpha=0$ on a $h(x) =  \int_{t=0}^x\frac{\sin t}t\,d t$, quantité
    bornée car l'intégrale converge en~$+\infty$.

    Pour $\alpha=1$ on a $h(x) = \cos x \int_{t=0}^x \frac{\cos t\sin t}t\,d t
                               + \sin x \int_{t=0}^x \frac{\sin^2 t}t\,d t$,
    quantité non bornée car la deuxième intégrale diverge en~$+\infty$.

    Pour $0<\alpha<1$, développer le $\cos(x-t)$ puis linéariser les produits
    obtenus. On obtient quatre intégrales convergentes, donc $h$ est bornée.}
\end{enumerate}
}