\uuid{TI41}
\exo7id{4693}
\titre{Convergence vers $0$ et monotonie}
\theme{Exercices de Michel Quercia, Suites convergentes}
\auteur{quercia}
\date{2010/03/16}
\organisation{exo7}
\contenu{
  \texte{Soit $(x_n)$ une suite de r{\'e}els strictement positifs convergeant vers 0.}
\begin{enumerate}
  \item \question{Montrer qu'il existe une infinit{\'e} d'indices $n$ tels que
    $x_n = \max(x_n,x_{n+1},x_{n+2},\dots)$.}
  \item \question{Montrer qu'il existe une infinit{\'e} d'indices $n$ tels que
    $x_n = \min(x_0,x_1,\dots,x_n)$.}
\end{enumerate}
\begin{enumerate}
  \item \reponse{Sinon, on construit une sous-suite strictement croissante.}
  \item \reponse{La suite $(\min(x_0,\dots,x_n))$ converge vers $0$, et prend
             une infinit{\'e} de valeurs diff{\'e}rentes.}
\end{enumerate}
}