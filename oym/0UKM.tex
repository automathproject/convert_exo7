\uuid{0UKM}
\exo7id{5830}
\titre{***}
\theme{Quadriques}
\auteur{rouget}
\date{2010/10/16}
\organisation{exo7}
\contenu{
  \texte{Trouver une équation du cône de sommet $S$ circonscrit à la surface $(\mathcal{S})$ quand}
\begin{enumerate}
  \item \question{$S(0,5,0)$ et $(\mathcal{S})$ : $x^2+y^2+z^2=9$,}
  \item \question{$S(0,0,0)$ et $(\mathcal{S})$ : $x^2+xy+z-1=0$.  (Préciser la courbe de contact.)}
\end{enumerate}
\begin{enumerate}
  \item \reponse{Ici $(\mathcal{S})$ est la sphère de centre $O$ et de rayon $3$ et  le point $S$ est extérieur à cette sphère. Donc

\begin{align*}\ensuremath
M(x,y,z)\in(C)&\Leftrightarrow M=S\;\text{ou}\;M\neq S\;\text{et}\;d(O,(SM))= 3\Leftrightarrow M=S\;\text{ou}\;M\neq S\;\text{et}\;\|\overrightarrow{SO}\wedge\overrightarrow{SM}\|= 3\|\overrightarrow{SM}\|\\
 &\Leftrightarrow\|\overrightarrow{SO}\wedge\overrightarrow{SM}\|= 3\|\overrightarrow{SM}\|\Leftrightarrow
\|(0,5,0)\wedge(x,y-5,z) \| = 3\|(x,y-5,z)\|\\
 &\Leftrightarrow(5z)^2+(5x)^2 =9(x^2+(y-5)^2+z^2)\Leftrightarrow16x^2-9(y-5)^2 +16z^2= 0.
\end{align*}}
  \item \reponse{Soit $M_0(x_0,y_0,z_0)$ un point de $(\mathcal{S})$ (c'est-à-dire tel que $x_0^2+x_0y_0+z_0-1 = 0$). $(\mathcal{S})$ est une surface du second degré. Une équation du plan tangent à $(\mathcal{S})$ en $M_0$ est fournie par la règle de dédoublement des termes : 

\begin{center}
$xx_0+\frac{1}{2}(y_0x+x_0y)+ \frac{1}{2}(z+z_0)-1 = 0$.
\end{center}

Ce plan tangent contient le point $S(0,0,0)$ si et seulement si $z_0 = 2$ ce qui montre déjà que la courbe de contact admet pour système d'équations  $\left\{
\begin{array}{l}
x^2+xy+z-1=0\\
z=2
\end{array}
\right.$ ou encore$\left\{
\begin{array}{l}
x^2+xy+1=0\\
z=2
\end{array}
\right.$. C'est une hyperbole du plan d'équation $z = 2$.

Le cône de sommet $S$ circonscrit à $(\mathcal{S})$ est alors le cône de sommet $S$ et de directrice $(\mathcal{C})$ d'équations $\left\{
\begin{array}{l}
x^2+xy+1=0\\
z=2
\end{array}
\right.$. On trouve la surface d'équation $4x^2+4xy+z^2 = 0$. C'est un cône du second degré.}
\end{enumerate}
}