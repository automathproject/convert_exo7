\uuid{4422}
\titre{Centrale P' 1996}
\theme{Exercices de Michel Quercia, Séries numérique}
\auteur{quercia}
\date{2010/03/14}
\organisation{exo7}
\contenu{
  \texte{}
  \question{Montrer que la série $\sum_{n=1}^\infty \frac{n^2}{(1+n^2)^2}$ converge.
Calculer une valeur approchée à $10^{-4}$ près de sa somme.}
  \reponse{$\frac{n^2}{(n^2+1)^2}-\frac{1}{n^2-1}
         = -\frac{3n^2+1}{(n^2+1)^2(n^2-1)} \ge -\frac{4}{n^4}$ pour
         $n\ge 3$.

Donc $S = \sum_{n=1}^N \frac{n^2}{(n^2+1)^2}
        + \sum_{n=N+1}^\infty \frac{1}{n^2-1} + R_N$ avec
     $-\frac{4}{3N^3}\le R_N\le 0$ et
     $\sum_{n=N+1}^\infty \frac{1}{n^2-1} = \frac{N+\frac12}{N(N+1)}$.

Pour $N=25$ on obtient~: $0.76981 < S < 0.76990$.}
}