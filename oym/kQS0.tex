\uuid{kQS0}
\exo7id{6978}
\titre{Exercice 6978}
\theme{Fonctions circulaires et hyperboliques inverses, Fonctions hyperboliques inverses}
\auteur{blanc-centi}
\date{2014/05/06}
\organisation{exo7}
\contenu{
  \texte{Simplifier les expressions suivantes:}
\begin{enumerate}
  \item \question{$\ch(\Argsh x),\quad \tanh(\Argsh x),\quad \sh(2\Argsh x)$.}
  \item \question{$ \sh(\Argch x),\quad \tanh(\Argch x),\quad \ch(3\Argch x)$.}
\end{enumerate}
\begin{enumerate}
  \item \reponse{\begin{enumerate}}
  \item \reponse{On sait que $\ch^2 u=1+\sh^2u$. Comme de plus la fonction $\ch$ est à valeurs positives, 
$\ch u=\sqrt{1+\sh^2u}$ et donc $\ch(\Argsh x)=\sqrt{1+\sh^2(\Argsh x)} = \sqrt{1+x^2}$.}
  \item \reponse{Alors
$$\tanh(\Argsh x)=\frac{\sh(\Argsh x)}{\ch(\Argsh x)}=\frac{x}{\sqrt{1+x^2}}.$$}
  \item \reponse{Et $\sh(2\Argsh x)=2\ch(\Argsh x)\sh(\Argsh x)=2x\sqrt{1+x^2}$.}
\end{enumerate}
}