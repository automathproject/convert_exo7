\uuid{4964}
\titre{Perpendiculaire commune à deux droites}
\theme{Exercices de Michel Quercia, Géométrie euclidienne en dimension 3}
\auteur{quercia}
\date{2010/03/17}
\organisation{exo7}
\contenu{
  \texte{}
  \question{Dans un \emph{rond} on donne les droites
$D  : \begin{cases}x +2y - z = 1  \cr 2x - y +2z = 2 \cr\end{cases}$ et
$D' : \begin{cases}x + y + z = 3  \cr  x - y +2z = 0.\cr\end{cases}$

Calculer $d(D,D')$.}
  \reponse{$H:(-2/19, 28/19, 35/19)$, $K:(-6/19, 40/19, 23/19)$, $d=4\sqrt{19}$.}
}