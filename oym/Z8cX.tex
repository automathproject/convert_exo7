\uuid{Z8cX}
\exo7id{2925}
\titre{{\'E}quations affines}
\theme{Exercices de Michel Quercia, Nombres complexes}
\auteur{quercia}
\date{2010/03/08}
\organisation{exo7}
\contenu{
  \texte{}
\begin{enumerate}
  \item \question{Montrer que toute droite du plan admet pour {\'e}quation complexe :
    $az + \overline{az} = b$ avec $a \in \C^*, b \in \R$.}
  \item \question{Soient $a,b,c \in \C$, $a,b$ non tous deux nuls. Discuter la nature
    de $E = \{ z \in \C$ tq $az + b\overline{z} = c \}$.}
\end{enumerate}
\begin{enumerate}
  \item \reponse{si $|a| \ne |b|$ : une solution unique, \\
         si $|a| = |b|$ : une droite ou $\varnothing$.}
\end{enumerate}
}