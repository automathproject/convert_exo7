\uuid{7172}
\titre{Exercice 7172}
\theme{}
\auteur{megy}
\date{2017/07/26}
\organisation{exo7}
\contenu{
  \texte{}
  \question{%[application directe d'AM>GM]
Un magasin vend au même prix deux lots de trois cristaux. Le premier lot comporte trois cristaux cubiques de côté $a$, $b$ et $c$ respectivement. Le second lot comporte trois cristaux identiques en forme de parallélépipède de dimensions $a\times b \times c$. Quel lot est-il préférable d'acheter ?}
  \reponse{Il s'agit de savoir laquelle des deux quantités
\[ 
a^3+b^3+c^3
\text{ et }
3abc
\]
est la plus grande.

Or, en appliquant l'inégalité arithmético-géométrique à $a^3$, $b^3$ et $c^3$, on obtient directement:
\[
\frac13\left(  a^3+b^3+c^3\right)
\geq 
\sqrt[3]{a^3b^3c^3} = abc.
\]
Il est donc préférable d'acheter les trois cubes.}
}