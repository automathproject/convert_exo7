\uuid{4038}
\titre{maximum de $x\cos^nx$}
\theme{}
\auteur{quercia}
\date{2010/03/11}
\organisation{exo7}
\contenu{
  \texte{On note $f_n(x) = x\cos^nx$. Soit $x_n \in \left[0,\frac \pi2 \right]$ tel que
$f_n(x_n)$ soit maximal.}
\begin{enumerate}
  \item \question{Existence et unicité de $x_n$ ?}
  \item \question{Chercher $\lim_{n \to \infty} x_n$.}
  \item \question{Montrer que $x_n^2 \sim \frac 1n$ ($n\to\infty$).}
  \item \question{Trouver un équivalent de $f_n(x_n)$.}
\end{enumerate}
\begin{enumerate}
  \item \reponse{$f_n'(x) = 0 \iff \mathrm{cotan} x = nx$.}
  \item \reponse{$0$.}
  \item \reponse{$x_n\tan x_n = \frac 1n$.}
  \item \reponse{$\ln\left(\frac{y_n}{x_n}\right) \to -\frac 1e
                 \Rightarrow  y_n \sim \frac 1{\sqrt{ne}}$.}
\end{enumerate}
}