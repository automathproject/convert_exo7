\uuid{2107}
\titre{Exercice 2107}
\theme{Groupes, sous-groupes, ordre}
\auteur{debes}
\date{2008/02/12}
\organisation{exo7}
\contenu{
  \texte{}
  \question{On consid\`ere sur $\R$ la loi de composition d\'efinie par $x\star y=
x+y-xy$. Cette loi est-elle associative, commutative? Admet-elle un \'el\'ement neutre?
 Un r\'eel $x$ admet-il un inverse pour cette loi? Donner une formule pour la puissance
$n$-i\`eme d'un \'el\'ement $x$ pour cette loi.}
  \reponse{Pour la derni\`ere question, v\'erifier par r\'ecurrence que $\displaystyle x^{\star\hskip 2pt n}=
\sum_{k=1}^n (-1)^{k-1} C_n^k x^k$.}
}