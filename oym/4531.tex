\uuid{4531}
\titre{Fonction $\zeta$ de Riemann et constante d'Euler}
\theme{Exercices de Michel Quercia, Suites et séries de fonctions}
\auteur{quercia}
\date{2010/03/14}
\organisation{exo7}
\contenu{
  \texte{}
  \question{Soit $\zeta(x) = \sum_{n=1}^\infty \frac1{n^x}$ et
$\gamma = \lim_{n\to\infty}\left(\frac 11 + \dots + \frac 1n -\ln(n) \right)$.

Montrer que $\gamma = 1 + \sum_{n=2}^\infty\left(\frac 1n + \ln\left(1-\frac 1n\right)\right)$
puis que $\gamma = 1 - \sum_{k=2}^\infty\frac{\zeta(k)-1}k$.}
  \reponse{}
}