\uuid{SObF}
\exo7id{3908}
\titre{$\alpha+\beta+\gamma = \pi$}
\theme{Exercices de Michel Quercia, Fonctions usuelles}
\auteur{quercia}
\date{2010/03/11}
\organisation{exo7}
\contenu{
  \texte{Soient $\alpha,\beta,\gamma \in \R$ tels que $\alpha+\beta+\gamma = \pi$.}
\begin{enumerate}
  \item \question{Démontrer que : $1 - \cos\alpha + \cos\beta + \cos\gamma =
                 4\sin\frac\alpha2 \cos\frac\beta2 \cos\frac\gamma2$.}
  \item \question{Simplifier $\tan\frac\alpha2 \tan\frac\beta2  +
             \tan\frac\beta2  \tan\frac\gamma2 +
             \tan\frac\gamma2 \tan\frac\alpha2 $.}
\end{enumerate}
\begin{enumerate}
  \item \reponse{$1 - \cos\alpha = 2\sin\frac\alpha2 \cos\frac {\beta+\gamma}2$,

$\cos\beta + \cos\gamma = 2\sin\frac\alpha2 \cos\frac {\beta-\gamma}2$.}
  \item \reponse{$=1$.}
\end{enumerate}
}