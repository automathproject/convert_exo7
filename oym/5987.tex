\uuid{5987}
\titre{Exercice 5987}
\theme{Probabilité et dénombrement ; indépendance}
\auteur{quinio}
\date{2011/05/18}
\organisation{exo7}
\contenu{
  \texte{}
  \question{Amédée, Barnabé, Charles tirent sur un oiseau; si les
probabilités de succès sont pour Amédée : $70$\%, Barnabé : $50$\%, Charles : $90$\%, quelle est la probabilité que l'oiseau soit
touché?}
  \reponse{Considérons plutôt l'événement complémentaire :
l'oiseau n'est pas touché s'il n'est touché ni par Amédée,
ni par Barnabé, ni par Charles.
Cet événement a pour probabilité : $(1-0.7)\cdot (1-0.5)\cdot (1-0.9)=0.015$.
La probabilité que l'oiseau soit touché est donc : $1-0.015=0.985$.}
}