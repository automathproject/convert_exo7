\uuid{1166}
\titre{Exercice 1166}
\theme{}
\auteur{cousquer}
\date{2003/10/01}
\organisation{exo7}
\contenu{
  \texte{}
  \question{Résoudre, suivant les valeurs de $m$~:
$$(S_1)\; \left\{\begin{array}{rcl}
    x+(m+1)y &=& m+2 \\
    mx+(m+4)y &=&3 
\end{array}\right.
\qquad
(S_2)\; \left\{\begin{array}{rcl}
    mx+(m-1)y & =& m+2 \\
    (m+1)x-my &=&5m+3
\end{array}\right.$$}
  \reponse{$(S_1)$~: solution unique si $m^2 \not=4$, impossible 
sinon. $(S_2)$~: solution unique si  $m^2\not=1/2$, infinité sinon.}
}