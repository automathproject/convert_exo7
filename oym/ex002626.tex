\uuid{2626}
\titre{Exercice 2626}
\theme{}
\auteur{debievre}
\date{2009/05/19}
\organisation{exo7}
\contenu{
  \texte{}
  \question{Soit $f\colon\R^3 \rightarrow \R$ une
fonction de classe $C^1$
et soit  $g\colon \R^3 \rightarrow \R$
la fonction d\'efinie par
\[
g(x,y,z) = f(x-y,y-z,z-x). 
\]
Montrer que
\begin{equation}
\frac{\partial g}{\partial x}  + \frac{\partial g}{\partial y}  + \frac{\partial g}{\partial z} = 0.
\label{ex3}
\end{equation}}
  \reponse{\begin{align*}
\frac{\partial g}{\partial x}
&=
\frac{\partial f}{\partial x}-\frac{\partial f}{\partial z}
\\
\frac{\partial g}{\partial y}
&=
\frac{\partial f}{\partial y}-\frac{\partial f}{\partial x}
\\
\frac{\partial g}{\partial z}
&=
\frac{\partial f}{\partial z}-\frac{\partial f}{\partial y}
\end{align*}
d'o\`u \eqref{ex3}.}
}