\uuid{6267}
\titre{Exercice 6267}
\theme{}
\auteur{queffelec}
\date{2011/10/16}
\organisation{exo7}
\contenu{
  \texte{On va proposer trois démonstrations possibles de l'exercice classique suivant :
soit $f$ une application de classe $C^1$ de ${\Rr}$ dans lui-même, telle que
$|f'(x)|\leq k$ pour tout $x$ réel, où $k\in ]0,1[$. Alors $F$ définie par
$F(x,y)=(x+f(y),y+f(x))$ est un difféomorphisme de classe $C^1$ de ${\Rr}^2$
{\bf sur} lui-même.}
\begin{enumerate}
  \item \question{Remarquer que $F$ est injective et $F'(x,y)\in$ Isom(${\Rr}^2,{\Rr}^2$)
pour tout $(x,y)$.

Reste à établir la surjection.}
  \item \question{1ère méthode : Montrer que $F$ est propre ($\lim ||F(x,y)||=+\infty$ quand
$||(x,y)||\to+\infty$) et que si $(a,b)\in{\Rr}^2$, la fonction
$g(x,y)=||F(x,y)-(a,b)||^2$ est différentiable et atteint sa borne inférieure
en un point annulant $g'(x,y)$; conclure.}
  \item \question{2ème méthode : Montrer que $F({\Rr}^2)$ est à la fois ouverte et fermée.
Conclure.}
  \item \question{3ème méthode : Si $(a,b)\in{\Rr}^2$, appliquer le théorème du point fixe à
l'application $\phi(x,y)=(a-f(y),b-f(x))$; conclure.}
\end{enumerate}
\begin{enumerate}

\end{enumerate}
}