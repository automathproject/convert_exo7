\uuid{5223}
\titre{**}
\theme{Suites}
\auteur{rouget}
\date{2010/06/30}
\organisation{exo7}
\contenu{
  \texte{}
  \question{Soit $(u_n)_{n\in\Nn}$ une suite arithmétique ne s'annulant pas. Montrer que pour tout entier naturel $n$, on a $\sum_{k=0}^{n}\frac{1}{u_ku_{k+1}}=\frac{n+1}{u_0u_{n+1}}$.}
  \reponse{Soit $r$ la raison de la suite $u$.
Pour tout entier naturel $k$, on a 

\begin{center}
$\frac{r}{u_ku_{k+1}}=\frac{u_{k+1}-u_k}{u_ku_{k+1}}=\frac{1}{u_k}-\frac{1}{u_{k+1}}$.
\end{center}
En sommant ces égalités, on obtient~:

$$r\sum_{k=0}^{n}\frac{1}{u_ku_{k+1}}=\sum_{k=0}^{n}\left(\frac{1}{u_k}-\frac{1}{u_{k+1}}\right)=\frac{1}{u_0}-\frac{1}{u_{n+1}}=\frac{u_{n+1}-u_0}{u_0u_{n+1}}=\frac{(n+1)r}{u_0u_{n+1}}.$$
Si $r\neq0$, on obtient $\sum_{k=0}^{n}\frac{1}{u_ku_{k+1}}=\frac{(n+1)}{u_0u_{n+1}}$, et si $r=0$ (et $u_0\neq0$), $u$ est constante et le résultat est immédiat.}
}