\uuid{2995}
\titre{Relation d'{\'e}quivalence avec deux sous-groupes}
\theme{}
\auteur{quercia}
\date{2010/03/08}
\organisation{exo7}
\contenu{
  \texte{Soient $H,K$ deux sous-groupes d'un groupe $G$.
Pour $x,y \in G$, on pose :
$$x \sim y \iff \exists\ h \in H,\ \exists\ k \in K \text{ tq } y = hxk.$$}
\begin{enumerate}
  \item \question{Montrer que c'est une relation d'{\'e}quivalence.}
  \item \question{Pour $x \in G$, soit $G_x = \{(h,k) \in H\times K$ tq $hxk^{-1} = x\}$.
    Montrer que $G_x$ est un sous-groupe de $H\times K$.}
  \item \question{Si $H$ et $K$ sont finis, montrer que chaque classe d'{\'e}quivalence est finie
    de cardinal divisant $\mathrm{Card}\,(H)\mathrm{Card}\,(K)$.}
\end{enumerate}
\begin{enumerate}

\end{enumerate}
}