\exo7id{4554}
\titre{Centrale MP 2002}
\theme{}
\auteur{quercia}
\date{2010/03/14}
\organisation{exo7}
\contenu{
  \texte{On pose $\phi (x)= d(x,\Z)=\inf\{ |x-n| \text{ tel que } n\in\Z\}$.}
\begin{enumerate}
  \item \question{Montrer que $f$ : $\R\ni x \mapsto \sum_{n=0}^{+\infty} (\frac34) ^n\phi (4^nx)$
    est définie et continue.}
  \item \question{Montrer que $\phi$ est lipschitzienne. Que peut-on en déduire pour $f$~?}
  \item \question{Montrer que $f$ n'est dérivable en aucun point.}
\end{enumerate}
\begin{enumerate}
  \item \reponse{La série converge normalement et $\phi$ est continue.}
  \item \reponse{$\phi$ est $1$-lipschitzienne, mais on ne peut rien en déduire pour $f$~:

pour $N$ fixé et $0<h\le \frac1{2.4^N}$, on a $|f(h)-f(0)| = f(h) \ge \sum_{n=1}^N3^nh = \frac{3^{N+1}-3}2h$
donc $f$ n'est pas lipschitzienne au voisinage de~$0$.}
  \item \reponse{D'après ce qui précède, le taux d'accroissement de~$f$ en $0$ est arbitrairement
grand, donc $f$ n'est pas dérivable en~$0$. On montre de même que $f$ n'est pas dérivable
en $x\in\R$.}
\end{enumerate}
}