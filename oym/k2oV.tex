\uuid{k2oV}
\exo7id{4680}
\titre{{\'E}tude de $C_{2n}^n/4^n$}
\theme{Exercices de Michel Quercia, Suites convergentes}
\auteur{quercia}
\date{2010/03/16}
\organisation{exo7}
\contenu{
  \texte{On pose $u_n = \frac {1\times 3\times 5 \times \dots \times (2n-1)}
                      {2\times 4\times 6 \times \dots \times (2n)}$.}
\begin{enumerate}
  \item \question{Exprimer $u_n$ {\`a} l'aide de factorielles.}
  \item \question{Montrer que la suite $(u_n)$ est convergente.}
  \item \question{Soit $v_n = (n+1)u_n^2$. Montrer que la suite $(v_n)$ converge.
    Que pouvez-vous en d{\'e}duire pour $\lim_{n\to\infty}u_n$ ?}
  \item \question{On note $\alpha = \lim_{n\to\infty}v_n$. En {\'e}tudiant la suite $(nu_n^2)$,
    montrer que $\alpha > 0$.}
\end{enumerate}
\begin{enumerate}
  \item \reponse{$u_n = \frac {(2n)!}{4^n(n!)^2}$.}
\end{enumerate}
}