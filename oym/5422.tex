\uuid{5422}
\titre{**}
\theme{Fonctions réelles d'une variable réelle dérivables (exclu études de fonctions)}
\auteur{rouget}
\date{2010/07/06}
\organisation{exo7}
\contenu{
  \texte{}
  \question{Soit $f$ de classe $C^1$ sur $\Rr_+^*$ telle que $\lim_{x\rightarrow +\infty}xf'(x)=1$. Montrer que $\lim_{x\rightarrow +\infty}f(x)=+\infty$.}
  \reponse{Puisque $\lim_{x\rightarrow +\infty}xf'(x)=1$, $\exists A>0/\;\forall x>0,\;(x\geq A\Rightarrow xf'(x)\geq\frac{1}{2})$.

Soit $x$ un réel fixé supérieur ou égal à $A$. $\forall t\in[A,x],\;f'(t)\geq\frac{1}{2x}$ et donc, par croissance de l'intégrale, $\int_{A}^{x}f'(t)\;dt\geq\int_{A}^{x}\frac{1}{2t}\;dt$ ce qui fournit~:

$$\forall x\geq A,\;f(x)\geq f(A)+\frac{1}{2}(\ln x-\ln A),$$

et montre que $\lim_{x\rightarrow +\infty}f(x)=+\infty$.}
}