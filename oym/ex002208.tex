\uuid{2208}
\titre{Exercice 2208}
\theme{}
\auteur{debes}
\date{2008/02/12}
\organisation{exo7}
\contenu{
  \texte{}
  \question{Soit $G$ un groupe d'ordre $399$. \smallskip

(a) Montrer que $G$ admet un unique $19$-Sylow $P$ qui est distingu\'e dans $G$.
\smallskip

(b) Soit $Q$ un $7$-Sylow. Montrer que $N=PQ$ est un sous-groupe d'ordre $133$    
de $G$ et que ce groupe est cyclique.
\smallskip

(c) On suppose que $Q$ n'est pas distingu\'e dans $G$. Montrer que $G$ admet $57$
sous-groupes cycliques d'ordre $133$ distincts deux \`a deux. Quel serait le nombre
d'\'el\'ements d'ordre $133$ dans $G$? Aboutir \`a  une contradiction. En d\'eduire
que $Q$ est distingu\'e dans $G$ et que $N$ est distingu\'e dans $G$.
\smallskip

(d) Montrer que $G=NR$, o\`u $R$ est un $3$-Sylow. En d\'eduire que $G$ est
isomorphe au produit semi-direct d'un groupe cyclique d'ordre $133$ par un groupe
cyclique d'ordre $3$.}
  \reponse{(a) Le nombre de $19$-Sylow de $G$ est $\equiv 1\
[\hbox{\rm mod}\ 19]$ et divise $21$; ce ne peut \^etre que $1$. Le groupe $G$ a
donc un unique $19$-Sylow $P$ qui est distingu\'e.
\smallskip

(b) Comme $P$ est distingu\'e dans $G$, $N=PQ$ est un sous-groupe de $G$. De
$P\cap Q=\{1\}$, on d\'eduit que $PQ/P \simeq Q$ et donc que $PQ$ est d'ordre
$7.19=133$. D'apr\`es l'exercice 15, le groupe $N$ est isomorphe au produit direct
$\Z/19\Z \times \Z/7\Z$, lequel est isomorphe au groupe cyclique $\Z/133\Z$ par le
lemme chinois.
\smallskip

(c) Le nombre de $7$-Sylow de $G$ est $\equiv 1\
[\hbox{\rm mod}\ 7]$ et divise $57$. Les seules possibilit\'es sont $1$ et $57$. Or
ce n'est pas $1$ non plus car on suppose que $Q$ n'est pas distingu\'e. Le groupe
$G$ admet donc $57$ $7$-Sylow, et donc $57$ sous-groupes cycliques d'ordre $133$
par la question pr\'ec\'edente. Ces $57$ groupes d'ordre $133$ sont bien distincts
car deux
$7$-Sylow distincts engendrent avec $P$ deux groupes cycliques d'ordre $133$
distincts puisque le $7$-Sylow est l'unique sous-groupe d'ordre $7$ du groupe
cyclique. Par cons\'equent leurs ensembles de g\'en\'erateurs sont deux \`a deux
disjoints. On obtient ainsi $57 \times \phi(133)= 57\times 6\times 18$ \'el\'ements
d'ordre
$133$ dans $G$ ($\phi$ d\'esigne ici la fonction indicatrice d'Euler), ce qui est
manifestement absurde. On peut donc conclure que $Q$ est distingu\'e dans $G$ et
que l'unique sous-groupe cyclique $N=PQ$ d'ordre $133$ l'est aussi.
\smallskip

(d) Comme $N$ est distingu\'e dans $G$, $NR$ est un sous-groupe de $G$. De
$N\cap R=\{1\}$, on d\'eduit que $NR/N \simeq R$ et donc que $NR$ est d'ordre
$133.3=399$. Ainsi $G=NR$ et l'isomorphisme pr\'ec\'edent $G/N \simeq R$ montre que
l'inclusion $R\rightarrow G$ est une section de la suite exacte $1\rightarrow
N\rightarrow G
\rightarrow R \rightarrow 1$. Le groupe $G$ est donc
isomorphe au produit semi-direct du groupe cyclique $N$ d'ordre $133$ par le groupe
cyclique $R$ d'ordre $3$.}
}