\uuid{7061}
\titre{Exercice 7061}
\theme{}
\auteur{megy}
\date{2017/01/11}
\organisation{exo7}
\contenu{
  \texte{}
  \question{% collège
% orthocentre, triangle rectangle inscrit 
% cercle circonscrit, hauteurs
Soit $ABC$ un triangle. Le cercle $\mathcal C$ (resp. $\mathcal C'$) de diamètre $[BC]$  (resp. $[CA]$) coupe la droite $(CA)$ (resp. la droite $(BC)$) en $P$ (resp. $Q$). Les cercles $\mathcal C$  et $\mathcal C'$ se recoupent en un second point $R$. Montrer que $(CR)$, $(BP)$ et $(AQ)$ sont concourantes.

\begin{center}
\includegraphics{../images/img007061-1}
\end{center}}
  \reponse{Un triangle dont un des côtés est un diamètre du cercle circonscrit est rectangle. On en déduit que les trois droites sont les hauteurs de $ABC$. Elles sont donc concourantes.

\begin{center}
\includegraphics{../images/img007061-1}
\end{center}}
}