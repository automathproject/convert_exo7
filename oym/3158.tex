\uuid{3158}
\titre{Carr{\'e}s dans $\Z/p\Z$}
\theme{Exercices de Michel Quercia, Propriétés de $\Zz/n\Zz$}
\auteur{quercia}
\date{2010/03/08}
\organisation{exo7}
\contenu{
  \texte{}
  \question{Soit $p$ un nombre premier impair.
Montrer que $\dot k$ est un carr{\'e} dans l'anneau $\Z/p\Z$ si et seulement si
$k^{(p+1)/2} \equiv k (\mathrm{mod}\, p)$.}
  \reponse{}
}