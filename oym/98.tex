\uuid{98}
\titre{Exercice 98}
\theme{}
\auteur{ridde}
\date{1999/11/01}
\organisation{exo7}
\contenu{
  \texte{}
  \question{Soit $ (a, b, c, d) \in \Rr^4$ tel que $ad-bc = 1$ et $c \neq 0$. Montrer que
si $z \neq -\dfrac dc$ alors $\Im (\dfrac{az + b}{cz + d}) = \dfrac{\Im (z)}
{ \left|(cz + d)\right|^2}$.}
  \reponse{}
}