\uuid{4699}
\titre{Polytechnique MP$^*$ 2000}
\theme{Exercices de Michel Quercia, Suites convergentes}
\auteur{quercia}
\date{2010/03/16}
\organisation{exo7}
\contenu{
  \texte{}
  \question{Soit $h$ croissante de~$\R^+$ dans $\R^+$, tendant vers~$+\infty$ en~$+\infty$,
et telle que $h(x+1)-h(x)$ tend vers~$0$ en~$+\infty$.
Soit~$V$ l'ensemble des valeurs d'adh{\'e}rence de la suite de terme g{\'e}n{\'e}ral~$e^{ih(n)}$
Montrer que $V$ est exactement le cercle trigonom{\'e}trique
(i.e. $\{z\in\C,\ |z|=1\}$).}
  \reponse{Si $e^{i\alpha}$ n'est pas valeur d'adh{\'e}rence alors il existe
$\delta>0$ tel que $|e^{ih(n)}-e^{i\alpha}|> \delta$ pour tout~$n$
assez grand donc l'ensemble $\bigcup_{k\in\N}[\alpha-\delta+2k\pi,\alpha-\delta+2k\pi]$
ne contient aucun terme de la suite $(h(n))$ pour $n$ assez grand ce qui
contredit les hypoth{\`e}ses $h(n)\xrightarrow[n\to\infty]{}+\infty$ et
$h(n+1)-h(n)\xrightarrow[n\to\infty]{}0$.}
}