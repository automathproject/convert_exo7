\uuid{NVdd}
\exo7id{5043}
\titre{Courbure de $M$ cste $ \Rightarrow $ courbure de $I$ cste}
\theme{Exercices de Michel Quercia, Courbes dans l'espace}
\auteur{quercia}
\date{2010/03/17}
\organisation{exo7}
\contenu{
  \texte{Soit $\mathcal{C}$ une courbe de l'espace, et $\Gamma$ la courbe décrite par le centre
de courbure, $I$, en un point $M$ de $\mathcal{C}$.
On suppose que la courbure de $\mathcal{C}$ est constante et sa torsion non nulle.}
\begin{enumerate}
  \item \question{Montrer que la courbure de $\Gamma$ est aussi constante.}
  \item \question{Chercher la torsion de $\Gamma$ en $I$ en fonction de la courbure et la
    torsion de $\mathcal{C}$ en $M$.}
\end{enumerate}
\begin{enumerate}
  \item \reponse{$\frac{d\vec I}{ds} = -\frac \tau c \vec B  \Rightarrow 
              \vec T_1 = \vec B$, $\frac {ds_1}{ds} = -\frac \tau c$,
              $\vec N_1 = -\vec N$, $c_1=c$.}
  \item \reponse{$\tau_1 = -\frac{c^2}{\tau}$.}
\end{enumerate}
}