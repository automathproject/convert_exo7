\uuid{X6c5}
\exo7id{4603}
\titre{Suite récurrente}
\theme{Exercices de Michel Quercia, Séries entières}
\auteur{quercia}
\date{2010/03/14}
\organisation{exo7}
\contenu{
  \texte{Soit $(a_n)$ la suite réelle définie par :
$a_0 = 1$, $2a_{n+1} = \sum_{k=0}^n C_n^k a_ka_{n-k}$.
On pose $f(x) = \sum_{n=0}^\infty \frac{a_n}{n!}x^n$.}
\begin{enumerate}
  \item \question{Montrer que le rayon de convergence est non nul.}
  \item \question{Calculer $f(x)$.}
  \item \question{En déduire $a_n$ en fonction de $n$.}
\end{enumerate}
\begin{enumerate}
  \item \reponse{$a_n \le n!$ par récurrence.}
  \item \reponse{$2f' = f^2  \Rightarrow  f(x) = \frac1{1-x/2}$.}
  \item \reponse{$a_n = n!\,2^{-n}$.}
\end{enumerate}
}