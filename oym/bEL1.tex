\uuid{bEL1}
\exo7id{574}
\titre{Exercice 574}
\theme{Suites, Limites}
\auteur{ridde}
\date{1999/11/01}
\organisation{exo7}
\contenu{
  \texte{Soit $n \geq 1$.}
\begin{enumerate}
  \item \question{Montrer que l'\'equation $\sum\limits_{k = 1}^n{x^k} = 1$ admet une unique solution,
notée $a_n$, dans $[0, 1]$.}
  \item \question{Montrer que $ (a_n)_{n \in \Nn}$ est d\'ecroissante minor\'ee par $\frac12$.}
  \item \question{Montrer que $ (a_n)$ converge vers $\frac 12$.}
\end{enumerate}
\begin{enumerate}
  \item \reponse{La fonction $f_n$ est continue
sur $[0,1]$. De plus $f_n(0) = -1 < 0$ et $f_n(1) = n-1\geqslant 0$.
D'apr\`es le th\'eor\`eme des valeurs interm\'ediaires, $f_n$, admet un
z\'ero dans l'intervalle $[0,1]$. De plus elle strictement
croissante (calculez sa d\'eriv\'ee) sur $[0,1]$ donc ce z\'ero est
unique.}
  \item \reponse{Calculons $f_n(a_{n-1})$.
\begin{align*}
f_n(a_{n-1}) &= \sum_{k=1}^{n} a_{n-1}^k  - 1 \\
     &= a_{n-1}^n + \sum_{k=1}^{n-1} a_{n-1}^k  - 1 \\
     &= a_{n-1}^n + f_{n-1}(a_{n-1})  \\
     &= a_{n-1}^n \text{\ \  (car $f_{n-1}(a_{n-1})=0$ par d\'efinition de $a_{n-1}$).}
\end{align*}

Nous obtenons l'in\'egalit\'e
$$ 0 = f_n(a_n) < f_n(a_{n-1}) = a_{n-1}^n.$$
Or $f_n$ est strictement croissante, l'in\'egalit\'e ci-dessus
implique donc $ a_n < a_{n-1}$.
Nous venons de d\'emontrer que la suite $(a_n)_n$ est d\'ecroissante.


Remarquons avant d'aller plus loin que $f_n(x)$ est la somme d'une
suite g\'eom\'etrique :
$$f_n(x) = \frac{1-x^{n+1}}{1-x}-2.$$

\'Evaluons maintenant $f_n(\frac12)$, \`a l'aide de l'expression
pr\'ec\'edente
$$f_n(\frac12) = \frac{1-(\frac12)^{n+1}}{1-{\frac12}}-2 = -\frac 1 {2^n} < 0.$$
Donc $ f_n(\frac12) < f_n(a_n)=0$ entra\^{\i}ne $\frac12 < a_n$.

Pour r\'esumer, nous avons montré que la suite $(a_n)_n$ est
strictement d\'ecroissante et minor\'ee par $\frac12$.}
  \item \reponse{Comme $(a_n)_n$ est
d\'ecroissante et minor\'ee par $\frac12$ alors elle converge, nous
notons $\ell$ sa limite :
$$ \frac 12 \leqslant \ell < a_n.$$
Appliquons $f_n$ (qui est strictement croissante) \`a cette
in\'egalit\'e :
 $$ f_n\left(\frac 12\right) \leqslant f_n(\ell) < f_n(a_n),$$
qui s'\'ecrit aussi :
$$ -\frac 1 {2^n} \leqslant f_n(\ell) < 0,$$
et ceci quelque soit $n\geqslant 1$. La suite $(f_n(\ell))_n$ converge
donc vers $0$ (th\'eor\`eme des  ``gendarmes''). Mais nous savons
aussi que
$$f_n(\ell) = \frac{1-\ell^{n+1}}{1-\ell}-2 ;$$
donc $(f_n(\ell))_n$ converge vers $\frac{1}{1-\ell}-2$ car
$(\ell^n)_n$ converge vers $0$. Donc
$$\frac{1}{1-\ell}-2 = 0, \text{ d'o\`u } \ell = \frac12.$$}
\end{enumerate}
}