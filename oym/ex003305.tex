\exo7id{3305}
\titre{Supplémentaire commun, X MP$^*$ 2005}
\theme{}
\auteur{quercia}
\date{2010/03/09}
\organisation{exo7}
\contenu{
  \texte{}
\begin{enumerate}
  \item \question{Soit $A = \{P\in\R[X]$ tq $P=(1-X)Q(X^2)$ avec $Q\in\R[X]\}$.
  \begin{enumerate}}
  \item \question{Montrer que $A$ est un $\R$-ev et que l'on a
    $R[X] = A \oplus \{$ polynômes pairs $\}$.

    A-t-on $R[X] = A \oplus \{$ polynômes impairs $\}$~?}
  \item \question{Que peut-on dire si l'on remplace $Q(X^2)$ par une fonction $f$ paire~?}
\end{enumerate}
\begin{enumerate}
  \item \reponse{\begin{enumerate}}
  \item \reponse{Soit $P\in\R[X]$ que l'on décompose en $P=P_1(X^2) + XP_2(X^2)$ .
    Alors $P = (P_1+P_2)(X^2) - (1-X)P_2(X^2) = (1-X)P_1(X^2) + X(P_1+P_2)(X^2)$,
    ce qui prouve que les deux sommes sont égales à $\R[X]$. Ces sommes
    sont facilement directes.}
  \item \reponse{Cela ne change pas~$A$~: les éléments de~$A$ sont ceux dont les parties
    paire et impaire sont opposées (au facteur $X$ près), in\-dé\-pen\-da\-ment du fait (vrai) que ces parties sont
    des polynômes.}
\end{enumerate}
}