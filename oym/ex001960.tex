\uuid{1960}
\titre{Exercice 1960}
\theme{}
\auteur{liousse}
\date{2003/10/01}
\organisation{exo7}
\contenu{
  \texte{On considère le triangle 
$ABC$ dont les c\^otés ont pour équations
$(AB) : x+2y=3 , (AC) : x+y=2 , (BC) : 2x+3y=4$.}
\begin{enumerate}
  \item \question{Donner les coordonnées des points $A, B, C$.}
  \item \question{Donner les coordonnées des milieux $A', B', C'$ des segments $[BC]$, $[AC]$ et $[AB]$ respectivement.}
  \item \question{Donner une équation de chaque médiane et vérifier qu'elles sont concourantes.}
\end{enumerate}
\begin{enumerate}
  \item \reponse{Le point $A$ est l'intersection des droites $(AB)$ et $(AC)$.
Les coordonnées $(x,y)$ de $A$ sont donc solutions du système :
$\left\{ \begin{array}{l}
x+2y=3 \\ x+y=2 \\ 
\end{array} \right.$ donné par 
les équations des deux droites.
La seule solution est $(x,y)=(1,1)$.
On a donc $A=(1,1)$. On fait de même pour obtenir le point $B=(-1,2)$ et $C=(2,0)$.}
  \item \reponse{Notons $A'$ le milieu de $[BC]$ alors les coordonnées se trouvent par la formule suivante
$A'= (\frac{x_B+x_C}{2},\frac{y_B+y_C}{2}) = (\frac12,1)$.
De même on trouve $B'=(\frac32,\frac12)$ et $C'=(0,\frac32)$.}
  \item \reponse{\begin{enumerate}}
  \item \reponse{Les médianes ont pour équations : 
$(AA') : (y=1)$ ; $(BB') : (3x+5y=7)$ ; $(CC') : (3x+4y=6)$.}
  \item \reponse{Vérifions que les trois médianes sont concourantes (ce qui est vrai quelque soit le triangle).
On calcule d'abord l'intersection $I=(AA')\cap (BB')$, les coordonnées du point $I$ d'intersection vérifient 
donc le système 
$\left\{ \begin{array}{l}
y=1 \\ 3x+5y=7 \\ 
\end{array} \right.$. On trouve $I=(\frac23,1)$.

Il ne reste plus qu'à vérifier que $I$ appartient à la droite $(CC')$ d'équation $3x+4y=6$.
En effet $3x_I+4y_I=6$ donc $I \in (CC')$.

Conclusion : les médianes sont concourantes au point $I=(\frac23,1)$.}
\end{enumerate}
}