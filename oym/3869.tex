\uuid{3869}
\titre{$f(x+y)f(x-y) = f^2(x)f^2(y)$}
\theme{Exercices de Michel Quercia, Fonctions continues}
\auteur{quercia}
\date{2010/03/11}
\organisation{exo7}
\contenu{
  \texte{}
  \question{Trouver toutes les fonctions $f : {\R} \to {\R}$ continues telles
que : $\forall\ x,y \in \R,\ f(x+y)f(x-y) = f^2(x)f^2(y)$.}
  \reponse{Si $f$ n'est pas identiquement nulle, alors $f(0) = \pm1$ et $f$ est paire,
de signe constant.

Par récurrence, $\forall\ p \in \N,\ f(px) = \pm f^{p^2}(x)$
$ \Rightarrow $ par densité, $f(x) = \pm\lambda^{x^2}$.}
}