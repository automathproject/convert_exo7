\uuid{3554}
\titre{$A^p = I$ et $\mathrm{spec}(A) \subset \R  \Rightarrow  A^2 = I$}
\theme{Exercices de Michel Quercia, Réductions des endomorphismes}
\auteur{quercia}
\date{2010/03/10}
\organisation{exo7}
\contenu{
  \texte{}
  \question{Soit $A \in \mathcal{M}_n(\R)$. On suppose que les valeurs propres de $A$ sont réelles
et qu'il existe $p \ge 1$ tel que $A^p = I$.
Montrer que $A^2 = I$.}
  \reponse{$A$ est $\C$-diagonalisable (polynôme annulateur à racines simples)
$ \Rightarrow  \dim(E_1) + \dim(E_{-1}) = n$.
Les dimensions sont conservées sur $\R$.}
}