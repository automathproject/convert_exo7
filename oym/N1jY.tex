\uuid{N1jY}
\exo7id{4543}
\titre{Ensi PC 1999}
\theme{Exercices de Michel Quercia, Suites et séries de fonctions}
\auteur{quercia}
\date{2010/03/14}
\organisation{exo7}
\contenu{
  \texte{Soit $f_n(x) = \frac{(-1)^n\cos^nx}{n+1}$.}
\begin{enumerate}
  \item \question{\'Etudier la convergence de $f(x) = \sum_{n=0}^\infty f_n(x)$.}
  \item \question{Montrer la convergence de la série de terme général
    $u_n =  \int_{x=0}^{\pi/2} f_n(x)\,d x$.}
  \item \question{En déduire $\sum_{n=0}^\infty u_n$ sous forme d'une intégrale.}
\end{enumerate}
\begin{enumerate}
  \item \reponse{cva si $|\cos x| < 1$, scv si $\cos x = 1$, dv si $\cos x = -1$.}
  \item \reponse{TCM en regroupant les termes deux par deux.}
  \item \reponse{$ \int_{x=0}^{\pi/2} -\ln(1-\cos x)\,d x$.}
\end{enumerate}
}