\uuid{3654}
\titre{Somme directe dans $E$ $ \Rightarrow $ somme directe dans $\mathcal{L}(E)$}
\theme{}
\auteur{quercia}
\date{2010/03/10}
\organisation{exo7}
\contenu{
  \texte{Soit $E$ un $ K$-ev de dimension finie $n$ et
${\cal B} = ({\vec e}_1,\dots,{\vec e}_n)$ une base de $E$.

On note $F_i = \{ u \in \mathcal{L}(E)$ tq $\Im u \subset \text{vect}({\vec e}_i)\}$.}
\begin{enumerate}
  \item \question{Caractériser matriciellement les éléments de $F_i$.}
  \item \question{Montrer que $F_1 \oplus F_2 \oplus \dots \oplus F_n = \mathcal{L}(E)$.}
\end{enumerate}
\begin{enumerate}

\end{enumerate}
}