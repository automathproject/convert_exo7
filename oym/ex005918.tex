\uuid{5918}
\titre{Exercice 5918}
\theme{}
\auteur{tumpach}
\date{2010/11/11}
\organisation{exo7}
\contenu{
  \texte{}
  \question{Montrer qu'une fonction \emph{monotone} sur $[a,b]$ est
Riemann-int\'egrable sur $[a,b]$.}
  \reponse{Soit $f$ une fonction croissante $[a,b]$. Pour montrer que $f$ est
Riemann-int\'egrable, il suffit de trouver, pour tout
$\varepsilon>0$ donn\'e, une subdivision de $[a, b]$ telle que
$\overline{S}_{f}^{\sigma} - \underline{S}_{f}^{\sigma} <
\varepsilon$. Soit $\sigma=\{a_{0}=a < \dots < a_{n} = b\}$ la
subdivision r\'eguli\`ere de $[a,b]$, de pas
$\left(\frac{b-a}{n}\right)$. On a
\begin{equation*}
\inf_{]a_{k-1}, a_{k}[} f = f(a_{k-1}) \quad \text{et} \quad
\sup_{]a_{k-1}, a_{k}[} f = f(a_{k}).
\end{equation*}
Ainsi~:
\begin{eqnarray*}
\overline{S}_{f}^{\sigma} - \underline{S}_{f}^{\sigma} &=&
\sum_{k=1}^{n} (a_{k} - a_{k-1})\left( f(a_{k}) - f(a_{k-1})
\right) \\ &= &\left(\frac{b-a}{n}\right) \sum_{k=1}^{n}\left(
f(a_{k}) - f(a_{k-1}) \right)\\& =&
\left(\frac{b-a}{n}\right)\left(f(b)- f(a)\right).
\end{eqnarray*}
Pour $n$ assez grand, la subdivision r\'eguli\`ere de $[a,b]$
satisfait $\overline{S}_{f}^{\sigma} - \underline{S}_{f}^{\sigma}
< \varepsilon$. D'autre part, si $g$ est d\'ecroissante, $f = -g$
est croissante, donc $g$ est Riemann-int\'egrable par l'exercice
pr\'ec\'edent (question 2.) avec $\lambda = -1$.}
}