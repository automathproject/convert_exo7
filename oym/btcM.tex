\uuid{btcM}
\exo7id{4479}
\titre{Développement factoriel}
\theme{Exercices de Michel Quercia, Séries numérique}
\auteur{quercia}
\date{2010/03/14}
\organisation{exo7}
\contenu{
  \texte{Soit $\cal S$ l'ensemble des suites croissantes d'entiers $(q_i)$ telles que $q_0\ge 2$.}
\begin{enumerate}
  \item \question{Si $s = (q_i) \in \cal S$, montrer que la série
$\sum_{k=0}^\infty \frac1{q_0\dots q_k}$ converge. On note $\Phi(s)$ sa somme.}
  \item \question{Montrer que l'application $\Phi : {\cal S} \to {]0,1]}$ est bijective.}
  \item \question{Soit $s = (q_i) \in \cal S$. Montrer que $\Phi(s) \in\Q$ si et seulement
    si $s$ est stationnaire.}
\end{enumerate}
\begin{enumerate}

\end{enumerate}
}