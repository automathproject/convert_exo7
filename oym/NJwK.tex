\uuid{NJwK}
\exo7id{2252}
\titre{Exercice 2252}
\theme{Anneaux et idéaux}
\auteur{barraud}
\date{2008/04/24}
\organisation{exo7}
\contenu{
  \texte{}
\begin{enumerate}
  \item \question{Si $x\cdot y$ est inversible dans un anneau $A$,
alors $x$ et $y$ sont inversibles.}
  \item \question{Dans un anneau, un \'el\'ement inversible n'est pas diviseur de z\'ero
et un diviseur de z\'ero n'est pas inversible.}
\end{enumerate}
\begin{enumerate}
  \item \reponse{Si $xy\in A^{\times}$, soit $z\in A, (xy)z=1$. Alors $x(yz)=1$ et
    $(zx)y=1$ donc $x$ et $y$ sont inversibles.}
  \item \reponse{Soit $x\in A^{\times}$, et $y\in A, xy=0$. Alors $x^{-1}xy=y=0$.
     Donc $x$ n'est pas diviseur de $0$.}
\end{enumerate}
}