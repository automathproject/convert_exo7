\uuid{6877}
\titre{Exercice 6877}
\theme{}
\auteur{gammella}
\date{2012/05/29}
\organisation{exo7}
\contenu{
  \texte{}
  \question{Quel est le champ vectoriel qui dérive du potentiel
$$U(x,y,z)=1+x+xy+xyz ?$$}
  \reponse{Le champ vectoriel qui dérive du potentiel $U$ est
$$\vec{\mathrm{grad}}(U)= (  \frac{\partial u}{\partial x},   \frac{\partial u}{\partial y},  \frac{\partial u}{\partial z}).$$ 
Il s'agit donc du champ vectoriel de composantes :
$$\vec{\mathrm{grad}}(U)=(1+y+yz, x+xz, xy).$$}
}