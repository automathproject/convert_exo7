\uuid{5505}
\titre{*T}
\theme{}
\auteur{rouget}
\date{2010/07/15}
\organisation{exo7}
\contenu{
  \texte{}
  \question{Dans $\Rr^3$ affine, déterminer un repère de la droite $(D)$ $\left\{
\begin{array}{l}
x-y+2z+7=0\\
2x+2y+3z-5=0
\end{array}
\right.$.}
  \reponse{Puisque $\left|
\begin{array}{cc}
1&2\\
2&3
\end{array}\right|=-1$, on choisit d'exprimer $x$ et $z$ en fonction de $y$.
Soit $M(x,y,z)\in\Rr^3$. D'après les formules de \textsc{Cramer}, on a
\begin{align*}\ensuremath
M\in(D)&\Leftrightarrow\left\{
\begin{array}{l}
x-y+2z+7=0\\
2x+2y+3z-5=0
\end{array}
\right.\Leftrightarrow\left\{
\begin{array}{l}
x+2z=y-7\\
-2x-3z=2y-5
\end{array}
\right.\\
 &\Leftrightarrow x=\frac{1}{1}\left|
\begin{array}{cc}
y-7&2\\
2y-5&-3
\end{array}
\right|\;\text{et}\;z=\frac{1}{1}\left|
\begin{array}{cc}
1&y-7\\
-2&2y-5
\end{array}
\right|\\
 &\Leftrightarrow\left\{\begin{array}{l}
 x=31-7y\\
 z=-19+4y
 \end{array}\right..
\end{align*}

\begin{center}
\shadowbox{
$(D)$ est la droite passant par $A(31,0,-19)$ dirigée par le vecteur $u(-7,1,4)$.
}
\end{center}}
}