\uuid{vvzH}
\exo7id{3122}
\titre{pgcd$( a^n - 1, a^m - 1 )$}
\theme{Exercices de Michel Quercia, Pgcd}
\auteur{quercia}
\date{2010/03/08}
\organisation{exo7}
\contenu{
  \texte{Soient $a,m,n\ \in \N^{*}$, $a\ge 2$, et $d = (a^n - 1) \wedge (a^m - 1)$.}
\begin{enumerate}
  \item \question{Soit $n = qm + r$ la division euclidienne de $n$ par $m$.
      D{\'e}montrer que $a^n \equiv a^r (\mathrm{mod}\, {a^m - 1})$.}
  \item \question{En d{\'e}duire que $d = (a^r - 1) \wedge (a^m - 1)$,
      puis $d = a^{(n \wedge m)} - 1$.}
  \item \question{A quelle condition $a^m - 1$ divise-t-il $a^n-1$ ?}
\end{enumerate}
\begin{enumerate}
  \item \reponse{$a^m - 1\mid (a^{qm} - 1)a^r = a^n - a^r$.}
  \item \reponse{$A\wedge(AQ+R) = A\wedge R$.
      Algorithme d'Euclide sur les exposants de $a$.}
  \item \reponse{ssi $m\mid n$.}
\end{enumerate}
}