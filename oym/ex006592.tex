\uuid{6592}
\titre{Exercice 6592}
\theme{}
\auteur{hueb}
\date{2011/10/16}
\organisation{exo7}
\contenu{
  \texte{Localiser les
singularités de chacune des fonctions suivantes et les
caractériser.}
\begin{enumerate}
  \item \question{$\frac {z^2}{(z+1)^3}$,}
  \item \question{$\frac {2z^3-z+1}{(z-4)^2 (z-i)(z-1+2i)}$,}
  \item \question{$\frac {\sin mz}{z^2 +2z +2}$,}
  \item \question{$\frac {1-\cos z}z$,}
  \item \question{$e^{-\frac 1{(z-1)^2}}$,}
\end{enumerate}
\begin{enumerate}

\end{enumerate}
}