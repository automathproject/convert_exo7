\uuid{3730}
\titre{Polytechnique MP$^*$ 2000}
\theme{Exercices de Michel Quercia, Formes quadratiques}
\auteur{quercia}
\date{2010/03/11}
\organisation{exo7}
\contenu{
  \texte{}
  \question{On considère sur $\R^n$ la forme quadratique : $q(x)=\alpha \| x\| ^2 +\beta (x|a)^2$ où
$\alpha >0$, $\beta $ réel et $a\in \R^n$.
Discuter de la signature et du rang de $q$.}
  \reponse{Pour $a=0$, $q$ est définie positive.
Pour $a\ne 0$ prendre une base orthonormale commen\c cant par~$a$~;
la matrice de~$q$ dans cette base est $\mathrm{Diag}(\alpha+\beta\|a\|^2,\alpha,\dots,\alpha)$.}
}