\uuid{fB13}
\exo7id{4813}
\titre{Applications lin{\'e}aires sur les polyn{\^o}mes}
\theme{Exercices de Michel Quercia, Topologie dans les espaces vectoriels normés}
\auteur{quercia}
\date{2010/03/16}
\organisation{exo7}
\contenu{
  \texte{Soit $E = \R[x]$ muni de la norme :
$\bigl\|\sum\limits_i a_ix^i\bigr\| = \sum\limits_i|a_i|$.}
\begin{enumerate}
  \item \question{Est-ce que $\varphi : P  \mapsto P(x+1)$ est continue ?}
  \item \question{Est-ce que $\psi : P  \mapsto AP$ est continue ? ($A \in E$ fix{\'e})}
  \item \question{Reprendre les questions pr{\'e}c{\'e}dentes avec la norme :
    $\|P\| = \sup\{e^{-|t|}|P(t)|,\ t\in \R\}$.}
\end{enumerate}
\begin{enumerate}
  \item \reponse{Non : $\|(x^2+1)^n\| = 2^n$.}
  \item \reponse{Oui : $\|\psi\| = \|A\|$.}
  \item \reponse{$\|\phi\| = e$, $\|\psi(x^n)\|/\|x^n\| \text{ si }m n  \Rightarrow  \psi$
             est discontinue.}
\end{enumerate}
}