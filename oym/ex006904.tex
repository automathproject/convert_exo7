\exo7id{6904}
\titre{Exercice 6904}
\theme{}
\auteur{ruette}
\date{2013/01/24}
\organisation{exo7}
\contenu{
  \texte{L'entreprise Luminex fabrique des lampes, dont 80\% durent plus de 3000 heures. 
Des tests sont effectués sur des échantillons de taille $n = 15$.}
\begin{enumerate}
  \item \question{Quelle est le nombre moyen de lampes qui ont une durée de vie inférieure à  3000 heures dans un échantillon de taille 15 ?}
  \item \question{Quelle est la probabilité que toutes les lampes de l'échantillon durent plus de 3000 heures ?}
  \item \question{Quelle est la probabilité que 13 lampes ou plus, dans un échantillon de taille 15, durent plus de 3000 heures ?}
\end{enumerate}
\begin{enumerate}
  \item \reponse{Une lampe tirée au hasard a une probabilité de $0,2$ d'avoir une durée de vie 
inférieure à 3000 heures. Le nombre $X$ de lampes qui ont une durée de vie 
inférieure à 3000 heures dans un échantillon de taille 15 tiré au hasard est 
la somme de 15 variables de Bernoulli de paramètre $p=0,2$. Par conséquent, 
il suit une loi binomiale $B(15;0,2)$. Son espérance vaut $E(X)=15\times 0,2=3$.}
  \item \reponse{C'est $p(X=0)={15\choose{0}}(0,2)^0 (0,8)^{15}\sim 0,0352$.}
  \item \reponse{C'est $p(X=0)+p(X=1)+p(X=2)$
\begin{eqnarray*}
&=&{15\choose{0}}(0,2)^0 (0,8)^{15}+{15\choose{1}}(0,2)^1 (0,8)^{14}+
{15\choose{2}}(0,2)^2 (0,8)^{13}\\&=&0,0352+0,1319+0,2309\simeq 0,398.
\end{eqnarray*}}
\end{enumerate}
}