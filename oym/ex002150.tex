\uuid{2150}
\titre{Exercice 2150}
\theme{}
\auteur{debes}
\date{2008/02/12}
\organisation{exo7}
\contenu{
  \texte{}
  \question{Soit $f$ un morphisme de groupes $f : \Q  \rightarrow \Q ^{\times}
_{>0}$, $\Q$ \'etant muni de l'addition et $\Q ^{\times} _{>0}$ muni de la
multiplication. Calculer $f(n)$ en fonction de $f(1)$ pour tout entier $n>0$. Montrer
que les deux groupes pr\'ec\'edents ne sont pas isomorphes.}
  \reponse{On a $f(n)=f(1)^n$ pour tout entier $n>0$. Mais on a aussi $f(1/n)^n = f(1)$ pour tout
$n>0$. Cela n'est pas possible car un nombre rationnel positif $\not=0,1$ ne peut \^etre une
puissance $n$-i\`eme dans $\Q$ pour tout $n>0$. (Pour ce dernier point, noter
par exemple qu'\^etre une puissance $n$-i\`eme dans $\Q$ entra\^\i ne que tous les
exposants de la d\'ecomposition en facteurs premiers sont des multiples de $n$). 
Les deux groupes $(\Q,+)$ et $(\Q_+^\times,\times)$ ne sont donc pas isomorphes.}
}