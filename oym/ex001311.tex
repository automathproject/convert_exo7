\uuid{1311}
\titre{Exercice 1311}
\theme{}
\auteur{ortiz}
\date{1999/04/01}
\organisation{exo7}
\contenu{
  \texte{Les ensembles suivants, pour les lois
consid\'er\'ees, sont-ils des groupes ?}
\begin{enumerate}
  \item \question{$]-1,1[$ muni de la loi d\'efinie par $x\star y=\frac{x+y}{1+xy}$ ;}
  \item \question{$\{z\in\Cc:{|z|}=2\}$ pour la multiplication usuelle ;}
  \item \question{$\Rr_+$ pour la multiplication usuelle;}
  \item \question{$\left\{x\in\Rr\mapsto ax+b : a\in\Rr\setminus\left\{0\right\},b\in\Rr\right\}$
pour la loi de composition des applications.}
\end{enumerate}
\begin{enumerate}
  \item \reponse{Oui.}
  \item \reponse{Non. Le seul élément qui peut être l'élément neutre
  est $1$ qui n'appartient pas à l'ensemble.}
  \item \reponse{Non. $0$ n'a pas d'inverse.}
  \item \reponse{Oui.}
\end{enumerate}
}