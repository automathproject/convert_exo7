\uuid{4342}
\titre{Fonction définie par une intégrale}
\theme{Exercices de Michel Quercia, Intégrale dépendant d'un paramètre}
\auteur{quercia}
\date{2010/03/12}
\organisation{exo7}
\contenu{
  \texte{}
  \question{On pose pour $x \ge 0$ :
$f(x) =  \int_{t=0}^{+\infty} \frac{\ln(x^2+t^2)}{1+t^2}\,d t$.

Calculer explicitement $f'(x)$ et en déduire $f(x)$
(on calculera $f(0)$ à
l'aide du changement de variable $u = 1/t$).}
  \reponse{$f'(x) = \frac\pi{x+1}$, $f(x) = \pi\ln(x+1)$.}
}