\uuid{4404}
\titre{$\lim((1+z/n)^n)$}
\theme{Exercices de Michel Quercia, Fonction exponentielle complexe}
\auteur{quercia}
\date{2010/03/12}
\organisation{exo7}
\contenu{
  \texte{}
  \question{Soit $z\in\C$. Montrer que $\left(1+\frac zn\right)^n \to e^z$ lorsque $n\to\infty$.}
  \reponse{Mettre $1+\frac zn$ sous forme trigonométrique.}
}