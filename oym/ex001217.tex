\uuid{1217}
\titre{Exercice 1217}
\theme{}
\auteur{ridde}
\date{1999/11/01}
\organisation{exo7}
\contenu{
  \texte{Calculer les limites de}
\begin{enumerate}
  \item \question{$\dfrac{\sin x \ln (1 + x^2)}{x \tan x} \text{ en } 0$.}
  \item \question{$\dfrac{\ln (1 + \sin x)}{\tan (6x)} \text{ en } 0$.}
  \item \question{$ (\ln (e + x))^{\frac 1x} \text{ en } 0$.}
  \item \question{$(\ln (1 + e^{-x}))^{\frac 1x} \text{ en }  + \infty$.}
\end{enumerate}
\begin{enumerate}
  \item \reponse{$\displaystyle{\lim _{x\rightarrow 0}{\frac
{\sin(x)\ln (1+{x}^{2})}{x\tan(x)}}=0 }$.}
  \item \reponse{$\displaystyle{\lim _{x\rightarrow 0}{\frac {\ln (1+\sin(x))}{\tan(6\,x)}}=1/6 }$.}
  \item \reponse{$\displaystyle{\lim _{x\rightarrow 0}\left (\ln ({e}+x)\right )^{{x}^{-1}}={e^{{e^{-1}}}} }$.}
  \item \reponse{$\displaystyle{\lim _{x\rightarrow \infty }\left (\ln (1+{e^{-x}})\right )^{{x}^{-1}}={e^{-1}} }$.}
\end{enumerate}
}