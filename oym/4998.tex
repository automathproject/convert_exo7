\uuid{4998}
\titre{Ensi Chimie P' 93}
\theme{Exercices de Michel Quercia, Courbes définies par une condition}
\auteur{quercia}
\date{2010/03/17}
\organisation{exo7}
\contenu{
  \texte{}
  \question{$$
\includegraphics[width=4cm]{../images/img004998-1}
$$
 Trouver les courbes $\mathcal{C}$ telles que $MN = ON$.}
  \reponse{La tangente ne doit pas être parallèle à $Oy$, donc on peut paramétrer
$\mathcal{C}$ sous la forme : $y=f(x)$, ce qui donne l'équation :
$$|x+yy'| = |y|\sqrt{1+y'^2} \Leftrightarrow
2xyy' = y^2-x^2.$$
(équation homogène) on obtient : $y = \pm\sqrt{\lambda x - x^2}$. Les courbes
cherchées sont des arcs de cercles centrés sur $Ox$ passant par $O$.}
}