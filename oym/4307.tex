\uuid{4307}
\titre{Ensi PC 1999}
\theme{Exercices de Michel Quercia, Intégrale généralisée}
\auteur{quercia}
\date{2010/03/12}
\organisation{exo7}
\contenu{
  \texte{}
  \question{Soient $I =  \int_{u=0}^{+\infty} \frac{d u}{(1+u^2)(1+u^n)}$
et $J =  \int_{u=0}^{+\infty} \frac{u^n\,d u}{(1+u^2)(1+u^n)}$
($n\in\N$).


Prouver que ces intégrales convergent, qu'elles sont égales et les calculer.}
  \reponse{$I=J$ par changement $u \mapsto1/u$.\par
$I+J = \frac\pi2  \Rightarrow  I = J = \frac\pi4$.}
}