\uuid{3809}
\titre{Calcul d'inverse}
\theme{Exercices de Michel Quercia, Problèmes matriciels}
\auteur{quercia}
\date{2010/03/11}
\organisation{exo7}
\contenu{
  \texte{}
  \question{Soit $A = \begin{pmatrix} a &-b &-c &-d \cr
                    b & a & d &-c \cr
                    c &-d & a & b \cr
                    d & c &-b & a \cr \end{pmatrix} \in \mathcal{M}_4(\R)$,
avec $a,b,c,d$ non tous nuls.

Démontrer que $A$ est inversible et calculer $A^{-1}$.}
  \reponse{$A^tA = (a^2+b^2+c^2+d^2)I$.}
}