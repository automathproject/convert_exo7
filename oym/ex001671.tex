\uuid{1671}
\titre{Exercice 1671}
\theme{}
\auteur{roussel}
\date{2001/09/01}
\organisation{exo7}
\contenu{
  \texte{Soit $B$ une matrice diagonalisable ~de $\mathcal{M}_n \left (\mathbb{R} \right )$.
On d\'efinit son rayon spectral par
$$\rho(B)= \max \left \{ |\lambda| \mbox{~~avec~~} \lambda \mbox{~~est une
valeur propre ~de~~} B \right \}.$$}
\begin{enumerate}
  \item \question{Montrer que $\lim _{k \longrightarrow + \infty} B^k= 0$.}
  \item \question{En d\'eduire que $I-B$ est inversible et que $(I-B)^{-1}=
\displaystyle{\sum _{k=0}^{+ \infty}} B^{k}$.}
\end{enumerate}
\begin{enumerate}

\end{enumerate}
}