\uuid{397}
\titre{Exercice 397}
\theme{}
\auteur{legall}
\date{2003/10/01}
\organisation{exo7}
\contenu{
  \texte{}
  \question{Soit $P(X)= a_nX^n+\cdots + a_0$ un polyn\^ome \`a 
coefficients entiers premiers entre eux (c'est \`a dire tels que les 
seuls diviseurs
communs \`a tous les $a_i$ soient $-1$ et $1$). Montrer que si 
$r=\dfrac{p}{q}$ avec $p$ et $q$ premiers entre eux est une racine 
rationnelle de $P$
alors $p$ divise $a_0$ et $q$ divise $a_n.$}
  \reponse{}
}