\exo7id{262}
\titre{Exercice 262}
\theme{}
\auteur{cousquer}
\date{2003/10/01}
\organisation{exo7}
\contenu{
  \texte{Les nombres $a$, $b$, $c$, $d$ étant des éléments non nuls de $\mathbb{Z}$, 
dire si les propriétés suivantes sont vraies ou 
fausses, en justifiant la réponse.}
\begin{enumerate}
  \item \question{Si $a$ divise $b$ et $c$, alors $c^2-2b$ est multiple de~$a$.}
  \item \question{S'il existe $u$ et $v$ entiers tels que $au+bv=d$ alors 
$\mbox{pgcd}(a,b)=\vert d\vert$.}
  \item \question{Si $a$ est premier avec $b$, alors $a$ est premier avec $b^3$.}
  \item \question{Si $a$ divise $b+c$ et $b-c$, alors $a$ divise $b$ et $a$ divise 
$c$.}
  \item \question{Si $19$ divise $ab$, alors $19$ divise $a$ ou $19$ divise $b$.}
  \item \question{Si $a$ est multiple de~$b$ et si $c$ est multiple de~$d$, alors 
$a+c$ est multiple de $b+d$.}
  \item \question{Si $4$ ne divise pas $bc$, alors $b$ ou $c$ est impair.}
  \item \question{Si $a$ divise $b$ et $b$ ne divise pas $c$, alors $a$ ne divise 
pas $c$.}
  \item \question{Si $5$ divise $b^2$, alors $25$ divise $b^2$.}
  \item \question{Si $12$ divise $b^2$, alors $4$ divise $b$.}
  \item \question{Si $12$ divise $b^2$, alors $36$ divise $b^2$.}
  \item \question{Si $91$ divise $ab$, alors $91$ divise $a$ ou $91$ divise $b$.}
\end{enumerate}
\begin{enumerate}

\end{enumerate}
}