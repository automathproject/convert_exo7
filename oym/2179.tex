\uuid{2179}
\titre{Exercice 2179}
\theme{Action de groupe}
\auteur{debes}
\date{2008/02/12}
\organisation{exo7}
\contenu{
  \texte{}
  \question{Soit $G$ un groupe et $H$ un sous-groupe d'indice fini $n$. Montrer que
l'intersection $H^\prime$ des conjugu\'es de $H$ par les \'el\'ements de $G$ est un
sous-groupe distingu\'e de $G$ et d'indice fini dans $G$. Montrer que c'est le plus
grand sous-groupe distingu\'e de $G$ contenu dans $H$.}
  \reponse{}
}