\uuid{5253}
\titre{***I}
\theme{}
\auteur{rouget}
\date{2010/07/04}
\organisation{exo7}
\contenu{
  \texte{}
  \question{Montrer que  $\sum_{k=0}^{n}k!\sim n!$.}
  \reponse{Pour $n\geq2$, on a

$$\frac{1}{n!}\sum_{k=0}^{n}k!=1+\frac{1}{n}+\sum_{k=0}^{n-2}\frac{k!}{n!}.$$

Mais, pour $0\leq k\leq n-2$, $\frac{k!}{n!}=\frac{1}{n(n-1)...(k+1)}\leq\frac{1}{n(n-1)}$ (le produit contenant au moins les deux premiers facteurs. Par suite,

$$0\leq\sum_{k=0}^{n-2}\frac{k!}{n!}\leq\frac{n-2}{n(n-1)}.$$

On en déduit que $\sum_{k=0}^{n-2}\frac{k!}{n!}$ tend vers $0$ quand $n$ tend vers $+\infty$. Comme $\frac{1}{n}$ tend aussi vers $0$ quand $n$ tend vers $+\infty$, on en déduit que $\frac{1}{n!}\sum_{k=0}^{n}k!$ tend vers $1$ et donc que

$$\sum_{k=0}^{n}k!\sim n!.$$}
}