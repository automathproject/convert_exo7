\uuid{1672}
\titre{Endomorphisme diagonalisable de $\Rr^2$}
\theme{}
\auteur{gineste}
\date{2001/11/01}
\organisation{exo7}
\contenu{
  \texte{}
  \question{On
considère l'endomorphisme $a$ de $E=\Rr^2$ dont la matrice
représentative $A=[a]_e^e$ dans la base canonique $e$ est $\left[
\begin{array}{ll}7&-10\\5&-8 \end{array} \right].$ Calculer la
trace, le déterminant, le polynôme caractéristique et le spectre
de $a$. Quel théorème du cours garantit l'existence d'une base
$f=(\vec{f}_1,\vec{f}_2)$ de vecteurs propres? Choisir ensuite $f$
telle que $[\mathrm{id}_E]_f^e$ et $[\mathrm{id}_E]_e^f$ soient à
coefficients entiers. Dessiner $\vec{f}_1$ et $\vec{f}_2$, en
prenant des unités d'axes assez petites. Dessiner quelques
vecteurs $\vec{x}$ et leurs images $a(\vec{x})$ à l'aide de $f$.

Trouver deux matrices $P$ et $D$ carrées d'ordre 2 telles que $D$ soit
diagonale, $P$ inversible et $A=PDP^{-1}$. Calculer $[a^{50}]_f^f$,
$[a^{50}]_e^e$ et $A^{50}$. Calculer $\lim_{n\infty}\frac{1}{3^{2n}}a^{2n}.$}
  \reponse{$\mathrm{tr} a=\mathrm{tr} A=-1,$ $\det a=\det A=-6$
$$P_a(X)=X^2-\mathrm{tr} X+\det a=X^2+X-6=(X-2)(X+3).$$ Donc le spectre est
$\{2,-3\}$, il est de taille 2 comme l'espace est de dimension 2.
D'apr\`es le cours, $a$ est diagonalisable et les espaces propres
de dimension 1. L'espace propre associ\'e \`a la valeur propre 2
est l'ensemble des $(x,y)$ tels que $7x-10y=2x$ ou $x=2y.$ On peut
prendre $\vec{f}_1=(2,1)$ pour base de cet espace propre. L'espace
propre associ\'e \`a la valeur propre $-3$ est l'ensemble des
$(x,y)$ tels que $7x-10y=-3x$ ou $x=y.$ On peut prendre
$\vec{f}_2=(1,1)$ pour base de cet espace propre. Alors si
$f=(\vec{f}_1,\vec{f}_2)$ on a

$$P=[\mathrm{id}_E]_f^e=\left[\begin{array}{rr}2&1\\1&1\end{array}\right],\ \
P^{-1}=[\mathrm{id}_E]_e^f=\left[\begin{array}{rr}
1&-1\\-1&2\end{array}\right],\ \
D=[a]_f^f=\left[\begin{array}{rr}2&0\\0&-3\end{array}\right].$$

$$D^{50}=[a^{50}]_f^f=\left[\begin{array}{rr}2^{50}&0\\0&(-3)^{50}\end{array}\right],
\ A^{50}=[a^{50}]_e^e=PD^{50}P^{-1}=\left[\begin{array}{rr}
2.2^{50}-(-3)^{50}&-2.2^{50}+2.(-3)^{50}\\2^{50}-(-3)^{50}&-2^{50}+2.(-3)^{50}
\end{array}\right]$$
Donc $\lim_{n\infty}\frac{1}{3^{2n}}[a^{2n}]_f^f=L=
\left[\begin{array}{rr}0&0\\0&1\end{array}\right],$ et
$\lim_{n\infty}\frac{1}{3^{2n}}[a^{2n}]_e^e=PLP^{-1}=\left[\begin{array}{rr}
-1&2\\-1&2\end{array}\right].$}
}