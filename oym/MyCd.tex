\uuid{MyCd}
\exo7id{3373}
\titre{Commutant d'une matrice diagonale}
\theme{Exercices de Michel Quercia, Matrices}
\auteur{quercia}
\date{2010/03/09}
\organisation{exo7}
\contenu{
  \texte{Soit $A \in \mathcal{M}_n(K)$ et ${\cal C}_A = \{ M \in \mathcal{M}_n(K) \text{ tq } AM = MA \}$
({\it commutant de $A$}).}
\begin{enumerate}
  \item \question{Montrer que ${\cal C}_A$ est une sous-algèbre de $\mathcal{M}_n(K)$.}
  \item \question{Soit $A = \text{diag}(\lambda_1, \lambda_2, \dots, \lambda_n)$ une matrice
    diagonale dont tous les $\lambda_i$ sont distincts.
  \begin{enumerate}}
  \item \question{Chercher ${\cal C}_A$.}
  \item \question{Soit $\phi : {\mathcal{M}_n(K)} \to {\mathcal{M}_n(K)}, M \mapsto {MA-AM}$
         \par
         Montrer que $\Im \phi$ est l'ensemble des matrices à diagonale nulle.}
\end{enumerate}
\begin{enumerate}

\end{enumerate}
}