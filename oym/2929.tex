\uuid{2929}
\titre{Sommets d'un carr{\'e}}
\theme{Exercices de Michel Quercia, Nombres complexes}
\auteur{quercia}
\date{2010/03/08}
\organisation{exo7}
\contenu{
  \texte{}
  \question{Soient $a,b,c,d \in \C$ tels que $$\begin{cases}a+ib &= c+id\cr a+c &= b+d.\end{cases}$$

Que pouvez-vous dire des points d'affixes $a,b,c,d$ ?

En d{\'e}duire qu'il existe $z \in \C$ tel que
$(z-a)^4 = (z-b)^4 = (z-c)^4 = (z-d)^4$.}
  \reponse{Les diagonales se coupent en leurs milieux, ont m{\^e}me longueur,
         et sont perpendiculaires $ \Rightarrow $ carr{\'e}.}
}