\uuid{2413}
\titre{Exercice 2413}
\theme{Théorème de Stone-Weierstrass -- Théorème d'Ascoli}
\auteur{mayer}
\date{2003/10/01}
\organisation{exo7}
\contenu{
  \texte{}
  \question{Soient $E,F$ des espaces norm\'es et $(f_n)$ une suite d'applications  de $E$ dans $F$
\'equicontinue en $a\in E$. Montrer que, si la suite $(f_n(a))$ converge vers $b$,
alors $(f_n(x_n))$ converge \'egalement vers $b$, si $(x_n)$ est une suite de $E$
telle que $\lim_{n\to \infty} x_n =a$.

L'\'equicontinuit\'e est-elle n\'ecessaire ici?}
  \reponse{\begin{enumerate}
  \item 
  \begin{enumerate}
    \item 
  Soit $(x_n)$ une suite convergeant vers $a$, alors
$$|f_n(x_n)-b| \le |f_n(x_n)-f_n(a)|+|f_n(a)-b|.$$

    \item Soit $\epsilon >0$, il existe $N_1$ tel que pour $n\ge N_1$ on ait
$|f_n(a)-b| < \frac \epsilon 2$.

    \item $(f_n)$ est équicontinue en $a$, donc il existe $\eta >0$ tel que
pour tout $n\in \Nn$ et tout $x\in E$, $(|x-a| < \eta \Rightarrow |f_n(x)-f_n(a)| < \frac \epsilon 2)$.
    
    \item Comme $x_n\rightarrow a$ alors il existe $N_2$ tel que pour $n\ge N_2$  on ait $|x_n-a| < \eta$.

    \item Donc pour $n \ge \max(N_1,N_2)$ on a $|f_n(x_n)-b| \le |f_n(x_n)-f_n(a)|+|f_n(a)-b| < \frac \epsilon 2 + \frac \epsilon 2 = \epsilon$. Donc
$(f_n(x_n))$ converge vers $b$.
  \end{enumerate}

  \item  Soit des fonctions réelles définies par $f_n(x) = (1+x)^n$.
Prenons $x_n= \frac 1 n$, alors $x_n \rightarrow a=0$.
Par contre $f_n(a) = f_n(0)=1$ pour tout $n$. Mais $f_n(x_n) = f_n(\frac 1n) =
(1+\frac 1n)^n$ converge vers $e$. L'équicontinuité est donc bien nécessaire.
\end{enumerate}}
}