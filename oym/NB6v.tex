\uuid{NB6v}
\exo7id{4287}
\titre{Sommes de Riemann}
\theme{Exercices de Michel Quercia, Intégrale généralisée}
\auteur{quercia}
\date{2010/03/12}
\organisation{exo7}
\contenu{
  \texte{Soit $f : {[a,b[} \to {\R^+}$ continue croissante.
On pose $S_n = \frac{b-a}n \sum_{k=0}^{n-1} f\Bigl(a+k\frac{b-a}n \Bigr)$.}
\begin{enumerate}
  \item \question{Si $ \int_{t=a}^b f(t)\,d t$ converge, montrer que $S_n \to  \int_{t=a}^b f(t)d t$ lorsque $n\to\infty$.}
  \item \question{Si $ \int_{t=a}^b f(t)\,d t$ diverge, montrer que  $S_n \to +\infty$ lorsque $n\to\infty$.}
\end{enumerate}
\begin{enumerate}

\end{enumerate}
}