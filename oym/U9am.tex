\uuid{U9am}
\exo7id{4896}
\titre{Caractérisation du barycentre par les surfaces}
\theme{Exercices de Michel Quercia, Propriétés des triangles}
\auteur{quercia}
\date{2010/03/17}
\organisation{exo7}
\contenu{
  \texte{Soit $ABC$ un triangle, et $M \in (ABC)$. On note $\alpha,\beta,\gamma$ les
aires des triangles $MBC$, $MCA$, $MAB$.}
\begin{enumerate}
  \item \question{On suppose que $M$ est dans l'enveloppe convexe de $\{A,B,C\}$.\par
    Montrer que : $(\alpha = \beta = \gamma) \iff
                   M = \text{Bar}(A:1,\ B:1,\ C:1)$.}
  \item \question{Quels sont tous les points du plan $(ABC)$ tels que
    $\alpha = \beta = \gamma$ ?}
\end{enumerate}
\begin{enumerate}
  \item \reponse{$G$ et les symétriques de $A,B,C$ par rapport aux milieux des côtés opposés.}
\end{enumerate}
}