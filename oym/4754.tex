\uuid{4754}
\titre{Suites de fonctions}
\theme{Exercices de Michel Quercia, Topologie dans les espaces vectoriels normés}
\auteur{quercia}
\date{2010/03/16}
\organisation{exo7}
\contenu{
  \texte{}
  \question{Soient $E = \mathcal{C}([a,b]\to\R)$, $(f_n)$ une suite de fonctions de $E$
et $f \in E$.
Comparer les {\'e}nonc{\'e}s :
$$1 : \|f_n-f\|_1 \xrightarrow[n\to\infty]{} 0
  \qquad
  2 : \|f_n-f\|_2 \xrightarrow[n\to\infty]{}0  \qquad
  3 : \|f_n-f\|_\infty \xrightarrow[n\to\infty]{} 0.$$}
  \reponse{}
}