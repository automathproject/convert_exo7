\exo7id{5095}
\titre{**}
\theme{}
\auteur{rouget}
\date{2010/06/30}
\organisation{exo7}
\contenu{
  \texte{Simplifier les expressions suivantes}
\begin{enumerate}
  \item \question{$\sin(2\Arcsin x)$,}
  \item \question{$\cos(2\Arccos x)$,}
  \item \question{$\sin^2\left(\frac{\Arccos x}{2}\right)$,}
  \item \question{$\ln(\sqrt{x^2+1}+x)+\ln(\sqrt{x^2+1}-x)$,}
  \item \question{$\Argsh\left(\frac{x^2-1}{2x}\right)$,}
  \item \question{$\Argch(2x^2-1)$,}
  \item \question{$\Argth\left(\sqrt{\frac{\ch x-1}{\ch x+1}}\right)$,}
  \item \question{$\frac{\ch(\ln x)+\sh(\ln x)}{x}$.}
\end{enumerate}
\begin{enumerate}
  \item \reponse{Pour tout réel $x$ de $[-1,1]$, $\sin(2\Arcsin x)=2\sin(\Arcsin x)\cos(\Arcsin x)=2x\sqrt{1-x^2}$.}
  \item \reponse{Pour tout réel $x$ de $[-1,1]$, $\cos(2\Arccos x)=2\cos^2(\Arccos x)-1=2x^2-1$.}
  \item \reponse{Pour tout réel $x$ de $[-1,1]$,
$\sin^2(\frac{1}{2}\Arccos x)=\frac{1}{2}(1-\cos(\Arccos x))=\frac{1-x}{2}$.}
  \item \reponse{Soit $x\in\Rr$.

$$\sqrt{x^2+1}>\sqrt{x^2}=|x|=\mbox{Max}\{x,-x\}.$$
Donc, $\sqrt{x^2+1}+x>0$ et $\sqrt{x^2+1}-x>0$. L'expression proposée existe pour tout réel $x$. De plus,

$$\ln(\sqrt{x^2+1}+x)+\ln(\sqrt{x^2+1}-x)=\ln\left((\sqrt{x^2+1}+x)(\sqrt{x^2+1}-x)\right)=\ln(x^2+1-x^2)=\ln1=0.$$}
  \item \reponse{L'expression proposée est définie sur $\Rr^*$ et impaire. Soit alors $x>0$.

\begin{align*}
\Argsh\left(\frac{x^2-1}{2x}\right)&=\ln\left(\frac{x^2-1}{2x}+\sqrt{\frac{(x^2-1)^2}{(2x)^2}+1}\right)
=\ln\left(\frac{1}{2x}(x^2-1+\sqrt{x^4-2x^2+1+4x^2})\right)\\
 &=\ln\left(\frac{1}{2x}(x^2-1+\sqrt{(x^2+1)^2})\right)=\ln\left(\frac{1}{2x}(x^2-1+x^2+1)\right)=\ln x
\end{align*}
Par imparité, si $x<0$, $\Argsh\left(\frac{x^2-1}{2x}\right)=-\ln(-x)$. En résumé, en notant $\varepsilon$ le signe de
$x$,

\begin{center}
\shadowbox{
$\forall x\in\Rr^*,\;\Argsh\left(\frac{x^2-1}{2x}\right)=\varepsilon\ln|x|.$
}
\end{center}}
  \item \reponse{L'expression proposée existe si et seulement si $2x^2-1\in[1,+\infty[$ ou encore $x^2\in[1,+\infty[$ ou enfin
$x\in]-\infty,-1]\cup[1,+\infty[$. Cette expression est paire. Soit donc $x\in[1,+\infty[$.

\begin{align*}
\Argch(2x^2-1)&=\ln(2x^2-1+\sqrt{(2x^2-1)^2-1})=\ln(2x^2-1+2x\sqrt{x^2-1})=\ln\left(\left(x+\sqrt{x^2-1}\right)^2\right)\\
 &=2\ln\left(x+\sqrt{x^2-1}\right)=2\Argch x
\end{align*}
Par parité, on en déduit que

\begin{center}
\shadowbox{
$\forall x\in]-\infty,-1]\cup[1,+\infty[,\;\Argch(2x^2-1)=2\Argch|x|.$
}
\end{center}}
  \item \reponse{Soit $x\in\Rr$.

\begin{align*}
\Argth\sqrt{\frac{\ch x-1}{\ch x+1}}\;\mbox{existe}&\Leftrightarrow\ch x+1\neq0\;\mbox{et}\;\frac{\ch x-1}{\ch
x+1}\geq0\;\mbox{et}\;\sqrt{\frac{\ch x-1}{\ch x+1}}\in]-1,1[\\
 &\Leftrightarrow\frac{\ch x-1}{\ch x+1}\in[0,1[
\end{align*}
Mais, d'une part, $\frac{\ch x-1}{\ch x+1}\geq0$ et d'autre part, $\frac{\ch x-1}{\ch x+1}=\frac{\ch x+1-2}
{\ch x+1}=1-\frac{2}{\ch x+1}<1$. L'expression proposée existe donc pour tout réel $x$ et est paire. Ensuite, pour $x$ réel positif, on a

\begin{align*}
\frac{1+\sqrt{\frac{\ch x-1}{\ch x+1}}}{1-\sqrt{\frac{\ch x-1}{\ch x+1}}}
&=\frac{\sqrt{\ch x+1}+\sqrt{\ch x-1}}{\sqrt{\ch x+1}-\sqrt{\ch x-1}}=
\frac{(\sqrt{\ch x+1}+\sqrt{\ch x-1})^2}{(\ch x+1)-(\ch x-1)}=\frac{2\ch x+2\sqrt{\ch^2x-1}}{2}\\
 &=\ch x+\sqrt{\sh^2x}=\ch x+|\sh x|=\ch x+\sh x=e^x
\end{align*}
Par suite, $x$ étant toujours positif,

$$\Argth\sqrt{\frac{\ch x-1}{\ch x+1}}=\frac{1}{2}\ln(e^x)=\frac{x}{2}.$$
Par parité, on a alors

\begin{center}
\shadowbox{
$\forall x\in\Rr,\;\Argth\left(\sqrt{\frac{\ch x-1}{\ch x+1}}\right)=\frac{|x|}{2}.$
}
\end{center}
(Remarque. Pour 5), 6) et 7), on peut aussi dériver chaque expression)}
  \item \reponse{Pour $x>0$,

\begin{align*}
\frac{\ch(\ln x)+\sh(\ln x)}{x}&=\frac{1}{2x}\left(x+\frac{1}{x}+x-\frac{1}{x}\right)=1.
\end{align*}}
\end{enumerate}
}