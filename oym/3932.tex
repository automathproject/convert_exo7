\uuid{3932}
\titre{Limite double}
\theme{Exercices de Michel Quercia, Dérivation}
\auteur{quercia}
\date{2010/03/11}
\organisation{exo7}
\contenu{
  \texte{}
  \question{Soit $f:\R \to \R$ continue en $0$.
Montrer que $f$ est dérivable en $0$, et $f'(0) = \ell$ si et seulement si :
$$\forall\ \varepsilon > 0,\ \exists\ \delta > 0 \text{ tq }
  \forall\ h,k\in {]0,\delta[},\
  \left|\frac {f(h) - f(-k)}{h+k} - \ell\right| \le \varepsilon.$$}
  \reponse{}
}