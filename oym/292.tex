\uuid{292}
\titre{Exercice 292}
\theme{Arithmétique dans $\Zz$, pgcd, ppcm, algorithme d'Euclide}
\auteur{bodin}
\date{1998/09/01}
\organisation{exo7}
\contenu{
  \texte{}
  \question{D\'eterminer les couples d'entiers naturels de pgcd
18 et de somme 360. De m\^eme avec pgcd 18 et produit 6480.}
  \reponse{Soient $a,b$ deux entiers de pgcd $18$ et de somme $360$. Soit
$a',b'$ tel que $a = 18a'$ et $b=18b'$. Alors $a'$ et $b'$ sont
premiers entre eux, et leur somme est $360/18 = 20$.

Nous pouvons facilement \'enum\'erer tous les couples d'entiers
naturels $(a',b')$ ($a'\leqslant b'$) qui v\'erifient cette condition,
ce sont les couples  :
$$(1,19),(3,17),(7,13),(9,11).$$

Pour obtenir les couples $(a,b)$ recherch\'es ($a\leqslant b$), il
suffit de multiplier les couples pr\'ec\'edents par $18$ :
$$(18,342),(54,306),(126,234),(162,198).$$}
}