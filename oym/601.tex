\uuid{601}
\titre{Exercice 601}
\theme{}
\auteur{gourio}
\date{2001/09/01}
\organisation{exo7}
\contenu{
  \texte{}
  \question{Soit $(u_{n})_{n\in \Nn}$ la suite d\'{e}finie par $u_{n+1}=u_{n}+u_{n}^{2}. $
L'\'{e}tudier et, en utilisant $v_{n}=\frac{1}{u_{n}},$ en donner un
\'{e}quivalent dans le cas $u_{0}\in ]-1;0]$. Que dire dans le cas $u_{0}\in
]0;\infty [$ ? (On \'{e}tudiera $v_{n}=\frac{\ln (u_{n})}{2^{n}}$.)}
  \reponse{}
}