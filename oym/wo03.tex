\uuid{wo03}
\exo7id{4353}
\titre{Développement en série, Ensam PSI 1998, Mines MP 1999}
\theme{Exercices de Michel Quercia, Intégrale dépendant d'un paramètre}
\auteur{quercia}
\date{2010/03/12}
\organisation{exo7}
\contenu{
  \texte{Soit $I(\alpha) =  \int_{x=0}^{+\infty} \frac{\sin\alpha x}{e^x-1}\,d x$.}
\begin{enumerate}
  \item \question{Justifier l'existence de $I(\alpha)$.}
  \item \question{Déterminer les réels $a$ et~$b$ tels que~:
    $I(\alpha) = \sum_{n=1}^\infty \frac a{b+n^2}$.}
  \item \question{Donner un équivalent de $I(\alpha)$ quand $\alpha\to+\infty$.}
\end{enumerate}
\begin{enumerate}
  \item \reponse{$a=\alpha$, $b=\alpha^2$.}
  \item \reponse{comparaison série-intégrale $ \Rightarrow  I(\alpha)\to\frac\pi2$ lorsque $\alpha\to+\infty$.}
\end{enumerate}
}