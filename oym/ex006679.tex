\exo7id{6679}
\titre{Exercice 6679}
\theme{}
\auteur{queffelec}
\date{2011/10/16}
\organisation{exo7}
\contenu{
  \texte{Pour $n\ge 1$ et $0\le k\le n$, on désigne par $C_n^k$ le
coefficient binomial. Pour $r>0$, soit $c_r\colon t\mapsto re^{it}$, $t\in
[0,2\pi]$.}
\begin{enumerate}
  \item \question{Montrer que 
$$C_n^k={1\over 2i\pi}\int_{c_r}(1+z)^n{dz\over z^{k+1}}.$$
En déduire que $C_{2n}^n\le 4^n$.}
  \item \question{Montrer que 
$$C_{2n}^n={1\over 2i\pi}\int_{c_r}\left( {1\over z}
+2+z\right) ^n{dz\over z}$$
(on pourra utiliser 1.). En déduire que 
$$\sum_{n=0}^{+\infty}C_{2n}^n{1\over 5^n}={1\over
2i\pi}\int_{c_r}{dz\over 3z-1-z^2)}=\sqrt 5 ,$$
à condition que $r_1<r<r_2$, où $r_1<r_2$ sont les deux racines de
$3z-1-z^2=0$.}
  \item \question{Montrer que
$${1\over 2i\pi}\int_{c_r}(1+z)^n\left( 1+{1\over z}\right)^n{dz\over
z}=\sum_{k=0}^n\left( C_n^k\right) ^2.$$
En déduire que 
$$\sum_{k=0}^n\left( C_n^k\right) ^2=C_{2n}^n.$$}
  \item \question{Montrer que
$${1\over 2i\pi}\int_{c_1}{(z-1)^{2n}(z+1)^n\over z^{2n+1}}dz=
\sum_{k=0}^n(-1)^kC_n^kC_{2n}^k.$$
Montrer que si $z\in c_1^*$, $\vert z-1\vert ^2\vert z+1\vert \le
{16\over 9}\sqrt 3$. En déduire que
$$\sum_{k=0}^n(-1)^kC_n^kC_{2n}^k\le \left( {16\over 9}\sqrt 3\right)
^n.$$}
\end{enumerate}
\begin{enumerate}

\end{enumerate}
}