\uuid{XWe6}
\exo7id{4129}
\titre{Résolution approchée de $y' = f(y,t),\ y(a) = y_0$  par la méthode d'Euler}
\theme{Exercices de Michel Quercia, \'Equations différentielles non linéaires (II)}
\auteur{quercia}
\date{2010/03/11}
\organisation{exo7}
\contenu{
  \texte{On suppose que $f$ est bornée par $M$ et $|f(y,s)-f(z,t)| \le K(|y-z|+|s-t|)$.
On divise $[a,b]$ en $n$ intervalles $[a_k, a_{k+1}]$,
$a_k = a + k\frac {b-a}n$ et on approche
la solution $y$ par la fonction $z$, continue affine par morceaux définie
par :
$$\begin{cases}
z(a_0) = y_0 \cr
\text{sur } ]a_k, a_{k+1}[,\ z' = f(z(a_k),a_k). \cr\end{cases}$$}
\begin{enumerate}
  \item \question{Soit $\varepsilon_k = |z(a_k) - y(a_k)|$.
     Montrer que : $\forall\ t \in {[a_k, a_{k+1}]},\quad |y(t)-z(t)| \le kh^2(M+1) + (1+Kh)\varepsilon_k$
     $\Bigl(h = \frac {b-a}n \Bigr)$.}
  \item \question{En déduire que $\sup|y-z| \le (M+1)(e^{K(b-a)}-1) \frac {b-a}n$.}
\end{enumerate}
\begin{enumerate}

\end{enumerate}
}