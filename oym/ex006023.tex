\exo7id{6023}
\titre{Exercice 6023}
\theme{}
\auteur{quinio}
\date{2011/05/20}
\organisation{exo7}
\contenu{
  \texte{Aux dernières élections présidentielles en France, le candidat A
a obtenu $20$\% des voix. On prend au hasard dans des bureaux de vote de
grandes villes des lots de $200$ bulletins: on note $X$ la variable aléatoire 
<<nombre de voix pour A dans les différents bureaux>>.}
\begin{enumerate}
  \item \question{Quelle est la loi de probabilité de $X$?}
  \item \question{Comment peut-on l'approcher?}
  \item \question{Quelle est alors la probabilité pour que : $X$ soit supérieur 
à $45$? $X$ compris entre $30$ et $50$?}
  \item \question{Pour un autre candidat B moins heureux le pourcentage des voix est de $2$\%.
En notant $Y$ le nombre de voix pour B dans les différents bureaux, sur $100$ bulletins, 
reprendre les questions 1 et 2. Quelle est alors la probabilité pour que : $Y$ soit supérieur à
$5$? $Y$ compris entre $1$ et $4$ ?}
\end{enumerate}
\begin{enumerate}
  \item \reponse{Lorsque l'on tire un bulletin au hasard, la probabilité que ce soit
un bulletin pour A est de $0.2.$}
  \item \reponse{Il y a suffisamment de bulletins de vote en tout pour que l'on puisse
assimiler ces tirages à des tirages avec remise; alors la loi de
probabilité de $X$ est une loi binomiale de paramètres $n=200$ et $p=0.2;$ or
 $np=40$; on peut faire l'approximation normale. L'espérance de $X$ est 
donc $m=40$ et l'écart-type: $\sqrt{40\times 0.8}=4\sqrt{2}$.}
  \item \reponse{$P[X\geq 45]=1-P[X\leq 44]\simeq 1-F(\frac{44.5-40}{4\sqrt{2}})\simeq 21\%$,
c'est la probabilité pour que le nombre de voix pour A
soit supérieur à $45$ dans un lot de $200$ bulletins.
De même, $P[30\leq X\leq 50]\simeq F(\frac{50.5-m}{\sigma })-F(\frac{29.\text{$5$}-m}{\sigma })\simeq 93.6\%$.}
  \item \reponse{Reprenons le calcul pour le candidat B qui n'a obtenu que $2$\% des voix.
Alors pour $n=100$ et $p=0.02$ l'approximation par une loi de Poisson d'espérance $\lambda =2$ est légitime.
On peut dire que $P[Y\geq 5]=1-P[Y\leq 4]=1-\sum_{k=0}^{4}\frac{e^{-2}2^{k}}{k!}$, de l'ordre de $5$\%.

Enfin $P[1\leq Y\leq 4]=\sum_{k=1}^{4}\frac{e^{-2}2^{k}}{k!}\simeq 0.812$.}
\end{enumerate}
}