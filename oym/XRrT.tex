\uuid{XRrT}
\exo7id{2614}
\titre{Exercice 2614}
\theme{Sujets de l'année 2008-2009, Examen}
\auteur{delaunay}
\date{2009/05/19}
\organisation{exo7}
\contenu{
  \texte{Soient ${a\in \R}$, ${b\in \R}$ et $A$ la matrice 
$$\begin{pmatrix}
1&a&0 \\ 
0&1&b \\ 
0&0&2 \\ 
\end{pmatrix}$$}
\begin{enumerate}
  \item \question{Donner les valeurs de $a$ et de $b$ pour lesquelles la décomposition de Dunford de $A$ est
$$A=\begin{pmatrix}
1 & 0 & 0  \\ 
0 & 1 & 0  \\ 
0 & 0 & 2  \\ 
\end{pmatrix} + 
\begin{pmatrix}
 0 & a & 0  \\ 
0 & 0 & b  \\ 
0 & 0 & 0  \\ 
\end{pmatrix} 
$$}
  \item \question{On suppose dans la suite que $b=1$ et $a\ne 0$
   \begin{enumerate}}
  \item \question{Déterminer les sous espaces propres et les sous espaces caractéristiques de $A$.}
  \item \question{Trouver $D$ diagonalisable et $N$ nilpotente telles que $D$ commute avec $N$ et 
$$A=D+N.$$}
\end{enumerate}
\begin{enumerate}
  \item \reponse{{\it Donnons les valeurs de $a$ et de $b$ pour lesquelles la décomposition de Dunford de $A$ est}



$$A=\begin{pmatrix}
1 & 0 & 0 \cr
0 & 1 & 0 \cr
0 & 0 & 2 \cr
\end{pmatrix} + 
\begin{pmatrix}
 0 & a & 0 \cr
0 & 0 & b \cr
0 & 0 & 0 \cr
\end{pmatrix} 
$$
Notons $D=\begin{pmatrix}
1 & 0 & 0 \cr
0 & 1 & 0 \cr
0 & 0 & 2 \cr
\end{pmatrix}$ et $N=\begin{pmatrix}
 0 & a & 0 \cr
0 & 0 & b \cr
0 & 0 & 0 \cr
\end{pmatrix}$.
Cette décomposition de $A=D+N$ est sa décomposition de Dunford si et seulement si $N$ est nilpotente (il est clair que $D$ est diagonale) et si $ND=DN$.

Vérifions que $N$ est nilpotente :
$$N^2=\begin{pmatrix}
 0 & a & 0 \cr
0 & 0 & b \cr
0 & 0 & 0 \cr
\end{pmatrix}.\begin{pmatrix}
 0 & a & 0 \cr
0 & 0 & b \cr
0 & 0 & 0 \cr
\end{pmatrix}=\begin{pmatrix}
 0 & 0 & ab \cr
0 & 0 & 0 \cr
0 & 0 & 0 \cr
\end{pmatrix},\quad N^3=\begin{pmatrix}
 0 & 0 & 0 \cr
0 & 0 & 0 \cr
0 & 0 & 0 \cr
\end{pmatrix}$$
ainsi la matrice $N$ est bien nilpotente quelques soient les valeurs de $a$ et $b$. Déterminons pour quelles valeurs de $a$ et $b$ les matrices commutent.

$$N.D=\begin{pmatrix}
 0 & a & 0 \cr
0 & 0 & b \cr
0 & 0 & 0 \cr
\end{pmatrix}.\begin{pmatrix}
1 & 0 & 0 \cr
0 & 1 & 0 \cr
0 & 0 & 2 \cr
\end{pmatrix}=\begin{pmatrix}
0 & a & 0 \cr
0 & 0 & 2b \cr
0 & 0 & 0 \cr
\end{pmatrix}$$ et 
$$D.N=\begin{pmatrix}
1 & 0 & 0 \cr
0 & 1 & 0 \cr
0 & 0 & 2 \cr
\end{pmatrix}.\begin{pmatrix}
 0 & a & 0 \cr
0 & 0 & b \cr
0 & 0 & 0 \cr
\end{pmatrix}=\begin{pmatrix}
 0 & a & 0 \cr
0 & 0 & b \cr
0 & 0 & 0 \cr
\end{pmatrix}.$$
Ainsi, $ND=DN$ si et seulement si $b=2b$, c'est-à-dire si $b=0$. Le paramètre $a$ peut prendre n'importe quelle valeur.}
  \item \reponse{On suppose dans la suite que $b=1$ et $a\neq 0$.
$$A=\begin{pmatrix} 1&a&0\cr
0&1&1\cr
0&0&2\cr
\end{pmatrix}$$

 \begin{enumerate}}
  \item \reponse{{\it Déterminons les sous espaces propres et les sous espaces caractéristiques de $A$}.
 
Commençons par déterminer les valeurs propres de $A$, ce qui est immédiat car $A$ est sous forme triangulaire. Elle admet donc deux valeurs propres, $1$ valeur propre double et $2$ valeur propre simple.
 
Notons $E_1$ et $E_2$ les sous-espaces propres de $A$.

$$E_{1}=\{\vec u=(x,y,z)\in\R^3,\ A\vec u =\vec u\}.$$
On a 
$$\vec u\in E_{1}\iff \left\{\begin{align*} x+ay&=x\cr y+z&=y\cr 2z&=z\end{align*}\right.\iff\left\{\begin{align*}y&=0\cr z&=0\end{align*}\right.$$ 
L'espace $E_1$ est donc la droite vectorielle engendrée par le vecteur $(1,0,0)$, ce sous-espace propre associé à la valeur propre double $1$ est de dimension $1$, la matrice n'est pas diagonalisable.
 
$$E_{2}=\{\vec u=(x,y,z)\in\R^3,\ A\vec u =2\vec u\}.$$
On a 
$$\vec u\in E_{2}\iff \left\{\begin{align*}x+ay&=2x\cr y+z&=2y\cr 2z&=2z\end{align*}\right.\iff\left\{\begin{align*}x&=ay\cr y&=z\end{align*}\right.$$ 
L'espace $E_2$ est donc la droite vectorielle engendrée par le vecteur $(a,1,1)$. La valeur propre $2$ étant simple, le sous-espace caractéristique $N_2$ associé est égal à l'espace $E_2$.
 
Déterminons le sous-espace caractéristique $N_1$ associé à la valeur propre $1$. On a 

$N_1=\ker(A-I)^2$. Calculons la matrice $(A-I)^2$.
$$(A-I)^2=\begin{pmatrix} 0&a&0\cr
0&0&1\cr
0&0&1\cr
\end{pmatrix}\begin{pmatrix} 0&a&0\cr
0&0&1\cr
0&0&1\cr
\end{pmatrix}=\begin{pmatrix} 0&0&a\cr
0&0&1\cr
0&0&1\cr
\end{pmatrix}.$$
Ainsi, le noyau de $(A-I)^2$ est le plan engendré par les vecteurs $\vec e_1=(1,0,0)$ et $\vec e_2=(0,1,0)$.}
  \item \reponse{{\it Déterminons $D$ diagonalisable et $N$ nilpotente telles que $D$ commute avec $N$ et} 
$$A=D+N.$$
Notons $\vec e_3=(a,1,1)$, dans la base $(\vec e_1,\vec e_2,\vec e_3)$, la matrice associée à l'endomorphisme représenté par $A$ s'écrit 
$$B=\begin{pmatrix} 1&a&0\cr
0&1&0\cr
0&0&2\cr
\end{pmatrix}=\underbrace{\begin{pmatrix} 1&0&0\cr
0&1&0\cr
0&0&2\cr
\end{pmatrix}}_\Delta+\underbrace{\begin{pmatrix} 0&a&0\cr
0&0&0\cr
0&0&0\cr
\end{pmatrix}}_M .$$
Par construction, c'est la décomposition de Dunford de $B$ et on a $A=PBP^{-1}$ avec 
$$P=\begin{pmatrix} 1&0&a\cr
0&1&1\cr
0&0&1\cr
\end{pmatrix}\ {\hbox{et}}\ P^{-1}=\begin{pmatrix} 1&0&-a\cr
0&1&-1\cr
0&0&1\cr
\end{pmatrix}.$$
Les matrices $D=P\Delta P^{-1}$ et $N=PMP^{-1}$ vérifient, $N$ nilpotente, $D$ diagonalisable et $ND=DN$. Calculons les
$$D=P\Delta P^{-1}=\begin{pmatrix} 1&0&a\cr
0&1&1\cr
0&0&1\cr
\end{pmatrix}\begin{pmatrix} 1&0&0\cr
0&1&0\cr
0&0&2\cr
\end{pmatrix}\begin{pmatrix} 1&0&-a\cr
0&1&-1\cr
0&0&1\cr
\end{pmatrix}=\begin{pmatrix} 1&0&a\cr
0&1&1\cr
0&0&2\cr
\end{pmatrix}$$
$$N=PMP^{-1}=\begin{pmatrix} 1&0&a\cr
0&1&1\cr
0&0&1\cr
\end{pmatrix}\begin{pmatrix} 0&a&0\cr
0&0&0\cr
0&0&0\cr
\end{pmatrix}\begin{pmatrix} 1&0&-a\cr
0&1&-1\cr
0&0&1\cr
\end{pmatrix}=\begin{pmatrix} 0&a&-a\cr
0&0&0\cr
0&0&0\cr
\end{pmatrix}=A-D.$$
Ainsi
$$A=\begin{pmatrix} 1&a&0\cr
0&1&1\cr
0&0&2\cr
\end{pmatrix}=N+D=\begin{pmatrix} 0&a&-a\cr
0&0&0\cr
0&0&0\cr
\end{pmatrix}+\begin{pmatrix} 1&0&a\cr
0&1&1\cr
0&0&2\cr
\end{pmatrix}.$$}
\end{enumerate}
}