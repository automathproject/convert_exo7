\uuid{BQ31}
\exo7id{2520}
\titre{partiel du 5 d\'ecembre 1999}
\theme{Théorème des accroissements finis}
\auteur{queffelec}
\date{2009/04/01}
\organisation{exo7}
\contenu{
  \texte{Soit $f:{\Rr^2}\to{\Rr^2}$ d\'efinie par
$f(x,y)=(x^2-y,x^2+y^2)$ et $g=f\circ f$.}
\begin{enumerate}
  \item \question{Montrer que $f$ et $g$ sont de classe $C^1$.}
  \item \question{Calculer en tout point $(x,y)\in{\Rr^2}$ la matrice
jacobienne de $f$ not\'ee $Df{(x,y)}$; calculer la matrice
jacobienne de $g$ au point $(0,0)$ not\'ee $Dg{(0,0)}$.}
  \item \question{Montrer qu'il existe $\rho>0$ tel que pour tout $(x,y)\in
\overline{B_\rho((0,0))}$ (la boule ferm\'ee de centre $(0,0)$ et
de rayon $\rho$) on a $||Dg{(x,y)}||\leq {\frac 1 2}$.}
  \item \question{Montrer que la fonction $g$ admet un unique point fixe dans
$\overline{B_\rho((0,0))}$.}
\end{enumerate}
\begin{enumerate}
  \item \reponse{$f$ est de classe $C^\infty$ car ses coordonn\'ees le sont
(polynômes). $g$ l'est car c'est la compos\'ee de deux fonctions
$C^\infty$.}
  \item \reponse{La matrice jacobienne de $f$ est:
$$Df(x,y)=\left(\begin{array}{cc}
2x & -1 \\ 2x & 2y
\end{array}\right)
$$ D'ap\`es la formule de diff\'erentielle d'une compos\'ee, on a
$$Dg(x,y)=Df(f(x,y))\circ Df(x,y).$$
Or $f(0,0)=0$ et $$Df(0,0)=\left(\begin{array}{cc}0& -1 \\0& 0
\end{array}\right)$$
et donc $$Dg(0,0)= \left(\begin{array}{cc}0& -1 \\0& 0
\end{array}.\right)\left(\begin{array}{cc}0& -1 \\0& 0
\end{array}\right)=0.$$}
  \item \reponse{Par continuit\'e de $Dg(x,y)$ \`a l'origine et en prenant
$\epsilon=1/2$ on a:
$$\exists \rho> 0, ||(x,y)-(0,0)||\leq \rho \Rightarrow
||Dg(x,y)-Dg(0,0)|| \leq 1/2$$ d'o\`u le r\'esultat demand\'e.}
  \item \reponse{D'apr\`es les accroissements finis, pour tous $X,Y \in
\mathbb{R}^2$, on a
$$||g(X)-g(Y)|| \leq \sup_{Z \in \overline B_\rho((0,0))}||Dg(Z)||.||X-Y|| \leq 1/2||X-Y||$$
et donc $g$ est contractante. Le Boule $\overline B_\rho((0,0))$
la boule $\overline B_\rho((0,0))$ \'etant compacte et compl\`ete,
le th\'eor\`eme du point fixe permet de conclure.}
\end{enumerate}
}