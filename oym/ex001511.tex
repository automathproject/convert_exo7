\uuid{1511}
\titre{Exercice 1511}
\theme{}
\auteur{ortiz}
\date{1999/04/01}
\organisation{exo7}
\contenu{
  \texte{Soit $q$ la forme quadratique de $\Rr^3$ de
matrice $A=
\left(\begin{smallmatrix}
2 & 1 & 1 \\
1 & 1 & 1 \\
1 & 1 & 2
\end{smallmatrix}\right)
$ dans la base canonique
$\mathcal{B}=(e_1,e_2,e_3)$ de $\Rr^3.$}
\begin{enumerate}
  \item \question{Donner l'expression analytique de $q$ dans $\mathcal{B}$ et
expliciter sa forme polaire $f$.}
  \item \question{V\'erifier que $\mathcal{B}^{^{\prime }}=(e_1,-\frac
12e_1+e_2,-e_2+e_3)$ est une base $\Rr^3$ et
donner la matrice de $q$ dans cette base.
Expliciter $q$ dans cette base.}
  \item \question{Trouver le rang et la signature de $q$.}
\end{enumerate}
\begin{enumerate}

\end{enumerate}
}