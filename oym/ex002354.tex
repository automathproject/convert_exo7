\exo7id{2354}
\titre{Exercice 2354}
\theme{}
\auteur{queffelec}
\date{2003/10/01}
\organisation{exo7}
\contenu{
  \texte{}
\begin{enumerate}
  \item \question{Soit $C$ l'espace des fonctions continues r\'eelles sur $[0,1]$ muni de la
m\'etrique
$d_1(f,g)=\int_0^1\vert f-g\vert\ dx$, puis de la m\'etrique
$d_\infty(f,g)=\sup_x\vert f(x)-g(x)\vert$. V\'erifier que l'applica\-tion
$f\to \int_0^1 |f| dx$ de
$C$ dans
$\Rr$ est $1$-lipschitzienne dans les deux cas.}
  \item \question{Soit $c$ l'espace des suites r\'eelles convergentes, muni de la m\'etrique
$d(x,y)=\sup_n\vert x(n)-y(n)\vert$. Si on d\'esigne par $\ell(x)$ la limite de la
suite $x$, montrer que $\ell$ est une application continue de $c$ dans $\Rr$. En
d\'eduire que $c_0$ est ferm\'e dans $c$.}
\end{enumerate}
\begin{enumerate}
  \item \reponse{Soit $F$ l'application définie par
$F(f) = \int_0^1 |f|.$ Alors 
$$|F(f)-F(g)| = | \int_0^1 |f|-|g| | \le \int_0^1 |f-g| = d_1(f,g) \le d_\infty (f,g).$$ Donc pour les deux distances $d_1$ et $d_\infty$, $F$ est lipschitzienne de rapport $1$.}
  \item \reponse{Soit $\epsilon >0$ alors en posant $\eta = \epsilon$ on obtient la continuité : 
si $d(x,y) < \epsilon$ alors $$|\ell(x)-\ell(y)| \leq \epsilon.$$ 
Donc $\ell$ est continue, et $c_0 = \ell{-1}(\{0\})$ est un fermé , car c'est l'image réciproque du fermé $\{0\}$ par l'application continue $\ell$.}
\end{enumerate}
}