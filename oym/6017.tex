\uuid{6017}
\titre{Exercice 6017}
\theme{Loi normale et approximations}
\auteur{quinio}
\date{2011/05/20}
\organisation{exo7}
\contenu{
  \texte{}
  \question{Des machines fabriquent des crêpes destinées à être empilées dans des paquets de 10.
Chaque crêpe a une épaisseur qui suit une loi normale de paramètres $m=0.6$mm et $\sigma =0.1$.
Soit $X$ la variable aléatoire <<épaisseur du paquet en mm>>.
Calculez la probabilité pour que $X$ soit compris entre 6.3mm et 6.6mm.}
  \reponse{Par des méthodes analogues on trouve que la probabilité pour que $X$ 
soit compris entre 6.3mm et 6.6 mm est
$14.3$.}
}