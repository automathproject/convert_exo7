\uuid{2186}
\titre{Exercice 2186}
\theme{}
\auteur{debes}
\date{2008/02/12}
\organisation{exo7}
\contenu{
  \texte{}
  \question{\label{ex:deb86}
Soit $G$ un groupe fini et $X$ un $G$-ensemble. Si $k$ est
un entier ($1\leq k$),
on dit que $X$ est $k$-transitif, si pour tout couple de $k$-uplets 
$(x_1, \dots ,x_k)$ et
$(y_1, \dots , y_k)$ d'\'el\'ements de $X$ distincts deux \`a deux,
il existe au moins un \'el\'ement $g$
de
$G$ tel que pour tout $i$, $1\leq i \leq k$, $g.x_i =y_i$. Un $G$-ensemble
$1$-transitif est donc
simplement un
$G$-ensemble transitif.
\smallskip

(a) Montrer que si $X$ est $k$-transitif, il est aussi $l$-transitif pour
tout $l$, $1\leq l \leq k$.
\smallskip

(b) Montrer que $X$ est $2$-transitif si et seulement si le fixateur d'un \'el\'ement $x$
de $X$ agit transitivement sur $X\setminus\{x\}$.

(c) Montrer que si $X$ est imprimitif, il n'est pas $2$-transitif.
\smallskip

(d) Montrer qu'un groupe cyclique $C$ d'ordre premier consid\'er\'e comme
$C$-ensemble par l'action de
translation de $C$ sur lui-m\^eme, est primitif mais n'est pas $2$-transitif.
\smallskip

(e) Montrer que l'ensemble $\{ 1, \dots ,n \} $ muni de l'action du groupe
$S_n$ est $k$-transitif
pour tout
$k$,
$1\leq k\leq n$. En d\'eduire que
l'ensemble $\{ 1, \dots ,n \} $ muni de l'action du groupe $S_n$ est
primitif.
\smallskip

(f) Montrer que le fixateur de $1$ dans $S_n$ est isomorphe \`a $S_{n-1}$.
Dans la suite on
identifie $S_{n-1}$ \`a ce fixateur. D\'eduire de l'exercice
19 que $S_{n-1}$ est un
sous-groupe propre maximal de $S_n$.}
  \reponse{}
}