\uuid{5983}
\titre{Exercice 5983}
\theme{Probabilité et dénombrement ; indépendance}
\auteur{quinio}
\date{2011/05/18}
\organisation{exo7}
\contenu{
  \texte{}
  \question{Une entreprise décide de classer $20$ personnes
susceptibles d'être embauchées; leurs CV étant très proches,
le patron décide de recourir au hasard : combien y-a-il de classements
possibles : sans ex-aequo; avec exactement $2$ ex-aequo ?}
  \reponse{Classements possibles : sans ex-aequo, il y en a 20!.

Avec exactement $2$ ex-aequo, il y en a :
\begin{enumerate}
\item Choix des deux ex-aequo : $\binom{20}{2}=$ $190$ choix;
\item Place des ex-aequo : il y a $19$ possibilités;
\item Classements des $18$ autres personnes, une fois les ex-aequo placés : il y a $18!$ choix.
\end{enumerate}
Il y a au total : $19\binom{20}{2}(18!)$ choix possibles.}
}