\uuid{1447}
\titre{Exercice 1447}
\theme{}
\auteur{ortiz}
\date{1999/04/01}
\organisation{exo7}
\contenu{
  \texte{Les questions sont ind\'ependantes.}
\begin{enumerate}
  \item \question{\begin{enumerate}}
  \item \question{Montrer que l'application $f:\Zz^2\to \Zz$, $(x,y)\mapsto3x+6y$
est un morphisme de groupes.}
  \item \question{D\'eterminer le noyau $\ker f$ de $f$ et montrer qu'il n'existe pas
$(p,q)\in\Zz^2$ tel que $\ker f=p\Zz\times q\Zz.$}
  \item \question{Montrer que le groupe-quotient $\Zz^2/\Zz(-2,1)$ est isomorphe au groupe $3\Zz.$}
\end{enumerate}
\begin{enumerate}
  \item \reponse{\begin{enumerate}}
  \item \reponse{$f((x,y)+(x',y'))=f(x+x',y+y')=3(x+x')+6(y+y')=3x+6y+3x'+6y'=f(x,y)+f(x',y')$.}
  \item \reponse{$\mathrm{Ker} f = \{ (x,y) ; f(x,y)=0\} = \{ (x,y) ; 3x+6y=0\}=
\{ (x,y) ; x=-2y \}= \{ (-2k,k) ; k\in \Zz\}$. Si $\mathrm{Ker} f =
p\Zz\times q\Zz$ alors $f(p,0)=0$ donc $3p=0$ soit $p=0$. De même
$f(0,q)=0$ implique $q=0$ et alors $\mathrm{Ker} f = \{ (0,0) \}$, ceci
contredit le fait que $f(-2,1)=0$.}
  \item \reponse{On a $f(\Zz^2) = 3\Zz$, le morphisme
$f : \Zz^2 \longrightarrow 3\Zz$ définit par passage au quotient
par le noyau un morphisme injectif $\bar{f} : \Zz^2/\mathrm{Ker} f
\longrightarrow 3\Zz$ (c'est le théorème de factorisation). De
plus comme $f$ est surjectif alors $\bar{f}$ l'est aussi. Ainsi
$\bar{f}$ est un isomorphisme entre $\Zz^2/\mathrm{Ker} f
=\Zz^2/(-2,1)\Zz$ et $3\Zz$.}
\end{enumerate}
}