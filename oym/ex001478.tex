\uuid{1478}
\titre{Exercice 1478}
\theme{}
\auteur{barraud}
\date{2003/09/01}
\organisation{exo7}
\contenu{
  \texte{}
  \question{Dans $\R^{3}$ muni de son produit scalaire canonique, déterminer la
  projection orthogonale sur le plan d'équation $x+y+z=0$ de $(1,0,0)$,
  et plus généralement d'un vecteur $(x,y,z)$ quelconque.

  Donner la  matrice de cette projection ainsi que celle de la symétrie
  orthogonale par rapport à ce plan.
  \medskip

  Dans un espace euclidien de dimension $n$, on considère un sous-espace
  $F$ de dimension $r$ et $(f_{1},...,f_{r})$ une base de orthonormée de
  cet espace. On not $p_{F}$ la projection orthogonale sur $F$, c'est à
  dire la projection sur $F$ associée à la décomposition $E=F\oplus
  F^{\bot}$. Montrer que :
  $$
  \forall v\in F, \qquad p_{F}(v)=<v,f_{1}>f_{1}
                                 +<v,f_{2}>f_{2}
                                 +\cdots
                                 +<v,f_{r}>f_{r}
  $$}
  \reponse{}
}