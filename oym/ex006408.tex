\exo7id{6408}
\titre{Exercice 6408}
\theme{}
\auteur{potyag}
\date{2011/10/16}
\organisation{exo7}
\contenu{
  \texte{Le but de cet exercice est démontrer que pour
chaque triplet $\alpha,\ \beta,\ \gamma\in\   ]0,\frac{\pi}{2}[$ tel que $\alpha+\beta+\gamma>\pi$ il
existe un triangle sphérique d'angles intérieurs égaux à $\alpha,\ \beta,\
\gamma$.}
\begin{enumerate}
  \item \question{Montrer que $\forall\ \alpha\in\  ]0,\frac{\pi}{2}[$\  $\forall\ d\in\  ]0,
  \alpha]$
   il existe un triangle
  sphérique $\triangle ABC$ tel que $\angle C=\frac{\pi}{2},\ \angle
  A=\alpha$ et $a=\vert BC\vert=d$ (dans les notations précédentes).}
  \item \question{En utilisant 1. et les formules de l'exercice \ref{exo:apres}
 démontrer le résultat.}
\end{enumerate}
\begin{enumerate}

\end{enumerate}
}