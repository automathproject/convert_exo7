\uuid{2674}
\titre{Exercice 2674}
\theme{}
\auteur{matexo1}
\date{2002/02/01}
\organisation{exo7}
\contenu{
  \texte{}
  \question{Calculer par la m{\'e}thode des r{\'e}sidus l'int{\'e}grale de Wallis
$$ W_n = \int_0^{\pi \over 2} \cos^{2n} \theta\,d\theta. $$}
  \reponse{On a, sachant que $2\cos\theta = e^{i\theta}+e^{-i\theta}$:
$$ W_n = {1\over 4}\int_0^{2\pi } \cos^{2n}\theta\,d\theta
 = {1\over 2^{2n+2}} \int_C \left(z+{1\over z}\right)^{2n}\ {dz\over iz}$$
o{\`u} $C$ est le cercle trigonom{\'e}trique et $z=e^{i\theta}$. En posant
$$f(z) = {1\over z} \left(z+{1\over z}\right)^{2n} $$
on a imm{\'e}diatement
$$ W_n = {\pi \over 2^{2n+1}} \mbox{\rm Res}(f, 0) $$
car $0$ est le seul p{\^o}le de $f$. Le r{\'e}sidu est en fait le coefficient en $1/z$ dans
le d{\'e}veloppement de $f$, c'est-{\`a}-dire le coefficient constant dans $(z+1/z)^{2n}$,
soit par la formule du bin{\^o}me, $C_{2n}^n$. Donc
$$ W_n =  {\pi \over 2^{2n+1}} {(2n)!\over (n!)^2}.$$}
}