\exo7id{1651}
\titre{Exercice 1651}
\theme{}
\auteur{legall}
\date{1998/09/01}
\organisation{exo7}
\contenu{
  \texte{Soit $  E  $ un $  \mathbb{K}$-espace vectoriel. Une application
$  p \in \mathcal{L} (E)  $ est nomm\'ee projecteur lorsque $  p^2=p  .$}
\begin{enumerate}
  \item \question{Montrer que si $  p  $ est un projecteur $  1-p  $ est un projecteur. Montrer que $  \hbox{Im}(p)\oplus \hbox{Ker}(p)=E  .$}
  \item \question{On suppose que $  \mathbb{K} = \R   .$ Soient $  p  $ et $ q   $ deux projecteurs
tels que $  p+q
$ soit aussi un projecteur. Montrer que~:
    \begin{enumerate}}
  \item \question{$  pq=qp=0  .$}
  \item \question{$  \hbox{Im}(p+q)=\hbox{Im}(p)+\hbox{Im}(q)  .$}
  \item \question{$  \hbox{Ker}(p+q)=\hbox{Ker}(p)\cap \hbox{Ker}(q)  .$}
\end{enumerate}
\begin{enumerate}

\end{enumerate}
}