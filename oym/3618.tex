\uuid{3618}
\titre{Polytechnique MP$^*$ 2000}
\theme{Exercices de Michel Quercia, Réductions des endomorphismes}
\auteur{quercia}
\date{2010/03/10}
\organisation{exo7}
\contenu{
  \texte{}
  \question{Soit $E$ un espace vectoriel de dimension finie, $f$ un endomorphisme de~$E$
tel que tout sous-espace de~$E$ admette un supplémentaire stable par~$f$.
Que peut-on dire de~$f$~? Réciproque~?}
  \reponse{Soit~$F$ un hyperplan de~$E$,
$<\!e\!>$ un supplémentaire stable et~$H$ un supplémentaire
de~$<\!e\!>$ stable. Si $K$ est un sev de~$H$, alors $K$ admet un supplémentaire
$K'$ dans~$E$ stable et $H\cap K'$ est un sev de~$H$ stable, en somme directe
avec~$K$.
$K'\not\subset H$ car $K\subset H$ et $K\oplus K' = E$ donc $K'+H = E$
et $\dim(H\cap K') = \dim(H)+\dim(K')-\dim(E) = \dim(H)-\dim(K)$
soit $K\oplus(H\cap K') = H$. $f_{|H}$ vérifie la même propriété que~$f$ et
on obtient par récurrence que $f$ est diagonalisable.

Réciproquement, soit $f$ diagonalisable, $F$ un sev de~$E$ et $(e_1,\dots,e_n)$
une base propre pour~$f$. On montre que $F$ admet un supplémentaire stable
par récurrence sur $\mathrm{codim}(F)$~: si $F=E$ alors $\{0\}$ convient
et si $F\ne E$ alors il existe $i$ tel que $e_i\notin F$ d'où $F\oplus{<\!e_i\!>}$
est un sur-espace strict de~$F$, admettant un supplémentaire~$G$ stable, d'où
$G\oplus{<\!e_i\!>}$ est supplémentaire de~$F$ stable.}
}