\uuid{Hva6}
\exo7id{4798}
\titre{Points fixes, ULM-Lyon-Cachan MP$^*$ 2005}
\theme{Exercices de Michel Quercia, Topologie dans les espaces vectoriels normés}
\auteur{quercia}
\date{2010/03/16}
\organisation{exo7}
\contenu{
  \texte{}
\begin{enumerate}
  \item \question{Montrer que les points fixes de $f$, continue sur $[0,1]$, {\`a} valeurs dans $[0,1]$,
    forment un ensemble ferm{\'e} non vide.}
  \item \question{Montrer que tout ferm{\'e} de $[0,1]$ non vide est l'ensemble des points fixes
    d'une fonction continue de $[0,1]$ dans~$[0,1]$.}
\end{enumerate}
\begin{enumerate}
  \item \reponse{Soit $F$ est un tel ferm{\'e}, et $a\in F$.
On prend $f(x) = x + d(x,F)$ si $0\le x\le a$
et $f(x) = x-d(x,F)$ si $a\le x\le 1$.}
\end{enumerate}
}