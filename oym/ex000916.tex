\uuid{916}
\titre{Exercice 916}
\theme{}
\auteur{ridde}
\date{1999/11/01}
\organisation{exo7}
\contenu{
  \texte{}
  \question{Soit $\alpha \in \Rr$ et
soit $f_{\alpha} : \Rr\rightarrow \Rr$ la fonction définie par 
$$
\begin{cases}
f_{\alpha}(x)= 1 & \text{ si } x = \alpha \\
f_{\alpha}(x)= 0 & \text{ si } x \neq \alpha
 \end{cases}
.$$ 
Montrer que la famille $ (f_{\alpha})_{\alpha \in\Rr}$ est libre.}
  \reponse{\`A partir de la famille $(f_\alpha)_{\alpha\in \Rr}$ nous
  consid\'erons une combinaison lin\'eaire (qui ne correspond qu'\`a un
  nombre \emph{fini} de termes).

  
  Soient $\alpha_1,\ldots,\alpha_n$ des r\'eels distincts, consid\'erons la
  famille (finie) : $(f_{\alpha_i})_{i=1,\ldots,n}$. Supposons qu'il
  existe des r\'eels $\lambda_1,\ldots, \lambda_n$ tels que
  $\sum_{i=1}^n \lambda_i f_{\alpha_i}=0$. Cela signifie que, quelque
  soit $x \in \Rr$, alors $\sum_{i=1}^n \lambda_i f_{\alpha_i}(x) = 0$
  ; en particulier pour $x = \alpha_j$ l'\'egalit\'e devient $\lambda_j =
  0$ car $f_{\alpha_i}(\alpha_j)$ vaut $0$ si $i\neq j$ et $1$ si
  $i=j$. En appliquant le raisonnement ci-dessus pour $j=1$ jusqu'\`a
  $j=n$ on obtient : $\lambda_j = 0$, $j=1,\ldots,n$. Donc la famille
  $(f_\alpha)_{\alpha}$ est une famille libre.}
}