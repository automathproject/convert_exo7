\uuid{341}
\titre{Exercice 341}
\theme{}
\auteur{bodin}
\date{1998/09/01}
\organisation{exo7}
\contenu{
  \texte{}
\begin{enumerate}
  \item \question{Montrer par r\'ecurrence que $\forall n\in \Nn ,\forall k\geqslant 1$ on a :
$$2^{2^{n+k}}-1=\left( 2^{2^n}-1 \right) \times \prod_{i=0}^{k-1}(2^{2^{n+i}}+1).$$}
  \item \question{On pose $F_n=2^{2^n}+1$. Montrer que pour $m\not= n$,
 $F_n$ et $F_m$ sont premiers entre eux.}
  \item \question{En d\'eduire qu'il y a une infinit\'e de nombres premiers.}
\end{enumerate}
\begin{enumerate}
  \item \reponse{Fixons $n$ et montrons la r\'ecurrence sur $k \ge 1$.
La formule est vraie pour $k=1$.
Supposons la formule vraie au rang $k$.
Alors
\begin{align*}
(2^{2^n}-1) \times \prod_{i=0}^{k}{(2^{2^{n+i}}+1)} 
&= (2^{2^n}-1) \times \prod_{i=0}^{k-1}{(2^{2^{n+i}}+1)} \times (2^{2^{n+k}}+1) \\
&= (2^{2^{n+k}}-1)\times (2^{2^{n+k}}+1) = (2^{2^{n+k}})^2-1
= 2^{2^{n+k+1}}-1.\\
\end{align*}
Nous avons utiliser l'hypoth\`ese de r\'ecurrence dans ces \'egalit\'es.
Nous avons ainsi montrer la formule au rang $k+1$. Et donc par
le principe de r\'ecurrence elle est vraie.}
  \item \reponse{\'Ecrivons $m=n+k$, alors l'\'egalit\'e pr\'ec\'edente devient:
$$F_m+2 = (2^{2^n}-1) \times \prod_{i=n}^{m-1} {F_i}.$$
Soit encore :
$$F_n \times (2^{2^n}-1) \times \prod_{i=n+1}^{m-1} {F_i} \ \ \  - \ \ F_m = 2.$$
Si $d$ est un diviseur de $F_n$ et $F_m$ alors $d$ divise $2$
(ou alors on peut utiliser le th\'eor\`eme de B\'ezout). En cons\'equent $d=1$ ou $d=2$. Mais $F_n$ est impair donc $d=1$. Nous avons montrer
que tous diviseurs de $F_n$ et $F_m$ est $1$, cela signifie que
$F_n$ et $F_m$ sont premiers entre eux.}
  \item \reponse{Supposons qu'il y a un nombre fini de nombres premiers.
Nous les notons alors $\{p_1,\ldots,p_N\}$. Prenons alors $N+1$
nombres de la famille $F_i$, par exemple $\{F_1,\ldots,F_{N+1}\}$.
Chaque $F_i$, $i=1,\ldots,N+1$ est divisible par (au moins) un facteur 
premier $p_j$, $j=1,\ldots,N$. Nous avons $N+1$ nombres $F_i$ et seulement $N$ facteurs premiers $p_j$. Donc par le principe des tiroirs
il existe deux nombres distincts $F_k$ et $F_{k'}$ 
(avec $1 \leq k,k' \leq N+1$) qui ont un facteur premier en commun.
En cons\'equent $F_k$ et $F_{k'}$ ne sont pas premiers entre eux. Ce qui contredit la question pr\'ec\'edente. Il existe donc une infinit\'e de nombres premiers.}
\end{enumerate}
}