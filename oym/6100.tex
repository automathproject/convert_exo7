\uuid{6100}
\titre{Exercice 6100}
\theme{}
\auteur{queffelec}
\date{2011/10/16}
\organisation{exo7}
\contenu{
  \texte{}
  \question{On se donne une application $f:\Rr\to\Rr^n$, et on note $d$ la distance
euclidienne sur $\Rr^n$. A quelles conditions sur
$f$, $\delta(x,y)=d(f(x),f(y))$ définit-elle une distance sur $\Rr$
équivalente topologiquement à la distance usuelle (ie définissant la même
topologie.)?}
  \reponse{}
}