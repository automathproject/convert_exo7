\uuid{3wCR}
\exo7id{7777}
\titre{Les sous-groupes distingués de $\mathcal{S}_n$}
\theme{Exercices de Christophe Mourougane, Théorie des groupes et géométrie}
\auteur{mourougane}
\date{2021/08/11}
\organisation{exo7}
\contenu{
  \texte{Le but de l'exercice est de déterminer les sous-groupes distingués de
$\mathcal{S}_n$ (pour $n\geq 5$).}
\begin{enumerate}
  \item \question{Soit $H$ un sous-groupe distingué de $\mathcal{S}_n$.
Montrer que $H\cap \mathcal{A}_n$ est un sous-groupe distingué de $\mathcal{A}_n$.
En déduire que $H$ contient $\mathcal{A}_n$ ou que $H\cap\mathcal{A}_n=\{id\}$ ?}
  \item \question{On suppose que $H\cap \mathcal{A}_n=\{id\}$. Montrer que la restriction à $H$ du morphisme signature est injective.
Montrer que dans ce cas que tous les éléments de $H$ sont dans le centre de $\mathcal{S}_n$ et en déduire que $H=\{id\}$.}
  \item \question{On suppose que $H$ contient $\mathcal{A}_n$. Montrer alors que $H=\mathcal{S}_n$ ou $H=\mathcal{A}_n$ suivant l'indice de $H$ dans $\mathcal{S}_n$.}
  \item \question{Conclure~: si $n\geq 5$, les seuls sous-groupes distingués de $\mathcal{S}_n$ sont $\{id\}$, $\mathcal{A}_n$ et $\mathcal{S}_n$..}
\end{enumerate}
\begin{enumerate}

\end{enumerate}
}