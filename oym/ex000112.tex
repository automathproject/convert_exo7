\exo7id{112}
\titre{Exercice 112}
\theme{}
\auteur{bodin}
\date{1998/09/01}
\organisation{exo7}
\contenu{
  \texte{Nier les assertions suivantes :}
\begin{enumerate}
  \item \question{tout triangle rectangle poss\`ede un angle droit ;}
  \item \question{dans toutes les \'ecuries, tous les chevaux sont noirs ;}
  \item \question{pour tout entier $x$, il existe un entier $y$ tel que, pour tout entier $z$, 
la relation $z<x$ implique la relation $z<x+1$ ;}
  \item \question{$\forall \epsilon >0 \quad \exists \alpha >0 \qquad (|x-7/5|<\alpha \Rightarrow |5x-7|<\epsilon)$.}
\end{enumerate}
\begin{enumerate}
  \item \reponse{``Il existe un triangle rectangle qui n'a pas d'angle droit." Bien sûr cette dernière phrase est fausse !}
  \item \reponse{``Il existe une \'ecurie dans laquelle il y a (au moins) un cheval
dont la couleur n'est pas noire."}
  \item \reponse{Sachant que la proposition en langage math\'ematique s'\'ecrit
$$\forall x\in\Zz \ \  \exists y\in\Zz\ \  \forall z\in\Zz \quad (z<x \Rightarrow z<x+1),$$
la n\'egation est
$$\exists x\in\Zz\ \  \forall y\in\Zz\ \  \exists z\in\Zz \quad (z<x \text{ et } z\geq x+1).$$}
  \item \reponse{$\exists \epsilon>0\ \  \forall \alpha>0 \quad (|x-7/5|<\alpha \text{ et } |5x-7|\geq\epsilon).$}
\end{enumerate}
}