\uuid{WLkJ}
\exo7id{3663}
\titre{Inversion}
\theme{Exercices de Michel Quercia, Produit scalaire}
\auteur{quercia}
\date{2010/03/11}
\organisation{exo7}
\contenu{
  \texte{Soit $E$ un espace vectoriel euclidien. On pose pour $\vec x \ne \vec 0$ :
$i(\vec x) = \frac{\vec x}{\|\vec x\,\|^2}$.}
\begin{enumerate}
  \item \question{Montrer que $i$ est une involution et conserve les angles de vecteurs.}
  \item \question{Vérifier que : $\forall\ \vec x,\vec y \in E \setminus\{\vec 0\}$,
    $\|i(\vec x) - i(\vec y)\| = \frac{\|\vec x - \vec y\,\|}{\|\vec x\,\|\,\|\vec y\,\|}$.}
  \item \question{Déterminer l'image par $i$ :
  \begin{enumerate}}
  \item \question{d'un hyperplan affine ne passant pas par $\vec 0$.}
  \item \question{d'une sphère passant par $\vec 0$.}
  \item \question{d'une sphère ne passant pas par $\vec 0$.}
\end{enumerate}
\begin{enumerate}
  \item \reponse{\'Elever au carré.}
  \item \reponse{\begin{enumerate}}
  \item \reponse{$(\vec x\mid \vec u) = 1
                  \Leftrightarrow (i(\vec x) \mid \vec u-i(\vec x)) = 0$ :
                  sphère passant par $\vec 0$.}
  \item \reponse{Hyperplan ne passant pas par $\vec 0$.}
  \item \reponse{$\|\vec x-\vec a\,\|^2 = R^2 \Leftrightarrow
                  \left\|\vec x -\frac{\vec a}{\|\vec a\,\|^2-R^2}\right\|^2 =
                  \frac{R^2}{(\|\vec a\,\|^2-R^2)^2}$ :
                  sphère ne passant pas par $\vec 0$.}
\end{enumerate}
}