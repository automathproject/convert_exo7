\uuid{utRI}
\exo7id{4631}
\titre{Calcul de séries}
\theme{Exercices de Michel Quercia, Séries de Fourier}
\auteur{quercia}
\date{2010/03/14}
\organisation{exo7}
\contenu{
  \texte{Soit $f$ la fonction $2\pi$-périodique telle que :
$\forall\ x \in {[-\pi,\pi[},\ f(x) = e^x$.}
\begin{enumerate}
  \item \question{Chercher le développement en série de Fourier de $f$.}
  \item \question{En déduire les sommes des séries :
    $S = \sum_{n=1}^\infty \frac1{n^2+1}$ et
    $S' = \sum_{n=1}^\infty \frac{(-1)^n}{n^2+1}$.}
\end{enumerate}
\begin{enumerate}
  \item \reponse{$a_0 = \frac{2\sh\pi}\pi$,
             $a_n = \frac{2(-1)^n\sh\pi}{\pi(1+n^2)}$, $b_n = -na_n$.}
  \item \reponse{$S = \frac{\pi-\tanh\pi}{2\tanh\pi}$,
             $S' = \frac{\pi-\sh\pi}{2\sh\pi}$.}
\end{enumerate}
}