\uuid{3357}
\titre{X MP$^*$ 2001}
\theme{Exercices de Michel Quercia, Applications linéaires en dimension finie}
\auteur{quercia}
\date{2010/03/09}
\organisation{exo7}
\contenu{
  \texte{}
  \question{Soit $G$ un sous-groupe fini de $GL(\R^n)$ et $F=\bigcap_{g\in G} \mathrm{Ker}(g-\mathrm{id})$. 
Montrer que $\mathrm{Card}\,(G)\times\dim F= \sum_{g\in G} \mathrm{tr}(g)$.}
  \reponse{Soit $p=\frac1{\mathrm{Card}\, G}\sum_{g\in G}g$. Alors $g\circ p = p$,
pour tout $g\in G$ donc $p^2=p$,
$F\subset \Im p$ et si $x\in\Im p$, on a $p(x)=x$ d'où $g(x) = x$ pour tout $g\in G$
c'est-à-dire $x\in F$. Donc $F = \Im p$ et $\dim F = \mathrm{rg}(p) = \mathrm{tr}(p)$
(trace d'un projecteur).}
}