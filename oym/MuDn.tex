\uuid{MuDn}
\exo7id{4990}
\titre{Cochléoïde}
\theme{Exercices de Michel Quercia, Courbes paramétrées}
\auteur{quercia}
\date{2010/03/17}
\organisation{exo7}
\contenu{
  \texte{}
\begin{enumerate}
  \item \question{Tracer la courbe $\mathcal{C}$ d'équation polaire $\rho = \frac{\sin\theta}\theta$
    (cochléoïde)}
  \item \question{Une droite passant par $O$ coupe $\mathcal{C}$ en un certain nombre de points.
    Montrer que les tangentes à $\mathcal{C}$ en ces points sont concourantes.}
\end{enumerate}
\begin{enumerate}
  \item \reponse{Repère $(0,\vec u_\theta,\vec v_\theta)$ avec $\theta$ constant
             $ \Rightarrow $ point de concours : $X=\cos\theta$, $Y=\sin\theta$.}
\end{enumerate}
}