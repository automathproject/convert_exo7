\uuid{1598}
\titre{Exercice 1598}
\theme{}
\auteur{legall}
\date{1998/09/01}
\organisation{exo7}
\contenu{
  \texte{}
  \question{Soit $A \in \mathcal{O}_n(\R)$. Montrer que si $-1$ n'est pas valeur
propre de $A$, alors il existe une matrice $Q$ antisym\'etrique (i.e.
${}^tQ = -Q$) telle que $A = (I + Q)^{-1}(I - Q) = (I - Q)(I + Q)^{-1}$ et
qu'on a $A \in \mathcal{SO}_n(\R)$. R\'eciproque ?}
  \reponse{}
}