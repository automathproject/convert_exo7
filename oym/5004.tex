\uuid{5004}
\titre{$I$ reste sur un cercle}
\theme{Exercices de Michel Quercia, Courbes définies par une condition}
\auteur{quercia}
\date{2010/03/17}
\organisation{exo7}
\contenu{
  \texte{}
  \question{Trouver les courbes planes $\mathcal{C}$ telles que le centre de courbure reste
sur un cercle $\mathcal{C}(O,r)$ fixe.
(On prendra $\varphi$ comme paramètre)}
  \reponse{Développante de cercle :
         $\frac{ d\vec I}{d s} = \frac{d R}{d s}\vec N  \Rightarrow 
          \frac{d R}{ d\varphi} = r$.\par
         $ \Rightarrow 
            x = x_0 + r(\cos\varphi-1+\varphi\sin\varphi),\quad
            y = y_0 + r(\sin\varphi-\varphi\cos\varphi)$.}
}