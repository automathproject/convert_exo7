\uuid{4541}
\titre{Ensi Chimie P' 93}
\theme{}
\auteur{quercia}
\date{2010/03/14}
\organisation{exo7}
\contenu{
  \texte{}
  \question{\'Etudier la convergence de la suite de fonctions définies par :
$f_n(x) = \frac{n(n+1)}{x^{n+1}} \int_0^x (x-t)^{n-1}\sin t\,d t$.}
  \reponse{Poser $t = xu$ puis intégrer deux fois par parties :
         $f_n(x) = 1 -  \int_{u=0}^1 (1-u)^{n+1}x\sin(xu)\,d u$
         donc $(f_n)$ converge simplement vers la fonction constante $1$, et
         la convergence est uniforme sur tout intervalle borné.}
}