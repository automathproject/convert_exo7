\uuid{4358}
\titre{Transformée de Laplace}
\theme{}
\auteur{quercia}
\date{2010/03/12}
\organisation{exo7}
\contenu{
  \texte{Soit $f : {[0,+\infty[} \to \R$ continue telle que $ \int_{t=0}^{+\infty}f(t)\,d t$ converge (pas forcément absolument).

On pose $\varphi(a) =  \int_{t=0}^{+\infty}e^{-at}f(t)\,d t$.}
\begin{enumerate}
  \item \question{Montrer que $\varphi$ est de classe $\mathcal{C}^\infty$ sur $]0,+\infty[$.}
  \item \question{Montrer que $\varphi$ est continue en $0$.}
\end{enumerate}
\begin{enumerate}
  \item \reponse{IPP.}
  \item \reponse{Soit $F(x) =  \int_{t=x}^{+\infty}f(t)\,d t$.
    On a $\varphi(a) = F(0) -a \int_{t=0}^{+\infty} e^{-at}F(t)\,d t$.

    Soit $\varepsilon > 0$ et $A$ tel que $x>A  \Rightarrow  |F(x)| \le \varepsilon$.
    On a~:
    $$\left|a \int_{t=0}^{+\infty} e^{-at}F(t)\,d t\right| \le
      a\sup{|F(t)|, t\in{[0,A]}} + \varepsilon e^{-aA} \le 2\varepsilon$$
    pour $a$ suffisament petit.}
\end{enumerate}
}