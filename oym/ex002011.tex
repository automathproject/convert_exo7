\uuid{2011}
\titre{Exercice 2011}
\theme{}
\auteur{liousse}
\date{2003/10/01}
\organisation{exo7}
\contenu{
  \texte{}
\begin{enumerate}
  \item \question{Trouver une équation du plan $(P)$ 
défini par les éléments suivants.


\begin{enumerate}}
  \item \question{$A$, $B$ et $C$ sont des points de $(P)$

\begin{enumerate}}
  \item \question{$A(0,0,1)$, $B(1,0,0)$ et $C(0,1,0)$.}
  \item \question{$A(1,1,1)$, $B(2,0,1)$ et $C(-1,2,4)$.}
\end{enumerate}
\begin{enumerate}
  \item \reponse{\begin{enumerate}}
  \item \reponse{Une équation d'un plan est $ax+by+cz+d=0$. 
Si un point appartient à un plan cela donne une condition linéaire sur $a,b,c,d$. 
Si l'on nous donne trois point cela donne un système linéaire de trois équations à trois inconnues
(car l'équation est unique à un facteur multplicatif non nul près). On trouve : 
    \begin{enumerate}}
  \item \reponse{$x+y+z-1=0$}
  \item \reponse{$3x+3y+z-7=0$}
\end{enumerate}
}