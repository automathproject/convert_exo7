\uuid{4033}
\titre{DL de $(1-e^x)^n$}
\theme{}
\auteur{quercia}
\date{2010/03/11}
\organisation{exo7}
\contenu{
  \texte{}
  \question{Développer de deux manières $(1-e^x)^n$ en $0$ à l'ordre $n+2$.

En déduire $\sum_{k=0}^n(-1)^kC_n^kk^p$ pour $p = 0,1,\dots,n+2$.}
  \reponse{$(1-e^x)^n = \sum_{k=0}^n(-1)^kC_n^ke^{kx}
 = \sum_{p=0}^{n+2} \Bigl(\sum_{k=0}^n(-1)^kC_n^kk^p \Bigr) \frac {x^p}{p!}
   + o (x^{n+2})$,

$(1-e^x)^n = (-x)^n\left({1 + \frac {nx}2 + \frac {n(3n+1)}{24}x^2 + o  (x^2)}\right)$.}
}