\uuid{1429}
\titre{Exercice 1429}
\theme{}
\auteur{legall}
\date{1998/09/01}
\organisation{exo7}
\contenu{
  \texte{}
  \question{Soit $  G  $ un groupe, $  A  $ une partie non vide de $  G  .$
On note $  N(A)=\{ g\in G  ;   gAg^{-1}=A\}  $ et $  C(A)=\{ g\in G  ; \forall a \in A  ;
gag^{-1}=a\}  .$ Montrer que $  N(A)   $ et $  C(A)  $ sont des sous-groupes de $  G  $ et que $  C(A)  $ est un sous-groupe distingu\'e de $  N(A)  .$}
  \reponse{}
}