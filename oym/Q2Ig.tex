\uuid{Q2Ig}
\exo7id{4510}
\titre{$f(nx)$, $f(x/n)$}
\theme{Exercices de Michel Quercia, Suites et séries de fonctions}
\auteur{quercia}
\date{2010/03/14}
\organisation{exo7}
\contenu{
  \texte{Soit $f : {\R^+} \to \R$ continue, non identiquement nulle, telle que $f(0) = 0$
et $f(x) \to 0$ lorsque $x\to+\infty$.

On pose $f_n(x) = f(nx)$ et $g_n(x) = f\left(\frac xn\right)$.}
\begin{enumerate}
  \item \question{Donner un exemple de fonction $f$.}
  \item \question{Montrer que $f_n$ et $g_n$ convergent simplement vers la fonction nulle,
    et que la convergence n'est pas uniforme sur $\R^+$.}
  \item \question{Si $ \int_{t=0}^{+\infty} f(t)\,d t$ converge, chercher
       $\lim_{n\to\infty}  \int_{t=0}^{+\infty} f_n(t)\,d t$ et
       $\lim_{n\to\infty}  \int_{t=0}^{+\infty} g_n(t)\,d t$.}
\end{enumerate}
\begin{enumerate}

\end{enumerate}
}