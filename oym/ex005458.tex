\uuid{5458}
\titre{**}
\theme{}
\auteur{rouget}
\date{2010/07/10}
\organisation{exo7}
\contenu{
  \texte{}
  \question{Montrer que  $\sum_{k=1}^{n}\sin\frac{k}{n}=\frac{1}{2}+\frac{1}{2n}+o(\frac{1}{n})$.}
  \reponse{Soit $n\in\Nn^*$.

\begin{align*}\ensuremath
\sum_{k=1}^{n}\sin\frac{k}{n^2}&=\mbox{Im}(\sum_{k=1}^{n}e^{ik/n^2})=\mbox{Im}\left(e^{i/n^2}\frac{1-e^{ni/n^2}}{1-e^{i/n^2}}\right)
=\mbox{Im}\left(e^{i(1+\frac{n}{2}-\frac{1}{2})/n^2}\frac{\sin\frac{1}{2n}}{\sin\frac{1}{2n^2}}\right)
=\frac{\sin\frac{n+1}{2n^2}\sin\frac{1}{2n}}{\sin\frac{1}{2n^2}}\\
 &\underset{n\rightarrow+\infty}{=}(\frac{1}{2n}+\frac{1}{2n^2}+o(\frac{1}{n^2}))(\frac{1}{2n}+o(\frac{1}{n^2}))(\frac{1}{2n^2}+o(\frac{1}{n^3}))^{-1}\\
 &=(1+\frac{1}{n}+o(\frac{1}{n}))(\frac{1}{2}+o(\frac{1}{n}))(1+o(\frac{1}{n}))^{-1}=\frac{1}{2}+\frac{1}{2n}
 +o(\frac{1}{n}),
\end{align*}

(on peut aussi partir de l'encadrement $\frac{k}{n^2}-\frac{k^3}{6n^6}\leq\sin\frac{k}{n^2}\leq\frac{k}{n^2}$).}
}