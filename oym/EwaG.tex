\uuid{EwaG}
\exo7id{7419}
\titre{Dans $\Rr^3$, à partir d'équations}
\theme{Exercices de Christophe Mourougane, Géométrie euclidienne}
\auteur{mourougane}
\date{2021/08/10}
\organisation{exo7}
\contenu{
  \texte{Soit l'espace affine $\Rr^3$ muni d'un repère affine
$A_0,A_1,A_2,A_3$.}
\begin{enumerate}
  \item \question{Montrer que le sous-ensemble $A$ de l'espace affine $\Rr^3$ d'équation 
$$M\left(\begin{array}{c}
x\\y\\z\end{array}\right)\in A\iff \left\{ 
\begin{array}{c}
2x-y-z-3=0\\x+y-2z=3
\end{array}
\right.$$
est un sous-espace affine. Préciser sa dimension, l'espace vectoriel
directeur et un repère affine.}
  \item \question{Même question avec $B$ d'équation 
$$M\left(\begin{array}{c}
x\\y\\z\end{array}\right)\in B\iff \left\{ 
\begin{array}{c}
x+y=2\\2x+2y=3z+1\\5x+5y=10z
\end{array}
\right.$$}
\end{enumerate}
\begin{enumerate}

\end{enumerate}
}