\exo7id{2211}
\titre{Quelques identit\'es pour le calcul d'inverses}
\theme{}
\auteur{matos}
\date{2008/04/23}
\organisation{exo7}
\contenu{
  \texte{D\'emontrer l'identit\'e
$$ (A + UBV)^{-1} = A^{-1} -A^{-1} U( I+BVA^{-1}U)^{-1} BVA^{-1}$$
en pr\'ecisant:
\begin{itemize}
\item son domaine de validit\'e;
\item les types des matrices $A, U, B, V$.
\end{itemize}
Quelques cas particuliers:}
\begin{enumerate}
  \item \question{Supposons $B=\beta$ scalaire, $U=u\in\Rr^n$, $V=v^T \in\Rr^n$. Retrouver la formule de Shermann--Morrison qui permet le calcul de l'inverse d'une matrice qui apparait comme perturbation de rang 1 d'une matrice dont on connait l'inverse.}
  \item \question{Soient $A\in\Rr^{n\times n} $ r\'eguli\`ere et $u,v\in\Rr^n$ tels que $1+v^TA^{-1}u=0$. Montrer que
$$B=\left(\begin{array}{cc}
A+uv^T&u\\
v^T& 0\end{array}\right) \mbox{ est r\'eguli\`ere.}$$
Calculer $B^{-1}$ en remarquant que
$$B=\left[\begin{array}{cc}
A&0\\
0&-1\end{array}\right]
 +\left[\begin{array}{c}
u\\
1\end{array}\right] \left[\begin{array}{cc}
v^T&1
\end{array}\right]$$}
  \item \question{Soit 
$$D=\left[\begin{array}{cc}
P&Q\\
R&S\end{array}\right] \mbox{ matrice inversible  avec } P\in\Rr^{p\times p}, Q\in\Rr^{p\times q}, S\in\Rr^{q\times q}$$
Calculer $D^{-1}$ en remarquant que
$$D=\left[\begin{array}{cc}
P&0\\
R&I\end{array}\right] +\left[\begin{array}{c}
Q\\S-I\end{array}\right]\left[\begin{array}{cc}
0&I\end{array}\right]$$}
  \item \question{{\it Calcul r\'ecursif de l'inverse}: on pose
$$A_n=\left[\begin{array}{cc}A_{n-1}&v\\
u^T&s\end{array}\right] \mbox{ avec } A_{n-1}\in\Rr^{(n-1)\times (n-1)} u,v\in\Rr^{n-1}, s\in\Rr$$
Utiliser la formule pr\'ec\'edente pour calculer $A_n^{-1}$ en fonction de $A_{n-1}^{-1}$. En d\'eduire un algorithme r\'ecursif pour le calcul de l'inverse d'une matrice carr\'ee de taille $n$.}
\end{enumerate}
\begin{enumerate}
  \item \reponse{On obtient la formule de Shermann-Morrisson:
$$(A+\beta uv^T)^{-1} = A^{-1} - \frac{\beta}{1+\beta v^TA^{-1}u} A^{-1} uv^T A^{-1}$$
qui permet le calcul de l'inverse d'une matrice qui apparait comme perturbation de rang 1 d'une matrice dont on connait l'inverse.}
  \item \reponse{$$B\left(\begin{array}{c} X\\y\end{array}\right)=0\Leftrightarrow \left\{ \begin{array}{l} (A+uv^T)x +yu =0\\ v^Tx=0\end{array}\right. \Leftrightarrow \left\{ \begin{array}{l} x=-yA^{-1}u\\v^Tx=-yv^TA^{-1}u=0\end{array}\right.$$
ce qui donne $x=0, y=0$ et donc $B$ est inversible.}
  \item \reponse{En appliquant la formule g\'en\'erale on obtient
$$B^{-1}=\left(\begin{array} {cc}
A^{-1} -A^{-1}uv^TA^{-1} & A^{-1}u\\ v^TA^{-1} & 0\end{array}\right)$$}
  \item \reponse{En appliquant la m\^eme formule on obtient
$$D^{-1}=\left(\begin{array}{cc}P^{-1} +P^{-1}Q\Delta^{-1} RP^{-1} & -P^{-1}Q\Delta^{-1}\\ -\Delta^{-1} RP^{-1} & \Delta^{-1}\end{array}\right)$$
avec $\Delta = S-RP^{-1}Q$.}
  \item \reponse{Calcul r\'ecursif de l'inverse : on dispose de $A_{n-1}^{-1}$ de taille $(n-1)\times (n-1)$ et on veut calculer l'inverse de
$$A_n=\left(\begin{array}{cc} A_{n-1}& v\\
u^T & s\end{array}\right)\mbox{ avec } u,v\in\Rr^{n-1}, s\in\Rr $$
en utilisant la formule pr\'ec\'edente on obtient
$$A_n^{-1}=\left(\begin{array}{cc}
A_{n-1}^{-1} +\frac{1}{\delta} A_{n-1}^{-1} v u^T A_{n-1}^{-1} & - A_{n-1}^{-1}\frac{v}{\delta}\\
-u^TA_{n-1}^{-1}/\delta & \frac{1}{\delta}\end{array}\right)$$
avec $\delta=s-u^TA_{n-1}^{-1}v$.

et on en d\'eduit facilement l'algorithme.}
\end{enumerate}
}