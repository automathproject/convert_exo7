\uuid{U3PW}
\exo7id{4929}
\titre{Sections circulaires}
\theme{Exercices de Michel Quercia, Quadriques}
\auteur{quercia}
\date{2010/03/17}
\organisation{exo7}
\contenu{
  \texte{}
\begin{enumerate}
  \item \question{On considère la forme quadratique $q(x,y,z) = ax^2 + by^2 + cz^2$
    avec $a \in {[b,c]}$.
  \begin{enumerate}}
  \item \question{Montrer qu'il existe $y,z \in \R$ tels que $y^2 + z^2 = 1$ et
        $by^2 + cz^2 = a$.}
  \item \question{En déduire qu'il existe une base orthonormée de $\R^3$ dans laquelle la
        matrice de $q$ est de la forme :
        $M = \begin{pmatrix} a &0 &* \cr 0 &a &* \cr * &* &* \cr\end{pmatrix}$.}
\end{enumerate}
\begin{enumerate}
  \item \reponse{Non lorsque la valeur propre médiane est nulle
             (paraboloïde hyperbolique, cylindre hyperbolique,
	     cylindre parabolique, plans).}
\end{enumerate}
}