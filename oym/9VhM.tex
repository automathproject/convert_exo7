\uuid{9VhM}
\exo7id{3242}
\titre{Mines MP 2001}
\theme{Exercices de Michel Quercia, Racines de polynômes}
\auteur{quercia}
\date{2010/03/08}
\organisation{exo7}
\contenu{
  \texte{Soit $ K$ un corps de caract{\'e}ristique $p$.}
\begin{enumerate}
  \item \question{Montrer que $\sigma\ :\ x \mapsto x^p$ est un morphisme de corps.}
  \item \question{Montrer que $\sigma$ est surjectif si et seulement si tout polyn{\^o}me
    $P\in K[X]$ irr{\'e}ductible v{\'e}rifie $P'\ne 0$.}
\end{enumerate}
\begin{enumerate}
  \item \reponse{$p$ est premier car $ K$ est int{\`e}gre.

    On a $1^p = 1$,
    $(xy)^p = x^py^p$ (un corps est commutatif) et
    $(x+y)^p = x^p + y^p + \sum_{k=1}^{p-1}C_p^kx^ky^{p-k} = x^p + y^p$
    car $p$ divise $C_p^k$ si $1\le k \le p-1$.}
  \item \reponse{Remarquer que $P'=0 \Leftrightarrow P\in  K[X^p]$.
    
    On suppose $\sigma$ surjectif. Soit $P(X) = Q(X^p) = a_0 + \dots + a_kX^{kp}$
    un polyn{\^o}me non constant {\`a} d{\'e}riv{\'e}e nulle. Il existe $b_0,\dots,b_k$
    tels que $b_i^p = a_i$. Alors $P(X) = Q(X)^p$ est r{\'e}ductible.
    
    On suppose que tout polyn{\^o}me irr{\'e}ductible a une d{\'e}riv{\'e}e non nulle.
    Soit $a\in  K$ et $P(X) = X^p - a$. $P'=0$ donc $P$ est r{\'e}ductible.
    Soit $Q$ un facteur unitaire irr{\'e}ductible de $X^p-a$. Alors $Q^p$ et $X^p-a$
    ont $Q$ en facteur commun donc leur pgcd, $D$, est non constant.
    Mais $Q^p$ et $X^p-a$ appartiennent {\`a} $ K[X^p]$ donc $D$, obtenu par l'algorithme
    d'Euclide aussi, d'o{\`u} $D = X^p-a$ et $X^p-a$ divise $Q^p$. Par unicit{\'e} de
    la d{\'e}composition de $Q^p$ en facteurs irr{\'e}ductibles, il existe $r\in\N$
    tel que $X^p-a = Q^r$. Par examen du degr{\'e} on a $r=p$ donc $\deg Q = 1$,
    $Q = X-b$ et finalement $b^p=a$.}
\end{enumerate}
}