\uuid{yiQN}
\exo7id{3450}
\titre{Formule de Cauchy-Binet}
\theme{Exercices de Michel Quercia, Déterminants}
\auteur{quercia}
\date{2010/03/10}
\organisation{exo7}
\contenu{
  \texte{Soit $M=(a_{ij})\in\mathcal{M}_{np}(K)$ et $q\in[[1,\min(n,p)]]$.

Pour $X=\{x_1,\dots,x_q\}$
et $Y=\{y_1,\dots,y_q\}$
avec $1\le x_1< x_2< \dots< x_q\le n$ et $1\le y_1< y_2< \dots< y_q\le p$
on note $\Delta_{X,Y}(M)$ le déterminant de la matrice $q\times q$ de terme général $a_{x_i,y_j}$.}
\begin{enumerate}
  \item \question{Soient $M\in\mathcal{M}_{np}(K)$ et $N\in\mathcal{M}_{pn}(K)$ avec $n\le p$.
    Montrer que $\det(MN) = \sum_{X\subset[[1,p]];\mathrm{Card}\, X =
    n}\Delta_{[[1,n]],X}(M)\Delta_{X,[[1,n]]}(N)$
    (considérer les deux membres comme des fonctions des colonnes de~$N$).}
  \item \question{Donner une formule pour $\det(MN)$ quand $n>p$.}
  \item \question{Soient $M\in\mathcal{M}_{np}(K)$, $N\in\mathcal{M}_{pq}(K)$ et $r\in{[[1,\min(n,q)]]}$.
    Montrer, pour $X\subset{[[1,n]]}$ et $Y\subset{[[1,q]]}$ avec $\mathrm{Card}\,(X) =
    \mathrm{Card}\,(Y) = r$~: $\Delta_{X,Y}(MN) = \sum_{Z\subset{[[1,p]]} ;\mathrm{Card}\,(Z)=r}\Delta_{X,Z}(M)\Delta_{Z,Y}(N)$.}
\end{enumerate}
\begin{enumerate}

\end{enumerate}
}