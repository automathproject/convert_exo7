\exo7id{6061}
\titre{Exemple de topologie non séparée}
\theme{}
\auteur{queffelec}
\date{2011/10/16}
\organisation{exo7}
\contenu{
  \texte{Dans $\Cc$, on note $[z_0\rightarrow[$ la demi-droite $\{\rho e^{i\theta_0}\
;\ \rho\geq\rho_0\}$, si $z_0=\rho_0 e^{i\theta_0}$. On déclare ouvert toute
réunion (éventuellement vide) de telles demi-droites.}
\begin{enumerate}
  \item \question{Montrer qu'on a ainsi défini sur $\Cc$ une topologie $\cal T$ non séparée.}
  \item \question{Montrer que l'adhérence du point $\{z_0\}$ pour cette topologie est
$[0,z_0]$.}
  \item \question{En déduire que les fermés de $\cal T$ sont les ensembles étoilés par rapport
à $0$ ($A$ est dit ``étoilé par rapport à $0$" si, pour tout $z\in A$, le
segment $[0,z]$ est encore dans $A$).}
\end{enumerate}
\begin{enumerate}
  \item \reponse{L'ensemble $\emptyset$ est réunion vide de demi-droites. Il est clair
que $\C$ est dans $\cal T$ si, lorsque $z_0=0$, toute demi-droite
$[0,\rightarrow[$ est admissible (ce qui aurait dû être précisé dans
l'énoncé...Excuses!); par ailleurs cette famille d'ensembles est bien
stable par réunion quelconque. Remarquons que si $U\in\cal T$, $z\in
U\Longleftrightarrow [z,\rightarrow[\subset U$ (c'est-à-dire les
demi-droites forment une base d'ouverts). En effet si $z\in U$ il existe
$z'$ tel que $z\in [z',\rightarrow[\subset U$ et a fortiori
$[z,\rightarrow[\subset U$. Ainsi la famille est stable par intersection
finie : si $z\in U\cap U'$, $[z,\rightarrow[\subset U\cap U'$ et $U\cap
U'\in\cal T$. On a bien une topologie.

Cette topologie n'est pas séparée puisque, si $z\in [z',\rightarrow[$,
tout voisinage de $z'$ contient $z$.}
  \item \reponse{La topologie n'étant pas même quasi-séparée, un singleton peut ne pas
être fermé. Soit $z_0$; $z\in \overline{\{z_0\}}$ si et seulement si tout
voisinage de
$z$ rencontre $z_0$; puisque tout voisinage de $z$ contient $[z,\rightarrow[$,
c'est équivalent à 
$z_0\in [z,\rightarrow[$ ou encore $z\in [0,z_0]$.

$\{0\}$ est le seul singleton fermé.}
  \item \reponse{On déduit de ce qui précède que si $A\subset X$, $\overline
A=\cup_{z\in A}[0,z]$. En effet si $z\in A$,  $\overline A$ contient
$\overline {\{z\}}$ et donc le segment
$[0,z]$; réciproquement, si $z'\in[0,z]$ avec $z\in A$, $[z',\rightarrow[$,
et donc tout voisinage de $z'$, rencontre $A$; ainsi $z'\in\overline A$.


Supposons $A$ étoilé
par rapport à $0$; $A$ contient $\cup_{z\in A}[0,z]=\overline A$ et $A$ est
fermé.

Réciproquement supposons $A$ fermé; $A=\cup_{z\in A}[0,z]$ et en particulier
$A$ est étoilé par rapport à $0$.}
\end{enumerate}
}