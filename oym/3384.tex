\uuid{3384}
\titre{$A > 0, X > 0$ et $A^kX = X$}
\theme{Exercices de Michel Quercia, Matrices}
\auteur{quercia}
\date{2010/03/09}
\organisation{exo7}
\contenu{
  \texte{}
  \question{Soit $A \in \mathcal{M}_{n,p}(\R)$. On dit que $A$ est positive si tous ses
coefficients sont strictement positifs.

Soit $M \in \mathcal{M}_n(\R)$ positive. On suppose qu'il existe $X \in \mathcal{M}_{n,1}(\R)$
positif et $k\in\N^*$ tels que $M^kX = X$.
Montrer qu'il existe $Y \in \mathcal{M}_{n,1}(\R)$ positif tel que $MY = Y$.}
  \reponse{}
}