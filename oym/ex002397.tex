\uuid{2397}
\titre{Exercice 2397}
\theme{}
\auteur{mayer}
\date{2003/10/01}
\organisation{exo7}
\contenu{
  \texte{\label{exocomp}
On consid\`ere l'espace des fonctions continues $X={\cal
C}([a,b])$.}
\begin{enumerate}
  \item \question{Soit $\omega \in X$ une fonction qui ne s'annule pas sur
$[a,b]$. Posons $$d_\omega (f,g) = \sup_{t\in[a,b]} |\omega (t)
(f(t)-g(t))| \; .$$
 L'espace $(X,d_\omega)$ est-il complet?}
  \item \question{Montrer que l'espace $(X, \|.\|_1)$ n'est pas complet (o\`u
 $\|f\|_1 = \int _0^1 |f(t)| \, dt$).}
\end{enumerate}
\begin{enumerate}
  \item \reponse{\begin{enumerate}}
  \item \reponse{Montrons que $(X,d_\omega)$ est complet. Soit $(f_n)_n$ une suite de Cauchy pour cet distance. Alors pour chaque $t\in[a,b]$, $(f_n(t))_n$ est une suite de Cauchy pour $(\Rr,|.|)$. Comme $\Rr$ est complet alors cette suite converge, notons $f(t)$ sa limite.

Il faut montrer deux choses : premièrement que $(f_n)$ converge vers $f$ pour la distance considérée, deuxièmement que $f$ est bien dans l'espace $X$.}
  \item \reponse{Comme $(f_n)$ est une suite de Cauchy. Pour $\epsilon > 0$. Il existe $n\ge0$ tel que pour tout $p\ge 0$ : $d_\omega(f_n,f_{n+p})  < \epsilon.$
Donc 
$$ \sup_{t\in[a,b]} |\omega (t)
(f_n(t)-f_{n+p}(t))| < \epsilon.$$
On fait tendre $p$ vers $+\infty$ et on obtient :
$\sup_{t\in[a,b]} |\omega (t) (f_n(t)-f(t))| < \epsilon.$
Donc $(f_n)$ converge vers $f$ pour la distance $d_\omega$.}
  \item \reponse{$\omega$ est une fonction non nulle sur le compact $[a,b]$, donc 
il existe $\alpha >0$ tel que $\omega(t)>\alpha$ pour tout $t\in[a,b]$.
On en déduit que 
$$\|f_n-f\|_\infty \le \frac 1 \alpha d_\omega(f_n,f).$$
Comme $ d_\omega(f_n,f)$ tend vers $0$ alors $f_n$ converge vers $f$ pour la norme infini. Donc $f$ est continue.

Conclusion : $(X,d_\omega)$ est complet.}
\end{enumerate}
}