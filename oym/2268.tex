\uuid{2268}
\titre{Exercice 2268}
\theme{Anneaux de polynômes I}
\auteur{barraud}
\date{2008/04/24}
\organisation{exo7}
\contenu{
  \texte{}
  \question{Soient
$$
f(x)=(x-a_1)(x-a_2)\dots (x-a_n)-1,\quad
g(x)=(x-a_1)^2(x-a_2)^2\dots (x-a_n)^2+1
$$
o\`u $a_1,\dots a_n\in \Zz$ soient deux \`a deux distincts.
Montrer que  $f$ et $g$ sont irr\'eductibles dans $\Qq[x]$.}
  \reponse{Commençons par montrer que ces polynômes sont irréductibles sur $\Zz$.


  \paragraph{-Le cas de $f=\prod_{i=1}^{n}(X-a_{i})-1$}
  Soit $P,Q\in\Zz[X]$ tels que $f=PQ$. On peut supposer sans perte de
  généralité que $P$ et $Q$ ont des coefficients dominants positifs (i.e.
  sont unitaires).

  On a~: $\forall i, f(a_{i})= P(a_{i})Q(a_{i})=-1$ donc
  $$
   P(a_{i})=\pm1
  \qquad\text{ et }\qquad
   Q(a_{i})=\mp1
  $$
  Soit $I=\{i, P(a_{i})=-1\}$ et $J=\{1,\dots,n\}\setminus I$. On notera
  $|I|$ et $|J|$ le nombre d'éléments de $I$ et $J$.

  \textit{Supposons $I\neq\emptyset$ et $J\neq\emptyset$}: Alors
  $\prod_{i\in I}(X-a_{i})|(P+1)$ et $\prod_{i\in J}(X-a_{i})|(Q+1)$.
  Ainsi $\deg(P+1)\geq |I|$ et $\deg(Q+1)\geq |J|=n-|I|$, et comme
  $\deg(P)+\deg(Q)=n$, on en déduit que $\deg(P)=|I|$ et $\deg(Q)=|J|$,
  puis que (puisque $P$ et $Q$ sont unitaires)~:
  $$
    P=\prod_{i\in I}(X-a_{i})-1
    \qquad\text{ et }\qquad
    Q=\prod_{i\in J}(X-a_{i})-1.
  $$
  
  Ainsi $f=\prod_{k\in I\cup J}(X-a_{k})-1=(\prod_{i\in
    I}(X-a_{i})-1)(\prod_{j\in J}(X-a_{j})-1)=f-\big(\prod_{i\in
    I}(X-a_{i})+\prod_{j\in J}(X-a_{j})-2\big)$, donc $\prod_{i\in
    I}(X-a_{i})+\prod_{j\in J}(X-a_{j})-2=0_{\Zz[X]}$, ce qui est faux.

  Ainsi $I=\emptyset$ ou $J=\emptyset$. On peut supposer sans perte de
  généralité que $I=\emptyset$. Alors $\forall i\in\{1,\dots,n\},
  Q(a_{i})=-1$. Donc les $a_{i}$ sont tous racine de $Q+1$. Comme
  $\deg(Q+1)\leq n$ et $Q+1\neq 0$, on en déduit que $Q=f$, et $P=1$. $f$
  est donc bien irréductible dans $\Zz[X]$.

  \medskip
  \paragraph{-Le cas de $g=\prod_{i=1}^{n}(X-a_{i})^{2}+1$}. Supposons que $g=PQ$,
  avec $P,Q\in\Zz[X]$.On a $g(a_{i})=1=P(a_{i})Q(a_{i})$, donc
  $P(a_{i})=Q(a_{i})=\pm1$.

  Comme $g$ n'a pas de racine réelle, il en va de même de $P$ et $Q$, qui
  sont donc de signe constant (théorème des valeurs intermédiaires pour
  les fonctions continues sur $\Rr$~!). On peut donc supposer sans perte
  de généralité que $P$ et $Q$ sont positifs.

  Alors $P(a_{i})=Q(a_{i})=1$. Ainsi, tous les $a_{i}$ sont racines de
  $P-1$ et de $Q-1$. On a donc $\prod_{i=1}^{n}(X-a_{i})|P-1$ et
  $\prod_{i=1}^{n}(X-a_{i})|Q-1$.

  En particulier, si $P-1\neq0$ et $Q-1\neq0$, $\deg(P)\geq n$ et
  $\deg(Q)=2n-\deg(P)\geq n$. Ainsi $\deg(P)=\deg(Q)=n$. Comme en plus
  $P$ et $Q$ sont unitaires, on en déduit que 
  $$
    P-1=\prod_{i=1}^{n}(X-a_{i}) \qquad\text{ et }
    Q-1=\prod_{i=1}^{n}(X-a_{i}).
  $$
  On devrait donc avoir
  $(\prod_{i=1}^{n}(X-a_{i})+1)^{2}=\prod_{i=1}^{n}(X-a_{i})^{2}+1$, ce
  qui est faux ($\prod_{i=1}^{n}(X-a_{i})\neq 0_{\Zz[X]}$)~!

  Ainsi $P-1=0$ ou $Q-1=0$, et on en déduit bien que $g$ est irréductible
  dans $\Zz[X]$.


  
  \bigskip
  \paragraph{Irréductibilité dans $\Qq[X]$}
 On a le lemme suivant~:

  Si $P\in\Zz[X]$ est unitaire et irréductible dans $\Zz[X]$, alors il
  l'est aussi dans $\Qq[X]$.

  L'ingrédient de base de la démonstration est la notion de
  \emph{contenu} d'un polynôme $P\in\Zz[X]$~: c'est le $\pgcd$ de ses
  coefficients, souvent noté $c(P)$. Il satisfait la relation suivante :
  $$
    c(PQ)=c(P)c(Q).
  $$

  Supposons que $P=QR$, avec $Q,R\in\Qq[X]$, $Q$ et $R$ unitaires. En
  réduisant tous leurs coefficients de au même dénominateur, on peut
  mettre $Q$ et $R$ sous la forme~:
  $$
    Q=\frac{1}{a}Q_{1} 
    \qquad\text{ et }\qquad
    R=\frac{1}{b}R_{1} 
  $$
  avec $a,b\in\Zz$, $Q_{1},R_{1}\in\Zz[X]$ et $c(Q_{1})=1$, $c(R_{1})=1$.

  Alors $abP=Q_{1}R_{1}$, donc $c(abP)=c(Q_{1})c(R_{1})=1$. Comme
  $ab|c(abP)$, on a $ab=\pm1$, et en fait $P,Q\in\Zz[X]$.}
}