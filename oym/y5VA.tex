\uuid{y5VA}
\exo7id{2340}
\titre{Exercice 2340}
\theme{Topologie générale}
\auteur{queffelec}
\date{2003/10/01}
\organisation{exo7}
\contenu{
  \texte{}
\begin{enumerate}
  \item \question{Rappeler les d\'efinitions d'une borne sup\'erieure (inf\'erieure) d'un ensemble de
nombres r\'eels. Si $A$ et $B$ sont deux ensembles born\'es non vides de $\Rr$, comparer 
avec $\sup A$, $\inf A$, $\sup B$ et $\inf B$ les nombres suivants : 

\quad (i) $\sup(A+B)$, \quad (ii) $\sup(A\cup B)$, \quad (iii) $\sup(A\cap
B)$, \quad (iv) $\inf(A\cup B)$, \quad (v) $\inf(A\cap B)$.}
  \item \question{Pour $x\in\Rr^n$ et $A\subset \Rr^n$ on d\'efinit $d(x,A)=\inf_{a\in A}
||x-a||$. Trouver $d(0,\Rr-\Qq)$, $d(\sqrt2,\Qq)$, $d(M,{\cal D})$ o\`u
$M=(x,y,z)\in\Rr^3$ et $\cal D$ est la droite de vecteur unitaire $(a,b,c)$.}
  \item \question{Pour  $A, B\subset \Rr^n$ on d\'efinit $d(A,B)=\inf_{a\in A,b\in B}
||a-b||$. Trouver $d(A,B)$ lorsque $A$ est une branche de l'hyperbole
$\{(x,y)\in{\Rr^2}\ ;\  xy=1\}$ et $B$ une asymptote.}
  \item \question{On d\'efinit $\hbox {diam} A=\sup_{a,b\in A}||a-b||$. Quel est $\hbox
{diam} (]0,1[\cap\Qq)$ ? $\hbox{diam} ([0,1]\cap\Rr-\Qq)$ ?}
\end{enumerate}
\begin{enumerate}
  \item \reponse{$A$ une partie non vide de $\Rr$, un \emph{majorant} de $A$ est un 
réel $M\in \Rr$ tel que
$$ \forall x \in A \qquad x \leq M.$$
Si $A$ est un partie non vide et majorée, alors par définition $\sup A$ est le plus petit des majorants.
On a les propriétés suivantes :
  \begin{enumerate}}
  \item \reponse{$\sup(A+B)=\sup A+\sup B$ ;}
  \item \reponse{$\sup(A \cup B)=\max(\sup A,\sup B)$ ;}
  \item \reponse{$\max(\inf A, \inf B) \leq \sup(A\cap B) \leq \min(\sup A, \sup B)$ si $A\cap B\neq \varnothing$ ;}
  \item \reponse{$\inf(A \cup B)=\min(\inf A, \inf B)$ ;}
  \item \reponse{$\max(\inf A, \inf B) \leq \inf(A\cap B) \leq \min(\sup A, \sup B)$ si $A\cap B\neq \varnothing$ ;}
\end{enumerate}
}