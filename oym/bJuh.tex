\uuid{bJuh}
\exo7id{4350}
\titre{Fonction définie par une intégrale}
\theme{Exercices de Michel Quercia, Intégrale dépendant d'un paramètre}
\auteur{quercia}
\date{2010/03/12}
\organisation{exo7}
\contenu{
  \texte{Soit $f:\R \to \R$ continue. On définit pour $x \in \R^*$ et $y \in \R$ :
$g(x,y) = \frac 1x  \int_{t=x}^{xy} f(t)\,d t$.}
\begin{enumerate}
  \item \question{Montrer que $g$ peut être prolongée en une fonction continue sur $\R^2$.}
  \item \question{On suppose de plus $f$ dérivable en 0. Montrer que $g$ est de classe $\mathcal{C}^1$.}
\end{enumerate}
\begin{enumerate}
  \item \reponse{$g(x,y) =  \int_{u=1}^y f(ux)\,d u$.}
  \item \reponse{$\frac{\partial g}{\partial x} =
             \frac1x \Bigl( yf(xy)-f(x) -  \int_{u=1}^y f(ux)\,d u\Bigr)
             \to \frac{y^2-1}2f'(0)$ lorsque $x\to0$.}
\end{enumerate}
}