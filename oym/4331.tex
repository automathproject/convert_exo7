\uuid{4331}
\titre{$(1-x/n)^n$, Ensi PSI 1998}
\theme{Exercices de Michel Quercia, Intégrale dépendant d'un paramètre}
\auteur{quercia}
\date{2010/03/12}
\organisation{exo7}
\contenu{
  \texte{}
  \question{Soit $x\in{[0,n]}$. Montrer que $(1-x/n)^n \le e^{-x}$.
En déduire $\lim_{n\to\infty}  \int_{x=0}^n(1-x/n)^n\,d x$.}
  \reponse{Soit $f_n(x) = (1-x/n)^n$ si $0\le x \le n$ et $f_n(x) = 0$ si
$x>n$. Alors $f_n(x)$ converge simplement vers $e^{-x}$ et il y a
convergence dominée.}
}