\uuid{rVF0}
\exo7id{2735}
\titre{Exercice 2735}
\theme{Résolution de systèmes linéaires par la méthode du Pivot de Gauss}
\auteur{tumpach}
\date{2009/10/25}
\organisation{exo7}
\contenu{
  \texte{Soit $a$ un nombre r\'eel. On \'etudie le syst\`eme lin\'eaire suivant~:
$$
\mathcal{S}_{a}~: \left\{\begin{array}{ccccccc} 
x & - & 2y & + & 3z & = & 2\\
x & + & 3y & - & 2z & = & 5\\
2x & - & y & + & az & = & 1
\end{array}\right.
$$}
\begin{enumerate}
  \item \question{En fonction des valeurs du param\`etre $a$, d\'eterminer si le syst\`eme $\mathcal{S}_{a}$ peut~:
\begin{enumerate}}
  \item \question{[(i)] n'admettre aucune solution ;}
  \item \question{[(ii)] admettre exactement une solution ;}
  \item \question{[(iii)] admettre une infinit\'e de solutions.}
\end{enumerate}
\begin{enumerate}

\end{enumerate}
}