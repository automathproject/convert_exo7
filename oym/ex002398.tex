\uuid{2398}
\titre{Exercice 2398}
\theme{}
\auteur{mayer}
\date{2003/10/01}
\organisation{exo7}
\contenu{
  \texte{Soit $X={\cal C}^1([a,b])$.}
\begin{enumerate}
  \item \question{Est-ce un espace complet si on le muni de la norme
uniforme $\| . \| _\infty $?}
  \item \question{Consid\'erons maintenant, pour $f\in X$, la norme
$$ N(f) = \sup _{t\in [a,b]} \|f(t)\| + \sup _{t\in [a,b]}
\|f'(t)\|\; .$$ L'espace $(X,N)$ est-il complet?}
\end{enumerate}
\begin{enumerate}
  \item \reponse{On reprend l'exemple de l'exercice \ref{exocomp}. Et on définit
 $g_n$ sur $[0,1]$ par $g_n(x) = \int_0^x f_n(t)dt$. Alors $g_n$ est $\mathcal{C}^1$, et converge (donc en particulier $(g_n)$ est de Cauchy). Elle converge vers $g$ qui n'est pas une fonction $\mathcal{C}^1$. Donc ce n'est pas un espace complet.}
  \item \reponse{Soit $(f_n)$ une suite de Cauchy pour la norme $N$.
Pour chaque $t\in[a,b]$, $(f_n(t))_n$ est une suite de Cauchy de $\Rr$
donc converge. Notons $f(t)$ sa limite. 
De même $(f_n'(t))_n$ est une suite de Cauchy de $\Rr$  donc converge vers $g(t)$.
Nous allons montrer que $f$
est dans $X$ et que $f_n$ converge vers $f$ pour $N$ et que $f'=g$.
Soit $\epsilon > 0$. Il existe $n\in \Nn$ tel que
Pour tout $p\ge 0$, 
$$N(f_{n}-f_{n+p}) < \epsilon.$$
En faisant tendre $p$ vers $+\infty$, $f_{n+p}$ converge (simplement) vers $f$.
On obtient que
$\|f_n-f\|_\infty$ et que $\|f_n'-g\|_\infty$ tendent vers $0$.
Donc $f_n$ converge uniformément vers $f$. Comme les $f_n$ sont continues
alors $f$ est continue. De même $f_n'$ converge uniformément vers $g$
donc $g$ est continue. De plus cela implique que $g=f'$.(Rappel : si $(f_n)$ est une suite de fonctions $\mathcal{C}^1$ sur $[a,b]$ qui converge simplement vers $f$, et tel que $(f_n')$ converge uniformément vers $g$, alors $f$ est $\mathcal{C}^1$ et sa dérivée est $f'=g$.)
Nous avons donc montrer que $N(f_n-f)$ tend vers $0$ et que $f$ est dans $X$.
Donc $(X,N)$ est complet.}
\end{enumerate}
}