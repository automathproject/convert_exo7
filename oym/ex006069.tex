\exo7id{6069}
\titre{Exercice 6069}
\theme{}
\auteur{queffelec}
\date{2011/10/16}
\organisation{exo7}
\contenu{
  \texte{Soit $f$ une application de $\R$ dans $\R$ et $\omega$ sa fonction
oscillation définie pour $x_0\in \R$ et $\delta>0$ par 
$$\omega(x_0,\delta)=\sup_{\{|x_0-y|=\delta,|x_0-z|=\delta\}}|f(y)-f(z)|.$$}
\begin{enumerate}
  \item \question{Remarquer que $f$ est continue en $x_0$ si et seulement si
$$\omega(x_0)=\inf_{\delta>0}\omega(x_0,\delta)=0.$$}
  \item \question{Montrer que pour tout $\varepsilon >0$, $O_\varepsilon=\{x\ ;\
\omega(x)<\varepsilon\}$ est un ouvert.

En déduire que $C(f)$, l'ensemble des points de continuité de $f$, est un
$G_\delta$.}
\end{enumerate}
\begin{enumerate}

\end{enumerate}
}