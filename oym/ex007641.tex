\uuid{7641}
\titre{Singularités}
\theme{}
\auteur{mourougane}
\date{2021/08/10}
\organisation{exo7}
\contenu{
  \texte{}
  \question{Que vaut, en fonction du nombre réel $r>0$, l'intégrale
 $$I_r:=\int_{\Delta_r} \frac{dz}{2z^2-5z+2} \ \ ?$$
 On précisera les valeurs de $r$ exclues.}
  \reponse{Les zéros du polynôme $2z^2-5z+2=(z-2)(2z-1)$ et donc les singularités de l'application méromorphe $f(z)=\frac{1}{2z^2-5z+2}$ sont $2$ et $1/2$. Ce sont des pôles simples.
Leurs résidus respectifs sont $res_{2}(f)=\lim_{z\to 2}(z-2)\frac{1}{2z^2-5z+2}=\frac{1}{3}$ et $res_{1/2}(f)=\lim_{z\to 1/2}(z-\frac{1}{2})\frac{1}{2z^2-5z+2}=-\frac{1}{3}$. Leurs indices par rapport au chemin orienté ${\partial\Delta_r}$ sont $0$ ou $1$, suivant qu'ils sont ou pas dans le disque ${\Delta_r}$.
On trouve donc, par le théorème des résidus appliqué à l'application méromorphe $f$ : (les valeurs de $r$ exclues sont $1/2$ et $2$ car les cercles correspondants passent par une singularité.)
\begin{enumerate} 
\item si $0<r<1/2$, $I_r=0$.
\item si $1/2<r<2$, $I_r=-2i\pi/3$.
\item si $2<r$, \ \ \ \ \ \ \ \ $I_r=1/3-1/3=0$.
\end{enumerate}}
}