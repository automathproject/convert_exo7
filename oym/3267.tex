\uuid{3267}
\titre{$2X^3 + 5X^2 - X + \lambda$ a une racine de module 1}
\theme{Exercices de Michel Quercia, Fonctions symétriques}
\auteur{quercia}
\date{2010/03/08}
\organisation{exo7}
\contenu{
  \texte{}
  \question{Trouver $\lambda \in \R$ tel que $2X^3 + 5X^2 - X + \lambda$
ait une racine de module 1.}
  \reponse{racine 1 : $\lambda = -6$.\par
 racine -1 : $\lambda = -4$.\par
 racine $\alpha \in \mathbb{U}\setminus\{\pm1\}$ : les autres sont $\frac 1\alpha$ et
 $-\lambda  \Rightarrow  \lambda = 6$, $\alpha = \frac{1+i\sqrt{15}}4$.}
}