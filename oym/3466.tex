\uuid{3466}
\titre{Factorisation, Centrale P' 1996}
\theme{Exercices de Michel Quercia, Rang de matrices}
\auteur{quercia}
\date{2010/03/10}
\organisation{exo7}
\contenu{
  \texte{}
  \question{Soit la matrice carrée d'ordre~$n$, $I_p$ ($p\le n$), telle que le $i$-ème
terme diagonal vaut 1 si $i$ est compris entre $p$ et~$n$, tous les autres
coefficients étant nuls.
Quelle sont les conditions sur~$A$ (matrice carrée d'ordre~$n$) pour
qu'il existe~$B$ telle que $AB=I_p$~?}
  \reponse{Les colonnes de~$A$ engendrent les $n-p$ derniers vecteurs de la
base canonique.}
}