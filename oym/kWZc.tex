\uuid{kWZc}
\exo7id{4258}
\titre{Plus grande fonction convexe minorant $f$}
\theme{Exercices de Michel Quercia, Intégrale de Riemann}
\auteur{quercia}
\date{2010/03/12}
\organisation{exo7}
\contenu{
  \texte{}
\begin{enumerate}
  \item \question{Soit $(f_i)$ une famille de fonctions convexes sur un intervalle $I$.

On suppose que : $\forall\ x \in I,\ f(x) = \sup(f_i(x))$ existe.
Montrer que $f$ est convexe.}
  \item \question{Soit $f : I \to {\R}$ minorée. Montrer qu'il existe une plus grande fonction
convexe minorant $f$. On la note $\tilde f$.}
  \item \question{Soit $f : {[0,1]} \to {\R^+}$ croissante.
Montrer que $ \int_{t=0}^1 \tilde f(t)d t \ge \frac 12 \int_{t=0}^1 f(t)d t$
(commencer par le cas où $f$ est en escalier).}
\end{enumerate}
\begin{enumerate}

\end{enumerate}
}