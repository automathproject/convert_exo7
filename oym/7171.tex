\uuid{7171}
\titre{Exercice 7171}
\theme{}
\auteur{megy}
\date{2017/07/26}
\organisation{exo7}
\contenu{
  \texte{}
  \question{%[application directe de'AM>GM]
Soit $n>0$ un entier. On considère $n$ réels positifs dont le produit vaut  $1$. Leur somme a-t-elle une valeur minimale et si oui laquelle et dans quel(s) cas?}
  \reponse{Notons $a_1$, ..., $a_n$ les nombres de l'énoncé. On a 
\[
a_1 a_2 \dots a_n =  \prod_{i=1}^n a_i=1.\]
L'inégalité arithmético-géométrique fournit :
\[ \frac{\sum_{i=1}^n a_i}{n} = \frac{a_1+..+a_n}{n}\geq \sqrt[n]{\prod_{i=1}^n a_i} = \sqrt[n]{1}=1\]
avec égalité ssi tous les $a_i$ sont égaux. On en déduit que la somme est supérieure à $n$, avec égalité ssi tous les réels sont égaux (et donc égaux à  $1$).}
}