\uuid{FGOE}
\exo7id{4914}
\titre{Points mobiles avec $PQ =$ constante}
\theme{Exercices de Michel Quercia, Coniques}
\auteur{quercia}
\date{2010/03/17}
\organisation{exo7}
\contenu{
  \texte{Soient $P$ un point mobile sur $Ox$, et $Q$ un point mobile sur $Oy$ tels
que $PQ$ reste constante.}
\begin{enumerate}
  \item \question{Pour $\alpha\in \R$, déterminer le lieu, ${\cal C}_\alpha$,
    de Bar$(P:1-\alpha,Q:\alpha)$.}
  \item \question{Soit $R$ le quatrième point du rectangle $OPQR$. Démontrer que la tangente à
    ${\cal C}_\alpha$ en un point $M$ est perpendiculaire à $(RM)$.}
\end{enumerate}
\begin{enumerate}
  \item \reponse{$\frac{x^2}{(1-\alpha)^2} + \frac{y^2}{\alpha^2} = d^2$.}
\end{enumerate}
}