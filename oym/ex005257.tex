\uuid{5257}
\titre{**T}
\theme{}
\auteur{rouget}
\date{2010/07/04}
\organisation{exo7}
\contenu{
  \texte{Soit $u$ l'endomorphisme de $\Rr^3$ dont la matrice dans la base canonique $(i,j,k)$ de $\Rr^3$ est~:

$$M=\left(
\begin{array}{ccc}
2&1&0\\
-3&-1&1\\
1&0&-1
\end{array}
\right)
.$$}
\begin{enumerate}
  \item \question{Déterminer $u(2i-3j+5k)$.}
  \item \question{Déterminer $\mbox{Ker}u$ et $\mbox{Im}u$.}
  \item \question{Calculer $M^2$ et $M^3$.}
  \item \question{Déterminer $\mbox{Ker}u^2$ et $\mbox{Im}u^2$.}
  \item \question{Calculer $(I-M)(I+M+M^2)$ et en déduire que $I-M$ est inversible. Préciser $(I-M)^{-1}$.}
\end{enumerate}
\begin{enumerate}
  \item \reponse{Soit $X=\left(
\begin{array}{c}
2\\
-3\\
5
\end{array}
\right)$. $MX=\left(
\begin{array}{ccc}
2&1&0\\
-3&-1&1\\
1&0&-1
\end{array}
\right)\left(
\begin{array}{c}
2\\
-3\\
5
\end{array}
\right)=\left(
\begin{array}{c}
1\\
2\\
-3
\end{array}
\right)$  et $u(2i-3j+5k)=i+2j-3k$.}
  \item \reponse{Soit $X=\left(
\begin{array}{c}
x\\
y\\
z
\end{array}
\right)\in\mathcal{M}_{3,1}(\Rr)$ . 

$$MX=0\Leftrightarrow\left(
\begin{array}{ccc}
2&1&0\\
-3&-1&1\\
1&0&-1
\end{array}
\right)\left(
\begin{array}{c}
x\\
y\\
z
\end{array}
\right)=\left(
\begin{array}{c}
0\\
0\\
0
\end{array}
\right)\Leftrightarrow
\left\{
\begin{array}{l}
2x+y=0\\
-3x-y+z=0\\
x-z=0
\end{array}
\right.\Leftrightarrow\left\{
\begin{array}{l}
y=-2x\\
z=x
\end{array}
\right..$$

Donc, $\mbox{Ker}u=\mbox{Vect}(i-2j+k)$. En particulier, $\mbox{dim}(\mbox{Ker}u)=1$ et, d'après le théorème du rang, $\mbox{rg}u=2$. Or, $u(j)=i-j$ et $u(k)=j+k$ sont deux vecteurs non colinéaires de $\mbox{Im}u$ qui est un plan vectoriel et donc $\mbox{Im}u=\mbox{Vect}(i-j,j-k)$.}
  \item \reponse{$$M^2=\left(
\begin{array}{ccc}
2&1&0\\
-3&-1&1\\
1&0&-1
\end{array}
\right)\left(
\begin{array}{ccc}
2&1&0\\
-3&-1&1\\
1&0&-1
\end{array}
\right)=\left(
\begin{array}{ccc}
1&1&1\\
-2&-2&-2\\
1&1&1
\end{array}
\right)$$

et 

$$M^3=M^2.M=\left(
\begin{array}{ccc}
1&1&1\\
-2&-2&-2\\
1&1&1
\end{array}
\right)\left(
\begin{array}{ccc}
2&1&0\\
-3&-1&1\\
1&0&-1
\end{array}
\right)=0.$$}
  \item \reponse{$\mbox{Ker}u^2$ est à l'évidence le plan d'équation $x+y+z=0$. Une base de $\mbox{Ker}u^2$ est $(i-j,j-k)$ et donc $\mbox{Ker}u^2=\mbox{Im}u=\mbox{Vect}(i-j,j-k)$.

D'après le théorème du rang, $\mbox{Im}u^2$ est une droite vectorielle. Mais $u^3=0$ s'écrit encore $u\circ u^2=0$, et donc $\mbox{Im}u^2$ est contenue dans $\mbox{Ker}u$ qui est une droite vectorielle. Donc, $\mbox{Im}u^2=\mbox{Ker}u=\mbox{Vect}(i-2j+k)$.}
  \item \reponse{$(I-M)(I+M+M^2)=I-M^3=I$. Par suite, $I-M$ est inversible à droite et donc inversible et 

$$(I-M)^{-1}=I+M+M^2=\left(
\begin{array}{ccc}
1&0&0\\
0&1&0\\
0&0&1
\end{array}
\right)+\left(
\begin{array}{ccc}
2&1&0\\
-3&-1&1\\
1&0&-1
\end{array}
\right)+\left(
\begin{array}{ccc}
1&1&1\\
-2&-2&-2\\
1&1&1
\end{array}
\right)=\left(
\begin{array}{ccc}
4&2&1\\
-5&-2&-1\\
2&1&1
\end{array}
\right).$$}
\end{enumerate}
}