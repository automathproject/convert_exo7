\exo7id{7201}
\titre{Exercice 7201}
\theme{}
\auteur{megy}
\date{2019/07/23}
\organisation{exo7}
\contenu{
  \texte{Soit $E=\R^\R$ l'ensemble des fonctions de $\R$ dans $\R$ et $f, g\in E$. On dit que $f$ et $g$ ont \og même germe en zéro\fg{} et on note $f \underset{0}{=} g$ si:
\[  \exists \epsilon>0, f|_{]-\epsilon,\epsilon[} = g|_{]-\epsilon,\epsilon[} \]}
\begin{enumerate}
  \item \question{Montrer que $\underset{0}{=}$ est une relation d'équivalence sur $E$.}
  \item \question{Montrer  que si $f \underset{0}{=} g$ alors $f(0)=g(0)$, mais que la réciproque est fausse.}
  \item \question{Montrer également que pour tout $a\in \R^*$, il existe deux fonctions $f$ et $g$ avec $f \underset{0}{=} g$ et $f(a)\neq g(a)$.}
\end{enumerate}
\begin{enumerate}

\end{enumerate}
}