\uuid{3474}
\titre{Calcul de rang}
\theme{Exercices de Michel Quercia, Rang de matrices}
\auteur{quercia}
\date{2010/03/10}
\organisation{exo7}
\contenu{
  \texte{}
  \question{Soit $M = \Bigl(\cos (i+j-1)\theta \Bigr) \in \mathcal{M}_n(\R)$.
Déterminer $\mathrm{rg} M$ en fonction de $\theta$.}
  \reponse{$M = \Re\left[\begin{pmatrix} e^{i\theta} \cr \vdots \cr e^{ni\theta} \cr \end{pmatrix}
              \Bigl(\begin{matrix} 1 & \dots & e^{(n-1)i\theta} \cr \end{matrix}\Bigr)\right]
  \Rightarrow  \mathrm{rg} M \le 2$.

Le premier mineur $2\times2$ vaut $-\sin^2\theta
 \Rightarrow  \mathrm{rg} M = 2$ si $\theta \not \equiv 0 (\mathrm{mod}\, \pi)$.
Sinon, $\mathrm{rg} M = 1$.}
}