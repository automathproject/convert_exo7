\uuid{xR0A}
\exo7id{3708}
\titre{Exponentielle d'une application antisymétrique}
\theme{Exercices de Michel Quercia, Espace vectoriel euclidien orienté de dimension 3}
\auteur{quercia}
\date{2010/03/11}
\organisation{exo7}
\contenu{
  \texte{Soit $E$ un espace vectoriel euclidien orienté de dimension 3\ et
$\vec a \in E\setminus\{\vec 0\}$. On note $\alpha = \|\vec a\,\|$.
Soit $f$ l'endomorphisme de $E$ défini par : $f(\vec x) = \vec a\wedge\vec x$.}
\begin{enumerate}
  \item \question{Vérifier que $f^3 = -\alpha^2f$.}
  \item \question{On pose $g(\vec x) = \sum_{k=0}^\infty \frac{f^k(\vec x)}{k!}$.
    Simplifier $g(\vec x)$ et en déduire que $g$ est une rotation.}
\end{enumerate}
\begin{enumerate}
  \item \reponse{$g(\vec x) = (\cos\alpha)\vec x
                        + (1-\cos\alpha)\frac{(\vec a\mid\vec x)\vec a}{\alpha^2}
                        + \frac{\sin\alpha(\vec a\wedge\vec x)}\alpha$.}
\end{enumerate}
}