\uuid{XWej}
\exo7id{3336}
\titre{$f$ tq $\Im f$ et $\mathrm{Ker} f$ sont imposés}
\theme{Exercices de Michel Quercia, Applications linéaires en dimension finie}
\auteur{quercia}
\date{2010/03/09}
\organisation{exo7}
\contenu{
  \texte{Soit $E$ un $ K$-ev de dimension finie et $H,K$ deux sev fixés de $E$.}
\begin{enumerate}
  \item \question{A quelle condition existe-t-il un endomorphisme $f \in \mathcal{L}(E)$ tel que
     $\Im f = H \text{ et } \mathrm{Ker} f = K$ ?}
  \item \question{On note ${\cal E} = \{ f \in \mathcal{L}(E) \text{ tq } \Im f = H \text{ et } \mathrm{Ker} f = K \}$.
     Montrer que $\cal E$ est un groupe pour $\circ$ si et seulement si
     $H \oplus K = E$.}
\end{enumerate}
\begin{enumerate}
  \item \reponse{$\dim H + \dim K = \dim E$.}
  \item \reponse{Si $H \oplus K \ne E$ alors $\cal E$ n'est pas stable pour
              $\circ$.}
\end{enumerate}
}