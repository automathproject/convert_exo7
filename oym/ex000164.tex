\uuid{164}
\titre{Exercice 164}
\theme{}
\auteur{cousquer}
\date{2003/10/01}
\organisation{exo7}
\contenu{
  \texte{On considère une suite $(u_n)_{n\in \mathbb{N}}$ telle que ~:
$$u_0=0 \quad \mbox{et}\quad u_1=1 \quad\mbox{et}\quad 
\forall n\geq1,\; u_{n+1}=u_n+2u_{n-1}$$
Démontrer que~:}
\begin{enumerate}
  \item \question{$\forall n\in \mathbb{N},\; u_n\in \mathbb{N}$,}
  \item \question{$\forall n\in \mathbb{N},\; u_n=\frac{1}{3}(2^n-(-1)^n)$.}
\end{enumerate}
\begin{enumerate}

\end{enumerate}
}