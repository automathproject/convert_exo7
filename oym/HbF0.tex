\uuid{HbF0}
\exo7id{6004}
\titre{Exercice 6004}
\theme{Probabilité conditionnelle}
\auteur{quinio}
\date{2011/05/20}
\organisation{exo7}
\contenu{
  \texte{Un débutant à un jeu effectue plusieurs parties successives. Pour la
première partie, les probabilités de gagner ou perdre sont les mêmes; puis, on suppose que:
\begin{itemize}
  \item Si une partie est gagnée, la probabilité de gagner la suivante est $0.6$.

  \item Si une partie est perdue, la probabilité de perdre la suivante est $0.7$.
\end{itemize}
Soit $G_n$ l'événement <<Gagner la partie $n$>>, et $u_n=P(G_n)$.
On note $v_n = P(\overline{G_n})$.}
\begin{enumerate}
  \item \question{Ecrire 2 relations entre $u_n$, $u_{n+1}$, $v_n$, $v_{n+1}$.}
  \item \question{A l'aide de la matrice mise en évidence en déduire $u_n$ et $v_n$.
Faire un calcul direct à l'aide de $u_n+v_n$.}
\end{enumerate}
\begin{enumerate}
  \item \reponse{$u_{n+1}=P(G_{n+1})=P(G_{n+1}/Gn)P(Gn)+P(G_{n+1}/\overline{G_n})P(\overline{G_n})
=0.6u_{n}+0.3v_{n}$.  

$v_{n+1}=0.4u_{n}+0.7v_{n}$.

Donc $\left( 
\begin{array}{c}
u_{n+1} \\ 
v_{n+1}\end{array}\right) =\left( 
\begin{array}{cc}
0.6 & 0.3 \\ 
0.4 & 0.7\end{array}\right) \left( 
\begin{array}{c}
u_{n} \\ 
v_{n}\end{array}\right)$

Comme $u_{n}+v_{n}=1$, $u_{n+1}=0.6u_{n}+0.3(1-u_{n})=0.3+0.3u_{n}$.
La suite $(u_{n}-\ell)$ est géométrique, où $\ell$ est solution
de $0.3+0.3\ell=\ell$, donc $\ell=\frac{3}{7}$.
Donc $u_{n}=\frac{3}{7}+u_{1}(0.3)^{n-1}=\frac{3}{7}+0.5 (0.3)^{n-1}$.}
\end{enumerate}
}