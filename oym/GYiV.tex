\uuid{GYiV}
\exo7id{6865}
\titre{Exercice 6865}
\theme{Calculs d'intégrales, Calculs de primitives}
\auteur{bodin}
\date{2012/04/13}
\organisation{exo7}
\contenu{
  \texte{Calculer les primitives suivantes par changement de variable.}
\begin{enumerate}
  \item \question{$\int (\cos x) ^{1234} \sin x \, d x$}
  \item \question{$\int \frac 1{x\ln x} \, dx$}
  \item \question{$\int \frac 1{3+\exp \left( -x\right)}dx$}
  \item \question{$\int \frac{1}{\sqrt{4x-x^2}}dx$}
\end{enumerate}
\begin{enumerate}
  \item \reponse{$\int (\cos x) ^{1234} \sin x \, d x$

En posant le changement de variable $u = \cos x$ on a $x=\arccos u$ et $du = - \sin x \, dx$ et 
on obtient 
$$\int (\cos x) ^{1234} \sin x \, d x = \int u^{1234} (-du) = - \frac 1{1235} u^{1235} + c=  -\frac 1{1235}(\cos x)^{1235}+c$$
Cette primitive est définie sur $\Rr$.}
  \item \reponse{$\int \frac 1{x\ln x} \, dx$

En posant le changement de variable $u=\ln x$ on a $x=\exp u$ et $du = \frac {dx}{x}$ on écrit :
$$\int \frac 1{x\ln x} \, dx = \int \frac 1{\ln x} \frac{dx}{x} = \int \frac 1 u du= \ln |u| + c = \ln \left| \ln x\right| +c$$
Cette primitive est définie sur $\left] 0,1\right[$ ou sur $\left] 1,+\infty \right[$ (la constante peut être différente pour chacun des intervalles).}
  \item \reponse{$\int \frac 1{3+\exp \left( -x\right)}dx$

Soit le changement de variable $u=\exp x$. Alors $x=\ln u$ et $du = \exp x \, dx$
ce qui s'écrit aussi $dx = \frac{du}{u}$.
$$\int \frac 1{3+\exp \left( -x\right) }dx = \int \frac{1}{3+\frac{1}{u}} \frac{du}{u}
= \int \frac{1}{3u+1} du = \frac 13 \ln |3u+1| + c = \frac13\ln \left( 3\exp
x+1\right) +c$$
Cette primitive est définie sur $\Rr$.}
  \item \reponse{$\int \frac{1}{\sqrt{4x-x^2}}dx$

Le changement de variable a pour but de se ramener à quelque chose de connu.
Ici nous avons une fraction avec une racine carrée au dénominateur et sous la racine un polynôme de degré $2$.
Ce que l'on sait intégrer c'est
$$\int \frac{1}{\sqrt{1-u^2}}du = \arcsin u + c,$$
car on connaît la dérivée de la fonction $\arcsin(t)$ c'est $\arcsin'(t)=\frac{1}{\sqrt{1-t^2}}$.
On va donc essayer de s'y ramener.
Essayons d'écrire ce qu'il y a sous la racine, $4x-x^2$ sous la forme $1-t^2$ :
$4x-x^2 =  4 - (x-2)^2 = 4 \bigg( 1 - \big(\frac 12 x - 1\big)^2\bigg)$.
Donc il est naturel d'essayer le changement de variable $u= \frac 12 x - 1$
pour lequel $4x-x^2=4(1-u^2)$ et $dx = 2du$. 

$$\int \frac{1}{\sqrt{4x-x^2}}dx = \int \frac{1}{\sqrt{4(1-u^2)}} 2du = \int \frac{du}{\sqrt{1-u^2}}
= \arcsin u + c = \arcsin \left( \frac 12x-1\right) +c$$
La fonction $\arcsin u$ est définie et dérivable pour $u\in]-1,1[$ alors cette primitive est 
définie sur $x \in \left] 0,4\right[$.}
\end{enumerate}
}