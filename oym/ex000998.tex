\exo7id{998}
\titre{Exercice 998}
\theme{}
\auteur{legall}
\date{1998/09/01}
\organisation{exo7}
\contenu{
  \texte{Soit $ a\in \R .$ On pose, pour tout $ p\in \Nn
: A_p(X)=(X-a)^p $
et $ B_p(X)=X^p .$}
\begin{enumerate}
  \item \question{Montrer que $\epsilon = \{ A_0, \ldots, A_n\}  $ est une base de
$ \R _n[X] .$}
  \item \question{Soit $ P\in \R _n[X] .$ Montrer que $ \displaystyle{ P(X)= \sum _{k=0}^n
\frac{1}{ k!}P^{(k)}(a)A_k(X)} .$ (On pourra montrer que l'ensemble $ E $
des \'el\'ement de $ \R _n[X] $ qui satisfont \`a cette \'egalit\'e
est un sous-espace vectoriel de $ \R _n[X] $ et contient une base.)}
\end{enumerate}
\begin{enumerate}

\end{enumerate}
}