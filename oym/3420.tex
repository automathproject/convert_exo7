\uuid{3420}
\titre{Combinaison de formes linéaires}
\theme{Exercices de Michel Quercia, \'Equations linéaires}
\auteur{quercia}
\date{2010/03/10}
\organisation{exo7}
\contenu{
  \texte{}
  \question{Soient $f,f_1,\dots,f_p$ des formes linéaires sur $ K^n$ linéairement
indépendantes.
Montrer que $f$ est combinaison linéaire de $f_1, \dots, f_p$ si et seulement si
$\mathrm{Ker} f \supset \mathrm{Ker} f_1 \cap \dots \cap \mathrm{Ker} f_p$.

Indication :
\'Etudier le système $$ \left\{\begin{array}{lll} f_1(x) &=& 0 \cr
                               \quad\vdots  \cr
                               f_p(x) &=& 0 \cr
                               f(x)   &=& 1.\cr \end{array}\right.$$


Le résultat est-il encore vrai si on ne suppose pas $(f_1, \dots, f_p)$
libre ?}
  \reponse{}
}