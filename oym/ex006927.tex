\uuid{6927}
\titre{Exercice 6927}
\theme{}
\auteur{ruette}
\date{2013/01/24}
\organisation{exo7}
\contenu{
  \texte{On suppose que le nombre de pièces sortant d'une
usine donnée en une journée est une variable aléatoire
d'espérance $50$.}
\begin{enumerate}
  \item \question{Peut-on estimer la probabilité que la production de demain
dépasse 75 pièces~?}
  \item \question{Que peut-on dire de plus sur cette probabilité si on sait que l'écart-type de la production quotidienne est
de $5$ pièces~?}
\end{enumerate}
\begin{enumerate}
  \item \reponse{Inégalité de Markov : $P(X\ge 75)\le \frac{E(X)}{75}=\frac 23$.}
  \item \reponse{Inégalité de Bienaymé-Tchebychev : $P(|X-50|\ge 25)\le \frac{\text{Var}(X)}{25^2}=
\frac{5^2}{25^2}=0,04$. Donc $P(X\ge 75)\le 0,04$.}
\end{enumerate}
}