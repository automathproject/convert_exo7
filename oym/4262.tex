\uuid{4262}
\titre{Mines MP 2001}
\theme{Exercices de Michel Quercia, Intégrale de Riemann}
\auteur{quercia}
\date{2010/03/12}
\organisation{exo7}
\contenu{
  \texte{}
  \question{Soit $a<0<b$ et $f$ continue sur $[0,1]$, à valeurs dans $[a,b]$
telle que $\int_0^1 f=0$. Montrer que $\int_0^1 f^2\le -ab$.}
  \reponse{Soit $g=f-a$. On a $0\le g \le b-a$ et $ \int_0^1 g = -a$ d'où
$ \int_0^1g^2 \le (b-a) \int_0^1g = -a(b-a)$ et 
$ \int_0^1f^2 =  \int_0^1g^2 +2a \int_0^1g + a^2\le -ab$.}
}