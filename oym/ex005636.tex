\uuid{5636}
\titre{***I}
\theme{}
\auteur{rouget}
\date{2010/10/16}
\organisation{exo7}
\contenu{
  \texte{}
  \question{On définit par blocs une matrice $A$ par $A=\left(
\begin{array}{cc}
B&D\\
0&C
\end{array}
\right)$ où $A$, $B$ et $C$ sont des matrices carrées de formats respectifs $n$, $p$ et $q$ avec $p+q=n$. Montrer que $\text{det}(A)=\text{det}(B)\times\text{det}(C)$.}
  \reponse{Soient $C\in \mathcal{M}_q(\Kk)$ et $D\in\mathcal{M}_{p,q}(\Kk)$. Soit $\begin{array}[t]{cccc}
\varphi~:&(\mathcal{M}_{p,1}(\Kk))^p&\rightarrow&\Kk\\
 &(C_1,\ldots,C_p)&\mapsto&\text{det}\left(
 \begin{array}{cc}
 X&D\\
 0&C
 \end{array}
 \right)
 \end{array}$ où $X=(C_1\ldots C_p)\in\mathcal{M}_p(\Kk)$.
 

\textbullet~$\varphi$ est linéaire par rapport à chacune des colonnes $C_1$,\ldots, $C_p$.

\textbullet~Si il existe $(i,j)\in\llbracket1,p\rrbracket^2$ tel que $i\neq j$ et $C_i=C_j$, alors $\varphi(C_1,\ldots,C_p)=0$.

Ainsi, $\varphi$ est une forme $p$-linéaire alternée sur l'espace $\mathcal{M}_{p,1}(\Kk)$ qui est de dimension $p$. On sait alors qu'il existe $\lambda\in\Kk$ tel que $\varphi=\lambda\;\text{det}_{\mathcal{B}_0}$ (où $\text{det}_{\mathcal{B}_0}$ désigne la forme déterminant dans la base canonique de $\mathcal{M}_{p,1}(\Kk)$) ou encore il existe $\lambda\in\Kk$ indépendant de $(C_1,\ldots,C_p)$ tel que $\forall (C_1,\ldots,C_p)\in(\mathcal{M}_{p,1}(\Kk))^p$, $f(C_1,\ldots,C_p)=\lambda\;\text{det}_{\mathcal{B}_0}(C_1,\ldots,C_p)$ ou enfin il existe $\lambda\in\Kk$ indépendant de $X$ tel que $\forall X\in\mathcal{M}_{p}(\Kk)$, $\text{det}\left(
 \begin{array}{cc}
 X&D\\
 0&C
 \end{array}
 \right)=\lambda\;\text{det}(X)$. Pour $X=I_p$, on obtient $\lambda=\text{det}\left(
 \begin{array}{cc}
 I_p&D\\
 0&C
 \end{array}
 \right)$ et donc
 
 \begin{center}
 $\forall B\in\mathcal{M}_p(\Kk)$, $\text{det}\left(
 \begin{array}{cc}
 B&D\\
 0&C
 \end{array}
 \right)=\text{det}(B)\times\text{det}\left(
 \begin{array}{cc}
 I_p&D\\
 0&C
 \end{array}
 \right)$.
 \end{center}
 

De même, l'application $Y\mapsto\text{det}\left(
 \begin{array}{cc}
 I_p&D\\
 0&Y
 \end{array}
 \right)$ est une forme $q$-linéaire alternée des lignes de $Y$ et donc il existe $\mu\in\Kk$ tel que $\forall Y\in\mathcal{M}_q(\Kk)$, $\text{det}\left(
 \begin{array}{cc}
 I_p&D\\
 0&Y
 \end{array}
 \right)=\mu\;\text{det}(Y)$ puis $Y=I_q$ fournit $\mu=\text{det}\left(
 \begin{array}{cc}
 I_p&D\\
 0&I_q
 \end{array}
 \right)$ et donc 
 
 \begin{center}
 $\forall B\in\mathcal{M}_p(\Kk)$, $\forall C\in\mathcal{M}_q(\Kk)$, $\forall D\in\mathcal{M}_{p,q}(\Kk)$, $\text{det}\left(
 \begin{array}{cc}
 B&D\\
 0&C
 \end{array}
 \right)=\text{det}(B)\times\text{det}(C)\times\text{det}\left(
 \begin{array}{cc}
 I_p&D\\
 0&I_q
 \end{array}
 \right)=\text{det}(B)\times\text{det}(C)$,
 \end{center}
 

(en supposant acquise la valeur d'un déterminant triangulaire qui peut s'obtenir en revenant à la définition d'un déterminant et indépendamment de tout calcul par blocs).

 \begin{center}
 \shadowbox{
 $\forall (B,C,D)\in\mathcal{M}_p(\Kk)\times\mathcal{M}_q(\Kk)\times\mathcal{M}_{p,q}(\Kk)$, $\text{det}\left(
 \begin{array}{cc}
 B&D\\
 0&C
 \end{array}
 \right)=\text{det}(B)\times\text{det}(C)$.
 }
 \end{center}}
}