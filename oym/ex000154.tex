\uuid{154}
\titre{Exercice 154}
\theme{}
\auteur{bodin}
\date{1998/09/01}
\organisation{exo7}
\contenu{
  \texte{}
  \question{\textbf{En quoi le raisonnement suivant est-il faux ?} \\
Soit $\mathcal{P}(n)$ : $n$ crayons de couleurs sont tous de la m\^eme couleur.
\begin{itemize}
    \item $\mathcal{P}(1)$ est vraie car un crayon de couleur est de la m\^eme couleur
que lui-m\^eme.
    \item Supposons $\mathcal{P}(n)$. Soit $n+1$ crayons. On en retire $1$. Les $n$
crayons restants sont de la m\^eme couleur par hypoth\`ese de r\'ecurrence.

Reposons ce crayon et retirons-en un autre ; les $n$ nouveaux crayons sont \`a nouveau
de la m\^eme couleur. Le premier crayon retir\'e \'etait donc bien de la m\^eme couleur que les $n$ autres. La proposition est donc vraie au rang $n+1$.
    \item On a donc d\'emontr\'e que tous les crayons en nombre infini d\'enombrable
sont de la m\^eme couleur.
\end{itemize}}
  \reponse{}
}