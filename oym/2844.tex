\uuid{2844}
\titre{Exercice 2844}
\theme{Prolongement analytique et résidus, Résidus}
\auteur{burnol}
\date{2009/12/15}
\organisation{exo7}
\contenu{
  \texte{}
  \question{Soit $\Omega$ un domaine, de
bord le cycle $\partial\Omega$ orienté dans le sens direct. 
Soit $f$ une fonction holomorphe sur
$\overline\Omega$, soient $z_1$ et $z_2$ deux points de
$\Omega$. Que vaut
\[ \int_{\partial\Omega}
\frac{f(z)\,dz}{(z-z_1)(z-z_2)}\;?\]
Qu'obtient-on pour $z_2\to z_1$, $z_1$ fixé ?}
  \reponse{On a
\begin{eqnarray*}
\int_{\partial\Omega} \frac{f(z)\,dz}{(z-z_1)(z-z_2)} &=& \int_\gamma \frac{f(z)\,dz}{(z-z_1)(z-z_2)} -\sum_{j=1}^N \int_{\gamma_j} \frac{f(z)\,dz}{(z-z_1)(z-z_2)} \\
&=& 2i\pi \Big( \mathrm{Res} (f, z_1)+\mathrm{Res} (f, z_2) \Big) = 2i\pi \Big( \frac{f(z_1)}{z_1-z_2}+\frac{f(z_2)}{z_2-z_1}\Big).
\end{eqnarray*}
D'o\`u
$$\lim_{z_2\to z_1} \int_{\partial\Omega} \frac{f(z)\,dz}{(z-z_1)(z-z_2)}= 2i\pi f'(z_1).$$}
}