\uuid{254}
\titre{Exercice 254}
\theme{Arithmétique dans $\Zz$, Divisibilité, division euclidienne}
\auteur{bodin}
\date{1998/09/01}
\organisation{exo7}
\contenu{
  \texte{}
  \question{D\'emontrer que le nombre $7^n+1$ est divisible par
$8$ si $n$ est impair ; dans le cas $n$ pair, donner le reste de
sa division par $8$.}
  \reponse{Raisonnons modulo $8$ :
$$7 \equiv -1 \pmod{8}.$$
Donc
$$7^n +1 \equiv (-1)^n + 1 \pmod{8}.$$

Le reste de la division euclidienne de $7^n+1$ par $8$ est donc
$(-1)^n+1$ donc Si $n$ est impair alors $7^n+1$ est divisible par
$8$. Et si $n$ est pair $7^n+1$ n'est pas divisible par $8$.}
}