\uuid{DyZr}
\exo7id{2738}
\titre{Exercice 2738}
\theme{Résolution de systèmes linéaires par la méthode du Pivot de Gauss}
\auteur{tumpach}
\date{2009/10/25}
\organisation{exo7}
\contenu{
  \texte{Soient $a$, $b$, et $c$ trois nombres r\'eels.}
\begin{enumerate}
  \item \question{Quelle relation doivent satisfaire les param\`etres $a$, $b$ et $c$ pour que le syst\`eme suivant ait au moins une solution~?
$$
\mathcal{S}_{abc}~: \left\{\begin{array}{ccccccc} 
x & + & 2y & - & 3z & = & a\\
2x & + & 6y & - & 11z & = & b\\
x & - & 2y & + & 7z & = & c
\end{array}\right.
$$}
  \item \question{Est-ce que le syst\`eme $\mathcal{S}_{abc}$ peut avoir une unique solution~?}
\end{enumerate}
\begin{enumerate}

\end{enumerate}
}