\exo7id{2043}
\titre{Exercice 2043}
\theme{}
\auteur{liousse}
\date{2003/10/01}
\organisation{exo7}
\contenu{
  \texte{Tout ce probl\`eme se situe dans l'espace euclidien tridimensionnel muni d'un rep\`ere
 orthonorm\'e direct $\mathcal R 
= (O,\overrightarrow{i}, \overrightarrow{j},\overrightarrow{k})$.
On d\'efinit les trois  points : $A=(3, \sqrt{6},3)$, $B=(3, -\sqrt{6},3)$  et $C=(4,0,0)$.}
\begin{enumerate}
  \item \question{\begin{enumerate}}
  \item \question{Montrer que les points $O$, $A$ et $B$ ne sont pas align\'es et donner
une \'equation cart\'esienne du plan $P$ contenant $O$, $A$ et $B$.}
  \item \question{Calculer les distances $OA$, $OB$ et $AB$. En d\'eduire la nature du triangle
$OAB$.}
  \item \question{Les points $O,A,B$ et $C$ sont-ils coplanaires ?}
\end{enumerate}
\begin{enumerate}

\end{enumerate}
}