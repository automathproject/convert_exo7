\uuid{4145}
\titre{Laplacien en dimension $n$}
\theme{}
\auteur{quercia}
\date{2010/03/11}
\organisation{exo7}
\contenu{
  \texte{}
  \question{Soit $f$ une application de classe $\mathcal{C}^2$ de $\R^{+*}$ dans $\R$.

On définit une application $F$ de $\R^n\setminus\{\vec 0\}$ dans $\R$ par :
$F(x_1,\dots,x_n) = f(\sqrt{x_1^2 + \dots + x_n^2}\,)$.

Calculer le laplacien de $F$ en fonction de $f$.}
  \reponse{$\Delta F = \frac{n-1}r f'(r) + f''(r)$.}
}