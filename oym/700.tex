\uuid{700}
\titre{Exercice 700}
\theme{Fonctions dérivables, Calculs}
\auteur{bodin}
\date{1998/09/01}
\organisation{exo7}
\contenu{
  \texte{}
  \question{Soit $f : \Rr^* \longrightarrow \Rr$ d\'efinie par
$\displaystyle{f(x)= x^2\sin \frac{1}{x} }$. Montrer que
$f$ est prolongeable par continuit\'e en $0$ ; on note encore
$f$ la fonction prolong\'ee. Montrer que $f$ est
d\'erivable sur $\Rr$ mais que $f'$ n'est pas continue en $0$.}
  \reponse{$f$ est $C^\infty$ sur $\Rr^*$.
\begin{enumerate}
  \item Comme $|\sin (1/x)| \leq 1$ alors
$f$ tend vers $0$ quand $x\rightarrow 0$. Donc en prolongeant $f$ par $f(0) =0$,
la fonction  $f$ prolongée est continue sur $\Rr$.
  \item Le taux d'accroissement est
$$\frac{f(x)-f(0)}{x-0}= x \sin\frac{1}{x}.$$
Comme ci-dessus il y a une limite (qui vaut $0$) en $x=0$.
Donc $f$ est d\'erivable en $0$ et $f'(0)=0$.
  \item Sur $\Rr^*$, $f'(x) = 2x\sin (1/x) -\cos(1/x)$,
Donc $f'(x)$ n'a pas de limite quand $x\rightarrow 0$.
Donc $f'$ n'est pas continue en $0$.
\end{enumerate}}
}