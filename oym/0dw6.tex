\uuid{0dw6}
\exo7id{4455}
\titre{$u_n/S_n$}
\theme{Exercices de Michel Quercia, Séries numérique}
\auteur{quercia}
\date{2010/03/14}
\organisation{exo7}
\contenu{
  \texte{Soit $(u_n)$ une suite à termes strictement positifs convergeant vers~$0$. On pose
$S_n = \sum_{k=0}^n u_k$.}
\begin{enumerate}
  \item \question{Si la série $\sum u_n$ converge, que dire de la série
    $\sum \frac{u_n}{S_n}$ ?}
  \item \question{Si la série $\sum u_n$ diverge, montrer que la série
    $\sum \frac{u_n}{S_n}$ diverge aussi.
    On pourra considérer $p_n = \prod_{k=1}^n \left(1-\frac{u_k}{S_k}\right)$.}
\end{enumerate}
\begin{enumerate}
  \item \reponse{$p_n = \frac{u_0}{S_n} \to 0$ donc la série de terme général
              $\ln\left(1-\frac{u_n}{S_n}\right)$ diverge.}
\end{enumerate}
}