\uuid{WOIk}
\exo7id{4168}
\titre{Fonction implicite}
\theme{Exercices de Michel Quercia, Dérivées partielles}
\auteur{quercia}
\date{2010/03/11}
\organisation{exo7}
\contenu{
  \texte{Soit $f:\R \to \R$ de classe $\mathcal{C}^1$.}
\begin{enumerate}
  \item \question{Montrer que, sous une condition à préciser, l'équation $y-zx=f(z)$ définit
    localement $z$ fonction implicite de $x$ et $y$.}
  \item \question{Montrer que l'on a alors : $\frac{\partial z}{\partial x} + z\frac{\partial z}{\partial y} = 0$.}
\end{enumerate}
\begin{enumerate}

\end{enumerate}
}