\uuid{1042}
\titre{Exercice 1042}
\theme{}
\auteur{liousse}
\date{2003/10/01}
\organisation{exo7}
\contenu{
  \texte{On consid\`ere 
les trois matrices suivantes :
$$A = \left( \begin{array}{cccc} 2 & -3 & 1 & 0 \\ 5 & 4 & 1 & 3 \\ 
6 & -2 & -1 & 7 \end{array} \right)\ \ \ \ \ 
B = \left( \begin{array}{cc} 7 & 2  \\ -5 & 2  \\ 3 & 1  \\ 
6 & 0 \end{array} \right) \ \ {\hbox { et } } \ \ 
C = \left( \begin{array}{ccc} -1 & 2 & 6  \\ 3 & 5 & 7 \end{array} \right)$$}
\begin{enumerate}
  \item \question{Calculer $AB$ puis $(AB)C$.\\}
  \item \question{Calculer $BC$ puid $A(BC)$.\\}
  \item \question{Que remarque-t-on ?}
\end{enumerate}
\begin{enumerate}

\end{enumerate}
}