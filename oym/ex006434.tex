\uuid{6434}
\titre{Exercice 6434}
\theme{}
\auteur{potyag}
\date{2011/10/16}
\organisation{exo7}
\contenu{
  \texte{En utilisant le théorème du cours  que la somme d'angles intérieurs d'un
polygone convexe à $n$ sommets dans $\mathbb{H}^2$ est inférieure à
$(n-2)\pi$ démontrer que :}
\begin{enumerate}
  \item \question{Soient $\theta_1, \theta_2, ..., \theta_n$ une collection
  ordonnée de nombres tels que $0\leq\theta_i < \pi\ (i=1,...,n)$. Pour
  qu'il existe un polygone convexe d'angles intérieurs $\theta_i$ il
  faut et il suffit que $\theta_1 +\theta_2 +...+ \theta_n < (n-2)\pi$.}
  \item \question{Un polygone à $n$ c\^otés et d'angles droits existe
  ssi $n\geq 5$.}
\end{enumerate}
\begin{enumerate}

\end{enumerate}
}