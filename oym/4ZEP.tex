\uuid{4ZEP}
\exo7id{2916}
\titre{In{\'e}galit{\'e}s pour la formule du crible}
\theme{Exercices de Michel Quercia, Ensembles finis}
\auteur{quercia}
\date{2010/03/08}
\organisation{exo7}
\contenu{
  \texte{Soient $A_1, \dots, A_n$\ $n$ ensembles finis, et $E = \bigcup_{i=1}^n A_i$.}
\begin{enumerate}
  \item \question{Montrer que $\mathrm{Card}\,(E) \le \sum_{i=1}^n \mathrm{Card}\,(A_i)$. Cas d'{\'e}galit{\'e} ?}
  \item \question{Montrer que $\mathrm{Card}\,(E) \ge \sum_{i=1}^n \mathrm{Card}\,(A_i) -
    \sum_{1\le i< j \le n} \mathrm{Card}\,(A_i\cap A_j)$. Cas d'{\'e}galit{\'e} ?}
\end{enumerate}
\begin{enumerate}
  \item \reponse{R{\'e}currence. {\'E}galit{\'e} pour $n \le 2$ ou les $A_i$ 3 {\`a} 3 disjoints.}
\end{enumerate}
}