\uuid{899}
\titre{Exercice 899}
\theme{}
\auteur{liousse}
\date{2003/10/01}
\organisation{exo7}
\contenu{
  \texte{On consid\`ere dans $\Rr^n$ 
une famille de $4$ vecteurs lin\'eairement ind\'ependants : 
$(\vec{e_1}, \vec{e_2}, \vec{e_3}, \vec{e_4})$.
Les familles suivantes sont-elles libres ?}
\begin{enumerate}
  \item \question{$(\vec{e_1}, 2\vec{e_2}, \vec{e_3})$.}
  \item \question{$(\vec{e_1}, \vec{e_3})$.}
  \item \question{$(\vec{e_1}, 2\vec{e_1}+\vec{e_4}, \vec{e_4})$.}
  \item \question{$(3\vec{e_1}+\vec{e_3}, \vec{e_3}, \vec{e_2}+\vec{e_3})$.}
  \item \question{$(2\vec{e_1}+\vec{e_2}, \vec{e_1}-3\vec{e_2}, 
\vec{e_4}, \vec{e_2}-\vec{e_1})$.}
\end{enumerate}
\begin{enumerate}

\end{enumerate}
}