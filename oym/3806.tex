\uuid{3806}
\titre{Rayon spectral, Centrale MP 2006}
\theme{Exercices de Michel Quercia, Endomorphismes auto-adjoints}
\auteur{quercia}
\date{2010/03/11}
\organisation{exo7}
\contenu{
  \texte{}
  \question{Soient $A,B$ des matrices de $\mathcal{M}_n(\R)$ symétriques et 
$f : \R \to \R, t \mapsto {\max(\mathrm{Sp}(A+tB)).}$
Montrer que $f$ est convexe.}
  \reponse{Pour $A$ symétrique réelle on a $\max(\mathrm{Sp}(A)) = \sup\{(x\mid Ax)/\|x\|^2,\ x\in\R^n\setminus\{0\}\}$
donc $f$ est la borne supérieure des fontions affines $t \mapsto((x\mid Ax) + t(x\mid Bx))/\|x\|^2$
lorsque $x$ décrit $\R^n\setminus\{0\}$. En tant que sup de fonctions convexes,
c'est une fonction convexe.}
}