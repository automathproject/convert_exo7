\uuid{EGEP}
\exo7id{3325}
\titre{Corps emboîtés}
\theme{Exercices de Michel Quercia, Espaces vectoriels de dimension finie}
\auteur{quercia}
\date{2010/03/09}
\organisation{exo7}
\contenu{
  \texte{Soient $H \subset  K \subset  L$ trois sous-corps de $\C$.}
\begin{enumerate}
  \item \question{Montrer que $ K$ et $ L$ sont des $ H$-ev et $ L$ est un $ K$-ev.}
  \item \question{Montrer que $ L$ est de dimension finie sur $ H$ si et seulement si
    $ K$ est de dimension finie sur $ H$ et
    $ L$ est de dimension finie sur $ K$.}
  \item \question{Application : Montrer que $\overline\Q$, la cloture algébrique de $\Q$ dans~$\C$, est
    un corps algébriquement clos (si $P\in\overline\Q[X]$, considérer le sous-corps
    de~$\C$ engendré par les coefficients de~$P$).}
\end{enumerate}
\begin{enumerate}

\end{enumerate}
}