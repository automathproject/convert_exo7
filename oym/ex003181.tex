\exo7id{3181}
\titre{$P(X) + P(X+1) = 2X^n$}
\theme{}
\auteur{quercia}
\date{2010/03/08}
\organisation{exo7}
\contenu{
  \texte{}
\begin{enumerate}
  \item \question{Montrer qu'il existe un uni\-que polyn{\^o}me
     $P_n \in { K[X]}$ tel que $P_n(X) + P_n(X+1) = 2X^n$.}
  \item \question{Chercher une relation de r{\'e}currence entre $P_n'$ et $P_{n-1}$.}
  \item \question{D{\'e}composer $P_n(X+1)$ sur la base $(P_k)_{k \in \N}$.}
  \item \question{D{\'e}montrer que $P_n(1-X) = (-1)^nP_n(X)$.}
\end{enumerate}
\begin{enumerate}
  \item \reponse{Isomorphisme $P \longmapsto P(X) + P(X+1)$.}
  \item \reponse{$P_n' = nP_{n-1}$.}
  \item \reponse{$P_n(X+1) = \sum_{k=0}^n C_n^k P_k$\quad (Taylor).}
  \item \reponse{$Q_n(X) = P_n(1-X)  \Rightarrow  Q_n(X) + Q_n(X+1) = 2(-1)^nX^n$.}
\end{enumerate}
}