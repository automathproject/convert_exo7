\uuid{6934}
\titre{Exercice 6934}
\theme{}
\auteur{ruette}
\date{2013/01/24}
\organisation{exo7}
\contenu{
  \texte{}
  \question{Sur un lot de  700 boulons soumis à des 
essais de rupture, 300  ont résisté. Sur un second lot de  225, 125  
ont résisté. Peut-on admettre, au seuil de  5\%, que ces deux lots 
appartiennent à la même population ?}
  \reponse{On choisit de comparer les fréquences observées sur les deux échantillons.

On fait l'hypothèse $H_0$ : les deux échantillons sont des tirages de 
variables de Bernoulli indépendantes ayant la même espérance $p_{pop}$. 

Sous l'hypothèse $H_{0}$, l'effectif $n_{i}F_{i}$ suit une loi binomiale 
$\mathcal{B}(p_{pop},n_{i})$. Les échantillons étant de grande taille 
($n_1$, $n_2 \geq 30$), on peut supposer que $F_{i}$ suit une loi 
normale $\mathcal{N}(p_{pop},\sqrt{p_{pop}(1-p_{pop})/n_{i}})$. De plus, 
pour le calcul de la variance, on peut estimer $p_{pop}$ au moyen de 
l'ensemble des données disponibles, i.e.
\begin{eqnarray*}
p_{pop}\sim \frac{300+125}{700+225}=0,46.
\end{eqnarray*}
La différence $D=F_{1}-F_{2}$ suit une loi normale d'espérance nulle et de variance
$
\text{Var}(D)=\text{Var}(F_{1})+\text{Var}(F_{2})=p_{pop}(1-p_{pop})
(\frac{1}{n_{1}}+\frac{1}{n_{2}})\sim 0,2484(\frac{1}{700}+\frac{1}{225})=0,00146.
$

Par conséquent, on choisit comme fonction discriminante $T=D/\sqrt{0.00146}=D/0,0382$, 
qui suit une loi normale standard.
La table donne $p(|T|>1,96)=0,05$.
Or, sur l'échantillon mesuré, la variable $T$ prend la valeur 
$t=\frac{-0,127}{0,0382}=-3,324$. Comme $|t|>1,96$, on rejette 
$H_{0}$. On conclut, avec au plus 5\% de chances de se tromper, 
que les deux lots de boulons ne sont pas tirés de la même population.}
}