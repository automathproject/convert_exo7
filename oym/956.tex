\uuid{956}
\titre{Exercice 956}
\theme{Applications linéaires, Injectivité, surjectivité, isomorphie}
\auteur{legall}
\date{1998/09/01}
\organisation{exo7}
\contenu{
  \texte{}
  \question{Pour les applications lin\'eaires suivantes,  d\'eterminer $\ker f_i$ et 
$\Im f_i$. En d\'eduire si $f_i$ est injective, surjective, bijective.
$$\begin{array}{rl}
f_1 : \Rr^2 \to \Rr^2 & f_1(x,y)=(2x+y,x-y)  \\
f_2 : \Rr^3 \to \Rr^3 & f_2(x,y,z)=(2x+y+z,y-z,x+y) \\
f_3 : \Rr^2 \to \Rr^4 & f_3(x,y)=(y,0,x-7y,x+y) \\
f_4 : \Rr_3[X] \to \Rr^3 & f_4(P) = \big( P(-1), P(0), P(1) \big) \\
\end{array}
$$}
  \reponse{Calculer le noyau revient à résoudre un système linéaire,
et calculer l'image aussi. On peut donc tout faire ``à la main''.

Mais on peut aussi appliquer un peu de théorie ! Noyau et image sont liés par la formule du rang : 
$\dim \ker f + \dim \Im f = \dim E$
pour $f : E \to F$. Donc si on a trouvé le noyau alors on connaît la dimension de l'image.
Et il suffit alors de trouver autant de vecteur de l'image.


\begin{enumerate}
  \item $f_1$ est injective, surjective (et donc bijective).

  \begin{enumerate}
  \item Faisons tout à la main. Calculons le noyau :
\begin{align*}
(x,y) \in \ker f_1
  & \iff f_1(x,y) = (0,0) 
  \iff (2x+y,x-y) = (0,0) \\
  &\iff \begin{cases}
         2x+y=0 \\
        x-y=0 \\   
        \end{cases}
  \iff (x,y)=(0,0) \\
\end{align*}
Ainsi $\ker f_1 = \{ (0,0) \}$ et donc $f_1$ est injective.

   \item Calculons l'image. Quels éléments $(X,Y)$ peuvent s'écrire $f_1(x,y)$ ?
\begin{align*}
f_1(x,y) = (X,Y) 
  & \iff (2x+y,x-y) = (X,Y) \\
  &\iff \begin{cases}
         2x+y=X \\
         x-y=Y \\   
        \end{cases} 
  \iff \begin{cases}
         x=\frac{X+Y}{3} \\
         y=\frac{X-2Y}{3} \\   
        \end{cases} \\
  & \iff (x,y)=\left(\frac{X+Y}{3},\frac{X-2Y}{3}\right) \\
\end{align*}
Donc pour n'importe quel $(X,Y)\in\Rr^2$ on trouve un antécédent $(x,y)=(\frac{X+Y}{3},\frac{X-2Y}{3})$
qui vérifie donc $f_1(x,y)=(X,Y)$. Donc $\Im f_1 = \Rr^2$. Ainsi $f_1$ est surjective.

  \item Conclusion : $f_1$ est injective et surjective donc bijective.
  \end{enumerate}


  \item 
\begin{enumerate}
  \item Calculons d'abord le noyau :
\begin{align*}
(x,y,z) \in \ker f_2
  & \iff f_2(x,y,z) = (0,0,0) \\
  & \iff (2x+y+z,y-z,x+y) = (0,0,0) \\
  &\iff \begin{cases}
         2x+y+z = 0 \\
         y-z = 0 \\   
         x+y = 0 \\ 
        \end{cases} \\
  & \iff  \begin{cases}
         x = -z \\
         y = z \\   
        \end{cases} \\
  & \iff 
  \begin{pmatrix}x\\y\\z\end{pmatrix} = \begin{pmatrix}-z\\z\\z\end{pmatrix}\\ 
  &\iff 
  \begin{pmatrix}x\\y\\z\end{pmatrix} \in 
  \text{Vect} \begin{pmatrix}-1\\1\\1\end{pmatrix} 
  = \left\lbrace \lambda\begin{pmatrix}-1\\1\\1\end{pmatrix} \mid \lambda \in \Rr \right\rbrace\\ 
\end{align*}
Ainsi $\ker f_2 =  \text{Vect}(-1,1,1) $ et donc $f_2$ n'est pas injective.

  \item Maintenant nous allons utiliser que $\ker f_2 = \text{Vect}(-1,1,1)$ , autrement dit
$\dim  \ker f_2 = 1$.
La formule du rang, appliquée à $f_2 : \Rr^3 \to \Rr^3$ s'écrit 
$\dim \ker f_2 + \dim \Im f_2 = \dim \Rr^3$. Donc $\dim \Im f_2 = 2$.
Nous allons trouver une base de $\Im f_2$. Il suffit donc de trouver deux vecteurs linéairement indépendants.
Prenons par exemple
$v_1 = f_2(1,0,0) =  (2,0,1) \in\Im f_2$
et $v_2 = f_2(0,1,0) = (1,1,1) \in \Im f_2$. Par construction ces vecteurs sont dans l'image de $f_2$ et 
il est clair qu'ils sont linéairement indépendants. Donc $\{v_1,v_2\}$ est une base de $\Im f_2$.

  \item $f_2$ n'est ni injective, ni surjective (donc pas bijective).
  \end{enumerate}


  \item Sans aucun calcul on sait $f_3: \Rr^2 \to \Rr^4$ ne peut être surjective 
car l'espace d'arrivée est de dimension strictement
supérieur à l'espace de départ. 
  \begin{enumerate}
  \item Calculons le noyau :
\begin{align*}
(x,y) \in \ker f_3
  & \iff f_3(x,y) = (0,0,0,0) \\
  & \iff  (y,0,x-7y,x+y) = (0,0,0,0) \\
  & \iff \begin{cases}
         y = 0 \\
         0  = 0 \\   
         x-7y = 0 \\ 
         x+y = 0 \\ 
        \end{cases} \\
  & \iff \cdots \\ 
  & \iff (x,y)=(0,0) \\
\end{align*}
Ainsi $\ker f_3 = \{ (0,0) \}$ et donc $f_3$ est injective.

  \item La formule du rang, appliquée à $f_3 : \Rr^2 \to \Rr^4$ s'écrit 
$\dim \ker f_3 + \dim \Im f_3 = \dim \Rr^2$. Donc $\dim \Im f_3 = 2$.
Ainsi $\Im f_3$ est un espace vectoriel de dimension $2$ inclus dans $\Rr^3$,
 $f_3$ n'est pas surjective.

Par décrire $\Im f_3$ nous allons trouver deux vecteurs indépendants de $\Im f_3$.
Il y a un nombre infini de choix : prenons par exemple 
$v_1 = f(1,0) = (0,0,1,1)$. Pour $v_2$ on cherche (un peu à tâtons) un vecteur linéairement indépendant de $v_1$.
Essayons $v_2 = f(0,1)=(1,0,-7,1)$. Par construction $v_1,v_2 \in \Im f$ ; ils sont clairement linéairement indépendants
et comme $\dim \Im f_3=2$ alors $\{v_1,v_2\}$ est une base de $\Im f_3$.

Ainsi $\Im f_3= \text{Vect}\{v_1,v_2\} =\big\{ \lambda(0,0,1,1) + \mu (1,0,-7,1) \mid \lambda,\mu \in \Rr \big\}$.
  \end{enumerate}

  
  
  \item $f_4 : \Rr_3[X] \to \Rr^3$ va d'un espace de dimension $4$ vers un espace de dimension strictement plus petit
et donc $f_4$ ne peut être injective.

\begin{enumerate}
  \item Calculons le noyau. \'Ecrivons un polynôme $P$ de degré $\le 3$ sous la forme
$P(X)= aX^3+bX^2+cX+d$. Alors $P(0) = d$, $P(1)=a+b+c+d$, $P(-1)=-a+b-c+d$.
\begin{align*}
P(X) \in \ker f_4 
  & \iff \big( P(-1), P(0), P(1) \big)= (0,0,0)  \\
  & \iff (-a+b-c+d,d,a+b+c+d) = (0,0,0) \\
  &\iff \begin{cases}
          -a+b-c+d = 0 \\
          d = 0 \\   
          a+b+c+d = 0 \\ 
        \end{cases}\\
 & \iff \cdots \\
 & \iff \begin{cases}
          a = -c \\
          b = 0 \\  
          d = 0 \\ 
        \end{cases} \\
  & \iff (a,b,c,d)=(t,0,-t,0) \quad t\in\Rr \\
  \end{align*}

Ainsi le noyau $\ker f_4 = \big\{ tX^3 - tX \mid t\in \Rr\big\} = \text{Vect}\{ X^3-X \}$.
$f_4$ n'est pas injective son noyau étant de dimension $1$.

  \item La formule du rang pour $f_4 : \Rr_3[X] \to \Rr^3$ s'écrit 
$\dim \ker f_4 + \dim \Im f_4 = \dim \Rr_3[4]$. Autrement dit
$1+ \dim \Im f_4 = 4$. Donc $\dim \Im f_4 = 3$. 
Ainsi $\Im f_4$ est un espace de dimension $3$ dans $\Rr^3$ donc 
$\Im f_4=\Rr^3$. Conclusion $f_4$ est surjective.
  \end{enumerate}

\end{enumerate}}
}