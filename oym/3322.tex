\uuid{3322}
\titre{$\dim H = \dim K \Leftrightarrow H$ et $K$ ont un supplémentaire commun}
\theme{Exercices de Michel Quercia, Espaces vectoriels de dimension finie}
\auteur{quercia}
\date{2010/03/09}
\organisation{exo7}
\contenu{
  \texte{}
  \question{Soient $H,K$ deux sev d'un ev $E$ de dimension finie.
Montrer que $\dim H = \dim K$ si et seulement si $H \text{ et } K$ ont un supplémentaire commun
(par récurrence sur codim$\,H$).}
  \reponse{codim$\,H = 0 :$ supplémentaire = $\{\vec 0\}$.\par
codim$\,H = p :$ Soit $\vec u \in E \setminus (H \cup K)$ :
Alors $H \oplus  K\vec u$ et $K \oplus  K\vec u$
ont un supplémentaire commun, $L$, donc $H$ et $K$ ont un supplémentaire
commun : $L \oplus  K\vec u$.}
}