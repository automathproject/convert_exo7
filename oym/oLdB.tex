\uuid{oLdB}
\exo7id{4689}
\titre{Somme des chiffres de $n$}
\theme{Exercices de Michel Quercia, Suites convergentes}
\auteur{quercia}
\date{2010/03/16}
\organisation{exo7}
\contenu{
  \texte{Pour $n \in \N^*$, on note $S(n)$ la somme des chiffres de l'{\'e}criture d{\'e}cimale de $n$.}
\begin{enumerate}
  \item \question{Encadrer $S(n+1)$ en fonction de $S(n)$.
    En d{\'e}duire que la suite $\left(\frac {S(n+1)}{S(n)}\right)$ est born{\'e}e.}
  \item \question{Chercher $\inf\left\{\frac {S(n+1)}{S(n)} \text{ tq } n \in \N^*\right\}$, et
             $\sup\left\{\frac {S(n+1)}{S(n)} \text{ tq } n \in \N^*\right\}$.}
  \item \question{La suite $\left(\frac {S(n+1)}{S(n)}\right)$ est-elle convergente ?}
\end{enumerate}
\begin{enumerate}
  \item \reponse{$1 \le S(n+1) \le S(n) + 1  \Rightarrow  0 \le \frac {S(n+1)}{S(n)} \le 2$.}
  \item \reponse{$\inf = 0$ $(99\dots99)$, $\sup = 2$ $(100\dots00)$.}
\end{enumerate}
}