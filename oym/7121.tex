\uuid{7121}
\titre{Trapèzes inscriptibles}
\theme{}
\auteur{megy}
\date{2017/02/08}
\organisation{exo7}
\contenu{
  \texte{}
  \question{% angles inscrits, facile, application directe
Montrer qu'un trapèze est isocèle si et seulement s'il est inscriptible.}
  \reponse{Commençons par rappeler deux points:
\begin{enumerate}
\item dans un trapèze, deux angles non adjacents à une même base sont supplémentaires, puisque les deux bases sont parallèles.
\item un quadrilatère non croisé est inscriptible ssi les angles opposés sont supplémentaires.
\end{enumerate}

Un trapèze est isocèle ssi les angles adjacents à une même base sont égaux, donc (par le premier point ci-dessus) ssi les angles opposés sont supplémentaires, donc (par le deuxième point) ssi il est inscriptible.

\begin{center}
\includegraphics{../images/img007121-1}
\end{center}}
}