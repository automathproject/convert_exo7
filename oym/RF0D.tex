\uuid{RF0D}
\exo7id{4425}
\titre{${a_n}/{(1+a_1)(1+a_2)\dots(1+a_n)}$}
\theme{Exercices de Michel Quercia, Séries numérique}
\auteur{quercia}
\date{2010/03/14}
\organisation{exo7}
\contenu{
  \texte{Soit $(a_n)$ une suite réelle positive.
On pose $u_n = \frac {a_n}{(1+a_1)(1+a_2)\dots(1+a_n)}$.}
\begin{enumerate}
  \item \question{Montrer que la série $\sum u_n$ converge.}
  \item \question{Calculer $\sum_{n=1}^\infty u_n$ lorsque $a_n = \frac 1{\sqrt n}$.}
\end{enumerate}
\begin{enumerate}
  \item \reponse{$u_1 + \dots + u_n = 1 - \frac1{(1+a_1)\dots(1+a_n)} \le 1$.}
  \item \reponse{$\ln\bigl((1+a_1)\dots(1+a_n)\bigr)
              = \sum_{k=1}^n \ln\left(1+\frac1{\sqrt k}\right)
              \to +\infty  \Rightarrow  \sum u_n = 1$ lorsque $n\to\infty$.}
\end{enumerate}
}