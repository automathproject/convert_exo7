\uuid{QCM6}
\exo7id{4897}
\titre{Coord. barycentriques du centre du cercle circonscrit}
\theme{Exercices de Michel Quercia, Propriétés des triangles}
\auteur{quercia}
\date{2010/03/17}
\organisation{exo7}
\contenu{
  \texte{Soit $ABC$ un triangle.
On note : $a = BC$, $b = CA$, $c = AB$,
$\alpha \equiv (\overline{ \vec{AB}, \vec{AC}})$,
$\beta  \equiv (\overline{ \vec{BC}, \vec{BA}})$,
$\gamma \equiv (\overline{ \vec{CA}, \vec{CB}})$.}
\begin{enumerate}
  \item \question{Montrer que pour tout point $M$ du cercle ($ABC$), on a :

      $$a\cos\alpha MA^2 + b\cos\beta MB^2 + c\cos\gamma MC^2 = abc.$$}
  \item \question{En déduire les coordonnées barycentriques du centre du cercle $(ABC)$.}
\end{enumerate}
\begin{enumerate}

\end{enumerate}
}