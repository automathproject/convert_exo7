\uuid{3583}
\titre{Réduction de Jordan (Mines MP 2003)}
\theme{Exercices de Michel Quercia, Réductions des endomorphismes}
\auteur{quercia}
\date{2010/03/10}
\organisation{exo7}
\contenu{
  \texte{}
  \question{Soit $f\in\mathcal{L}(\R^3)$ telle que $\mathrm{Spec}(f) = \{\lambda\}$ et $\dim(\mathrm{Ker}(f-\lambda\mathrm{id}))=2$.

Montrer qu'il existe une base~$\cal B$ dans laquelle $\mathrm{Mat}(f) = \begin{pmatrix}\lambda&0&0\cr0&\lambda&1\cr0&0&\lambda\cr\end{pmatrix}$.}
  \reponse{On se ramène à $\lambda=0$ en rempla\c cant $f$ par~$f-\lambda\mathrm{id}$.
$\Im f$ est de dimension~$1$ stable par~$f$ donc $f_{|\Im f}$ est une homothétie, c'est l'application nulle vu $\mathrm{Spec}(f)$. On en déduit~$\Im f\subset\mathrm{Ker} f$.
Soit~$e_2\in\Im f\setminus\{0\}$, $e_3$ un antécédent de~$e_2$ par~$f$ et $e_1\in\mathrm{Ker} f$ indépendant de~$e_2$.
Alors~${\cal B} = (e_1,e_2,e_3)$ convient.}
}