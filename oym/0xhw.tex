\uuid{0xhw}
\exo7id{4396}
\titre{Volumes}
\theme{Exercices de Michel Quercia, Intégrale multiple}
\auteur{quercia}
\date{2010/03/12}
\organisation{exo7}
\contenu{
  \texte{Calculer le volume des domaines suivants :}
\begin{enumerate}
  \item \question{$D$ est l'intersection du cylindre de révolution d'axe $Oz$ de rayon $a$
    et de la boule de centre $O$ de rayon 1 ($0 < a < 1$).}
  \item \question{$D$ est l'intersection de la boule de centre $O$ de rayon 1 et du cône de
    révolution d'axe $Oz$ et de demi-angle~$\frac\pi4$.}
  \item \question{$D$ est le volume engendré par la rotation d'un disque de rayon $r$ autour
    d'une droite coplanaire avec le disque, située à la distance $R > r$
    du centre du disque (tore de révolution ou chambre à air).}
\end{enumerate}
\begin{enumerate}
  \item \reponse{$V = \frac{4\pi}3 (1-\sqrt{1-a^2}^3\,)$.}
  \item \reponse{$V = \frac{2\pi}3 (2-\sqrt{2})$.}
  \item \reponse{$V = 2\pi^2Rr^2$.}
\end{enumerate}
}