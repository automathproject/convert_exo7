\uuid{Cdc7}
\exo7id{3859}
\titre{Cordes de longueur $1/n$}
\theme{Exercices de Michel Quercia, Fonctions continues}
\auteur{quercia}
\date{2010/03/11}
\organisation{exo7}
\contenu{
  \texte{Soit $f : {[0,1]} \to \R$ continue telle que $f(0) = f(1)$.}
\begin{enumerate}
  \item \question{Montrer qu'il existe $x \in \left[0,\frac 12\right]$ tel que
      $f(x) = f\left(x+\frac 12\right)$.}
  \item \question{Pour $n \in \N, n \ge 2$, montrer qu'il existe
      $x \in \left[0,1-\frac 1n\right]$ tel que $f(x) = f\left(x+\frac 1n\right)$.}
  \item \question{Trouver une fonction $f$ telle que : $\forall\ x\in \left[0,\frac35\right]$,
      $f(x) \ne f(x+\frac25)$.}
  \item \question{Montrer qu'il existe $a > 0$ tel que : $\forall\ b \in {]0,a]},\ \exists\ x
      \in {[0, 1-b]} \text{ tq } f(x) = f(x+b)$.}
\end{enumerate}
\begin{enumerate}

\end{enumerate}
}