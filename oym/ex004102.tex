\uuid{4102}
\titre{Système différentiel à coefficients positifs}
\theme{}
\auteur{quercia}
\date{2010/03/11}
\organisation{exo7}
\contenu{
  \texte{}
  \question{Soit $A : {\R^+} \to {\mathcal{M}_n(\R)}, t \mapsto {(a_{ij}(t))}$ continue avec~:
$\forall\ t\ge 0,\ \forall\ i,j,\ a_{ij}(t)\ge 0$ et $X$ une solution
du système différentiel $X'(t) = A(t)X(t)$.
Montrer que si toutes les coordonnées de $X(0)$ sont positives ou
nulles il en est de même pour $X(t)$ pour tout~$t$ (commencer par
le cas strictement positif).}
  \reponse{Si $x_i(0)>0$ pour tout~$i$ on obtient une contradiction
en considérant le plus petit $t$ tel qu'il existe $i$ avec $x_i(t)<0$.

Cas général~: dépendance continue de la solution par rapport aux conditions
initiales\dots}
}