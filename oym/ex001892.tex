\exo7id{1892}
\titre{Exercice 1892}
\theme{}
\auteur{legall}
\date{2003/10/01}
\organisation{exo7}
\contenu{
  \texte{Soit $ 
(E,\Vert \quad \Vert ) $ un espace vectoriel
norm\'e sur $\Rr$. On pose $$\mu (E) = \sup _{x,y \in 
E-(0,0)}\frac{\Vert x+y\Vert ^2+\Vert x-y\Vert ^2}{2(\Vert x\Vert 
^2+\Vert y\Vert ^2)}.$$}
\begin{enumerate}
  \item \question{Montrer que $1\leq \mu (E)\leq 2$.}
  \item \question{Calculer $\mu (\Rr ^2)$ lorsque $\R^2$ est muni de la norme 
euclidienne puis de la norme $\Vert (x,y)\Vert _{\infty}=\max \{\vert 
x\vert, \vert y\vert \}$.}
\end{enumerate}
\begin{enumerate}

\end{enumerate}
}