\uuid{2188}
\titre{Exercice 2188}
\theme{}
\auteur{debes}
\date{2008/02/12}
\organisation{exo7}
\contenu{
  \texte{}
  \question{Montrer que le groupe des isom\'etries de l'espace affine
euclidien de dimension $3$
qui laissent invariant un t\'etra\`edre r\'egulier de sommets $a_1,
a_2,a_3,a_4$ est isomorphe \`a $S_4$
et que le sous-groupe des isom\'etries directes qui laissent invariant le
t\'etra\`edre est isomorphe \`a
$A_4$.}
  \reponse{Notons $G$ le groupe des isom\'etries de l'espace euclidien de dimension $3$ laissant
invariant l'ensemble $\{a_1,\ldots,a_4\}$ des $4$ sommets d'un t\'etra\`edre r\'egulier.
Le fixateur $G(a_4)$ agit transitivement sur $\{a_1,a_2,a_3\}$: en effet ce sous-groupe
contient la rotation d'axe la droite joignant $a_4$ au centre de gravit\'e du triangle
de sommets $a_1,a_2,a_3$, laquelle agit sur ces points comme un $3$-cycle. D'apr\`es
l'exercice \ref{ex:deb86}, le groupe $G$ agit $2$-transitivement sur
$\{a_1,\ldots,a_4\}$. De plus $G(a_4)$ contient une isom\'etrie agissant sur
$\{a_1,\ldots,a_4\}$ comme une transposition, par exemple la sym\'etrie par rapport au
plan m\'ediateur $P$ du segment $[a_1,a_2]$, laquelle
\'echange $a_1$ et $a_2$ et fixe $a_3$ et $a_4$ qui sont dans $P$. D'apr\`es
l'exercice \ref{ex:deb85}, on a $G\simeq S_4$.
\smallskip

Notons $G_+$ le sous-groupe de $G$ constitu\'e de ses isom\'etries directes. Le groupe
$G_+$ est le noyau du morphisme $\hbox{\rm det}: G_+ \rightarrow \{1,-1\}$ qui \`a tout
$g\in G$ vu comme matrice associe son d\'eterminant. Comme ce morphisme est surjectif
(la rotation et la sym\'etrie consid\'er\'ees ci-dessus sont respectivement    
directe et indirecte), $G_+$ est d'indice $2$. D'o\`u $G\simeq A_4$ puisque $A_4$ est le
seul sous-groupe de $S_4$ d'indice $2$ (cf exercice \ref{ex:deb75}).}
}