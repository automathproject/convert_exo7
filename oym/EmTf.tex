\uuid{EmTf}
\exo7id{412}
\titre{Exercice 412}
\theme{Polynômes, Racines et factorisation}
\auteur{cousquer}
\date{2003/10/01}
\organisation{exo7}
\contenu{
  \texte{Dans $\mathbb{R}[X]$ et dans $\mathbb{C}[X]$, 
d\'ecomposer les polyn\^omes suivants en facteurs
irr\'eductibles.}
\begin{enumerate}
  \item \question{$X^3-3$.}
  \item \question{$X^{12}-1$.}
\end{enumerate}
\begin{enumerate}
  \item \reponse{$\left \lbrace\begin{array}{rcl}
  X^3-3 & = &(X-\sqrt[3]{3})(X^2+\sqrt[3]{3}\, X+\sqrt[3]{9})\\
  & = & (X-\sqrt[3]{3}) 
         (X+{\sqrt[3]{3}\over2}- i{\sqrt3\sqrt[3]{3}\over2})
         (X+{\sqrt[3]{3}\over2}+ i{\sqrt3\sqrt[3]{3}\over2}).
\end{array}\right.$}
  \item \reponse{$\left\lbrace\begin{array}{rcl}
  X^{12}-1 & = & \textstyle(X-1) (X+1) (X^2+1) (X^2-X+1) (X^2+X+1)\times{}\\
           &   & \qquad\textstyle(X^2-\sqrt3\,X+1) (X^2+\sqrt3\,X+1)\\
           & = & \textstyle(X-1) (X+1) (X- i) (X+ i)\times{}\\
           &   & \qquad\textstyle\bigl(X-{1+ i\sqrt3\over2}\bigr)
                 \bigl(X-{1- i\sqrt3\over2}\bigr)
                 \bigl(X-{-1+ i\sqrt3\over2}\bigr) 
                 \bigl(X-{-1- i\sqrt3\over2}\bigr)\times{}\\
           &   & \qquad\textstyle\bigl(X-{\sqrt3+ i\over2}\bigr) 
                 \bigl(X-{\sqrt3- i\over2}\bigr)
                 \bigl(X-{-\sqrt3+ i\over2}\bigr) 
                 \bigl(X-{-\sqrt3- i\over2}\bigr).
\end{array}\right.$}
\end{enumerate}
}