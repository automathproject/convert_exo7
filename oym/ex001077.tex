\uuid{1077}
\titre{Exercice 1077}
\theme{}
\auteur{gourio}
\date{2001/09/01}
\organisation{exo7}
\contenu{
  \texte{}
  \question{Soit $A\in M_{n}({\Rr}) $ nilpotente, on d\'{e}finit :
$$\exp A=\sum_{i\geq 0}\frac{A^{i}}{i!}, $$
la somme \'{e}tant finie et s'arr\^{e}tant par exemple au premier indice $i $
tel que $A^{i}=0.$
Montrer que si $A$ et $B$ sont nilpotentes et commutent, alors $\exp(A+B)=\exp(A)\exp(B).$
En d\'{e}duire que $\exp(A)$ est toujours inversible et calculer son inverse.}
  \reponse{}
}