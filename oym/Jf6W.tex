\uuid{Jf6W}
\exo7id{4490}
\titre{\'Etude de convergence}
\theme{Exercices de Michel Quercia, Familles sommables}
\auteur{quercia}
\date{2010/03/14}
\organisation{exo7}
\contenu{
  \texte{\'Etudier la finitude des sommes suivantes~:}
\begin{enumerate}
  \item \question{$\sum_{(i,j)\in(\N^*)^2}\frac1{(i+j)^\alpha}$.}
  \item \question{$\sum_{(i,j)\in(\N^*)^2}\frac1{\strut i^\alpha +j^\alpha}$.}
  \item \question{$\sum_{x\in\Q\cap[1,+\infty[}\frac1{x^2}$.}
  \item \question{$\sum_{(p,q)\in\N^2}\frac1{a^p+b^q}$, ${a>1,b>1}$.}
\end{enumerate}
\begin{enumerate}
  \item \reponse{Regroupement à $i+j$ constant $ \Rightarrow $ CV ssi $\alpha > 2$.}
  \item \reponse{Pour $\alpha\ge 1$ on a par convexité~:
             $2^{1-\alpha}(i+j)^\alpha \le i^\alpha+j^\alpha \le (i+j)^\alpha$
             donc il y a convergence ssi $\alpha > 2$.}
  \item \reponse{Il y a une infinité de termes supérieurs à~$1/4$.}
  \item \reponse{$\frac1{a^p+b^q} \le \frac1{2\sqrt a^p\sqrt b^q}  \Rightarrow $ sommable.}
\end{enumerate}
}