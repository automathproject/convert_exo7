\uuid{okww}
\exo7id{2736}
\titre{Exercice 2736}
\theme{Résolution de systèmes linéaires par la méthode du Pivot de Gauss}
\auteur{tumpach}
\date{2009/10/25}
\organisation{exo7}
\contenu{
  \texte{Les vecteurs complexes $(z, w)$ et $(z', w')$ sont li\'es par la formule $(z', w') = (z + iw, (1+i)z + (1-2i)w)$. Un \'etudiant qui n'aime pas les nombres complexes pose $z = x + iy$, $w = u + iv$, $z' = x' + i y'$ et $w' = u'+iv'$.}
\begin{enumerate}
  \item \question{Exprimer $(x', y', u', v')$ en fonction de $(x, y, u, v)$.}
  \item \question{R\'esoudre le syst\`eme $(x', y', u', v') = (1, 2, 3, 4)$.}
\end{enumerate}
\begin{enumerate}

\end{enumerate}
}