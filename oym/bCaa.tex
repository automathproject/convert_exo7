\uuid{bCaa}
\exo7id{3332}
\titre{Rang de $f+g$}
\theme{Exercices de Michel Quercia, Applications linéaires en dimension finie}
\auteur{quercia}
\date{2010/03/09}
\organisation{exo7}
\contenu{
  \texte{Soient $E,F$ deux ev, $E$ de dimension finie, et $f,g \in \mathcal{L}(E,F)$.}
\begin{enumerate}
  \item \question{Démontrer que $\mathrm{rg}(f+g) \le \mathrm{rg}(f) + \mathrm{rg}(g)$.}
  \item \question{Montrer qu'il y a égalité si et seulement si
    $\Im f \cap \Im g = \{\vec 0_F \}$ et $\mathrm{Ker} f + \mathrm{Ker} g = E$.}
\end{enumerate}
\begin{enumerate}

\end{enumerate}
}