\uuid{4723}
\titre{$u_{n+1} - u_n \to 0$}
\theme{Exercices de Michel Quercia, Topologie de $\mathbb{R}
\auteur{quercia}
\date{2010/03/16}
\organisation{exo7}
\contenu{
  \texte{}
  \question{Soit $f : {[0,1]} \to {[0,1]}$ continue, $u_0 \in {[0,1]}$ et $(u_n)$
la suite des it{\'e}r{\'e}es de $f$ en $u_0$.

On suppose que $u_{n+1} - u_n \xrightarrow[n\to\infty]{} 0$.
Montrer que la suite $(u_n)$ converge vers un point fixe de $f$.}
  \reponse{L'ensemble des valeurs d'adh{\'e}rence est un intervalle constitu{\'e} de
         points fixes de $f$ $ \Rightarrow $ la suite $(u_n)$ a une seule valeur
         d'adh{\'e}rence.}
}