\uuid{5958}
\titre{Exercice 5958}
\theme{Théorème de Fubini-Tonelli et convolutions , Théorème de Fubini-Tonelli}
\auteur{tumpach}
\date{2010/11/11}
\organisation{exo7}
\contenu{
  \texte{}
  \question{Montrer que la fonction $(x, y) \mapsto  e^{-y} \sin 2xy$ est
int\'egrable pour la mesure de Lebesgue sur $[0, 1]\times (0,
+\infty)$ ; en d\'eduire la valeur de
$$
\int_{0}^{+\infty} \frac{1}{y} (\sin y)^2 e^{-y}\,dy.
$$}
  \reponse{Le th\'eor\`eme de Tonelli donne~:
$$
\int_{[0, 1]\times (0, +\infty)} |e^{-y} \sin 2xy|\,dx dy \leq
\int_{0}^{+\infty} e^{-y}\,dy = 1 < +\infty,
$$
ce qui prouve que la fonction $(x, y)\mapsto e^{-y} \sin 2xy$ est
int\'egrable pour la mesure de Lebesgue sur $[0,1]\times
(0,+\infty)$.
\\
Le th\'eor\`eme de Fubini donne alors la valeur $I$ de
l'int\'egrale de cette fonction~:
\begin{eqnarray*}
I  = \int_{0}^{1} \,dx \int_{0}^{+\infty} e^{-y} \sin 2xy\,dy
\begin{array}{c} \text{(IPP)}\\= \end{array}
\int_{0}^{1} (2x) (1+4x^{2})^{-1}\,dx = \frac{\log 5}{4}\\
I  = \int_{0}^{+\infty} e^{-y} \,dy \int_{0}^{1} \sin 2xy \,dx =
\int_{0}^{+\infty} e^{-y}~ \frac{\sin^2 y}{y}\,dy.
\end{eqnarray*}}
}