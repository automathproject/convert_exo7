\uuid{2792}
\titre{Exercice 2792}
\theme{Dérivabilité au sens complexe, fonctions analytiques, Dérivabilité complexe}
\auteur{burnol}
\date{2009/12/15}
\organisation{exo7}
\contenu{
  \texte{}
  \question{\label{ex:burnol1.1.10}
  Sur un ouvert connexe $U$ on se donne une fonction
  holomorphe $f$ qui a la propriété de ne prendre que des
  valeurs réelles. En utilisant les équations de
  Cauchy-Riemann, montrer que $f$ est constante.}
  \reponse{Soit $f(z)=u(z)+iv(z)$ pour $z\in U$. Si $f$ ne prend que des valeurs r\'eelles, alors $v\equiv 0$. On tire des
\'equations de Cauchy-Riemann
$$\frac{\partial u}{\partial x} =\frac{\partial v}{\partial y}=0$$
$$\frac{\partial u}{\partial y}=-\frac{\partial v}{\partial x}\equiv 0.$$
La d\'eriv\'ee de $f$ est alors identiquement nulle sur l'ouvert connexe $\Omega$ ce qui implique que $f$ est constante
(voir l'exercice \ref{ex:burnol1.1.9}).}
}