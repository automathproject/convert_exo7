\uuid{7896}
\titre{Les groupes $SO(3)$ et $SU(2)$}
\theme{}
\auteur{mourougane}
\date{2021/08/11}
\organisation{exo7}
\contenu{
  \texte{}
\begin{enumerate}
  \item \question{Montrer que la classe de conjugaison $C$ de $SU(2)$ des matrices de trace nulle (i.e. la latitude $0$) 
est la sphère unité de l'espace euclidien $(V,\ll~~,\gg)$ des matrices anti-hermitienne de trace nulle.}
  \item \question{Montrer que $SU(2)$ agit par conjugaison sur l'espace $V$.}
  \item \question{Montrer que cette action est transitive.}
  \item \question{En déduire un morphisme de groupes $\phi$ de $SU(2)$ dans le groupe orthogonal de $(V,\ll~~,\gg)$.}
  \item \question{Déterminer le noyau de $\phi$.}
  \item \question{En utilisant la connexité de $SU(2)$ montrer que l'image de $\phi$ est incluse dans $SO(V)$.}
  \item \question{Montrer que l'image par $\phi$ du sous-groupe $D$ des matrices diagonales de $SU(2)$
est le sous-groupe des rotations de $V$ qui fixent $\begin{bmatrix}i&0\\0&-i\end{bmatrix}$.}
  \item \question{En déduire l'image de $\phi$.}
\end{enumerate}
\begin{enumerate}

\end{enumerate}
}