\uuid{a0hF}
\exo7id{3051}
\titre{D{\'e}nombrement de $\N^2$}
\theme{Exercices de Michel Quercia, Propriétés de $\Nn$}
\auteur{quercia}
\date{2010/03/08}
\organisation{exo7}
\contenu{
  \texte{Soit 
\begin{align*}
  f : {\N^2} &\to {\N}, \\ {(p,q)} &\mapsto {{\frac12}(p+q)(p+q+1) + p}.
\end{align*}}
\begin{enumerate}
  \item \question{Montrer pour $q > 0$ : $f(p+1,q-1) = f(p,q)+1$ et $f(0,p+1) = f(p,0)+1$.}
  \item \question{Montrer que : $f(0,p+q) \le f(p,q) < f(0,p+q+1)$.}
  \item \question{Montrer que $g$ : $n \mapsto f(0,n)$ est strictement croissante.}
  \item \question{Montrer que $f$ est injective (on supposera $f(p,q) = f(p',q')$ et on montrera
     dans un premier temps que $p+q = p'+q'$).}
  \item \question{Montrer que $f$ est surjective.}
\end{enumerate}
\begin{enumerate}

\end{enumerate}
}