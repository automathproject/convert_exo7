\uuid{3563}
\titre{X MP$^*$ 2004}
\theme{Exercices de Michel Quercia, Réductions des endomorphismes}
\auteur{quercia}
\date{2010/03/10}
\organisation{exo7}
\contenu{
  \texte{}
  \question{Caractériser les polynômes $P$ tels que~: $\forall\ A\in\mathcal{M}_n(\C)$, $(P(A) = 0)  \Rightarrow  (\mathrm{tr}(A)\in\Z)$.}
  \reponse{Aucun polynôme constant ne convient. Si $P$ est non constant et $\alpha$
est une racine de~$P$ alors en considérant $A=\alpha I_n$ on obtient une
première condition nécessaire~: $n\alpha \in\Z$. Si $P$ a une autre racine $\beta$
alors en prenant $A = \mathrm{diag}(\alpha,\dots,\alpha,\beta)$ on obtient
une deuxième condition nécessaire~: $\beta-\alpha\in\Z$. Ainsi les polynômes~$P$
cherchés ont la propriété suivante~: {\it $\deg(P)\ge 1$ et il existe $u\in\Z$ tel que
toutes les racines de~$P$ sont congrues à $u/n$ modulo~$1$.}
Cette condition est clairement suffisante.}
}