\uuid{1967}
\titre{Exercice 1967}
\theme{}
\auteur{gourio}
\date{2001/09/01}
\organisation{exo7}
\contenu{
  \texte{}
  \question{On appelle enveloppe convexe $co(A) $ d'une partie non vide $A$ d'un espace
affine $E $  l'intersection des ensembles convexes contenant $A$ ;
c'est le plus petit ensemble convexe contenant $A.$ Montrer que c'est aussi
l'ensemble des barycentres \`{a} coefficients positifs de points de $A.$ Que
sont $co(\{A,B\}), co(\{A,B,C\}) $ ?}
  \reponse{}
}