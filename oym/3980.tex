\uuid{3980}
\titre{$f^{2} + (1+f')^{2} \le 1$, Polytechnique MP$^*$ 2006}
\theme{Exercices de Michel Quercia, Dérivation}
\auteur{quercia}
\date{2010/03/11}
\organisation{exo7}
\contenu{
  \texte{}
  \question{Soit $f:\R \to \R$ dérivable telle que $f^{2} + (1+f')^{2} \le 1$.
Montrer que $f$ est nulle.}
  \reponse{On a $f'\le 0$ donc $f$ est décroissante sur~$\R$,
et en particulier elle admet des limites finies, $a$ et $b$ en $-\infty$ et $+\infty$
avec $-1\le b\le a\le 1$.

Supposons $a>0$~: soit $\alpha\in{]0,a[}$. Il existe $x_0\in\R$
tel que $\forall\ x\le x_0,\ f(x)\ge a-\alpha > 0$, d'où
$f'(x)\le -1+\sqrt{a-\alpha}< 0$. Ceci est incompatible avec le caractère
borné de~$f$, donc on a en fait $a\le 0$.
On montre de même que $b\ge 0$ et comme $b\le a$, on a finalement
$a=b=0$.}
}