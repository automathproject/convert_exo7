\uuid{3710}
\titre{Matrice circulante}
\theme{}
\auteur{quercia}
\date{2010/03/11}
\organisation{exo7}
\contenu{
  \texte{Pour tout triplet $(a,b,c)$  de réels on pose 
\[M(a,b,c) = \begin{pmatrix}a &b &c \cr c &a &b \cr b &c &a \cr\end{pmatrix}.\]}
\begin{enumerate}
  \item \question{Montrer que les assertions suivantes sont équivalentes:
\begin{enumerate}}
  \item \question{$M(a,b,c)$ est une matrice de rotation;}
  \item \question{$(a,b,c)$ appartient au cercle $\mathcal C \subset \R^3$ défini par l'intersection de la sphère unité de $\R^3$ et du plan $x+y+z-1=0$;}
  \item \question{$a$, $b$ et $c$ sont les racines d'un polynôme de la forme $P = X^3-X^2+\lambda$ avec
$\lambda \in \left[0,\frac4{27}\right]$.}
\end{enumerate}
\begin{enumerate}
  \item \reponse{On a
\[{}^tMM = I \Leftrightarrow (a^2+b^2+c^2=1 \text{ et } ab+bc+ca=0),\]
 et 
\[\det(M) = 1 \Leftrightarrow a^3+b^3+c^3-3abc=1.\]
En remarquant que 
\[ (a^3+b^3+b^3-3abc)  = (a+b+c)(a^2+b^2+c^2-ab-bc-ca), \] 
on en déduit (a)$\Leftrightarrow$ (b)$\Leftrightarrow$
($a+b+c = 1$ et $ab+ac+bc = 0$), d'où le polynôme $P$.

Or, $P$ a trois racines réelles si et seulement si
 $0 \le \lambda \le \frac 4{27}$(étude de la fonction associée), d'où l'équivalence avec la condition (c).}
  \item \reponse{On voit sur la matrice que $(1,1,1)$ est propre pour la valeur propre $a+b+c$. Si $M \in SO(3)$, c'est donc (l'identité ou) une matrice de rotation suivant l'axe $\R (1,1,1)$. L'angle $\theta$ de la rotation vérifie alors $\mathrm{Tr}(M) = 2\cos(\theta)+1$ d'où $|\theta| = \Arccos\frac{3a-1}{2}$.}
  \item \reponse{L'ensemble est celui de toutes les rotations d'axe $(1,1,1)$. C'est un groupe isomorphe à $\R/2\pi\Z$.}
\end{enumerate}
}