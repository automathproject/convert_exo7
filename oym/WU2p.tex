\uuid{WU2p}
\exo7id{5971}
\titre{Exercice 5971}
\theme{Compléments d'intégration, Théorème de Radon-Nikodym, fonction Bêta}
\auteur{tumpach}
\date{2010/11/11}
\organisation{exo7}
\contenu{
  \texte{\textbf{D\'efinition.}
Soient $\mu$ et $\nu$ deux mesures  sur un espace mesur\'e
$(\Omega, \Sigma)$. On dit que $\nu$ est absolument continue par
rapport \`a  $\mu$ et on \'ecrit $\nu<<\mu$ si
$$
\mu(S) = 0 \Rightarrow \nu(S) = 0
$$
pour tout $S\in\Sigma$. 

\bigskip

\textbf{Th\'eor\`eme de Radon-Nikodym.}
%\label{t}
Soient $\mu$ et $\nu$ deux mesures finies  sur un espace mesur\'e
$(\Omega, \Sigma)$. Si $\nu$ est absolument continue par rapport
\`a $\mu$, alors il existe une fonction positive $h\in L^1(\Omega,
\mu)$ telle que pour toute fonction positive mesurable $F$ on a~:
\begin{equation}\label{radon}
\int_{\Omega} F(x) \,d\nu(x) = \int_{\Omega} F(x) h(x)\,d\mu(x).
\end{equation}

\bigskip


Le but de cet exercice est de d\'emontrer ce th\'eor\`eme de Radon-Nikodym.}
\begin{enumerate}
  \item \question{Posons
$$
\alpha = \mu + 2\nu, \quad\quad \omega = 2\mu + \nu.
$$
On consid\`ere l'espace de Hilbert $L^2(\Omega, \alpha)$ des
fonctions de carr\'e int\'egrable par rapport \`a la mesure
$\alpha$ et l'application lin\'eaire $\varphi~: L^2(\Omega,
\alpha) \rightarrow \mathbb{C}$ donn\'ee par~:
$$
\varphi(f) = \int_{\Omega} f(x)\,d\omega(x).
$$
Montrer que $\varphi~: L^2(\Omega, \alpha) \rightarrow \mathbb{C}$
est une application lin\'eaire continue.}
  \item \question{En d\'eduire qu'il existe $g\in L^{2}(\Omega, \alpha)$ tel
que pour tout $f \in L^2(\Omega, \alpha)$~:
$$
\int_{\Omega} f(2g - 1) \,d\nu =  \int_{\Omega} f(2 - g) \,d\mu.
$$}
  \item \question{Montrer que les ensembles $S_{1l} := \{x\in\Omega, g(x) <
\frac{1}{2} - \frac{1}{l}\}$ et  $S_{2l} := \{x\in \Omega, g(x)
> 2 + \frac{1}{l}\}$ o\`u $l\in\mathbb{N}^*$ v\'erifient $\mu(S_{jl}) = \nu(S_{jl})  = 0$.
En d\'eduire que l'on peut choisir la fonction $g$ de telle
mani\`ere que $\frac{1}{2} \leq g \leq 2$.  Montrer que l'ensemble
$Z = \{ x\in \Omega~: g(x) = \frac{1}{2}\}$ est de $\mu$-mesure
$0$.}
  \item \question{Montrer que la fonction $$h(x) = \frac{2 - g(x)}{2g(x) -
1}$$ est bien d\'efinie, positive, appartient \`a $L^{1}(\Omega,
\mu)$ et satisfait \eqref{radon}.}
\end{enumerate}
\begin{enumerate}

\end{enumerate}
}