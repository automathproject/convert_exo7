\uuid{9rm4}
\exo7id{2220}
\titre{Exercice 2220}
\theme{Décomposition en valeurs singulières. Conditionnement}
\auteur{matos}
\date{2008/04/23}
\organisation{exo7}
\contenu{
  \texte{Soit $A=\left(\begin{array}{cc}
1&0\\
0&10^{-6}
\end{array}\right)$}
\begin{enumerate}
  \item \question{Calculer $\mbox{cond}_2(A)$, $\mbox{cond}_1(A)$ et $\mbox{cond}_\infty(A)$;}
  \item \question{R\'esoudre: 
\begin{itemize}}
  \item \question{$Ax=b$ pour $b=\left(\begin{array}{c}1\\10^{-6}\end{array}\right)$}
  \item \question{$Ay=b+\delta b$ pour $\delta b=\left(\begin{array}{c}10^{-6}\\0\end{array}\right)$ et $Az=b+\Delta b$ pour $\Delta b=\left(\begin{array}{c}0\\10^{-6}\end{array}\right)$
\end{itemize}}
  \item \question{Pour chacune des trois normes consid\'er\'ees, trouver une majoration th\'eorique de
$$\frac{\|y-x\|}{\|x\|} \mbox{ et } \frac{\|z-x\|}{\|x\|}$$
et comparer avec les valeurs exactes. Quelle conclusion?}
\end{enumerate}
\begin{enumerate}

\end{enumerate}
}