\uuid{4586}
\titre{$\sum P(n)x^n$, Ensi P 91}
\theme{}
\auteur{quercia}
\date{2010/03/14}
\organisation{exo7}
\contenu{
  \texte{}
  \question{Rayon et somme de $\sum P(n)x^n$ où $P$ est un polynôme de degré $p$.}
  \reponse{$R=1$. On décompose $P$ sous la forme :
	     $P = a_0 + a_1(X+1) + a_2(X+1)(X+2) + \dots + a_p(X+1)\dots(X+p)$.\par
	     Alors $\sum_{n=0}^\infty P(n)x^n = \frac{a_0}{1-x} +
	     \frac{a_1}{(1-x)^2} + \dots + \frac{p!\,a_p}{(1-x)^{p+1}}$.}
}