\uuid{1247}
\titre{Exercice 1247}
\theme{}
\auteur{vignal}
\date{2001/09/01}
\organisation{exo7}
\contenu{
  \texte{}
  \question{Calculer les limites suivantes
$$\lim_{x\rightarrow 0}\frac{e^{x^2}-\cos x}{x^2}
\quad\quad\lim_{x\rightarrow 0}\frac{\ln (1+x)-\sin x}{x}
\quad\quad \lim_{x\rightarrow 0}\frac{\cos x-\sqrt{1-x^2}}{x^4}$$}
  \reponse{\begin{enumerate}
  \item 
On a 
$$e^{x^2} = 1+x^2+\frac{x^4}{2!} + o(x^4) \quad \text{ et } \quad  \cos x= 1-\frac{x^2}{2!}+\frac{x^4}{4!} + o(x^4)$$

On s'aperçoit qu'en fait un dl à l'ordre $2$ suffit :
$$e^{x^2}-\cos x 
= \big(1+x^2 + o(x^2) \big) - \big(1-\frac{x^2}{2}+ o(x^2) \big)
= \frac32 x^2 + o(x^2)$$
Ainsi $\frac{e^{x^2}-\cos x}{x^2} = \frac32 + o(1)$ (où $o(1)$ désigne une fonction qui tend vers $0$)
 et donc
$$\lim_{x\rightarrow 0}\frac{e^{x^2}-\cos x}{x^2} = \frac32$$


  \item 
On sait que 
$$\ln(1+x)=x-\frac{x^2}{2}+\frac{x^3}{3}+o(x^3) 
\quad \text{ et } \quad \sin x = x-\frac{x^3}{3!} + o(x^3).$$

Les dl sont distincts dès le terme de degré $2$ donc un dl à l'ordre $2$ suffit :
$$\ln (1+x)-\sin x = \big(x - \frac{x^2}{2} + o(x^2) \big) - \big(x + o(x^2) \big) = -\frac{x^2}{2} + o(x^2)$$
donc 
$$\frac{\ln (1+x)-\sin x}{x} = -\frac{x}{2} + o(x)$$
et ainsi 
$$\lim_{x\rightarrow 0}\frac{\ln (1+x)-\sin x}{x} = 0.$$


  \item 
Sachant
$$\cos x= 1-\frac{x^2}{2!}+\frac{x^4}{4!}+ o(x^4)$$
et 
$$\sqrt{1-x^2} = 1-\frac12x^2-\frac18x^4 + o(x^4)$$ alors

\begin{align*}
\frac{\cos x-\sqrt{1-x^2}}{x^4}
 & =\frac{\big(1-\frac{x^2}{2}+\frac{x^4}{24}+ o(x^4)\big)-\big(1-\frac12x^2-\frac18x^4 + o(x^4)\big)}{x^4} \\
 & = \frac{\frac 16 x^4 + o(x^4)}{x^4}  \\
 &= \frac16+o(1) \\
\end{align*}
Ainsi 
$$\lim_{x\rightarrow 0}\frac{\cos x-\sqrt{1-x^2}}{x^4}=\frac16$$
\end{enumerate}}
}