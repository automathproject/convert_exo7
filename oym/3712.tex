\uuid{3712}
\titre{Ensi Physique 92}
\theme{Exercices de Michel Quercia, Espace vectoriel euclidien orienté de dimension 3}
\auteur{quercia}
\date{2010/03/11}
\organisation{exo7}
\contenu{
  \texte{}
  \question{Déterminer la matrice de la rotation $\cal R$ de $\R^3$ dans une base
orthonormée $(\vec i,\vec j,\vec k\,)$ telle que ${\cal R}(\vec u)=\vec u$
avec $\vec u\bigl(\frac1{\sqrt3},\frac{-1}{\sqrt3},\frac1{\sqrt3}\bigr)$
et ${\cal R}(\vec i) = \vec k$. Donner son angle de rotation.}
  \reponse{$M=\begin{pmatrix}0 &-1 &0\cr 0 &0 &-1\cr 1 &0 &0\cr\end{pmatrix}$, $\theta = \frac{2\pi}3$.}
}