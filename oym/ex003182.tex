\exo7id{3182}
\titre{$(1-X)^nP + X^nQ = 1$}
\theme{}
\auteur{quercia}
\date{2010/03/08}
\organisation{exo7}
\contenu{
  \texte{}
\begin{enumerate}
  \item \question{D{\'e}montrer qu'il existe $P,Q \in  K_{n-1}[X]$ uniques tels que
     $(1-X)^nP + X^nQ = 1$.}
  \item \question{Montrer que $Q = P(1-X)$.}
  \item \question{Montrer que : $\exists\ \lambda \in  K$ tel que $(1-X)P' - nP = \lambda X^{n-1}$.}
  \item \question{En d{\'e}duire $P$.}
\end{enumerate}
\begin{enumerate}
  \item \reponse{Bezout g{\'e}n{\'e}ralis{\'e}.}
  \item \reponse{$\bigl((1-X)P' - nP\bigr) (1-X)^{n-1} + \bigl(nQ + XQ'\bigr)X^{n-1} = 0$.}
  \item \reponse{$P^{(k+1)}(0) = (n+k)P^{(k)}(0)  \Rightarrow 
               P = \sum_{k=0}^{n-1} C_{n+k-1}^k X^k$.}
\end{enumerate}
}