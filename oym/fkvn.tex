\uuid{fkvn}
\exo7id{3300}
\titre{$E$ n'est pas union de sous-espaces stricts}
\theme{Exercices de Michel Quercia, Espaces vectoriels}
\auteur{quercia}
\date{2010/03/09}
\organisation{exo7}
\contenu{
  \texte{Soit $E$ un $ K$-ev non nul et $F_1,\dots,F_n$ des sev stricts de $E$.
On veut montrer que $E \ne F_1 \cup \dots \cup F_n$ :}
\begin{enumerate}
  \item \question{Traiter le cas $n = 2$.}
  \item \question{Cas général : on suppose $F_n \not\subset F_1 \cup \dots \cup F_{n-1}$
     et on choisit
     $\vec x \in F_n \setminus (F_1 \cup \dots \cup F_{n-1})$ et
     $\vec y \notin F_n$.
  \begin{enumerate}}
  \item \question{Montrer que : $\forall\ \lambda \in  K$, $\lambda\vec x + \vec y \notin F_n$.}
  \item \question{Montrer que : $\forall\ i \le n-1$, il existe au plus un $\lambda \in  K$ tel que
         $\lambda\vec x + \vec y \in F_i$.}
  \item \question{Conclure.}
\end{enumerate}
\begin{enumerate}

\end{enumerate}
}