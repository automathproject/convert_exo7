\uuid{1958}
\titre{Exercice 1958}
\theme{}
\auteur{liousse}
\date{2003/10/01}
\organisation{exo7}
\contenu{
  \texte{Dans l'exercice suivant, 
on consid\`ere des couples de deux droites $D_1$ et $D_2$ : on doit 
d\'eterminer si
elles sont s\'ecantes, parall\`eles ou confondues. Si elles sont s\'ecantes, 
on d\'eterminera les coordonn\'ees
du point d'intersection, et si elles 
sont parall\`eles ou confondues on d\'eterminera un vecteur directeur.}
\begin{enumerate}
  \item \question{$(D_1) : 3x + 5y - 2 = 0$ et $(D_2) : x - 2y + 3 = 0$}
  \item \question{$(D_1) : 2x - 4y + 1 = 0$ et $(D_2) : -5x + 10y + 3 = 0$}
  \item \question{$(D_1) : \left\{ \begin{array}{l}
                             x = 3 + 4t \\
                             y = 2 - t
                           \end{array} \right.$
 et $(D_2) : \left\{ \begin{array}{l}
                       x = 5 - s \\
                       y = 2 + 3s 
                     \end{array} \right.$}
  \item \question{$(D_1) : \left\{ \begin{array}{l}
                             x = 1 + 2t \\
                             y = 2 - 3t
                           \end{array} \right.$
 et $(D_2) : \left\{ \begin{array}{l}
                       x = 3 - 4s \\
                       y = -1 + 6s 
                     \end{array} \right.$}
  \item \question{$(D_1) : x - 2y + 3 = 0$ et $D_2 :\left\{ \begin{array}{l}
                                                     x = 2 + t  \\
                                                     y = 3 - 2t 
                                                    \end{array} \right.$}
  \item \question{$(D_1) : 3x - 2y + 1 = 0$ et $(D_2) : \left\{ \begin{array}{l}
                                                          x = 1 - 4t \\
                                                          y = 2 - 6t 
                                                        \end{array} \right.$}
\end{enumerate}
\begin{enumerate}

\end{enumerate}
}