\uuid{IycU}
\exo7id{3585}
\titre{Ensi PC 1999}
\theme{Exercices de Michel Quercia, Réductions des endomorphismes}
\auteur{quercia}
\date{2010/03/10}
\organisation{exo7}
\contenu{
  \texte{Soit $A=\begin{pmatrix}-1&2&1\cr2&-1&-1\cr-4&4&3\cr\end{pmatrix}$.}
\begin{enumerate}
  \item \question{Calculer $A^n$.}
  \item \question{Soit $U_0=\begin{pmatrix}-2\cr4\cr1\end{pmatrix}$ et $(U_n)$ défini par la relation~:
    $U_{n+1} = AU_n$. Calculer $U_n$ en fonction de~$n$.}
  \item \question{Soit $X(t) = \begin{pmatrix}x(t)\cr y(t)\cr z(t)\cr\end{pmatrix}$.
    Résoudre $\frac{d X}{d t} = AX$.}
\end{enumerate}
\begin{enumerate}
  \item \reponse{$A^{2k}=I$, $A^{2k+1}=A$.}
\end{enumerate}
}