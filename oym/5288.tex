\uuid{5288}
\titre{***}
\theme{Dénombrements}
\auteur{rouget}
\date{2010/07/04}
\organisation{exo7}
\contenu{
  \texte{}
  \question{Soit $(P)$ un polygone convexe à $n$ sommets. Combien ce polygone a-t-il de diagonales~?~En combien de points distincts des sommets se coupent-elles au maximum~?}
  \reponse{Soit $n\geq5$. De chaque sommet part $n-1$ droites (vers les $n-1$ autres sommets) dont $2$ sont des cotés et $n-3$ des diagonales. Comme chaque diagonale passe par 2 sommets , il y a $\frac{n(n-3)}{2}$ diagonales.

Ces diagonales se recoupent en $C_{n(n-3)/2}^2$ points distincts ou confondus. Dans ce décompte, chaque sommet a été compté autant de fois que l'on a choisi une paire de deux diagonales passant par ce sommet à savoir $C_{n-3}^2$. Maintenant, il y a $n$ sommets.

Réponse : 
\begin{align*}\ensuremath 
C_{n(n-3)/2}^2-nC_{n-3}^2&=\frac{1}{2}\frac{n(n-3)}{2}(\frac{n(n-3)}{2}-1)-n\frac{(n-3)(n-4)}{2}=\frac{n(n-3)}{8}(n(n-3)-2-4(n-4))\\
 &=\frac{n(n-3)}{8}(n^2-7n+14)
\end{align*}

Les diagonales se recoupent en $\frac{n(n-3)(n^2-7n+14)}{8}$ points distincts ou confondus et distincts des sommets (ou encore en $\frac{n(n-3)(n^2-7n+14)}{8}$ points au maximum).}
}