\uuid{ASSI}
\exo7id{3963}
\titre{$(1+k)(1+k^2)\dots(1+k^n)$}
\theme{Exercices de Michel Quercia, Dérivation}
\auteur{quercia}
\date{2010/03/11}
\organisation{exo7}
\contenu{
  \texte{}
\begin{enumerate}
  \item \question{Montrer que : $\forall\ x \ge -1,\ \ln(1+x) \le x$.}
  \item \question{Soit $k \in {]-1,1[}$. On pose $u_n = (1+k)(1+k^2)\dots(1+k^n)$.
    Montrer que la suite $(u_n)$ est convergente
    (traiter séparément les cas $k \ge 0$, $k < 0$).}
\end{enumerate}
\begin{enumerate}
  \item \reponse{Pour $k \ge 0$, la suite $(u_n)$ est croissante et $\ln u_n \le \frac k{1-k}$.

    Pour $k < 0$, $(u_{2n})$ décro\^\i t et converge, et $u_{2n+1} \sim u_{2n}$.}
\end{enumerate}
}