\exo7id{1551}
\titre{Exercice 1551}
\theme{}
\auteur{barraud}
\date{2003/09/01}
\organisation{exo7}
\contenu{
  \texte{\noindent\textbf{A --- } Soit $E$ un espace vectoriel et $u$ et $v$ deux
endomorphismes de
$E$ diagonalisables qui commutent (c'est à dire qui satisfont $u\circ
v=v\circ u$). On note $\lambda_{1}...\lambda_{k}$les valeurs propres de
$u$ et $E_{1}...E_{k}$ les espaces propres associés.}
\begin{enumerate}
  \item \question{Montrer que $v(E_{i})\subset E_{i}$.}
  \item \question{On note $v_{i}=v_{|_{E_{i}}}$ la restriction de $v$ à $E_{i}$. Soit
    $P\in\C[X]$, montrer que $P(v_{i})=P(v)_{|_{E_{i}}}$.}
  \item \question{En déduire que $v_{i}$ est diagonalisable. Soit $B_{i}$ une base de
    $E_{i}$ formée de vecteurs propres de $v_{i}$. 

    Montrer que $B=\bigcup\limits_{i=1}^{k}B_{i}$ est une base de $E$
    formée de vecteurs propres à la fois pour $u$ et pour $v$.}
  \item \question{En déduire que $u$ et $v$ sont diagonalisables dans une même base.
    Montrer que $u-v$ est diagonalisable.}
\end{enumerate}
\begin{enumerate}

\end{enumerate}
}