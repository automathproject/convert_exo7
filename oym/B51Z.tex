\uuid{B51Z}
\exo7id{3345}
\titre{Mines P' 1995}
\theme{Exercices de Michel Quercia, Applications linéaires en dimension finie}
\auteur{quercia}
\date{2010/03/09}
\organisation{exo7}
\contenu{
  \texte{Soit $E$ un $ K$-espace vectoriel de dimension finie et $f\in \mathcal{L}(E)$ nilpotente
d'indice~$n$.

Soit $\phi : {\mathcal{L}(E)} \to {\mathcal{L}(E)}, g \mapsto{f\circ g - g\circ f.}$}
\begin{enumerate}
  \item \question{Montrer que $\phi^p(g) = \sum_{k=0}^p (-1)^kC_p^k f^{p-k}\circ g \circ f^k$.
    En déduire que $\phi$ est nilpotente.}
  \item \question{Soit $a\in \mathcal{L}(E)$. Montrer qu'il existe $b\in \mathcal{L}(E)$ tel que
    $a\circ b \circ a = a$. En déduire l'indice de nilpotence de~$\phi$.}
\end{enumerate}
\begin{enumerate}

\end{enumerate}
}