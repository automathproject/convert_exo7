\uuid{3228}
\titre{Racines de $\sum_{k = 0}^n C_n^k(\sin k\theta)X^k$}
\theme{Exercices de Michel Quercia, Racines de polynômes}
\auteur{quercia}
\date{2010/03/08}
\organisation{exo7}
\contenu{
  \texte{}
  \question{Soit $\theta \in \R$ tel que $\sin n\theta \ne 0$.
D{\'e}montrer que le polyn{\^o}me $P = \sum_{k = 0}^n C_n^k(\sin k\theta)X^k$ a toutes ses
racines r{\'e}elles.}
  \reponse{Pour $x \in \R$, on a $P(x) = \Im\bigl((1+xe^{i\theta})^n\bigr)$.
\\
Donc $P(x) = 0 \iff \exists\ k \in \{0,\dots,n-1\}$ et $\lambda \in \R$
tels que : $1+xe^{i\theta} = \lambda e^{ik\pi/n}$.
\\
On obtient $x_k = \frac{\sin(k\pi/n)}{\sin(\theta-k\pi/n)}$,\quad
$0\le k \le n-1$.}
}