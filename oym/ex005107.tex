\uuid{5107}
\titre{**IT}
\theme{}
\auteur{rouget}
\date{2010/06/30}
\organisation{exo7}
\contenu{
  \texte{}
  \question{Pour $z\neq i$, on pose $f(z)=\frac{z+i}{z-i}$. Montrer que $f$ réalise une bijection
de $D=\{z\in\Cc/\;|z|<1\}$ sur $P=\{z\in\Cc/\;\Re(z)< 0\}$. Préciser $f^{-1}$.}
  \reponse{\begin{enumerate}
 \item  Montrons que la restriction de $f$ à $D$, notée $g$, est bien une application de $D$ dans $P$.
Soit $z\in D$. On a $|z|<1$ et en particulier $z\neq i$. Donc, $f(z)$ existe. De plus,

$$\Re(f(z))=\frac{1}{2}(f(z)+\overline{f(z)})=\frac{1}{2}\left(\frac{z+i}{z-i}+\frac{{\bar z}-i}{{\bar z}+i}\right)=
\frac{1}{2}\frac{2z{\bar z}-2}{(z-i)(\overline{z-i})}=\frac{|z|^2-1}{|z-i|^2}<0.$$
Donc, $f(z)$ est élément de $P$. $g$ est donc une application de $D$ dans $P$.
 \item  Montrons que $g$ est injective.
Soit $(z,z')\in D^2$.

$$g(z)=g(z')\Rightarrow\frac{z+i}{z-i}=\frac{z'+i}{z'-i}\Rightarrow iz'-iz=iz-iz'\Rightarrow 2i(z'-z)=0\Rightarrow
z=z'.$$
 \item  Montrons que $g$ est surjective.
Soient $z\in D$ et $Z\in P$.

$$g(z)=Z\Leftrightarrow\frac{z+i}{z-i}=Z\Leftrightarrow z=\frac{i(Z+1)}{Z-1}\;(\text{car}\;Z\neq1,$$
(ce qui montre que $Z$ admet au plus un antécédent dans $D$, à savoir $z=\frac{i(Z+1)}{Z-1}$ (mais on le sait déjà car
$g$ est injective). Il reste cependant à vérifier que $\frac{i(Z+1)}{Z-1}$ est effectivement dans $D$).
Réciproquement, puisque
$\Re(Z)<0$, 

\begin{center}
$\left|\frac{i(Z+1)}{Z-1}\right|=\frac{|Z+1|}{|Z-1|}<1$
\end{center}
($Z$ étant strictement plus proche de $-1$ que
de $1$) et $z\in D$. Finalement $g$ est une bijection de $D$ sur $P$, et~:~

\begin{center}
\shadowbox{
$\forall z\in
P,\;g^{-1}(z)=\frac{i(z+1)}{z-1}$.
}
\end{center}
\end{enumerate}}
}