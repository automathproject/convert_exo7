\uuid{156}
\titre{Exercice 156}
\theme{}
\auteur{bodin}
\date{1998/09/01}
\organisation{exo7}
\contenu{
  \texte{}
\begin{enumerate}
  \item \question{Dans le plan, on consid\`ere trois droites $\Delta_{1},\Delta_{2},\Delta_{3}$ formant un
``vrai'' triangle : elles ne sont pas concourantes, et il n'y en a pas deux parall\`eles.
Donner le nombre $R_{3}$ de r\'egions (zones blanches) d\'ecoup\'ees par ces trois droites.}
  \item \question{On consid\`ere quatre droites $\Delta_{1},\ldots,\Delta_{4}$, telles qu'il n'en existe pas
trois concourantes, ni deux parall\`eles. Donner le nombre $R_{4}$ de r\'egions d\'ecoup\'ees par
ces quatre droites.}
  \item \question{On consid\`ere $n$ droites $\Delta_{1},\ldots,\Delta_{n}$, telles qu'il n'en existe pas
trois concourantes, ni deux parall\`eles. Soit $R_{n}$ le nombre de r\'egions d\'elimit\'ees par
$\Delta_{1}\ldots\Delta_{n}$, et $R_{n-1}$ le nombre de r\'egions d\'elimit\'ees par
$\Delta_{1}\ldots\Delta_{n-1}$. Montrer que $R_{n}=R_{n-1}+n$.}
  \item \question{Calculer par r\'ecurrence le nombre de r\'egions d\'elimit\'ees par $n$ droites en position
g\'en\'erale, c'est-\`a-dire telles qu'il n'en existe pas trois concourantes ni deux parall\`eles.}
\end{enumerate}
\begin{enumerate}

\end{enumerate}
}