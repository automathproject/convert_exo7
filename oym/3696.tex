\uuid{3696}
\titre{$a\wedge b$, $b\wedge c$, $c\wedge a$ donnés}
\theme{Exercices de Michel Quercia, Espace vectoriel euclidien orienté de dimension 3}
\auteur{quercia}
\date{2010/03/11}
\organisation{exo7}
\contenu{
  \texte{}
  \question{Soit $E$ un espace vectoriel euclidien orienté de dimension 3.\par
Trouver $\vec a,\vec b,\vec c$ connaissant
$\vec u = \vec a\wedge\vec b$,
$\vec v = \vec b\wedge\vec c$
et $\vec w = \vec c\wedge\vec a$
(calculer $\vec u\wedge\vec v$).}
  \reponse{$p = [\vec a,\vec b,\vec c]  \Rightarrow 
 \begin{cases} \vec u\wedge\vec v = p\vec b \cr
         \vec v\wedge\vec w = p\vec c \cr
         \vec w\wedge\vec u = p\vec a \cr\end{cases}$
et $[\vec u,\vec v,\vec w] = p^2$.

si $[\vec u,\vec v,\vec w] < 0$ : pas de solutions.

si $[\vec u,\vec v,\vec w] = 0$ et $\mathrm{rg}(\vec u,\vec v,\vec w) > 1$ :
   pas de solutions.

si $[\vec u,\vec v,\vec w] = 0$ et $\mathrm{rg}(\vec u,\vec v,\vec w) \le 1$ :
   une infinité de solutions.

si $[\vec u,\vec v,\vec w] > 0$ : 2 solutions.}
}