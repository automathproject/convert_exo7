\exo7id{6361}
\titre{Exercice 6361}
\theme{}
\auteur{queffelec}
\date{2011/10/16}
\organisation{exo7}
\contenu{
  \texte{Soit $V$ un champ de vecteurs défini sur $\Omega\subset\Rr^n$. On dit qu'une
application $h:\Omega\to\Rr$ de classe $C^1$ est une {\it intégrale première}
de $V$, si $h\circ\varphi(t)$ est constante sur $J$ pour toute solution
$(\varphi, J)$ de l'équation autonome associée. On suppose le champ de classe
$C^1$ sur $\Omega$.}
\begin{enumerate}
  \item \question{Montrer que 
$h$ est une intégrale première de
$V$ si et seulement si 

$h'(x).V(x)=0$ pour tout $x\in\Omega$.}
  \item \question{Donner une intégrale première sur $\Rr^n$ du système différentiel $X'=AX$
où $A$ est une matrice antisymétrique $n\times n$ (commencer avec $n=2$).}
  \item \question{Soit $f$ une application de classe $C^\infty$ de $\Rr$ dans $\Rr$ telle que $f(0)=0$, et on note $F(x)=\int_0^xf(u)\ du$.

Montrer que la fonction $(x,y)\to y^2+2F(x)$ est une intégrale première sur
$\Rr^2$ du champ de vecteurs 
$V(x,y)=(y,\ -f(x))$ défini sur $\Rr^2$.
On suppose que $F$ tend vers $+\infty$ lorsque $x$ tend vers
$\pm\infty$. Montrer que si une solution  $(x(t),y(t))$ de $X'=V(X)$ est définie
sur un intervalle quelconque $I$, les fonctions $x$ et $y$ sont bornées sur $I$
(remarquer que $F$ est bornée inférieurement).}
\end{enumerate}
\begin{enumerate}

\end{enumerate}
}