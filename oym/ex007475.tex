\exo7id{7475}
\titre{Reconnaître une application affine}
\theme{}
\auteur{exo7}
\date{2021/08/10}
\organisation{exo7}
\contenu{
  \texte{Soit $E$ un plan affine euclidien. Soient $\mathcal{R}$ un repère 
cartésien orthonormé de $E$,
et $f : E\to E$ l'application affine
définie dans $\mathcal{R}$ par l'égalité
$$f \bigl((x,y) \bigr)=\biggl({3x+4y+8\over 5},{4x-3y-1\over 5}
\biggr).$$}
\begin{enumerate}
  \item \question{L'application $f$ possède-t-elle des points fixes ?}
  \item \question{Démontrer que lorsque $M$ décrit $E$, le milieu de 
$(M,f(M))$ décrit une droite dont on précisera
l'équation.}
  \item \question{Démontrer que $f$ est une isométrie dont on précisera la 
nature, l'axe et la composante translation.}
\end{enumerate}
\begin{enumerate}

\end{enumerate}
}