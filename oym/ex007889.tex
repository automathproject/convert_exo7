\uuid{7889}
\titre{Forme alternée}
\theme{}
\auteur{mourougane}
\date{2021/08/11}
\organisation{exo7}
\contenu{
  \texte{Une forme bilinéaire $f$ sur un $k$-espace vectoriel $E$ est dite alternée, si tout vecteur de $E$ est isotrope.
Soit $(E,f)$ un $k$-espace vectoriel de dimension finie muni d'une forme bilinéaire alternée.}
\begin{enumerate}
  \item \question{Soit $(V,f)$ un $k$-espace vectoriel de dimension $2$, muni d'une forme alternée non-dégénérée. Soit $x$ un vecteur non nul.
Montrer qu'il existe un vecteur isotrope $y$ tel que $f(x,y)=1$. On dit alors que $(V,f)$ est un plan symplectique.}
  \item \question{Soit $V$ un sous-espace vectoriel de $E$.
Montrer que si $(V,f_{\mid V})$ est un espace non singulier (i.e. $f_{\mid V}$ non dégénérée)
alors $E=V\oplus^\perp V^\perp$.}
  \item \question{Montrer que $E $ est somme directe orthogonale de droites isotropes et de plans symplectiques.}
  \item \question{Montrer que tous les sous-espaces isotropes maximaux de $E$ ont même dimension.
Déterminer cette dimension en fonction de la dimension de $E$ et du rang de $f$.}
  \item \question{Retrouver ce résultat en utilisant le théorème de Witt symplectique :
Soit $(E,f)$ et $(E',f')$ deux $k$-espaces vectoriels de dimension finie muni d'une forme symplectique (i.e. bilinéaire alternée non-dégénérée).
On suppose $(E,f)$ et $(E',f')$ isométriques.
Alors, toute isométrie d'un sous-espace de $(E,f)$ sur un sous-espace de $(E',f')$ se prolonge en une isométrie de $(E,f)$ sur $(E',f')$.}
\end{enumerate}
\begin{enumerate}

\end{enumerate}
}