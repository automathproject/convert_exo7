\uuid{km7q}
\exo7id{6008}
\titre{Exercice 6008}
\theme{Variables aléatoires discrètes}
\auteur{quinio}
\date{2011/05/20}
\organisation{exo7}
\contenu{
  \texte{L'oral d'un concours comporte au total 100 sujets; les candidats
tirent au sort trois sujets et choisissent alors le sujet traité parmi
ces trois sujets. Un candidat se présente en ayant révisé 60
sujets sur les 100.}
\begin{enumerate}
  \item \question{Quelle est la probabilité pour que le candidat ait révisé:
\begin{enumerate}}
  \item \question{les trois sujets tirés;}
  \item \question{exactement deux sujets sur les trois sujets;}
  \item \question{aucun des trois sujets.}
\end{enumerate}
\begin{enumerate}
  \item \reponse{Les trois sujets tirés ont été révisés : $P[X=3]=\frac{\binom{60}{3}}{\binom{100}{3}}$.}
  \item \reponse{Deux des trois sujets tirés ont été révisés: $P[X=2]=\frac{\binom{60}{2}.\binom{40}{1}}{\binom{100}{3}}$.}
  \item \reponse{Aucun des trois sujets: $P[X=0]=\frac{\binom{40}{3}}{\binom{100}{3}}$.}
\end{enumerate}
}