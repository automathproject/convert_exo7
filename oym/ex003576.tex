\uuid{3576}
\titre{Crochet de Lie (Ens Cachan MP$^*$ 2003)}
\theme{}
\auteur{quercia}
\date{2010/03/10}
\organisation{exo7}
\contenu{
  \texte{}
  \question{Soit $\Phi : {\mathcal{M}_n(\C)}\to {\mathcal{M}_n(\C)}$ un automorphisme d'ev tel que~:
$\forall\ A,B\in\mathcal{M}_n(\C),\ \Phi([A,B]) = [\Phi(A),\Phi(B)]$ où $[X,Y] = XY - YX$.
Montrer~: $\forall\ D\in\mathcal{M}_n(\C)$, ($D$ est diagonalisable) $\Leftrightarrow$ ($\Phi(D)$ est diagonalisable).

{\it Indication~: considérer $\phi_D$ : $X  \mapsto[D,X]$ et montrer que
($D$ est diagonalisable) $\Leftrightarrow$ ($\phi_D$ est diagonalisable).}}
  \reponse{Si $D$ est diagonalisable alors les applications $X \mapsto DX$ et $X \mapsto XD$
le sont (annulateur scindé à racines simples) et elles commutent, donc elles sont
simmultanément diagonalisables et leur différence, $\phi_D$, est aussi diagonalisable.

Pour la réciproque, on commence par constater que si $P$ est un polynôme
quelconque, alors~:
$$\forall\ X\in\mathcal{M}_n(\C),\ P(\phi_D)(X) = \sum_{k=0}^{\deg(P)} (-1)^kD^kX\frac{P^{(k)}(D)}{k!}
= \sum_{k=0}^{\deg(P)} (-1)^k\frac{P^{(k)}(D)}{k!}XD^k.$$
(formule du binôme pour $P=X^m$ et linéarité de chaque membre par rapport à~$P$
pour $P$ quelconque).

Supposons $\phi_D$ diagonalisable, prenons~$P$ annulateur scindé à racines simples de~$\phi_D$,
$X=U^tV$ où $U$ est un vecteur propre de~$D$
associé à une certaine valeur propre~$\lambda$ et $V$ un vecteur arbitraire. Donc~:
$$0 = \sum_{k=0}^{\deg(P)} (-1)^k\lambda^kU^tV\frac{P^{(k)}(D)}{k!}
= U^tV\sum_{k=0}^{\deg(P)} (-1)^k\lambda^k\frac{P^{(k)}(D)}{k!}
= U^tVP(D-\lambda I).$$
Comme $U\ne 0$, ceci implique ${}^tVP(D-\lambda I) = 0$ pour tout~$V$,
donc $P(D-\lambda I) = 0$. Ainsi $D-\lambda I$ est diagonalisable et $D$ itou.}
}