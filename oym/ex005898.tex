\exo7id{5898}
\titre{*** I}
\theme{}
\auteur{rouget}
\date{2010/10/16}
\organisation{exo7}
\contenu{
  \texte{Résoudre les équations aux dérivées partielles suivantes :}
\begin{enumerate}
  \item \question{$2 \frac{\partial f}{\partial x}- \frac{\partial f}{\partial y}= 0$ en posant $u = x+y$ et $v = x+2y$.}
  \item \question{$x \frac{\partial f}{\partial y}-y \frac{\partial f}{\partial x}=0$ sur $\Rr^2\setminus\{(0,0)\}$ en passant en polaires.}
  \item \question{$x^2 \frac{\partial^2f}{\partial x^2}+2xy \frac{\partial^2f}{\partial x\partial y}+y^2 \frac{\partial^2f}{\partial y^2}= 0$ sur $]0, +\infty[\times\Rr$ en posant $x = u$ et $y = uv$.}
\end{enumerate}
\begin{enumerate}
  \item \reponse{Soit $f$ une application de classe $C^1$ sur $\Rr^2$. Posons $f(x,y)=g(u,v)$ où $u = x+y$ et $v = x+2y$. L'application $(x,y)\mapsto(x+y,x+2y)=(u,v)$ est un automorphisme de $\Rr^2$ et en particulier un $C^1$-difféormorphisme de $\Rr^2$ sur lui-même.

\begin{center}
$ \frac{\partial f}{\partial x}= \frac{\partial}{\partial x}(g(u,v))= \frac{\partial u}{\partial x}
\times \frac{\partial g}{\partial u}+ \frac{\partial v}{\partial x}
\times \frac{\partial g}{\partial v}= \frac{\partial g}{\partial u}+ \frac{\partial g}{\partial v}$
\end{center}

De même, $ \frac{\partial f}{\partial y}= \frac{\partial g}{\partial u}+2 \frac{\partial g}{\partial v}$ et donc

\begin{center}
$2 \frac{\partial f}{\partial x}- \frac{\partial f}{\partial y}=2 \frac{\partial g}{\partial u}+2 \frac{\partial g}{\partial v}- \frac{\partial g}{\partial u}-2 \frac{\partial g}{\partial v}= \frac{\partial g}{\partial u}$.
\end{center}

Par suite, $2 \frac{\partial f}{\partial x}- \frac{\partial f}{\partial y}=0\Leftrightarrow \frac{\partial g}{\partial u}=0\Leftrightarrow\exists h\in C^1(\Rr,\Rr)/\;\forall(u,v)\in\Rr^2,\;g(u,v)=h(v)\Leftrightarrow\exists h\in C^1(\Rr,\Rr)/\;\forall(x,y)\in\Rr^2,\;f(x,y)=h(x+2y)$.

\begin{center}
\shadowbox{
Les solutions sont les $(x,y)\mapsto h(x+2y)$ où $h\in C^1(\Rr,\Rr)$.
}
\end{center}

Par exemple, la fonction $(x,y)\mapsto\cos\sqrt{(x+2y)^2+1}$ est solution.}
  \item \reponse{Soit $f$ une application de classe $C^1$ sur $\Rr^2\setminus\{(0,0)\}$. Posons $f(x,y)=g(r,\theta)$ où $x=r\cos\theta$ et $y=r\sin\theta$. L'application $(r,\theta)\mapsto(r\cos\theta,r\sin\theta)=(x,y)$ est un $C^1$-difféormorphisme de $]0,+\infty[\times[0,2\pi[$ sur $\Rr^2\setminus\{(0,0)\}$. De plus,

\begin{center}
$ \frac{\partial g}{\partial r}= \frac{\partial}{\partial r}(f(x,y))= \frac{\partial x}{\partial r} \frac{\partial f}{\partial x}+ \frac{\partial y}{\partial r} \frac{\partial f}{\partial y}=\cos\theta \frac{\partial f}{\partial x}+\sin\theta \frac{\partial f}{\partial y}$,
\end{center}

et

\begin{center}
$ \frac{\partial g}{\partial \theta}= \frac{\partial}{\partial \theta}(f(x,y))= \frac{\partial x}{\partial \theta} \frac{\partial f}{\partial x}+ \frac{\partial y}{\partial \theta} \frac{\partial f}{\partial y}=-r\sin\theta \frac{\partial f}{\partial x}+r\cos\theta \frac{\partial f}{\partial y}=x \frac{\partial f}{\partial y}-y \frac{\partial f}{\partial x}$.
\end{center}

Donc

\begin{align*}\ensuremath
 \frac{\partial f}{\partial y}-y \frac{\partial f}{\partial x}=0&\Leftrightarrow \frac{\partial g}{\partial \theta}=0\Leftrightarrow\exists h_1\in C^1(]0,+\infty[,\Rr)/\;\forall(r,\theta)\in]0,+\infty[\times[0,2\pi[,\;g(r,\theta)=h_1(r)\\
 &\Leftrightarrow\exists h_1\in C^1(]0,+\infty[,\Rr)/\;\forall(x,y)\in\Rr^2\setminus\{(0,0)\},\;f(x,y)=h_1\left(\sqrt{x^2+y^2}\right)\\
 &\Leftrightarrow\exists h\in C^1(]0,+\infty[,\Rr)/\;\forall(x,y)\in\Rr^2\setminus\{(0,0)\},\;f(x,y)=h(x^2+y^2).
\end{align*}

\begin{center}
\shadowbox{
Les solutions sont les $(x,y)\mapsto h(x^2+y^2)$ où $h\in C^1(]0,+\infty[,\Rr)$.
}
\end{center}}
  \item \reponse{Soit $f$ une fonction de classe $C^2$ sur $]0,+\infty[\times\Rr$. D'après le théorème de \textsc{Schwarz}, $ \frac{\partial^2f}{\partial x\partial y}= \frac{\partial^2f}{\partial y\partial x}$.

Soit $\begin{array}[t]{cccc}
\varphi~:&]0,+\infty[\times\Rr&\rightarrow&]0,+\infty[\times\Rr\\
 &(u,v)&\mapsto&(u,uv)=(x,y)
\end{array}
$. Donc si on pose $f(x,y)=g(u,v)$, on a $g=f\circ\varphi$.

Soit $(x,y,u,v)\in]0,+\infty[\times\Rr\times]0,+\infty[\times\Rr$.

\begin{center}
$\varphi(u,v)=(x,y)\Leftrightarrow\left\{
\begin{array}{l}
u=x\\
uv=y
\end{array}
\right.\left\{
\begin{array}{l}
u=x\\
v= \frac{y}{x}
\end{array}
\right.$.
\end{center}

Ainsi, $\varphi$ est une bijection de $]0,+\infty[$ sur lui-même et sa réciproque est l'application 

\begin{center}
$\begin{array}[t]{cccc}
\varphi^{-1}~:&]0,+\infty[\times\Rr&\rightarrow&]0,+\infty[\times\Rr\\
 &(x,y)&\mapsto&\left(x, \frac{y}{x}\right)=(u,v)
\end{array}
$.
\end{center}

De plus, $\varphi$ est de classe $C^2$ sur $]0,+\infty[\times\Rr$ et son jacobien

\begin{center}
$J_\varphi(u,v)=\left|
\begin{array}{cc}
1&0\\
v&u
\end{array}
\right|=u$
\end{center}

ne s'annule pas sur $]0,+\infty[\times\Rr$. On sait alors que $\varphi$ est un $C^2$-difféomorphisme de $]0,+\infty[\times\Rr$ sur lui-même.

Puisque $g=f\circ\varphi$ et que $\varphi$ est un $C^2$-difféomorphisme de $]0,+\infty[\times\Rr$ sur lui-même, $f$ est de classe $C^2$ sur $]0,+\infty[\times\Rr$ si et seulement si $g$ est de classe $C^2$ sur $]0,+\infty[\times\Rr$.

\textbullet~$ \frac{\partial f}{\partial x}= \frac{\partial u}{\partial x} \frac{\partial g}{\partial u}+ \frac{\partial v}{\partial x} \frac{\partial g}{\partial v}= \frac{\partial g}{\partial u}- \frac{y}{x^2} \frac{\partial g}{\partial v}$.

\textbullet~$ \frac{\partial f}{\partial y}= \frac{\partial u}{\partial y} \frac{\partial g}{\partial u}+ \frac{\partial v}{\partial y} \frac{\partial g}{\partial v}= \frac{1}{x} \frac{\partial g}{\partial v}$.

\textbullet~$ \frac{\partial^2f}{\partial x^2}= \frac{\partial}{\partial x}\left( \frac{\partial g}{\partial u}- \frac{y}{x^2} \frac{\partial g}{\partial v}\right)=\left( \frac{\partial^2g}{\partial u^2}- \frac{y}{x^2} \frac{\partial^2g}{\partial u\partial v}\right)+\left( \frac{2y}{x^3} \frac{\partial g}{\partial v}- \frac{y}{x^2} \frac{\partial^2g}{\partial u\partial v}+ \frac{y^2}{x^4} \frac{\partial^2g}{\partial v^2}\right)= \frac{\partial^2g}{\partial u^2}- \frac{2y}{x^2} \frac{\partial^2g}{\partial u\partial v}+ \frac{y^2}{x^4} \frac{\partial^2g}{\partial v^2}+ \frac{2y}{x^3} \frac{\partial g}{\partial v}$.

\textbullet~$ \frac{\partial^2f}{\partial x\partial y}= \frac{\partial}{\partial y}\left( \frac{\partial g}{\partial u}- \frac{y}{x^2} \frac{\partial g}{\partial v}\right)= \frac{1}{x} \frac{\partial^2g}{\partial v\partial  v}- \frac{1}{x^2} \frac{\partial g}{\partial v}- \frac{y}{x^3} \frac{\partial^2g}{\partial v^2}$.

\textbullet~$ \frac{\partial^2f}{\partial y^2}= \frac{\partial}{\partial y}\left( \frac{1}{x} \frac{\partial g}{\partial v}\right)= \frac{1}{x^2} \frac{\partial^2g}{\partial v^2}$.

Ensuite,

\begin{align*}\ensuremath
x^2 \frac{\partial^2f}{\partial x^2}+2xy \frac{\partial^2f}{\partial x\partial y}+y^2 \frac{\partial^2f}{\partial y^2}&=x^2 \frac{\partial^2g}{\partial u^2}-2y \frac{\partial^2g}{\partial u\partial v}+ \frac{y^2}{x^2} \frac{\partial^2g}{\partial v^2}+ \frac{2y}{x} \frac{\partial g}{\partial v}+2y \frac{\partial^2g}{\partial v\partial  v}- \frac{2y}{x} \frac{\partial g}{\partial v}- \frac{2y^2}{x^2} \frac{\partial^2g}{\partial v^2}+ \frac{y^2}{x^2} \frac{\partial^2g}{\partial v^2}\\
 &=x^2 \frac{\partial^2g}{\partial u^2}.
\end{align*}

Ainsi,

\begin{align*}\ensuremath
\forall(x,y)\in]0,+\infty[\times\Rr,&\;x^2 \frac{\partial^2f}{\partial x^2}(x,y)+2xy \frac{\partial^2f}{\partial x\partial y}(x,y)+y^2 \frac{\partial^2f}{\partial y^2}(x,y)=0\Leftrightarrow\forall(u,v)\in]0,+\infty[\times\Rr,\; \frac{\partial^2g}{\partial u^2}(u,v)=0\\
 &\Leftrightarrow\exists h\in C^2(\Rr,\Rr)/\;\forall(u,v)\in]0,+\infty[\times\Rr,\; \frac{\partial g}{\partial u}(u,v)=h(v)\\
  &\exists(h,k)\in(C^2(\Rr,\Rr))^2/\;\forall(u,v)\in]0,+\infty[\times\Rr,\;g(u,v)=uh(v)+k(v)\\
  &\exists(h,k)\in(C^2(\Rr,\Rr))^2/\;\forall(x,y)\in]0,+\infty[\times\Rr,\;f(x,y)=xh(xy)+k(xy).
\end{align*}

Les fonctions solutions sont les $(x,y)\mapsto xh(xy)+k(xy)$ où $h$ et $k$ sont deux fonctions de classe $C^2$ sur $\Rr$.}
\end{enumerate}
}