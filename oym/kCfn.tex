\uuid{kCfn}
\exo7id{3480}
\titre{Matrices antisymétriques}
\theme{Exercices de Michel Quercia, Rang de matrices}
\auteur{quercia}
\date{2010/03/10}
\organisation{exo7}
\contenu{
  \texte{Soit $A=(a_{ij}) \in \mathcal{M}_n(K)$ antisymétrique.}
\begin{enumerate}
  \item \question{On suppose $a_{12} \ne 0$, et on décompose $A$ sous la forme :
    $A = \begin{pmatrix} J &U \cr {-{}^tU} &V\cr\end{pmatrix}$
    avec $J = \begin{pmatrix} 0 &a_{12} \cr -a_{12} &0 \cr\end{pmatrix}$.\par
    Soit $P = \begin{pmatrix} I_2 &-J^{-1}U \cr 0 &I_{n-2}\cr\end{pmatrix}$.
   \begin{enumerate}}
  \item \question{Montrer que $P$ existe et est inversible.}
  \item \question{Calculer $AP$.}
  \item \question{En déduire que $\mathrm{rg}(A) = 2 + \mathrm{rg}({}^tUJ^{-1}U + V)$.}
\end{enumerate}
\begin{enumerate}

\end{enumerate}
}