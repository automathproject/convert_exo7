\uuid{5061}
\titre{Chimie P' 91}
\theme{Exercices de Michel Quercia, Surfaces paramétrées}
\auteur{quercia}
\date{2010/03/17}
\organisation{exo7}
\contenu{
  \texte{}
  \question{On considère la droite $\Delta$ d'équations : $x=a$, $z=0$.
    $P$ est un point décrivant $\Delta$ et $\mathcal{C}_P$ le cercle tangent à $Oz$ en
    $O$ et passant par $P$.
    Faire un schéma et paramétrer la surface engendrée par les cercles $\mathcal{C}_P$
    quand $P$ décrit $\Delta$.}
  \reponse{$x=\frac a2(1+\cos u)$,
	     $y=\frac v2(1+\cos u)$,
	     $z=\frac{\sqrt{a^2+v^2}}2\sin u$.}
}