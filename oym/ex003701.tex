\exo7id{3701}
\titre{Expression d'une rotation}
\theme{}
\auteur{quercia}
\date{2010/03/11}
\organisation{exo7}
\contenu{
  \texte{Soit $E$ un espace vectoriel euclidien orienté de dimension 3, $\vec u \in E$ unitaire,
$\alpha \in \R$ et $f$ la rotation autour de $\vec u$ d'angle de mesure $\alpha$.}
\begin{enumerate}
  \item \question{Exprimer $f(\vec x)$ en fonction de $\vec u$, $\vec x$ et $\alpha$.}
  \item \question{On donne les coordonnées de $\vec u$ dans une base orthonormée : $a,b,c$.
    Calculer la matrice de $f$ dans cette base.}
\end{enumerate}
\begin{enumerate}
  \item \reponse{$f(\vec x) = (\vec x|\vec u)\vec u
                        + \cos\alpha(\vec u\wedge\vec x)\wedge\vec u
                        + \sin\alpha(\vec u\wedge\vec x)$.}
  \item \reponse{$M = (\cos\alpha)I
                + (1-\cos\alpha)\begin{pmatrix}a^2 &ab &ac \cr
                                         ab &b^2 &bc \cr
                                         ac &bc &c^2 \cr \end{pmatrix}
                + \sin\alpha\begin{pmatrix}0  &-c &b \cr
                                     c  &0  &-a \cr
                                     -b &a  &0 \cr\end{pmatrix}$.}
\end{enumerate}
}