\uuid{2588}
\titre{Exercice 2588}
\theme{Sujets de l'année 2005-2006, Rattrapage}
\auteur{delaunay}
\date{2009/05/19}
\organisation{exo7}
\contenu{
  \texte{}
  \question{Soit $N$ une matrice nilpotente, il existe $q\in\N$ tel que $N^q=0$. Montrer que la matrice $I-N$ est inversible et exprimer son inverse en fonction de $N$.}
  \reponse{{\it Soit $N$ une matrice nilpotente, il existe $q\in\N$ tel que $N^q=0$. Montrons que la matrice $I-N$ est inversible et exprimons son inverse en fonction de $N$.}

On remarque que 
$(I-N)(I+N+N^2+\dots+N^{q-1})=I-N^q=I.$
Ainsi, la matrice $I-N$ est inversible, et son inverse est 
$(I-N)^{-1}=I+N+N^2+\dots+N^{q-1}.$}
}