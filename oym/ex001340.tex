\uuid{1340}
\titre{Exercice 1340}
\theme{}
\auteur{ortiz}
\date{1999/04/01}
\organisation{exo7}
\contenu{
  \texte{}
  \question{Soient  $G$ un groupe et $x\in G$ un \'el\'ement
d'ordre $n.$ Quel est l'ordre de $x^2$ ?}
  \reponse{Rappelons d'abord que pour $x$ un élément d'ordre $n$, alors
$$x^q=e \Longrightarrow n | q.$$
\begin{itemize}
  \item Si $n$ est pair alors $ord(x^2)=n/2$ : en effet
$(x^2)^\frac{n}{2} = x^n=e$ et pour $p \geq 1$ tel que $(x^2)^p =
e$ alors $x^{2p}= e$ et $n|2p$ donc $p \geq \frac n2$. Donc $n/2$
est le plus petit des entiers $q$ (non nul) tel que $x^q=e$ et par
conséquent $n/2$ est l'ordre de $x$.
  \item Si $n$ est impair alors $ord(x)=n$. Tout d'abord
$(x^2)^n=(x^n)^2=e$ et pour $p$ tel que $(x^2)^p=e$ alors $n|2p$
mais $2$ et $n$ sont premiers entre eux donc d'après le théorème
de Gauss, $n|p$ et en particulier $p \geq n$.
\end{itemize}}
}