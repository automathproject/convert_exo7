\uuid{FzCK}
\exo7id{290}
\titre{Exercice 290}
\theme{Arithmétique dans $\Zz$, pgcd, ppcm, algorithme d'Euclide}
\auteur{bodin}
\date{1998/09/01}
\organisation{exo7}
\contenu{
  \texte{Calculer le pgcd des nombres suivants :}
\begin{enumerate}
  \item \question{126, 230.}
  \item \question{390, 720, 450.}
  \item \question{180, 606, 750.}
\end{enumerate}
\begin{enumerate}
  \item \reponse{$126 = 2.3^2.7$ et $230 = 2.5.23$ donc le pgcd de $126$ et $230$ est $2$.}
  \item \reponse{$390 = 2.3.5.13$, $720 = 2^4.3^2.5$, $450 = 2.3^2.5^2$ et donc le pgcd de ces trois nombres est
$2.3.5=30$.}
  \item \reponse{$\pgcd(180,606,750) = 6$.}
\end{enumerate}
}