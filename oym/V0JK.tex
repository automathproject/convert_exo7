\uuid{V0JK}
\exo7id{3013}
\titre{Radical d'un id{\'e}al}
\theme{Exercices de Michel Quercia, Anneaux}
\auteur{quercia}
\date{2010/03/08}
\organisation{exo7}
\contenu{
  \texte{Soit $A$ un anneau commutatif et $I$ un id{\'e}al de $A$.

On note $\sqrt{I} = \{ x \in A \text{ tq } \exists\ n \in \N \text{ tq } x^n \in I \}$
(radical de $I$).}
\begin{enumerate}
  \item \question{Montrer que $\sqrt{I}$ est un id{\'e}al de $A$.}
  \item \question{Montrer que $\sqrt{\sqrt I} = \sqrt I$.}
  \item \question{Montrer que $\sqrt{I \cap J} = \sqrt I \cap \sqrt J$ et
    $\sqrt{I + J} \supset \sqrt I + \sqrt J$.}
  \item \question{Exemple : $A = \Z$, $I = 3648\Z$. Trouver $\sqrt I$.}
\end{enumerate}
\begin{enumerate}
  \item \reponse{Remarque : la r{\'e}ciproque fausse : $A = \Z[X]$, $I=(X)$, $J=(X+4)$.}
  \item \reponse{$114\Z$.}
\end{enumerate}
}