\uuid{7260}
\titre{Exercice 7260}
\theme{}
\auteur{mourougane}
\date{2021/08/10}
\organisation{exo7}
\contenu{
  \texte{Soit $E$ un plan euclidien orienté, muni d'un repère $(O,\vec{\imath},\vec{\jmath})$ orthonormé direct.}
\begin{enumerate}
  \item \question{Soit $n$ un entier naturel supérieur à $3$.
Exprimer à l'aide des fonctions trigonométriques $\cos$ et $\sin$,
 le périmètre $p_n$ d'un polygone régulier à $n$ côtés inscrit dans le cercle trigonométrique
(c'est à dire le cercle de centre $0$ et de rayon $1$.)}
  \item \question{On rappelle que pour tout $\theta\in ]0,\pi/2[$, $$\theta\cos\theta\leq \sin\theta\leq \theta.$$
Montrer que la suite $(p_n)_{n\in\mathbb{N}}$ admet une limite et déterminer cette limite.}
\end{enumerate}
\begin{enumerate}

\end{enumerate}
}