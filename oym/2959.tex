\uuid{2959}
\titre{$z = (1+ia)/(1-ia)$}
\theme{Exercices de Michel Quercia, Nombres complexes}
\auteur{quercia}
\date{2010/03/08}
\organisation{exo7}
\contenu{
  \texte{}
  \question{Soit $z \in \mathbb{U}$. Peut-on trouver $a \in \R$ tel que $z = \frac {1+ia}{1-ia}$ ?}
  \reponse{$z = e^{i\theta}  \Rightarrow  a = \tan\frac\theta2$
         pour $\theta \not \equiv \pi (\mathrm{mod}\,{2\pi})$.}
}