\uuid{sg5x}
\exo7id{3925}
\titre{\'Equations aux arctan}
\theme{Exercices de Michel Quercia, Fonctions circulaires inverses}
\auteur{quercia}
\date{2010/03/11}
\organisation{exo7}
\contenu{
  \texte{Résoudre :}
\begin{enumerate}
  \item \question{$\arctan 2x + \arctan 3x = \frac \pi4$.}
  \item \question{$\arctan\left(\frac{x-1}{x-2}\right) + \arctan\left(\frac{x+1}{x+2}\right) = \frac\pi4$.}
  \item \question{$\arctan\left(\frac1x\right) + \arctan\left(\frac{x-1}{x+1}\right) = \frac\pi4$.}
  \item \question{$\arctan(x-3) + \arctan(x) + \arctan(x+3) = \frac {5\pi}4$.}
\end{enumerate}
\begin{enumerate}
  \item \reponse{$x = \frac 16$.}
  \item \reponse{$x = \pm1{\sqrt 2}$.}
  \item \reponse{$x \in {]-\infty,-1[} \cup ]0,+\infty[$.}
  \item \reponse{$x^3 - 3x^2 - 12x + 10 = 0  \Rightarrow  x = 5,\,-1\pm\sqrt3$.
Seule la solution $x=5$ convient.}
\end{enumerate}
}