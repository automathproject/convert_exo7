\exo7id{4872}
\titre{Expressions analytiques}
\theme{}
\auteur{quercia}
\date{2010/03/17}
\organisation{exo7}
\contenu{
  \texte{On fixe un repère ${\cal R} = (O,\vec e_1, \vec e_2, \vec e_3)$ d'un espace
affine de dimension 3.
Déterminer les expressions analytiques des applications suivantes :}
\begin{enumerate}
  \item \question{Symétrie de base le plan d'équation $x+2y+z = 1$ et de direction
    vect$(\vec e_1 + \vec e_2 + \vec e_3)$.}
  \item \question{Symétrie de base la droite d'équations
    $\begin{cases} x+y+1 = 0 \cr 2y+z+2 = 0,\end{cases}$ \par
    de direction le plan vectoriel
    d'équation $3x + 3y - 2z = 0$.}
\end{enumerate}
\begin{enumerate}
  \item \reponse{$\begin{cases}2x' = x - 2y - z + 1 \cr
                     2y' = -x -z + 1      \cr
                     2z' = -x -2y +z + 1  \cr\end{cases}$}
  \item \reponse{$\begin{cases}2x' = -5x - 3y + 2z - 3 \cr
                     2y' = 3x + y - 2z - 1   \cr
                     z' = -3x -3y +z -3      \cr\end{cases}$}
\end{enumerate}
}