\uuid{thKp}
\exo7id{7322}
\titre{Exercice 7322}
\theme{Exercices de Christophe Mourougane, Arithmétique 2}
\auteur{mourougane}
\date{2021/08/10}
\organisation{exo7}
\contenu{
  \texte{Soit $P = a_0 + a_1X + \cdots + a_nX^n \in \Z[X]$. On veut déterminer si $P$ a des racines rationnelles.}
\begin{enumerate}
  \item \question{On suppose que $P$ a une racine rationnelle non nulle $x$, avec $x = \frac{p}{q}$ et $\pgcd(p,q) = 1$. Montrer que $p$ divise $a_0$ et $q$ divise $a_n$.}
  \item \question{Le polynôme $7X^3 - 5X^2 -9X +4$ a-t-il des racines rationnelles ? et $X^4 - 2X^2 - 3$ ?}
  \item \question{Soit $n \in \N^*$. Montrer que $\sqrt{n}$ est soit un entier, soit un irrationnel.}
\end{enumerate}
\begin{enumerate}

\end{enumerate}
}