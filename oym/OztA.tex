\uuid{OztA}
\exo7id{4797}
\titre{Thm du point fixe}
\theme{Exercices de Michel Quercia, Topologie dans les espaces vectoriels normés}
\auteur{quercia}
\date{2010/03/16}
\organisation{exo7}
\contenu{
  \texte{Soit $E$ un evn de dimension finie et $f : E \to E$ une fonction
$k$-lipchitzienne avec $k < 1$.
On choisit ${\vec u}_0 \in E$ arbitrairement, et on consid{\`e}re la suite
$({\vec u}_n)$ telle que pour tout $n$ :
${\vec u}_{n+1} = f({\vec u}_n)$.}
\begin{enumerate}
  \item \question{Montrer que $\|{\vec u}_{n+1} - {\vec u}_n\| \le k^n \|{\vec u}_1 - {\vec u}_0\|$.}
  \item \question{En d{\'e}duire que la suite $({\vec u}_n)$ est de Cauchy.}
  \item \question{Soit $\vec \ell = \lim({\vec u}_n)$. Montrer que $\vec \ell$ est l'unique solution
    dans $E$ de l'{\'e}quation $f(\vec x\,) = \vec x$.}
\end{enumerate}
\begin{enumerate}

\end{enumerate}
}