\uuid{9ZIA}
\exo7id{4558}
\titre{Fonction définie par une série}
\theme{Exercices de Michel Quercia, Suites et séries de fonctions}
\auteur{quercia}
\date{2010/03/14}
\organisation{exo7}
\contenu{
  \texte{On pose pour $x\in\R$~: $f(x) = \sum_{n=1}^\infty \frac{(-1)^{n-1}}{\sqrt{n^2+x^2}}$.}
\begin{enumerate}
  \item \question{Déterminer $\lim_{x\to\infty}f(x)$.}
  \item \question{Chercher un équivalent de~$f(x)$ en $+\infty$.}
\end{enumerate}
\begin{enumerate}
  \item \reponse{CSA~: $0\le f(x)\le \frac1{\sqrt{1+x^2}}$ donc $f(x)\to 0$ lorsque $x\to+\infty$.}
  \item \reponse{{$xf(x)$}
$= \sum_{p=0}^\infty \frac{x}{\sqrt{(2p+1)^2+x^2}} - \frac{x}{\sqrt{(2p+2)^2+x^2}}$\par
$= \sum_{p=0}^\infty \int_{t=2p+1}^{2p+2}\frac{xt}{(t^2+x^2)^{3/2}}\,d t$\par
$= \sum_{p=0}^\infty \int_{u=(2p+1)/x}^{(2p+2)/x}\frac{u}{(u^2+1)^{3/2}}\,d u$.\par


On a $ \int_{u=0}^\infty\frac u{(u^2+1)^{3/2}}\,d u = 1 = a + b$ avec~:

$a = \sum_{p=0}^\infty \int_{u=(2p)/x}^{(2p+1)/x}\frac{u}{(u^2+1)^{3/2}}\,d u$ et
$b = \sum_{p=0}^\infty \int_{u=(2p+1)/x}^{(2p+2)/x}\frac{u}{(u^2+1)^{3/2}}\,d u = xf(x)$.

$h$ : $u \mapsto\frac{u}{(u^2+1)^{3/2}}$ est croissante sur~$[0,\sqrt{\frac12}]$ et
décroissante sur $[\sqrt{\frac12},+\infty[$ donc $|a-b| \le \frac{3\|h\|_\infty}x$,
et $xf(x)\to \frac12$ lorsque $x\to+\infty$.}
\end{enumerate}
}