\uuid{4312}
\titre{X MP$^*$ 2000}
\theme{Exercices de Michel Quercia, Intégrale généralisée}
\auteur{quercia}
\date{2010/03/12}
\organisation{exo7}
\contenu{
  \texte{}
  \question{Donnez un équivalent pour $x\to+\infty$ de $ \int_{t=0}^x \Bigl|\frac{\sin t}{t}\Bigr|\,d t$.}
  \reponse{On pose $u_n= \int_{t=n\pi}^{(n+1)\pi} \Bigl|\frac{\sin t}{t}\Bigr|\,d t
= \int_{t=0}^{\pi} \frac{\sin t}{t+n\pi}\,d t$. Par encadrement du dénominateur
on a $u_n\sim \frac2{n\pi}$, d'où
$u_0+\dots+u_n\sim \frac{2\ln n}\pi$ et, par encadrement encore, l'intégrale
arrêtée en $x$ est équivalente à $\frac{2\ln x}\pi$.}
}