\uuid{1323}
\titre{Exercice 1323}
\theme{}
\auteur{legall}
\date{1998/09/01}
\organisation{exo7}
\contenu{
  \texte{}
  \question{Soit $  G  $ un groupe, $  H  $ et $  K  $ deux
sous-groupes de $ G  .$ Montrer que $  H\cup K  $ est un
sous-groupe de $  G  $ si et seulement si $  H\subset K  $ ou $
K\subset H  .$}
  \reponse{\begin{itemize}
    \item[$\bullet$] Si $H \subset K$ alors $H\cup K = K$, qui est un sous-groupe
de H. M\^eme chose si $K \subset H$.
    \item[$\bullet$] R\'eciproquement, supposons que
$H\cup K$ est un sous-groupe de $G$. Par l'absurde supposons que
$H \not\subset K$ et $K \not\subset H$. Alors il existe $x\in
H\setminus K$ et $y \in K\setminus H$. Comme $x,y \in H\cup K$ et
que $H\cup K$ est un groupe alors $x.y \in H\cup K$. Donc $x.y \in
H$ ou $x.y \in K$. Par exemple supposons $x.y \in H$ alors comme
$x\in H$, $x^{-1}\in H$ et donc comme $H$ est un groupe
$x^{-1}.x.y \in H$ et donc $y\in H$. Ce qui est en contradiction
avec l'hypoth\`ese $ y \in K\setminus H$. En conclusion, parmi les
sous-groupes $H,K$ l'un est inclus dans l'autre.
\end{itemize}}
}