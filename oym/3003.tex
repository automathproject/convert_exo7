\uuid{3003}
\titre{Partie g{\'e}n{\'e}ratrice d'un groupe fini}
\theme{Exercices de Michel Quercia, Groupes}
\auteur{quercia}
\date{2010/03/08}
\organisation{exo7}
\contenu{
  \texte{}
  \question{Soit $G$ un groupe fini de cardinal~$n$. Montrer qu'il existe une
partie g{\'e}n{\'e}ratrice de~$G$ de cardinal inf{\'e}rieur ou {\'e}gal {\`a}~$\log_2(n)$.}
  \reponse{Soir $\{e_1,\dots,e_p\}$ une partie g{\'e}n{\'e}ratrice de cardinal
minimal. Alors les $2^p$ {\'e}l{\'e}ments $e_1^{\alpha_1}\dots e_p^{\alpha_p}$
avec $\alpha_i\in\{0,1\}$ sont distincts (sinon un des $e_i$ appartient
au groupe engendr{\'e} par les autres) donc $n\ge 2^p$.}
}