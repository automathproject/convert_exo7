\uuid{6248}
\titre{Exercice 6248}
\theme{}
\auteur{queffelec}
\date{2011/10/16}
\organisation{exo7}
\contenu{
  \texte{Soit $X$ un espace métrique.}
\begin{enumerate}
  \item \question{Montrer que si $X$ n'est pas complet, 
il existe une suite de Cauchy $(a_n)$,
non convergente, et telle que $a_p\neq a_q$ pour $p\neq q$.}
  \item \question{Soit $(b_n)$ une suite de Cauchy non convergente; montrer que l'ensemble
$B=\{b_n,\ n\in\Nn\}$ est fermé dans $X$.}
  \item \question{Déduire des questions précédentes que si $X$ n'est pas complet, on peut
trouver une fonction continue $f:X\to [0,1]$ qui n'est pas uniformément
continue.

\emph{Indication :} Si $(a_n)$ est définie par 1., construire $f:X\to [0,1]$ telle que
$f(a_{2n})=0$ et $f(a_{2n+1})=1$.}
\end{enumerate}
\begin{enumerate}

\end{enumerate}
}