\uuid{806}
\titre{Exercice 806}
\theme{}
\auteur{cousquer}
\date{2003/10/01}
\organisation{exo7}
\contenu{
  \texte{}
  \question{Représenter la courbe définie par son équation polaire
$\rho=a\sin^3\frac{\theta}{3}$. Calculer sa longueur $L$ et les aires
$A_1$ et $A_2$ limitées par les deux boucles qu'elle forme.}
  \reponse{$\displaystyle L=\frac{3\pi a}{2},\quad
A_1=\frac{5\pi-9\sqrt3}{32}a^2,\quad
A_2=\frac{5\pi+18\sqrt3}{32}a^2$.}
}