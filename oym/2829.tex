\uuid{2829}
\titre{Exercice 2829}
\theme{Prolongement analytique et résidus, Principe du maximum}
\auteur{burnol}
\date{2009/12/15}
\organisation{exo7}
\contenu{
  \texte{}
  \question{Montrer que si une fonction entière $f$ a sa partie réelle
bornée supérieurement  alors elle est constante (considérer
$\exp(f)$).}
  \reponse{Soit $g(z) = e^{f(z)}$. On a $|g(z)| =e^{\Re (f(z))}$. Par cons\'equent $g$ est une fonction enti\`ere born\'ee. Elle est donc constante par d'Alembert-Liouville.}
}