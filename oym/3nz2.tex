\uuid{3nz2}
\exo7id{4497}
\titre{Centrale MP 2002}
\theme{Exercices de Michel Quercia, Familles sommables}
\auteur{quercia}
\date{2010/03/14}
\organisation{exo7}
\contenu{
  \texte{Soient $a,b,c\in\N^*$. On pose $f(z) = \sum_{n=0}^\infty \frac{z^{bn}}{1-z^{an+c}}$.}
\begin{enumerate}
  \item \question{Étudier la convergence de la série et montrer qu'on peut intervertir
    $b$ et $c$ dans la formule.}
  \item \question{Développer en série entière~: $\sum_{m=1}^\infty\frac{z^{2m}}{1-z^m}$.}
\end{enumerate}
\begin{enumerate}
  \item \reponse{Il y a convergence si $|z|<1$.
On a alors $f(z) = \sum_{(n,p)\in\N^2}z^{anp + bn+cp}$.
Il y a aussi convergence pour $|z|>1$ lorsque $a>b$ et on a dans ce cas~:
$f(z) = \sum_{(n,p)\in\N\times\N^*}z^{-anp + bn - cp}$ (non symétrique en $b,c$).}
  \item \reponse{$f(z) = \sum_{n=2}^\infty d_nz^n$ avec $d_n =$ nombre de diviseurs de~$n$
dans~$[[1,n-1]]$.}
\end{enumerate}
}