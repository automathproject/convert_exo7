\uuid{VuYn}
\exo7id{4602}
\titre{Fonction génératrice du nombre de partitions}
\theme{Exercices de Michel Quercia, Séries entières}
\auteur{quercia}
\date{2010/03/14}
\organisation{exo7}
\contenu{
  \texte{On note $T_n$ le nombre de partitions d'un ensemble à $n$ éléments.}
\begin{enumerate}
  \item \question{Montrer que $T_{n+1} = \sum_{k=0}^n {\binom{n}{k}}T_k$.}
  \item \question{Montrer que $\sum_{n=0}^\infty \frac{T_nx^n}{n!} = e^{e^x-1}$.}
\end{enumerate}
\begin{enumerate}
  \item \reponse{$f'(x) = e^xf(x)$.}
\end{enumerate}
}