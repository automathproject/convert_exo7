\uuid{jkOI}
\exo7id{4629}
\titre{DSF de $f*g$, Mines PSI 1998}
\theme{Exercices de Michel Quercia, Séries de Fourier}
\auteur{quercia}
\date{2010/03/14}
\organisation{exo7}
\contenu{
  \texte{Soient ${f,g} : \R \to \C$ continues $2\pi$-périodiques. On pose pour $x \in \R$ :
$h(x) = \frac1{2\pi} \int_{t=0}^{2\pi} f(x-t)g(t)\,d t$.}
\begin{enumerate}
  \item \question{Montrer que $h$ est $2\pi$-périodique, continue, et calculer les coefficients
de Fourier exponentiels de $h$ en fonction de ceux de $f$ et de $g$.}
  \item \question{Pour $g$ fixée, déterminer les valeurs et vecteurs propres de
$f \mapsto h$.}
\end{enumerate}
\begin{enumerate}
  \item \reponse{$c_k(h) = c_k(f)c_k(g)$.}
\end{enumerate}
}