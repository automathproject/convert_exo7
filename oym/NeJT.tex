\uuid{NeJT}
\exo7id{4072}
\titre{$AP' - nA'P = \lambda P$}
\theme{Exercices de Michel Quercia, \'Equations différentielles linéaires (I)}
\auteur{quercia}
\date{2010/03/11}
\organisation{exo7}
\contenu{
  \texte{Soit $A$ un polynôme à coefficients réels de degré 2 donné.
Au polynôme $P$ de degré inférieur ou égal à $2n$ on fait correspondre le
polynôme $Q = AP' - nA'P$.}
\begin{enumerate}
  \item \question{Montrer qu'on définit ainsi un endomorphisme $\Phi$ de $\R_{2n}[X]$.}
  \item \question{Chercher les valeurs propres et les vecteurs propres de $\Phi$ dans les
    cas particuliers :\par
   \begin{enumerate}}
  \item \question{$A = X^2-1$,}
  \item \question{$A = X^2$,}
  \item \question{$A = X^2+1$.}
\end{enumerate}
\begin{enumerate}
  \item \reponse{$\lambda = 2k$, $P = \alpha(X-1)^{n-k}(X+1)^{n+k}$ pour $-n \le k \le n$.}
  \item \reponse{$\lambda = 0$, $P = \alpha X^{2n}$.}
  \item \reponse{$\lambda = 0$, $P = \alpha (X^2+1)^n$.}
\end{enumerate}
}