\exo7id{2671}
\titre{Exercice 2671}
\theme{}
\auteur{matexo1}
\date{2002/02/01}
\organisation{exo7}
\contenu{
  \texte{}
\begin{enumerate}
  \item \question{Sur l'espace $\R^n$, d\'efinir une distance $d$, telle que
$\forall x,y \in \R^n$, on a $d(x,y)\leq 1$. Est-il possible de
choisir $d$ de sorte que, en plus, $x\mapsto \sqrt{d(x,0)}$ soit
une norme\,?}
  \item \question{On consid\`ere l'ensemble $S$ constitu\'e des points de $\R^3$
de norme euclidienne \'egale \`a 1 (``sph\`ere unit\'e''). Si $x$, $y$,
sont deux \'el\'ements de $S$, il existe au moins un cercle de rayon 1
trac\'e sur $S$ et passant par ces points (dans quels cas en existe-t-il
plusieurs\,?)\,; on note $d(x,y)$ la longueur du plus court arc de ce
cercle joignant $x$ et $y$ (``distance g\'eod\'esique'' --- comparer
avec la distance sur la sph\`ere terrestre). Montrer que $d$ est une
distance sur $S$\,; pourquoi $x\mapsto \sqrt{d(x,0)}$ n'est-elle pas
une norme\,?}
\end{enumerate}
\begin{enumerate}

\end{enumerate}
}