\uuid{MsNL}
\exo7id{3769}
\titre{Ensi Physique 93}
\theme{Exercices de Michel Quercia, Endomorphismes auto-adjoints}
\auteur{quercia}
\date{2010/03/11}
\organisation{exo7}
\contenu{
  \texte{Soit $E = \R_n[x]$. On pose pour $P,Q \in E :
{<P,Q>} =  \int_{-1}^1 P(t)Q(t)\,d t$ et on considère
$$u : E \to  {\R[x]}, {P(x)}  \mapsto {2xP'(x) + (x^2-1)P''(x).}$$}
\begin{enumerate}
  \item \question{Montrer que l'on définit un produit scalaire et que $u$ est un endomorphisme.}
  \item \question{Montrer que $u$ est diagonalisable et que si $P_k,P_\ell$ sont des vecteurs propres
    de valeurs propres distinctes, alors ${{<}P_k,P_\ell{>=} 0}$.}
  \item \question{\'Eléments propres de $u$ pour $n=3$ ?}
\end{enumerate}
\begin{enumerate}
  \item \reponse{$u$ est autoadjoint pour $<,>$.}
  \item \reponse{$P_0 = 1$, $P_2 = x$, $P_6 = 3x^2-1$, $P_{12} = 5x^3 - 3x$.}
\end{enumerate}
}