\uuid{Gwp9}
\exo7id{2938}
\titre{Sph{\`e}re de $\R^3$}
\theme{Exercices de Michel Quercia, Nombres complexes}
\auteur{quercia}
\date{2010/03/08}
\organisation{exo7}
\contenu{
  \texte{Soient $u,v \in \C$ tels que $u+v \ne 0$.
On pose $x = \frac {1+uv}{u+v}$,
        $y = i\frac {1-uv}{u+v}$,
        $z = \frac {u-v}{u+v}$.}
\begin{enumerate}
  \item \question{CNS sur $u$ et $v$ pour que $x,y,z$ soient r{\'e}els ?}
  \item \question{On suppose cette condition r{\'e}alis{\'e}e. Montrer que le point $M(x,y,z)$ dans
    l'espace appartient {\`a} la sph{\`e}re de centre $O$ et de rayon 1.}
  \item \question{A-t-on ainsi tous les points de cette sph{\`e}re ?}
\end{enumerate}
\begin{enumerate}
  \item \reponse{$z\in\R \Leftrightarrow \exists\ \alpha\in\R$ tq $u=\alpha v$.\par
    $x,y\in\R \Leftrightarrow \alpha = \frac1{|v|^2} \Leftrightarrow u = \frac 1{\bar v}$.}
  \item \reponse{Il manque seulement les deux p{\^o}les.}
\end{enumerate}
}