\uuid{4039}
\titre{Développement asymptotique}
\theme{}
\auteur{quercia}
\date{2010/03/11}
\organisation{exo7}
\contenu{
  \texte{Soit $f : x  \mapsto \frac{x+1}x e^x$.}
\begin{enumerate}
  \item \question{Tracer la courbe $\mathcal{C}$ représentative de $f$.}
  \item \question{Soit $\lambda \in \R^+$. Si $\lambda$ est assez grand, la droite d'équation
    $y=\lambda$ coupe $\mathcal{C}$ en deux points d'abscisses $a < b$.
 \begin{enumerate}}
  \item \question{Montrer que $a \sim \frac1\lambda$, et
            $e^b \sim \lambda$ pour $\lambda \to +\infty$.}
  \item \question{Chercher la limite de $b^a$ quand $\lambda$ tend vers $+\infty$.}
\end{enumerate}
\begin{enumerate}
  \item \reponse{\begin{enumerate}}
  \item \reponse{$a \sim e^{-b}  \Rightarrow  a\ln b \to 0  \Rightarrow  b^a \to 1$.}
\end{enumerate}
}