\uuid{3787}
\titre{$f$ quelconque, il existe une BON dont l'image est orthogonale}
\theme{Exercices de Michel Quercia, Endomorphismes auto-adjoints}
\auteur{quercia}
\date{2010/03/11}
\organisation{exo7}
\contenu{
  \texte{}
  \question{Soit $f \in \mathcal{L}(E)$. Montrer qu'il existe une base orthonormée
$(\vec e_1,\dots,\vec e_n)$ dont l'image par $f$ est une famille orthogonale.}
  \reponse{Soit ${\cal B}$ une BON fixée, $M = \text{Mat}_{\cal B}(f)$,
         ${\cal B}'$ la BON cherchée et $P$ la matrice de passage de $\cal B$
         à ${\cal B}'$. On veut que ${}^tM'M'$ soit diagonale avec
         $M' = {}^tPMP$, cad ${}^tP^tMMP$ diagonale.}
}