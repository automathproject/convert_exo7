\uuid{1381}
\titre{Exercice 1381}
\theme{}
\auteur{ortiz}
\date{1999/04/01}
\organisation{exo7}
\contenu{
  \texte{Soit
$M=\left\{aI_2+bJ\in\mathcal{M}_2(\Rr):a,b\in\Rr\right\}$
o\`u $I_2=
\left (
\begin{array}{ll}
1&0\\
0&1
\end{array}
\right ),
J=
\left (
\begin{array}{ll}
0&2\\
1&0
\end{array}
\right )$.}
\begin{enumerate}
  \item \question{Calculer $J^2$ et montrer que si $a,b\in \Rr$ et $ aI_2+ bJ =O$ alors $a=b=0$.}
  \item \question{Montrer que, muni des lois usuelles sur $\mathcal{M}_2(\Rr)$,
$M$ est un anneau. Cet anneau est-il commutatif,
int\`egre ?}
  \item \question{$M$ est-il un corps, une $\Rr$-alg\`ebre ?}
\end{enumerate}
\begin{enumerate}

\end{enumerate}
}