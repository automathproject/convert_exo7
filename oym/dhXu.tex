\uuid{dhXu}
\exo7id{4697}
\titre{Radicaux it{\'e}r{\'e}s}
\theme{Exercices de Michel Quercia, Suites convergentes}
\auteur{quercia}
\date{2010/03/16}
\organisation{exo7}
\contenu{
  \texte{Soit $u_n = \sqrt{ n + \sqrt{ n-1 + \dots + \sqrt 1 }}$.}
\begin{enumerate}
  \item \question{Montrer que la suite $\left(\frac{u_n}{\sqrt n}\right)$ est born{\'e}e.}
  \item \question{D{\'e}terminer $\lim_{n\to\infty} \left(\frac{u_n}{\sqrt n}\right)$.}
  \item \question{D{\'e}terminer $\lim_{n\to\infty}(u_n-\sqrt n\,)$.}
\end{enumerate}
\begin{enumerate}
  \item \reponse{Par r{\'e}currence, $\sqrt n \le u_n \le \sqrt{2n}$.}
  \item \reponse{$\sqrt n \le u_n \le \sqrt{n+\sqrt{2(n-1)}}  \Rightarrow  \lim = 1$.}
  \item \reponse{$\sqrt{n+\sqrt{n-1}} \le u_n \le \sqrt{n+\sqrt{n-1+\sqrt{2(n-2)}}}
               \Rightarrow  \lim = \frac 12$.}
\end{enumerate}
}