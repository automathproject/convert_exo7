\uuid{7094}
\titre{Théorème de Pappus, version affine}
\theme{}
\auteur{megy}
\date{2017/01/21}
\organisation{exo7}
\contenu{
  \texte{}
  \question{% Source :  Audin par exemple
Soient $D$ et $D'$ deux droites. Soient $A, B, C$ trois points sur $D$, et $A'$, $B'$ et $C'$ trois points sur $D'$. Si $(AB') // (BC')$ et $(BA') // (CB')$, alors $(AA') // (CC')$.}
  \reponse{Soit $O$ le pt d'intersection. On note $\phi$ l'homothétie qui envoie $A$ sur $B$, et $\psi$ celle qui envoie $B$ sur $C$. Alors $\phi\psi = \psi\phi$. L'image de $A$ est $C$ et l'image de $A'$ est $C'$, d'où le  parallélisme demandé.

Si les droites sont parallèles, on remplace les homothéties par des translations.}
}