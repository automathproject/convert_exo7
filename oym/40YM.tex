\uuid{40YM}
\exo7id{4538}
\titre{Fonction définie par une série}
\theme{Exercices de Michel Quercia, Suites et séries de fonctions}
\auteur{quercia}
\date{2010/03/14}
\organisation{exo7}
\contenu{
  \texte{}
\begin{enumerate}
  \item \question{\'Etudier la convergence simple, uniforme, de $f(x) = \sum_{n=0}^\infty
    \bigl(\Arctan(x+n) - \Arctan(n)\bigr)$.}
  \item \question{Montrer que $f$ est de classe $\mathcal{C}^1$ sur $\R$.}
  \item \question{Chercher une relation simple entre $f(x)$ et $f(x+1)$.}
  \item \question{Trouver $\lim_{x\to+\infty} f(x)$.}
\end{enumerate}
\begin{enumerate}
  \item \reponse{CVU sur tout $[a,b]$.}
  \item \reponse{$f(x+1) = f(x) + \frac\pi2 - \Arctan x$.}
  \item \reponse{$f(x+1) - f(x) \sim \frac 1x$ donc la suite $(f(n))$ diverge et
    $f$ est croissante $ \Rightarrow  \lim = +\infty$.}
\end{enumerate}
}