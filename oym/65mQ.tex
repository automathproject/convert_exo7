\uuid{65mQ}
\exo7id{2289}
\titre{Exercice 2289}
\theme{Anneaux de polynômes II, anneaux quotients}
\auteur{barraud}
\date{2008/04/24}
\organisation{exo7}
\contenu{
  \texte{Soit $I$ et $J$ deux id\'eaux de l'anneau $A$.
Consid\'erons la projection canonique 
\newline
$\pi_I : A\to A/I$ 
et l'image $\bar J=\pi_I(J)$ de l'id\'eal $J$.}
\begin{enumerate}
  \item \question{Montrer que  $\bar J$ est un id\'eal de l'anneau quotient $A/I$.}
  \item \question{D\'emontrer qu'on a l'isomorphisme suivant
: $(A/I)/\bar J\cong A/(I+J)$.

({\it Indication :.} Consid\'erer le morphisme $a+I\mapsto a+(I+J)$
de l'anneau $A/I$ vers l'anneau $A/(I+J)$.)}
\end{enumerate}
\begin{enumerate}
  \item \reponse{Soit $\alpha,\beta\in\bar{J}$ et $\lambda,\mu\in A/I$. Alors
     $\exists a,b\in J$, $l,m\in A$,
     $\alpha=\pi(a),\beta=\pi(b),\lambda=\pi(l),\mu=\pi(m)$. On a donc
     $\lambda\alpha+\mu\beta=\pi(la+mb)$. Or $la+mb\in J$ (car $J$ est un
     idéal), donc $\lambda\alpha+\mu\beta\in\bar{J}$. Donc $\bar{J}$ est
     un idéal de $A/I$.}
  \item \reponse{Comme dans l'exercice \ref{ex:bar49}, on a le diagramme suivant~:
     $$    
     \xymatrix{%
       A        \ar[r]^{\pi_{1}}\ar[d]^{\pi}\ar@(ur,ul)[rr]^{\pi_{2}\circ\pi_{1}} & 
       A/I      \ar[r]^{\pi_{2}}&(A/I)/\bar{J} \\ 
       A/(I+J) \ar[urr]_{\sim}
      }%
      $$
      En effet, si $x\in\ker(\pi_{2}\circ\pi_{1})$, alors
      $\pi_{1}(x)\in\ker\pi_{2}=\bar{J}$, donc $\exists y\in A,
      \pi_{1}(x)=\pi_{1}(y)$. Alors $x-y\in\ker\pi_{1}=I$, donc $\exists
      z\in I, x=y+z$~: on a donc $x\in I+J$. Réciproquement, si $x\in
      I+J$, alors $\exists (x_{1},x_{2})\in I\times J, x=x_{1}+x_{2}$.
      Alors $\pi_{1}(x)=\pi_{1}(x_{2})\in\bar{J}$, donc
      $\pi_{2}\circ\pi_{1}(x)=0$.

      Donc $\ker(\pi_{2}\circ\pi_{1})=I+J$. Donc
      $A/(I+J)\sim(A/I)/\bar{J}$.}
\end{enumerate}
}