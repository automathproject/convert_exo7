\uuid{CkL1}
\exo7id{4530}
\titre{Fonction $\zeta$ de Riemann}
\theme{Exercices de Michel Quercia, Suites et séries de fonctions}
\auteur{quercia}
\date{2010/03/14}
\organisation{exo7}
\contenu{
  \texte{Soit $\zeta(x) = \sum_{n=1}^\infty \frac1{n^x}$.}
\begin{enumerate}
  \item \question{Déterminer le domaine de définition de $\zeta$.
    Montrer que $\zeta$ est de classe $\mathcal{C}^\infty$ sur ce domaine.}
  \item \question{Prouver que $\zeta(x) \to 1$ lorsque $x\to+\infty$ $\Bigl($majorer
    $\sum_{n=2}^\infty \frac1{n^x}$ par comparaison à une intégrale$\Bigr)$.}
  \item \question{Prouver que $\zeta(x) \to +\infty$ lorsque $x\to1^+$.}
\end{enumerate}
\begin{enumerate}

\end{enumerate}
}