\exo7id{1852}
\titre{Exercice 1852}
\theme{}
\auteur{maillot}
\date{2001/09/01}
\organisation{exo7}
\contenu{
  \texte{On cherche les fonctions $f:\R^2\rightarrow \R$ telles que:
\begin{equation}\label{truc}
\frac{\partial f}{\partial u} (u,v) +2u \frac{\partial f}{\partial v} (u,v)=0 \qquad \text{pour tout $(u,v)\in\R^2$}.
\end{equation}
Soit $\phi:\R^2\rightarrow\R^2$ l'application d\'efinie par $\phi(x,y)=(x,y+x^2)$.}
\begin{enumerate}
  \item \question{En calculant l'application r\'eciproque, montrer que $\phi$ est
bijective. V\'erifier que $\phi$ et $\phi^{-1}$ sont de classe $\mathcal C^1$.}
  \item \question{Soit $f:\R^2\rightarrow \R$ une fonction de classe $\mathcal C^1$. Posons
$g=f\circ \phi$.
  \begin{enumerate}}
  \item \question{Montrer que $g$ est de classe $\mathcal C^1$.}
  \item \question{Montrer que $f$ est solution de (\ref{truc}) si et seulement si
  $\frac{\partial g}{\partial x}=0$.}
\end{enumerate}
\begin{enumerate}

\end{enumerate}
}