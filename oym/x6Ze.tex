\uuid{x6Ze}
\exo7id{3996}
\titre{Constante d'Euler}
\theme{Exercices de Michel Quercia, Fonctions convexes}
\auteur{quercia}
\date{2010/03/11}
\organisation{exo7}
\contenu{
  \texte{Soit $f : {[0,+\infty[} \to {\R}$ concave, dérivable, croissante.}
\begin{enumerate}
  \item \question{Montrer que : $\forall\ x\ge 1,\ f(x+1)-f(x) \le f'(x) \le f(x)-f(x-1)$.}
  \item \question{On pose : $\begin{cases} u_n = f'(1) + f'(2) + \dots + f'(n) - f(n) \cr
                       v_n = f'(1) + f'(2) + \dots + f'(n) - f(n+1).\cr\end{cases}$
    Montrer que ces suites convergent.}
  \item \question{On prend $f(x) = \ln x$. Soit $\gamma = \lim_{n\to\infty} u_n$
    (constante d'Euler). Calculer $\gamma$ à $10^{-2}$ près.}
\end{enumerate}
\begin{enumerate}

\end{enumerate}
}