\uuid{7155}
\titre{Point intérieur de Vecten}
\theme{}
\auteur{megy}
\date{2017/05/13}
\organisation{exo7}
\contenu{
  \texte{% composition de similitudes
% hauteurs, orthocentre
Soit $ABC$ un triangle direct non rectangle isocèle, et soit $P$ (resp. $Q$, $R$) tel que $BCP$ (resp. $CAQ$, $ABR$) soit direct isocèle rectangle en $P$ (resp. $Q$ et $R$). 

\begin{center}
\includegraphics{../images/img007155-1}
\end{center}}
\begin{enumerate}
  \item \question{Montrer que $\overrightarrow{AP}$ et $\overrightarrow{QR}$ sont orthogonaux et de même norme.}
  \item \question{Montrer que les droites $(AP)$, $(BQ)$ et $(CE)$ sont concourantes.}
\end{enumerate}
\begin{enumerate}
  \item \reponse{Suivons l'indication. Soit $s = s_2\circ s_1$. C'est une similitude directe de rapport $1$ et d'angle $\pi/2$. On vérifie également que $s(C')=C'$, donc c'est la rotation de centre $C'$ et d'angle $\pi/2$. 

Or, d'une part on a  $s(R)=A$, et d'autre part $s_1(Q) = C$ et $s_2(C) = P$, donc $s(Q) = P$. On en déduit que $\overrightarrow{AP}$ est l'image de $\overrightarrow{RQ}$ par une rotation de $\pi/2$. Les vecteurs ont donc même norme et sont orthogonaux.}
  \item \reponse{Les droites $(AP)$, $(BQ)$ et $(CE)$ sont les hauteurs de $PQR$ donc sont concourantes en l'orthocentre de $PQR$. Ce point est le \emph{point intérieur de Vecten} de $ABC$.}
\end{enumerate}
}