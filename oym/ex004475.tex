\exo7id{4475}
\titre{$u_n/R_n^p$}
\theme{}
\auteur{quercia}
\date{2010/03/14}
\organisation{exo7}
\contenu{
  \texte{Soit $(a_n)$ une série positive convergente, $A = \sum_{k=0}^\infty a_k$,
$R_n = \sum_{k=n}^\infty a_k$ et $p\in{]0,1[}$.}
\begin{enumerate}
  \item \question{Montrer qu'il existe $C_p \in \R$ tel que
    $\sum_{n=0}^\infty \frac {a_n}{R_n^p} \le C_p A^{1-p}$.}
  \item \question{Trouver la meilleure constante $C_p$.}
\end{enumerate}
\begin{enumerate}
  \item \reponse{TAF : $\exists\ x_n \in {[R_{n+1},R_n]} \text{ tel que }
             R_n^{1-p} - R_{n+1}^{1-p} = (1-p) \frac {R_n-R_{n+1}}{x_n^p}
             \ge (1-p) \frac {a_n}{R_n^p}$.
             Donc, $\sum_{n=0}^{\infty} \frac {a_n}{R_n^p} \le
             \frac {A^{1-p}}{1-p}$.}
  \item \reponse{C'est $\frac 1{1-p}$ : Pour $a_n = k^n$,
             $A^{p-1}\sum_{n=0}^{\infty} \frac {a_n}{R_n^p}
             = \frac {1-k}{1-k^{1-p}} \to \frac 1{1-p}$ lorsque $k\to1^-$.}
\end{enumerate}
}