\uuid{4641}
\titre{DSF d'une primitive de $f$}
\theme{Exercices de Michel Quercia, Séries de Fourier}
\auteur{quercia}
\date{2010/03/14}
\organisation{exo7}
\contenu{
  \texte{}
  \question{Soit $f$ continue $2\pi$-périodique, $F(x) =  \int_{t=0}^x f(t)\,d t$,
$a_n, b_n$ les coefficients de Fourier trigonométriques de~$f$ et
$C = \frac1{2\pi} \int_{t=0}^{2\pi} (\pi-t)f(t)\,d t$. Montrer~:
$$\forall\ x\in\R,\ F(x) =
\frac{a_0x}2 + C + \sum_{n=1}^\infty \frac{a_n\sin nx - b_n\cos nx}n.
$$}
  \reponse{}
}