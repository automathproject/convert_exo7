\exo7id{2499}
\titre{Exercice 2499}
\theme{}
\auteur{sarkis}
\date{2009/04/01}
\organisation{exo7}
\contenu{
  \texte{Soient $||.||_1$ et $||.||_2$
deux normes sur $\mathbb{R}^2$ et $M=\left (
\begin{array}{cc}
a & b \\
c & d
\end{array}
\right )$ une matrice de ${\cal M}_{n,n}(\mathbb{R} \mbox{ ou }
\mathbb{C})$. On d\'efinit la norme de $M$ (ou de l'application
lin\'eaire associ\'ee) de la mani\`ere suivante:
$$||M||=\sup_{X \in S_1(0,1)}||M.X||_2$$
o\`u $S_1(0,1)$ est la sph\`ere unit\'e pour la norme $||.||_1$.
Dans chacun des cas suivant, calculez la norme de $M$.}
\begin{enumerate}
  \item \question{$||(x,y)||_1=||(x,y)||_2=sup(|x|,|y|).$}
  \item \question{$||(x,y)||_1=||(x,y)||_2=\sqrt{x^2+y^2}.$}
  \item \question{$||(x,y)||_1=\sqrt{x^2+y^2}$ et $||(x,y)||_2=sup(|x|,|y|).$}
\end{enumerate}
\begin{enumerate}

\end{enumerate}
}