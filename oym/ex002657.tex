\uuid{2657}
\titre{Exercice 2657}
\theme{}
\auteur{matexo1}
\date{2002/02/01}
\organisation{exo7}
\contenu{
  \texte{}
  \question{Calculer
$$\ell = \lim_{x\to+\infty}\left(\frac{\ln(x+1)}{\ln x}\right)^x.$$
Donner un équivalent de 
$$\left(\frac{\ln(x+1)}{\ln x}\right)^x - \ell$$
lorsque $x \to +\infty$.}
  \reponse{$$\ln(x+1) = \ln \Big(x \times (1+\frac1x)\Big) = \ln x+\ln\big(1+{\frac1x}\big) = \ln x + \frac 1x + o(\frac1x)$$
Donc 
$$\frac{\ln(x+1)}{\ln x} = 1 + \frac1{x\ln x} + o(\frac1{x\ln x}).$$
Ainsi
\begin{align*}
\left(\frac{\ln(x+1)}{\ln x}\right)^x 
  & = \exp\left( x \ln \left( \frac{\ln(x+1)}{\ln x}\right) \right)  \\
  & = \exp\left( x \ln \left(1 + \frac1{x\ln x} + o(\frac1{x\ln x})\right)  \right)  \\
  & = \exp\left( x \left(\frac1{x\ln x} + o(\frac1{x\ln x})\right) \right)  \\
  & = \exp\left( \frac1{\ln x} + o(\frac1{\ln x})\right)  \\
  & = 1 + \frac1{\ln x} + o(\frac1{\ln x}) \\
\end{align*}
On en déduit immédiatement que 
$$\lim_{x\to+\infty}\left(\frac{\ln(x+1)}{\ln x}\right)^x = 1$$
et que lorsque $x\to + \infty$
$$\left(\frac{\ln(x+1)}{\ln x}\right)^x - 1 \sim \frac1{\ln x}.$$}
}