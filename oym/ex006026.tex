\uuid{6026}
\titre{Exercice 6026}
\theme{}
\auteur{quinio}
\date{2011/05/20}
\organisation{exo7}
\contenu{
  \texte{Un vol Marseille - Paris est assuré par un Airbus de $150$
places ; pour ce vol des estimations ont montré que la probabilité
pour qu'une personne confirme son billet est $p=0.75$. La
compagnie vend $n$ billets, $n>150$. Soit $X$ la variable aléatoire <<nombre de personnes
parmi les $n$ possibles, ayant confirmé leur réservation pour ce vol>>.}
\begin{enumerate}
  \item \question{Quelle est la loi exacte suivie par $X$ ?}
  \item \question{Quel est le nombre maximum de places que la compagnie peut vendre pour
que, à au moins $95$\%, elle soit sûre que tout le monde puisse
monter dans l'avion, c'est-à-dire $n$ tel que : $P[X>150] \leq 0.05$ ?}
  \item \question{Reprendre le même exercice avec un avion de capacité de $300$
places; faites varier le paramètre $p = 0.5$ ; $p=0.8$.}
\end{enumerate}
\begin{enumerate}

\end{enumerate}
}