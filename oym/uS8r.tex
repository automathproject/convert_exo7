\uuid{uS8r}
\exo7id{4900}
\titre{\'Equations du second degré}
\theme{Exercices de Michel Quercia, Coniques}
\auteur{quercia}
\date{2010/03/17}
\organisation{exo7}
\contenu{
  \texte{Déterminer la nature et les éléments de la courbe d'équation dans un
repère $(O,\vec i,\vec j)$ orthonormé :}
\begin{enumerate}
  \item \question{$16x^2 -24xy + 9y^2 + 35x -20y = 0$.}
  \item \question{$5x^2 + 7y^2 + 2xy\sqrt3 -(10+2\sqrt3\,)x - (14+2\sqrt3\,)y - 4+2\sqrt3 = 0$.}
  \item \question{$x^2+xy+y^2 = 1$.}
  \item \question{$x^2+2y^2+4xy\sqrt3 + x + y\sqrt3 + 1 = 0$.}
  \item \question{$mx^2 + 4mx + (m-1)y^2 + 2 = 0$ ($m \in \R$).}
\end{enumerate}
\begin{enumerate}
  \item \reponse{Parabole.
             axes = $\frac 15\begin{pmatrix}4\cr-3\end{pmatrix}$, $\frac 15\begin{pmatrix}3\cr4\end{pmatrix}$.
             sommet : $X = -\frac 45$, $Y = -\frac{16}5$.}
  \item \reponse{Ellipse.
             $\Omega = \begin{pmatrix}1\cr1\end{pmatrix}$,
             axes à $-\frac\pi3,\frac\pi6$,
             $a=2$, $b=\sqrt2$.}
  \item \reponse{Ellipse.
             $\Omega = \begin{pmatrix}0\cr0\end{pmatrix}$,
             axes à $-\frac\pi4,\frac\pi4$,
             $a=\sqrt2$, $b=\sqrt{\frac 23}$.}
  \item \reponse{Centre $\begin{pmatrix}-2\cr 0\end{pmatrix}$.\par
             $$\left\{\begin{array}{cccccc} %????????
                &  &m &< 0  & \Rightarrow  &\text{ellipse horizontale.}   \hfill\cr
             0  &< &m &<1/2 & \Rightarrow  &\text{hyperbole verticale.}   \hfill\cr
             1/2&< &m &<1   & \Rightarrow  &\text{hyperbole horizontale.} \hfill\cr
               1&< &m &     & \Rightarrow  &\text{ellipse verticale.}     \hfill\cr  \end{array}\right.$$}
\end{enumerate}
}