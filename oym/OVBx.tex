\uuid{OVBx}
\exo7id{5448}
\titre{**I Le lemme de \textsc{Lebesgue}}
\theme{Intégration}
\auteur{rouget}
\date{2010/07/10}
\organisation{exo7}
\contenu{
  \texte{}
\begin{enumerate}
  \item \question{On suppose que $f$ est une fonction de classe $C^1$ sur $[a,b]$. Montrer que $\lim_{\lambda\rightarrow +\infty}\int_{a}^{b}\sin(\lambda t)f(t)\;dt=0$.}
  \item \question{(***) Redémontrer le même résultat en supposant simplement que $f$ est continue par morceaux sur $[a,b]$ (commencer par le cas des fonctions en escaliers).}
\end{enumerate}
\begin{enumerate}
  \item \reponse{Puisque $f$ est de classe $C^1$ sur $[a,b]$, on peut effectuer une intégration par parties qui fournit pour $\lambda>0$~:

$$\left|\int_{a}^{b}f(t)\sin(\lambda t)\;dt\right|=\left|\frac{1}{\lambda}(-\left[\cos(\lambda t)f(t)\right]_{a}^{b}+\int_{a}^{b}f'(t)\cos(\lambda t)\;dt)\right|\leq\frac{1}{\lambda}(|f(a)|+|f(b)|+\int_{a}^{b}|f'(t)|\;dt).$$ 

Cette dernière expression tend vers $0$ quand $\lambda$ tend vers $+\infty$, et donc $\int_{a}^{b}f(t)\sin(\lambda t)\;dt$ tend vers $0$ quand $\lambda$ tend vers $+\infty$.}
  \item \reponse{Si $f$ est simplement supposée continue par morceaux, on ne peut donc plus effectuer une intégration par parties.

Le résultat est clair si $f=1$, car pour $\lambda>0$, $\left|\int_{a}^{b}\sin(\lambda t)\;dt\right|=...\leq\frac{2}{\lambda}$.

Le résultat s'étend aux fonctions constantes par linéarité de l'intégrale puis aux fonctions constantes par morceaux par additivité par rapport à l'intervalle d'intégration, c'est-à-dire aux fonctions en escaliers.

Soit alors $f$ une fonction continue par morceaux sur $[a,b]$. 

Soit $\varepsilon>0$. On sait qu'il existe une fonction en escaliers $g$ sur $[a,b]$ telle que $\forall x\in[a,b],\;|f(x)-g(x)|<\frac{\varepsilon}{2(b-a)}$.

Pour $\lambda>0$, on a alors 

\begin{align*}\ensuremath
\left|\int_{a}^{b}f(t)\sin(\lambda t)\;dt\right|&=\left|\int_{a}^{b}(f(t)-g(t))\sin(\lambda t)\;dt+\int_{a}^{b}g(t)\sin(\lambda t)\;dt\right|\\
 &\leq\int_{a}^{b}|f(t)-g(t)|\;dt+\left|\int_{a}^{b}g(t)\sin(\lambda  t)\;dt\right|\leq(b-a)\frac{\varepsilon}{2(b-a)}+\left|\int_{a}^{b}g(t)\sin(\lambda t)\;dt\right|\\
 &=\frac{\varepsilon}{2}+\left|\int_{a}^{b}g(t)\sin(\lambda t)\;dt\right|.
\end{align*}

Maintenant, le résultat étant établi pour les fonctions en esacliers, 

$$\exists A>0/\;\forall\lambda\in\Rr,\;
(\lambda>A\Rightarrow\left|\int_{a}^{b}g(t)\sin(\lambda t)\;dt\right|<\frac{\varepsilon}{2}).$$

Pour $\lambda>A$, on a alors $\left|\int_{a}^{b}f(t)\sin(\lambda t)\;dt\right|<\frac{\varepsilon}{2}+\frac{\varepsilon}{2}=\varepsilon$. On a montré que 

$$\forall\varepsilon>0,\exists A>0/\;\forall\lambda\in\Rr,\;(\lambda>A\Rightarrow\left|\int_{a}^{b}f(t)\sin(\lambda t)\;dt\right|<\varepsilon),$$

et donc que $\int_{a}^{b}f(t)\sin(\lambda t)\;dt$ tend vers $0$ quand $\lambda$ tend vers $+\infty$.}
\end{enumerate}
}