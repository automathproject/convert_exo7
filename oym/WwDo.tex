\uuid{WwDo}
\exo7id{3341}
\titre{$f^2 = -\mathrm{id}$}
\theme{Exercices de Michel Quercia, Applications linéaires en dimension finie}
\auteur{quercia}
\date{2010/03/09}
\organisation{exo7}
\contenu{
  \texte{Soit $E$ un $\R$-ev et $f \in \mathcal{L}(E)$ tel que $f\circ f = -\mathrm{id}_E$.
Pour $z = x + iy \in \C$ et $\vec u \in E$, on pose :
$z\vec u = x\vec u + yf(\vec u)$.}
\begin{enumerate}
  \item \question{Montrer qu'on définit ainsi une structure de $\C$-ev sur $E$.}
  \item \question{En déduire que $\dim_{\R}(E)$ est paire.}
\end{enumerate}
\begin{enumerate}

\end{enumerate}
}