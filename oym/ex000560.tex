\uuid{560}
\titre{Exercice 560}
\theme{}
\auteur{cousquer}
\date{2003/10/01}
\organisation{exo7}
\contenu{
  \texte{On considère la suite réelle définie par~:
$$x_0=1 \quad\mbox{et} \quad x_{n+1}= \sqrt{2x_n+1}.$$}
\begin{enumerate}
  \item \question{Montrer que $x_n$ est supérieur ou égal à~$1$ pour tout~$n$.}
  \item \question{Montrer que si $(x_n)$ converge, sa limite $l$ vérifie
$$l= \sqrt{2l+1}.$$}
  \item \question{$l$ étant définie par l'égalité de 2), est-il
possible de trouver $k\in\mathopen]0,1\mathclose[$ tel que
$$\vert x_n-l\vert \leq k \vert x_{n-1}-l\vert.$$
Si oui en déduire que $\vert x_n-l\vert \leq k^n \vert x_0-l\vert$.
Conclure.}
\end{enumerate}
\begin{enumerate}

\end{enumerate}
}