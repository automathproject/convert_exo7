\exo7id{7784}
\titre{Simplicité de $\mathcal{A}_5$}
\theme{}
\auteur{mourougane}
\date{2021/08/11}
\organisation{exo7}
\contenu{
  \texte{}
\begin{enumerate}
  \item \question{Faire la liste des classes de conjugaison de $\mathcal{S}_n$ dans $\mathcal{A}_n$ en les dénombrant.}
  \item \question{Montrer que les $3$-cycles sont conjugués dans $\mathcal{A}_n$.}
  \item \question{Montrer que les éléments d'ordre $2$ sont conjugués dans $\mathcal{A}_n$.}
  \item \question{Montrer que tout sous-groupe distingué $H$ de $\mathcal{A}_n$ qui contient un élément d'ordre $5$ les contient tous. (On remarquera que le groupe engendré par un élément d'ordre $5$ est un Sylow.)}
  \item \question{Montrer que tout sous-groupe distingué $H$ de $\mathcal{A}_n$ non réduit à $\{id\}$ contient au moins deux types d'éléments en plus de l'identité. Montrer alors que $H=\mathcal{A}_n$.}
\end{enumerate}
\begin{enumerate}

\end{enumerate}
}