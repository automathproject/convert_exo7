\uuid{qVlo}
\exo7id{2939}
\titre{Racines de l'unit{\'e}}
\theme{Exercices de Michel Quercia, Nombres complexes}
\auteur{quercia}
\date{2010/03/08}
\organisation{exo7}
\contenu{
  \texte{R{\'e}soudre :}
\begin{enumerate}
  \item \question{$(z+1)^n = (z-1)^n$.}
  \item \question{$(z+1)^n = z^n = 1$.}
  \item \question{$z^4 - z^3 + z^2 - z + 1 = 0$.}
  \item \question{$1 + 2z + 2z^2 + \dots + 2z^{n-1} + z^n = 0$.}
  \item \question{$\left(\frac{1+ix}{1-ix}\right)^n = \frac{1+i\tan a}{1-i\tan a}$.}
  \item \question{$\overline x = x^{n-1}$.}
  \item \question{$\left(\frac{z+1}{z-1}\right)^3 + \left(\frac{z-1}{z+1}\right)^3 = 0$.}
\end{enumerate}
\begin{enumerate}
  \item \reponse{$z=-i\mathrm{cotan}\frac {k\pi}n$.}
  \item \reponse{$6\mid n \Rightarrow  z = j$ ou $j^2$. Sinon, pas de solution.}
  \item \reponse{$z = \exp\frac{(2k+1)i\pi}5$, $k = 0,1,3,4$.}
  \item \reponse{$z = -1$ ou $z = \exp\frac{2ik\pi}n$, $1\le k< n$.}
  \item \reponse{$x = \tan\left(\frac {a+2k\pi}n\right)$.}
  \item \reponse{$z = \pm i,\ \pm i(2\pm\sqrt3)$.}
\end{enumerate}
}