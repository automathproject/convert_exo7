\uuid{h3BZ}
\exo7id{520}
\titre{Exercice 520}
\theme{Suites, Convergence}
\auteur{ridde}
\date{1999/11/01}
\organisation{exo7}
\contenu{
  \texte{Soit $H_n = 1 + \dfrac12 + \cdots + \dfrac1n$.}
\begin{enumerate}
  \item \question{En utilisant une int\'egrale, montrer que pour tout $n>0$ : $\dfrac1{n + 1} \leq
\ln (n + 1)-\ln (n) \leq \dfrac1n$.}
  \item \question{En d\'eduire que $\ln (n + 1) \leq H_n \leq \ln (n) + 1$.}
  \item \question{D\'eterminer la limite de $H_n$.}
  \item \question{Montrer que $u_n = H_n-\ln (n)$ est d\'ecroissante et positive.}
  \item \question{Conclusion ?}
\end{enumerate}
\begin{enumerate}
  \item \reponse{La fonction $t \mapsto \frac 1 t$ est d\'ecroissante
sur $[n,n+1]$ donc
$$\frac{1}{n+1} \leqslant \int_n^{n+1} \frac{dt}{t} \leqslant \frac 1n$$
(C'est un encadrement de l'aire de l'ensemble des points $(x,y)$
du plan tels que $x\in[n,n+1]$ et $0\leqslant y\leqslant 1/x$ par l'aire de
deux rectangles.) Par calcul de l'intégrale nous obtenons l'in\'egalit\'e :
$$\frac{1}{n+1} \leqslant \ln(n+1)-\ln(n)  \leqslant \frac 1n.$$}
  \item \reponse{$H_n = \frac1n+\frac{1}{n-1}+\cdots +\frac12+1$, nous majorons chaque terme de cette somme en utilisant l'in\'egalit\'e $\frac1k \leqslant \ln(k)-\ln (k-1)$ obtenue pr\'ec\'edemment : nous obtenons
$H_n \leqslant \ln(n)-\ln (n-1) + \ln(n-1)-\ln (n-2)+\cdots-\ln(2) +\ln (2) - \ln
(1) + 1$. Cette somme est t\'elescopique (la plupart des termes
s'\'eliminent et en plus $\ln (1) =0$) et donne $H_n \leqslant \ln (n) + 1$.

L'autre in\'egalit\'e  s'obtient de la fa\c{c}on similaire en
utilisant l'in\'egalit\'e $ \ln(k+1)-\ln(k) \leqslant \frac{1}{k}$ .}
  \item \reponse{Comme $H_n \geqslant \ln (n+1)$ et que $\ln(n+1) \rightarrow +\infty$ quand $n\rightarrow +\infty$ alors $H_n \rightarrow +\infty$ quand $n\rightarrow +\infty$.}
  \item \reponse{$u_{n+1}-u_n = H_{n+1}-H_n - \ln(n+1)+\ln(n) = \frac{1}{n+1}-(\ln (n+1)-\ln (n))\leqslant 0$ d'apr\`es la premi\`ere question. Donc $u_{n+1}-u_n  \leqslant 0$. Ainsi $u_{n+1} \leqslant u_n$ et la suite
$(u_n)$ est d\'ecroissante.

Enfin comme $H_n \geqslant \ln(n+1)$ alors $H_n \geqslant \ln (n)$ et donc
$u_n\geqslant 0$.}
  \item \reponse{La suite $(u_n)$ est d\'ecroissante et minor\'ee (par $0$) donc elle converge
vers un r\'eel $\gamma$. Ce r\'eel $\gamma$ s'appelle \emph{la constante d'Euler}
(d'après Leonhard Euler, 1707-1783, math\'ematicien d'origine suisse). Cette
constante vaut environ $0,5772156649\ldots$ mais on ne sait pas si
$\gamma$ est rationnel ou irrationnel.}
\end{enumerate}
}