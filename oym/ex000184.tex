\uuid{184}
\titre{Exercice 184}
\theme{}
\auteur{ridde}
\date{1999/11/01}
\organisation{exo7}
\contenu{
  \texte{Pour calculer des sommes portant sur deux indices, on a int\'erêt
\`a repr\'esenter la z{o}ne du plan couverte par ces indices et
\`a sommer en lignes, colonnes ou diagonales... Calculer :}
\begin{enumerate}
  \item \question{$\sum\limits_{1\leq i \leq j \leq n}ij$.}
  \item \question{$\sum\limits_{1\leq i < j \leq n}i (j-1)$.}
  \item \question{$\sum\limits_{1\leq i < j \leq n} (i-1)j$.}
  \item \question{$\sum\limits_{1\leq i \leq j \leq n} (n-i) (n-j)$.}
  \item \question{$\sum\limits_{1\leq p, q \leq n} (p + q)^2$ (on posera $k = p + q$).}
\end{enumerate}
\begin{enumerate}

\end{enumerate}
}