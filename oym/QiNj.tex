\uuid{QiNj}
\exo7id{2913}
\titre{Permutations de couples}
\theme{Exercices de Michel Quercia, Ensembles finis}
\auteur{quercia}
\date{2010/03/08}
\organisation{exo7}
\contenu{
  \texte{On doit placer autour d'une table ronde un groupe de $2n$ personnes, $n$ hommes
et $n$ femmes, qui constituent $n$ couples.
Combien existe-t-il de dispositions $\ldots$}
\begin{enumerate}
  \item \question{au total ?}
  \item \question{en respectant l'alternance des sexes ?}
  \item \question{sans s{\'e}parer les couples ?}
  \item \question{en remplissant les deux conditions pr{\'e}c{\'e}dentes ?}
\end{enumerate}
\begin{enumerate}
  \item \reponse{$(2n)!$.}
  \item \reponse{$2(n!)^2$.}
  \item \reponse{$2^{n+1}\times n!$.}
  \item \reponse{$4\times n!$.}
\end{enumerate}
}