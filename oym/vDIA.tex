\uuid{vDIA}
\exo7id{3397}
\titre{Calcul de $A^k$}
\theme{Exercices de Michel Quercia, Calcul matriciel}
\auteur{quercia}
\date{2010/03/10}
\organisation{exo7}
\contenu{
  \texte{Calculer $A^k$ pour $k \in \N$ :}
\begin{enumerate}
  \item \question{$A = \begin{pmatrix}1 &       &(2)\cr
                     &\ddots     \cr
                  (2)&       &1  \cr\end{pmatrix}$.}
  \item \question{$A = \begin{pmatrix} 1 &2 &3 &4 \cr
                    0 &1 &2 &3 \cr
                    0 &0 &1 &2 \cr
                    0 &0 &0 &1 \cr \end{pmatrix}$.}
  \item \question{$A = \begin{pmatrix} x^2 &xy  &xz  \cr
                    xy  &y^2 &yz  \cr
                    xz  &yz  &z^2 \cr \end{pmatrix}$.}
\end{enumerate}
\begin{enumerate}
  \item \reponse{$\begin{pmatrix} a & &b    \cr
                          &\ddots \cr
                        b & &a    \cr\end{pmatrix}$ avec
            $\begin{cases} a = \frac1n\bigl( (2n-1)^k + (n-1)(-1)^k \bigr) \cr
                     b = \frac1n\bigl( (2n-1)^k -      (-1)^k \bigr).\cr \end{cases}$}
  \item \reponse{$\begin{pmatrix} 1 &2k &2k^2+k &\frac 13(4k^3 + 6k^2 + 2k) \cr
                       0 &1  &2k     &2k^2+k                     \cr
                       0 &0  &1      &2k                         \cr
                       0 &0  &0      &1                          \cr \end{pmatrix}$.}
  \item \reponse{$\left[{ \begin{pmatrix} x\cr y\cr z \cr\end{pmatrix} \begin{pmatrix} x &y &z \cr\end{pmatrix} }\right]^k            = (x^2+y^2+z^2)^{k-1}A$.}
\end{enumerate}
}