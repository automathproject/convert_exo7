\uuid{5872}
\titre{**}
\theme{}
\auteur{rouget}
\date{2010/10/16}
\organisation{exo7}
\contenu{
  \texte{}
  \question{Pour $A\in\mathcal{M}_n(\Cc)$, calculer $\lim_{p \rightarrow +\infty}\left(I_n+ \frac{A}{p}\right)^p$.}
  \reponse{On munit $\mathcal{M}_n(\Cc)$ d'une norme sous-multiplicative notée $\|\;\|$. Soit $A\in\mathcal{M}_n(\Cc)$. Soit $p\in\Nn^*$.

\begin{center}
$\left\|\sum_{k=0}^{p} \frac{A^k}{k!}-\left(I+ \frac{A}{p}\right)^p\right\| =\left\|\sum_{k=0}^{p}\left( \frac{1}{k!}- \frac{C_p^k}{p^k}\right)A^k\right\|\leqslant\sum_{k=0}^{p}\left| \frac{1}{k!}- \frac{C_p^k}{p^k}\right|\|A\|^k $.
\end{center}

Maintenant, $\forall k\in\llbracket1,p\rrbracket$, $ \frac{1}{k!}- \frac{C_p^k}{p^k}= \frac{1}{k!}\left(1-  \frac{\overbrace{p\times(p-1)\times\ldots\times(p-k+1)}^{k}}{\underbrace{p\times p\times\ldots\times p}_{k}}\right)\geqslant 0$. Donc,

\begin{center}
$\sum_{k=0}^{p}\left| \frac{1}{k!}- \frac{C_p^k}{p^k}\right|\|A\|^k=\sum_{k=0}^{p} \frac{\|A\|^k}{k!}-\left(1+ \frac{\|A\|^p}{p}\right)^p\underset{n\rightarrow+\infty}{\rightarrow}e^{\|A\|}-e^{\|A\|}= 0$.
\end{center}

On en déduit que $\sum_{k=0}^{p} \frac{A^k}{k!}-\left(I+ \frac{A}{p}\right)^p$ tend vers $0$ quand $p$ tend vers $+\infty$ et puisque $\sum_{k=0}^{p} \frac{A^k}{k!}$ tend vers $\text{exp}(A)$ quand $p$ tend vers $+\infty$, il en est de même de $\left(I+ \frac{A}{p}\right)^p$.

\begin{center}
\shadowbox{
$\forall A\in\mathcal{M}_n(\Cc)$, $\text{exp}(A)=\lim_{p \rightarrow +\infty}\left(I_n+ \frac{A}{p}\right)^p$.
}
\end{center}}
}