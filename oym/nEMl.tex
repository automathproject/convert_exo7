\uuid{nEMl}
\exo7id{4594}
\titre{Coefficients équivalents $ \Rightarrow $ séries équivalentes}
\theme{Exercices de Michel Quercia, Séries entières}
\auteur{quercia}
\date{2010/03/14}
\organisation{exo7}
\contenu{
  \texte{Soit $(a_n)$ une suite de réels strictement positifs.
On suppose que le rayon de convergence de la série entière
$A(x) = \sum_{n=0}^\infty a_nx^n$ est 1 et que la série diverge pour $x = 1$.}
\begin{enumerate}
  \item \question{Montrer que $A(x) \to +\infty$ lorsque $x\to1^-$.}
  \item \question{Soit $(b_n)$ une suite telle que $b_n \sim a_n$ et
    $B(x) = \sum_{n=0}^\infty b_nx^n$.
    Montrer que $B(x) \sim A(x)$ pour $x\to1^-$.}
\end{enumerate}
\begin{enumerate}
  \item \reponse{Fonction croissante. $\lim_{x\to1^-} A(x) \ge \sum_{n=0}^N a_n$.}
  \item \reponse{Dém de type Césaro.}
\end{enumerate}
}