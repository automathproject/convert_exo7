\uuid{XtWd}
\exo7id{7851}
\titre{Groupe d'ordre $p^3$}
\theme{Exercices de Christophe Mourougane, Algèbre}
\auteur{mourougane}
\date{2021/08/11}
\organisation{exo7}
\contenu{
  \texte{Soit $G$ un groupe non abélien d'ordre $p^3$ où $p$ est un
nombre premier.}
\begin{enumerate}
  \item \question{Montrer que le centre de $G$ est d'ordre $p$ et égal à son sous-groupe dérivé $Z(G)=D(G)$.}
  \item \question{En déduire que le nombre de classes de conjugaison est $p^2+p-1$. (On pourra étudier l'action de $G$ sur lui-même par
conjugaison~: ses points fixes, l'orbite des éléments, le
stabilisateur des éléments et appliquer la formule de Burnside...)}
  \item \question{Montrer que $G/Z(G)$ est isomorphe à $\Z/p\Z\times \Z/p\Z$.}
  \item \question{Montrer que tout sous-groupe de $G$ d'ordre $p^2$ contient le centre $Z(G)$ de $G$,
et que donc $G$ n'est pas un produit semi-direct de son centre par son abélianisé.
% \href{https://groupprops.subwiki.org/wiki/Semidirect_product_of_cyclic_group_of_prime-square_order_and_cyclic_group_of_prime_order}{Voir sur les groupes d'ordre $p^3$}}
\end{enumerate}
\begin{enumerate}

\end{enumerate}
}