\uuid{4928}
\titre{Cône s'appuyant sur une ellipse}
\theme{}
\auteur{quercia}
\date{2010/03/17}
\organisation{exo7}
\contenu{
  \texte{Soit ${\cal E}$ l'ellipse d'équations :
$\begin{cases}\frac{x^2}{a^2} + \frac{y^2}{b^2} = 1 \cr
        z = 0 \cr\end{cases}$
et $\Omega = (x_0,y_0,z_0)$ avec $z_0 \ne 0$.
On note $\mathcal{C}$ le cône de sommet $\Omega$ engendré par ${\cal E}$.}
\begin{enumerate}
  \item \question{Chercher une équation cartésienne de $\mathcal{C}$.}
  \item \question{Quels sont les points $\Omega$ tels que $\mathcal{C} \cap Oyz$ soit un cercle ?}
\end{enumerate}
\begin{enumerate}
  \item \reponse{$\frac{(xz_0-x_0z)^2}{a^2} + \frac{(yz_0-y_0z)^2}{b^2}
              = (z-z_0)^2$.}
  \item \reponse{$y_0 = 0$, $\frac{x_0^2}{a^2} - \frac{z_0^2}{b^2} = 1$.}
\end{enumerate}
}