\exo7id{5021}
\titre{Perpendiculaire à $OM$ sur une ellipse}
\theme{}
\auteur{quercia}
\date{2010/03/17}
\organisation{exo7}
\contenu{
  \texte{Soit $\cal E$ une ellipse de centre $O$, de paramètres $a$ et $b$.
Pour $M \in \cal E$, soit $D$ la perpendiculaire en $M$ à $(OM)$.}
\begin{enumerate}
  \item \question{Donner les équations paramétriques de l'enveloppe des droites $D$.}
  \item \question{Tracer les enveloppes sur ordinateur pour différentes valeurs de $a/b$.}
  \item \question{\'Etudier les points stationnaires de l'enveloppe quand il y en a.}
\end{enumerate}
\begin{enumerate}
  \item \reponse{$x = \frac{\cos t}a(a^2 + (a^2-b^2)\sin^2t)$,
             $y = \frac{\sin t}b(b^2 - (a^2-b^2)\cos^2t)$.}
  \item \reponse{Point stationnaire ssi $a^2 > 2b^2$, obtenu pour
             $\sin^2t = \frac{a^2-2b^2}{3(a^2-b^2)}$. Rebroussement de $1^{\text{ère}}$ espèce.}
\end{enumerate}
}