\uuid{6851}
\titre{Exercice 6851}
\theme{}
\auteur{gijs}
\date{2011/10/16}
\organisation{exo7}
\contenu{
  \texte{Soit $a$ un réel, $a>1$. On considère la série
entière
$$\sum_{n=0}^{+\infty}{z^n\over a^{n^2}}.$$}
\begin{enumerate}
  \item \question{Montrer que sa somme, notée $f(z)$, est holomorphe dans $\C$.
Montrer qu'il existe $a_0>1$ tel que l'on ait, pour tout $a\ge a_0$,
$$\sum_{k=1}^{+\infty}{1\over a^{k^2}}\le {1\over 100}.$$}
  \item \question{\emph{Dans toute la suite, on supposera $a\ge a_0$.}\\
Soit $p$ entier, $p\ge 1$. Montrer que l'on a, pour tout $z$ vérifiant
$\vert z\vert =a^{2p}$,
$${1\over a^{p^2}}\left\vert f(z)-{z^p\over a^{p^2}}\right\vert\le
{2\over 100}.$$
En déduire que $f(z)$ a $p$ zéros, $z_1, \dots ,z_p$ dans le disque
ouvert $D(0,a^{2p})$. Montrer que, quel que soit $p\ge 1$, $z_p$ a les
propriétés suivantes :
\begin{enumerate}}
  \item \question{$a^{2(p-1)}<\vert z\vert <a^{2p}$,}
  \item \question{$z_p$ est un zéro simple,}
  \item \question{$z_p$ est un réel négatif.}
\end{enumerate}
\begin{enumerate}

\end{enumerate}
}