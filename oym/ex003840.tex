\uuid{3840}
\titre{Orthogonal d'un hyperplan}
\theme{}
\auteur{quercia}
\date{2010/03/11}
\organisation{exo7}
\contenu{
  \texte{Soit $E = \mathcal{C}([0,1],\R)$ muni du produit scalaire usuel et, pour $f \in E$ :
$\varphi(f) =  \int_{t=0}^{1/2} f(t)\,d t$.}
\begin{enumerate}
  \item \question{Montrer que $\varphi$ est continue.}
  \item \question{Montrer que $H = \mathrm{Ker}\varphi$ est fermé.}
  \item \question{Montrer que $H^\bot = \{0\}$.}
\end{enumerate}
\begin{enumerate}
  \item \reponse{Soit $g \in H^\bot$ non nulle. Les formes linéaires :
             $f  \mapsto  \int_{0}^{1/2} f$ et $f  \mapsto  \int_0^1 fg$ sont
             nulles sur $H$, donc proportionnelles, ce qui est impossible
             pour $g$ continue.}
\end{enumerate}
}