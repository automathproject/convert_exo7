\uuid{5016}
\titre{Cordes d'une hyperbole}
\theme{Exercices de Michel Quercia, Enveloppes}
\auteur{quercia}
\date{2010/03/17}
\organisation{exo7}
\contenu{
  \texte{}
  \question{Soit $\cal H$ une hyperbole de foyer $F$. Trouver l'enveloppe des cordes
$[P,Q]$ de $\cal H$ vues depuis $F$ sous un angle droit.}
  \reponse{équation polaire : $\rho = \frac p{1+e\cos\theta}  \Rightarrow 
\begin{cases}x = \frac {p(\cos\theta-\sin\theta)}{2 + e(\cos\theta-\sin\theta)}\cr
        y = \frac {p(\cos\theta+\sin\theta)}{2 + e(\cos\theta-\sin\theta)}\cr \end{cases}$
	 conique d'excentricité $\frac e{\sqrt2}$.}
}