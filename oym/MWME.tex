\uuid{MWME}
\exo7id{3010}
\titre{Anneau de caract{\'e}ristique 2}
\theme{Exercices de Michel Quercia, Anneaux}
\auteur{quercia}
\date{2010/03/08}
\organisation{exo7}
\contenu{
  \texte{Soit $A$ un anneau non nul tel que : $\forall\ x \in A,\ x^2 = x$.}
\begin{enumerate}
  \item \question{Exemple d'un tel anneau ?}
  \item \question{Quels sont les {\'e}l{\'e}ments inversibles de $A$ ?}
  \item \question{Montrer que : $\forall\ x \in A,\ x+x = 0$.
    En d{\'e}duire que $A$ est commutatif.}
  \item \question{Pour $x,y \in A$ on pose :
    $x \le y \iff \exists\ a \in A \text{ tel que } x=ay$.
    Montrer que c'est une relation d'ordre.}
\end{enumerate}
\begin{enumerate}
  \item \reponse{$1$.}
  \item \reponse{$x+y = (x+y)^2 = x^2+y^2+xy+yx = x+y+xy+yx  \Rightarrow  xy+yx = 0$.\par
             Pour $y = 1$ : $x+x = 0  \Rightarrow  1 = -1$.\par
             Pour $y$ quelconque : $xy = -yx = yx$.}
  \item \reponse{Antisym{\'e}trie : si $x = ay$, alors $xy = ay^2 = ay = x$.\par
             Donc $(x\le y)$ et $(y\le x)  \Rightarrow  xy = x = y$.}
\end{enumerate}
}