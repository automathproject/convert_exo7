\uuid{4846}
\titre{Norme pour les fonctions lipschitziennes}
\theme{Exercices de Michel Quercia, Espaces complets}
\auteur{quercia}
\date{2010/03/16}
\organisation{exo7}
\contenu{
  \texte{}
  \question{Soit $E = \{ \text{fonctions lipchitziennes } f:\R \to \R \}$.
Pour $f \in E$, on pose
$\|f\| = |f(0)| + \sup\limits_{x\ne y} \left|\frac{f(x)-f(y)}{x-y}\right|$.

Montrer que $E$ est complet.}
  \reponse{}
}