\uuid{PlTS}
\exo7id{7251}
\titre{Diagramme de Voronoï}
\theme{Exercices de Christophe Mourougane, Géométrie en petites dimensions}
\auteur{mourougane}
\date{2021/08/10}
\organisation{exo7}
\contenu{
  \texte{Soit \((A_i, P_i)_{i \in I}\) un diagramme de Voronoï. Si \(i, j \in I\) 
avec \(i \neq j\), notons \( \Pi_{i,j}\) le demi-plan de frontière la 
médiatrice du segment \([P_iPj]\) et qui contient le point \(P_i\).}
\begin{enumerate}
  \item \question{Montrer que, pour tout \(i \in I\), on~a:
\[
A_i = \bigcap_{j \in I \setminus \{i\}} \Pi_{i,j}.
\]}
  \item \question{Dessiner le diagramme de Voronoï de trois points formant un 
triangle équilatéral.}
\end{enumerate}
\begin{enumerate}

\end{enumerate}
}