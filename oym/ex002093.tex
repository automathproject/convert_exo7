\uuid{2093}
\titre{Exercice 2093}
\theme{}
\auteur{bodin}
\date{2008/02/04}
\organisation{exo7}
\contenu{
  \texte{Soit $F(x)=\displaystyle \int_x^{x^2}\frac{1}{\ln t}d t$}
\begin{enumerate}
  \item \question{Quel est l'ensemble de d\'efinition de $F$. $F$ est-elle continue,
d\'erivable sur son ensemble de d\'efinition ?}
  \item \question{D\'eterminer $\lim_{x\to 1^+} F(x)$ en comparant $F$ \`a $H(x)=\displaystyle \int_x^{x^2}\frac{1}{t\ln t}d t$.}
\end{enumerate}
\begin{enumerate}
  \item \reponse{$F$ est d\'efinie sur $]0,1[\cup ]1,+\infty[$. $F$ est continue et d\'erivable sur $]0,1[$
et sur $]1,+\infty[$. Pour vois cela il suffit d'\'ecrire $F(x)= \int_x^a \frac{dt}{\ln t}+\int_a^{x^2} \frac{dt}{\ln t}$. La premi\`ere de ces fonctions est continue et d\'erivable (c'est une primitive), la seconde est la compos\'ee de $x\mapsto x^2$ avec $x\mapsto \int_a^{x} \frac{dt}{\ln t}$ et est donc aussi continue et d\'erivable.
On pourrait m\^eme calculer la d\'eriv\'ee.}
  \item \reponse{Notons $f(t) = \frac 1{\ln t}$ et $g(t) = \frac 1{t\ln t}$. On se place sur $]1,+\infty[$.
Bien \'evidemment $g(t) \leqslant f(t)$, mais nous avons aussi que pour $\epsilon >0$ fix\'e il existe $x>1$ tel que
pour tout $t\in [1,x^2]$ on ait $\frac 1t \leqslant 1+\epsilon$ donc sur $]1,x^2]$ nous avons $f(t) \leqslant (1+\epsilon)g(t)$.
Par int\'egration de l'in\'egalit\'e $g(t) \leqslant f(t) \leqslant (1+\epsilon)g(t)$ sur $[x,x^2]$ nous obtenons pour $x$ assez proche de $1$ :
$$H(x) \leqslant F(x)  \leqslant (1+\epsilon)H(x).$$

Il ne reste plus qu'a calculer $H(x)$. En fait $g(t) = \frac 1{t\ln t}$ est la d\'eriv\'ee de la fonction $h(t) = \ln (\ln t)$. Donc 
\begin{align*}
H(x) = \int_x^{x^2}\frac{dt}{t\ln t} 
 &= [ \ln (\ln t) ]_x^{x^2}  = \ln (\ln (x^2))-\ln (\ln x) \\
 &= \ln (2 \ln x)-\ln (\ln x) = \ln \frac {2 \ln x}{\ln x} \\
 &= \ln 2. \\
\end{align*}

Nous obtenons alors, pour $\epsilon >0$ fix\'e et $x>1$ assez proche de $1$, l'encadrement 
$$\ln 2 \leqslant F(x) \leqslant (1+\epsilon) \ln 2.$$

Donc la limite de $F(x)$ quand $x\to 1^+$ est $\ln 2$.}
\end{enumerate}
}