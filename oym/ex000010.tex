\exo7id{10}
\titre{Exercice 10}
\theme{}
\auteur{bodin}
\date{1998/09/01}
\organisation{exo7}
\contenu{
  \texte{\'Etablir les \'egalit\'es suivantes :}
\begin{enumerate}
  \item \question{$(\cos (\pi/7) + i \sin (\pi/7))(\frac{1-i\sqrt{3}}{2})(1+i) = \sqrt{2}(\cos(5\pi/84)+i\sin(5\pi/84)),$}
  \item \question{$(1-i)(\cos (\pi/5) + i \sin (\pi/5))(\sqrt{3}-i) = 2\sqrt{2}(\cos(13\pi/60)-i\sin(13\pi/60)),$}
  \item \question{$\frac {\sqrt{2}(\cos (\pi/12) + i \sin (\pi/12))}{1+i} = \frac {\sqrt{3}-i}{2}.$}
\end{enumerate}
\begin{enumerate}

\end{enumerate}
}