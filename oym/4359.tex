\uuid{4359}
\titre{Exercice 4359}
\theme{Exercices de Michel Quercia, Intégrale dépendant d'un paramètre}
\auteur{quercia}
\date{2010/03/12}
\organisation{exo7}
\contenu{
  \texte{}
  \question{On pose pour $n\ge2$~: $v_n= \int_{x=0}^1 \frac 1{1+x^n}\, d x$.
Montrer que la suite $(v_n)$ converge. Nature de la série $\sum(v_n-1)$ ?}
  \reponse{$v_n\to 1$ (lorsque $n\to\infty$) par convergence dominée.
$v_n-1 =  \int_{x=0}^1\frac{x^n}{1+x^n}\, d x = \frac1n \int_{u=0}^1\frac{u^{-1/n}}{1+u}\,d u
\sim \frac{\ln 2}n$ donc la série diverge.}
}