\uuid{b08K}
\exo7id{4893}
\titre{Symétriques d'un point par rapport aux milieux des cotés}
\theme{Exercices de Michel Quercia, Propriétés des triangles}
\auteur{quercia}
\date{2010/03/17}
\organisation{exo7}
\contenu{
  \texte{Soit un triangle $ABC$, $A',B',C'$, les milieux des côtés, et $M$ un point
du plan $(ABC)$ de coordonnées barycentriques $(\alpha,\beta,\gamma)$.}
\begin{enumerate}
  \item \question{Chercher les coordonnées barycentriques de $P,Q,R$ symétriques de $M$
     par rapport aux points $A',B',C'$.}
  \item \question{Montrer que les droites $(AP)$, $(BQ)$, $(CR)$ sont concourantes en un
     point $N$.}
  \item \question{Montrer que $N$ est le milieu de $[A,P]$, $[B,Q]$, $[C,R]$.}
  \item \question{Reconnaître l'application $M \mapsto N$.}
\end{enumerate}
\begin{enumerate}
  \item \reponse{$N = \text{Bar}(A:1-\alpha, B:1-\beta, C:1-\gamma)$.}
  \item \reponse{homothétie de centre $G$, de rapport $-\frac 12$.}
\end{enumerate}
}