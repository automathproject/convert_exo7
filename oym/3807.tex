\uuid{3807}
\titre{$I + a(X^tY - Y^tX)$ inversible}
\theme{Exercices de Michel Quercia, Problèmes matriciels}
\auteur{quercia}
\date{2010/03/11}
\organisation{exo7}
\contenu{
  \texte{}
  \question{Soient $X,Y \in \mathcal{M}_{n,1}(\R)$ indépendantes, $a \in \R$ et $M$ la matrice
$n\times n$ telle que $m_{ij} = x_iy_j - x_jy_i$.

A quelle condition $I + aM$ est-elle inversible ?}
  \reponse{$M = X^tY - Y^tX$.

Soit $Z \in \mathcal{M}_{n,1}(\R)$ tq $(I + aM)Z = 0$.
Donc $Z \in \text{vect}(X,Y)$ : $Z = \lambda X + \mu Y$.

on remplace :
$$\left\{\begin{array}{lllll}
 (1-a{}^tYX)\lambda &-& a{}^tYY\mu      &=&0 \cr
 a{}^tXX\lambda     &+& (1+a{}^tYX)\mu  &=&0 \cr
\end{array}\right.$$

CNS $\iff a^2({}^tXX\,^tYY - ({}^tXY)^2) + 1 \ne 0$.}
}