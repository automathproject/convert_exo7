\uuid{IPgq}
\exo7id{4565}
\titre{Calculs de rayons}
\theme{Exercices de Michel Quercia, Séries entières}
\auteur{quercia}
\date{2010/03/14}
\organisation{exo7}
\contenu{
  \texte{Trouver le rayon de convergence de la série entière $\sum a_nz^n$ :}
\begin{enumerate}
  \item \question{$a_n \to \ell \ne 0$ lorsque $n\to\infty$}
  \item \question{$(a_n)$ est périodique non nulle.\par}
  \item \question{$a_n = \sum_{d\mid n} d^2$.}
  \item \question{$a_n = \frac{n^n}{n!}$.}
  \item \question{$a_{2n} = a^n$, $a_{2n+1} = b^n$,\par $0<a<b$.}
  \item \question{$a_{n^2} = n!$, $a_k = 0$ si $\sqrt k \notin\N$.}
  \item \question{$a_n = (\ln n)^{-\ln n}$.}
  \item \question{$a_n = e^{\sqrt n}$.}
  \item \question{$a_n = \frac{1.4.7\dots(3n-2)}{n!}$.}
  \item \question{$a_n = \frac1{\sqrt n^{\sqrt n}}$.}
  \item \question{$a_n = \left(1+\frac12+\dots+\frac1n\right)^{\ln n}$.}
  \item \question{$a_{n+2} = 2a_{n+1} + a_n$,\par\nobreak $a_0=a_1=1$.}
  \item \question{$a_n = C_{kn}^n$.}
  \item \question{$a_n = e^{(n+1)^2} - e^{(n-1)^2}$.}
  \item \question{$a_n =  \int_{t=0}^1 (1+t^2)^n\,d t$.}
  \item \question{$a_n=\sqrt[n]n-\sqrt[{n+1}]{n+1}$.}
  \item \question{$a_n=\frac{\cos n\theta}{\sqrt n+(-1)^n}$.}
\end{enumerate}
\begin{enumerate}
  \item \reponse{$R=1$.}
  \item \reponse{$R=1$.}
  \item \reponse{$R=1$.}
  \item \reponse{$R=\frac1e$.}
  \item \reponse{$R=\frac1{\sqrt b}$.}
  \item \reponse{$R=1$.}
  \item \reponse{$R=1$.}
  \item \reponse{$R=1$.}
  \item \reponse{$R=\frac13$.}
  \item \reponse{$R=1$.}
  \item \reponse{$R=1$.}
  \item \reponse{$R=\sqrt2-1$.}
  \item \reponse{$R=\frac{(k-1)^{k-1}}{k^k}$.}
  \item \reponse{$R=0$.}
  \item \reponse{$R=\frac12$, $2t\le 1+t^2\le2$.}
  \item \reponse{$R=1$, $a_n\sim\frac{\ln n}{n^2}$.}
  \item \reponse{$R=1$.}
\end{enumerate}
}