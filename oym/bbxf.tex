\uuid{bbxf}
\exo7id{4008}
\titre{Formule de Simpson}
\theme{Exercices de Michel Quercia, Formules de Taylor}
\auteur{quercia}
\date{2010/03/11}
\organisation{exo7}
\contenu{
  \texte{}
\begin{enumerate}
  \item \question{Soit $f:\R \to \R$ de classe $\mathcal{C}^5$, impaire, telle que $f'(0) = 0$
    et : $\forall\ x\in\R,\ |f^{(5)}(x)| \le M$.

    Montrer qu'il existe une constante $\lambda$ telle que :
    $\forall\ x \in \R,\ \left|f(x)-\frac x3f'(x)\right| \le \lambda M|x^5|$.}
  \item \question{Soit $f : {[a,b]} \to \R$ de classe $\mathcal{C}^5$ telle que :
    $$f'(a) = f'(b) = f'\left(\frac {a+b}2\right) = 0,  \qquad
      \text{et}\quad \forall\ x\in{[a,b]},\ |f^{(5)}(x)| \le M.$$
    Montrer que $\bigl|f(b)-f(a)\bigr| \le \frac {M(b-a)^5}{2880}$.}
\end{enumerate}
\begin{enumerate}
  \item \reponse{Formule de Taylor pour $f$ et $f'$ $ \Rightarrow  \lambda = 1/180$.}
\end{enumerate}
}