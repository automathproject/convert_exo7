\uuid{5161}
\titre{***I}
\theme{Valeurs absolues. Partie entière. Inégalités}
\auteur{rouget}
\date{2010/06/30}
\organisation{exo7}
\contenu{
  \texte{}
  \question{Montrer que $\forall n\in\Nn,\;(n\geq3\Rightarrow\sqrt{n}<\sqrt[n]{n!})$.

(commencer par vérifier que pour $k=2,3,...,n$,
on a~:~$(n-k+1)k>n$).}
  \reponse{Soit $n\in\Nn\setminus\{0,1,2\}$.

$$n!^2=\prod_{k=1}^{n}(n+1-k)\prod_{k=1}^{n}k=\prod_{k=1}^{n}k(n+1-k).$$

Maintenant, la fonction $x\mapsto x(n+1-x)$ est strictement croissante sur $[0,\frac{n+1}{2}]$ et
strictement décroissante sur $[\frac{n+1}{2},n+1]$. Puisque $f(1)=f(n)=n$, on en déduit que pour $x\in[2,n-1]$,
$f(x)>n$. Puisque $n\geq3$, on a $n-1\geq2$ et on peut écrire

$$n!^2=n^2\prod_{k=2}^{n-1}k(n+1-k)>n^2\prod_{k=2}^{n-2}n=n^n,$$

et donc,

$$\sqrt[n]{n!}=(n!^2)^{1/(2n)}>(n^n)^{1/2n}=\sqrt{n}.$$}
}