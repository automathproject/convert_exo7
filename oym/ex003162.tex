\uuid{3162}
\titre{$-3$ est-il un carr{\'e} ?}
\theme{}
\auteur{quercia}
\date{2010/03/08}
\organisation{exo7}
\contenu{
  \texte{Soit $p$ un nombre premier impair.}
\begin{enumerate}
  \item \question{Montrer qu'une {\'e}quation du second degr{\'e}~: $x^2 + ax + b = \dot 0$ admet
    une solution dans $\Z/p\Z$ si et seulement si son discriminant~: $a^2 - 4b$
    est un carr{\'e} dans $\Z/p\Z$.}
  \item \question{On suppose que $p\equiv 1 (\mathrm{mod}\, 3)$~: $p=3q+1$.
  \begin{enumerate}}
  \item \question{Montrer qu'il existe $a\in(\Z/p\Z)^*$ tel que $a^q\ne \dot 1$.}
  \item \question{En d{\'e}duire que $-\dot 3$ est un carr{\'e}.}
\end{enumerate}
\begin{enumerate}
  \item \reponse{\begin{enumerate}}
  \item \reponse{Le nombre de solutions de l'{\'e}quation $x^q = \dot 1$ est inf{\'e}rieur
    ou {\'e}gal {\`a}~$q < p-1$.}
  \item \reponse{$\dot 0 = a^{3q}-\dot 1 = (a^q - \dot 1)(a^{2q} + a^q + \dot 1)$
    donc $a^{2q}$ est racine de $x^2 + x + \dot 1 = \dot 0$, de discriminant
    $-\dot 3$.}
\end{enumerate}
}