\uuid{872}
\titre{Exercice 872}
\theme{}
\auteur{bodin}
\date{1998/09/01}
\organisation{exo7}
\contenu{
  \texte{On consid\`ere l'\'equation diff\'erentielle suivante :
$$
(E.D.) \quad y''-4y'+4y = d(x),
$$
o\`u $d$ est une fonction qui sera pr\'ecis\'ee plus loin.}
\begin{enumerate}
  \item \question{R\'esoudre l'\'equation diff\'erentielle homog\`ene (ou sans second membre)
associ\'ee \`a $(E.D.)$.}
  \item \question{Trouver une solution particuli\`ere de $(E.D.)$ lorsque $d(x)=e^{-2x}$
et lorsque $d(x)=e^{2x}$ respectivement.}
  \item \question{Donner la forme g\'en\'erale des solutions de $(E.D)$ lorsque
$$d(x) = \frac{e^{-2x}+e^{2x}}{4}.$$}
\end{enumerate}
\begin{enumerate}
  \item \reponse{L'\'equation caract\'eristique $r^2-4r+4=0$ a une racine (double) $r=2$
donc les solutions de l'\'equation homog\`ene sont les fonctions :
             $$ y(x) = (c_1x+c_2)e^{2x} \hbox{ o\`u } c_1,c_2 \in\R. $$}
  \item \reponse{Pour $d(x) = e^{-2x}$ on peut chercher une solution particuli\`ere de la forme : $ y_1(x)
= ae^{-2x} $ car $-2$ n'est pas racine de l'\'equation
caract\'eristique. On a $y_1'(x)= -2e^{-2x}$ et
$y_1''(x)=4ae^{-2x}$. Par cons\'equent $y_1$ est solution si et
seulement si :
             $$\forall x\in\R\quad  (4a -4(-2a)+4a)e^{-2x} = e^{-2x} $$
donc si et seulement si $ a =\frac{1}{16}$. \\ Pour $d(x) =e^{2x}$
on cherche une solution de la forme $y_2(x)=ax^2e^{2x}$, car $2$
est racine double de l'\'equation caract\'eristique. On a $y_2'(x)
= (2ax+2ax^2)e^{2x}$ et
$y_2''(x)=(2a+4ax+4ax+4ax^2)e^{2x}=(4ax^2+8ax+2a)e^{2x}$. Alors
$y_2$ est solution si et seulement si
$$\forall x\in\R\quad (4ax^2+8ax+2a-4(2ax+2ax^2)+4ax^2)e^{2x} =e^{2x} $$
donc si et seulement si $a=\frac{1}{2}$.}
  \item \reponse{On d\'eduit du principe de superposition que la
fonction $$
y_p(x)=\frac{1}{4}(y_1(x)+y_2(x))=\frac{1}{64}e^{-2x}+\frac{1}{8}x^2e^{2x}$$
est solution de l'\'equation pour le second membre donn\'e dans
cette question, et la forme g\'en\'erale des solutions est alors :
$$ y(x)=(c_1x+c_2)e^{2x}+\frac{1}{64}e^{-2x}+\frac{1}{8}x^2e^{2x} \hbox{ o\`u }c_1,c_2\in\R.$$}
\end{enumerate}
}