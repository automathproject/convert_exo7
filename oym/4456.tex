\uuid{4456}
\titre{Polytechnique MP$^*$ 2000}
\theme{Exercices de Michel Quercia, Séries numérique}
\auteur{quercia}
\date{2010/03/14}
\organisation{exo7}
\contenu{
  \texte{}
  \question{On donne une suite de réels strictement positifs $(a_n)$, décroissante
et de limite nulle. Montrer que la série de terme général
$\frac{a_n-a_{n+1}}{a_n}$ diverge.}
  \reponse{Méthode des rectangles~:
$\sum_{k=0}^n\frac{a_k-a_{k+1}}{a_{k+1}}\ge
 \int_{t=a_{n+1}}^{a_0}\frac{d t}t \to +\infty$ lorsque $k\to\infty$.

Si $a_k\sim a_{k+1}$ la série donnée diverge donc. Sinon, elle diverge
aussi car son terme général ne tend pas vers~$0$.}
}