\uuid{y0NP}
\exo7id{2918}
\titre{Parties ne contenant pas d'{\'e}l{\'e}ments cons{\'e}cutifs}
\theme{Exercices de Michel Quercia, Ensembles finis}
\auteur{quercia}
\date{2010/03/08}
\organisation{exo7}
\contenu{
  \texte{}
\begin{enumerate}
  \item \question{Quel est le nombre de parties {\`a} $p$ {\'e}l{\'e}ments de $\{1, \dots, n\}$
  ne contenant pas d'{\'e}l{\'e}ments cons{\'e}cutifs ?}
  \item \question{Soit $t_n$ le nombre de parties de $\{1, \dots, n\}$ de cardinal
  quelconque sans {\'e}l{\'e}ments cons{\'e}cutifs.
  \begin{enumerate}}
  \item \question{Montrer que $t_{n+2} = t_{n+1} + t_n$, $t_{2n+1} = t_n^2 +
    t_{n-1}^2$, et $t_{2n} = t_n^2 - t_{n-2}^2$.}
  \item \question{Calculer $t_{50}$.}
\end{enumerate}
\begin{enumerate}
  \item \reponse{Comme $\{ x_1-1, \dots, x_p-p\}$ est une partie quelconque de $\{0,
  \dots, n-p\}$, on a $N = C_{n-p+1}^p$.}
  \item \reponse{\begin{enumerate}}
  \item \reponse{32951280099.}
\end{enumerate}
}