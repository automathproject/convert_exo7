\uuid{mPQk}
\exo7id{4903}
\titre{Orthoptique d'une parabole}
\theme{Exercices de Michel Quercia, Coniques}
\auteur{quercia}
\date{2010/03/17}
\organisation{exo7}
\contenu{
  \texte{Soit $P$ une parabole de foyer $F$ et de directrice $D$.
Soit $M \in P$, et $M'$ le point de $P$ tel que les tangentes en $M$ et $M'$
sont orthogonales.}
\begin{enumerate}
  \item \question{Montrer que ces tangentes se coupent au milieu de $[H,H']$.}
  \item \question{Montrer que $M,F,M'$ sont alignés.

En déduire dans un repère $(O,\vec i,\vec j)$ donné, toutes les paraboles
tangentes aux axes de coordonnées.}
\end{enumerate}
\begin{enumerate}
  \item \reponse{Soit $O$ ce milieu. La tangente en $M$ est parallèle à $(FH')$,
             et passe par le milieu de $[F,H]$, donc par le milieu de $[H,H']$.}
  \item \reponse{Calcul d'angles. $\overline{(\vec{MO},\vec{MF})} \equiv
             \overline{(\vec{FH'},\vec{FM'})}$.}
\end{enumerate}
}