\uuid{XTqt}
\exo7id{3586}
\titre{Puissances de $A$}
\theme{Exercices de Michel Quercia, Réductions des endomorphismes}
\auteur{quercia}
\date{2010/03/10}
\organisation{exo7}
\contenu{
  \texte{Soit $A \in \mathcal{M}_{3}(\R)$ ayant pour valeurs propres $1,-2,2$, et $n\in \N$.}
\begin{enumerate}
  \item \question{Montrer que $A^n$ peut s'écrire sous la forme :
    $A^n = \alpha_n A^2 + \beta_n A + \gamma_n I$
    avec $\alpha_n,\beta_n,\gamma_n \in \R$.}
  \item \question{On considère le polynôme $P = \alpha_n X^2 + \beta_n X + \gamma_n$.
    Montrer que : $P(1) = 1$, $P(2) = 2^n$, $P(-2) = (-2)^n$.}
  \item \question{En déduire les coefficients $\alpha_n,\beta_n,\gamma_n$.}
\end{enumerate}
\begin{enumerate}

\end{enumerate}
}