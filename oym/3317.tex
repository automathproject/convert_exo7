\uuid{3317}
\titre{Essai de bases}
\theme{Exercices de Michel Quercia, Espaces vectoriels de dimension finie}
\auteur{quercia}
\date{2010/03/09}
\organisation{exo7}
\contenu{
  \texte{}
  \question{Montrer que dans $\R^3$, les trois vecteurs $\vec a = (1,0,1)$,
$\vec b = (-1,-1,2)$ et $\vec c = (-2,1,-2)$ forment une base, et calculer
les coordonnées dans cette base d'un vecteur $\vec x = (x,y,z)$.}
  \reponse{$$\left\{\begin{array}{lllllll} 
x'  &= &     & & 2y &+& z  \cr
                 3y' &= & -x  & &    &+& z  \cr
                 3z' &= & -x  &+& 3y &+& z. \cr
       \end{array}\right.$$}
}