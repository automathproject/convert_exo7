\uuid{1849}
\titre{Exercice 1849}
\theme{}
\auteur{gourio}
\date{2001/09/01}
\organisation{exo7}
\contenu{
  \texte{}
  \question{Soit $ f:{\Rr}^{2}\rightarrow {\Rr}$ une application $ C^{1 }$
homog\`{e}ne de degr\'{e} $ s>0$, i.e. telle que :
$$\forall \lambda \in {\Rr}^{+*},\forall x\in {\Rr}^{2},f\left( \lambda
x\right) =\lambda ^{s}f(x). $$
Montrer que les d\'{e}riv\'{e}es partielles de $ f$ sont homog\`{e}nes de
degr\'{e} $ s-1$ et :
$$sf(x)=x_{1}\frac{\partial f}{\partial x_{1}}(x)+x_{2}\frac{\partial f}{%
\partial x_{2}}(x). $$}
  \reponse{}
}