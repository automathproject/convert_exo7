\exo7id{7404}
\titre{Exercice 7404}
\theme{}
\auteur{mourougane}
\date{2021/08/10}
\organisation{exo7}
\contenu{
  \texte{Alice et Bernard décident d'utiliser l'algorithme d'El Gamal. Il utilise le corps $\mathbb{F}_{13}$ avec l'élément $G=2$.}
\begin{enumerate}
  \item \question{Quels sont les ordres possibles des éléments de $\mathbb{F}_{19}^\times$. Déterminer l'ordre de $2$ dans $\mathbb{F}_{19}^\times$.}
  \item \question{Bernard choisit sa clé privée $c=3$. Déterminer sa clé publique $C=G^c$.}
  \item \question{Alice choisit une clé temporaire privée
	$d=7$. Quelle est sa clé publique $D$ ? Elle souhaite envoyer le
	message $m=11$. Elle le chiffre en utilisant la clé publique $C$ de
	Bernard par $(M_1,M_2)=(D,mC^d)$. Expliciter ce message chiffré.}
  \item \question{Comment Bernard retrouve-t-il le message $m$ ?}
  \item \question{Dans un second envoi, Bernard reçoit $(8,3)$. Quel est le message
	$m$ envoyé cette fois par Alice ? Quelle clé privée a-t-elle utilisé
	cette fois ?}
\end{enumerate}
\begin{enumerate}
  \item \reponse{Le groupe $\mathbb{F}_{19}^\times$ est d'ordre 18, donc les ordres
possibles des éléments sont les diviseurs de 18 (et ce sont
exactement ceux-là car $\mathbb{F}_{19}^\times$ est cyclique), c'est-à-dire
1,2,3,6,9,18. Or $2^1=2$, $2^2=4$, $2^3=8$, $2^6=7$, $2^9=-1$, donc
2 est bien un générateur de $\mathbb{F}_{19}^\times$.}
  \item \reponse{$C=2^3=8$.}
  \item \reponse{$D=2^7=14$. $C^d=8^7=2^{21}=2^3=8$. $mC^d=11\times8=12$.
Finalement, $(M_1,M_2)=(14,12)$.}
  \item \reponse{Il utilise la formule : $m=M_2(M_1^c)^{-1}$. On vérifie en effet
que $12\times 14^{-3}=12\times(-5)^{-3}=12\times (-7)=-8=11$.}
  \item \reponse{Le message envoyé est $3\times8^{-3}=3\times 2^{-9}=-3=16$.
Pour crypter $m=16$, Alice a choisi une clé privée $k$ et a envoyé le
message $(2^k,\,16\times 8^k)$. Or on sait que $2^3=8$ et que le
logarithme discret est un isomorphisme, donc Alice a choisi cette
fois la clé privée 3. Et effectivement, on retrouve bien
$(2^3,\,16\times 8^3)=(8,3)$.}
\end{enumerate}
}