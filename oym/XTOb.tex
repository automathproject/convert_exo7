\uuid{XTOb}
\exo7id{4601}
\titre{Calcul de somme}
\theme{Exercices de Michel Quercia, Séries entières}
\auteur{quercia}
\date{2010/03/14}
\organisation{exo7}
\contenu{
  \texte{On pose $f(x) = \sum_{n=0}^\infty \frac{n!\,x^{2n+1}}{1.3.5\dots(2n+1)}$.}
\begin{enumerate}
  \item \question{Déterminer le rayon de convergence.}
  \item \question{\'Etudier la convergence aux bornes de l'intervalle de convergence.}
  \item \question{Calculer $f(x)$.}
\end{enumerate}
\begin{enumerate}
  \item \reponse{$R=\sqrt2$.}
  \item \reponse{Stirling $ \Rightarrow  a_n{\sqrt2\,}^{2n+1} \sim \frac2{\sqrt{\pi n}}
              \Rightarrow $ DV.}
  \item \reponse{$(x^2-2)y' + xy + 2 = 0  \Rightarrow 
             f(x) = \frac{2\Arcsin (x/\sqrt2\,)}{\sqrt{2-x^2}}$.}
\end{enumerate}
}