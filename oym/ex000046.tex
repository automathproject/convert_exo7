\exo7id{46}
\titre{Exercice 46}
\theme{}
\auteur{cousquer}
\date{2003/10/01}
\organisation{exo7}
\contenu{
  \texte{}
\begin{enumerate}
  \item \question{Montrer que, pour tout $n\in\mathbb{N}^*$ et tout nombre $z\in\mathbb{C}$, on a:
$$\left(z-1\right)\left(1+z+z^2+...+z^{n-1}\right)=z^n-1,$$
et en déduire que, si $z\neq 1$, on a:
$$1+z+z^2+...+z^{n-1}=\frac{z^n-1}{z-1}.$$}
  \item \question{\label{Q2} Vérifier que pour tout $x\in\mathbb{R}$ , on a
$  
\exp(ix)-1=2i\exp\left(\frac{ix}{2}\right)\sin\left(\frac{x}{2}\right).$}
  \item \question{\label{Q3} Soit $n\in\mathbb{N}^*$. Calculer pour tout $x\in\mathbb{R}$ la somme:
$$Z_n=1+\exp(ix)+\exp(2ix)+...+\exp((n-1)ix),$$ 
et en déduire les valeurs de
\begin{eqnarray*}
X_n&=&1+\cos(x)+\cos(2x)+...+\cos((n-1)x)\\
Y_n&=&\sin(x)+\sin(2x)+...+\sin((n-1)x).
\end{eqnarray*}}
\end{enumerate}
\begin{enumerate}

\end{enumerate}
}