\uuid{h4hE}
\exo7id{3058}
\titre{Ordre sur $\N^{\N}$}
\theme{Exercices de Michel Quercia, Propriétés de $\Nn$}
\auteur{quercia}
\date{2010/03/08}
\organisation{exo7}
\contenu{
  \texte{Soit $E = {\N}^{\N}$. Pour $f,g \in E$ avec $f\ne g$,
on note $n_{f,g} = \min\{k$ tq $f(k) \ne g(k)\}$.

On ordonne $E$ par :
$$\forall\ f,g\in E,\
  f \ll g \iff (f = g) \text{ ou }\bigl(f(n_{f,g}) < g(n_{f,g})\bigr).$$}
\begin{enumerate}
  \item \question{Montrer que c'est une relation d'ordre total.}
  \item \question{Montrer que toute partie de $E$ non vide admet une borne inf{\'e}rieure
      et toute partie de $E$ non vide et major{\'e}e admet une borne sup{\'e}rieure.}
\end{enumerate}
\begin{enumerate}

\end{enumerate}
}