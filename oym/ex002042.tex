\exo7id{2042}
\titre{Exercice 2042}
\theme{}
\auteur{liousse}
\date{2003/10/01}
\organisation{exo7}
\contenu{
  \texte{Tout ce probl\`eme se situe dans l'espace euclidien tridimensionnel muni d'un rep\`ere
 orthonorm\'e direct $\mathcal R 
= (0,\overrightarrow{i}, \overrightarrow{j},\overrightarrow{k})$.}
\begin{enumerate}
  \item \question{On consid\`ere les
deux droites $d$ et $D$ donn\'ees par les syst\`emes d'\'equations 
cart\'esiennes suivant :

$d \left\lbrace 
\begin{array}{ll} x +y-3z& = 0\cr y +z& =0 \end{array}\right.$\hfil et \hfil
$D\left\lbrace \begin{array}{ll}x -1& = 0 \cr y - z -1& =0 \end{array}\right.$  

\begin{enumerate}}
  \item \question{\begin{enumerate}}
  \item \question{Donner un point et un vecteur directeur de $d$.
 Donner un point et un vecteur directeur  de $D$.}
  \item \question{Dire si les droites $d$ et $D$ sont parall\`eles, s\'ecantes 
 ou non coplanaires.}
  \item \question{Justifier l'existence de deux plans parall\`eles  (en donnant pour chacun de ces 
deux plans un point et deux vecteurs directeurs) tels que $d$ est contenue dans l'un 
 et $D$ est contenue dans l'autre.}
\end{enumerate}
\begin{enumerate}

\end{enumerate}
}