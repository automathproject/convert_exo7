\uuid{3217}
\titre{$X^2-2X\cos\theta+1$ divise $X^{2n}-2X^n\cos(n\theta)+1$}
\theme{Exercices de Michel Quercia, Racines de polynômes}
\auteur{quercia}
\date{2010/03/08}
\organisation{exo7}
\contenu{
  \texte{}
  \question{Montrer que $X^2-2X\cos\theta+1$ divise $X^{2n}-2X^n\cos n\theta+1$.
Pour $\sin\theta \ne 0$, chercher le quotient.}
  \reponse{$X^{2n}-2X^n\cos n\theta+1 = (X^n-e^{in\theta})(X^n-e^{-in\theta})$.

$Q$
$=\Bigl(\sum_{k=0}^{n-1} X^ke^{i(n-1-k)\theta}\Bigr)
  \Bigl(\sum_{\ell=0}^{n-1} X^le^{-i(n-1-\ell)\theta}\Bigr)$ \par
$= \sum_{k=0}^{n-1} X^k\Bigl({\sum_{p=0}^k e^{i(k-2p)\theta}}\Bigr)
 + \sum_{k=n}^{2n-2} X^k\Bigl({\sum_{p=k-n+1}^{n-1} e^{i(k-2p)\theta}}\Bigr)$ \par
$= \sum_{k=0}^{n-1} X^k \frac {\sin(k+1)\theta}{\sin\theta}
 + \sum_{k=n}^{2n-2} X^k \frac {\sin(2n-k-1)\theta}{\sin\theta}$. \par}
}