\uuid{251}
\titre{Exercice 251}
\theme{Arithmétique dans $\Zz$, Divisibilité, division euclidienne}
\auteur{bodin}
\date{1998/09/01}
\organisation{exo7}
\contenu{
  \texte{}
  \question{Sachant que l'on a $96842=256\times 375 + 842$, d\'eterminer, sans faire
la division, le reste de la division du nombre $96842$ par chacun des nombres
$256$ et $375$.}
  \reponse{La seule chose \`a voir est que pour une division euclidienne le reste doit \^etre plus petit que le quotient.
Donc les divisions euclidiennes s'\'ecrivent :
$96842 = 256 \times 378 + 74$ et $96842 = 258 \times 375 + 92$.}
}