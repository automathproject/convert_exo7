\uuid{2806}
\titre{Exercice 2806}
\theme{}
\auteur{burnol}
\date{2009/12/15}
\organisation{exo7}
\contenu{
  \texte{}
  \question{Le Laplacien $\Delta = \frac{\partial^2}{\partial x^2}
+\frac{\partial^2}{\partial y^2}$ est un opérateur
différentiel qui joue un rôle important en analyse
complexe. Soit $f(z) = u(x,y) + i\, v(x,y)$ une fonction holomorphe sur
un ouvert du plan complexe. On sait que $f$, donc $u$ et
$v$, admettent des dérivées partielles de tous les
ordres. En
utilisant les équations de 
Cauchy-Riemann, montrer que $u$ et $v$ vérifient l'équation de Laplace:
\[ \frac{\partial^2 u}{\partial x^2}
+\frac{\partial^2 u}{\partial y^2} =0 \qquad
\frac{\partial^2 v}{\partial x^2} +\frac{\partial^2
v}{\partial y^2} =0\] On dit d'une fonction vérifiant
l'équation de Laplace qu'elle est harmonique. La fonction
holomorphe $f = u + i v$ est aussi une fonction harmonique
puisque $\Delta(f) = \Delta(u) + i \Delta(v) = 0$.}
  \reponse{On a les \'equations de Cauchy-Riemann $\frac{\partial u}{\partial x} = \frac{\partial v}{\partial y}$
et $\frac{\partial u}{\partial y} = -\frac{\partial v}{\partial x}$. D'o\`u :
$$\Delta u = \frac{\partial }{\partial x}\frac{\partial u}{\partial x}+ \frac{\partial }{\partial y}\frac{\partial u}{\partial y}=
\frac{\partial }{\partial x}\frac{\partial v}{\partial y}- \frac{\partial }{\partial y}\frac{\partial v}{\partial x}=0$$
par le th\'eor\`eme de Schwarz. On proc\`ede de la m\^eme fa\c{c}on pour $v$ pour en d\'eduire
qu'une fonction holomorphe est harmonique.}
}