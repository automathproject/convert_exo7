\uuid{3365}
\titre{Opérations par blocs}
\theme{}
\auteur{quercia}
\date{2010/03/09}
\organisation{exo7}
\contenu{
  \texte{}
\begin{enumerate}
  \item \question{Soient $A_1 \in \mathcal{M}_{n,p_1}(K)$, $A_2 \in \mathcal{M}_{n,p_2}(K)$,
    $B_1 \in \mathcal{M}_{p_1,q}(K)$, $B_2 \in \mathcal{M}_{p_2,q}(K)$.

    On pose $A = \begin{pmatrix} A_1 &A_2 \end{pmatrix} \in \mathcal{M}_{n,p_1+p_2}(K)$
    et $B = \begin{pmatrix} B_1 \cr B_2 \cr \end{pmatrix} \in \mathcal{M}_{p_1+p_2,q}(K)$.
    Montrer que $AB = A_1B_1 + A_2B_2$.}
  \item \question{Soit $M = \begin{pmatrix}A & B \cr 0 & C \cr\end{pmatrix}$
    où $A,B,0,C$ sont des matrices de tailles $p\times p$, $p \times q$,
    $q\times p$, $q \times q$ (matrice triangulaire par blocs).
    Montrer que $M$ est inversible si et seulement si $A$ et $C$ le sont.
    Le cas échéant, donner $M^{-1}$ sous la même forme.}
  \item \question{En déduire une nouvelle démonstration de la propriété :
    {\it L'inverse d'une matrice triangulaire est triangulaire.}}
\end{enumerate}
\begin{enumerate}

\end{enumerate}
}