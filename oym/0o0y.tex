\uuid{0o0y}
\exo7id{2915}
\titre{Formule du crible}
\theme{Exercices de Michel Quercia, Ensembles finis}
\auteur{quercia}
\date{2010/03/08}
\organisation{exo7}
\contenu{
  \texte{Soient $A_1, \dots, A_n$\ $n$ ensembles finis.}
\begin{enumerate}
  \item \question{\begin{enumerate}}
  \item \question{Calculer $\mathrm{Card}\,(A_1\cup A_2 \cup A_3)$ et $\mathrm{Card}\,(A_1\cup A_2\cup A_3\cup A_4)$.}
  \item \question{Sugg{\'e}rer une formule pour $\mathrm{Card}\,(A_1\cup \dots \cup A_n)$.
    \label{devine}}
\end{enumerate}
\begin{enumerate}
  \item \reponse{\begin{enumerate}}
  \item \reponse{$\sum_{k=1}^n (-1)^kC_n^k(n-k)^p$.}
  \item \reponse{$\sum_{k=1}^n \frac{(-1)^kn!}{k!}$.}
\end{enumerate}
}