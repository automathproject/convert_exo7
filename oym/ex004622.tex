\exo7id{4622}
\titre{Développements}
\theme{}
\auteur{quercia}
\date{2010/03/14}
\organisation{exo7}
\contenu{
  \texte{Calculer le développement des fonctions $f$ $2\pi$-périodiques telles que :}
\begin{enumerate}
  \item \question{$f(x) = \pi-|x|$ sur $]-\pi,\pi[$.}
  \item \question{$f(x) = \pi-x$ sur $]0,2\pi[$.}
  \item \question{$f(x) = x^2$ sur $]0,2\pi[$.}
  \item \question{$f(x) = \max(0,\sin x)$.}
  \item \question{$f(x) = |\sin x|^3$.}
\end{enumerate}
\begin{enumerate}
  \item \reponse{$a_0 = \pi$, $a_{2p} = 0$, $a_{2p+1} = \frac4{\pi(2p+1)^2}$,
             $b_n = 0$.}
  \item \reponse{$a_n = 0$, $b_n = \frac2n$.}
  \item \reponse{$a_0 = \frac{8\pi^2}3$, $a_n = \frac4{n^2}$,
             $b_n = -\frac{4\pi}n$.}
  \item \reponse{$a_0 = \frac 2\pi$, $a_{2p} = \frac{-2}{\pi(4p^2-1)}$,
             $a_{2p+1} = 0$, $b_1 = \frac 12$, $b_p = 0$.}
  \item \reponse{$a_{2p} = \frac{24}{\pi(4p^2-1)(4p^2-9)}$, $a_{2p+1} = 0$,
             $b_p = 0$.}
\end{enumerate}
}