\uuid{4908}
\titre{Longueur minimale d'une corde normale, Ensi Physique 93}
\theme{}
\auteur{quercia}
\date{2010/03/17}
\organisation{exo7}
\contenu{
  \texte{}
  \question{Soit $\cal P$ une parabole de paramètre $p$ et $A \in \cal P$. Soit $B$ le
point où la normale à $\cal P$ en $A$ recoupe $\cal P$. Déterminer la longueur
minimale de $AB$.}
  \reponse{$A:(t^2/2p,t)$, $B:(u^2/2p,u)$ avec $t(t+u) = -2p^2$.
         $AB$ est minimal pour $t^2=2p^2$ et vaut alors $3p\sqrt3$.}
}