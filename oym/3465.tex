\uuid{3465}
\titre{Matrice à trou}
\theme{Exercices de Michel Quercia, Rang de matrices}
\auteur{quercia}
\date{2010/03/10}
\organisation{exo7}
\contenu{
  \texte{}
  \question{Soient
$A \in \mathcal{M}_{3,2}(K),\ B \in \mathcal{M}_{2,2}(K),\ C \in \mathcal{M}_{2,3}(K)$ telles que
$ABC = \begin{pmatrix} 1&1&\phantom-2 \cr -2&x&1 \cr 1&-2&1 \end{pmatrix}$.
Trouver $x$.}
  \reponse{$\mathrm{rg} ABC \le 2  \Rightarrow  x = 13$.
         $M = \begin{pmatrix} 1&0 \cr 0&1 \cr \frac35 &-\frac 15 \cr \end{pmatrix}
              \begin{pmatrix} 1&0 \cr 0&1 \cr \end{pmatrix}
              \begin{pmatrix} 1&1&2 \cr -2&13&1 \end{pmatrix}$.}
}