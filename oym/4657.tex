\uuid{4657}
\titre{Formule sommatoire de Poisson}
\theme{Exercices de Michel Quercia, Séries de Fourier}
\auteur{quercia}
\date{2010/03/14}
\organisation{exo7}
\contenu{
  \texte{}
  \question{Soit $f : \R \to \C$ de classe $\mathcal{C}^1$. On suppose qu'il existe $a>1$ tel que
$f(x) =  O(1/|x|^a)$ et $f'(x) =  O(1/|x|^a)$ lorsque $|x|\to\infty$, et on pose
$F(x) = \sum_{n\in \Z}f(x+2n\pi)$.

Montrer que $F$ est bien définie, $\mathcal{C}^1$ et $2\pi$-périodique.
En déduire la formule sommatoire de Poisson~:
$$\sum_{n\in\Z}f(2n\pi) = \frac1{\sqrt{2\pi}} \sum_{n\in\Z} \hat f(n).$$}
  \reponse{}
}