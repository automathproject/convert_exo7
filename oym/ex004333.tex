\exo7id{4333}
\titre{$\tan^nt$, Ensi Physique P 94}
\theme{}
\auteur{quercia}
\date{2010/03/12}
\organisation{exo7}
\contenu{
  \texte{On pose $I_n =  \int_0^{\pi/4} \tan^nt\,d t$.\par}
\begin{enumerate}
  \item \question{Montrer que $I_n \to 0$  lorsque $n\to\infty$.}
  \item \question{Calculer $I_n$ en fonction de~$n$.}
  \item \question{Que peut-on en déduire~?}
\end{enumerate}
\begin{enumerate}
  \item \reponse{$I_n+I_{n+2} = \frac1{n+1}$.}
  \item \reponse{$I_{2k} = \frac1{2k-1} - \frac1{2k-3} + \dots + \frac{(-1)^{k-1}}1 + (-1)^k\frac\pi4$,\par
         $I_{2k+1} = \frac1{2k} - \frac1{2k-2} + \dots + \frac{(-1)^{k-1}}2 - (-1)^k\ln\sqrt2$.}
  \item \reponse{$\frac11-\frac13+\frac15 - \dots = \frac\pi4$ et $\frac11-\frac12+\frac13-\dots=\ln 2$.}
\end{enumerate}
}