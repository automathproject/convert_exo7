\uuid{2349}
\titre{Exercice 2349}
\theme{}
\auteur{queffelec}
\date{2003/10/01}
\organisation{exo7}
\contenu{
  \texte{}
\begin{enumerate}
  \item \question{Montrer que 
$||f||_\infty= \sup_{0\leq x\leq
1}|f(x)|$ et  $||f||_1=\int_0^1|f(t)|\ dt$ sont deux normes 
sur $C([0,1],{\Rr})$. Sont-elles \'equivalentes ?}
  \item \question{Les deux m\'etriques associ\'ees sont-elles topologiquement \'equivalentes ?}
\end{enumerate}
\begin{enumerate}
  \item \reponse{$\| f \|_1 = \int_0^1|f(t)|dt \le \int_0^1 \|f\|_\infty dt \le \|f\|_\infty$.
Donc $\| f \|_1 \le \|f\|_\infty$ Par contre il n'existe aucune constante $C>0$ tel que $\|f\|_\infty \le C \| f \|_1$ pour tout $f$. Pour montrer ceci par l'absurde, supposons qu'il existe une constante $C>0$ telle que
$\|f\|_\infty \le C \| f \|_1$ pour tout $f$ de $ C([0,1],{\Rr})$.
Regardons les fonctions $f_k$ définies par $f_k(x) = 2k(1-kx)$ si
$x\in[0,\frac1k]$ et $f_k(x)=0$ si $x > \frac1k$. 
Alors $f_k \in C([0,1],{\Rr})$ et  $\| f_k \|_\infty = 2k$ alors que $\| f_k \|_1 = 1$. On obtient $2k \le C.1$ ce qui est contradicoire pour
$k$ assez grand. Cela prouve que les normes ne sont pas équivalentes.}
  \item \reponse{Comme les métriques sont définies par des normes et que les normes ne sont pas équivalentes alors les métriques ne définissent pas la même topologie.}
\end{enumerate}
}