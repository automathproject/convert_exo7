\exo7id{1419}
\titre{Exercice 1419}
\theme{}
\auteur{ortiz}
\date{1999/04/01}
\organisation{exo7}
\contenu{
  \texte{${\cal A}_4$ d\'esigne le groupe des permutations
paires sur l'ensemble $E=\left\{ 1,2,3,4\right\}
.$}
\begin{enumerate}
  \item \question{Quels sont les ordres des \'el\'ements de ${\cal A}_4$ ? En d\'eduire
la liste de ces \'el\'ements sous forme d\'ecompos\'ee en produit de cycles
\`a supports disjoints.}
  \item \question{Montrer que les permutations $s=(1\;2)(3\;4)$ et $r=(1\;2\;3)$
engendrent ${\cal A}_4.$}
  \item \question{Montrer que ${\cal A}_4$ admet un unique sous-groupe $H$ d'ordre 4 (on
examinera d'abord les ordres des \'el\'ements d'un
tel sous-groupe) et que ce sous-groupe est un
sous-groupe distingu\'e de ${\cal A}_4$.}
\end{enumerate}
\begin{enumerate}

\end{enumerate}
}