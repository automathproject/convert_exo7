\uuid{200}
\titre{Exercice 200}
\theme{Injection, surjection, bijection}
\auteur{bodin}
\date{1998/09/01}
\organisation{exo7}
\contenu{
  \texte{}
  \question{Soit $ f:\Rr \rightarrow \Cc, \ t\mapsto e^{it}$. 
Changer les ensembles de départ et d'arrivée afin que
(la restriction de) $f$ devienne bijective.}
  \reponse{Considérons la restriction suivante de $f$ : $f_|:[0,2\pi[ \longrightarrow \mathbb{U}$, 
$t\mapsto e^{it}$. Montrons que cette nouvelle application $f_|$ est bijective. Ici $\mathbb{U}$
est le cercle unit\'e de $\Cc$ donn\'e par l'\'equation $(|z|=1)$.
\begin{itemize}
    \item[$\bullet$] $f_|$ est surjective car tout nombre complexe de $\mathbb{U}$ s'\'ecrit
sous la forme polaire $e^{i\theta}$, et l'on peut choisir $\theta
\in [0,2\pi[$.

    \item[$\bullet$] $f_|$ est injective :
\begin{align*}
f_|(t) = f_|(t') &\Leftrightarrow e^{it}=e^{it'}\\
&\Leftrightarrow t=t' +2k\pi \text{ avec } k\in \Zz\\
&\Leftrightarrow t=t' \text{ car } t,t'\in[0,2\pi[ \text{ et donc }k=0.\\
\end{align*}
\end{itemize}
En conclusion $f_|$ est injective et surjective donc bijective.}
}