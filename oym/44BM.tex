\uuid{44BM}
\exo7id{3727}
\titre{$1/(\lambda_i+\lambda_j)$}
\theme{Exercices de Michel Quercia, Formes quadratiques}
\auteur{quercia}
\date{2010/03/11}
\organisation{exo7}
\contenu{
  \texte{}
\begin{enumerate}
  \item \question{Soit ${f_1,\dots,f_n} : I \to \R$ des fonctions continues de carrés
intégrables sur l'intervalle~$I$.
On pose $a_{i,j} =  \int_I f_if_j$. Montrer que la matrice $(a_{i,j})$
est définie positive ssi la famille $(f_1,\dots,f_n)$ est libre.}
  \item \question{En déduire que si $\lambda_1$,\dots,$\lambda_n$ sont des réels
strictement positifs distincts alors la matrice de terme général
$1/(\lambda_i+\lambda_j)$ est définie positive.}
\end{enumerate}
\begin{enumerate}

\end{enumerate}
}