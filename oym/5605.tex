\uuid{5605}
\titre{***}
\theme{Algèbre linéaire II}
\auteur{rouget}
\date{2010/10/16}
\organisation{exo7}
\contenu{
  \texte{}
  \question{Soient $n$ un entier naturel non nul puis $A\in\mathcal{M}_n(\Kk)$.
Soit $f$ l'endomorphisme de $M_n(\Kk)$ qui à une matrice $X$ associe $AX+XA$. Calculer $\text{Tr}(f)$.}
  \reponse{On note $\mathcal{B}=(E_{i,j})_{1\leqslant i,j\leqslant n}$ la base canonique de $\mathcal{M}_n(\Kk)$.

$\text{Tr}f=\sum_{1\leqslant i,j\leqslant n}^{}\alpha_{i,j}$ où $\alpha_{i,j}$ désigne la $(i,j)$-ème coordonnée de $f(E_{i,j})=AE_{i,j}+E_{i,j}A$ dans la base $\mathcal{B}$.

Mais pour $(i,j)\in\llbracket1,n\rrbracket^2$ donné, 

\begin{center}
$AE_{i,j}=\sum_{1\leqslant k,l\leqslant n}^{}a_{k,l}E_{k,l}E_{i,j}=\sum_{k=1}^{n}a_{k,i}E_{k,j}$
\end{center}

et de même,

\begin{center}
$E_{i,j}A=\sum_{1\leqslant k,l\leqslant n}^{}a_{k,l}E_{i,j}E_{k,l}=\sum_{l=1}^{n}a_{j,l}E_{i,l}$.
\end{center}

Donc $\forall(i,j)\in\llbracket1,n\rrbracket^2$, $\alpha_{i,j}=a_{i,i}+a_{j,j}$ puis

\begin{center}
$\text{Tr}f=\sum_{1\leqslant i,j\leqslant n}^{}(a_{i,i}+a_{j,j})=2\sum_{1\leqslant i,j\leqslant n}^{}a_{i,i}=2\sum_{j=1}^{n}\left(\sum_{i=1}^{n}a_{i,i}\right)=2\sum_{j=1}^{n}\text{Tr}A=2n\text{Tr}A$.
\end{center}

\begin{center}
\shadowbox{
$\text{Tr}f=2n\text{Tr}A$.
}
\end{center}}
}