\uuid{773}
\titre{Exercice 773}
\theme{}
\auteur{bodin}
\date{1998/09/01}
\organisation{exo7}
\contenu{
  \texte{}
  \question{D\'emontrer les in\'egalit\'es : 
$$ x- \frac{x^2}{2} < \ln (1+x) \text{ pour } x>0 \quad \text{et} \quad
1+x \le e^x \text{ pour tout $x$ r\'eel}.$$}
  \reponse{\begin{enumerate}
    \item Soit $f(x) = \ln(1+x)-x+x^2/2$ alors $f'(x)=\frac{1}{1+x}
-1+x=\frac{x^2}{1+x} >0$. Donc $f$ est strictement croissante sur $[0,+\infty[$ et comme $f(0)=0$ alors $f(x) > f(0)=0$ pour $x>0$. Ce qui donne l'inégalité recherchée.
  \item De même avec $g(x) = e^x-x-1$, $g'(x)=e^x-1$.
Sur $[0,+\infty[$ $g'(x) \geq 0$ et $g$ est croissante
sur $]-\infty,0]$, $g'(x)\leq 0$ et $g$ est décroissante. Comme
 $g(0)=0$ alors pour tout $x\in\Rr$ $g(x)\geq 0$.
\end{enumerate}}
}