\uuid{U2EP}
\exo7id{3337}
\titre{Thms de factorisation}
\theme{Exercices de Michel Quercia, Applications linéaires en dimension finie}
\auteur{quercia}
\date{2010/03/09}
\organisation{exo7}
\contenu{
  \texte{Soient $E,F,G$ trois $ K$-ev avec $\dim(G)$ finie.}
\begin{enumerate}
  \item \question{Soient $u \in \mathcal{L}(F,E)$ et $v \in \mathcal{L}(G,E)$.
    Montrer qu'il existe $h \in \mathcal{L}({G,F})$
    tel que $v = u\circ h$ si et seulement si $\Im v \subset \Im u$.}
  \item \question{Soient $u \in \mathcal{L}(E,F)$ et $v \in \mathcal{L}(E,G)$.
    Montrer qu'il existe $h \in \mathcal{L}({G,F})$
    tel que $u = h\circ v$ si et seulement si $\mathrm{Ker} v \subset \mathrm{Ker} u$.}
\end{enumerate}
\begin{enumerate}

\end{enumerate}
}