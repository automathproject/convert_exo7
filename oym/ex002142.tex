\uuid{2142}
\titre{Exercice 2142}
\theme{}
\auteur{debes}
\date{2008/02/12}
\organisation{exo7}
\contenu{
  \texte{}
  \question{\label{ex:le7}
Soit $G$ un groupe et $K\subset H\subset G$ deux sous-groupes. On
suppose que $H$ est distingu\'e  dans $G$ et que $K$ est caract\'eristique dans $H$ (i.e.
stable par tout automorphisme de $H$). Montrer qu'alors $K$ est distingu\'e dans
$G$.

Donner un exemple de groupe $G$ et de deux sous-groupes $K\subset H \subset G$, $H$ \'etant
distingu\'e dans $G$ et $K$ \'etant distingu\'e dans $H$, mais $K$ n'\'etant pas distingu\'e dans
$G$.}
  \reponse{Pour tout $g\in G$, la conjugaison $c_g:G\rightarrow G$ par $g$ induit un automorphisme de
$H$ si $H$ est distingu\'e dans $G$. Si de plus $K$ est caract\'eristique dans $H$, alors
$K$ est stable par $c_g$. D'o\`u $K$ est alors distingu\'e dans $G$.
\smallskip

Le sous-ensemble $V_4$ du groupe sym\'etrique $S_4$ consistant en l'identit\'e et
les trois produits de transpositions disjointes: $(1 \hskip 2pt 2)(3 \hskip 2pt 4)$, $(1
\hskip 2pt 3)(2 \hskip 2pt 4)$ et $(1 \hskip 2pt 4)(2 \hskip 2pt 3)$ est un sous-groupe
(v\'erification imm\'ediate) qui est distingu\'e: cela r\'esulte de la formule $g (i \hskip
2pt j)(k \hskip 2pt l) g^{-1} = (g(i) \hskip 2pt g(j))(g(k) \hskip 2pt g(l))$ pour
$i,j,k,l \in \{1,2,3,4\}$ distincts. Le sous-groupe $K$ (d'ordre $2$) engendr\'e par $(1
\hskip 2pt 2)(3 \hskip 2pt 4)$ est distingu\'e dans $V_4$ (car $V_4$ est ab\'elien). Mais $K$
n'est pas distingu\'e dans $S_4$ (comme le montre encore la formule pr\'ec\'edente).}
}