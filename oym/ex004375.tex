\uuid{4375}
\titre{Lemme de Lebesgue, Centrale MP 2004}
\theme{}
\auteur{quercia}
\date{2010/03/12}
\organisation{exo7}
\contenu{
  \texte{Soit $f$ continue par morceaux définie sur $\R$, à valeurs dans $\C$.}
\begin{enumerate}
  \item \question{Soient $a,b\in\R$.
Montrer que $ \int_{t=a}^b f(t) \cos(nt)\,d t\to0$ lorsque $n\to\infty$.}
  \item \question{On suppose que $f$ est intégrable sur $]0,+\infty[$.
Soit $u_n=  \int_{t=0}^{n\pi}  \sin^2(nt) f(t)\,d t $.
Montrer que $(u_n)_{n \in \N}$ admet une limite quand $n\to\infty$ et la préciser.}
\end{enumerate}
\begin{enumerate}
  \item \reponse{$\sin^2(nt) = \frac{1-\cos(2nt)}2$, donc il suffit d'étudier
$I_n =  \int_{t=0}^{n\pi}\cos(2nt)f(t)\,d t$.

Posons $I_{n,p} =  \int_{t=0}^{\min(n,p)\pi}\cos(2nt)f(t)\,d t$~:
on a $|I_n-I_{n,p}| \le
 \int_{t=p\pi}^{+\infty}|f(t)|\,d t$, quantité indépendante de~$n$ et
tendant vers $0$ quand $p\to\infty$ donc le théorème d'interversion
des limites s'applique~:

$\lim_{n\to\infty}I_n
= \lim_{n\to\infty}\lim_{p\to\infty}I_{n,p}
= \lim_{p\to\infty}\lim_{n\to\infty}I_{n,p} = 0$.
On en déduit $u_n \to \frac12 \int_{t=0}^{+\infty} f(t)\,d t$ lorsque $n\to\infty$.}
\end{enumerate}
}