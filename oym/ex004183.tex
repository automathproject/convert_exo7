\uuid{4183}
\titre{Contre-exemple au théorème de Leibniz}
\theme{}
\auteur{quercia}
\date{2010/03/11}
\organisation{exo7}
\contenu{
  \texte{}
  \question{On pose~:
$f(x,y) = \begin{cases}
 x            &\text{ si } y\ge0 \text{ et } 0\le x\le \sqrt y~;\cr
 2\sqrt y - x &\text{ si } y\ge0 \text{ et } \sqrt y < x\le 2\sqrt y~;\cr
 0            &\text{ si } y\ge0 \text{ et } 2\sqrt y < x \text{ ou } x\le 0~;\cr
 -f(x,-y)     &\text{ si } y<0.\cr\end{cases}
$
et~: $F(y) =  \int_{x=0}^1 f(x,y)\,d x$.

Faire un dessin, vérifier que $f$ est continue sur~$\R^2$,
calculer $F(y)$ pour $-\frac14 \le y \le \frac 14$, $F'(0)$ et
$ \int_{x=0}^1 \frac{\partial f}{\partial y}(x,0)\,d x$.}
  \reponse{}
}