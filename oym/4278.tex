\uuid{4278}
\titre{Calcul par récurrence}
\theme{Exercices de Michel Quercia, Intégrale généralisée}
\auteur{quercia}
\date{2010/03/12}
\organisation{exo7}
\contenu{
  \texte{}
  \question{Soit $\alpha \in\, ]0,\pi[$ et
$I_n =  \int_{t=0}^\pi \frac{\cos nt\,d t}{1-\sin\alpha\cos t} $.

Calculer $I_n+I_{n+2}$ en fonction de $I_{n+1}$ puis
exprimer $I_n$ en fonction de $\alpha$ et $n$.}
  \reponse{$I_n+I_{n+2} = \frac{2I_{n+1}}{\sin\alpha}  \Rightarrow 
          I_n = \frac\pi{\cos\alpha} \Bigl(\tan\frac\alpha2 \Bigr)^n$.}
}