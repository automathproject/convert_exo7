\uuid{5212}
\titre{**IT}
\theme{Les rationnels, les réels}
\auteur{rouget}
\date{2010/06/30}
\organisation{exo7}
\contenu{
  \texte{}
  \question{Soit $A$ une partie non vide et bornée de $\Rr$. Montrer que $\mbox{sup}\{|x-y|,\;(x,y)\in A^2\}=\mbox{sup }A-\mbox{inf }A$.}
  \reponse{Posons $B=\{|y-x|,\;(x,y)\in A^2\}$.
$A$ est une partie non vide et bornée de $\Rr$, et donc $m=\mbox{inf }A$ et $M=\mbox{sup }A$ existent dans $\Rr$.
Pour $(x,y)\in A^2$, on a $m\leq x\leq M$ et $m\leq y M$, et donc $y-x\leq M-m$ et $x-y\leq M-m$ ou encore $|y-x|\leq M-m$.
Par suite, $B$ est une partie non vide et majorée de $\Rr$. $B$ admet donc une borne supérieure.
Soit $\varepsilon>0$. Il existe $(x_0,y_0)\in A^2$ tel que $x_0<\mbox{inf }A+\frac{\varepsilon}{2}$ et $y_0>\mbox{sup }A-\frac{\varepsilon}{2}$.

Ces deux éléments $x_0$ et $y_0$ vérifient, 

$$|y_0-x_0|\geq y_0-x_0>\left(\mbox{sup }A-\frac{\varepsilon}{2}\right)-\left(\mbox{inf }A+\frac{\varepsilon}{2}\right)=\mbox{sup }A-\mbox{inf }A-\varepsilon.$$
En résumé, 
\begin{enumerate}
 \item  $\forall(x,y)\in A^2,\;|y-x|\leq\mbox{sup }A-\mbox{inf }A$ et  
 \item  $\forall\varepsilon>0,\;\exists(x,y)\in A^2/\;|y-x|>\mbox{sup }A-\mbox{inf }A-\varepsilon$.
\end{enumerate}
Donc, $\mbox{sup }B=\mbox{sup }A-\mbox{inf }A$.

\begin{center}
\shadowbox{
$\mbox{sup }\{|y-x|,\;(x,y)\in A^2\}=\mbox{sup }A-\mbox{inf }A$.
}
\end{center}}
}