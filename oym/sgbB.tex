\uuid{sgbB}
\exo7id{3877}
\titre{Borne supérieure de fonctions croissantes}
\theme{Exercices de Michel Quercia, Fonctions monotones}
\auteur{quercia}
\date{2010/03/11}
\organisation{exo7}
\contenu{
  \texte{Soit $f : {\R} \to {\R}$ une fonction bornée.

On note ${\cal E} = \{g : \R \to \R \text{ croissantes tq } g \le f \}$,
et pour $x\in\R$ : $\tilde f(x) = \sup\{ g(x) \text{ tq } g \in {\cal E} \}$.}
\begin{enumerate}
  \item \question{Montrer que $\tilde f \in {\cal E}$.}
  \item \question{On suppose $f$ continue. Montrer que $\tilde f$ est aussi continue.\par
      (S'il existe un point $x_0 \in \R$ tel que
      $\lim_{x\to x_0^-} \tilde f(x) < \lim_{x\to x_0^+} \tilde f(x)$,
      construire une fonction de $\cal E$ supérieure à~$\tilde f\,$)}
\end{enumerate}
\begin{enumerate}

\end{enumerate}
}