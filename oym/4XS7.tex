\uuid{4XS7}
\exo7id{6969}
\titre{Exercice 6969}
\theme{Fractions rationnelles, Décompositions en éléments simples}
\auteur{exo7}
\date{2014/04/08}
\organisation{exo7}
\contenu{
  \texte{Décomposer les fractions suivantes en éléments simples sur $\Rr$.}
\begin{enumerate}
  \item \question{\`A l'aide de divisions euclidiennes successives :
% De 447, cousquer
$$F=\frac{4X^6-2X^5+11X^4-X^3+11X^2+2X+3}{X(X^2+1)^3}$$}
  \item \question{\`A l'aide d'une division selon les puissances croissantes :
% De 446, cousquer
$$G=\frac{4X^4-10X^3+8X^2-4X+1}{X^3(X-1)^2}$$}
  \item \question{Idem pour :
% De 444, cousquer
$$H=\frac{X^4+2X^2+1}{X^5-X^3}$$}
  \item \question{A l'aide du changement d'indéterminée $X=Y+1$ :
% De 444, cousquer
$$K=\frac{X^5+X^4+1}{X(X-1)^4}$$}
\end{enumerate}
\begin{enumerate}
  \item \reponse{$F=\frac{4X^6-2X^5+11X^4-X^3+11X^2+2X+3}{X(X^2+1)^3}$.

  \begin{enumerate}}
  \item \reponse{La décomposition en éléments simples de $F$ est de la forme
$F=\frac{a}{X}+\frac{bX+c}{(X^2+1)^3}+\frac{dX+e}{(X^2+1)^2}+\frac{fX+g}{X^2+1}$.
Il est difficile d'obtenir les coefficients par substitution.
% $a,b,c$, (pour ces derniers : multiplication des deux membres par $X^2+1$, simplification, puis remplacement de $X$ par $i$ ou $-i$, avec séparation des parties
% réelle et imaginaire), mais c'est insuffisant pour conclure: il faut
% encore soustraire $\frac{bX+c}{(X^2+1)^3}$, simplifier par $X^2+1$, calculer $d$
% et $e$,\ldots}
  \item \reponse{On va ici se contenter de trouver $a$ :
on multiplie $F$ par $X$, puis on remplace $X$ par $0$, on obtient $a=3$.}
  \item \reponse{On fait la soustraction $F_1=F-\frac{a}{X}$. 
  On sait que la fraction $F_1$ \emph{doit} se simplifier par $X$. 
  On trouve $F_1=\frac{X^5-2X^4+2X^3-X^2+2X+2}{(X^2+1)^3}$.}
  \item \reponse{La fin de la décomposition se fait par divisions euclidiennes successives.
  Tout d'abord la division du numérateur $X^5-2X^4+2X^3-X^2+2X+2$
par $X^2+1$:
$$X^5-2X^4+2X^3-X^2+2X+2=(X^2+1)(X^3-2X^2+X+1)+X+1$$
puis on recommence en divisant le quotient obtenu par $X^2+1$, pour obtenir 
$$X^5-2X^4+2X^3-X^2+2X+2=(X^2+1)\big((X^2+1)(X-2)+3\big)+X+1$$
On divise cette identité par $(X^2+1)^3$ :

$F_1 = \frac{(X^2+1)\big((X^2+1)(X-2)+3\big)+X+1}{(X^2+1)^3} 
= \frac{X+1}{(X^2+1)^3}+\frac{3}{(X^2+1)^2}+\frac{X-2}{X^2+1}$

Ainsi 
$$F=
\frac{3}{X}+\frac{X+1}{(X^2+1)^3}+\frac{3}{(X^2+1)^2}+\frac{X-2}{X^2+1}$$}
\end{enumerate}
}