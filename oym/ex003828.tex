\uuid{3828}
\titre{Calcul de minimums}
\theme{}
\auteur{quercia}
\date{2010/03/11}
\organisation{exo7}
\contenu{
  \texte{}
\begin{enumerate}
  \item \question{Soit $\varphi : {\R^n} \to \R$ définie par
    $\varphi(x_1,\dots,x_n) =  \int_{t=0}^1 (1+tx_1+\dots+t^nx_n)^2\,d t$.
    Montrer que $\varphi$ admet un minimum absolu et le calculer lorsque $n=3$.}
  \item \question{Même question avec
    $\psi(x_1,\dots,x_n) =  \int_{t=0}^{+\infty} e^{-t}(1+tx_1+\dots+t^nx_n)^2\,d t$.}
\end{enumerate}
\begin{enumerate}
  \item \reponse{$\frac1{16}$.}
  \item \reponse{$\frac14$.}
\end{enumerate}
}