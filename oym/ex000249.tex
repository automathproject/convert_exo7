\uuid{249}
\titre{Exercice 249}
\theme{}
\auteur{bodin}
\date{1998/09/01}
\organisation{exo7}
\contenu{
  \texte{}
  \question{Combien $15!$ admet-il de diviseurs ?}
  \reponse{\'Ecrivons la d\'ecomposition de  $15 !=1.2.3.4\ldots15$ en facteurs premiers. $15 !  = 2^{11}.3^6.5^3.7^2 .11.13$.
Un diviseur de $15 !$ s'\'ecrit $d = 2^{\alpha}.3^\beta.5^\gamma.7^\delta .11^\epsilon.13^\eta$
avec $0 \leq \alpha \leq 11$, $0 \leq \beta \leq 6$, $0 \leq \gamma \leq 3$, $0 \leq \delta \leq 2$,
$0 \leq \epsilon \leq 1$, $0 \leq \eta \leq 1$. De plus tout nombre $d$ de cette forme est un diviseur de $15 !$.
Le nombre de diviseurs est donc $(11+1)(6+1)(3+1)(2+1)(1+1)(1+1) = 4032$.}
}