\uuid{4886}
\titre{Polygone des milieux}
\theme{Exercices de Michel Quercia, Barycentres}
\auteur{quercia}
\date{2010/03/17}
\organisation{exo7}
\contenu{
  \texte{}
  \question{Soit $P = A_1A_2\dots A_n$ un polygone à $n$ sommets :
on lui associe le polygone $P' = A'_1A'_2\dots A'_{n-1}A'_n$
où $A'_i$ est le milieu de $A_i$ et $A_{i+1}$ ($A_{n+1} = A_1$).

On définit alors une suite de polygones par récurrence :
$\begin{cases} P_0 = P\cr P_{k+1} = (P_k)'.\cr \end{cases}$

Montrer que chaque sommet de $P_k$ converge vers le centre de gravité de $P_0$
lorsque $k$ tend vers l'infini.

(\'Ecrire un sommet de $P_k$ comme barycentre de $A_1,\dots,A_n$)}
  \reponse{$A_i^{(k)} = \text{Bar}\left(
          A_j : \frac 1{2^k}\sum_{l\equiv j(\mathrm{mod}\, n)} C_k^{|l-i|}\right)$.}
}