\uuid{3766}
\titre{Diagonalisation de matrices symétriques}
\theme{}
\auteur{quercia}
\date{2010/03/11}
\organisation{exo7}
\contenu{
  \texte{Diagonaliser dans une base orthonormée :}
\begin{enumerate}
  \item \question{$A = \begin{pmatrix} 6 &-2 &2 \cr -2 &5 &0 \cr 2 &0 &7 \cr \end{pmatrix}$.}
  \item \question{$A = \frac19\begin{pmatrix} 23 &2 &-4 \cr 2 &26 &2 \cr -4 &2 &23 \cr \end{pmatrix}$.}
\end{enumerate}
\begin{enumerate}
  \item \reponse{$P = \frac13\begin{pmatrix}2 &-1 &2 \cr 2 &2 &-1 \cr -1 &2 &2 \cr\end{pmatrix}$,
             $D = \text{Diag}(3\ 6\ 9)$.}
  \item \reponse{$P = \frac13\begin{pmatrix}2 &-1 &2 \cr 2 &2 &-1 \cr -1 &2 &2 \cr\end{pmatrix}$,
             $D = \text{Diag}(3\ 3\ 2)$.}
\end{enumerate}
}