\uuid{4747}
\titre{Suites de Cauchy}
\theme{Exercices de Michel Quercia, Topologie dans les espaces vectoriels normés}
\auteur{quercia}
\date{2010/03/16}
\organisation{exo7}
\contenu{
  \texte{}
  \question{Soient $(u_n)$, $(v_n)$ deux suites d'un evn $E$ telles que
$u_n - v_n \xrightarrow[n\to\infty]{} 0$ et $(u_n)$ est de Cauchy.
Montrer que $(v_n)$ est de Cauchy.}
  \reponse{}
}