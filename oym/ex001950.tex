\uuid{1950}
\titre{Exercice 1950}
\theme{}
\auteur{legall}
\date{2003/10/01}
\organisation{exo7}
\contenu{
  \texte{Soit $f$ une fonction int\'egrable au sens de Riemann p\'eriodique de 
p\'eriode $2\pi $. On d\'esigne par~:
$\displaystyle{ \frac{a_0}{2}+ \sum _{k=1}^\infty(a_k \cos (kx)+ b_k 
\sin (kx))}$ sa s\'erie de Fourier et on pose, pour tout $n\in \Nn :
\displaystyle{ S_n(x)=\frac{a_0}{2}+ \sum _{k=1}^n(a_k \cos (kx)+ b_k 
\sin (kx))}.$}
\begin{enumerate}
  \item \question{Soit $\theta \in \Rr-2\pi \Zz .$ Montrer $\displaystyle{ 
\frac{1}{2}+ \sum _{k=1}^n \cos (k\theta)=\frac{\sin 
(n+\frac{1}{2})\theta}
{2\sin \frac{\theta }{2}}}.$}
  \item \question{Etablir que $
\displaystyle{ S_n(x)=\frac{1}{\pi}\int _{-\pi}^\pi \frac {\sin 
(n+\frac{1}{2})(x-t)}{2\sin \frac{(x-t)}{2}}f(t)dt}$.}
  \item \question{En d\'eduire $ \displaystyle{ S_n(x)=\frac{1}{2\pi}\int 
_{-\pi}^\pi f(x+\theta ) \frac{\sin (n+\frac{1}{2})\theta}
{\sin \frac{\theta }{2}}d\theta } $.}
  \item \question{Calculer $\displaystyle{ \int _{0}^\pi \frac{\sin (n+\frac{1}{2})\theta}
{\sin \frac{\theta }{2}}d\theta } .$}
\end{enumerate}
\begin{enumerate}

\end{enumerate}
}