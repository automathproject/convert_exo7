\uuid{3678}
\titre{Coefficients diagonaux dans la méthode de Schmidt}
\theme{Exercices de Michel Quercia, Produit scalaire}
\auteur{quercia}
\date{2010/03/11}
\organisation{exo7}
\contenu{
  \texte{}
  \question{Soit $E$ un espace euclidien, ${\cal B} = (\vec u_1,\dots,\vec u_n)$ une base de $E$
et ${\cal B}' = (\vec e_1,\dots,\vec e_n)$ la base orthonormée déduite de
$\cal B$ par la méthode de Schmidt.
On note $P$ la matrice de passage de $\cal B$ à ${\cal B}'$.

Montrer que $P_{ii} \times d(\vec u_i, \text{vect}(\vec u_1,\dots,\vec u_{i-1})) = 1$.}
  \reponse{}
}