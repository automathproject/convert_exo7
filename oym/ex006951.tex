\exo7id{6951}
\titre{Convergence en probabilité **}
\theme{}
\auteur{ruette}
\date{2013/01/24}
\organisation{exo7}
\contenu{
  \texte{}
\begin{enumerate}
  \item \question{Montrer que pour tout $x>0$,
$$
e^{-x^2/2}\left(\frac{1}{x}-\frac{1}{x^3}\right)\leq
\int_x^{+\infty}e^{-t^2/2}dt\leq e^{-x^2/2}\frac{1}{x}.
$$
{\it Indication : on pourra intégrer par parties $t^{-1}te^{-t^2/2}$.}}
  \item \question{Soit $(X_n)_{n\geq 1}$ une suite de variables aléatoires indépendantes de même loi
$\mathcal{N}(0,1)$. Montrer que $\displaystyle\frac{\max_{1\leq i\leq n} X_i}{\sqrt{2\ln n}}$
tend vers~$1$ en probabilité.}
\end{enumerate}
\begin{enumerate}
  \item \reponse{Intégration par parties : $u=1/t, v'=te^{-t^2/2}, u'=-1/t^2, v=-e^{-t^2/2}$ :\\
$\displaystyle\int_x^{+\infty}e^{-t^2/2}dt=\frac{1}{x}e^{-x^2/2}-
\int_x^{+\infty}\frac{1}{t^2}e^{-t^2/2}dt\leq \frac{1}{x}e^{-x^2/2}
\mbox{ car l'intégrale de droite est positive}.$\\
On fait une nouvelle intégration par parties :
$\displaystyle
\int_x^{+\infty}\frac{1}{t^2}e^{-t^2/2}dt=\frac{1}{x^3}e^{-x^2/2}-
\int_x^{+\infty}\frac{3}{t^4}e^{-t^2/2}\,dt\leq \frac{1}{x^3}e^{-x^2/2},
$
donc $\displaystyle \int_x^{+\infty}e^{-t^2/2}dt\geq \frac{1}{x}e^{-x^2/2}-
\frac{1}{x^3}e^{-x^2/2}$.}
  \item \reponse{Soit $\displaystyle Y_n=\max_{1\leq i\leq n} X_i$. On a
$\displaystyle P(Y_n\leq t)=\prod_{i=1}^n P(X_i\leq t)=P(X_1\leq t)^n$ et de même $\displaystyle P(Y_n\geq t)=P(X_1\geq t)^n$.

On veut montrer que 
$$\forall\epsilon>0,\  P(|\frac{Y_n}{\sqrt{2\ln n}}-1|\geq
\epsilon)\to 0\quad \mbox{quand}\quad n\to +\infty\quad 
\mbox{(convergence en probabilité)}.$$
On peut écrire
$$\{|\frac{Y_n}{\sqrt{2\ln n}}-1|\geq\epsilon\}=\{|Y_n-\sqrt{2\ln n}|\geq \epsilon
\sqrt{2\ln n}\}=\{Y_n\geq (1+\epsilon)\sqrt{2\ln n}\}\cup
\{Y_n\leq (1-\epsilon)\sqrt{2\ln n}\}$$ 
et cette union est disjointe donc
$$P(|\frac{Y_n}{\sqrt{2\ln n}}-1|\geq\epsilon)=P(Y_n\geq (1+\epsilon)\sqrt{2\ln n})
+P(Y_n\leq (1-\epsilon)\sqrt{2\ln n}). $$
On va montrer que ces 2 probabilités tendent vers $0$.

\begin{eqnarray*}
P(Y_n\leq (1-\epsilon)\sqrt{2\ln n})&=&P(X_1\leq (1-\epsilon)\sqrt{2\ln n})^n
=\left(1-P(X_1>(1-\epsilon)\sqrt{2\ln n}\right)^n\\
&=&\left(1-\frac{1}{\sqrt{2\pi}}\int_{(1-\epsilon)\sqrt{2\ln n}}^{+\infty}
e^{-t^2/2}\,dt\right)^n\\
&\leq & \left(1-\frac{1}{\sqrt{2\pi}}e^{-(\ln n)(1-\epsilon)^2}
\left(\frac{1}{(1-\epsilon)\sqrt{2\ln n}}-\frac{1}{((1-\epsilon)\sqrt{2\ln n})^3}
\right)\right)^n\mbox{ par 1.}
\end{eqnarray*}

Comme $e^{-(\ln n)(1-\epsilon)^2}=n^{-(1-\epsilon)^2}$, la quantité
$$
n\ln\left(1-\frac{1}{\sqrt{2\pi}}e^{-(\ln n)(1-\epsilon)^2}
\left(\frac{1}{(1-\epsilon)\sqrt{2\ln n}}-\frac{1}{((1-\epsilon)\sqrt{2\ln n})^3}
\right)\right)$$ est équivalent à
$$-\frac{1}{\sqrt{2\pi}} n^{2\epsilon-\epsilon^2}
\left(\frac{1}{(1-\epsilon)\sqrt{2\ln n}}-\frac{1}{((1-\epsilon)\sqrt{2\ln n})^3}
\right),\quad \mbox{donc à}\quad \displaystyle -\frac{n^{2\epsilon-\epsilon^2}}{\sqrt{2\pi}(1-\epsilon)
\sqrt{2\ln n}} .$$

Si $\epsilon>0$ est assez petit, $\alpha=2\epsilon-\epsilon^2>0$, donc
$\displaystyle \frac{n^{2\epsilon-\epsilon^2}}{\sqrt{\ln n}}\to +\infty$. \\
En reprenant l'exponentielle, on obtient que
$P(Y_n\leq (1-\epsilon)\sqrt{2\ln n})$ tend vers $0$ quand $n\to +\infty$.
$$
P(Y_n\geq (1+\epsilon)\sqrt{2\ln n})=(P(X_1\geq (1+\epsilon)\sqrt{2\ln n}))^n
=\left(\frac{1}{\sqrt{2\pi}}\int_{(1+\epsilon)\sqrt{2\ln n}}^{+\infty}
e^{-t^2/2}\,dt\right)^n.$$ 
Donc par 1., 
$\displaystyle\ln (P(Y_n\geq (1+\epsilon)\sqrt{2\ln n}))$ est inférieur ou égal à
$$
n\ln\left(\frac{1}{\sqrt{2\pi}}e^{-(\ln n)(1+\epsilon)^2}\frac{1}{(1+\epsilon)
\sqrt{2\ln n}}\right)=-n (\ln n)(1+\epsilon)^2-n\ln(\sqrt{2\pi}(1+\epsilon)\sqrt{2\ln n})
\to -\infty$$
donc $\displaystyle P(Y_n\geq (1+\epsilon)\sqrt{2\ln n})$
tend vers $0$ quand $n\to +\infty$. Conclusion :
$\displaystyle\frac{Y_n}{\sqrt{2\ln n}}\stackrel{P}{\longrightarrow}1$.}
\end{enumerate}
}