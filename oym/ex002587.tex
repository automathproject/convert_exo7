\uuid{2587}
\titre{Exercice 2587}
\theme{}
\auteur{delaunay}
\date{2009/05/19}
\organisation{exo7}
\contenu{
  \texte{}
  \question{Soit $\displaystyle A=\begin{pmatrix}a&c \\  c&d\end{pmatrix}\in M_2(\R)$,
montrer que $A$ est diagonalisable sur $\R$.}
  \reponse{{\it Soit $\displaystyle A=\begin{pmatrix}a&c \\  c&d\end{pmatrix}\in M_2(\R)$,
 on montre que $A$ est diagonalisable sur $\R$.}

Le polyn\^ome caract\'eristique $P_A(X)$ est \'egal \`a
$$P_A(X)=\begin{vmatrix}a-X&c \\  c& d-x\end{vmatrix}=(a-X)(d-X)-c^2=X^2-(a+d)X+ad-c^2,$$
d\'eterminons ses racines : calculons le discriminant :
\begin{align*}
\Delta&=(a+d)^2-4(ad-c^2) \\  &=a^2+d^2+2ad-4ad+4c^2 \\ &=a^2+d^2-2ad+4c^2 \\ &=(a-d)^2+4c^2\geq 0
\end{align*}
On a $\Delta=0\iff a-d=0\ {\hbox{et}}\ c=0$, mais, si $c=0$, la matrice $A$ est d\'ej\`a diagonale. Sinon $\Delta>0$ et le polyn\^ome caract\'eristique admet
deux racines r\'eelles distinctes, ce qui prouve que la matrice est toujours diagonalisable dans $\R$.}
}