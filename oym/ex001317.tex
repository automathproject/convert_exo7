\uuid{1317}
\titre{Exercice 1317}
\theme{}
\auteur{ortiz}
\date{1999/04/01}
\organisation{exo7}
\contenu{
  \texte{}
\begin{enumerate}
  \item \question{L'ensemble $\Rr\setminus\left\{-1\right\}$ muni de la loi $\star$
    d\'efinie par $\forall a,b\in
\Rr,a\star b=a+b+ab$ est-il un groupe~~?}
  \item \question{L'ensemble $E=\left\{-1,1,i,-i \right\}\subseteq \Cc$ muni de la loi usuelle de
multiplication dans $\Cc$ est-il un groupe~?}
  \item \question{L'ensemble $E=\left\{\left(\begin{smallmatrix} a&0\\0&0
\end{smallmatrix}\right) :a\in \Rr\setminus\left\{0\right\}\right\}$ muni
de la loi de multiplication usuelle des matrices
de $\mathcal{M}_2(\Rr)$ est-il un groupe ?}
  \item \question{L'ensemble $\mathcal{S}_2(\Rr) $ des matrices sym\'etriques
r\'eelles d'ordre 2 muni de la loi de
multiplication usuelle des matrices de
$\mathcal{M}_2(\Rr)$ est-il un groupe ?}
\end{enumerate}
\begin{enumerate}

\end{enumerate}
}